\documentclass[a4paper, twoside, parskip, 10pt, smallheadings]{scrbook}
%\input{praeambel_neu}
\usepackage[ansinew]{inputenc}
\usepackage{amsfonts}
\usepackage{amsmath}
\usepackage{amssymb}
\usepackage{amsthm}
\usepackage[ngerman]{babel}
%\usepackage[T1]{fontenc}
%\usepackage{graphicx}

%%%%%%%%%%%%%%%%%%%%\usepackage[final]{pst-pdf}
\usepackage{pst-all}
\usepackage{bm}
\date{2007-12-13}


\usepackage{bm}


\usepackage{color}
\usepackage{longtable}
\usepackage{hyperref}
\usepackage{multicol}
\usepackage{framed}



\usepackage{bbm} % für Zahlenmengen

\usepackage{enumitem}
\usepackage[top=4.5cm,left=3cm,headsep=1cm]{geometry}
\usepackage{chngpage}

\usepackage{hyperref}

 % \pagestyle{scrheadings}

\definecolor{dg}{gray}{0.55}
\definecolor{lg}{gray}{0.85}

 \sloppy
%Inhaltsverzeichnis% 
\setcounter{secnumdepth}{3} \setcounter{tocdepth}{1}



 \sloppy
%Inhaltsverzeichnis% 
\setcounter{secnumdepth}{3} \setcounter{tocdepth}{1}
%%Eurosymbol zeichnen:
 \newcommand\euro{{\sffamily C%
    \makebox[0pt][l]{\kern-.70em\mbox{--}}%
    \makebox[0pt][l]{\kern-.68em\raisebox{.25ex}{--}}}}
\renewcommand{\baselinestretch}{1.1}



\theoremstyle{plain}
\newtheorem{Satz}{Satz}%[chapter]
\newtheorem*{Merke}{Merke}
\newtheorem{HS}{Hilfssatz}%[chapter]
\renewcommand{\proofname}{Beweis}
\theoremstyle{definition}
\newtheorem{Def}{Definition}%[chapter]
\newtheorem{Bsp}{Beispiel}%[chapter]
\newtheorem{Auf}{Aufgabe}%[chapter]
\newtheorem*{Folg}{Folgerung}
\newtheorem*{Bem}{Bemerkung}
\newtheorem*{Mer}{Merke}



\definecolor{dunkelgelb}{rgb}{1,0.94,0.05}
\definecolor{orange}{rgb}{1,0.52,0}

\newenvironment{fshaded}{%
\def\FrameCommand{\fcolorbox{framecolor}{shadecolor}}%
\MakeFramed {\FrameRestore}}%
{\endMakeFramed}


%Formatierung Satz - Rahmenfarbe und Hintergrundfarbe
\newenvironment{fsatz}[1][]{\definecolor{shadecolor}{rgb}{.973,.513,.27}%
\definecolor{framecolor}{rgb}{1,0.25,0}%
\begin{fshaded}\begin{Satz}[#1]}{\end{Satz}\end{fshaded}}
%Formatierung Definition - Rahmenfarbe und Hintergrundfarbe
\newenvironment{fdef}[1][]{\definecolor{shadecolor}{rgb}{1,1,.29}%
\definecolor{framecolor}{rgb}{1,1,0}%
\begin{fshaded}\begin{Def}[#1]}{\end{Def}\end{fshaded}}
%Formatierung Beispiele - Rahmenfarbe und Hintergrundfarbe
\newenvironment{fbsp}[1][]{\definecolor{shadecolor}{rgb}{.57,1,.28}%
\definecolor{framecolor}{rgb}{.25, .63, 0}%
\begin{fshaded}\begin{Bsp}[#1]}{\end{Bsp}\end{fshaded}}
%Formatierung Aufgaben - Rahmenfarbe und Hintergrundfarbe
\newenvironment{fauf}[1][]{\definecolor{shadecolor}{rgb}{.58,.788,1}%
\definecolor{framecolor}{rgb}{.13,.25,.9}%
\begin{fshaded}\begin{Auf}[#1]}{\end{Auf}\end{fshaded}}


\newcommand{\bb}{\begin{fbsp}}
\newcommand{\eb}{\end{fbsp}}
\newcommand{\ba}{\begin{fauf}}
\newcommand{\ea}{\end{fauf}}
\newcommand{\bn}{\begin{enumerate}}
\newcommand{\en}{\end{enumerate}}
\newcommand{\bi}{\begin{itemize}}
\newcommand{\ei}{\end{itemize}}
\newcommand{\bd}{\begin{fdef}}
\newcommand{\ed}{\end{fdef}}
\newcommand{\bs}{\begin{fsatz}}
\newcommand{\es}{\end{fsatz}}
\newcommand{\bme}{\begin{Mer}}
\newcommand{\eme}{\end{Mer}}
\newcommand{\bt}{\begin{tabular}}
\newcommand{\et}{\end{tabular}}



\newcommand{\beq}{\begin{equination}}
\newcommand{\eeq}{\end{equination}}

\definecolor{blau}{rgb}{0.09, 0.1, 0.64}
\definecolor{hellblaugrau}{rgb}{0.82, 0.906, 0.871}
\newcommand{\mygrid}{\psgrid[gridlabels=0pt, subgriddiv=2, gridwidth=0.5pt, subgridwidth=0.5pt, gridcolor=hellblaugrau, subgridcolor=hellblaugrau]}

\usepackage{scrpage2}        % Seitenkopf gestalten%
\special{papersize=210mm,297mm}         % DVI das DIN A4 Format mitteilen %
\renewcommand{\baselinestretch}{1}        % Zeilenabstand: 1.4-zeilig %
\parskip1.5ex                   % Absatzabstand %
\ohead{\pagemark}
\ihead{\headmark}
\cfoot{Realgymnasium Albert Einstein - Meran}
%\ofoot{\includegraphics[width=1.5cm]{einstein.jpg}}
\ifoot{}
\setheadsepline{.04pt}
\setfootsepline{.04pt}

\usepackage{scrpage2}        % Seitenkopf gestalten%
\special{papersize=210mm,297mm}         % DVI das DIN A4 Format mitteilen %
\renewcommand{\baselinestretch}{1}        % Zeilenabstand: 1.4-zeilig %
\parskip1.5ex                   % Absatzabstand %


\newcommand{\zitat}[2]{ \begin{quote}       % Zitate am Anfang jedes Kapitels
            \textit{%
            \begin{center}%
            {\small#1}\\%
            \end{center}%
            \begin{flushright}%
            {\footnotesize#2}%
            \end{flushright}}%
            \end{quote}}

 
%\usepackage{rotating}

\newcommand{\mybox}[4]{%
    \ifthenelse{\isodd{#4}}
        {% ungerade
            \begin{tabular}{p{3mm}ll}%
            \colorbox{dg}{\parbox[c][#1]{5mm}{\begin{turn}{90}{\Large \sf%
            \textbf{#3}}\end{turn}}}&%
                \colorbox{lg}{\parbox[c][#1]{0.97\textwidth}{\sf#2}}\\%
            \end{tabular}%
        }{%gerade
            \begin{tabular}{l@{}l}%
            \colorbox{lg}{\parbox[c][#1]{0.97\textwidth}{\sf#2}}&%
            \colorbox{dg}{\parbox[c][#1]{5mm}{\begin{turn}{-90}{\Large \sf%
            \textbf{#3}}\end{turn}}}
            \end{tabular}%
        }
    }



\topmargin-1.4cm \oddsidemargin  0.7cm 
\evensidemargin -0.7cm 
\textwidth15.5cm \textheight25cm


\pagestyle{scrheadings}
% Schriftart
\setkomafont{pagenumber}{\normalfont\normalcolor\sffamily}
\setkomafont{pagefoot}{\normalfont\normalcolor\sffamily}
\setkomafont{pagehead}{\normalfont\normalcolor\sffamily}

\renewcommand{\familydefault}{\sfdefault}


\usepackage{framed, color, pifont}
\usepackage{color}
%%%%%%%%%%%%%%%%%%%%%%%%%%%%\usepackage{pst-eucl}
\usepackage{multicol}
%\usepackage{sty/secnum}
\SpecialCoor

\usepackage{pstricks-add}
% Dies ist die Datei makros.tex in der Fassung vom 4. Juli 1997,
% die alle in der Aufgabensammlung enthaltenen Makros und Umgebungen
% enthaelt.
%
% Die Namen aller nur fuer die Aufgabensammlung verwendeten Makros
% werden gross geschrieben.
%
% In den TeX-Blaettern werden die Aufgaben mit dem Makro ABINPUT
% eingelesen, dem der Dateiname mit und ohne geschweifte Klammern
% folgen kann. Der Titel eines Blattes wird mit der Umgebung
% ABTITEL gekennzeichnet, damit es vom Programm zur Auswahl der
% Aufgaben gefunden wird.
%
% Die Quelltexte der Aufgaben duerfen keine INPUT-Anweisungen,
% etwa zum Einlesen von Graphiken, enthalten, da die zugehoerigen
% Dateien vom Programm ABSELECT, das die Auswahl der Aufgaben
% erledigt, nicht gefunden werden.
%
%*********************************************************************************************
%
% Um bei der Verwendung von KOMA-Script (srcbook) bei enumerate (2. Ebene)
% die Doppelklammer im Label zu haben [(a) statt a)] (RR, 2001)
%
\renewcommand*\labelenumii{(\theenumii)}
%
%*********************************************************************************************
%
%% Kopf- und Fusszeilen auf den Blaettern
% Die Makros KOPFTEXT und FUSSTEXT legen Kopf- und Fusszeile jedes
% TeX-Blattes fest. Durch die Leerdefinitionen von KopfU, KopfG usw.
% zu Beginn des Uebersetzungsvorganges ist eine einheitliche Definition
% von Kopf- und Fusszeilen im gesamten zum Ausdruck benoetigten
% TeX-Dokument moeglich.
%---------------------------------------------------------------------------------------------
\newcommand{\KopfU}{}
\newcommand{\KopfG}{}
\newcommand{\FussU}{}
\newcommand{\FussG}{}
\newcommand{\KOPFTEXT}[1]{%
\renewcommand{\KopfU}{\parbox{\textwidth}%
{\sf\small #1 \hfill\thepage\\%
\rule[2ex]{\textwidth}{0.2mm}}}
\renewcommand{\KopfG}{\parbox{\textwidth}%
{\sf\small\thepage\hfill #1\\%
\rule[2ex]{\textwidth}{0.2mm}}}}
% Fusszeile
\newcommand{\FUSSTEXT}[1]{%
\renewcommand{\FussU}{\parbox{\textwidth}%
{\sf\small\rule[-0.9ex]{\textwidth}{0.2mm}\\%
#1 \\%
\rule[2ex]{\textwidth}{0.2mm}}}
\renewcommand{\FussG}{\parbox{\textwidth}%
{\sf\small\rule[-0.9ex]{\textwidth}{0.2mm}\\%
\hfill #1 \\%
\rule[2ex]{\textwidth}{0.2mm}}}
}
%*********************************************************************************************
%
%% ABTITEL
% Diese Umgebung kennzeichnet den Titel eines TeX-Blattes und muss auf
% jedem Blatt vorhanden sein, da das Programm zur Aufgabenauswahl nach
% ABTITEL sucht.
% Der Schalter PRIVATE erlaubt es, beim Gesamtausdruck fuer die
% TeX-Blaetter numerierte Ueberschriften zu erzeugen.
%---------------------------------------------------------------------------------------------
\newif\ifPRIVATE
\PRIVATEfalse
\newenvironment{ABTITEL}{\ifPRIVATE\begin{center}%
\setbox0=\vbox\bgroup\else\begin{center}\bf \large\fi}%
{\ifPRIVATE\egroup\fi\end{center}}%
%*********************************************************************************************
%
%% ABINPUT, FILE
% Das Makro ABINPUT ersetzt die TeX-eigene INPUT-Anweisung und dient
% zum Einlesen des Quelltextes einer Aufgabe. Das Makro uebernimmt ausserdem
% die Funktion des item-Befehls, zeigt also auch die Aufgabennummer auf dem
% jeweiligen Blatt an. Dies hat zur Folge, dass die AUFGABEN-Umgebung ein
% enumerate enthalten muss.
%
% Der Schalter FILE bestimmt, ob die Dateinamen der Aufgaben
% ausgegeben werden. Der Dateiname ist ggf. Teil der Itemmarkierung.
%---------------------------------------------------------------------------------------------
\newif\ifFILE
\def\ABINPUT #1 {%
\ifFILE{\item[\hspace{-2em}{\footnotesize\sf\fbox{#1}}\hspace{2em}%
\addtocounter{enumi}{1}\labelenumi]\input{#1}}%
\else{\item\input{#1}\ }\fi}
%*********************************************************************************************
%
%% AUFGABEN
% Eine Umgebung fuer Aufgaben, zunaechst im Wesentlichen eine Verkleidung
% fuer enumerate, kann aber spaeter weitere Stilmerkmale aufnehmen.
% Gibt zur Zeit den Namen des TeX-Blattes rechts unten auf der letzten
% Seite aus, falls die entsprechenden Anweisungen nicht auskommentiert
% sind.
%---------------------------------------------------------------------------------------------
\newenvironment{AUFGABEN}{\begin{enumerate}}
                         {\end{enumerate}}
                          %\vfill
                          %\hfill\tiny\jobname}
%*********************************************************************************************
%
%% ANG, ANGABE  (RR 2001)
% ANGABE ist eine Boolsche Variable, um zu entscheiden, ob Angaben
% gedruckt werden. ANG ist eine Umgebung fuer Angaben,
% im Wesentlichen, um diese auf Wunsch ein- und ausblenden zu koennen.
%---------------------------------------------------------------------------------------------
\newif\ifANGABE
\newcommand{\ANG}[1]{
    \ifANGABE{#1}\else{}\fi
  }
%*********************************************************************************************
%
%% LSG, LOESUNG
% LOESUNG ist eine Boolsche Variable, um zu entscheiden, ob Loesungen
% gedruckt werden. LSG ist eine Umgebung fuer Loesungen,
% im Wesentlichen, um diese auf Wunsch ein- und ausblenden zu koennen.
% Gelegentlich kann der Loesungstext auch methodische und didaktische
% Hinweise enthalten.
%---------------------------------------------------------------------------------------------
%%\newif\ifLOESUNG
%%\newdimen\mydim
%%\newenvironment{LSG}{\ifLOESUNG{{\it L"osung:}\quad}\else{}\fi%
%%\mydim=\hsize%
%%\advance\mydim by -30mm%
%%\setbox0=\vbox\bgroup\hsize=\mydim\begin{minipage}[t]{0.98\hsize}
%%\begin{small}}%
%%{\end{small}\end{minipage}\egroup\ifLOESUNG{\box0}\else{}\fi}
%
% Diese neue Definition des LSG-Makros erlaubt Seitenumbrueche in der
% Loesung und Loesungen, die laenger als eine Seite sind. 
% Aus typographischen Gruenden wurde der Schriftstil \small
% statt \it gewaehlt. (RR 1998, 2001)
%
\newif\ifLOESUNG
\newcommand{\LSG}[1]%
  { \ifANGABE
      { \ifLOESUNG\begingroup
        \begin{small}
        \begin{list}{\it L�sung:\hfill}%
                { \setlength{\itemindent}{0mm}
                  \setlength{\labelwidth}{14mm}
                  \setlength{\leftmargin}{0mm}
                }
        \item #1
        \end{list}\end{small}\endgroup\else{}\fi
      }
    \else
      { \ifLOESUNG{
        \begin{small}%
        #1
        \end{small}%
        }\else{}\fi%
      }\fi
  }
%%
%
%*********************************************************************************************
%
%% BILD
% Das folgende Makro dient der Einbindung von PCX-Grafiken.
% Diese muessen alle in einem Verzeichnis direkt unterhalb des
% jeweiligen Verzeichnisses fuer die Jahrgangsstufe
% liegen.
% Auch bei den Grafiken muss die Trennung nach Jahrgangsstufen
% beibehalten werden, da jede Jahrgangsstufe gesondert per FTP
% aktualisiert werden koennen muss.
% Da PCX-Bilder recht umfangreich sind und zudem nicht vom WWW-Server
% nicht ohne weiteres uebertragen werden koennen, muss ihre Verwendung
% auf das unbedingt notwendige Mass beschraenkt bleiben.
%---------------------------------------------------------------------------------------------
\newcommand{\BILD}[2]{%
\special{em: graph c:/ab/#1/grf/#2}}
%*********************************************************************************************
% 
%% Einbindung von EPS-Graphiken (RR 1999)
%
\newcommand{\EPSBASIS}[3]{
    \ifthenelse{\equal{#2}{}}
         {\ifthenelse{\equal{#3}{}}
                {}
                {\includegraphics[height=#3]{#1}}
         }
         {\ifthenelse{\equal{#3}{}}
                {\includegraphics[width=#2]{#1}}
                {\includegraphics[width=#2,height=#3]{#1}}
         }
  } 
%
%
%
\newcommand{\EPS}[4]{ \raisebox{2.5mm}{\parbox[t]{#2}{
                          \vspace{0pt}
                          #4
                          \EPSBASIS{#1}{#2}{#3}
                          }}
                        }
%
\newcommand{\EPSB}[4]{ \setlength{\fboxsep}{0pt}
                       \raisebox{2.5mm}{\parbox[t]{#2}{
                          \vspace{0pt}
                          #4
                          \fbox{\EPSBASIS{#1}{#2}{#3}}
                        }}
                       \setlength{\fboxsep}{3pt}
                      }
%
\newcommand{\EPSR}[6]{\parbox[t]{#1}{%\vspace{0pt}
                          #6}\hfill
                          \raisebox{2.5mm}{\parbox[t]{#3}{
                          \vspace{0pt}
                          #5
                          \EPSBASIS{#2}{#3}{#4}
                          }}
                        }
%
\newcommand{\EPSC}[4]{\begin{center}
                            \raisebox{2.5mm}{\parbox[t]{#2}{
                            \vspace{0pt}
                            #4
                            \EPSBASIS{#1}{#2}{#3}
                            }}
                            \end{center}
                           }
%
\newcommand{\EPSRB}[6]{\parbox[t]{#1}{%\vspace{0pt}
                          #6}\hfill
                          \raisebox{2.5mm}{\parbox[t]{#3}{
                          \setlength{\fboxsep}{0pt}
                          \vspace{0pt}
                          #5
                          \fbox{\EPSBASIS{#2}{#3}{#4}}
                          \setlength{\fboxsep}{3pt}
                          }}
                        }
%
\newcommand{\EPSCB}[4]{\begin{center}
                            \raisebox{2.5mm}{\parbox[t]{#2}{
                            \setlength{\fboxsep}{0pt}
                            \vspace{0pt}
                            #4
                            \fbox{\EPSBASIS{#1}{#2}{#3}}
                            \setlength{\fboxsep}{3pt}
                            }}
                            \end{center}
                           }
%
\newcommand{\PSF}[2]{\psfrag{#1}{\scriptsize #2}}
\newcommand{\PSFM}[2]{\psfrag{#1}[][]{\scriptsize #2}}
\newcommand{\PSFROT}[3]{\psfrag{#2}[c][c][1][#1]{\scriptsize #3}}
%
%% AUFZAEHLUNG, TEILAUFGABE, TAG
% Automatisches Durchzaehlen nebeneinander angeordneter Teilaufgaben
% Beispiel \begin{AUFZAEHLUNG}{l}
%          \TEILAUFGABE{......}
%          \end{AUFZAEHLUNG}
%---------------------------------------------------------------------------------------------
\newcounter{TAG}
\newenvironment{AUFZAEHLUNG}[1]{\setcounter{TAG}{1}%
\begin{tabular}{#1}}{\end{tabular}\hspace{-2em}}
%
\newcommand{\ZAEHLE}{\alph{TAG})}
\newcommand{\TEILAUFGABE}[1]{\ZAEHLE\hspace{1em}#1\hspace{1em}%
\addtocounter{TAG}{1}}
%*********************************************************************************************
%
%% PUNKTE
% Das folgende Makro dient dazu, am Rand einer Schulaufgabe die
% Bewertungseinheiten bei jeder Teilaufgabe zu notieren.
%---------------------------------------------------------------------------------------------
\newcommand{\PUNKTE}[1]{\mbox{}\marginpar{\hspace*{1em}\fbox{#1}}}
%*********************************************************************************************
%
%% TIMMS
% Das folgende Makro dient dazu, am Rand der Aufgaben einen Hinweis
% auf die Art der Aufgabe ("neue Aufgabenkultur") zu plazieren.
%---------------------------------------------------------------------------------------------
\newcommand{\TIMMS}[1]{\mbox{}\marginpar{\hspace*{3em}\fbox{#1}}}
%*********************************************************************************************
%
%% EGY
% Aufgaben f"ur das Europ"aische Gymnasium 
% veraltet
%---------------------------------------------------------------------------------------------
%\newcommand{\EGY}{\mbox{}\marginpar[]{\fbox{\small{EGY}}}}
\newcommand{\EGY}{}
%*********************************************************************************************
%
%% D, T, SC, SSC
% Einschalten von Displaystil bzw. Textstil beim Mathematiksatz
%---------------------------------------------------------------------------------------------
\newcommand{\D}{\displaystyle}
\newcommand{\T}{\textstyle}
\newcommand{\SC}{\scriptstyle}
\newcommand{\SSC}{\scriptscriptstyle}
%*********************************************************************************************
%
%% GF
% Anfuehrungszeichen fuer Zitate
%---------------------------------------------------------------------------------------------
\newcommand{\GF}[1]{\glqq{}#1\grqq{}}
%*********************************************************************************************
%
%% VPFEIL
% Zeichen fuer Vektorpfeil, nur im Mathmode verwenden
%---------------------------------------------------------------------------------------------
%\newcommand{\VPFEIL}[1]{\stackrel{\D\longrightarrow}{#1}}
\newcommand{\VPFEIL}[1]{\overrightarrow{#1}}
\newcommand{\VPFEILRM}[1]{\overrightarrow{\mathrm{#1}}}
%*********************************************************************************************
%
%% RVEKTOR, EVEKTOR
% Raeumliche und ebene Spaltenvektoren
\newcommand{\RVEKTOR}[4]{\left(\begin{array}{#1}\negthickspace#2\negthickspace\\\negthickspace#3\negthickspace\\\negthickspace#4\negthickspace\end{array}\right)}
\newcommand{\EVEKTOR}[3]{\left(\begin{array}{#1}\negthickspace#2\negthickspace\\\negthickspace#3\negthickspace\end{array}\right)}
%*********************************************************************************************
%%
%*********************************************************************************************
%
%% EPUNKT, RPUNKT
% Ebene und raeumliche Punktkoordinaten
\newcommand{\EPUNKT}[3]{
  {\rm #1}\left(\,#2\;\vline\;#3\,\right)}
\newcommand{\RPUNKT}[4]{
  {\rm #1}\left(\,#2\;\vline\;#3\;\vline\;#4\,\right)}
%
%*********************************************************************************************
%
%% POLAR
% Polarform komplexer Zahlen in der Form (r|phi)_p
% Verwendung: \POLAR{r}{phi}
\newcommand{\POLAR}[2]{
  \left(\,#1\;\vline\;#2\,\right)_{\mathrm{p}}}
%
%*********************************************************************************************
%
%% POLARE
% Polarform komplexer Zahlen in der Form r E(phi)
% Verwendung: \POLARE{r}{phi}
\newcommand{\POLARE}[2]{
  #1\cdot\mathrm{E}\left(#2\right)}
%
%*********************************************************************************************
%
%% KREIS
% Kreis in der Form k(M;r) im Mathemodus
% Verwendung: \KREIS{M}{r}
\newcommand{\KREIS}[2]{
  \text{k}(\text{#1;}#2)}
\newcommand{\KREISi}[2]{
  \text{k}_{\text{i}}(\text{#1;}#2)}
\newcommand{\KREISa}[2]{
  \text{k}_{\text{a}}(\text{#1;}#2)}
%
%*********************************************************************************************
%
%% DANN (daraus folgt), GENAUDANN
% Das logische Zeichen "daraus folgt" und "genau dann, wenn"
\newcommand{\DANN}{\Longrightarrow}
\newcommand{\GENAUDANN}{\Longleftrightarrow}
%
%*********************************************************************************************
%
%% OHNE (Mengenzeichen)
\newcommand{\OHNE}{\mathrel{\setminus}}
%
%*********************************************************************************************
%
%% SATZB{Zeile1\\Zeile2\\...}
% Ein gerahmter und zentrierter Satz, wobei sich die Rahmenbreite der
% Textbreite anpasst. SATZB{} stellt eine mehrzeilige math-Umgebung
% bereit, die Zeilen sind zentriert. 
% Normaler Text wird mit \text{} eingegeben.
\newcommand{\SATZB}[1]{\begin{displaymath}\boxed{\begin{gathered}
                       #1\end{gathered}}\end{displaymath}}
%
%*********************************************************************************************
%
%% RFRAC{z}{n}
% Bruch in Roman (fuer Einheiten)
\newcommand{\RFRAC}[2]{\frac{\rm #1}{\rm #2}}
%
%*********************************************************************************************
%
%% ggT, kgV
% im mathem. Modus in Roman gesetzt
\newcommand{\ggT}{\mathrm{ggT}}
\newcommand{\kgV}{\mathrm{kgV}}
%
%*********************************************************************************************
%
%% \RMd
% im mathem. Modus in Roman gesetztes d (Differentiale)
\newcommand{\RMd}{\mathrm{d}}
%
%*********************************************************************************************
%
%% EURO und MEUR (EURO im Mathemodus), Ct fuer Cent
\DeclareFontFamily{OT1}{mvs}{}
\DeclareFontShape{OT1}{mvs}{m}{n}{<-> fmvr8x}{}
\def\mvs{\usefont{OT1}{mvs}{m}{n}}
\def\mvchr{\mvs\char}
\def\textmvs#1{{\mvs #1}}
\def\EURhv{{\mvchr99}}\def\EURcr{{\mvchr100}}\def\EURtm{{\mvchr101}}
\def\EUR{{\mvchr164}}
\newcommand{\MEUR}{\mbox{\EUR}}
\newcommand{\Ct}{\ensuremath{\mbox{Ct}}}
%
%*********************************************************************************************
%
%% Mmu
%   aufrechtes mu im Mathemodus
%\newcommand{\Mmu}{\mbox{\textmu}}
\newcommand{\Mmu}{\mu}
%
%*********************************************************************************************
%
%% GC
%  Grad Celsius (im Mathemodus)
\newcommand{\GC}{\,^{\circ}\mathrm{C}}
%
%*********************************************************************************************
%

%
% Neben ein Befehlswort kommt eine \fbox bis Zeilenende
% Beispiel: \FBEFEHL{Beweise:\quad}{Sind ...}
%
\newlength{\LBefehl}
\newlength{\BBefehl}
\newcommand{\FBEFEHL}[2]{\settowidth{\LBefehl}{#1}\setlength{\BBefehl}{\linewidth}%
\addtolength{\BBefehl}{-\LBefehl}%\addtolength{\BBefehl}{-7mm}%
#1\fbox{\parbox[t]{\BBefehl}{#2}}\\}  
% das Gleiche ohne \fbox
\newcommand{\BEFEHL}[2]{\settowidth{\LBefehl}{#1}\setlength{\BBefehl}{\linewidth}%
\addtolength{\BBefehl}{-\LBefehl}%\addtolength{\BBefehl}{-2mm}%
#1\parbox[t]{\BBefehl}{#2}\\} 
%
%*********************************************************************************************
%
\newcommand{\relsqrt}{\sqrt{1-\beta^{2}}}
\newcommand{\relgamma}{\frac{1}{\sqrt{1-\beta^{2}}}}
%
%*********************************************************************************************
%
% Zeichen in einem Kreis
%\newcommand{\IMKREIS}[1]{ \unitlength 1mm 
%                          \begin{picture}(3,3)
%                          \put(1,1){\circle{4}}
%                          \put(1,1){\makebox(0,0){#1}}
%                          \end{picture}
%}
\def\IMKREIS#1{ ${%
	\psset{unit=1pt}
	\setbox0=\hbox{#1}\relax
	\dimen0=\the\wd0\dimen1=\the\ht0
	\divide\dimen0 by 2\divide\dimen1 by 2
	\ifdim\dimen0>\dimen1\dimen2=\dimen0\else\dimen2=\dimen1\fi
	\advance\dimen2 by +5pt
	\rput(\dimen0, \dimen1){\pscircle{\dimen2}}
	\box0%
}$ }
%*********************************************************************************************
%
% plus - nicht minus, nicht plus - minus
\newcommand{\PNM}{\raisebox{-1mm}{$\SC\,\stackrel{+}{(-)}\,$}}
\newcommand{\NPM}{\raisebox{-1mm}{$\SC\,\stackrel{(+)}{-}\,$}}

%
%% \BOX{Breite}{Hoehe}{Zeichen}
%  Zeichen in Box einer festen Breite und Hoehe
\newcommand{\BOX}[3]{\framebox[#1]{\rule[-1.5mm]{0mm}{#2}#3}}
%
%*********************************************************************************************
%% NN, BB, ZZ, QQ, RR, CC, GG, LL, DD, WW
% Makros zur Bezeichnung von Zahlenmengen
%---------------------------------------------------------------------------------------------
\newcommand{\NN}{\mathds{N}}
\newcommand{\BB}{\mathds{B}}
\newcommand{\CC}{\mathds{C}}
\newcommand{\ZZ}{\mathds{Z}}
\newcommand{\QQ}{\mathds{Q}}
\newcommand{\PP}{\mathds{P}}
\newcommand{\RR}{\mathds{R}}
\newcommand{\GG}{G}
\newcommand{\DD}{D}
\newcommand{\LL}{L}
\newcommand{\WW}{W}
%\def\NN{{\sf I\!N}} % natuerliche Zahlen
%
%\def\BB{{\sf I\!B}} % Brueche
%
%\def\ZZ{% ganze Zahlen
%{\mathchoice {\hbox{$\sf\textstyle Z\kern-0.4em Z$}}
%{\hbox{$\sf\textstyle Z\kern-0.4em Z$}}
%{\hbox{$\sf\scriptstyle Z\kern-0.3em Z$}}
%{\hbox{$\sf\scriptscriptstyle Z\kern-0.2em Z$}}}}
%
%\def\QQ{% rationale Zahlen
%{\mathchoice {\setbox0=\hbox{$\displaystyle\sf Q$}\hbox{\raise
%0.15\ht0\hbox to0pt{\kern0.3\wd0\vrule height0.8\ht0\hss}\box0}}
%{\setbox0=\hbox{$\textstyle\sf Q$}\hbox{\raise
%0.15\ht0\hbox to0pt{\kern0.3\wd0\vrule height0.8\ht0\hss}\box0}}
%{\setbox0=\hbox{$\scriptstyle\sf Q$}\hbox{\raise
%0.15\ht0\hbox to0pt{\kern0.3\wd0\vrule height0.7\ht0\hss}\box0}}
%{\setbox0=\hbox{$\scriptscriptstyle\sf Q$}\hbox{\raise
%0.15\ht0\hbox to0pt{\kern0.3\wd0\vrule height0.7\ht0\hss}\box0}}}}
%
%\def\RR{{\sf I\!R}} % reelle Zahlen
%
%\def\CC{% komplexe Zahlen
%{\mathchoice {\setbox0=\hbox{$\displaystyle\sf C$}\hbox{\hbox
%to0pt{\kern0.4\wd0\vrule height0.9\ht0\hss}\box0}}
%{\setbox0=\hbox{$\textstyle\sf C$}\hbox{\hbox
%to0pt{\kern0.4\wd0\vrule height0.9\ht0\hss}\box0}}
%{\setbox0=\hbox{$\scriptstyle\sf C$}\hbox{\hbox
%to0pt{\kern0.4\wd0\vrule height0.9\ht0\hss}\box0}}
%{\setbox0=\hbox{$\scriptscriptstyle\sf C$}\hbox{\hbox
%to0pt{\kern0.4\wd0\vrule height0.9\ht0\hss}\box0}}}}
%
%\def\GG{% Grundmenge
%{\mathchoice {\setbox0=\hbox{$\displaystyle\sf G$}\hbox{\raise
%0.03\ht0\hbox to0pt{\kern0.4\wd0\vrule height0.9\ht0\hss}\box0}}
%{\setbox0=\hbox{$\textstyle\sf G$}\hbox{\raise
%0.03\ht0\hbox to0pt{\kern0.4\wd0\vrule height0.9\ht0\hss}\box0}}
%{\setbox0=\hbox{$\scriptstyle\sf G$}\hbox{\raise
%0.03\ht0\hbox to0pt{\kern0.4\wd0\vrule height0.87\ht0\hss}\box0}}
%{\setbox0=\hbox{$\scriptscriptstyle\sf G$}\hbox{\raise
%0.03\ht0\hbox to0pt{\kern0.4\wd0\vrule height0.87\ht0\hss}\box0}}}}
%
%\def\LL{{\sf I\!L}}% Loesungsmenge
%
%\def\DD{{\sf I\!D}}% Definitionsmenge, ich schlage ein serifenloses D vor
%
%\def\WW{
%{\sf W\hspace{-0.825em}W}}% Wertemenge, ich schlage ein serifenloses W vor
%{\mathchoice {\displaystyle\sf W\hspace{-0.825em}W}
%{\textstyle\sf W\hspace{-0.825em}W}
%{\scriptstyle\sf W\hspace{-0.7em}W}
%{\scriptscriptstyle\sf W\hspace{-0.7em}W}}}
%
% Die Zeichen fuer "Winkel" und "entspricht" werden umdefiniert
%
\renewcommand{\angle}{<\hspace{-0.6em})\hspace{0.3em}}
\renewcommand{\equiv}{\mathrel{\widehat{=}}}
%*********************************************************************************************
%*********************************************************************************************
% Realschulmakros
\renewcommand{\leq}{\leqq}
\renewcommand{\geq}{\geqq}
\renewcommand{\subseteq}{\subseteqq}
\renewcommand{\supseteq}{\supseteqq}
\newcommand{\LQ}{\leqq}
\newcommand{\GQ}{\geqq}
%*********************************************************************************************
% Die nachfolgenden Makros wurden bisher (4. Juli 1996) nicht verwendet.
% Sie verbleiben in der Makrosammlung, um in Zukunft die Erstellung
% von PICTEX-Graphiken zu erleichtern.
%*********************************************************************************************
%*********************************************************************************************
%% XPOS, YPOS, XNEU, YNEU, STRECKE, SIZE, DRAW, MOVE, DRAWTO, MOVETO,
%% POLARDRAW, POLARMOVE, BOGEN, BEZEICHNE, INIT
%
% Die folgenden Makros dienen der einfacheren Erstellung von Zeichnungen
% in PICTEX. Sie werden zum Teil auch von einem speziellen Logo-Programm
% verwendet, mit dem sich Igelgrafiken in PICTEX-Bilder umsetzen lassen.
% Naehere Informationen dazu finden sich in der Beschreibung dieses Logo-
% Programms mit dem Namen TEXUTILS, das sich im Unterverzeichnis
% UTILTIES des Verzeichnisses AB der Aufgabenbank befindet.
%---------------------------------------------------------------------------------------------
%
\newlength\XPOS% aktuelle x-Koordinate des Zeichenstiftes
\newlength\YPOS% aktuelle y-Koordinate ...
\newlength\XNEU% x-Koordinate nach Ausfuehrung der jeweiligen Bewegung
\newlength\YNEU% y-Koordinate nach ...
\newlength\STRECKE% Laenge der Zeichenstrecke bei POLAR...
\newlength\SIZE% legt die Groesse der jeweiligen Zeichnung fest
%
\newcommand{\DRAW}[2]{% Zeichnen relativ zur alten Position
\advance\XNEU by #1%
\advance\YNEU by #2%
\plot {\XPOS} {\YPOS} {\XNEU} {\YNEU} /
\XPOS\XNEU\YPOS\YNEU}
%
\newcommand{\MOVE}[2]{% Bewegen relativ zur alten Position
\advance\XNEU by #1%
\advance\YNEU by #2%
\XPOS\XNEU\YPOS\YNEU}
%
\newcommand{\DRAWTO}[2]{% Zeichnen von alter auf neue -absolute- Position
\XNEU#1%
\YNEU#2%
\plot {\XPOS} {\YPOS} {\XNEU} {\YNEU} /
\XPOS\XNEU\YPOS\YNEU}
%
\newcommand{\MOVETO}[2]{% Bewegen von alter auf neue -absolute- Position
\XNEU#1%
\YNEU#2%
\XPOS\XNEU\YPOS\YNEU}
%
\newcommand{\POLARDRAW}[3]{% Zeichnet relativ zur alten Position
% cos und sin des Winkels zur Nordrichtung und die Streckenlaenge
% sind anzugeben
\STRECKE#3%
\advance\XNEU by #1\STRECKE%
\advance\YNEU by #2\STRECKE%
\plot {\XPOS} {\YPOS} {\XNEU} {\YNEU} /
\XPOS\XNEU\YPOS\YNEU}
%
\newcommand{\POLARMOVE}[3]{% Bewegt den Zeichenstift relativ zur
% alten Position.
% cos und sin des Winkels zur Nordrichtung und die Streckenlaenge
% sind anzugeben
\STRECKE#3%
\advance\XNEU by #1\STRECKE%
\advance\YNEU by #2\STRECKE%
\XPOS\XNEU\YPOS\YNEU}
%
\newcommand{\BOGEN}[5]{% Zeichnen eines Bogens, Winkel, Ausgangspunkt
% und Zentrum sind anzugeben
\circulararc #1 degrees from #2 #3 center at #4 #5 }
%
\newcommand{\BEZEICHNE}[3]{% Beschriften einer Zeichnung, Text und
% Versatz sind anzugeben
\put {$#1$}[tl] <#2,#3> at {\XPOS} {\YPOS} }
%
\newcommand{\INIT}{% Initialisieren der Grafikroutinen
\XPOS0pt\YPOS0pt\XNEU0pt\YNEU0pt\SIZE0.3mm}
%*********************************************************************************************
%*********************************************************************************************

%\input{realschulmakros}
\ANGABEtrue
\LOESUNGtrue
%\LOESUNGfalse
%\FILEtrue
\FILEfalse
\begin{document}
\begin{titlepage} 
\begin{center}
 {\Large Realgymnasium "`Albert Einstein"'}
\vspace{6cm}

 {\bf {\Huge Mathematik 2}}

\vspace{1.5cm}
\includegraphics[width=7cm]{MatheCover_einstein.jpg}

\vspace{3cm}

{\bf \large{ Schuljahr 2008-2009}}
\end{center}  
\end{titlepage}
 \newpage

{\tiny .}

\tableofcontents
% \chapter{�bungen zur Wiederholung 1. Klasse}

\ba
\bn


\item Berechne:

\bt{ll} a) $\D{\bigl(\frac{xy^4 z^5}{(xy^2)^{-1}z}\bigr)^2= }$
\qquad \qquad & b) $(a^2 b^3 c^4)^3 \cdot (a^{-4} b^{-3}
c^{-2})^3=$ \et


\item Berechne:

\bt{lll} a) $(x-5)^2= $ &  b) $(7a+2b)(-a+3b)-(2a-2b)^2=$ \\
          c) $\D{(\frac{1}{7}c-\frac{3}{4}b)^2=}$ &  d) $\D{(4x+\frac{1}{2}y)^2-5(4xy+y)(4xy-y)=}$\\
          e) $\D{(\frac{5}{6}u^2 v^3+\frac{4}{7}z^6)(\frac{5}{6}u^2 v^3-\frac{4}{7}z^6)=} $
          & f) $(2s-2)^3=$ \\
          g) $(10x^5 y^7 -11)(10x^5 y^7 +11)=$ & h) $[(x^2 +1)(x^2 -1)]^2=$ \\ f) $(4x^2+2y^3)^3=$ & \et

\item Berechne:

\bt{lll} a) $\D{\frac{5a^2 -5ab}{(a-b)^2}=}$ &
         b) $\D{\frac{x-y}{x+y}+\frac{x+y}{x-y}=}$ &
         c) $\D{\frac{x+6}{(x-3)^2}+\frac{x}{x^2-9}+\frac{2}{x+3}=}$
         \\[0.2cm]
         d) $\D{\frac{2x-50x^3}{25x^2 -10x+1}=}$ &
         e) $\D{\frac{b}{b^2 -a^2}-\frac{a}{b^2 -a^2}=}$ &
         f) $\D{\frac{a^2}{a-b}-\frac{4ab^3}{(a^2-b^2)(a+b)}-\frac{b^2(a-b)}{(a+b)^2}=}$
         \\[0.2cm]
         g) $\D{\frac{x^2 -8x+16}{3x^2 -48}=}$ &
         h) $\D{\frac{xy}{xy-y^2}(x^2-xy)=}$ &
         i) $\D{\frac{35z}{49z^2-4}-1+\frac{14z-1}{14z+4}=}$
         \\[0.2cm]
         j) $\D{\frac{a^2+1}{a^2-1}:\frac{a+1}{a-1}=}$ &
         k) $\D{\frac{5y-5x}{3y-2x}:\frac{x^2-y^2}{4x^2-9y^2}=}$ &
         l) $\D{\frac{x^2+4y^2}{x^2-4y^2}:(\frac{x}{x-2y}-\frac{2y}{x+2y})=}$ \et

\item Bestimme den Definitionsbereich und berechne die
L�sungsmenge:

\bt{ll}  a) $\D{\frac{6x-3}{x}=5}$ &
         b) $\D{\frac{4}{x-2}-\frac{3}{x-1}=\frac{1}{x}}$ \\[0.2cm]
         c) $\D{\frac{5x}{x-2}-\frac{x}{x+2}=4}$ &
         d) $\D{\frac{x}{x^2-6x+9}-\frac{1}{x^2-3x}=\frac{1}{x}}$ \\[0.2cm]
         e) $\D{\frac{x}{x-4}-\frac{x}{x+4}=0}$ &
         f) $\D{\frac{x+2}{2-x}+\frac{x-2}{2+x}=\frac{4}{4-x^2}}$ \\[0.2cm]
         g) $\D{\frac{x+3}{2x-4}=\frac{x+9}{2x}}$ &
         h) $\D{\frac{x-1}{(x+1)^2}=\frac{1}{x-1}-\frac{2}{x^2-1}}$  \et

\item Bestimme Definitionsbereich und L�sungsmenge:

\bt{llll} a) $5(x-10)\leq \frac{10x+5}{7}+38 \qquad$ &
          b) $\frac{x}{x-3}<0 \qquad$ &
          c) $\frac{2x-3}{4-x}>2 \qquad$ &
          d)$\frac{4x+1}{x-5}\leq 7 $
\et

\item Berechne die Funktionsvorschrift der beiden Geraden $g$ und
$h$, die durch den Punkt $P(4,-3)$ gehen und parallel bzw.
senkrecht zur Geraden $j$ sind, welche durch die Punkte $P_1
(2,-1)$ und $P_2 (-2,5)$ erzeugt wird. Berechne weiters die
Nullstellen aller drei Geraden und �berpr�fe, ob
$(-\frac{2}{3},3)$ auf einer dieser Geraden liegt. Zeichne alle
drei Geraden.

\item 60000 \euro \quad sollen unter drei Preistr�gern derart verteilt
werden, dass auf den zweiten Preis $\frac{2}{3}$ des ersten
Preises und auf den dritten die H�lfte des zweiten Preises
entfallen. Welche Betr�ge entfallen auf die drei Preise?

\item Wie viel kg einer $22\%-$igen Salzl�sung sind zu 6 kg einer
$15\%$-igen Salzl�sung hinzuzuf�gen, um eine $19\%$-ige Salzl�sung
zu erhalten?

\item L�se die folgenden Gleichungssysteme:

$(a)\;%
\begin{array}{c}
\frac{x+2}{3}-\frac{3y-5}{4}=\frac{y+3}{6} \\
\frac{2x+13}{7}=\frac{3x-2}{5}-\frac{y-10}{7}%
\end{array}
$ 

$ (b)\;%
\begin{array}{c}
\frac{4}{x}-\frac{5}{y}=0 \\
-\frac{2}{x}+\frac{1}{y}=4%
\end{array}
$

$ (c)\;%
\begin{array}{c}
2x+2y+2z=1 \\
5x+6y=2 \\
-3x-4y-3z=-1%
\end{array}%
$
\en
\ea
%\input{2te/linearegleichungssysteme/linearegleichungssysteme.tex}
% \input{2te/funktion/funktion.tex}
%\chapter{Lineare Funktionen}
Nachdem wir den Begriff der Funktion allgemein untersucht haben,
beginnen wir in diesem Abschnitt mit der n\"{a}heren Betrachtung
einzelner Funktionstypen. Die einfachste dieser Funktionenklassen
sind die {\it linearen Funktionen}.










\section{Definition und Eigenschaften}
\subsection{Begriff und Beispiele}
\bd 
Eine Funktion $f$ nennt man {\bf linear}, wenn ihre
Funktionsgleichung auf die Form $$y=k\cdot x + d \qquad \qquad (k, d \in
\mathbb{R})$$ gebracht werden kann.
 \ed  



\ba  Entscheide, welche der folgenden Funktionsgleichungen
linear sind und bestimme gegebenenfalls $k$ und $d$.
\[\begin{array}{lll}
  a.)\quad 3x-4y =7 &   b.)\quad x=y-3x+6 &   c.)\quad \frac{1}{y}=\frac{x}{y}+4 \\
  d.)\quad y=-\frac{2x}{3}+\frac{4}{7} &   e.)\quad y=2x^2-3&   f.)\quad
  y=6-x
\end{array}\]
\ea 

\subsection{Eigenschaften}
Lineare Funktionen sind f\"{u}r uns nicht neu. Sie sind uns schon im
vorhergehenden Abschnitt begegnet und wir haben in Beispielen und
Hausaufgaben sogar schon sechs Graphen von linearen Funktionen
gezeichnet.
\begin{longtable}{p{5cm}p{5cm}p{5cm}}
  Graph 1: $y=-\frac{x}{3}$ &
  Graph 2: $y=\frac{7x-3}{2}$ & 
   Graph 3: $y=\frac{1}{2}x$ \\
      (Bsp. \ref{2.1} Seite \pageref{2.1}) &
   (Auf. \ref{2.2} Seite \pageref{2.2}) & 
  (Bsp. \ref{2.3} Seite \pageref{2.3})\\
   
\includegraphics[width=4.5cm]{2te/linearefunktion/bilder/lf1.jpg} & \includegraphics[width=4.5cm]{2te/linearefunktion/bilder/lf2.jpg}&
\includegraphics[width=4.5cm]{2te/linearefunktion/bilder/lf3.jpg} \\

Graph 4: $y=-2x+1$ 
  & Graph 5: $y=-x$  &
  Graph 6: $y=-x+5$  \\
 (Bsp. \ref{2.4} Seite \pageref{2.4})&
  (Auf. \ref{2.5} Seite \pageref{2.5}) &
   (Bsp. \ref{2.6} Seite \pageref{2.6})  \\
   
   \includegraphics[width=4.5cm]{2te/linearefunktion/bilder/lf4.jpg}&
  \includegraphics[width=4.5cm]{2te/linearefunktion/bilder/lf5.jpg} & \includegraphics[width=4.5cm]{2te/linearefunktion/bilder/lf6.jpg}\\
\end{longtable}



\ba {\bf Computerraum}

Wir haben nun die Aufgabe festzustellen, wie die Graphen der
linearen Funktionen von $k$ und $d$ abh\"{a}ngen. 

�ffne dazu die Datei {\tt ../materialien/linearefunktion.gxt} und arbeite den Arbeitsauftrag sorgf�ltig durch:

\begin{enumerate}

\item Stelle zun�chst f�r die Konstanten (falls nicht schon
eingestellt) folgende Werte ein: $k=0,d=0$. �ndere nun mit der
Maus den Wert f�r $d$. Beobachte, was sich genau ver�ndert und
halte deine Beobachtungen fest!

\item Setze nun $k=1$ und ver�ndere den Wert f�r
$d$. Halte wie oben deine Beobachtungen fest! 

\item $d$ wird "`$y-$Achsenabschnitt"' genannt. Warum glaubst du, ist das so?

\item Setze $d$ auf den Wert $0$. Ver�ndere nun den Wert $k$ und
halte alle deine Beobachtungen fest! Gehe besonders auf die nachfolgenden Fragen ein

\begin{enumerate}
\item Was passiert f�r ein positives bzw. negatives $k$?
\item Gibt es eine M�glichkeit anhand der Steigung der Geraden zu erkennen, wie gro� $k$ gew�hlt wurde? 
\item $k$ wird "`Steigung"' genannt. Wie k�nnte Steigung, auch im Alltagsleben, definiert sein (z.B. im Sta�enverkehr)?

\end{enumerate}

\item W�hle nun selbst einige Werte f�r $k$ und $d$. Halte deine Beobachtungen fest.

\item �berpr�fe, ob �berall \textbf{ausf�hrlich in ganzen S�tzen}
geantwortet wurde. Versuche schlie�lich die Antworten auf obige
Fragen auf eine Seite (maximal zwei) zu bringen und in einer
ansehnlichen Form darzustellen. Drucke die Seite(n) aus!

\item \textbf{F�r all jene, die fr�her fertig werden:} �berlege dir eine Methode, mit der man erstens ohne Berechnung einer Wertetabelle den Graphen einer linearen Funktion finden kann, und zweitens anhand eines Graphen die Funktionsvorschrift bestimmen kann.

\end{enumerate}




\ea

Die Ergebnisse dieser Beobachtung formulieren wir als
Merksatz:

\bme \label{3satz}
\begin{enumerate}
\item Der Graph einer linearen Funktion ist immer $\dots$.
\item Die Steigung dieser Geraden h\"{a}ngt von der Variablen $k$ ab:
\begin{enumerate}
\item Ist k positiv ($k>0$), so $\dots$
\item Ist k negativ ($k<0$), so $\dots$
\item Ist $k =0$, so $\dots$
\end{enumerate}
\item  Je gr\"{o}{\ss}er $|k|$ ist, umso $\dots$ \hspace{2cm} verl\"{a}uft die Gerade.
\item Der Graph schneidet die y-Achse im Punkt $\dots$
\item Wenn $d=0$ ist, die lineare Funktion also die Form $y= \dots$ hat,
so verl\"{a}uft der Graph immer durch den $\dots$.

\end{enumerate}
\eme

 \bd  
\bi \item Gegeben ist eine lineare Funktion $f(x)=kx+d$. Die Variable $k$ nennt man {\bf Steigung}, die Variable $d$
ist der {\bf y-Achsenabschnitt} des Graphen.
\item 
Den Schnittpunkt der x-Achse mit der y-Achse in einem
Koordinatenkreuz nennt man {\bf Ursprung} oder {\bf Nullpunkt}
$(0/0)$.\ei
 \ed  




\ba 

\bn \item  Welcher Graf geh�rt zu welcher Gleichung?

	\begin{minipage}{6cm}
	\bi \item $f_1(x)=3x-4$;\item $f_2(x)=-3x-4$; \item $f_3(x)=\frac13x-4$; \item $f_4(x)=-\frac13x$; \item $f_5(x)=-4$\ei
	\end{minipage} 
	\begin{minipage}{8cm}
	\begin{center}
		\includegraphics[width=8cm]{2te/linearefunktion/bilder/200502.jpg}
		\end{center}
	\end{minipage}


	\item "`9 Bildchen"'

	\begin{minipage}{6cm}
	$k>0$, $k=0$, $k<0$ sind 3 verschiedene F�lle. Das gleiche gilt f�r $d$. Somit gibt es $9$ verschiedene M�glichkeiten. Zeichne alle Beispiele und beschrifte aussagend.
	\end{minipage} 
	\begin{minipage}{8cm}
	\begin{center}
		\includegraphics[width=8cm]{2te/linearefunktion/bilder/200501.jpg}
		\end{center}
	\end{minipage}


\item \bn \item Es sei $k$ eine fix gew�hlte Variable. Was haben alle m�glichen Funktionsgrafen der Funktionen $y=kx+d$ gemeinsam? Zeichne ein Schaubild.
\item Es sei $d$ eine fix gew�hlte Variable. Was haben alle m�glichen Funktionsgrafen der Funktionen $y=kx+d$ gemeinsam? Zeichne ein Schaubild.
\en \en
\ea 




\subsection{Der Graph der linearen Funktion}
\subsubsection{Praktische Zeichnung von linearen Funktionsgraphen}

Da der Graph einer linearen Funktion immer eine Gerade ist, so
ben\"{o}tigt man zum genauen Zeichnen des Graphen nur {\it 2 Punkte},
denn eine Gerade ist durch 2 Punkte ja eindeutig festgelegt.

Dabei brauchen wir effektiv nur mehr einen Punkt berechnen, da
einer schon aus der Funktionsgleichung herausgelesen werden kann,
n\"{a}mlich: ($\dots$). (Warum?)

\bb  
Zeichne den Graphen von $y=\frac{x}{2}-4$
\eb  


\subsubsection{Besondere Graphen}

 \bd  \label{defdsdf}
\bi \item Eine Funktion $f$ hei�t {\bf konstant}, falls $f(x)=c$ mit $c \in \mathbb{R}$ ist.

\item Die {\bf Betragsfunktion} $$f(x)=|x|$$ ordnet jedem Zahl ihren Abstand vom Nullpunkt auf dem Zahlenstrahl zu.

\item Die {\bf Signumfunktion} ist die sog. Vorzeichenfunktion, sie ordnet allen positiven Zahl die Zahl $+1$, den negativen die Zahl $-1$ und der Zahl $0$ die $0$ zu.

\item Die {\bf Treppenfunktion} $f(x)=[x]$ entsteht dadurch, dass jeder Zahl ihr "`Gr��tes Ganzes"' zugeordnet wird. \ei
 \ed  

\ba Zeichne jeweils einen bzw. den Funktionsgraf zu Definition \ref{defdsdf}.\ea


\subsection{Nullstellen und lineare Gleichungen}

Wir haben schon in der Einf\"{u}hrung von Derive gesehen, dass jede
beliebige Gleichung zeichnerisch mit Hilfe des Funktionsgraphen
gel\"{o}st werden kann: die L\"{o}sung(en) der Gleichung ergibt bzw.
ergeben sich als Nullstelle(n) des Funktionsgraphen.

Der Vorteil dieser f\"{u}r die Mathematik wichtigen Erkenntnis liegt
darin, dass damit eine Methode bekannt ist, um komplizierte
Gleichungen zumindest n\"{a}herungsweise zu l\"{o}sen. Denn es gibt
Gleichungen, die mit algebraischen Umformungsmethoden nicht exakt
gel\"{o}st werden k\"{o}nnen. So k\"{o}nnen beispielsweise bei Gleichungen
h\"{o}heren Grades (z.B. $3x^5-3x^3+4x=9$) die ersten Schwierigkeiten
auftreten.

In diesem Kapitel hat die graphische Methode noch nicht eine so gro{\ss}e Bedeutung, da ja die Berechnung von
Nullstellen einer linearen Funktion auf eine \hspace{3cm} Gleichung f\"{u}hrt, die ja mit den Methoden der ersten
Klasse einfach gel\"{o}st werden kann.


\ba 
Ermittle die Nullstellen von $y=\frac14 x-2$ zeichnerisch und
\ea 

Die letzte Aufgabe und das letzte Beispiel f\"{u}hren auf die Frage,
ob eine lineare Funktion immer eine Nullstelle hat und wenn ja, ob
es immer genau eine ist:

\bs  \label{sdfsfgg}
Eine lineare Funktion $y=kx+d$ besitzt
\begin{enumerate}
\item genau eine Nullstelle, falls $\dots$
\item unendlich viele Nullstellen, falls $\dots$
\item keine Nullstelle, falls $\dots$
\end{enumerate}
\es  

Diesen Satz \ref{sdfsfgg} kann man sehr sch\"{o}n graphisch interpretieren:

\begin{center}
\begin{pspicture}(15,4)
\mygrid
\end{pspicture}
\end{center}



\subsection{Berechnung der Funktionsgleichung}
In der Praxis treffen wir selten auf Funktionsgleichungen. Dagegen
sind sehr oft zwei Gr\"{o}{\ss}en gegeben, die voneinander abh\"{a}ngen
($\rightarrow$ Funktionen!). Dabei sind konkret meistens Werte
gegeben, also mathematisch gesehen Zahlenpaare $(x,y)$, mit denen
man dann evtl. ein Schaubild oder Diagramm ($\rightarrow$
Funktionsgraph) zeichnet, oder mit denen man andere Werte
ausrechnen soll ($\rightarrow$ Funktionswerte/Nullstellen).

Wenn es um konkrete Beispiele aus der Praxis geht, dann k�nnen verschiedene Fragestellungen sehr einfach mit Hilfe einer vorhandenen Funktionsgleichung beantwortet werden.\bn \item Es kann zu jedem beliebigen $x$ das entsprechende $f(x)$ sehr einfach durch
Einsetzen des Wertes in die Funktionsgleichung berechnet werden. \item es kann zu jedem gegebenen $y$ durch
das Aufl\"{o}sen der (linearen!) Gleichung das entsprechende $x$ berechnet werden.\en

Diesen Sachverhalt soll das folgende (in Wahrheit etwas vereinfachte Beispiel - in der Praxis ist eine
"`Kostenfunktion"' sehr wahrscheinlich nicht linear) Beispiel noch einmal unterstreichen:

\bb Ein Betrieb produziert Schildm\"{u}tzen  und hat dabei (fixe und variable) Kosten, die von den produzierten
St\"{u}ckzahlen abh\"{a}ngen. Im letzten Monat hat die Produktion von $5.000$ St\"{u}ck Kosten in H\"{o}he von $30.000$ \euro
\, ergeben. Im Vormonat ergaben sich Kosten in H\"{o}he von $36.000$ \euro bei einer Produktion von $6.000$
St\"{u}ck. \bn \item Wie hoch sid die Kosten bei einer Produktion von $5.000$ M\"{u}tzen? \item Wie viele M\"{u}tzen
k\"{o}nnen mit einem Budget von $50.000$ \euro \, produziert werden? \en\eb


An dieser Stelle soll gezeigt werden, wie man aus gegebenen Werten die Funtionsgleichung berechnet. Im
n\"{a}chsten Abschnitt treffen wir auf zahlreiche Anwendungsgebiete von linearen Funktionen.

Setzen wir die zwei gegebenen Punkte $(x_1/y_1)$ und $(x_2/y_2)$  in die Funktionsgleichung $y=kx+d$ ein, so
ergeben sich $2$ (lineare) Gleichungen mit $2$ Unbekannten, n\"{a}mlich $k$ und $d$. Die Aufgabe lautet nun aus
den gegebenen $2$ Gleichungen die $2$ Unbekannten auszurechnen. 

\bb \label{bboben} W�hle selbst zwei beliebige Punkte $P_1$ und $P_2$. Berechne die lineare Funktion, deren Graph durch diese zwei Punkte verl�uft. Kontrolliere dein Ergebnis anhand einer Zeichnung.\eb


%Dies f\"{u}hrt uns auf das folgende Kapitel: \input{lin_gs.tex}

Ist die Funktionsgleichung $y=kx+d$ gesucht, dann hei{\ss}t dies, dass
die Steigung $k$ und der y-Achsenabschnitt $d$ zu berechnen sind.
Eine erste Hilfe bietet dabei der folgende Satz:

\bs  \label{ssoben}
Sind zwei Punkte $(x_1/y_1)$ und $(x_2/y_2)$ gegeben, so kann die
Steigung $k$ daraus wie folgt berechnet werden:
\[k=\frac{y_2-y_1}{x_2-x_1}\]
\es  
\begin{proof}
1. Geometrisch ({\bf Steigungsdreieck})\hspace{5cm} 2. Rechnerisch 

\begin{center}
\begin{pspicture}(15,5)
\mygrid
\end{pspicture}
\end{center}

\end{proof}


\bb Berechne Beispiel \ref{bboben} mit Hilfe von Satz \ref{ssoben}.\eb





\subsection{Lineare Gleichungssysteme und lineare Funktionen}\label{linfkl_lings}


\subsubsection{Die Funktionsgleichung $y=kx+d$ ist gesucht}
Jedem leuchtet die geometrische Aussage "`Durch zwei Punkte verl\"{a}uft genau eine Gerade"' ein. Diese Aussage
kann demnach dazu benutzt werden, um die Funktionsgleichung zu berechnen, falls zwei Punkte des
Funktionsgraphen bekannt sind:

\bb Welche lineare Funktion verl\"{a}uft durch $P=(-3/2)$ und $Q=(\frac{1}{2}/-\frac{1}{2})$? \eb

Achtung: Falls wir die Funktionsgleichung wie \"{u}blich mit $y=kx+d$ bezeichnen, so sind jetzt nicht $x$ und
$y$, sondern $k$ und $d$ gesucht!

Ist die Steigung $k$ oder der y-Achsenabschnitt $d$ und ein weitere Punkt bekannt, so erh\"{a}lt man durch das
Einsetzen in die Funktionsgleichung eine lineare Gleichung mit einer Unbekannten, die sehr einfach gel\"{o}st
werden kann.

Damit haben wir alle n\"{o}tigen Kenntnisse gesammelt und k\"{o}nnen nun anhand der verschiedenen Aufgaben dieses
neue Wissen ein\"{u}ben:
\ba 
\begin{enumerate}
\item Welche lineare Funktion verl\"{a}uft durch die Punkte $(3/-4)$ und
$(-2/3)$?
\item Berechne die Funktionsgleichung der Funktion mit der
Steigung $\frac{2}{3}$, die die y-Achse bei $-\frac{2}{5}$ schneidet.
\item Welcher Funktionsgraph geht durch den Punkt
$(23,2/\frac{25}{7})$ und hat die Steigung $0,2$?
\item Liegen die folgenden drei Punkte auf einer Geraden?
A=$(1/4)$, B=$(2/1)$ und C=$(3/-1)$ - \"{U}berpr\"{u}fe rechnerisch und zeichnerisch. 

\item Von einer Geraden sind zwei Punkte $P_1=(1/1)$ und
$P_2=(7/-5)$ bekannt.
\begin{enumerate}
\item Zeichne die Gerade
\item Wie lautet die Funktionsgleichung? Versuche diese Frage zuerst
graphisch anhand von a.) n\"{a}herungsweise zu l\"{o}sen und \"{u}berpr\"{u}fe dein Ergebnis rechnerisch.
\item Berechne die Nullstelle - kontrolliere das Ergebnis mit a.).
\end{enumerate}

\item 
{\bf Computerraum} "`St�ckweise definierte Funktionsgrafen"'
Zeichne mit Hilfe einer passenden Software eine "`nette"' Figur. Z.B. Namen schreiben.
\end{enumerate}
\ea 

\subsubsection{Der Schnittpunkt zweier lin. Funktionen  ist gesucht}
Diese Art der Anwendung haben wir schon kennengelernt und muss hier eigentlich nicht mehr wiederholt werden.

\bb {\tiny .}

a) $\quad$ \includegraphics[width=5.5cm]{2te/linearefunktion/bilder/modem.jpg}   \qquad \qquad   b) $\quad$  \includegraphics[width=5.5cm]{2te/linearefunktion/bilder/aol.jpg}
 \eb

 \ba Gegeben sind die Funktionen $f(x)=\frac{2}{3}x+3$ und $g(x)=x-\frac{1}{3}$. Bestimme den Schnittpunkt der
 Funktionsgraphen! \ea

\ba
�bung Computerraum:

�ffne die Seite \begin{verbatim}
http://www.cybernautenshop.de/virtuelle_schule/lehrinhalte_index2.html
\end{verbatim}
und gehe zu den abgebildeten Kapiteln. Gute Arbeit!

\begin{center}
\includegraphics[width=3cm]{2te/linearefunktion/bilder/funktionen.png}
\end{center}

\ea 

\section{Anwendungen der linearen Funktion}
Anhand der folgenden Beispiele soll gezeigt werden wie verbreitet lineare Funktionen sind. Praktisch findet man in
allen Berufen Beispiele f\"{u}r diese Funktionsart.

\bb  
\begin{enumerate}
\item {\bf Proportionalit\"{a}ten}

$3$ kg einer Ware kosten $4,5$ Euro.
\begin{enumerate}
\item Wieviel kosten $7$ kg der Ware?
\item Wie lautet die Funktionsgleichung?
\item Zeichne den Graphen f\"{u}r $x \in\{1, \dots, 10\}$.
\item Kontrolliere dein Ergebnis aus a.) zeichnerisch!
\item Wieviel $kg$ bekomme ich f\"{u}r $10$ Euro. Beantworte die Frage zeichnerisch und rechnerisch!
\end{enumerate}
\item {\bf Prozentrechnung}

Die Mehrwertsteuer betr\"{a}gt zur Zeit $20\%$ vom Warenwert. Sie h\"{a}ngt also direkt vom Warenwert ab.
\begin{enumerate}
\item Wieviel Euro betr\"{a}gt die MWSt., wenn der Warenpreis $20$, $30$, $50$, $70$, $85$ bzw. $100$ Euro betr\"{a}gt?
\item Wie lautet die Funktionsgleichung?
\item Zeichne den Graphen bis $x=100$.
\end{enumerate}

\item {\bf Fixe und variable Kosten}

 Ein Gaswerk berechnet f\"{u}r $1m^3$ Gas einen Preis von $0,2$ Euro bei einer monatlichen Grundgeb\"{u}hr von $3$ Euro.
\begin{enumerate}
\item Bilde die Funktionsgleichung, die den Zusammenhang zwischen Gasmenge und Rechnungspreis angibt.
\item Zeichne das Bild der Funktion bis zu einem Verbrauch von $10m^3$.
\item Zu welcher Zahlenmenge geh\"{o}ren die y-Werte?
\item Wieviel bezahlt man bei einem Verbrauch von $16m^3$ Gas? L\"{o}se rechnerisch und zeichnerisch!
\item Wieviel hat man verbraucht, wenn auf der Betrag von $59$ Euro zu bezahlen ist? L\"{o}se rechnerisch und zeichnerisch!
\end{enumerate}

\item {\bf Gegeben: Zahlenpaare}
Das Gehalt eines Vertreters setzt sich aus einem Grundgehalt und einem Prozentsatz an Provision zusammen. Aufgrund
seiner letzten zwei Gehaltsstreifen wei{\ss} er, dass er bei einem Umsatz von $40.000$ Euro ein Gehalt von $4.000$ Euro und
bei einem Umsatz von $20.000$ Euro ein Gehalt von $2.800$ Euro erh\"{a}lt. Kannst du aus diesen Informationen sein
Grundgehalt und seinen Provisionssatz ermitteln?

\item {\bf Bestimmung des Schnittpunktes}
In einem Beh\"{a}lter befinden sich $40$ Liter Wasser. In jeder Sekunde flie{\ss}en $0,5$ Liter hinzu. In einem zweiten
Beh\"{a}lter befinden sich $25$ Liter Wasser. Der Zufluss betr\"{a}gt $1,2$ Liter pro Sekunde. Nach wievielen Sekunden befindet
sich in beiden Beh\"{a}ltern die gleiche Wassermenge und wieviel? L\"{o}se diese Aufgabe zeichnerisch und rechnerisch.
\end{enumerate}
\eb  

\section*{�bungen - Vorbereitung auf die Schularbeit}
\ba 
\begin{enumerate}
\item $1$ Dollar kostet $1,5$ Euro.
\begin{enumerate}
\item Formuliere den Zusammenhang zwischen den W\"{a}hrungen als lineare Funktion.
\item Zeichne den Graphen dieser Funktion.
\item Wie viel Dollar bekommt man f\"{u}r $6$ Euro. Beantworte diese Frage zeichnerisch und rechnerisch.
\item Wie viel Euro bekommt man f\"{u}r $30$ Dollar? Beantworte diese Frage ebenfalls zeichnerisch und rechnerisch.
\end{enumerate}
\item Die Zinsen h\"{a}ngen direkt vom Kapital ab. Angenommen der Zinssatz betr\"{a}gt $5\%$, so bekommt man f\"{u}r ein Kapital von
$100$ Euro $5$ Euro an Zinsen.
\begin{enumerate}
\item Formuliere den Zusammenhang zwischen Kapital und Zinsen als lineare Funktion.
\item Zeichne den Graphen dieser Funktion.
\item Wie viel betragen die Zinsen bei einem Kapital von $1.600$ Euro? Beantworte diese Frage zeichnerisch und rechnerisch.
\item Bei welchem Kapital bekommt man $50$ Euro an Zinsen? Beantworte diese Frage ebenfalls zeichnerisch und rechnerisch.
\end{enumerate}

\item Die fixen Kosten eines Betriebes betragen monatlich $12.000$ Euro. Der Kostenpreis pro St\"{u}ck des produzierten
Artikels betr\"{a}gt hingegen $6$ Euro.
\begin{enumerate}
\item Wie lautet die Funktionsgleichung, die den Zusammenhang zwischen St\"{u}ckpreis und Gesamtkosten darstellt?
\item Zeichne das Bild dieser Funktion.
\item Wieviel St\"{u}ck k\"{o}nnen produziert werden, damit der gesamte Kostenpreis nicht mehr als $20.000$ Euro betr\"{a}gt.
Beantworte diese Frage zeichnerisch und rechnerisch.
\end{enumerate}
\item Das neue Tarifsystem der SAD f\"{u}r \"{U}berlandfahrten setzt sich aus einer einmaligen Tagesgeb\"{u}hr und ein variabler Betrag
pro gefahrener Kilometer zusammen. Wie hoch ist diese Tagesgeb\"{u}hr und wieviel bezahlt man pro Kilometer, wenn
jemand von St. Martin nach Meran (18km) $1,4$ \euro \, bezahlt und am n\"{a}chsten Tag von Meran nach Lana (9km)
f\"{u}r $0,85$ \euro \, weiterf\"{a}hrt? Wie lautet schlie{\ss}lich die Funktionsgleichung, die den gefahrenen
Tageskilometern den Preis zuordnet?



\item

%\begin{minipage}{6cm}
Bestimme die Gleichungssysteme, welche zu den entsprechenden Grafen passen.
%\end{minipage}
%\begin{minipage}{10cm}
\begin{center}
\includegraphics[width=10cm]{2te/linearefunktion/bilder/scanimage200502.jpg}
\end{center}


%\end{minipage}



\item Ein �ltank enth�lt  $1000$ l Heiz�l. Pro Tag werden $35$ l verbraucht.
\bn \item Stelle die Restmenge $R(t)$, die nach $t$ Tagen �brig ist, als Funktion von $t$ dar.
\item Wie viel �l ist nach $14$ Tagen noch im Tank?
\item Wann ist der Tank leer? \en

\item In einer Stadt waren im Jahr 1990 ca. $7200$ PKW zugelassen, im Jahr 2000 ca. $11700$. Man kann annehmen, dass die Anzahl der PKW linear w�chst.
\bn \item Wie viele PKW werden pro Jahr neu zugelassen?
\item Stelle die Anzahl der PKW als Funktion der Zeit dar (1990 = Jahr 0).
\item Wie viele PKW sind im Jahr 2010 zu erwarten?
\item Wann wird es voraussichtlich $18.000$ PKW geben?\en




\item

%\begin{minipage}{6cm}
Drei verschiedene Geraden k�nnen unterschiedlich viele Schnittpukte miteinander haben. Erstelle f�r
alle vier F�lle ein Gleichungssystem.
%\end{minipage}
%\begin{minipage}{10cm}
\begin{center}
\includegraphics[width=8cm]{2te/linearefunktion/bilder/scanimage200503.jpg}
\end{center}


%\end{minipage}

\item Eine Taxifahrt kostet $2,50$  Euro Grundgeb�hr und $0,96$ Euro pro gefahrenem Kilometer.
\bn \item Stelle den Fahrpreis $F(x)$ als Funktion der Strecke $x$ dar.
\item Wie viel kostet eine $6$ km lange Fahrt?
\item Wie weit kann man mit $10$  Euro  fahren? \en



\item Die Nachbarsh�ndin Senta jagt oft die Katze Minka. 

%\begin{minipage}{6cm}
\bn \item Erfinde sinnvolle Geschichten zu den folgenden Graphen.
\item Stelle zu den drei Abbildungen passende Geradengleichungen auf.
\item Versuche jeweils de Geschwindigkeit von der H�ndin und der Katze zu bestimmen. Wo findest du diese in der jeweiligen Geradengleichung wieder?
\en
%\end{minipage}
%\begin{minipage}{10cm}
\begin{center}

\includegraphics[width=10cm]{2te/linearefunktion/bilder/scanimage200504.jpg}
\end{center}

%\end{minipage}





\item Internetkosten

\bi \item Der Internet-Provider  www.freewebster.com bietet als Superkn�ller die Surfstunde zu $0,50$ Euro an. Allerdings muss man eine monatliche Grundgeb�hr von $5$ Euro bezahlen. 

\item Die Firma www.billigsurfen.com wirbt damit, dass ihr Angebot, eine Surfstunde von $1,50$ Euro pro Stunde und ohne Grundgeb�hr, das g�nstigste Angebot derzeit auf dem Markt sei.
\ei

L�se grafisch und rechnerisch.



\item Ein Elektrizit\"{a}tswerk bietet zwei Tarife an:\\
Tarif1: Grundgeb\"{u}hr $9$ Euro und $0,12$ Euro pro Kilowattstunde.\\ Tarif2: Grundgeb\"{u}hr $12$ Euro und $0,08$ Euro pro
Kilowattstunde.
\begin{enumerate}
\item Bilde beide Funktionsgleichungen!
\item Bei welchen Stromverbrauch ergibt sich der gleiche Rechnungsbetrag. Wann ist also Tarif1 und wann Tarif2
g\"{u}nstiger. Beantworte diese Frage zeichnerisch und rechnerisch!
\end{enumerate}


\item R�tsel:




	\begin{center}
	\includegraphics[width=9cm,angle=90]{2te/linearefunktion/bilder/scanimage200506.jpg}
	\end{center}
	
	\begin{center}
	\includegraphics[width=8cm,angle=90]{2te/linearefunktion/bilder/scanimage200505.jpg}
	\end{center}
\end{enumerate}
\ea 


\subsection*{Noch mehr �bungen (mit L�sugen)}

\ba
.../skripten/2te/linearefunktion/uebungenlinearefunktion.tex
\ea


% \chapter{Die reellen Zahlen}

\begin{minipage}{9cm}
\bd  Die {\bf Quadratwurzel} einer positiven Zahl $a$ ist diejenige {\it positive} Zahl, deren Quadrat gleich $a$ ist. Bezeichnung: $\sqrt{a}$: $$\left(\sqrt{a}\right)^2=a$$\ed  
\end{minipage}
\begin{minipage}{4cm}
\includegraphics[width=4cm]{2te/reellezahlen/bilder/logo2.jpg}
\end{minipage}


\ba \bn \item  {\bf Wiederholung} \bn \item Wof�r stehen die Symbole $\mathbb{N, Z,}$ und $\mathbb{Q}$? \item Was sind periodische Dezimalzahlen? \item Wie k�nnen endliche, wie periodische Dezimalzahlen als Bruch dargestellt werden?\en 


\item Vereinfache ohne Taschenrechner:
$$\begin{array}{lll}
a.) \quad \sqrt{64}= \qquad & \qquad \qquad & b.) \quad \sqrt{\frac49}= \qquad \\
c.) \quad \sqrt{\frac{32}{18}}= \qquad & \qquad \qquad & d.) \quad \sqrt{6+\frac14}= \qquad \\
e.) \quad \sqrt{2,25}= \qquad & \qquad \qquad & f.) \quad \sqrt{0,\overline{1}}= \qquad \\
\end{array}$$
\item Warum muss $a$ positiv sein?
\item Wenn jeder positiven Zahl ihre Wurzel zugeordnet wird, dann handelt es sich ja wieder um eine Funktion. Nenne Definitions- und Wertemenge dieser Funktion und zeichne ihren Grafen.
\en \ea


\section{Die irrationalen Zahlen}
\subsection{Die Zahl $\sqrt{2}$}

\begin{minipage}{7cm}\bb Gegeben ist ein Quadrat mit der Seitenl�nge $1$. Wie lang ist eine Diagonale?\eb
\end{minipage}
\begin{minipage}{5cm} \end{minipage}

\vspace{1.5cm}

\ba {\bf �berlegungen zu $\sqrt{2}$} Was verbirgt sich nun hinter $\sqrt{2}$? Wie sieht die Zahl aus? Wie kann man sie hinschreiben? Kann man sie hinschreiben???

\bn \item  Versuche $\sqrt{2}$ m�glichst gut mit einem Bruch auszudr�cken. Wer schafft es auf m�glichst viele Stellen genau?\footnote{Mein Bruch, den es zu "`schlagen"' gilt: $\frac{11}{8}$.}

\item Der Taschenrechner: Vielleicht hilft uns der Taschenrechner? Berechne die $\sqrt{2}$ mit Hilfe des Taschenrechners, notiere das Ergebnis und mache die Probe. Was stellst du fest?

\item Der Zahlenstrahl: Wo befindet sich die Zahl auf dem Zahlenstrahl? Kannst du die Stelle geometrisch exakt einzeichnen? 

\vspace{2cm}


\vspace{1cm}
\en
\ea


\bs  $\sqrt{2}$ kann nicht als Bruch dargestellt werden, also handelt es sich nicht um eine rationale Zahl, kurz: $$\sqrt{2} \in ...$$
\es 

\begin{minipage}{10cm}
\begin{proof}Beispiel eines {\it indirekten Beweises}. Man nimmt eine Aussage als wahr an, folgert daraus eine Widerspruch und hat damit gezeigt dass die angenommene Aussage falsch war.

Hier hei�t die Aussage: Es gibt einen Bruch $\frac{a}{b}$ mit $a, b \mathbb{N}$ geben, so dass gilt:  $$\frac{a}{b}=\sqrt{2}$$
Wir wissen: jede nat�rliche Zahl kann eindeutig in ein Produkt von Primfaktoren zerlegt werden. Vergleicht man die Anzahl der Primzahlen von $2a^2$ und $b^2$, so ist sie nicht gleich! Also, kann die Gleichung $\frac{a^2}{b^2}=2$ nicht wahr sein und damit war die Aussage falsch. Also: $\sqrt{2}$ ist keine rationale Zahl. 
\end{proof}
\end{minipage}
\begin{minipage}{5cm}
	\begin{center}
	\includegraphics[width=5cm,bb=0 0 501 489]{2te/reellezahlen/bilder/indirekterbeweis.jpg}
% indirekterbeweis.jpg: 83dpi, width=15.33cm, height=14.96cm, bb=0 0 501 489
	\end{center}
\end{minipage}



\begin{minipage}{10cm}
\begin{Folg}
Man kann $\sqrt{2}$ nicht als endliche Dezimalzahl hinschreiben, lediglich n�herungsweise:
\bn \item mit Hilfe eines Taschenrechners (... Stellen): ...
\item Gedicht von Aigner: daneben findest du ein Gedicht von Prof. Alexander Aigner. Damit kann die Quadratwurzel von $2$ auf $42$ Stellen genau angegeben werden:

Jedes Wort hat eine Buchstabenzahl, die die Ziffer an der entsprechenden Dezimalstele angibt. Ist die Buchstabenanzahl gr��er als $9$, musst du solange $10$ abziehen, bis eine Zahl kleiner als $10$ herauskommt. (Sonderzeichen nicht mitz�hlen). Vor dem Komma steht eine $1$, da "`O"' ein Buchstabe ist. Nach dem Komma folge eine $4$, da "`kann"' aus $4$ Buchstaben besteht.
 
\item Computer (Millionen von Stellen) (z.B. mit Mupad).

\en\end{Folg}
\end{minipage}
\begin{minipage}{5.2cm}
\begin{flushright}
\includegraphics[width=5.2cm]{2te/reellezahlen/bilder/aigner.jpg}
\end{flushright}
\end{minipage}

\paragraph*{Fazit:}

\bs  Alle nat�rlichen Zahlen, die keine Quadratzahlen (z.B. $1, 4, 9, 16, 25, ...$) sind, haben keine rationale Quadratwurzel.\es 
 


\bd  \bi \item Eine {\bf irrationale Zahl} ist eine unendliche, nicht periodische Dezimalzahl.
\item Eine {\bf reelle Zahl} ist eine (endliche oder unendliche, periodische oder nicht periodische Dezimalzahl. 
 \item Die Menge alle reellen Zahlen wird mit $\mathbb{R}$ bezeichnet.\ei\ed 


\ba
 \bn \item Berechne (ohne Taschenrechner):

\bt{llllllll}
 a) & $\sqrt{49} $ & b) & $\sqrt{\frac{1}{16}} $ &
  c) & $\sqrt{\frac{50}{72}} $ & d) & $\sqrt{7+\sqrt{81}} $ \et
\item Stelle fest, ob folgende Zahlen rational oder irrational
sind und kreuze an!

\bt{l|l|c|c}
 & Zahl & rational & irrational \\ \hline
 a) & $5,121212\dots$ & $\Box$ & $\Box$ \\
 b) & $5,121221222122221\dots$ & $\Box$ & $\Box$   \\
 c) & $5,121$ & $\Box$ & $\Box$  \\
 d) & $0,343344333444\dots$ & $\Box$ & $\Box$   \\
 e) & $0,343344333444$ & $\Box$ & $\Box$  \\ \et


\item Welche der folgenden Aussagen ist richtig? Kreuze an!

\bt{lll}

a)& Jede nat\"{u}rliche Zahl ist eine reelle Zahl.& $\Box$ \\

b)& Jede nat\"{u}rliche Zahl ist eine irrationale Zahl. & $\Box$ \\

c)& Jede rationale Zahl ist eine irrationale Zahl. & $\Box$ \\

d)& Jede ganze Zahl ist eine rationale Zahl. & $\Box$ \\

e)& Jede ganze Zahl ist eine reelle Zahl. & $\Box$ \\

f)& Jede rationale Zahl ist eine reelle Zahl. & $\Box$ \\

g)& Jede nat\"{u}rliche Zahl ist eine rationale Zahl. & $\Box$ \\

h)& Jede reelle Zahl ist eine irrationale Zahl. & $\Box$ \\

i)& Jede reelle Zahl ist eine rationale Zahl. & $\Box$  \et 

 \item Zeichne rechts von obiger Definition ein aussagekr�ftiges Mengendiagramm f�r $\mathbb{N, Z, Q, R}$ und zeichne jeweils ein Zahlenbeispiel an.  \item Beweise, dass  $\sqrt{8}$ irrational ist. \item Warum funktioniert der Beweis f�r $\sqrt{4}$ nicht?
 \item {\it In jedem Intervall $[a, b]$ mit $a, b \in \mathbb{R}$ gibt es unendlich viele irrationale Zahlen!}

Was h�ltst du von dieser Aussage?

\item Die irrationalen Zahlen historisch betrachtet: seit wann etwa sind irrationale Zahlen anerkannt? Wer leistete dabei Beachtliches?  Wie gingen die antiken Griechen (Pythagoras und andere) mit diesem Ph�nomen um? Recherchiere dazu im Internet.

\item Gehe auf die Seite \href{http://www.mathe-online.at/tests/zahlen/zahlenmengen.html}{http://www.mathe-online.at/tests/zahlen/zahlenmengen.html} und f�hre den Multiple Choice Test mit Mehrfachantworten zum Thema Zahlenmengen durch.
\en 
\ea


\section{Darstellung einer irrationalen Zahl}

\subsection{N�herungsverfahren}
\paragraph{Kostruieren und Ablesen}:

\begin{minipage}{7cm}
Wie kann man $\sqrt{n}$ mit $n \in \mathbb{N}$ konstruieren? Die Antwort gibt die nebenstehende Grafik und der Satz des Pythagoras. Der Haken dieser Konstruktion: es muss die $\sqrt{n-1}$ schon konstruiert sein. Man spricht dabei von {\it rekursiv}.
\end{minipage}
\begin{minipage}{8cm}
\includegraphics[width=2.5cm]{2te/reellezahlen/bilder/konstruktionwurzel.jpg}
\end{minipage}


\begin{minipage}{7cm}
(Siehe */materialien/RechtecksflaecheQuadratWurzelfunktion.gxt). Sucht man zu einem gegebenen Rechteck mit der Fl�che $F$ ein fl�chengleiches Quadrat, so kann diese Aufgabe konstruktiv mit Hilfe des H�hensatzes gel�st werden. Dabei ergibt sich f�r die Seitenl�nge $s$ des Quadrates: $$s=\sqrt{A}$$
Welche Ortskurve erh�lt man nun, wenn die Rechtecksbreite gleich $1$ (eins) gew�hlt wird?
\end{minipage}
\begin{minipage}{8cm}
\includegraphics[width=8cm]{2te/reellezahlen/bilder/rechtecksflaeche.png}
\end{minipage}

\paragraph{Intervallschachtelung}
\ba {\bf Computerraum} Schau dir die Pr�sentation "`Intervallschachtelung"' konzentriert an.\ea

Das {\bf Intervallschachtelungsprinzip} bildet in der Numerischen Mathematik die Grundlage f�r einige L�sungsverfahren.

Das Prinzip ist einfach: Man f�ngt mit einem Intervall an und "`nimmt"' sich aus diesem Intervall ein Intervall, das komplett in dem vorherigen Intervall drin liegt, und "`nimmt"' sich dort wieder ein Intervall raus und so weiter. Die Intervalle sind ineinander verschachtelt. Durch die unendliche Hintereinanderausf�hrung zieht sich das Intervall immer mehr zusammen bis es nur noch ein einziger Punkt ist.  Eine Strategie k�nnte sein, das Intervall fortlaufend zu {\it halbieren}, dann spricht man vom sog. {\bf Intervallhalbierungsverfahren}.

Der Vorteil dieses Verfahrens ist, dass es recht einfach, der Nachteil, dass es sehr aufwendig ist!
	\begin{center}
	\includegraphics[width=14cm,bb=0 0 829 273]{2te/reellezahlen/bilder/intervallschachtelung1.jpg}
% intervallschachtelung1.jpg: 83dpi, width=25.37cm, height=8.35cm, bb=0 0 829 273
	\end{center}

\bd  Die (unendlich vielen) Intervalle rationaler Zahlen bilden eine {\bf Intervallschachtelung}, wenn gilt:
\bn \item Jedes Intervall ist im vorangehenden enthalten. \item Die Intervalll�ngen nehmen ab und werden beliebig klein. \en \ed

\begin{Folg}
Die Intervallschachtelung bestimmt auf der Zahlengeraden genau einen Punkt.
\end{Folg}

\bb F�hre das Intervallhalbierungsverfahren f�r $\sqrt{20}$ durch. Gesucht ist eine auf $5$ Dezimalstellen genaue L�sung.\eb

\ba Bestimme eine Intervallschachtelung f�r $\frac19=0,\overline{1}$.\ea



\paragraph{Das Heron-Verfahren}
Mit diesem Verfahren kann jeder Ausdruck $\sqrt{n}$ (mit $n \in \mathbb{N}$)  auf beliebig viele Dezimalstellen genau angegeben werden.

Schon um $500$ vor Christus wussten die Griechen, wie man Wurzeln berechnet. Heron von Alexandria hat dies im 1. Jahrhundert nach Christus niedergeschrieben und Isaac Newton (1643-1727) hat dieses Verfahren verallgemeinert.
Die Idee ist genauso genial wie einfach.


\bs  Nach Wahl eines geeigneten Startwertes $x_0$ liefert die wiederholte Berechnung
(Iteration\footnote{(lat.): wiederholen}) nach der Vorschrift

$$x_{n+1}=\frac12\left(x_n+\frac{a}{x_n}\right)\qquad \qquad (a \geq 0; \quad n \in \mathbb{N})$$
eine Folge von immer genaueren N�herungswerten f�r $\sqrt{a}$.\es 


\begin{minipage}{8cm}
\begin{proof}Die n�herungsweise Berechnung von $\sqrt{a}$ f�hrt auf das gleiche Problem, wie die
Bestimmung der Seitenl�nge eines Quadrates mit gegebenen Fl�cheninhalt $A$ - dabei sei $A=a$. Um
die Seitenl�nge eines Quadrates mit dem Fl�cheninhalt $A$ zu berechnen, betrachtete Heron eine
Folge von Rechtecken, die alle den Fl�cheninhalt $A$ haben und sich dem gesuchten Quadrat
annh�hern. 
\end{proof}
\end{minipage}
\begin{minipage}{7cm}
\includegraphics[width=7cm]{2te/reellezahlen/bilder/heron.jpg}
\end{minipage}

\bn
\item Es wird eine beliebige Seitenl�nge $x_0$ des Rechtecks gew�hlt.
\item F�r die  andere Seitenl�nge folgt aus $x_0\cdot y_0=A$: $y_0=\frac{A}{x_0}$.
\item Um ein Rechteck zu erhalten, welches dem Quadrat angen�hert ist, bildet man aus den beiden
Seitenl�ngen das {\bf arithmetische Mittel}: $$x_{neu}=\frac12\left(x+y\right)$$
\item F�r $y_2$ folgt dann: $y_2=\frac{A}{x_2}$
\item Diese Schritte 3 und 4 werden nun beliebig oft wiederholt. Die Folge der dabei errechneten
Rechtecksl�ngen n�hert sich immer mehr der gesuchten Quadratsseitenl�nge.
\item So ergibt sich $$x_{n+1}=\frac12\left(x_n+y_n\right)=\frac12\left(x_n+\frac{A}{x_n}\right)$$\en


\bb Berechne mit dem Heron-Verfahren  einen auf $7$ Stellen genauen N�herungswert f�r $\sqrt{2}$:
%
\begin{center}
\begin{tabular}{|c|p{5cm}|p{5cm}|}
  \hline
  % after \\: \hline or \cline{col1-col2} \cline{col3-col4} ...
  $n$ & $x_n$ & $x_{n+1}$ \\
  \hline
  1 &  &  \\
  2 &  &  \\
  3 &  &  \\
  4 &  &  \\
  $\dots$ &  &  \\
     \hline
\end{tabular}
\end{center}

Da sich schon bei $n=4$ die Werte von $x_n$ und $x_{n+1}$ nicht mehr �ndern, ist die gesuchte
Genauigkeit erreicht. \eb



\ba Berechne mit dem Heron-Verfahren einen auf $7$ Stellen genauen N�herungswert f�r $\sqrt{14}$.
\ea


\ba {\bf Computerraum}

\bn \item Tabellenkalkulation

Erstelle ein Tabellenblatt, das es erlaubt f�r $\sqrt{a}$ jedes beliebige $a$ und einen beliebigen
Startwert einzugeben und sofort (mit Hilfe von Formeln) eine ausgef�llte Tabelle (wie oben) mit
N�herungswerten anzugeben. Weiters soll die Genauigkeit angegeben werden. Bei welcher Stellenanzahl
st��t ein Tabellelkalkulation an seine Grenzen?

\item Derive

Derive hat f�r iterative Berechnungen eine vorgegebene Funktion enthalten. - Hier die Syntax:

\begin{center}{\tt Iterates(Iterationsausdruck, Iterationsvariable, Startwert, Anzahl der
Iterationen)}\end{center}

Das Ergebnis ist eine Folge von Br�chen, welche durch "`Approximieren\footnote{Ann�herung}"' als
Dezimalzahlen dargestellt werden.

(Achtung: sei vorsichtig mit der Anzahl der Iterationen - eine hohe Zahl zwingt auch moderne
Rechner in die Knie...)

 
 \bn \item F�hre jetzt einige Berechnungen durch $\sqrt{2}, \sqrt{14}$

 \item Berechne $\sqrt{216}$ mit Mupad auf $15$ Stellen nach dem Komma genau. Variiere dabei die
 Startwerte: $2; 12,6; 14,7; 21,6; 216$.

 Welchen Einfluss hat der Startwert auf Anzahl der notwendigen Iterationen? Wie sollte man den
 Startwert w�hlen?

 \item �berpr�fe mit Mupad die $42.$ Stelle von $\sqrt{2}$ (vgl. mit dem Gedicht von Prof. Aigner:
 "`Die Irrationale"'.)
\en  \en

\ea

\ba Taschenrechner sind pr�zise und machen nie Fehler. So m�chte man meinen. \bn \item  Dann versuche mal 25-mal die Quadratwurzel aus 10 zu ziehen und anschlie�end 25-mal das Ergebnis zu quadrieren, es m�sste gelten:
$(\sqrt{\sqrt{\sqrt{...\sqrt{10}}}})^2)^2)^2)...)^2=10$. Was stellst du fest?
\item Versuche nun ein CAS zu benutzen, was stellst du fest?
\item Wie schauts mit einer Tabellenkalkulation aus?
\en 
Das Problem ist, dass die Stellenanzeige eines jeden elektronischen Ger�tes begrenzt ist, die Zahlen aber oft unendliche Dezimalzahlen sind. In der Praxis merkt man davon meistens nicht. Auf den Seiten 129 bis 132 sind im Buch "`Mathematik f�r Sonntagmorgen"' einige interessante Episoden aufgelistet, wo diese Rundung doch zu Problemen f�hrte: Raketen, Wetter, Wirtschaft, ... - in unserer Bibliothek.
\ea



\ba F�r Profis - h�here Wurzeln 
\bn \item Bestimme $\sqrt[3]{16}$ auf $100$ Stellen genau.
Schreibe die letzten $10$ Stellen auf. Benutze dabei die Iterationsformel
$$x_{n+1}=\frac13\left(2x_n+\frac{a}{x_n^2}\right)$$ von Heron zur n�herungsweisen Berechnung der $\sqrt[3]{a}$.
\item Welche Zahl kann man m�glicherweise mit folgender Iterationsvorschrift von Heron
bestimmen?$$x_{n+1}=\frac14\left(3x_n+\frac{35}{x_n^3}\right)$$
\item Verfolge den Gedanken von Heron. Gib eine Iterationsvorschrift f�r die $\sqrt[11]{a}$ an. \item Gib
eine Iterationsvorschrift f�r die $\sqrt[m]{a} \quad (m \in \mathbb{N})$ an.
 \en \ea




\subsection{Exakte Berechnung}

\begin{minipage}{11cm}
So wie es einen "`Divisionsalgorithmus"' (Rechenverfahren) zum Dividieren von zwei Zahlen gibt, so gibt es auch ein Rechenverfahren zum Ziehen der Quadratwurzel? Hast du es vielleicht gelernt?
Informiere dich im Internet (z.B. \href{http://tinohempel.de/mathe.htm}{http://tinohempel.de/mathe.htm}) dar�ber - erlerne dieses Verfahren (zum Spa� - denn zeitgem�� ist nat�rlich das Arbeiten mit dem Taschenrechner).
\end{minipage}
\begin{minipage}{3cm}
\includegraphics[width=3cm]{2te/reellezahlen/bilder/namenlos.jpg}
\end{minipage}




\section{Rechnen mit Wurzeln}

Generell l\"{a}sst man Wurzeln innerhalb einer Rechnung fast immer
stehen, d.h. man ersetzt z.B. $\sqrt{2}$ nicht durch einen
N\"{a}herungswert der unendlichen Dezimalzahl $1,414213\dots$, sondern
betrachtet das Symbol $\sqrt{2}$ als Abk\"{u}rzung derselben. Wir
d\"{u}rfen also von der (irrationalen) Zahl $\sqrt{2}$ sprechen. Durch
die Multiplikation von $\sqrt{2}$ mit $6$ entsteht so z.B. die
Zahl $6 \sqrt{2}$. Durch anschlie{\ss}ende Addition von $7$ ergibt
sich die Zahl $7+ 6 \sqrt{2}$. Erst als Endergebnis kann man (muss
aber nicht) die ungef\"{a}hre Dezimalzahl anf\"{u}hren: $7+6 \sqrt{2}
\approx 15,48528$


\bs  {\bf Wurzelgesetze} F�r alle $a, b \in \mathbb{R}_0^+$ gilt: 

\bn \item Definitionsmenge von $\sqrt{a}$: $a \in \mathbb{R}^+_0$

\item $\sqrt{a^2}=\left(\sqrt{a}\right)^2=a$ falls $a\geq0$ ist!

\item Addieren und Subtrahieren: (nur {\bf gleiche} Wurzeln!):
$$m\cdot \sqrt{a}\pm n \cdot \sqrt{a}=(m \pm n)\sqrt{a}$$
\item Multiplizieren und Dividieren ($b\neq 0$):
$$\sqrt{a\cdot b}=\sqrt{a} \cdot \sqrt{b} \qquad \qquad \sqrt{\frac{a}{b}}=\frac{\sqrt{a}}{\sqrt{b}}$$

\item Teilweise die Wurzel ziehen:
$$\sqrt{a^2\cdot b}=a \cdot \sqrt{b}$$

\item Nenner rational machen (keine Wurzel im Nenner):

$$\frac{a}{\sqrt{b}}=\frac{a\cdot\sqrt{b}}{\sqrt{b}\cdot \sqrt{b}}=\frac{a\cdot \sqrt{b}}{b}$$


\en\es  

\subsection{Summen (Differenzen) von Wurzeln}
\bb \bt{llll}
 a) & $4\sqrt{2}+\sqrt{2}=(4+1) \sqrt{2}= ......... $ & b) & $-3\sqrt{5}+2\sqrt{5}=(...........) \sqrt{5}=........... $ \\
 c) & $\sqrt{7}-5\sqrt{7}=........$  & d) & $\sqrt{7} -\sqrt{6}=...........$ \et\eb

\subsection{Das Produkt zweier Wurzeln}

\bb \bt{llll}
 a) & $\sqrt{2}\cdot\sqrt{3}=\sqrt{2\cdot 3}=....... $ & b) &
 $\sqrt{2}\cdot\sqrt{8}=......$ \et \eb

Die Wurzel aus einem Produkt kann somit faktorweise gezogen
werden; dies ist einsichtig, wenn man den obigen Satz von "`rechts
nach links"' liest: $\sqrt{a \cdot b}=\sqrt{a}\cdot \sqrt{b}$.

\bb {\tiny .}

\bt{llll}
 a) & $\sqrt{200}=\sqrt{2}\cdot\sqrt{100}=10\sqrt{2}$ & b) & $\sqrt{6400}=\sqrt{64}\cdot\sqrt{100}=....... $ \\
 c) & $\sqrt{81\cdot 10^4}=..... $ &
 & \et\eb

\subsubsection{Achtung!}

Gr\"{o}{\ss}te Vorsicht ist bei folgenden Beispielen geboten:

\bb \label{sddsdgge12} Es ist $ \sqrt{16+9}=\sqrt{25}=5$, \textbf{nicht aber} $
\sqrt{16+9}=\sqrt{16}+\sqrt{9}=4+3=7$. Folglich ist $ \sqrt{16+9} \neq \sqrt{16}+\sqrt{9}$.

Ebenso ist $ \sqrt{100-36}=\sqrt{64}=8$ \textbf{nicht aber}
$\sqrt{100-36}=\sqrt{100}-\sqrt{36}=10-6=4$. Damit gilt auch hier
$ \sqrt{100-36} \neq \sqrt{100}-\sqrt{36}$. \eb

\bme Der obige Satz gilt \textbf{nicht} f\"{u}r Summe und Differenz
d.h. \[ \sqrt{a\pm b} \neq \sqrt{a}\pm \sqrt{b} \quad !!! \] \eme

\bb Vereinfache die folgenden Terme:

\bt{llllll}
 a) & $\sqrt{3}\cdot\sqrt{7}= $ &  b) & $\sqrt{18}\cdot\sqrt{8}= $ &
 c) & $\sqrt{2}\cdot\sqrt{3}\cdot\sqrt{6}= $ \\
 d) & $\sqrt{64}+\sqrt{36}= $ &  e) & $\sqrt{64}\cdot\sqrt{9}= $ &  f) & $\sqrt{\sqrt{36}-\sqrt{4}}=$
 \et  \eb

\subsection{Der Quotient zweier Wurzeln}


\bb Vereinfache:

\bt{llllll}
 a) & $\frac{\sqrt{27}}{\sqrt{3}}=\sqrt{\frac{27}{3}}=\sqrt{9}=\dots $ &
 b) & $\sqrt{8+\frac{1}{4}}=....$ &
 c) & $\frac{\sqrt{24}}{\sqrt{40}}=.... $ \\
 d) & $\sqrt{0,0002}=.... $ & e) & $\sqrt{\frac{7}{3}}\cdot\sqrt{\frac{7}{12}}=.... $ &
 f) & $\frac{\sqrt{3}}{\sqrt{5}}\cdot\frac{\sqrt{10}}{\sqrt{18}}=....$
 \et
\eb



\subsection{Potenzen von Wurzeln}

Auch Potenzen von Wurzeln stellen kein allzu gro{\ss}es Problem
dar:

 \bn\item Quadrate von Wurzeln:

\bb \bt{llll}
 a) & $\sqrt{2}^2=\sqrt{2}\cdot\sqrt{2}= ...... $ & b) & $\sqrt{7,1}^2=....... $ \\
\et \eb

\item H\"{o}here Potenzen:

\bb\bt{llll}
 a) & $\sqrt{13}^3=\sqrt{13}\cdot\sqrt{13}\cdot\sqrt{13}= ....... $ & b) & $\sqrt{7}^4=....... $ \\
\et \eb \en

\ba Berechne:

\bt{llllllll}
 a) & $\sqrt{1,5}^2=$ & b) & $\left( \sqrt{\frac{2}{7}}\right)
 ^4=$ & c) & $(-\sqrt{5})^2=$ & d) & $(-\sqrt{3})^3=$ \et\ea


\subsection{Das teilweise Wurzelziehen}

Die Multiplikationsregel $\sqrt{a \cdot b}=\sqrt{a}\cdot \sqrt{b}$
erlaubt in gewissen F\"{a}llen das sogenannte "`teilweise Ziehen"'
einer Wurzel.

\bb {\tiny .}

\bt{llllll}
 a) & $\sqrt{18}=\sqrt{9\cdot 2}=\sqrt{9}\cdot\sqrt{2}=........ $ &
 b) & $\sqrt{700}=.... $ &  c) & $\sqrt{0,07}=.... $
\et \eb

\bme $ \sqrt{a^2 b}=a\sqrt{b}$ !! \eme

\bb
\[5\sqrt{12}+\sqrt{27}-\sqrt{48}-\sqrt{75}=5\sqrt{4\cdot 3}+\sqrt{9\cdot
3}-\sqrt{16\cdot 3}-\sqrt{25\cdot3}=.......\] \eb

\bme $ -\sqrt{a^2 b}=-a\sqrt{b}$ !! \eme

\subsubsection{Umkehrung:} Genauso kann man, wenn man die beiden obigen
Merks\"{a}tze von rechts nach links liest, einen Faktor vor der Wurzel
in diese "`hineinziehen"'.

\bb Bringe alles unter eine Wurzel:

\bt{llll}
 a) & $2\sqrt{\frac{7}{2}}= .... $ & b) & $-3x\sqrt{\frac{2}{x}}=....$\et
\eb

\ba Versuche teilweise die Wurzel zu ziehen und umzuformen:

\bt{llll}
 a) & $-\sqrt{45}= $ & b) & $4\sqrt{12}(2\sqrt{3}+\sqrt{6})= $ \\
 c) & $\sqrt{\frac{3a^2}{4}}= $ & d) & $\sqrt{c^2+\left(\frac{c}{2}\right)^2}= $ \\
 e) & $\sqrt{x^2 y+x^2 z}= $ &
 f) & $
 \sqrt{8}+\sqrt{12}+\sqrt{18}+\sqrt{48}-\sqrt{72}-\sqrt{108}=$\et\ea


\section{Wiederholung - �bungen}
%\input{smart-reelle.tex}
\ba
\bn 

\item Gibt es f\"{u}r jede reelle Zahl eine Darstellung
$\frac{a}{b}$? Wenn ja, warum; wenn nein, gib ein Gegenbeispiel
an!

\item Warum ist es notwendig, den Zahlenbereich um die reellen Zahlen zu erweitern? Gib ein Beispiel an!

\item Wie bringe ich einen beliebigen Faktor unter die Wurzel? Gib
ein Beispiel an!

\item Richtig oder falsch?

\bt{llllll}
 a) & $-\sqrt{25}=-5$ & b) & $\sqrt{-25}=-5 $ & c) & $\sqrt{36}=6 $ \\
 d) & $-\sqrt{36}=-6 $ & e) & $(-\sqrt{7})^2 =-7 $ & f) & $(-\sqrt{7})^2=7 $ \et



\item Wo ist der Fehler?

\begin{eqnarray*}
1-\sqrt{2}=\sqrt{(1-\sqrt{2})^2}&=&\sqrt{(\sqrt{2}-1)^2}=\sqrt{2}-1\\
1-\sqrt{2}&=&\sqrt{2}-1\\
2& = & 2\sqrt{2}\\
1& = & \sqrt{2}\\
1& = & 2\\
\end{eqnarray*}

\item Berechne:

\bt{llllll}
 a) & $\sqrt{32}\cdot\sqrt{50} $ &  b) & $\sqrt{3a^2 b^3}\cdot\sqrt{24bc^2} $ &
 c) & $\sqrt{75}:\sqrt{27} $ \\
 d) & $\frac{\sqrt{3x^2 y^5}}{\sqrt{x y^3}} $ &  e) & $(3\sqrt{5}-5\sqrt{3})\sqrt{15} $ &
 f) & $\sqrt{72}\cdot\sqrt{3}\cdot\sqrt{6} $ \\
 g) & $\frac{3\sqrt{5}+5\sqrt{3}}{\sqrt{15}} $ &  h) & $\sqrt{\frac{10}{7}}\cdot \sqrt{\frac{21}{15}}\cdot\sqrt{18} $ &
 i) & $\frac{\sqrt{33}}{\sqrt{10}}\cdot\sqrt{\frac{3}{7}}\cdot\sqrt{\frac{14}{4}}\cdot\sqrt{33} $
\et

\item Bringe den wurzelfreien Faktor unter die Wurzel und vereinfache:

\bt{llllll}
 a) & $7\sqrt{\frac{5}{7}} $ &  b) & $8\sqrt{\frac{22}{128}} $ &  c) & $x\sqrt{x^3} $ \\
 d) & $-x^2 \sqrt{\frac{1}{x^3}} $ &  e) & $\frac{1}{x}\sqrt{x^2} $ &  f) & $\frac{z^2}{\sqrt{3z^2}} $
\et

\item Vereinfache durch teilweises Wurzelziehen:

\bt{llll}
 a) & $\sqrt{27} $ & b) & $\sqrt{\frac{169ab^4 c^2}{32}} $ \\
 c) & $\sqrt{\frac{48}{27}} $ & d) & $\sqrt{20}+\sqrt{45}-\sqrt{80}-\sqrt{5} $ \\
 e) & $\sqrt{16x+32y} $ & f) & $3\sqrt{48}-5\sqrt{75}-2\sqrt{12}+7\sqrt{27} $ \\
 g) & $\sqrt{3r^2 -2r^4 s} $ & h) & $12\sqrt{98}+4\sqrt{8}-10\sqrt{50}-14\sqrt{72}+8\sqrt{32}+6\sqrt{8} $
\et 


\item
Bestimme mit dem Heron-Verfahren den Wert von $\sqrt{7}$ auf 6 Dezimalen genau.W"ahle dazu
als Startwert $x=1$.
\item
Untersuche, ob $x$ rational ist! Schreibe zu jeder Antwort eine kurze, aber logisch einwandfreie
Begr"undung! Falls $x$ rational ist, ist $x$ als vollst"andig gek"urzter Bruch darzustellen!
\begin{enumerate}
\item $x = 2,314113111411113111114.......$
\item $x = -0,0545454.......$
\item $x = 0,12636363.......$
\item $x^2 = 21$
\item $x^2 = -4$
\end{enumerate}
\item
 Gib die ersten f"unf Intervalle einer Intervallschachtelung f"ur $3,461057...$ an.

\item
Berechne mit der Halbierungsmethode eine Intervallschachtelung f"ur $\sqrt{11}$!\\
Starte mit dem Intervall $I_0$ (Intervall"ange 1), das nat"urliche Zahlen als
Grenzen hat und brich ab, wenn die Intervall"ange kleiner als 0,1 ist!\\
Wie oft mu"s halbiert werden, wenn als Abbruchbedingung die Intervall"ange 0,00001 steht?

\item
 Welche der folgenden Intervalle bilden {\bf nicht} den Beginn einer Intervallschachtelung,
wenn man sinngem"a"s fortsetzt? Begr"unde kurz.
\begin{enumerate}
\item
$[4;5]$, $[4,8;5]$, $[4,89;5]$, $[4,899;5]$, \dots
\item
$[7;8]$, $[7,7;7,8]$, $[7,77;7,78]$, $[7,777;7,778]$ \dots
\item
$[{2\over3};1]$, $[{2\over5};{1\over 2}]$, $[{2\over7};{1\over 3}]$, $[{2\over9};{1\over 4}]$,
$[{2\over11};{1\over 5}]$ \dots
\end{enumerate}
\item
Welche Gleichung der Form $x^2=a$ hat als L"osung

(a)\qquad $-2$~ ?\hspace{2cm} (b)\qquad $ \frac{1}{\sqrt{5}}$~? 
\item
 Bestimme die Definitionsmenge von:
\begin{enumerate}
\item
$\sqrt{c + 4}$
\item
$\sqrt{-c^2}$
\item
$\sqrt{(- c)^2}$
\item
$\sqrt{c^3}$
\end{enumerate}
\item
Sind die folgenden Aussagen wahr oder falsch? Gib gegebenenfalls an, was zus"atzlich vorausgesetzt
werden mu"s, damit die Aussagen wahr werden.
\begin{enumerate}
\item
Ist $0 < x < y $, so ist $\sqrt{x} < \sqrt{y}$
\item
Es gilt stets $\sqrt{x} < x$.
\item
$\left( \sqrt{-x}\right)^2 = \sqrt{x^2}$
\end{enumerate}
\item
Bestimme die Definitionsmenge des folgenden Terms:
\[\sqrt \frac{-2}{x\cdot(x-4)}\]
\item
 Vereinfache und radiziere soweit wie m"oglich:
\begin{enumerate}
\item
$\sqrt{16 x^2 + 56 x + 49}$
\item
$\sqrt{\sqrt{81c^2}}$
\item
$\sqrt{0,00000175}$
\item
$\left(\sqrt{\frac{  1}{  a^3}} \cdot \sqrt{\frac{  b^6}{  c}} \right) : \sqrt{\frac{  bc}{ 
a^4}}$
\item
$3\sqrt{75} + \sqrt{147} - 4 \sqrt{27} - \sqrt{3}$
\end{enumerate}
\item
Radiziere und vereinfache so weit wie m"oglich:
$$
\sqrt{4a^2} - \sqrt{a^2-4a+4}, \ \ (a<0)
$$
\item
Vereinfache:
$$
\sqrt{275} + \sqrt{343} - \sqrt{112} - \sqrt{99}
$$
\item
Radiziere und vereinfache soweit wie m"oglich:
$$
\sqrt{27a^3+81a^2b \over {(a+3b)}^3},\quad a,b>0
$$
\item
 Vereinfache soweit wie m"oglich:
$$
\sqrt{4a^2\cdot(x-3) \over (x+3)} \cdot \sqrt{(x^2-9)\cdot a^2}; \quad x<-3
$$
\item
Radiziere so weit wie m"oglich und bestimme jeweils zus"atzlich die Bedingungen an die Variablen,
damit der Term definiert ist:
\begin{enumerate}
\item   $\sqrt{(-4)^2x^{14}y^{27}z^{7}}$
\item   $\sqrt{a^4b^3-a^4b^2}$
\end{enumerate}

\item
Radiziere, gegebenenfalls mit Fallunterscheidung:
\[\sqrt{4x^2+64-32x}\]
\item
Gib an, f"ur welche Werte der Variablen die folgenden Wurzelterme definiert sind und radiziere dann
so weit wie m"oglich:
\[\mbox{(a)}~~\sqrt\frac{x^5y^2}{z^4}\hspace{3cm}
\mbox{(b)}~~\sqrt\frac{(a^2+2)\cdot b^3}{c^2-8c+16}\]
\item
 Stelle rationale Nenner her und vereinfache soweit wie m"oglich:
\begin{enumerate}
\item
$\frac{  240}{  \sqrt{180}}$
\item
$\frac{  9 \sqrt{2}}{  \sqrt{98} + \sqrt{72}}$
\end{enumerate}
\item
Mache den Nenner rational und vereinfache soweit wie m"oglich:
$$
3+ 2\sqrt{3} \over 3 -\sqrt{3}
$$
\en
\ea

\section{Wurzelgleichungen}
\bd  Tritt in einer Gleichung die Unbekannte mindestens einmal unter einer Wurzle auf, so nennt sie {\bf Wurzelgleichung}. Dabei beschr�nken wir uns hier auf Quadratwurzeln.\ed  

\bme \bn \item {\bf Gleichung potenzieren!}

Eine Wurzelgleichung l�st man durch Potenzieren (eine Quadratwurzelgleichung durch quadrieren). 

Dabei soll die Gleichung so umgeformt werden, dass die Wurzel isoliert (also alleine) auf einer Seite der Gleichung steht und anschlie�end die beiden Seiten der Gleichung mit dem Wurzelexponenten potenziert. Falls n�tig, wiederholt man dieses Verfahren, bis alle Wurzeln eliminiert sind.

\item {\bf Probe Pflicht!}

Das Potenzieren einer Gleichung ist KEINE �quivalenzumformung, folglich kann sich die gesuchte L�sungsmenge �ndern. Ein solcher Rechenschritt kann n�mlich aus einer falschen Aussage wie $2=-2$ eine wahre Aussage $(2^2=(-2)^2)$ machen. Daher k�nnen beim Potenzieren "`Scheinl�sungen"' hinzukommen, die keine L�sungen der urspr�nglichen Gleichung sind. Die Probe ist daher unverzichtbar!
\en 
\eme 

\ba Bestimme jeweils die Definitions- und die L�sungsmenge:
$$\begin{array}{lll}
a) \quad \sqrt{x}=1 & b) \quad \sqrt{x+7}=-4 & c) \quad \sqrt{x-1}+10=12\\
d) \quad \sqrt{x+30}=6\sqrt{x-5} & e) \quad 2\sqrt{x+1}=\sqrt{x+13} & f) \quad \sqrt{x}=\sqrt{x+8}-2\\
g) \quad \sqrt{x-4}-\sqrt{x+11}+3=0 & h) \quad \sqrt{8-2x}=1+\sqrt{5-x} & i) \quad \sqrt{x-10}+\sqrt{x+10}=10\\
j) \quad \sqrt{2x-\sqrt{7x+4}}=\sqrt{3x+2} & k) \quad \sqrt{x+1}+x=5 & l) \quad \sqrt{x-5}\cdot\sqrt{x+3}=x \\
m) \quad \sqrt{13x+12}=2\sqrt{x-3}+3\sqrt{x} & n) \quad \sqrt{4x-9}+\sqrt{6x-7}=&\sqrt{5x-3}+\sqrt{5x-13}  \\
\end{array}$$

\ea


\ba {\bf mit L�sungen} \begin{eqnarray}
\begin{array}{llc@{\qquad}r}
1)&\quad \sqrt{x}-\sqrt{2x+1}=-1        & &[0;4]\\%[0.4cm]
2)&\quad \sqrt{3x-5}-\sqrt{2x-5}=1      & &[3;7]\\%[0.4cm]
3)&\quad \sqrt{3x+1}+\sqrt{x-4}=\sqrt{4x+5} & &[5]\\%[0.4cm]
4)&\quad \sqrt{2x+2}=2+\sqrt{3x+15}         & &[\{ \}]\\%[0.4cm]
5)&\quad \D{\frac{\sqrt{x^2-16}}{\sqrt{x-3}}}+\sqrt{x+3}=
    \D{\frac{7}{\sqrt{x-3}}}        & &[5]\\%[0.4cm]
%\end{array}
%\eeq
%\beq
%\begin{array}{llc@{\qquad}r}
6)&\quad\sqrt{2x-3}-\sqrt{x+2}=1        & &[14]\\%[0.4cm]
7)&\quad\sqrt{x^2+16}-\sqrt{6x+7}=3-x       & &[-\frac{7}{6};3]\\%[0.4cm]
8)&\quad\sqrt{3x+1}+\sqrt{x-4}=\sqrt{4x+5}  & &[5]\\%[0.4cm]
9)&\quad\sqrt{4x-11}-\sqrt{21-4x}=2\sqrt{x-4}   & &[4;5]\\%[0.4cm]
10)&\quad\sqrt{4x-3}-\sqrt{13-4x}=2\sqrt{x-2}   & &[2;3]\\%[0.4cm]
11)&\quad\sqrt{3x^2+21}+\sqrt{x^2+3}=2\sqrt{x^2+18} & &[\pm 3]\\%[0.4cm]
12)&\quad\sqrt{x-4}=\sqrt{x-9}-\sqrt{x-1}   & &[\{\}]\\%[0.4cm]
13)&\quad\sqrt{3-4x}+\sqrt{1+9x}=\sqrt{4+5x}    & &[\frac{3}{4};-\frac{1}{9}]\\%[0.4cm]
14)&\quad\sqrt{2x+8}-\sqrt{x+5}=7       & &[284]\\%[0.4cm]
\end{array}
 \end{eqnarray}  \begin{eqnarray}
\begin{array}{llc@{\qquad}r}
15)&\quad2\sqrt{x+3a^2}-5a=\sqrt{x-5a^2}    & &[\frac{94a^2}{9},6a^2]\\%[0.4cm]
16)&\quad\sqrt{x-2}=\sqrt{4x+1}-\sqrt{x+3}  & &[6]\\%[0.4cm]
17)&\quad\sqrt{3x-6}+\sqrt{2x+4}=2\sqrt{x}  & &[2]\\%[0.4cm]
18)&\quad\sqrt{x+2}+\sqrt{x-3}=\sqrt{3x+4}  & &[7]\\%[0.4cm]
19)&\quad2\sqrt{3x+1}+3\sqrt{5x-4}=4\sqrt{6x+1} & &[8]\\%[0.4cm]
20)&\quad\sqrt{9x^2+b^2}-3x=\sqrt{b^2-6x}   & &[0,\frac{b^2-1}{6}]\\%[0.4cm]
\end{array}
 \end{eqnarray}
\ea 


 
% \chapter{Die quadratische Funktion}
%\paragraph*{Einstieg: Stationenunterricht}
%In den Pflicht- und Wahl/Pflichtstationen wurden die wichtigen Inhalte zum Thema erarbeitet. Im Kapitel \ref{zufquadfkt} wird das %Wesentliche zusammengefasst.


\bd  Eine {\bf quadratische Funktion} (auch ganzrationale Funktion 2. Grades oder Polynom 2. Grades) ist eine Funktion, die als Funktionsterm ein Polynom vom Grad 2 besitzt, also von der Form

    $$f(x) = ax^2 + bx + c \qquad \mbox{(mit }a \neq 0)$$

ist. Der Graph ist eine {\it Parabel}. F�r $a = 0$ ergibt sich eine lineare Funktion.\ed 

Zun�chst die Frage: welche Rolle spielen die Koeffizienten $a, b$ und $c$. Dazu zwei �bungen im Computerraum:

\ba \bn \item Computerraum: Die quadratische Funktionen

Die Arbeitsanweisungen m�ssen \textbf{genauestens gelesen} werden.
Alle Beobachtungen sind in \textbf{ganzen S�tzen} festzuhalten!

Verwende eine Tabellenkalkulation und ein Textverarbeitungsprogramm.
�ffne die Datei \texttt{DynABQuadratischeFunktion.xls} Sie
enth�lt zwei Arbeitsbl�tter mit den Namen "`Scheitelform"' und
"`allgemeine Form"'. Mit der Maus k�nnen die Konstanten $a,d,e$
der Scheitelform $y=a(x+d)^2+e$ bzw. $a,b,c$ der allgemeinen Form
$y=ax^2+bx+c$ ge�ndert werden; die Funktion selbst ver�ndert sich
automatisch mit.

Textverarbeitung-�berschrift:
"`Arbeitsblatt - Quadratische Funktionen"'; Kopfzeile: dein Name.





\begin{enumerate}

\item Wechsle in der Excel- Datei zum Arbeitsblatt "`Scheitelform"'. 
Erstelle  eine kleinere �berschrift
\textbf{"`Scheitelform: $y=a(x+d)^2+e$"'} und bearbeite die
folgenden Fragen:

\begin{enumerate}

\item Stelle zun�chst f�r die Konstanten (falls nicht schon
eingestellt) folgende Werte ein: $a=1,d=0,e=0$. �ndere nun mit der
Maus den Wert f�r $a$. Beobachte, was sich genau ver�ndert und
halte deine Beobachtungen fest!

\item Setze $a$ wieder auf den Wert $1$. Ver�ndere nun den Wert
$d$ und halte wie oben deine Beobachtungen fest!

\item Setze $d$ auf den Wert $0$. Ver�ndere nun den Wert $e$ und
halte deine Beobachtungen fest!

\item Versuche (evtl. mit Hilfe verschiedener selbst gew�hlter
Werte) eine allgemeine Formel f�r den Scheitelpunkt $S$ ($=$
h�chster oder tiefster Punkt der Kurve) in Abh�ngigkeit der
Parameter $a,d,e$ zu finden!

\end{enumerate}

\item Wechsle in der Excel- Datei zum Arbeitsblatt "`Allgemeine
Form"'. Erstelle 
 eine kleinere �berschrift
\textbf{"`Allgemeine Form: $y=ax^2+bx+c$"'} und bearbeite die
folgenden Fragen:

\begin{enumerate}

\item Stelle zun�chst f�r die Konstanten (falls nicht schon
eingestellt) folgende Werte ein: $a=1,b=0,c=0$. �ndere nun mit der
Maus den Wert f�r $a$. Beobachte, was sich genau ver�ndert und
halte deine Beobachtungen fest!

\item Setze $a$ wieder auf den Wert $1$. Ver�ndere nun den Wert
$b$ und halte wie oben deine Beobachtungen fest! Betrachte
insbesondere die Lage der Nullstellen.

\item Setze $b$ auf den Wert $0$. Ver�ndere nun den Wert $c$ und
halte deine Beobachtungen fest!

\item Versuche (evtl. mit Hilfe verschiedener selbst gew�hlter
Werte) eine allgemeine Formel f�r den $x-$Wert des Scheitelpunkts
$S$ ($x_{S}$) in Abh�ngigkeit der Parameter $a,b,c$ zu finden!

\end{enumerate}

\item �berpr�fe, ob �berall \textbf{ausf�hrlich in ganzen S�tzen}
geantwortet wurde. Versuche schlie�lich die Antworten auf obige
Fragen auf eine Seite (maximal zwei) zu bringen und in einer
ansehnlichen Form darzustellen. Drucke die Seite(n) aus!

\item \textbf{F�r all jene, die fr�her fertig werden:} Versuche
mit Hilfe der Excel- Datei auf den beiden Arbeitsbl�ttern
Funktionen mit denselben Graphen ausfindig zu machen. Vergleiche
die jeweiligen Funktionsvorschriften und �berlege dir, wie man
eine Funktionsgleichung in Scheitelform in die allgemeine Form
umrechnen k�nnte und umgekehrt!

\end{enumerate}
\item Computerraum: Zeichne mit Geonext drei Schieberegler f�r die Parameter $a$, $b$ und $c$. (evtl. Strecken von $-5$ bis $5$). Zeichne den Funktionsgraphen in Abh�ngigkeit dieser Gleiter und beobachte ihren Einfluss auf den Graphen.

\en 
\ea 





\section{Eigenschaften der quadratischen Funktion - Scheitelpunktform}\label{zufquadfkt}
\bme 
\begin{description}
\item[Die quadratische Funktion $y=x^2$]:

	\begin{minipage}{11cm}
	\bi
	\item Definitionsmenge $D=\mathbb{R}$
	\item Wertemenge $W=\mathbb{R}^+$
	\item der Graf ist eine {\bf Parabel}
	\item der Graf ist achsensymmetrisch bez�glich der $y$-Achse
	\item ber�hrt die $x$-Achse im Ursprung - Scheitel der Parabel: $S(0|0)$
	\ei 
	\end{minipage} 
	\begin{minipage}{3cm}
	\includegraphics[width=3cm]{2te/quadratischefunktion/bilder/reinquadfkt2.jpg}
	\end{minipage}


\item[Die quadratische Funktion $y=ax^2$]:

	\begin{minipage}{11cm}
	\bi
		\item der Graf ist eine {\bf Parabel}, sie ist 
	\bi \item nach oben ge�ffnet, falls $a>0$ ist
		\item nach unten ge�ffnet, falls $a<0$ ist
		\item keine quadratische Funktion ($y=0$), falls $a=0$ ist.
	\ei

	\item Sie entsteht, indem man die Grundparabel in Richtung der $y$-Achse
		\bi \item {\it streckt}, falls $|a|>1$ ist (sie ist dann steiler als die Grundparabel)
		\item  {\it staucht}, falls $|a|<1$ ist (sie ist dann flacher als die Grundparabel)
		\item und sie, falls $a<0$, anschlie�end noch an der $x-Achse$ spiegelt.
		\ei 
		Welche Grafen sind rechts gezeichnet?
	
		$..............$ $..............$. $.................$
	\item der Graf ist achsensymmetrisch bez�glich der $y$-Achse
	\item ber�hrt die $x$-Achse im Ursprung - Scheitel der Parabel: $S(0|0)$
	\ei 
	\end{minipage} 
	\begin{minipage}{3cm}
	\includegraphics[width=3cm]{2te/quadratischefunktion/bilder/reinquadfkt3.jpg}
	\end{minipage}


\item[Die quadratische Funktion $y=a(x-d)^2$]:

	\begin{minipage}{11cm}
	Der Graf dieser quadratischen Funktion ist eine Parabel mit dem Scheitel $S(d|0)$ auf der $x$-Achse. Sie entsteht durch eine Horizontalverschiebung (entlang der $x$-Achse) der Parabel $y=ax^2$ um $d$ Einheiten (nach rechts, falls $d$ negativ ist, nach links, falls $d$ positiv ist).

	Welche Funktionsgrafen sind rechts gezeichnet?

	$................$ $...................$
	\end{minipage} 
	\begin{minipage}{3cm}
	\includegraphics[width=3cm]{2te/quadratischefunktion/bilder/allgquadfthorizontaleverschiebung.jpg}
	\end{minipage}


\item[Die quadratische Funktion $y=ax^2+e$]:

	\begin{minipage}{11cm}
	Der Graf dieser quadratischen Funktion ist eine Parabel mit dem Scheitel $S(0|e)$ auf der $y$-Achse. Sie entsteht durch eine Vertikalverschiebung (entlang der $y$-Achse) der Parabel $y=ax^2$ um $e$ Einheiten (nach oben, falls $e$ positiv ist, nach unten, falls $e$ negativ ist).

	Welche Funktionsgrafen sind rechts gezeichnet?

	$................$ $...................$
	\end{minipage} 
	\begin{minipage}{3cm}
	\includegraphics[width=3cm]{2te/quadratischefunktion/bilder/allgquadftvertikaleverschiebung.jpg}
	\end{minipage}

\item[Die quadratische Funktion $y=a(x-d)^2+e$ (Scheitelpunktform)]:

	\begin{minipage}{11cm}
	Der Graf dieser quadratischen Funktion ist eine Parabel mit dem Scheitel $S(d|e)$. Sie entsteht durch eine Horizontal- und eine Vertikalverschiebung  der Parabel $y=ax^2$.

	Welche Funktionsgrafen sind rechts gezeichnet?

	$................$ $...................$
	\end{minipage} 
	\begin{minipage}{3cm}
	\includegraphics[width=3cm]{2te/quadratischefunktion/bilder/allgquadfthorizontaleundvertikaleverschiebung.jpg}
	\end{minipage}


\item [Die allgemeine quadratische Funktion $y=ax^2+bx+c$]:
	
	\begin{minipage}{11cm}
	$ax^2$ nennt man quadratisches Glied, $bx$ lineares Glied und $c$ hei�t absolutes Glied.

	Bringt man die quadratische Funktion $y=ax^2+bx+c$ durch quadratische Erg�nzung auf die Scheitelpunktform $y=a(x-d)^2+e$, so ergeben sich die Koordinaten des {\bf Scheitels} $$S\left(-\frac{b}{2a}|\frac{4ac-b^2}{4a}\right)$$

	Der Graf entsteht also aus dem Graphen der Parabel $y=ax^2$ durch eine Horizontal- (um $d$-Einheiten) und eine Vertikalverschiebung (um $e$-Einheiten).

	
	

	Welche Funktionsgrafen sind rechts gezeichnet?

	$................$ $...................$
	\end{minipage} 
	\begin{minipage}{3cm}
	\includegraphics[width=3cm]{2te/quadratischefunktion/bilder/allgquadfkt9.jpg}
	\end{minipage} 



\end{description}
\eme 

\section{Die Nullstellen der quadratischen Funktion - die quadratische Gleichung}

Sucht man die Nullstellen einer Funktion, so setzt man die Funktionsgleichung gleich Null und l�st die entsprechende Gleichung. Bei einer quadratischen Funktion ergibt sich dabei die sog. quadratische Gleichung.

\bd Eine {\bf quadratische Gleichung} hat die Form $$ax^2 +bx+c=0$$ oder kann auf diese Form gebracht werden.\ed 

\bs  Die quadratische Gleichung kann mit Hilfe der Formel $$x_{1/2}=\frac{-b\pm \sqrt{b^2-4ac}}{2a}$$ gel�st werden. Dabei nennt man den Ausdruck $b^2-4ac$ auch Diskriminante (lat. {\it discriminare} = unterscheiden)  und bezeichnet ihn mit $D$.\es 

\begin{proof}
Die Formel kannmit Hilfe der quadratischen Erg�nzung bewiesen werden: ... 
\end{proof}

\bme  
	\begin{itemize}
		\item Die reinquadratische Gleichung $x^2=d$:

		\begin{minipage}{10cm}Die {\bf reinquadratische Gleichung} hat die Form $$x^2=d$$ oder l�sst sich auf diese Form bringen. Es gilt:
		\bi \item F�r $d>0$ gibt es zwei L�sungen $x_1=+\sqrt{d}$ und $x_2=-\sqrt{d}$
		\item F�r $d=0$ gibt es eine L�sung $x=0$
		\item F�r $d<0$ gibt es keine L�sung $\mathbb{L}=\{ \}$
		\ei
		Rechts Geometrische Interpretation: die Funktion $y=x^2+d$ hat keine, eine oder zwei Nullstellen. Welche Funktionsgrafen sind gezeichnet?\\
		..............      ..............      ..............
		\end{minipage} 
		\begin{minipage}{3cm}
		\includegraphics[width=3cm]{2te/quadratischefunktion/bilder/reinquadfkt.jpg}
		\end{minipage}

		\item Die allgemeine quadratische Gleichung $ax^2+bx+c=0$

		\begin{minipage}{10cm}Bei der {\bf allgemeinen quadratischen Gleichung} $$ax^2+bx+c=0$$ h�ngt die L�sung von der {\bf Diskriminante} $$D=b^2-4ac$$ ab:
		\bi \item F�r $D>0$ gibt es zwei L�sungen\\ $x_1=\frac{-b+\sqrt{D}}{2a}$ und $x_2=\frac{-b-\sqrt{D}}{2a}$
		\item F�r $D=0$ gibt es eine L�sung $x=-\frac{b}{2a}$
		\item F�r $D<0$ gibt es keine L�sung $\mathbb{L}=\{ \}$
		\ei
		Rechts: Geometrische Interpretation: die Funktion $y=ax^2+bx+c$ hat keine, eine oder zwei Nullstellen. Welche Funktionsgrafen sind gezeichnet?\\
		..............      ..............      ..............
		\end{minipage} 
		\begin{minipage}{3cm}
		\includegraphics[width=3cm]{2te/quadratischefunktion/bilder/allgquadfkt.jpg}
		\end{minipage}
		
		\item Die Zerlegung von quadratischen Polynomen

		Jedes quadratische Polynom $ax^2+bx+c$ mit positiver Diskriminante (die entsprechende Funktion hat also $2$ Nullstellen $x_1$ und $x_2$) kann in ein Produkt von Linearfaktoren zerlegt werden: $$ax^2+bx+c=a(x-x_1)(x-x_2)$$
	\end{itemize}
\eme 

\section{Zusammenfassung: quadratische Gleichungen}

\bs
Eine Gleichung der Form: $ax^2+bx+c=0$ hei{\ss}t {\bf Normalform} der quadratischen
Gleichung. (Der Grad der Gleichung ist gleich dem
Exponenten der Variablen) \\
Die L\"{o}sung der quadratischen Gleichung berechnet sich aus: 
\begin{eqnarray}
x_{1,2}=\frac{-b\pm\sqrt{b^2-4ac}}{2a} \end{eqnarray}
 der Wert $D=b^2-4ac$ unter der Wurzel
hei{\ss}t {\bf Diskriminante}  und gibt an, ob die Gleichung zwei L\"{o}sungen $(D>0)$, genau
eine L\"{o}sung $(D=0)$, oder
keine L\"{o}sung $(D<0)$ besitzt.\\
Besitzt eine quadratische Gleichung zwei L\"{o}sungen $x_1,x_2$ kann sie auf folgende Art
in Faktoren zerlegt werden: \begin{eqnarray}
ax^2+bx+c=a(x-x_1)(x-x_2), \quad \mbox{mit} \quad
 c= ax_1x_2  \end{eqnarray}
Besitzt sie genau eine L\"{o}sung, n\"{a}mlich $-\frac{b}{2a}$ ist die Zerlegung:
$ax^2+bx+c=a(x+\frac{b}{2a})^2$, $c$ entspricht dann dem Wert, $\frac{b^2}{4a}$, wie
durch einfaches Ausmultiplizieren leicht ersichtlich
wird.
\es


\begin{center}
\begin{figure}[hbt]
\includegraphics[width=5cm]{2te/quadratischefunktion/bilder/quadratischehexe.png}
\caption{Quadratische Gleichungen machen Lehrer noch nicht zu Monstern ...}
\end{figure}
\end{center}

\ba
\bn \item $  (9+x)(7-x)+(9-x)(7+x)=76$ \hfill (L�sung: $ [\pm 5] $)
\item $  (x+9)^2=2(x+7)^2-17$ \hfill (L�sung: $ [0,-10] $)
\item $ \D{\frac{a+1}{a-1}}x=\D{\frac{x^2+x}{x-1}}  $ \hfill (L�sung: $ [0,a] $)
\item $ (x+a)^2+2ax=a^2  $ \hfill (L�sung: $ [0,-4a] $)
\item $ (a+x)(b-x)+(a-x)(b+x)=0  $ \hfill (L�sung: $  [\pm \sqrt{ab}]$)
\item $ \D{\frac{x+a}{x-a}}+\D{\frac{x-a}{x+a}}
    =\D{\frac{2(a^2+1)}{(1+a)(1-a)}} $ \hfill (L�sung: $ [\pm 1] $)
\item $ \D{\frac{4x-3}{x-2}}+\D{\frac{1-3x}{x-1}}=0 $ \hfill (L�sung: $  [\{\}] $)
\item $ \D{\frac{x^2-6}{(1-x)(x-3)}}+2=\D{\frac{x-2}{x-1}}
    +\D{\frac{x-2}{x-3}} $ \hfill (L�sung: $ [\pm 2]$)
\item $ \D{\frac{5}{6}}-\D{\frac{1}{x+2}}=\D{\frac{2}{x^2-4}}
    -\D{\frac{1}{x-2}} $ \hfill (L�sung: $ [\pm 2\sqrt{\D{\frac{2}{5}}}]$)
\item $ (x-3)(x-5)=(x-2)(x-1)+(x-3)^2-16 $ \hfill (L�sung: $ [5,-4] $)
\item $ (2x+3)^2=(x-1)(x-2)+25 $ \hfill (L�sung: $ [-6;1] $)
\item $ \D{\frac{(3x+1)(2x-3)}{21}}+\D{\frac{x^2+3}{7}}
    =\D{\frac{x^2+x-2}{3}} $ \hfill (L�sung: $ [2;5]$)
\item $ (x+1)(x+2)-(x+1)(x+3)+(x+2)(x+3)=2 $ \hfill (L�sung: $ [-3;-1] $)
\item $ \D{\frac{7(x-5)}{8}}+x-2=\left(x-\D{\frac{9}{2}}\right)\left(
    x-\D{\frac{11}{4}}\right) $ \hfill (L�sung: $ [6;\D{\frac{25}{8}}] $)
\item $  (1-x^2)+2-3x=\D{\frac{x-5}{3}}-x^2-\D{\frac{x^2-5x}{3}}$ \hfill (L�sung: $[1;2]  $)
\item $ \D{\frac{x^2-1}{\sqrt{2}-1}}-\D{\frac{x^2+1}{\sqrt{2}+1}}=
    \D{\frac{\sqrt{2}(2-x+\sqrt{2})}{\sqrt{2}+1}} $ \hfill (L�sung: $ \left[\sqrt{2};-\D{\frac{2+\sqrt{2}}{2}}\right]  $)
\item $ \D{\frac{7x^2+5}{12}}-\D{\frac{(x-1)^2}{2}}=\D{\frac{2x^2-4x}{3}}-
    \D{\frac{(x-2)^2}{3}} $ \hfill (L�sung: $ [-1;5] $)
\item $ \D{\frac{7(x-1)(x-7)}{6}}- \D{\frac{2x^2-19x+17}{3}}=
    \D{\frac{(1-x)(15x-53)}{14}}   $ \hfill (L�sung: $ [1;4] $)
\item $ \D{\frac{(x+1)(5x-3)}{4}}+\D{\frac{3x-2x^2}{3}}=\D{\frac{8x}{3}}+1 $ \hfill (L�sung: $  [-1;3]$)
\item $ x^2+\D{\frac{x^2-1}{3}}=\D{\frac{x-2}{2}} $ \hfill (L�sung: $[\{\}]  $)
\item $ \D{\frac{x+1}{x-1}}+\D{\frac{x+3}{x-3}}=-2 $ \hfill (L�sung: $ [0;2] $)
\item $ (3x+1)^2-4x(2x-3)=(x+7)^2 $ \hfill (L�sung: $  $)
\item $ (2x+3)^2=(x-1)(x-2)+25 $ \hfill (L�sung: $ [-6;1] $)
\item $(1-x)^2+2-3x=\D{\frac{x-5}{3}}-x^2-
    \D{\frac{x^2-5x}{3}}    $ \hfill (L�sung: $ [1,2] $)
\item $ \D{\frac{(3x+1)(2x-3)}{21}}+\D{\frac{x^2+3}{7}}
    =\D{\frac{x^2+x-2}{3}}  $ \hfill (L�sung: $  [2;5]$)
\item $  \D{\frac{4-9x}{4x+6}}+\D{\frac{6x+16}{9-4x^2}}-1=
    \D{\frac{2-10x}{6x-9}}   $ \hfill (L�sung: $ [2;\D{\frac{45}{3}}] $)
\item $  \D{\frac{a+1}{a-1}x=\frac{x^2+x}{x-1}}$ \hfill (L�sung: $[0,a]  $)
\item $ 2x^2-ax=bx  $ \hfill (L�sung: $ [\D{\frac{a+b}{2}}] $)
\item $ (x-8)^2-9x^2+12x=2(x-1)^2+4  $ \hfill (L�sung: $ [\D{\pm\sqrt{\frac{29}{5}}}] $)
\item $  \D{\frac{x+a}{x-a}-\frac{2a-x}{x+a}=1+\frac{5}{4}}$ \hfill (L�sung: $ [-7a,3a] $)
\item $  $ \hfill (L�sung: $  $)
\en
\ea


\ba \bn 


\item
 
L"ose mit quadratischer Erg"anzung und mache die Probe: 
$$3x^2 - 10x + 3 = 0$$




\item
 
L"ose mit der L"osungsformel:
\begin{enumerate}
\item
$5x^2 + 6x - 8 = 0$
\item
$- \frac{1}{6}y^2 = - \frac{2}{3}y + \frac{1}{2}$
\end{enumerate}



\item
 
Bestimme die L"osungsmenge! Die Ergebnisse sind mit rationalem Nenner
anzugeben!
$$10z=2\sqrt{5}+\sqrt{5}z^2$$


\item
 

Bestimme die L"osungsmenge:
\[(1-2x)\cdot \left(4-\frac{8}{9}x\right)=\left(2-\frac{5}{3}x\right)^2\]

\item
 

Gib die Definitionsmenge an und bestimme die L"osungsmenge:
$$
\frac{x+21}{x-3} + \frac{16x}{6-2x} = 3x + 2; \ \ G= Q
$$
  

\item

Bestimme die L"osungen der Gleichung
\[\frac{3x+1}{4x-6}+\frac{7x+2}{6x+9}=\frac{8x^2-3x+2}{4x^2-9}\]

\item

L"ose folgende Gleichung mit Hilfe quadratischer Erg"anzung, wobei der
Parameter $a$ von Null verschieden sei:
\[a\cdot x^2+2x-\frac{1}{a}=0\]



\item
 

L"ose folgende Gleichung "uber $G=R$:
\[3b\cdot (x-2a)-2a\cdot (3b-x)=x^2-6ab~;~~a,b \quad{\rm sind~ Formvariablen}\]


\item
 
F"ur welche $a \in R$ besitzt die Gleichung $ ax^2 - x - 0,25 = 0$ 
genau eine L"osung? Gib diese L"osungen an!\\


\item
 
F"ur welche Parameterwerte $t \in R$ besitzt die Gleichung $x(3x + 4t) = 15 + 10tx + 18t$
genau eine L"osung? Wie lautet jeweils diese L"osung?\\


\item
 
Bestimme Definitions- und L"osungsmenge folgender Gleichung "uber der 
Grundmenge $R$:
$$\frac{2}{2x-3} + \frac{1}{1+x}=\frac{3}{2x^2-x-3}$$


\item
 

Bestimme alle L"osungen der Gleichung 
\[x^5+3x^3-40x=0\]


\item
 
L"ose folgende Gleichungen:
\begin{enumerate}
\item
$\frac{1}{3} x^4 - \frac{5}{3} x^2 = 12$
\item
$(x+1)^4 - 3 (x+1)^2 - 4 = 0$
\end{enumerate}


\item
 

L"ose folgende Gleichung "uber $G=R$:
\[  3\cdot \left(x^2-\frac{1}{3}\right)^2\,+\,6 \cdot \left(x^2-\frac{1}{3}
\right)\,-\,\frac{7}{3}\,=\,0\]


\item
 
Bestimme Definitions- und L"osungsmenge! (Vergi"s die Probe nicht!)
$$\sqrt{2x+8} - \sqrt{5+x} = 1$$

\item 
 

Gegeben ist die Wurzelgleichung
\[  -\sqrt{2-x}+\frac{6}{\sqrt{4-x}}-\sqrt{4-x}=0~.\]
\begin{enumerate}
\item	L"ose die Gleichung "uber $G=R$!
\item	F"uhre eine ausf"uhrliche Probe durch!
\end{enumerate}




\item
 
L"ose durch Ausklammern: $$2x^3 - 32x = 0$$


\item
 
Bestimme Definitions- und L"osungsmenge folgender Gleichung\\
(Grundmenge ist $R$):
$$\frac{  x^2}{  x^2 - x - 6} + \frac{  1}{  x - 3} = 
\frac{  x}{  2(x + 2)}$$


\item
 
Welche rationale Zahl hat folgende Eigenschaft:\\
Das Produkt der um 1 kleineren Zahl und der um 1 gr"o"seren Zahl
ist um 31 gr"o"ser als das halbe Quadrat der gesuchten Zahl!\\
Fertige einen x-Ansatz an!

\item
 
In einem Quader mit der Oberfl"ache $286\,{\rm cm}^2$ ist die mittlere Kante
7\,cm lang. Sie unterscheidet sich von der gr"o"sten Kante ebensoviel 
wie von der kleinsten. Wie lang sind die Kanten dieses Quaders?

\item
 
\begin{minipage}[t]{9cm}
In ein wei"ses Kreuz der Seitenl"ange $s=\sqrt{2} {\rm\,m}$ 
ist ein schwarzes Kreuz symmetrisch eingezeichnet (vgl. Abbildung). 
Wie breit ist das Kreuz, wenn der Fl"acheninhalt des Kreuzes 
genauso gro"s ist wie der des Hintergrundes?
\end{minipage}
\begin{minipage}{5.5cm}
\includegraphics[width=5.5cm]{2te/quadratischefunktion/bilder/kreuz.jpg}
\end{minipage}



\item Die Zinsen eines Kapitals von 800 Euro werden am Ende jeden Jahres zum
Kapital geschlagen und dieses zus"atzlich noch um 100 Euro vermehrt,
so da"s es am Anfang des dritten Jahres auf 1069,28 Euro angewachsen war.
Wie hoch war der ("uber dem gesamten Zeitraum als konstant angenommene) 
Zinssatz?

\item Ein Sch"uler hatte f"ur einen Ferienaufenthalt $252\,$ Euro gespart. Nachdem 
sich die Tageskosten um $7\,$ Euro erh"oht hatten, mu"ste er seinen Aufenthalt 
um drei Tage verk"urzen. Wie viele Tage wollte er urspr"unglich bleiben?


\en 

\ea 

\section{Ungleichungen 2.Grades, Ungleichungssysteme}
Ungleichungen k\"{o}nnen auf dieselbe Art umgeformt werden wie Gleichungen. Ausnahmen
bilden jedoch die Multiplikation (Division) mit einer negativen Zahl und das
Vertauschen. In diesen F\"{a}llen dreht sich das
Ungleichheitszeichen um.\\
In einer \underline{\bf Bruchungleichung} muss daher eine Fallunterscheidung gemacht
werden:\begin{eqnarray}
  \frac{3x+2}{x-4}>2 \quad D=\mathbb{R}\{4\}
\end{eqnarray} \underline{1. Fall:} Unter der Bedingung, dass $x-4>0$ ist, wird die gegebene
Bruchungleichung mit einem positiven Term multipliziert. Das Relationszeichen bleibt
damit erhalten. \begin{eqnarray}
  x>4 \:&\wedge& \: 3x+2>2x-8\\[0.4cm]
  \Leftrightarrow x>4 \:&\wedge& \: x>-10\\[0.4cm]
  L_1&=&\{x|x>4\}
\end{eqnarray} \underline{2. Fall:} Die gegebene Bruchungleichung wird unter der Bedingung, dass
$x-4<0$ ist, mit einem negativen Term multipliziert. Das Relationszeichen wird
umgedreht. \begin{eqnarray}
  x<4 \:&\wedge& \: 3x+2<2x-8\\
  \Leftrightarrow x<4 \:&\wedge& \: x<-10\\
  L_2&=&\{x|x<-10\}
\end{eqnarray} Nun sind beide F\"{a}lle zusammenzufassen. \begin{eqnarray} L=L_1\cup L_2 \Rightarrow L=\{x|x>4  \vee x<-10\}
\end{eqnarray}

\noindent F\"{u}r die L\"{o}sung von \underline{\bf Ungleichungen 2. Grades}
\begin{eqnarray}
ax^2+bx+c>0 \label{groesser}\\
ax^2+bx+c<0 \label{kleiner}
\end{eqnarray}
betrachten wir deren Diskriminante $D$. \\
\underline{1. Fall:} Ist $D>0$ besitzt die Gleichung $ax^2+bx+c=0$ zwei L\"{o}sungen
$x_1,x_2$ wobei wir annehmen, dass $x_1<x_2$. Die Gleichung l\"{a}{\ss}t sich deshalb in
Faktoren zerlegen: $ax^2+bx+c=a(x-x_1)(x-x_2)$. Die Ungleichung \ref{groesser} ist
erf\"{u}llt f\"{u}r alle Werte von $x$ f\"{u}r die gilt $x\not\in (x_1,x_2)$, w\"{a}hrend die
Ungleichung \ref{kleiner}
erf\"{u}llt ist f\"{u}r alle Werte von $x$ f\"{u}r die gilt $x\in (x_1,x_2)$.\\
\underline{2. Fall:} Ist $D=0$ besitzt die Gleichung $ax^2+bx+c=0$ die L\"{o}sung
$x_1=x_2=-\frac{b}{2a}$. Die Gleichung l\"{a}{\ss}t sich in Faktoren zerlegen:
$ax^2+bx+c=a(x-x_1)^2=a(x+\frac{b}{2a})^2$. Die Ungleichung \ref{groesser} ist erf\"{u}llt
f\"{u}r alle Werte von $x$ f\"{u}r die gilt $x\not=-\frac{b}{2a}$, w\"{a}hrend die Ungleichung
\ref{kleiner}
keine L\"{o}sung besitzt.\\
\underline{3. Fall:} Ist $D<0$ besitzt die Gleichung $ax^2+bx+c=0$ keine reelle
L\"{o}sung, das Polynom kann jedoch umgeformt werden:
     \begin{eqnarray}   ax^2+bx+c=a(x^2+\frac{b}{a}x+\frac{c}{a})=\\
       a(x^2+\frac{b}{a}x+\frac{c}{a}+\frac{b^2}{4a^2}-\frac{b^2}{4a^2})=\\
        a[(x^2+\frac{b}{2a})^2-\frac{b^2-4ac}{4a^2}]
     \end{eqnarray}
 Da nun $b^2-4ac$ negativ ist, wird das Polynom f\"{u}r jeden Wert von $x$ positiv.
Die Ungleichung \ref{groesser} ist also f\"{u}r alle Werte von $x$ erf\"{u}llt,
w\"{a}hrend die Ungleichung \ref{kleiner} keine L\"{o}sung besitzt.\\
\underline{\"{U}bungen:}\\
a) lineare Bruchungleichungen: \begin{eqnarray} &1)& \quad \frac{x}{x+1}>0 \qquad \qquad
\quad2)\quad
    \frac{2x}{x+1}\geq 1 \\
&3)& \quad \frac{x-7}{x}<\frac{1-3x}{x} \qquad 4) \quad
    \frac{8-2x}{x+1}\geq1\\
&5)& \quad \frac{x-1}{x}-1<0 \qquad \quad 6)\quad
    \frac{13}{x+4}\leq\frac{15}{2x-3}\\
&7)& \quad \frac{6}{2x-1}>\frac{5}{x-2}\\
&8)& \quad \frac{2}{x-1}\leq\frac{1}{x^2-x}+\frac{1}{x}\\
\end{eqnarray}b) quadratische Ungleichungen: \begin{eqnarray}
\begin{array}{llc@{\qquad}r}
1)& \quad x^2-3x-4 <0   &&[-1<x<4]\\
2)& \quad 2x^2+5x-3 >0  &&[x<-3,x>\frac{1}{2}]\\
3)& \quad x^2-10+25>0   &&[x\in \mathbb{R}\backslash\{5\}\\
4)& \quad 3x^2-5x+9>0   &&[x\in \mathbb{R}]\\
5)& \quad 25-x^2\leq0   &&[x\leq-9; x\geq9]\\
6)& \quad x(3-x)>0  &&[0<x<3]\\
7)& \quad (2x+5)(x-1)<0 &&[-\frac{5}{2}<x<1]\\
8)& \quad (x-1)(x+2)>0  &&[x<-2;x>1]\\
9)& \quad x(x-3)+\D{\frac{x}{2}}>(2x-5)^2+\D{\frac{7x^2}{3}}    &&\\[0.4cm]
10)& \quad 5x^2-11<(2x-1)^2-\D{\frac{x^2-1}{4}}+11      &&\\[0.4cm]
11)& \quad (3x-2)^2+3<5x-(2x-1)^2   &&[\frac{8}{13}<x<1]\\[0.4cm]
12)& \quad \D{\frac{x^2+4x}{6}}>\D{\frac{9x+36}{2}}-35 &&[x<6;x>17]\\[0.4cm]
13)& \quad 2(x-1)-(x+3)^2+4x-6<0    &&[x\in \mathbb{R}]\\[0.4cm]
14)& \quad \D{\frac{x-1}{5}+\frac{1}{3}}<\D{\frac{x^2-5x+6}{6}}
            &&[x<1,x>\frac{26}{5}]\\[0.4cm]
15)& \quad \D{\frac{(x+3)^2}{5}-\frac{7(x-6)}{2}}\leq\D{\frac{(2x-20)
    (x-12)}{3}}-\frac{x}{2}     &&[x\leq\frac{39}{7},x\geq22]\\[0.4cm]
15)& \quad \D{\frac{x+3}{x}}>\D{\frac{x}{x+3}}
\end{array}
\end{eqnarray}

\underline{Ungleichungssysteme:}\\
Bei Ungleichungsystemen l\"{o}st man die gegebenen Ungleichungen und bildet dann die
Schnittmenge der einzelnen L\"{o}sungen.
%\input{funktion_quadrat}


\section{Der Scheitel der Parabel}

Jede Parabel hat einen gr��ten bzw. kleinen Wert, diesen nennt man {\bf Scheitel}. 

\bs  Eine quadratische Funktion $y=ax^2+bx+c$ hat immer einen Scheitelpunkt $S(x_s, y_s)$ mit $$x_s=-\frac{b}{2a}$$ Die Parabel ist immer symmetrisch bez�glich der Geraden durch den Scheitel, also bez�glich $x=x_s$.\es 

\begin{proof}
Die Formel kann auf zwei Arten bewiesen werden: \bn \item Mit Hilfe der quadratischen Erg�nzung. \item Das arithmetische Mittel der Nullstellen ...  \en 
\end{proof}






\ba \bn \item 
Gegeben ist die Funktion $f: y=x^2+x-3,75$.
\begin{enumerate}
\item
Bestimme die maximale Definitions- und Wertemenge der Funktion.
\item
Gib die Koordinaten des Scheitelpunktes $S(s_1|s_2)$ an.
\item
Zeichne den Graphen der Funktion im Intervall $[s_1 -3;s_1 +3]$.\\
(1 L"angeneinheit = $1 \, \rm cm$)
\item
Bestimme \underline{rechnerisch} die Nullstellen der Funktion $f$.
\end{enumerate}

\item
Gegeben ist die Funktion $f(x) = -1,5x^2 + 9x - 12$.\\
Bestimme die Koordinaten des Scheitels sowie die Bereiche auf der 
$x$-Achse, in denen die Funktion steigt bzw. f"allt.

\item
 
Gegeben ist die Funktion $p: y= -0,5x^2 + x + 1,5$.
\begin{enumerate}
\item
Zeige, da"s der Punkt $S(1|2)$ Scheitel der zu p geh"orenden Parabel ist.
\item
Bestimme die Symmetrieachse, Wertemenge und die Schnittpunkte des Graphen
mit den Koordinatenachsen.
\item
Zeichne den Graphen der Funktion im Intervall $[-3;5]$ ohne Verwendung
einer Wertetabelle. (1 L"angeneinheit = $1 \, \rm cm$)
\end{enumerate}

\item
 
Bestimme die Scheitelpunktsform folgender Funktionen und gib jeweils
die Koordinaten des Scheitels an:
\begin{enumerate}
\item
$ x \mapsto 3 x^2 - 6x - 3$
\item
$ x \mapsto 2x^2 + 4x - 3$
\end{enumerate}
\en
\ea 

\ba Computerraum
\begin{enumerate}
\item Besuche die Seite
http://www.mathe-online.at/galerie/fun1/fun1.html
und f�hre die folgenden 3 Java-Applets aus - bitte konzentriert arbeiten!
	a.) Funktionen erkennen 1 	
	b.) Graphen erkennen 1
      c.) Graphen zeichnen

\item Puzzlespiele:

\bn \item \url{http://johnny.ch/math/puzzle/lin_funkt_.php}

\item  \url{http://johnny.ch/math/puzzle/quad_funkt2.php}

\item \url{http://www.mathe-online.at/tests/fun1/erkennen.html}
\en 

\item Besuche die Seite 
	\url{http://de.wikipedia.org/wiki/Quadratische_Funktion}
und lies die Abschnitte 1.1 bis 1.5 konzentriert durch - was ist neu? Notiere dir das Wesentliche!

\item Der Mathemillion�r: \url{http://www.thorsten-vogelsang.de}

\item �bung Computerraum:

�ffne die Seite \begin{verbatim}
http://www.cybernautenshop.de/virtuelle_schule/lehrinhalte_index2.html
\end{verbatim}
und gehe zu den abgebildeten Kapiteln. Gute Arbeit!

\begin{center}
\includegraphics[width=3cm]{2te/quadratischefunktion/bilder/quadratischefunktionen.png}
\end{center}


\end{enumerate}
\ea 




\section{Schnittpunkte von Funktionen}

Sucht man die Schnittpunkte von zwei Funkionsgraphen $f_1$ und $f_2$, so muss man 
\bi \item die Funktionen gleichsetzen $f_1(x)=f_2(x)$
    \item die Gleichung l�sen.
\ei 

Schneidet man nun eine quadratische Funktion mit einer linearen, oder mit einer weiteren quadratischen, so entsteht immer eine quadratische Gleichung, welche keine, eine oder zwei L�sungen haben kann. Entsprechend gibt es keinen Schnittpunkt, einen sog. {\it Ber�hrungspunkt} oder zwei Schnittpunkte.

\begin{center}
\begin{tabular}{|p{4.5cm}p{4.5cm}p{4.5cm}|} \hline 
\includegraphics[width=4.5cm]{2te/quadratischefunktion/bilder/schnittparabel04.png}  & 
\includegraphics[width=4.5cm]{2te/quadratischefunktion/bilder/schnittparabel05.png}  & 
\includegraphics[width=4.5cm]{2te/quadratischefunktion/bilder/schnittparabel06.png}  \\
kein Schnittpunkt, Gerade ist eine {\it Passante} & ein Ber�hrungspunkt, Gerade ist eine {\it Tangente} & zwei Schnittpunkte, Gerade ist eine {\it Sekante} \\
\includegraphics[width=4.5cm]{2te/quadratischefunktion/bilder/schnittparabel01.png}  & 
\includegraphics[width=4.5cm]{2te/quadratischefunktion/bilder/schnittparabel02.png}  & 
\includegraphics[width=4.5cm]{2te/quadratischefunktion/bilder/schnittparabel03.png}  \\ \hline 
\end{tabular}
\end{center}

\bb Grundaufgaben
\bn \item Berechne die Schnittpunkte einer Geraden mit einer Parabel.
\item Berechne die Schnittpunkte zweier Parabeln.
\item Berechne einen Parameter, so dass es keine, bzw. genau einen, bzw. zwei Schnittpunkte gibt.
\en  
\eb


\ba \bn \item 
Gegeben ist die quadratische Funktion

$ y=-\frac{2}{3}\,x^2-3x+\frac{13}{8}$ ~~mit der Definitionsmenge $D=[-6;0]$.
\begin{enumerate}
\item   Zeichne den Graphen nach Berechnung der Scheitelkoordinaten sowie
        der Randpunkte sauber in ein Koordinatensystem ein!
\item   Gib die Wertemenge $\mathbb{W}$ an!
\item   Die Gerade mit der Gleichung $y=\frac{7}{3}$ schneidet den 
        Funktionsgraphen in zwei verschiedenen Punkten $P$ und $Q$.\\
        Trage $P$ und $Q$ in die Zeichnung ein und \underline{berechne}
        die Koordinaten dieser Punkte!
\end{enumerate}


\item
 Gegeben sind die Parabeln $p_1: \; y=0,5x^2+x+1,5$ und $p_2: \; y=-x^2+4x$\\
Untersuche \underline{rechnerisch}, ob sich die Parabeln schneiden.\\
Gib gegebenenfalls die Koordinaten gemeinsamer Punkte an.


\item Gegeben ist die Funktion $y=-1/2x^2+4x-5$ und die Geradenschar $y=kx+3$. F�r welche Werte von $k$ ist die Gerade Tangente an die Parabel?
\en

\ea 


\section{Die quadratische Interpolation}


\begin{minipage}{10cm}Sind zwei Punkte gegeben und eine Gerade (also eine lineare Funktion $y=kx+d$) gesucht, so f�hrt die Aufgabe auf ein lineares Gleichungssystem mit zwei Unbekannten (vgl. Kapitel \ref{linfkl_lings} auf Seite \pageref{linfkl_lings}). - Man nent diese Aufgabe auch {\bf lineare Interpolation}.

Bei einer {\bf quadratischen Interpolation} sucht man nun eine quadratische Funktion, also Werte f�r $a,b$ und $c$, so dass der Graph durch 3 gegebene Punkte verl�uft. Die Vorgehensweise ist dabei wieder dieselbe: Man muss dabei die drei Punkte in diese Gleichung einsetzen und  ein lineares Gleichungssystem mit drei Gleichungen und drei Unbekannten l�sen.

Welche Funktion verl�uft durch die  gezeichneten Punkte rechts?\\
		..............      
		\end{minipage} 
		\begin{minipage}{4cm}
		\includegraphics[width=4cm]{2te/quadratischefunktion/bilder/quadinterpol.jpg}
		\end{minipage}

\ba 
\bn \item Bestimme ausf"uhrlich die Gleichung derjenigen Parabel, welche durch
         die Punkte $P(-1|2),~Q(3|-22),~R(-7|-7)$ verl"auft!


\item Von einer Parabel ist der Scheitel $S(2|3)$ gegeben sowie der Punkt $(7|-1)$ Wie lautet die Funktionsgleichung?

\item
Der Graph der Funktion $x \mapsto ax^2 + bx + c$ ber"uhrt die x-Achse 
im Punkt $P(7|0)$ und geht durch den Punkt $Q(2|-75)$.\\
Bestimme a, b und c und gib diese Funktion an!

\item 
 
Der Graph einer Funktion $y=ax^2 + bx + c$ hat den Scheitel $S(10|-1)$
und geht durch den Punkt $P(9|2)$.
Bestimme a, b und c.
\en 
\ea 



\section{Extremwertaufgaben mit Parabeln}
Mit Extremwertaufgaben besch�ftigen wir uns eigentlich erst in der 5. Klasse (Differentialrechnung). Es geht im Allgemeinen darum, einen kleinsten oder gr��ten Funktionswert (in einem bestimmten Bereich) zu finden. Bei Parablen f�hrt diese Aufgabe auf die Berechnung des Scheitels. Dabei ist die $x$-Koordinate des Scheitels die {\it Extremstelle} und die $y$-Koordinate des Scheitels der {\bf Extremwert}.


\ba 
\bn 
\item 
Hat die Funktion $y = -0,8 x^2 + 0,2 x + 4$ einen gr"o"sten oder 
kleinsten Funktionswert? Begr"undung! Bestimme diesen Wert und gib
an, f"ur welchen x-Wert sie ihn annimmt. In welchem Bereich
(der x-Werte) steigt, in welchem f"allt der Graph der Funktion?

\item
 
Bilde ein Produkt, dessen erster Faktor um 1 gr"o"ser als $x$ und dessen
zweiter Faktor um 3 kleiner als $x$ ist. F"ur welche Zahl $x$
ist der Wert dieses Produkts am kleinsten?


\item
 
\begin{center}
\includegraphics[width=4cm]{2te/quadratischefunktion/bilder/kanone.jpg}
\end{center}

Die Flugbahn einer Kanonenkugel ist eine Parabel. Der Scheitel der
Flugbahn hat die Koordinaten S$(\,400\,{\rm m}\,|\,675\,{\rm m}\,)$,
der Abschusspunkt liegt in einer Felswand bei 
A$(\,200\,{\rm m}\,|\,375\,{\rm m}\,)$.
\begin{enumerate}
\item Berechne die Gleichung der Flugbahn in der Form $y=a\,x^2+b\,x+c$.
\item Bei welcher $x$-Koordinate f"allt die Kugel ins Meer?
\item \parfillskip0pt Die Flugbahn wird parallel zur $y$-Achse soweit
  nach oben verschoben, bis der Auftreffpunkt im Meer  
\end{enumerate}

\begin{enumerate}
\item[] bei $x=800\,$m liegt. 

Berechne die H"ohe $h'$ des neuen Abschusspunktes A$'(\,200\,{\rm m}\,|\,h'\,)$.
\item[(d)] Zeichne die beiden Flugbahnen in {\bf ein} Koordinatensystem 
($1\,{\rm cm}\,\widehat{=}\,100\,$m).
\end{enumerate}

\item
 
Zeige: Von allen Rechtecken mit Umfang $8\,\rm cm$ hat das Quadrat mit
Seitenl"ange $2\,\rm cm$ den gr"o"sten Fl"acheninhalt.


\item

 \begin{minipage}[t]{8cm}
Ein Quadrat ABCD hat die Seitenl"ange $4\,$cm. Tr"agt man von der Ecke C auf 
beiden anliegenden Seiten jeweils x cm ab, so erh"alt man die Punkte P 
und Q.\\
F"ur welchen x-Wert hat das Dreieck APQ den gr"o"sten Fl"acheninhalt?\\
Wie gro"s ist dieser?
\end{minipage}
\begin{minipage}[t]{6cm}
\includegraphics[width=6cm]{2te/quadratischefunktion/bilder/quadrat2.jpg}\end{minipage}


\en 

\ea 
% \input{2te/beschreibendestatistik/beschreibendestatistik.tex}
 %\input{2te/geometrie2/geometrie2.tex}
%\chapter{Trigonometrie}



Unter {\it Trigonometrie} (grch. {\it tri}=drei; {\it gonia}=Winkel, {\it metrein}=messen) versteht
man die Berechnung unbekannter Seiten/Winkel eines beliebigen Dreiecks aus einigen gemessenen
Seiten/Winkeln. In der {\it ebenen} Trigonometrie handelt es sich dabei um ebene Dreiecke, in der
{\it sph�rischen} Trigonometrie (grch. {\it sphaira}=Kugel, Ball) um Dreiecke auf einer
Kugeloberfl�che (z.B. die Erdkugel). Ob die Berechnung der gesuchten St�cke aus den bekannten
m�glich ist, dar�ber geben die {\it Kongruenzs�tze} (lat. {\it congruere}=�bereinstimmen) Auskunft.

\begin{center}
\includegraphics[width=5cm]{2te/trigonometrie1/bilder/landvermessung.jpg} $\qquad$
\includegraphics[width=5cm]{2te/trigonometrie1/bilder/kugeldreieeck.jpg}
\end{center}

% Mit Hilfe der Trig. ist beispielsweise folgende Aufgabenstellung l�sbar:
% 
% \bb  Wie kann man zum Beispiel herausfinden, wie weit $A$ und $B$ voneinander entfernt sind, wenn
% dazwischen ein Fluss oder ein Berg liegt? \eb
% 
%  \ba Gibt es zwei nicht kongruente Dreiecke, die in f�nf St�cken �bereinstimmen?\ea

\paragraph{Anwendung Erdoberfl�che}
Handelt es sich bei einer praktischen Vermessungsaufgabe um relativ "`kleine"' Dreiecke auf der
Erdoberfl�che, so kann man wegen des sehr gro�en Radius dieser Kugel ($R=6.371km$) ohne gro�e
Fehler von der Kr�mmung absehen und die Methoden der ebenen Trig. benutzen. Bei "`gro�en"'
Dreiecken jedoch (mit Seitenl�ngen �ber ca. $50km$), sind die sph�rischen Verfahren anzuwenden.

% \ba \label{12455ddkj} Wie gro� ist der Unterschied zwischen der Sehnenl�nge $s$ und der L�nge $b$
% des zugeh�rigen Kreisbogenst�cks beim Erdradius und f�r $s=50,  200$ und
% $1000km$?\footnote{Zur vollst�ndigen L�sung brauchen wir noch eine Vorbereitung. Die Aufgabe folgt
% nochmals sp�ter.}\ea

Zun�chst ben�tigen wir einen neuen Begriff aus der Geometrie: die �hnlichkeit.

\section[Geometrie: �hnlichkeit]{Geometrie 2: Strahlens�tze und �hnlichkeit}

\subsection{Wiederholung Geometrie 1}
\ba {\bf Wiederholung}

Rufen wir uns zun�chst wichtige S�tze und geometrische Grundlagen aus der ersten Klasse in Erinnerung:

\bn \item Was besagt die Dreiecksungleichung? \item Die Satzgruppe des Pythagoras besteht aus $3$ S�tzen. 
	\bn  	\item Wie hei�en sie? 
		\item Was besagen sie? 
		\item Interpretiere geometrisch! 
		\item Unter welcher wesentlichen Voraussetzung gelten sie?
	\en 

\item Die Heronsche Fl�chenformel erlaubt die direkte Berechnung des Fl�cheninhalts eines Dreiecks aus den Seitenl�ngen $a$, $b$ und $c$. Wie lautet sie?

\item Die Winkelsumme (Summe der Innenwinkel) in einem $n$-Eck kann in Abh�ngigkeit von $n$ berechnet werden. Wie hei�t die entsprechende Formel? (Anleitung: �berlege dir zun�chst die Summe im Dreieck, Viereck, F�nfeck)

\item Stichwort: Kongruenz. \bn \item Wann sind zwei Figuren kongruent? \item Unter welchen Bedingungen sind zwei Dreiecke kongruent? \en 

\item Stichwort: Symmetrie. Man unterscheidet zwischen Achsen- und Punktsymmetrie. \bn \item Welche Gro�buchstaben des deutschen Alphabets sind achsen-, welche punktsymmetrisch? \item Ist ein Kreis symmetrisch? \item Ist ein Mensch symmetrisch? Wer hat sich u.a. damit befasst? \en 

\item Stichwort: besondere Punkte und Linien im Dreieck
\bn \item Welche besonderen $4$ Punkte in einem Dreieck kennst du? \item Als Schnittpunkt welcher Linien ergeben sich jeweils die $4$ Punkte? \item  An welches Zahlenverh�ltnis erinnert dich der Schwerpunkt und warum? \en 

\item Gegeben ist ein gleichseitiges Dreieck mit der Seitenl�nge $s$. Gib in Abh�ngigkeit von $s$ \bn \item den Umfang \item die H�he des Dreiecks, \item den Fl�cheninhalt an. \item Wie gro� ist der Fl�cheninhalt eines regelm��igen Sechsecks? \en

\item Zerlegung einer Strecke in $n$ gleich lange Teile. Wie geht man konstruktiv vor? Zerlege z.B. eine beliebige Strecke in $5$ gleich lange Teile.

\item Teilung einer Strecke im Verh�ltnis $m$ zu $n$. Wie geht man konstruktiv vor? Z.B. soll der Punkt $T$ die Strecke $\overline{AB}$ im Verh�ltnis $1:5$ teilen.
\en 

\ea 

\subsection{�hnlichkeit}

\begin{minipage}{8cm}
Mit �hnlichkeit hat man es immer zu tun, wenn von einer Zeichnung (von einem Plan, einer Fotografie, ...) eine Vergr��erung oder eine Verkleinerung hergestellt wird. 

�hnliche Figuren sehen bis auf die Gr��e gleich aus. Sie haben gleiche Winkel und gleiche
Streckenverh�ltnisse.
\end{minipage}
\begin{minipage}{7cm}
\begin{flushright}
\includegraphics[width=5.5cm]{2te/geometrie2/bilder/aehnlichkeit.jpg}
\end{flushright}
\end{minipage}



\bd  Dreiecke (bzw. ebene Figuren, die aus Dreiecken zusammengesetzt sind) hei�en  {\bf �hnlich}, wenn sie in entsprechenden Winkeln �berein stimmen. F�r �hnliche Figuren verwendet man das Zeichen $\sim$.\ed 

\bs  Zwei Dreiecke $ABC$ und $A'B'C'$ sind genau dann �hnlich wenn eine der folgenden Bedingungen erf�llt ist: \bi 
\item  Sie haben lauter gleiche Winkel
\item Sie haben lauter gleiche Streckenverh�ltnisse
%\item Sie haben einen gemeinsamen Winkel und ein gemeinsames Streckenverh�ltnis, und zwar das Verh�ltnis der Seiten, die diesen Winkel einschlie�en (oder von zwei anderen Seiten, sofern die gr��ere Seite dem Winkel gegen�ber liegt).
\ei \es   

\ba  \bn \item Sind �hnliche Figuren immer kongruent? Sind kongruente Figuren immer �hnlich? \item  Gegeben ist das rechtwinklige Dreieck $ABC$ mit $\gamma=90�$. Der H�henfu�punkt von $h_c$ sei $F$. Erkennst du �hnliche Dreiecke?  \item Sind zwei gleichschenklige, zwei gleichseitige Dreicke, zwei Quadrate, zwei Rechtecke, zwei Kreise   immer �hnlich?   \item Konstruiere zwei �hnliche Dreiecke mit der Eigenschaft $a:c=3:4$ und $\gamma=60�$.
\item Ein Dreieck hat die Seiten $a=5$, $b=6$ und $c=7$. \bn \item Konstruiere ein dazu �hnliches Dreieck,  \item Konstruiere ein dazu �hnliches Dreieck, dessen Umfang um $6$cm kleiner ist.\en 

\en \ea

\subsection{Strahlens�tze}

Die Umkehrung des obigen Satzes f�hrt auf den wichtigen Strahlensatz:

\bs  Gegeben sind zwei Geraden mit dem Schnittpunkt $S$. Die Geraden werden von zwei weiteren {\it parallelen} Geraden geschnitten. Dann gilt:


\begin{minipage}{8cm}
\paragraph{1. Strahlensatz:} Das Verh�ltnis der L�ngen von zwei Abschnitten auf einem Strahl ist gleich dem Verh�ltnis der L�ngen der zwei entsprechenden Abschnitte auf dem zweiten Strahl, kurz:

$$\frac{\overline{SA_1}}{\overline{SA_2}}=\frac{\overline{SB_1}}{\overline{SB_2}}$$

\end{minipage}
\begin{minipage}{6cm}
\begin{flushright}
\includegraphics[width=7cm]{2te/geometrie2/bilder/ss1.jpg}
\end{flushright}
\end{minipage}

\paragraph{2. Strahlensatz:}  Das Verh�ltnis der L�ngen von zwei Abschnitten auf einem Strahl ist gleich dem Verh�ltnis der L�ngen der zwei entsprechenden Abschnitte auf den parallelen Geraden, kurz: 

$$\frac{\overline{SA_1}}{\overline{SA_2}}=\frac{\overline{A_1B_1}}{\overline{A_2B_2}}$$
\es  

\begin{proof}
 Da  $\triangle SA_1B_1 \sim \triangle SA_2B_2$ ist, verhalten sich entsprechende Seiten gleich.
\end{proof}

%\subsection{Die Zentrische Streckung}
%\bd  Eine {\bf zentrische Streckung} ist eine Abbildung mit einem Fixpunkt $Z$ und einer Zahl $k\neq 0$, bei der f�r jeden Punkt $P$ und seinen Bildpunkt $P'$ gilt: $$\overline{ZP'}=k \cdot \overline{ZP} \qquad \mbox{bzw.} \frac{\overline{ZP'}}{\overline{ZP}}=\frac{k}{1}\qquad $$\ed

%\newpage 


%\section*{�bungen}


\section{Winkelmessung} 
\bd  Ein {\bf Winkel} ist gegeben durch einen Punkt, den {\it
Scheitelpunkt} des Winkels, und zwei von diesem Scheitelpunkt ausgehende Halbgeraden, den {\it
Schenkeln} des Winkels.

Die {\bf Gr��e eines Winkels}  wird als das Verh�ltnis der �ffnung zwischen seinen Schenkeln im
Vergleich zum vollen Kreis bestimmt.\ed 

\subsection{Gradma�} Misst man Winkel im {\it Gradma�}, so entspricht dem vollen Kreis $360$,
d.h. ein Winkel von $1$ ist den $360$sten Teil eines Vollkreises ge�ffnet. Zur feineren Messung
wird und wurde jedes Grad in sechzig Minuten $(60')$ und jede Minute wiederum in sechzig Sekunden
$(60'')$ unterteilt\footnote{Diese Einteilung geht auf die Babylonier zur�ck, die im
Sexagesimal-System rechneten. Die Bezeichnungen Minute und Sekunde stammen aus dem Lateinischen:
{\it minuere}=vermindern, verkleinern; {\it secundus}=der Zweite, d.h. Sekunde bedeutet den zweiten
Verkleinerungsschritt.}. Heute werden Winkel aber auch sehr oft als (einfache) Dezimalzahl
angegeben.



\subsection{Neugrad} Im Vermessungswesen wird der Vollkreis auch in $400$gon (Neugrad) eingeteilt.
Bruchteile von $1$gon werden dabei im Dezimalsystem angegeben.


\subsection{Bogenma�} Eine andere M�glichkeit f�r die Gr��enangabe eines Winkels besteht im {\it
Bogenma�}. Dazu wird die L�nge des Bogens angegeben, den der Winkel aus dem Kreis vom Radius $1$
herausschneidet. Da der Umfang des vollen Einheitskreises gleich $\dots$ ist, besitzt ein rechter
Winkel das Bogenma�$\dots$ . Bezeichnet $\alpha$ das Gradma� eines Winkels und $x$ das Bogenma�
so gilt:
$$\frac{\alpha}{360}=\frac{}{}$$

\subsection{Zusammenfassung}

{\bf Wichtig:} F\"{u}r alle Berechnungen muss am Taschenrechner das korrekte Ma{\ss} eingestellt
sein, ansonsten liefert dieser unbrauchbare Ergebnisse. Z.B. F\"{u}r die Bererchnung von
$\sin(30)$ m\"{u}ssen die Altgrad "`Deg"' eingestellt sein. Jeder MUSS in der Lage sein, bei
seinem Taschenrechner die entsprechenden Einstellungen zu kontrollieren und gegebenenfalls zu
\"{a}ndern!

\ba \bn \item Erg�nze:

\begin{center}
\begin{tabular}{|c|c|c|c|c|}
  % after \\: \hline or \cline{col1-col2} \cline{col3-col4} ...
\hline  Taschenrechner & Kreis & Umfang & Halbkreis & Viertelkreis \\ \hline
  DEG & Altgrad &  &  &  \\
  GRAD & Neugrad &  $400^g$ (gon) &  &  \\
  RAD & Bogenma{\ss} &  &  &  \\ \hline
\end{tabular}
\end{center}


\item  Was versteht man unter den folgenden Begriffen? \bn \item rechter Winkel \item senkrecht oder
orthogonal \item spitze und stumpfe Winkel \en 

\item \bn \item Rechne in Minuten und Sekunden um: $3,2579�$.
\item Rechne in eine Dezimalzahl um: $25�30'65''$
\item Kontrolliere deine Aufgabe mit Hilfe der Taschenrechnertasten \frame{$D^\circ M'S$} und %$\frame{\leftrightarrow DEG}$? Findest du diese Tasten auf dem Taschenrechner?
\en 

\item Rechne ins Gradma{\ss} um: $5\pi$; $\frac{\pi}{4}$; $\frac{5\pi}{12}$

\item Rechne ins Bogenma{\ss} um:  $0$; $40$; $1000$

\item Wie gro� ist das Gradma� eines Winkel, dessen Bogenma� gleich $1$ ist? Wie gro� ist das
Bogenma� eines Winkels, der das Gradma� $1$ besitzt? Wie gro� ist das Gradma� eines Winkel von
$1$gon, wie gro� ist ein Winkel von $1$ in Neugrad?

\item(Matura 2004 Frage 8) Winkel werden unter Verwendung eines geeigneten Winkelma{\ss}es gemessen.
Die gebr\"{a}uchlichsten sind Altgrad (sexagesimale Grade), Radianten und Neugrad. Welches sind
ihre jeweiligen Definitionen? \en\ea

\subsection{Orientierung eines Winkels}

Man kann durch Unterscheidung von erstem und zweitem Schenkel eines Winkels diesen orientieren,
wodurch auch {\it negative} Werte f�r die Gr��e eines Winkels m�glich sind. �blicherweise wird der
Wert positiv gerechnet, wenn beim �bergang vom ersten zum zweiten Schenkel der Scheitelpunkt im
mathematisch positiven Sinn umlaufen wird, also entgegegen dem Uhrzeigersinn.

\ba Woran liegt es eigentlich, dass sich die Zeiger unserer Analoguhren alle "`im Uhrzeigersinn"'
drehen?\ea

\section{Winkelfunktionen}

\subsection{Sinus und Kosinus}


\begin{minipage}{10cm} 
\bd  Der {\bf Einheitskreis} ist ein Kreis mit Mittelpunkt $(0|0)$ und Radius $1$.\ed  
Legt man einen orientierten Winkel $x$ mit seinem Scheitelpunkt in den Ursprung $0$ eines
kartesischen Koordinatensystems (R. Descartes, 1596-1650) und den ersten Schenkel auf die
positive $x$-Achse, so schneidet der zweite Schenkel den Einheitskreis in einem Punkt $P$ mit den
Koordinaten $(x_0|y_0)$.

\ba  Wie gro� ist der Umfang im Einheitskreis? \ea
\end{minipage}
\begin{minipage}{6cm}
  \psset{unit=2cm}
  \begin{pspicture}(-1.25,-1.25)(1.25,1.25)
    \psaxes{->}(0,0)(-1.25,-1.25)(1.25,1.25)
    \uput[225](1.25,0){$\bm{x}$}
    \uput[225](0,1.25){$\bm{y}$}
    \pscircle(0,0){1}
    \psline(0,0)(0.866,0.5)
    \psline[linestyle=dashed, linewidth=0.01](0.866,0)(0.866,0.5)
    \psline[linestyle=dashed,  linewidth=0.01](0,0.5)(0.866,0.5)
    \uput[45](0.866,0.5){$P(x_0|y_0)$}
    \uput[270](0.866,0){$x_0$}
    \uput[180](0,0.5){$y_0$}
    \psarc(0,0){0.25}{1}{29}
    \uput[45](0.25;5){$x$}
  \end{pspicture}
\end{minipage}



Die Koordinaten $x_0$ und $y_0$ des Punktes $P$ h�ngen vom Winkel $x$ ab, wir definieren deshalb
zwei wichtige (Winkel-)Funktionen:

\bd   Es sei $x$ der Winkel im Einheitskreis. Man bezeichnet $y_0$ als den {\bf Sinus} von $x$ und $x_0$ als {\bf Kosinus} von $x$:
$$\sin(x)=y_0 \qquad \mbox{und} \qquad \cos(x)=x_0$$

Der Sinus- und der Kosinus ist nicht nur f�r Winkel im 1. Quadranten definiert. Erg�nze f�r stumpfe und �berstumpfe Winkel die Grafik!
  \psset{unit=2cm}
  \begin{pspicture}(-1.25,-1.25)(1.25,1.25)
    \psaxes{->}(0,0)(-1.25,-1.25)(1.25,1.25)
    \uput[225](1.25,0){$\bm{x}$}
    \uput[225](0,1.25){$\bm{y}$}
    \pscircle(0,0){1}
    \psline[linewidth=0.3pt](0,0)(0.866,0.5)
    \psline[linestyle=dashed, linewidth=0.3pt](0.866,0)(0.866,0.5)
    \psline[linestyle=dashed, linewidth=0.3pt](0,0.5)(0.866,0.5)
    \uput[45](0.866,0.5){$P(x_0|y_0)$}
%     \uput[270](0.866,-0.5){$x_0$}
%     \uput[180](-0.5,0.5){$y_0$}
%     \psdot*[dotsize=4pt](0.866,0.5)
    \psarc[linewidth=0.3pt](0,0){0.25}{1}{29}
    \uput[45](0.22;3){$x$}
    \psbrace[ref=c, nodesepB=-5pt,linewidth=0.5pt, rot=90](0,-0.05)(0.866,-0.05){$\cos x$}
    \psbrace[ref=c, nodesepA=-5pt, linewidth=0.5pt, rot=270](-0.05,0.5)(-0.05,0){$\sin x$}
  \end{pspicture}
  \begin{pspicture}(-1.25,-1.25)(1.25,1.25)
    \psaxes{->}(0,0)(-1.25,-1.25)(1.25,1.25)
    \uput[225](1.25,0){$\bm{x}$}
    \uput[225](0,1.25){$\bm{y}$}
    \pscircle(0,0){1}
%     \psline(0,0)(0.866,0.5)
%     \psline[linestyle=dashed](0.866,0)(0.866,0.5)
%     \psline[linestyle=dashed](0,0.5)(0.866,0.5)
%     \uput[45](0.866,0.5){$P(x_0|y_0)$}
%     \uput[270](0.866,0){$x_0$}
%     \uput[180](0,0.5){$y_0$}
%     \psarc(0,0){0.25}{1}{29}
%     \uput[45](0.22;7){$x$}
  \end{pspicture}
  \begin{pspicture}(-1.25,-1.25)(1.25,1.25)
    \psaxes{->}(0,0)(-1.25,-1.25)(1.25,1.25)
    \uput[225](1.25,0){$\bm{x}$}
    \uput[225](0,1.25){$\bm{y}$}
    \pscircle(0,0){1}
%     \psline(0,0)(0.866,0.5)
%     \psline[linestyle=dashed](0.866,0)(0.866,0.5)
%     \psline[linestyle=dashed](0,0.5)(0.866,0.5)
%     \uput[45](0.866,0.5){$P(x_0|y_0)$}
%     \uput[270](0.866,0){$x_0$}
%     \uput[180](0,0.5){$y_0$}
%     \psarc(0,0){0.25}{1}{29}
%     \uput[45](0.22;7){$x$}
  \end{pspicture}

Jedem Winkel $x\in \mathbb{R}$ wird also eine L�nge, also eine Zahl, zugeordnet, es handelt sich also jeweils um eine Funktion, der {\bf Sinusfunktion} und der {\bf Kosinusfunktion}.

\begin{eqnarray*}
\sin: \mathbb{R}  \longrightarrow [-1,1] \qquad & &  \qquad  \cos: \mathbb{R}  \longrightarrow [-1,1] \\
x \longmapsto \sin(x) \qquad & &  \qquad \qquad  \quad x \longmapsto \cos(x) 
\end{eqnarray*}


\ed  

\subsection{Eigenschaften der Funktionen $\sin$ und $\cos$}


\bs   \label{satzdege} \bn 



\item Es gilt der sog. trigonometrische Pythagoras: $$\sin^2(x)+\cos^2(x)=1$$

\item Die Winkelfunktionen $\sin$ und $\cos$ sind {\bf periodische} Funktionen mit der
Periode $2\pi$, d.h. es gilt f�r alle $k \in \mathbb{Z}$ und Winkel $x$

$$\sin(x)=\sin(x+2k\pi) \qquad \mbox{und} \qquad \cos(x)=\dots$$

\item F�r die {\bf Nullstellen} dieser Funktionen gilt

\begin{eqnarray*}\sin(x)&=&0 \qquad \Leftrightarrow \qquad \dots \\
\cos(x)&=&0 \qquad \Leftrightarrow \qquad x=\frac{\pi}{2}+k\pi \qquad \mbox{f�r ein}\qquad k \in
\mathbb{Z}\end{eqnarray*}

\item Die Funktion $\sin$ ist {\it ungerade}, die Funktion $\cos$ {\it gerade}, d.h. f�r alle
Winkel $x$ gilt $$\sin(-x)=... \qquad \qquad \mbox{und}  \qquad \qquad \cos(-x)=\dots$$

\item F�r alle Winkel $x$ gilt $$\sin(\pi -x)=\sin(x) \qquad \qquad \mbox{und} \qquad \qquad
\cos(\pi-x)=\dots$$

\en \es  

\ba \bn \item Veranschauliche all diese Eigenschaften graphisch am Einheitskreis.
\item Warum enth�lt der Wertebereich von Sinus und Kosinus nicht Zahlen gr��er als $1$, bzw. kleiner als $-1$?

\item  ({\bf Matura2005F9}) Berechnen Sie, ohne Benutzung des Taschenrechners, den Wert von: $$\sin^2(35�)+\sin^2(55�)$$ Die
Gr��en der Winkel sind im Sexagesimalsystem gegeben.
\item Formuliere Satz \ref{satzdege} im Gradma�.\en\ea

\subsection{Die Funktionsgraphen}
\paragraph{Die Berechnung von Sinus- und Kosinuswerten} Wie findet man Zahlenwerte von Sinus und Kosinus.
Fr\"{u}her gab es dazu Tabellen in denen nachgeschaut werden musste, heutzutage im Zeitalter des
Taschenrechners, braucht man nur den Winkel einzugeben und die Taste {\it $\sin$} oder {\it $\cos$}
dr\"{u}cken. Ist der Sinus oder Kosinus eines Winkels gegeben und der Winkel ist gesucht, so muss
genau die {\it Umkehrung} durchgef\"{u}hrt werden, die entsprechende Taste wird oft mit $\sin^{-1}$
bzw. mit $\cos^{-1}$ bezeichnet.

\ba \bn \item Zeichne die zwei Funktionsgraphen in ein Koordinatensystem ein (�berlege dir dabei die Nullstellen und die Extremwerte der beiden Funktionen!)  (Tipp: W�hle den Winkel im
Bogenma�).



\begin{center}
\psset{xunit=7mm, yunit=5mm}
 \begin{pspicture}(-10,-3.5)(10,4)
 	\psset{plotpoints=1000}
 	\psaxes[labels=none, ticks=none]{->}(0,0)(-9.7,-3.3)(9.7,3.7)
% %Problem: missing number - treated by zero ????????????
%    \uput[-90](9.5,-0.05){$x$}
%    \uput[180](-1,3.2){$y$}
   \uput[-90](-9.43,-0.3){$-3\pi$}
   \uput[-90](-6.28,-0.3){$-2\pi$}
   \uput[-90](-3.14,-0.3){$-\pi$}
   \uput[-90](3.14,-0.3){$\pi$}
   \uput[-90](6.28,-0.3){$2\pi$}
   \uput[-90](9.43,-0.3){$3\pi$}
   \uput[180](0,-3.14){$-1$}
   \uput[180](0,3.14){$1$}
%   \rput(0.3,-0.3){O}
%    \psplot{-3.14159}{3.1415}{x}
%    \psplot{3.14159}{9.43}{x 6.283 neg add}
%    \psplot{-9.43}{-3.1415}{x 6.283 add}
  	\psline[linewidth=0.75pt](-9.43,-0.25)(-9.43,0.25)
   	\psline[linewidth=0.75pt](-6.28,-0.25)(-6.28,0.25)
  	\psline[linewidth=0.75pt](6.28,-0.25)(6.28,0.25)
  	\psline[linewidth=0.75pt](9.43,-0.25)(9.43,0.25)
   	\psline[linewidth=0.75pt](-3.14,-0.25)(-3.14,0.25)
  	\psline[linewidth=0.75pt](3.14,-0.25)(3.14,0.25)
   	\psline[linewidth=0.75pt](-0.25,-3.14,)(0.25,-3.14)
   	\psline[linewidth=0.75pt](-0.25,3.14,)(0.25,3.14)
%  	\psline[linewidth=0.5pt,linestyle=dashed](-3.14159,-3.14159)(-3.14159,3.14159)
%  	\psline[linewidth=0.5pt,linestyle=dashed](3.14159,-3.14159)(3.14159,3.14159)
 \end{pspicture}
%}
\end{center}

\item Geonext: Konstruiere einen Einheitskreis mit einem Gleiter. Nun sollen sich die Funktionen Sinus und Kosinus jeweils als Spur ergeben, indem man den Gleiter animiert (verwende dabei verschiedene Farben).
\en 
\ea


\subsection{Tangens und Kotangens}





\begin{minipage}{9cm}
\bd  \bn \item F�r alle Winkel $x \neq \frac{\pi}{2}+k\pi$ $(k \in \mathbb{Z})$ definiert man als
{\bf Tangensfunktion}

$$\tan(x)=\frac{\sin(x)}{\cos(x)}$$

  \item F�r alle Winkel $x \neq k\pi$ $(k \in \mathbb{Z})$ definiert man als
{\bf Kotangensfunktion}

$$\cot(x)=\frac{1}{\tan(x)}=\frac{\cos(x)}{\sin(x)}$$
\en \ed  
\end{minipage}
\begin{minipage}{6cm}
\includegraphics[width=6cm]{2te/trigonometrie1/bilder/einheitskreissincostancotan.jpg}
\end{minipage}


\ba Zeichne die zwei Funktionsgraphen in ein Koordinatensystem ein (Tipp: W�hle den Winkel im
Bogenma�)

\begin{center}
\psset{xunit=7mm, yunit=7mm}
 \begin{pspicture}(-10,-3.5)(10,4)
 	\psset{plotpoints=1000}
 	\psaxes[labels=none, ticks=none]{->}(0,0)(-9.7,-3.3)(9.7,3.7)
% %Problem: missing number - treated by zero ????????????
%    \uput[-90](9.5,-0.05){$x$}
%    \uput[180](-1,3.2){$y$}
   \uput[-90](-9.43,-0.3){$-3\pi$}
   \uput[-90](-6.28,-0.3){$-2\pi$}
   \uput[-90](-3.14,-0.3){$-\pi$}
   \uput[-90](3.14,-0.3){$\pi$}
   \uput[-90](6.28,-0.3){$2\pi$}
   \uput[-90](9.43,-0.3){$3\pi$}
%    \uput[180](0,-3.14){$-1$}
%    \uput[180](0,3.14){$1$}
% %   \rput(0.3,-0.3){O}
%    \psplot{-3.14159}{3.1415}{x}
%    \psplot{3.14159}{9.43}{x 6.283 neg add}
%    \psplot{-9.43}{-3.1415}{x 6.283 add}
  	\psline[linewidth=0.75pt](-9.43,-0.25)(-9.43,0.25)
   	\psline[linewidth=0.75pt](-6.28,-0.25)(-6.28,0.25)
  	\psline[linewidth=0.75pt](6.28,-0.25)(6.28,0.25)
  	\psline[linewidth=0.75pt](9.43,-0.25)(9.43,0.25)
   	\psline[linewidth=0.75pt](-3.14,-0.25)(-3.14,0.25)
  	\psline[linewidth=0.75pt](3.14,-0.25)(3.14,0.25)
%    	\psline[linewidth=0.75pt](-0.25,-3.14,)(0.25,-3.14)
%    	\psline[linewidth=0.75pt](-0.25,3.14,)(0.25,3.14)
%  	\psline[linewidth=0.5pt,linestyle=dashed](-3.14159,-3.14159)(-3.14159,3.14159)
%  	\psline[linewidth=0.5pt,linestyle=dashed](3.14159,-3.14159)(3.14159,3.14159)
 \end{pspicture}
%}
\end{center}\ea



\bs   
	\bn \item Die Winkelfunktionen $\tan$ und $\cot$ sind {\bf periodische} Funktionen mit der
	Periode $\pi$, d.h. es gilt f�r alle $k \in \mathbb{Z}$ und Winkel $x$

	$$\tan(x)=\dots \qquad \mbox{und} \qquad \cot(x)=\dots$$

	\item F�r die {\bf Nullstellen} dieser Funktionen gilt

	\begin{eqnarray*}\tan(x)&=&0 \qquad \Leftrightarrow \qquad \dots \\
	\cot(x)&=&0 \qquad \Leftrightarrow \qquad x=\dots \qquad \mbox{f�r ein}\qquad k \in
	\mathbb{Z}\end{eqnarray*}

	\item Die Funktionen $\tan$ und $\cot$  sind {\it \dots},  d.h. f�r alle
	Winkel $x$ gilt $$\tan(-x)=... \qquad \qquad \mbox{und}  \qquad \qquad \cot(-x)=\dots$$

	\item F�r alle Winkel $x$ gilt $$\tan(\pi -x)=-\tan(x) \qquad \qquad \mbox{und} \qquad \qquad
	\cot(\pi-x)=\dots$$

	\en 
\es  

\subsection*{�bungen}
\bn \item  Beweise diese Eigenschaften!


\item Beweise die Periode des Tangens: $\tan(x+\pi)=\tan(x)$ mit $\tan x=\frac{\sin x}{\cos x}$

\item Erg�nze: $\sin(x+\pi)=...$, $\cos(x+\pi)=...$, $\tan(x+\pi)=...$!

\item Erg�nze: $\sin(\frac{\pi}{2}\pm x)=...$, $\cos(\frac{\pi}{2}\pm x)=...$, 



\item Zeichne jeweils $\sin \alpha$ und $\cos \alpha$ ein. Nun sollte es dir nicht schwer fallen, die
Vorzeichentabelle f\"{u}r den Sinus, Cosinus und Tangens auszuf\"{u}llen


  \includegraphics[width=\textwidth]{2te/trigonometrie1/bilder/klaus3.jpg}

\item  Ermittle die Werte nicht mit dem Taschenrechner sondern anhand von �berlegungen am
Einheitskreis: $\sin \pi$; $\tan \frac{\pi}{2}$; $\cos 3\pi $ 


\item \bn \item {\bf Computerraum} Diese Seiten sind im Computerraum mit Hilfe der folgenden
Internetadressen zu erg\"{a}nzen:

\bi \item http://www.mathe-online.at/galerie/fun2/fun2.html\#sincostan

Beachte die M\"{o}glichkeit zwischen dem Gradma{\ss} und dem Bogenma{\ss} zu wechseln!

\item http://www.walter-fendt.de/m11d/sincostan.htm

\item http://www.ies.co.jp/math/java/trig/

Beschr\"{a}nke dich hier auf den Abschnitten "`Sine Function Box"', "`Cosine Function Box"' und
"`Tangent Function Box"' bzw. "`The graph of $y=\sin x$"',  "`The graph of $y=\cos x$"'. \ei

\item Die trigonometrischen Funktionen und Excel: Erstelle eine Wertetabelle (im Bogenma{\ss}) von
-10 bis 10 mit der Schrittweite 0,2 f\"{u}r die Funktionen Sinus und Kosinus und lass dir die Werte
in einem einzigen Diagramm (Achtung: Diagrammtyp ist nicht f�rei w\"{a}hlbar - w\"{a}hle:
"`Punkt(xy)"' und lass dir die Linien verbinden) einzeichnen. Formatiere es und f\"{u}ge es in die
Worddatei ein. {\bf Hinweis}: Excel rechnet im Bogenma{\ss} - bei Bedarf m\"{u}sste man die Werte
ins Gradma{\ss} umrechnen.) \en

\item Erg�nze:   \bi \item Die Sinusfunktion hat den Definitionsbereich $\mathbb{D}=$\\
und den Wertebereich $\mathbb{W}=$

\item Die Kosinusfunktion hat den Definitionsbereich $\mathbb{D}=$\\
und den Wertebereich $\mathbb{W}=$

\item Die Tangensfunktion hat den Definitionsbereich $\mathbb{D}=$\\
und den Wertebereich $\mathbb{W}=$ \ei


 Den maximalen Wert ($=1$) nimmt die \bi \item Sinusfunktion f\"{u}r alle Werte $x=$
\item Kosinusfunktion f\"{u}r alle Werte $x= $ an.\ei

Den minimalen Wert ($=-1$) nimmt die \bi \item Sinusfunktion f\"{u}r alle Werte $x=$
\item Kosinusfunktion f\"{u}r alle Werte $x=$ an.\ei

\item Welche Aussagen sind richtig?

\bn \item Sinus und Kosinus haben die gleichen Nullstellen. \dotfill $\Box$
\item Kosinus und Tangens haben die gleichen Nullstellen. \dotfill $\Box$
\item Die Asymptoten der Tangenskurve schneiden die x-Achse in den Nullstellen der Kosinusfunktion. \dotfill
$\Box$
\item F\"{u}r $-\frac{\pi}{4}\leq x \leq \frac{\pi}{4}$ ist $-1 \leq \tan(x) \leq 1$. \dotfill
$\Box$
\en 
\en

\ba  �bung Computerraum:

�ffne die Seite \begin{verbatim}
http://www.cybernautenshop.de/virtuelle_schule/lehrinhalte_index2.html
\end{verbatim}
und gehe zum den abgebildeten Kapiteln. Gute Arbeit!

\begin{center}
\includegraphics[width=4cm]{2te/trigonometrie1/bilder/uebungen_trigo.png}
\end{center}

\ea 


% 
% \subsection{Die allgemeine Sinusfunktion $y\! =\!A\sin \left[ b \left(x+c \right) \right] +d$}
% \begin{enumerate}
% \paragraph{Die Funktionen $y\! =\!A \sin x$}
% \item
% Zeichne die Funktionen $y\! =\!\sin x$, $y\! =\!-\sin x$ und $y\! =\!2\sin x$ in ein Schaubild (nicht zu mickrig!)!\\[1em]
% Wir bezeichnen bei der Funktion $y = A \sin x$ den Wert $|A|$ als {\bf Amplitude}.\\
% Wir erhalten die Kurve der Funktion $y = -A \sin x$ aus der Kurve der Funktion $y = A \sin x$, indem wir die Kurve an der x-Achse spiegeln.\\
% Falls $|A|< 1$ ist, dann ist die Kurve im Vergleich zu $y\! =\!\sin x$ in y-Richtung gestaucht,\\
% f�r $|A| > 1$ ist die Kurve in y-Richtung gestreckt.
% \item
% Zeichne zu den obigen Aussagen Schaubilder, die den Sachverhalt verdeutlichen!\\[1em]
% \paragraph{Die Funktion $y\! =\!\sin \left(b x \right)$, wobei b $\neq 0$}
% \item
% Zeichne die Funktionen $y\! =\!\sin x$, $y\! =\!\sin \left(2 x \right)$ und $y\! =\!\sin \left(\dfrac{x}{2} \right)$ in ein Schaubild\\[1em]
% Falls $|b| < 1$ ist, dann ist die Kurve im Vergleich zu $y\! =\!\sin x$ in x-Richtung gestaucht,\\
% f�r $|b| > 1$ ist die Kurve in x-Richtung gestreckt.\\
% Wir erhalten die Periode der Sinusfunktion $y\! =\!\sin \left(b x \right)$ durch $\dfrac{2 \pi}{\|b\|}$\\
% F�r das Zeichnen einer Sinuskurve ist es wichtig, mindestens eine Periode einzutragen. Empfohlene Schrittweite: mindestens $\dfrac{Periode}{8}$\\
% Zeichne zu den obigen Aussagen Schaubilder, die den Sachverhalt verdeutlichen!\\[1em]
% \paragraph{Die Funktion $y\! =\!\sin \left(x+c \right)$}
% \item
% Zeichne die Funktionen $y\! =\!\sin x$, $y\! =\!\sin \left( x- \dfrac{\pi}{2} \right)$ und $y\! =\!\sin \left( x+\dfrac{\pi}{2} \right)$ in ein Schaubild\\[1em]
% Wir nennen c die Phasenverschiebung, damit wird eine Sinuskurve in x-Richtung verschoben.
% Falls $c < 0$ ist, dann ist die Kurve nach rechts verschoben,\\
% falls $c > 0$ ist, dann ist die Kurve nach links verschoben.\\
% Zeichne zu den obigen Aussagen Schaubilder, die den Sachverhalt verdeutlichen!\\[2em]
% \paragraph{Die Funktion $y\! =\!\sin x +d$}
% \item
% Zeichne die Funktionen $y\! =\!\sin x$, $y\! =\!\sin x\! +\! 1$ und $y\! =\!\sin x\! -\! 1$ in ein Schaubild\\[2em]
% d bewirkt eine Verschiebung in y-Richtung.\\
% Ist $d < 0$, so wird die Kurve nach unten verschoben, \\
% ist $d > 0$, so wird die Kurve nach oben verschoben.\\
% Zeichne zu den obigen Aussagen Schaubilder, die den Sachverhalt verdeutlichen!\\[1em]
% 
% 
% \paragraph{Die Funktion $y\! =\!A\sin \left[ b \left(x+c \right) \right] +d$}
% \item
% Zeichne die Funktion $y\! =\!-\sin \left[ 2 \left(x+ \dfrac{\pi}{4} \right) \right] +1$\\
% 
% \item
% Gib den Funktionsterm folgender Sinuskurven an!
% 
% \includegraphics[width=12cm]{2te/trigonometrie1/bilder/klaus7.jpg}
% \end{enumerate}
% 
% 
% {\bf Anwendung: Modellierung eines periodischen Vorganges mit Excel}
% 
% \includegraphics[width=\textwidth]{2te/trigonometrie1/bilder/allgemeinesinusfunktion.jpg}
% 
% {\it Datei: Sonnenscheindauer allgemeine Sinusfunktion.xls}
% 
% 
% \subsection{Arkusfunktionen - die trigonometrischen Umkehrfunktionen}
% 
%  \label{umkehr}
% 
% 
% 
% %\input{umkehrfunktionen.tex}
% %\subsubsection{Die Arkusfunktionen}
% 
%  \label{Arkus}\bd  Die Umkehrfunktionen der trigonometrischen Funktionen
% ($\sin, \cos, \tan, \cot$) nennt man {\bf Arkusfunktionen} (oder auch {\it zyklometrische
% Funktionen}).\ed  
% 
% Die Arkusfunktionen haben im Moment f\"{u}r uns nur die Bedeutung, dass das L\"{o}sen einer trig.
% Gleichung der Art $\sin x=c$ algebraisch machbar wird. Sie werden uns dann auch bei der
% Integralrechnung (5. Klasse) begegnen.
% 
% \paragraph{Wichtigste Eigenschaften:}
%  Fertige eine Tabelle mit den 5 Spalten: Arkusfunktion, Bezeichnung auf dem Taschenrechner,
% Definitionsbereich, Wertebereich, Schaubild an und f\"{u}lle sie richtig aus.
% 
% \ba \bn \item B3F2: Zeichnen Sie das Schaubild der Funktion $f(x)=\sin \arcsin x$ . 
% 
% \item B10F5: Bestimme den Definitionsbereich  $D$  der Funktion
% $$f(x)=\arcsin(3x-1),$$
% (zeige, dass sie in  $D$  monoton steigend ist und) ermittle ihre Umkehrfunktion. 
% \en
% \ea




\section{Dreiecksberechnung}
\subsection{Berechnungen im  rechtwinkligen Dreieck}
\bd  Nehmen wir den Winkel $\alpha$ als Ausgangspunkt, so gibt es eine Kathete, die einen Schenkel von
$\alpha$ bildet, und eine Kathete, die mit $\alpha$ direkt nichts zu tun hat. Die erste Kathete
hei{\ss}t {\bf Ankathete} von $\alpha$, die andere {\bf Gegenkathete} von $\alpha$. \ed  

Die Bezeichnungen Ankathete und Gegenkathete beziehen sich immer auf einen Winkel. Nimmt man einen
anderen Winkel, so muss man die Bezeichnungen entsprechend \"{a}ndern. (Wie?)

\bs  Gegeben sind 2 rechtwinklige Dreiecke $\Delta A_1B_1C_1$ und $\Delta A_2B_2C_2$, die in einem
weiteren Winkel (z.B. in $\alpha_1=\alpha_2$) gleich gro� sind. So gilt:

\[\frac{\mbox{Ankathete von} \;\alpha_1}{\mbox{Hypotenuse von }\; \Delta_1}=
  \frac{\mbox{Ankathete von} \;\alpha_2}{\mbox{Hypotenuse von }\; \Delta_2}\]
\es  

\begin{proof}
Dieser Sachverhalt ist mit der \"{A}hnlichkeit von Dreiecken einsichtig:
\end{proof}


\bs   Gegeben ist ein rechtwinkliges Dreieck mit dem rechten Winkel $\gamma$ und weiters den Winkeln
$\alpha$  und $\beta$. F�r die Verh\"{a}ltnisse von zwei Seiten gilt:
\[  \sin(\alpha)=\frac{\mbox{Gegenkathete von }\;\alpha}{\mbox{Hypotenuse}}  \qquad
\cos(\alpha)=\frac{\mbox{Ankathete von }\;\alpha}{\mbox{Hypotenuse}}
\]
\[  \tan(\alpha)=\frac{\mbox{Gegenkathete von }\;\alpha}{\mbox{Ankathete von }\;\alpha}  \qquad
\cot(\alpha)=\frac{\mbox{Ankathete von }\;\alpha}{\mbox{Gegenkathete von }\;\alpha}
\]
Gleiches gilt f\"{u}r den Winkel $\beta$. \es   

\begin{proof}
Mit Hilfe der �hnlichkeit wird obige Aussage einsichtig. ...
\end{proof}

\bme Eine "`Eselsbr�cke"' zum obigen Sachverhalt k�nnte die GAGA-{\bf H}�hner{\bf h}of{\bf
a}ktien{\bf g}esellschaft sein.

...\vspace{1.5cm}
 \eme

 \ba "`Google"' dich mal zur GAGA-H�hnerhofaktiengesellschaft.\ea

Sind {\it zwei St\"{u}cke} - das hei{\ss}t Seiten oder Winkel - eines rechtwinkligen Dreiecks
bekannt, so lassen sich in der Regel die restlichen unbekannten St\"{u}cke ausrechnen. Es gibt
lediglich eine Ausnahme, n\"{a}mlich der Fall, dass alle 3 Winkel gegeben sind. Denn das reicht
nicht aus, um auf die 3 fehlenden St\"{u}cke, also auf die 3 Seiten zu schlie{\ss}en (es gibt
unendlich viele Dreiecke, die diese Bedingung erf\"{u}llen). Es verbleiben aber insgesamt 5
unterschiedliche l�sbare Grundaufgaben:
\begin{enumerate} \item Die Hypotenuse und eine Kathete sind bekannt.
\item Die beiden Katheten sind bekannt.
\item Ein Winkel und die Hypotenuse sind bekannt.
\item Ein Winkel und die Ankathete   sind bekannt.
\item Ein Winkel und die Gegenkathete   sind bekannt.
\end{enumerate}
.



Neben den neuen trigonometrischen Begriffen sind die folgenden 2 S\"{a}tze aus der Geometrie
n\"{o}tig und werden deshalb an dieser Stelle wiederholt:

\bs  {\bf Pythagor\"{a}ische Lehrsatz}

In jedem {\it rechtwinkligen} Dreieck ist die Summe der Quadrate der Katheten gleich dem Quadrat
der Hypotenuse; d.h. mit den Bezeichnungen von oben: $$a^2+b^2=c^2$$
\es

\bs {\bf Summenwinkelsatz}

In jedem {\it beliebigen} Dreieck ergibt die Summe der 3 Winkel $180$, d.h.
\[\alpha+\beta+\gamma=180\]
\es 

\bb  
\bn \item Finde zu jedem der 5 F�lle ein konkretes Beispiel, \item berechne die fehlenden Seiten
und Winkel \item und gib die Formeln allgemein an.\en
\eb  

Damit haben wir alle f�nf Grundaufgaben gel\"{o}st.

\"{A}hnliche Zusammenh\"{a}nge gelten auch in beliebigen Dreiecken. Das erf\"{a}hrst du im
n\"{a}chsten Abschnitt. Die folgenden Textaufgaben f\"{u}hren immer auf eine Grundaufgabe. Bevor du
ans Ausrechnen gehst, solltest du dir eine Zeichnung machen, in der die gegebenen St\"{u}cke
farblich hervorgehoben sind. Dann kannst du leicht erkennen, um welchen Aufgabentyp es sich
handelt.



\ba 
\begin{enumerate}
        \item Gegeben seien das rechtwinklige Dreieck $\Delta ABC$, wobei
        $\gamma$ der rechte Winkel sein soll, sowie die folgenden St\"{u}cke. Die restlichen St\"{u}cke
        sind gesucht!
                \begin{enumerate}
                \item $c=24; \alpha=13$
                \item $a=5; \alpha=66$
                \item $a=7,5; b= 6,8$
                \item $c=9,5; \beta= 50$
                \end{enumerate}


        \item Eine Leiter der L\"{a}nge $3,5$m lehnt an einer Mauer. Der Fuss
        der Leiter ist $80$cm von der Mauer entfernt. In welcher H\"{o}he ber\"{u}hrt die Leiter die Mauer?
        Welchen Neigungswinkel hat die Leiter gegen den Boden, welchen gegen die Wand?

        \item {\bf Sehnenl�nge} Man l�se die Aufgabe \ref{12455ddkj}!




        \item Herr Spielmitmir l\"{a}sst seinen Drachen steigen. Schlie{\ss}lich
        ist die $50$m lange Drachenschnur ganz abgewickelt. Schnur und Boden bilden einen Winkel von $60�$
        miteinander. Wie hoch steht der Drachen?


        \item Den Steigungshinweisen auf Verkehrsschildern liegt die
        Angabe zugrunde, wie viel Meter die Stra{\ss}e auf hundert Meter Entfernung ansteigt.
        Beispielsweise spricht man von einer zwanzigprozentigen Steigung wenn die Stra{\ss}e in $100$m um
        $20$m steigen w\"{u}rde.
                \begin{enumerate}
                \item Berechne den Steigungswinkel $\alpha$.
                \item Welche Steigungswinkel entsprechen den Steigungen $8\%$,
                $12\%$, $17\%$?
                \item Welche Steigungen (anzugeben in Prozent) geh\"{o}ren zu den
        Steigungswinkeln $10$, $25$, $45$, $50$?
\en



        



        


        \item Die Seiten eines Rechtecks sind $6$cm und $4$cm lang.
        Berechne den Winkel
                \begin{enumerate}
                \item zwischen einer Diagonalen und einer Rechteckseite;
                \item zwischen den Beiden Diagonalen.
                \end{enumerate}


    \item Die Spitze eines $25$m hohen, direkt am Flussufer stehenden
    Hochspannungsmastes erscheint vom anderen Ufer aus gesehen unter einem Winkel von $20$ (bezogen
    auf die Wasserfl\"{a}che). Wie breit ist der Fluss?

        \item Begr�nde mit Hilfe eines rechtwinkligen Dreiecks und merke:
        \begin{center}
        \begin{equation}
        \begin{tabular}{|r@{\,= \,}l|c|c|c|} \hline
        \multicolumn{2}{|c|}{Winkel} & \multicolumn{3}{c|}{Funktion} \\ \cline{3-5}
        \multicolumn{2}{|c|}{$\alpha$} & \multicolumn{1}{c|}{$sin \left(\alpha \right)$}
        & \multicolumn{1}{c|}{$cos \left(\alpha \right)$}
        & \multicolumn{1}{c|}{$tan \left(\alpha \right)$}\\ \hline
        0&0 rad & 0 & 1 & 0\\\hline 30& & & & \\\hline 45&& & & \\\hline 60& & & & \\\hline 90& & & &
        \\\hline
        \end{tabular}
        \end{equation}
        \end{center}


        \item
        (Matura 2005: Frage 1) Beweise, dass die Seite des regelm��igen Zehnecks, das in einen Kreis eingeschrieben wird, den Radius im Goldenen Schnitt teilt! Nutze das Ergebnis um $\sin{18^\circ}$ und  $\sin{36^\circ}$ zu berechnen!\\[1ex]
        \textit{Anleitung:} Ein Punkt T teilt die Strecke $\overline{AB}$ im \textit{Goldenen Schnitt (``stetige Teilung'')}, wenn sich die L�nge der ganzen Strecke zu der des gr��eren Teilabschnittes so verh�lt wie die L�nge des gr��eren zu der des kleineren Teilabschnittes.\\[1ex]
        Verbinden wir die Ecken eines regelm��igen Zehnecks mit dem Mittelpunkt seines Umkreises, dann erhalten wir 10 gleichschenklige Dreiecke, die sogenannten \textit{Bestimmungsdreiecke des Zehnecks}.\\[1ex]
        \begin{minipage}{9cm}
            \begin{enumerate}
            \item
            Was ist zu zeigen?
            \item
            Warum ist der Winkel an der Spitze $36^\circ$?
            \item
            Wir halbieren den Winkel $\angle \left(MAB \right)$ und erhalten den Schnittpunkt C. Welche Winkel
            erhalten wir im $\Delta \; ABC$? Zeichne sie ein!
            \item
            Wie lang ist $\overline{CM}$?
            \item
            Welche Dreiecke sind �hnlich? Was folgt daraus?
            \item
            Zeige: $r:\overline{MC}=\overline{MC}:\overline{CB}$
            \item
            Wie hei�t dieses Verh�ltnis?
            \item
            Berechne $\dfrac{\overline{AB}}{r}$!
            \item
            Berechne $\sin{18^\circ}$ und  $\sin{36^\circ}$!
            \end{enumerate}
        \end{minipage}
        \begin{minipage}[c]{4cm}
        \includegraphics[width=4cm]{2te/trigonometrie1/bilder/klaus1.jpg}
        \end{minipage}\\[0.1em]


        \item Erg�nzung: Zehneckskonstruktion\\[0.7em]


        \begin{minipage}{8cm}
        Die Punkte A, B, C, D und E sind in alphabetischer Reihenfolge konstruiert.
        \begin{enumerate}
        \item
        Schreibe eine Konstruktionsbeschreibung!
        \item
        Rechne nach, dass $\overline{AE}$ wirklich gleich $\dfrac{\sqrt{5}-1}{2}r$\ ist!
        \item
        Wie kannst du damit ein regelm��iges F�nfeck konstruieren?
        \item
        Welche regelm��igen n-Ecke lassen sich mit Zirkel und Lineal konstruieren?
        \end{enumerate}
        \end{minipage}
        \begin{minipage}[c]{4cm}
        \includegraphics[width=4cm]{2te/trigonometrie1/bilder/klaus2.jpg}
        \end{minipage}



\end{enumerate}
\ea 

\subsection{Berechnungen im  allgemeinen Dreieck}


Mit den neuen Begriffen Sinus, Kosinus und Tangens l\"{a}sst sich, wie wir gesehen haben, allerhand
anfangen. Allerdings sind diese hilfreichen Gr\"{o}ssen bis jetzt nur f\"{u}r {\it rechtwinklige}
Dreiecke definiert. Diese Einschr\"{a}nkung ist aber nicht einschneidend, denn man kann ja jedes
Dreieck durch das Einzeichnen einer Dreiecksh\"{o}he in zwei rechtwinklige Dreiecke zerlegen. So
werden wir uns S\"{a}tze herleiten, die in beliebigen Dreiecken gelten, die die Dreiecksberechnung
dann erleichtern oder \"{u}berhaupt erst erm\"{o}glichen. Diese S\"{a}tze sind der {\it Sinus-} und
{\it Kosinussatz}:

\bs {\bf Sinussatz}

In einem beliebigen Dreieck $\Delta ABC$ verhalten sich zwei Seiten wie die Sinuse der
gegen\"{u}berliegenden Winkel, d.h.
\[\frac{a}{b}=\frac{\sin(\alpha)}{\sin(\beta)}; \qquad
\frac{b}{c}=\frac{\sin(\beta)}{\sin(\gamma)}; \qquad
\frac{a}{c}=\frac{\sin(\alpha)}{\sin(\gamma)}\]
\es  
\begin{proof}
Beweisidee: Das gegebene Dreieck wird durch das Einzeichnen der H\"{o}he in 2 rechtwinklige
Dreiecke zerlegt, dort wird jeweils f\"{u}r die entsprechenden Winkel die Sinusdefinition verwendet
und die sich daraus ergebenen Gleichungen werden gleichgesetzt. Ausf\"{u}hrung: siehe Heft.
\end{proof}
\bme \bn \item {\it Wenn das Dreieck stumpfwinklig ist ...}

In diesen \"{U}berlegungen sind wir von einem Dreieck ausgegangen, in welchem alle Winkel kleiner
als $90$ sind. Dies muss aber nicht sein. Handelt es sich bei $\beta$ beispielsweise um einen
stumpfen Winkel, so muss der obige Satz nochmal nachgewiesen werden.


\item {\it Achtung beim Sinus ...}
Rechnest du mit dem Sinussatz einen Winkel aus, so musst du folgendes ber\"{u}cksichtigen: Wenn der
Winkel $\alpha$ zwischen $0$ und $90$ liegt, so haben die Winkel $\alpha$ und $180-\alpha$
denselben Sinuswert. Also gibt es umgekehrt zwei Winkel, wenn der Sinuswert gegeben ist.
Prinzipiell m\"{u}ssen demnach alle zwei L\"{o}sungen einer Gleichung der Art (geg: $c\in
\mathbb{R}$, ges: $\alpha$):
\[\sin(\alpha)=c\]
ber\"{u}cksichtigt werden. Aus diesen Sachverhalt macht der Taschenrechner {\it nicht} aufmerksam!

Deshalb muss je nach Aufgabenstellung evtl. auch die L\"{o}sung $180-\alpha$ in Betracht gezogen
werden.
\en 
\eme 

\bs  {\bf Kosinussatz}

Weiters gelten in einem beliebigen Dreieck $\Delta ABC$ die folgenden Beziehungen:
\begin{eqnarray*}
  a^2 & = & b^2+c^2-2bc\cdot \cos(\alpha) \\
  b^2 & = & a^2+c^2-2ac\cdot \cos(\beta) \\
  c^2 & = & a^2+b^2-2ab\cdot \cos(\gamma)
\end{eqnarray*}
\es  

\begin{proof}
Vervollst�ndige den Beweis, erg�nze richtig, orientiere dich an der Skizze.

Im linken Teildreieck soll der Satz des Pythagoras angewandt werden, um einen Rechenausdruck f�r $c^2$ zu finden. Dazu ben�tigt man die Quadrate der Kathetenl�ngen dieses Teildreiecks:
$$h^2=....$$ 
(Satz des Pythagoras f�r das rechte Teildreieck)
$$d^2=(a-....$$ (binomische Formel)\\

\begin{minipage}{10cm}

Nach Pythagoras gilt f�r das linke Teildreieck: $$c^2=$$
Es m�ssen also die beiden oben gefundenen Rechenausdr�cke addiert werden:
$$c^2=b^2 ...$$
Nun gilt aber $$\cos  \gamma=$$
mit der Folgerung     $e=...$
Einsetzen dieses Zwischenergebnisses in die Gleichung f�r $c^2$ ergibt die Behauptung: $$c^2=$$
\end{minipage}
\begin{minipage}{4.5cm}
\includegraphics[width=4.5cm]{2te/trigonometrie1/bilder/kosinussatz.jpg}
\end{minipage}
\end{proof}
\begin{Bem}
Der Kosinussatz weist eine gewisse \"{A}hnlichkeit mit dem Satz des Pythagoras auf; es existiert
auch tats\"{a}chlich der folgende Zusammenhang: in einem rechtwinkligen Dreieck f\"{u}hrt der
Kosinussatz genau auf dem Satz des Pythagoras.
\end{Bem}
Mit Hilfe des Sinus- und des Kosinussatzes lassen sich aus drei bekannten St\"{u}cken alle anderen
Seiten bzw. Winkel eines beliebigen Dreiecks berechnen. Dies gilt wieder f\"{u}r die folgenden
F\"{a}lle, jedoch der Fall, dass 3 Winkel gegeben sind, kann nicht (bzw. nicht eindeutig)
gel\"{o}st werden:
\begin{enumerate}
\item SSS-Fall: drei Seiten sind bekannt
\item SWS-Fall: zwei Seiten und der von ihnen eingeschlossene
Winkel ist bekannt
\item WSW-Fall: zwei Winkel und die dazwischenliegende Seite ist
gegeben
\item SsW-Fall: Zwei Seiten und ein Winkel sind bekannt, wobei der
Winkel der gr��eren Seite gegen\"{u}berliegt
\item Sonderfall: zwei Seiten und der Winkel, der der kleineren
Seite gegen\"{u}berliegt, hier kann das Dreieck nicht eindeutig gel\"{o}st werden.
\end{enumerate}
Die obigen vier F\"{a}lle entsprechen im \"{u}brigen genau den vier Kongruenzs\"{a}tzen der
Dreiecksgeometrie (siehe Mathematik 1. Klasse).


\bb  
Der obige 5. Fall ist nicht eindeutig l�sbar. Dieses Beispiel  f\"{u}hrt auf zwei Dreiecke, es ist also
nicht eindeutig l\"{o}sbar: Gegeben:

$a=6,9; c= 5; \gamma=40$ Gesucht: $b, \alpha, \beta$
\eb  
\ba 
\begin{enumerate}
\item Entscheide, in welchen F\"{a}llen eine eindeutige Berechnung des
Dreiecks m\"{o}glich ist. F\"{u}hre die m\"{o}glichen Berechnungen durch.
\begin{enumerate}
\item $a=13,8; b=25,7; \beta=130�$
\item $\alpha=35�; \beta=105,5�; \gamma=39,5�$
\item $a=7,4; b=3,6; c=9$
\item $a=3; b=7; \alpha=115�$
\item $a=4,7; c=5,2; \alpha=46�$
\item $a= 12,2; b=4,5; c= 16,7$
\end{enumerate}
\item Berechne die H\"{o}hen der folgenden Dreiecke:

a) $a=6,5; b=5; \alpha=50�$, b) $\alpha=35�; \beta=75�; c=9$


\hfill ({\it $h_a\approx 4,98; h_b\approx 6,485; h_c\approx 3,83$})
\item  \bn \item Die Ortschaften  Meran und Brixen sollen durch ein Tunnel
verbunden werden. Wie lang wird dieser Tunnel sein, wenn Meran-Bozen $30$km und Bozen-Brixen $45$km
voneinander entfernt sind und weiters die Orte Meran und Brixen von Bozen aus unter einem Winkel
von $35$ erscheinen?
\item (F�r Profis:) Ber�cksichtige die unterschiedlichen Meeresh�hen der St�dte und f�hre die Berechnungen im Raum durch.
\en 

\hfill ({\it $\approx 26,7km$})

\item Wie \"{a}ndern sich die Berechnungen, wenn nicht der Winkel in
Bozen, sondern der in Meran gemessen wird: $75,11�$!
\item Die Punkte $A$ und $C$, deren Entfernung $b$ ermittelt
werden soll, sind durch einen Fluss voneinander getrennt. Man misst deshalb, ausgehend von $A$,
eine Strecke $c$ bis zu einem Punkt $B$ ab und bestimmt die beiden Winkel $\alpha$ und $\beta$.
Dann l\"{a}sst sich $b$ berechnen. Diese Methode wird in der Vermessungskunde {\bf
Vorw\"{a}rtseinschneiden} genannt.

Welche Werte f\"{u}r die gesuchte Entfernung findet man, wenn man folgende Messresultate
erh\"{a}lt:
\begin{enumerate}
\item $c=50$m, $\alpha=50$, $\beta=40$;

\hfill ({\it L�sung: $b \approx 32,14m$})

\item $c=100$m, $\alpha=48,5$, $\beta=62$;
\end{enumerate}
\item Schon wieder ein Tunnel. Diesmal sollen die gleichhoch
liegenden Ortschaften Saltaus i.P. und Sarnthein verbunden werden. Genau \"{u}ber der gesuchten
Verbindungsstrecke befindet sich die Hirzerseilbahn, die (Annahme) bis zum Gipfel hinauff\"{a}hrt,
eine L\"{a}nge von $950$m hat und unter einem Winkel von $40$ steigt. Von Sarnthein aus sieht man
die Hirzerspitze und einem Winkel von $55$.
\begin{enumerate}
\item Berechne den H\"{o}henunterschied zwischen Saltaus und der
Hirzerspitze.
\item Wie lang ist der Tunnel?
\end{enumerate}

\hfill ({\it L�sung: $6a.) \quad h \approx 610,6m \qquad 6.) \quad x \approx 1155,32m$; })



\item In einem beliebigen Viereck $ABCD$ ist die Diagonale
$\overline{AC}=500m$ bekannt. Gesucht ist die L\"{a}nge der zweiten Diagonale $\overline{BD}$. Dazu
werden die Winkel $\angle CAD=65,5$, $\angle BAC=39,5$, $\angle DCA = 72,1$ und $\angle ACB=
70$ bestimmt. Wie gro� ist $\overline{BD}$?


\hfill ({\it L�sung: $ \overline{BD}\approx 963,523$})

\item In dem Viereck
 $ABCD$ sind die Seiten $c=6,5$ und $d=8,2$
gegeben sowie die Winkel $\angle DBA=40$, $\angle ADB$ und $\angle BDC = 21$. Zu berechnen sind
die fehlenden Seiten und Winkel sowie die Diagonale $\overline{BD}$!

\hfill ({\it L�sung: $\overline{BD}\approx 11,1574; b \approx 5,6$})


\item Gegeben ist ein Quader (z.B. ein Kleiderschrank) mit den
Abmessungen $a=2$m (Breite), $b=1,4$m (H\"{o}he) und $c=1$m (Tiefe). Zu berechnen sind die
Fl\"{a}chendiagonalen $d, e$ und $f$ sowie die Winkel zwischen diesen.


\hfill ({\it L�sung: $\alpha \approx 86,16; \beta \approx 39,84; \gamma \approx 54; d \approx 1,7; e \approx 2,4; f \approx 2,2$})

\item ({\bf Matura2003F3}) Vom Punkt $A$ aus, den man erreichen kann, ist der Punkt $B$ sichtbar, den man aber auf keine Weise erreichen
kann, so dass ein direktes Messen der Entfernung $AB$ unm\"{o}glich ist. Vom Punkt $A$ aus kann man einen Punkt
$P$ erreichen, von dem aus, au{\ss}er $A$ auch $B$ sichtbar ist. Die Entfernung $PB$ bleibt weiterhin nicht
direkt messbar, die Entfernung $AP$ kann aber gemessen werden. Angenommen Sie verf\"{u}gen \"{u}ber die notwendigen
Messinstrumente und Sie wissen dass der Punkt $P$ nicht auf einer Geraden mit $A$ und $B$ liegt, erkl\"{a}ren
Sie, wie der Sinussatz benutzt werden kann, um die Entfernung $AB$ zu berechnen.

\item ({\bf Matura2007F2}) Die Seitenl�ngen eines Dreiecks sind $40$, $60$ und $80$cm. Berechnen Sie mit Hilfe eines Taschenrechners die im Gradma� auf Grad und Minuten gerundeten Gr��en der Winkel des Dreiecks.

\end{enumerate}
\ea 


%\newpage


%\input{additionstheoreme.tex}

%\input{goniometrie.tex}
\end{document}

