\section{Auflösung von Gleichungen durch Radikale}

\begin{Def}
Sei $K$ ein Körper.
\begin{enum}

\item Eine einfache Körpererweiterung $L=K(\alpha)$ heißt
\emp{elementare (oder einfache) Radikalerweiterung}, wenn entweder

\begin{enumerate}
\item[(i)] $\alpha$ ist eine Einheitswurzel.
\item[(ii)] $\alpha$ ist Nullstelle von $X^n - \gamma$ für ein
$\gamma \in K$ und char$(K) \nmid n$
\item[(iii)] $\alpha$ ist Nullstelle von $X^p - X - \gamma$ für
$\gamma \in K$, char$(K) = p$
\end{enumerate}

\item Eine endliche Körpererweiterung $L/K$ heißt
\emp{Radikalerweiterung}, wenn es eine Körpererweiterung $L'/L$ gibt
und eine Kette $K=L_0 \subset L_1 \subset \dots \subset L_n = L'$
von Zwischenkörpern, so daß $L_{i+1}/L_i$ elementare
Radikalerweiterung ist für $i=0,\dots,n-1$

\item Ist $f \in K[X]$ separabel, nicht konstant, so heißt die
Gleichung $f(X) = 0$ \emp{durch Radikale auflösbar}, wenn der
Zerfällungskörper von $f$ Radikalerweiterung ist.
\end{enum}

\bsp{$K=\mathbb{Q}$, $f(X) = X^3 - 3X + 1$

\textbf{Beh.}: Ist $\alpha$ Nullstelle von $f$, so ist
$\mathbb{Q}(\alpha)$ Zerfällungskörper von $f$, hat also Grad $3$
über $\mathbb{Q}$. $\mathbb{Q}(\alpha)/\mathbb{Q}$ ist
\textbf{keine} einfache Radikalerweiterung.

Die Nullstellen von $f$ sind: \[\ds\begin{array}{l}\alpha_1 =
e^{2\pi i/9} + e^{16\pi i / 9} \\ \alpha_2 = e^{8\pi i/9} + e^{10
\pi i/9}
\\ \alpha_3 = e^{14 \pi i / 9} + e^{4\pi i /9} \end{array}\]

Es ist $\alpha_1^2 = e^{4\pi i /9} + e^{14 \pi i/9} + 2 = \alpha_3+2
\Ra \alpha_3 \in \mathbb{Q}(\alpha_1) \Ra \alpha_2 = -\alpha_1 -
\alpha_3 \in \mathbb{Q}(\alpha_1)$}
\end{Def}

\begin{Satz}
Sei $K$ ein Körper, $f \in K[X]$ separabel, nicht konstant.
\begin{enum}

\item Die Gleichung $f(X) = 0$ ist genau dann durch Radikale
auflösbar, wenn ihre Galoisgruppe auflösbar ist (dh. $G$ hat
Normalreihe $G=G_0 \vartriangleright \dots \vartriangleright G_n =
\{e\}$ mit $G_i/G_{i+1}$ abelsch).

\item Eine endliche Körpererweiterung $L/K$ ist genau dann
Radikalerweiterung, wenn es eine endliche Galoiserweiterung $L'/K$
gibt mit $L\subseteq L'$, so daß Gal$(L'/K)$ auflösbare Gruppe ist.
\end{enum}

\bsp{ $X^5 - 4X + 2$ hat Galoisgruppe $S_5$ und ist deshalb nicht
durch Radikale auflösbar, denn $S_5 \supset A_5 \supset \{e\}$ ist
Kompositionsreihe. Nach Jordan-Hölder tritt $A_5$ in jeder
Kompositionsreihe für $S_5$ als Faktorgruppe auf.}

\sbew{1.0}{
\newline ''$\Ra$'': Sei $K=L_0 \subset L_1 \subset \dots
\subset L_m$ Kette wie in Def. mit $L \subseteq L_m$.

\textbf{Induktion über $\mathbf{m}$:}
\begin{description}
\item[m=1:] Ist $L_1/K$ vom Typ (i), so ist $L_1 = K(\zeta)$ für
eine primitive $n$-te Einheitswurzel $\zeta$ und Gal$(K(\zeta)/K)
\subseteq (\mathbb{Z}/n\mathbb{Z})^x$, also auflösbar.

Ist $L_1/K$ vom Typ (ii), so ist $L_1/K$ galoissch und Gal$(L_1/K) =
\mathbb{Z}/p\mathbb{Z}$ \\Sei $L_1/K$ vom Typ (iii). Enthält $K$
eine primitive $n$-te Einheitswurzel, so ist $K(\alpha)/K$ galoissch
und Gal$(K(\alpha)/K) \cong \mathbb{Z}/n\mathbb{Z}$

Andernfalls sei $F=K(\zeta)$ der Zerfällungskörper von $X^n-1$ über
$K$ und $L_1' = L_1(\zeta) = F(\alpha) = F \cd L_1$ das
''\emp{Kompositum}'' von $F$ und $L_1$.

$L_1'$ ist galoissch über $K$ (Zerfällungskörper von $X^n - \gamma$
über $K$) und es gibt exakte Sequenz
\[ 1 \ra \underset{\mbox{\scriptsize zyklisch}}{\underbrace{\mbox{Gal}(L_1'/F)}}
\ra \mbox{Gal}(L_1'/K) \ra \underset{\mbox{\scriptsize
abelsch}}{\underbrace{\mbox{Gal}(F/K)}} \ra 1 \]

$\Ra$ Gal$(L_1'/K)$ auflösbar.

\item[m$>$1:] Eine endliche Körpererweiterung heißt \emp{auflösbar},
wenn es eine endliche Erweiterung $L'/L$ gibt, so daß $L'/K$
galoissch und Gal$(L'/K)$ auflösbar ist.

Nach Induktionsvoraussetzung ist $L_{m-1}/K$ auflösbar. Außerdem ist
$L_m/L_{m-1}$ auflösbar. (m=1)

zu zeigen also: Sind $K \subset \underset{=L_{m-1}}{\underbrace{L}}
\subset \underset{=L_m}{\underbrace{M}}$ Körpererweiterungen und ist
$L/K$ auflösbar und $M/L$ auflösbar, so ist $M/K$ auflösbar.

Seien dazu $L'/L$ und $M'/M$ Erweiterungen wie in Def.:

%\[\begindc{\undigraph} \obj(1,1){$K$}[\south]
%                      \obj(2,1){$L$}[\south]
%                      \obj(3,1){$M$}[\south]
%                      \obj(4,1){$M'$}[\east]
%                      \obj(2,2){$L'$}[\north]
%                      \obj(4,2){$L'M'$}[\north]
%                      \mor{$K$}{$L$}{}
%                      \mor{$L$}{$M$}{}
%                      \mor{$M$}{$M'$}{}
%                      \mor{$L$}{$L'$}{}
%                      \mor{$M'$}{$L'M'$}{}
%                      \mor{$L'$}{$L'M'$}{}
%                      \mor{$K$}{$L'$}{gal.}
%                      \cmor((2,1)(3,0)(4,1))
%                            \pup(4,0){gal.}
%\enddc\]

\textbf{Beh.}: $L'M'/L'$ ist galoissch und Gal$(L'M'/L)$ ist
auflösbar.

\textbf{denn}: Nach Voraussetzung ist $M'/L$ galoissch, also
Zerfällungskörper eines Polynoms $f \in L[X] \Ra M'L'$ ist
Zerfällungskörper von $f \in  L'[X]$ über $L'$.

Außerdem: Gal$(L'M'/L') \ra$ Gal$(M'/L)$, $\sigma \mapsto
\sigma_{|M'} \overset{(!)}{\in}$ Gal$(M'/L)$ ist wohldefiniert
und injektiv: Ist $\sigma_{|M'} = id_{M'}$, so ist $\sigma =
id_{L'M}$, da $\sigma_{|L'} = id_{L'}$ nach Voraussetzung.

Also \OE $L=L'$, $L'M' = M$.
\end{description}}

\sbew{1.0}{ \begin{description}
\item[m$>$1 (Forts.)] Ist $M/K$ galoissch, so ist Gal$(M/K)$
auflösbar, da dann \[ 1 \ra \underset{\mbox{\scriptsize
auflösbar}}{\underbrace{\mbox{Gal}(M/K)}} \ra \mbox{Gal}(M/K) \ra
\underset{\mbox{\scriptsize
auflösbar}}{\underbrace{\mbox{Gal}(L/K)}} \ra 1\] exakt ist.

Andernfalls sei $\wt{M}/M$ (minimale) Erweiterung, so daß $\wt{M}/K$
galoissch ist. $\wt{M}$ wird (über $K$) erzeugt von den $\sigma(M)$,
$\sigma \in$ Hom$_K(M,\bar K)$. ($\bar K$ fest gewählter
algebraischer Abschluß von $K$) Für jedes $\sigma \in$ Hom$_K(M,\bar
K)$ ist $\sigma(M)$ Galoiserweiterung von $\sigma(L) = L$.

Dann ist \[\begin{array}{ccc} \mbox{Gal}(\wt{M}/L) &\ra
&\prod_{\sigma \in \mbox{
\scriptsize Hom}_K(M,\bar K)}\mbox{ Gal}(\sigma(M)/L) \\
\tau &\mapsto &(\tau_{|\sigma(M)})_\sigma \end{array}\] injektiver
Gruppenhomomorphismus.

Für jedes $\sigma \in$ Hom$_K(M,\bar K)$ ist Gal$(\sigma(M)/L)
\cong$ Gal$(M/L)$, also auflösbar $\Ra \prod_\sigma$
Gal$(\sigma(M)/L)$ ist auflösbar. (!) $\Ra$ Gal$(\wt{M}/L)$
auflösbar (als Untergruppe einer auflösbaren Gruppe) $\Ra$
Gal$(\wt{M}/K)$ ist auflösbar wegen $1 \ra$ Gal$(\wt{M}/L) \ra$
Gal$(\wt{M}/K) \ra$ Gal$(L/K) \ra 1$ exakt.
\end{description}

''$\Leftarrow$'':
\newline $G \defeqr$ Gal$(L'/K)$ sei auflösbar, $G
= G_0 \supset G_1 \supset \dots \supset G_m = \{1\}$ Normalreihe, so
daß $G_{i+1}$ Normalteiler in $G_i$ und $G_i/G_{i+1} \cong
\mathbb{Z}/p\mathbb{Z}$ mit Primzahlen $p_i,\;i=0,\dots,m-1$ ist.
\newline \newline Dazu gehört eine Kette von Zwischenkörpern $K = K_0 \subset K_1
\subset \dots K_m = L'$, in der $K_i/K_{i-1}$ Galoiserweiterung ist
und Gal$(K_i/K_{i-1}) \cong \mathbb{Z}/p_i\mathbb{Z}$.
\newline \newline Ist $p_i =$ char$(K)$, so ist $K_i/K_{i-1}$
Artin-Schreier-Erweiterung. (dh. Typ (iii)) (Folgerung zu Satz 19)
\newline Ist $p_i \neq$ char$(K)$, so ist $K_i/K_{i-1}$ vom Typ (ii), \textbf{falls}
$K_{i-1}$ eine primitive $n$-te Einheitswurzel $\zeta$ enthält.
\newline \newline Sei also $d \defeqr \prod_{\substack{p \mbox{ \scriptsize
prim} \\ p |\;|G|}}$ und $F$ der Zerfällungskörper von $X^d - 1$ über $K$.
$\\\Ra F/K$ ist Erweiterungskörper vom Typ (i).
\newline \newline Sei $\wt{L} = F L' \Ra \wt{L}/F$ ist Galoiserweiterung (siehe
Diagramm oben) und Gal$(\wt{L}/F) \subset$ Gal$(L'/K)$, also
auflösbar. Beginne von vorne mit $\wt{L}$ und $F$ statt $L'$ und
$K$. Erhalte Kette $K \subset F \subset F_1 \subset \dots \subset
F_r = \wt{L}$ von Zwischenkörpern, $F_i/F_{i-1}$ Galoiserweiterung,
Gal$(F_i/F_{i-1}) \cong \mathbb{Z}/p_i\mathbb{Z}$ elementare
Radikalerweiterung.}
\end{Satz}