\documentclass{article}
\newcounter{chapter}
\usepackage{ana}

\author{Joachim Breitner}
\title{Konvergenzkriterien}
\setcounter{chapter}{12}


\begin{document}
\maketitle

\begin{satz}[Leibnizkriterium]
Sei $(b_n)$ eine monoton fallende Nullfolge und $a_n := (-1)^{n+1}b_n$. Dann ist $\reihe{a_n}$ konvergent.
\end{satz}
\begin{beweis}
Wie bei $\reihe{(-1)^{n+1}\frac{1}{n}}$. Von $(b_n)=(\frac{1}{n})$ wurde nur benutzt: $\frac{1}{n}$ ist eine fallende Nullfolge.
\end{beweis}
\begin{bemerkung}
Gilt $a_n=b_n \ffa n\in\MdN$, so gilt: $\reihe{a_n}$ ist genau dann konvergent, wenn $\reihe{b_n}$ konvergent ist.
\end{bemerkung}

\begin{satz}[Majoranten- und Minorantenkriterium]
\begin{liste}
\item \begriff{Majorantenkriterium}: Gilt $|a_n|\le b_n \ffa n\in\MdN$ und  ist $\reihe{b_n}$ konvergent so gilt: $\reihe{a_n}$ ist absolut konvergent.
\item \begriff{Minorantenkriterium}: Gilt $|a_n|\ge b_n \ge 0 \ffa n\in\MdN$ und  ist $\reihe{b_n}$ divergent so gilt: $\reihe{a_n}$ ist absolut divergent.
\end{liste}
\end{satz}

\begin{beweise}
\item $s_n := b_1 +b_2+ \ldots + b_n$, $\sigma_n := |a_1|+\ldots+|a_n| \ \forall n\in\MdN$. O.b.d.A.: $|a_n| \le b_n \ \forall n\in\MdN$. $(s_n)$ ist konvergent $\folgtnach{6.1} (s_n)$ ist beschränkt $\folgt \exists c\ge0: a_n \le c \ \forall n\in\MdN \folgt 0 \le \sigma_n = |a_1| + |a_2| + \ldots + |a_n| \le b_1 + b_2 + \ldots b_n = s_n \le c \ \forall n\in\MdN \folgt (\sigma_n)$ ist beschränkt $\folgtnach{11.1(1)} (\sigma_n)$ konvergent.
\item Annahme: $\reihe{a_n}$ ist konvergent $\folgtnach{(1)}$ $\reihe{b_n}$ ist konvergent. Widerspruch!
\end{beweise}

\begin{beispiele}
\item $\reihe{\frac{1}{(n+1)^2}}$, $a_n=\frac{1}{(n+1)^2} = \frac{1}{n^2+2n+1} \le \frac{1}{n^2+2n} \le \frac{1}{n(n+1)} =: b_n$. Bekannt: $\reihe{b_n}$ konvergent $\folgtnach{12.2(2)}$ $\reihe{a_n}$ ist konvergent. Folgerung: $\reihe{\frac{1}{n^2}}$ ist konvergent.
\item $\reihe{\frac{1}{n^2-n+\frac{1}{8}}}$, $a_n=\frac{1}{n^2-n+\frac{1}{8}}$, $b_n:=\frac{1}{n^2}$, $\frac{a_n}{b_n} = \frac{n^2}{n^2-n+\frac{1}{8}} \to 1 \ (n\to\infty) \folgt \exists m\in\MdN: \frac{a_n}{b_n} \le 2\ \forall n\ge m \folgt a_n \le 2 b_n \ \forall n\ge m\ (|a_n|=a_n)$\\
$\reihe{2b_n}$ ist konvergent $\folgtnach{12.2(1)}$ $\reihe{a_n}$ ist konvergent.
\item Sei $\alpha \in (0,1] \cap \MdQ$: $\frac{1}{n^\alpha} \ge \frac{1}{n} \ \forall n\in\MdN \folgtnach{12.2(2)} \reihe{\frac{1}{n^\alpha}}$ ist divergent.
\item Sei $\alpha \ge 2, \alpha \in \MdQ$: $\frac{1}{n^\alpha} \le \frac{1}{n^2} \ \forall n\in\MdN \folgtnach{12.2(1)} \reihe{\frac{1}{n^\alpha}}$ ist konvergent.
\item In der Übung gezeigt: Ist $\alpha > 0$, $\alpha \in\MdQ$: $\reihe{\frac{1}{n^\alpha}}$ ist konvergent genau dann, wenn $\alpha > 1$. Bemerkung: Ist später die allgemeine Potenz $a^x \ (a>0,x\in\MdR)$ bekannt, so zeigt man analog: $\reihe{\frac{1}{n^\alpha}} \equizu \alpha > 1 \ \forall \alpha\in\MdR$.
\end{beispiele}

\begin{definition}[$\infty$ als Limes Superior]
Ist $(\alpha_n)$ eine Folge und $\alpha_n \ge 0 \ \forall n\in\MdN$ und ist $(\alpha_n)$ nicht nach oben beschränkt, so setzte $\limsup \alpha_n := \limsup_{n\to\infty} \alpha_n := \infty$.
\end{definition}
\begin{vereinbarung}
$x < \infty \ \forall x \in\MdR$
\end{vereinbarung}

\begin{satz}[Wurzelkriterium]
Sei $(a_n)$ eine Folge und $\alpha := \limsup \sqrt[n]{|a_n|}$.
\begin{liste}
 \item Ist $\alpha<1 \folgt \reihe{a_n}$ absolut konvergent
 \item Ist $\alpha>1 \folgt \reihe{a_n}$ divergent
 \item Ist $\alpha=1$, so ist keine allgemeine Aussage möglich.
\end{liste}
\end{satz}

\begin{beweise}
\item $\alpha < 1 $. Sei $\ep>0$ so, dass $x:= \alpha+x<1$. 9.2 $\folgt \sqrt[n]{|a_n|} < \alpha + \ep = x \ffa n\in\MdN \folgt |a_n| < x^n \ffa n\in\MdN$. $\reihe{x^n}$ ist konvergent $\folgtnach{12.1(1)}$ Behauptung.
\item 
 \begin{liste}
 \item $\alpha>1$, $\alpha<\infty$: Sei $\ep>0$ so, dass $\alpha-\ep>1$. 9.2 $\folgt \sqrt[n]{|a_n|}>\alpha-\ep>1$ für unendlich viele $n\in\MdN$ \folgt $|a_n|>1$ für unendlich viele $n\in\MdN$ \folgt $a_n \to 0 \folgtnach{11.1} \reihe{a_n}$ ist divergent.
 \item $\alpha = \infty \folgt \sqrt[n]{|a_n|} > 1$ für unendlich viele \natn \folgtnach{wie eben} $\reihe{a_n}$ ist divergent.
 \end{liste}
\item Siehe Beispiele
\end{beweise}

\begin{beispiele}
\item $\reihe{\frac{1}{n}}$ ist divergent. $\sqrt[n]{\frac{1}{n}} = \frac{1}{\sqrt[n]{n}} \to 1$, also $\alpha = 1$.
\item $\reihe{\frac{1}{n^2}}$ ist konvergent. $\sqrt[n]{\frac{1}{n^2}} = (\frac{1}{\sqrt[n]{n}})^2 \to 1$, also $\alpha = 1$.
\item $\reihe{\underbrace{(-1)^{n}(1+\frac{1}{n})^{-n^2}}_{=:a_n}}$. $\sqrt[n]{|a_n|} = (1+\frac{1}{n})^{-n} = \frac{1}{(1+\frac{1}{n})^n} \to \frac{1}{e} < 1 \folgt \reihe{a_n}$ ist absolut konvergent.
\item Sei $(a_n)$ eine Folge und $x\in\MdR$ mit $a_n:= \begin{cases}\frac{1}{2^n} & \text{für }n\text{ gerade} \\ n\cdot x^n & \text{für }n\text{ ungerade}\end{cases}$. \\
Betrachte $\reihe{a_n}$. $a_n := \sqrt[n]{|a_n|} = \begin{cases}\frac{1}{2} &\text{für }n\text{ gerade} \\ \sqrt[n]{n}|x| &\text{für }n\text{ ungerade}\end{cases}$. \\
$a_{2n} = \frac{1}{2} \to \frac{1}{2}$. $a_{2n-1} = \sqrt[2n-1]{2n-1}\cdot|x| \to |x|$. A16 $\folgt \H(a_n) = \{\frac{1}{2}, |x|\}$. \\
Ist $|x| < 1 \folgt \limsup \sqrt[n]{|a_n|} < 1 \folgt \reihe{a_n}$ konvergiert absolut.\\
Ist $|x| > 1 \folgt \limsup \sqrt[n]{|a_n|} > 1 \folgt \reihe{a_n}$ divergiert.\\
Sei $|x|=1$: $|a_{2n_-1}| = | (2n-1)x^{2n-1}| = 2n-1 \folgt a_n \nrightarrow 0 \folgt \reihe{a_n}$ ist divergent.
\item Sei $p\in\MdN$ und $q\in\MdR$ und $|q|<1$. \textbf{Behauptung:} $\lim_{n\to\infty} n^pq^n=0$. \textbf{Beweis:} $a_n := n^pq^n$. $\sqrt[n]{|a_n|} = \sqrt[n]{n^p}|q| = (\sqrt[n]{n})^p|q| \to |q|<1 \folgtnach{12.3} \reihe{a_n}$ ist absolut konvergent $\folgt a_n \to 0$.
\end{beispiele}

\begin{satz}[Quotientenkriterium]

Sei $(a_n)$ eine Folge in $\MdR$ und $a_n \ne 0 \ffa \natn$. $\alpha_n := \frac{a_{n+1}}{a_n}$ ($ffa n\in\MdN$).
\begin{liste}
\item Ist $|\alpha_n| \ge 1 \ffa\natn \folgt \reihe{a_n}$ ist divergent.
\item Es sei $(\alpha_n)$ beschränkt, $\beta := \liminf |a_n|$ und $\alpha := \limsup |a_n|$.
\begin{liste}
\item Ist $\beta > 1 \folgt \reihe{a_n}$ ist divergent.
\item Ist $\alpha < 1 \folgt \reihe{a_n}$ ist absolut konvergent.
\item Ist $\alpha = \beta = 1$, so ist keine allgemeine Aussage möglich.
\end{liste}
\end{liste}
\end{satz}

\begin{beweis}
O.B.d.A.: $a_n \ne 0 \ \forall\natn$
\begin{liste}
\item O.B.d.A.: $a_n \ne 0 \ \forall\natn$. Dann: $|a_2|\ge |a_1|>0$, $|a_3|\ge|a_2|\ge|a_1|>0$, \ldots allgemein: $|a_n|\ge|a_1|>0\ \forall\natn \folgt a_n\nrightarrow 0 \folgt$ die Behauptung.
\item 
\begin{liste}
\item Sei $\beta >1$, Sei $\ep>0$ so, dass $\beta-\ep>1$. 9.2 $\folgt |\alpha_n|>\beta-\ep>1 \ffa n\in\MdN \folgt$ die Behauptung.
\item Sei $\alpha < 1$. Sei $\ep>0$ so, dass $\beta-\ep>1$. 9.2 $\folgt |\alpha_n|< x \ffa\natn$. Dann: $|a_2|\le|a_1|x$, $|a_3|\le|a_2|x\le|a_1|x^2$,\ldots allgemein: $|a_n|\le|a_n1|x^{n-1} \ffa\natn$. $\reihe{|a_1|x^{n-1}}$ ist konvergent \folgtnach{12.2} $\reihe{a_n}$ ist absolut konvergent.
\item siehe Beispiele
\end{liste}
\end{liste}
\end{beweis}

\begin{beispiele}
\item $\reihe{\frac{1}{n}}$ ist divergent. $\left|\frac{a_{n+1}}{a_n}\right| = \frac{n}{n+1} \to 1$, also $\alpha = \beta = 1$.
\item $\reihe{\frac{1}{n^2}}$ ist konvergent. $\left|\frac{a_{n+1}}{a_n}\right| = \frac{n^2}{(n+1)^2} \to 1$, also $\alpha = \beta = 1$.
\end{beispiele}

\begin{wichtigesbeispiel}[Exponentialfunktion]
Für $x\in\MdR$ betrachte die Reihe
$$\reihenull{\frac{x^n}{n!}} = 1 + x + \frac{x^2}{2!} + +\frac{x^3}{3!} + \frac{x^4}{4!} + \ldots$$
Für welche $x\in\MdR$ konvergiert diese Reihe (absolut)?.

Klar: für $x=0$ konvergiert die Reihe.

Sei $x\ne0$ und $a_n=\frac{x^n}{n!}$;
$$\left|\frac{a_{n+1}}{a_n}\right| = \left| \frac{x^{n+1}}{(n+1)!} \cdot \frac{n!}{x^n}\right| = \frac{|x|}{n+1} \to 0 \quad (n\to\infty) \quad (\text{also } \alpha = \beta = 0)$$
12.4 $\folgt \reihenull{\frac{x^n}{n!}}$ ist absolut konvergent für alle $x\in\MdR$.

Also wird durch $E(x) := \reihenull{\frac{x^n}{n!}} \quad (x\in\MdR)$ eine Funktion $E: \MdR \to \MdR$ definiert. Diese Funktion $E$ heißt die \begriff{Exponentialfunktion}.

$E(0) = 1$, $E(1)=\reihenull{\frac{1}{n!}} = e$.

\begin{bemerkung}
Später zeige wir: $E(r) = e^r \ \forall r\in\MdQ$. Dann \textit{definieren} wir $e^x := E(x) \quad (x\in\MdR)$.
\end{bemerkung}
\end{wichtigesbeispiel}

\begin{motivation}
$b_n := (-1)^n \quad (n\in\MdN)$, $b_n \nrightarrow 0 \folgt \reihe{b_n}=b_1 + b_2 + \ldots$ ist divergent. \\
$a_1 := b_1 + b_2$, $a_2 := b_2 + b_4$, \ldots also: $a_n = 0 \ \forall\natn \folgt \reihe{a_n} = (b_1+b_2) + (b_3+b_4) + \ldots$ ist konvergent. Also: \glqq Im Allgemeinen darf man Klammern in konvergenten Reihen nicht weglassen.\grqq
\end{motivation}

\begin{satz}[In konvergenten Folgen darf man Klammern setzen]

Sei $\reihe{a_n}$ konvergent und es seien $n_1, n_2, \ldots \in\MdN$ mit $n_1<n_2<\ldots$. Setze $b_1 := a_1 + \ldots + a_{n_1}$, $b_2 := a_{n_1+1} + \ldots + a_{n_2}$, allgemein: $b_k := a_{n_{k-1}+1} + \ldots + a_{n_k} \quad (k\ge2)$. Dann ist $\reihe{b_k}$ konvergent und $\reihe{b_k} = \reihe{a_n}$.
\end{satz}

\begin{beweis}
$s_n := a_1 + a_2 + \ldots + a_n$; $\sigma_k := b_1 + b_2 + \ldots b_k$. Es ist $\sigma_k = a_1 + a_2 + \ldots + a_{n_k} = s_{n_k} \folgt \sigma_k$ ist eine Teilfolge von $s_n \folgtnach{8.1(3)} (\sigma_k)$ konvergent und $\lim_{k\to0}\sigma_k=\lim_{n\to\infty}s_n$.
\end{beweis}
\end{document}
