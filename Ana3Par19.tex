\documentclass{article}
\newcounter{chapter}
\setcounter{chapter}{19}
\usepackage{ana}
\usepackage{mathrsfs}

\title{Lineare Differentialgleichungen $m$-ter Ordnung}
\author{Jonathan Picht und Ferdinand Szekeresch}
% Wer nennenswerte �nderungen macht, schreibt sich bei \author dazu

\begin{document}
\maketitle

In diesem Paragraphen: $I \subseteq \MdR$ ein Intervall, $a_0, a_1, \ldots,
a_{m-1}, b \in C(I,\MdR), x_0, y_0, \ldots, y_{m-1} \in \MdR$. Die
Differentialgleichung $y^{(m)} + a_{m-1}(x)y^{(m-1)} + \ldots + a_1(x)y' +
a_0(x)y = b(x)$ hei�t eine \begriff{lineare Differentialgleichung $m$-ter
Ordnung}.

Setze $Ly:= y^{(m)} + a_{m-1}(x)y^{(m-1)} + \ldots + a_1(x)y' + a_0(x)y$. Dann schreibt sich obige Gleichung in der Form
$$Ly = b(x)$$.

Diese Gleichung hei�t \begriff{homogen}, falls $b\equiv 0$, anderenfalls \begriff{inhomogen}.
Das zur Gleichung $Ly=b$ geh�rende System (S) aus � 18 lautet
$$z'=A(x)z+b_0(x)$$
mit $b_0(x) = (0, \ldots 0, b(x))$ und $A(x) =\begin{pmatrix}
0 & 1 & 0 & \ldots & 0\\
0 & 0 & 1 & & 0\\
\vdots & \vdots & & \ddots & 0\\
0 & 0 & 0 & & 1\\
-a_0(x) & \ldots & \ldots & \ldots & -a_{m-1}(x)\\
\end{pmatrix}$

Die Beweise der folgenden S�tze 19.1 bis 19.4 folgen aus den Paragraphen 16 und 18.

\begin{satz} %19.1
Das Anfangswertproblem $\begin{cases}Ly=b(x)\\y(x_0)=y_0, y'(x_0)=y_1, \ldots, y^{(m-1)}(x_0)=y_{m-1}\end{cases}$ hat auf $I$ genau eine L�sung.
\end{satz}

{\bf Wie in � 16:} Ist $J \subseteq I$ ein Intervall und $\hat{y}:J\rightarrow \MdR$
eine L�sung von $Ly=b$ auf $J$, so existiert eine L�sung $y:I\rightarrow \MdR$
der Gleichung $Ly=b$ auf $I$ mit $\hat{y}=y|_J$.

Daher betrachten wir immer L�sungen $y:I\rightarrow \MdR$.

Die zu $Ly=b$ geh�rende \emph{homogene} Gleichung lautet: $(\text{H})\quad Ly=0$.

\begin{satz} %19.2
Sei $y_s$ eine spezielle L�sung der Gleichung $Ly=b$ und $y:I\rightarrow \MdR$ eine Funktion.

Dann: $y$ ist eine L�sung von $Ly=b \equizu \exists \; y_0:I\rightarrow \MdR: y_0$ ist eine L�sung von (H) und $y = y_0 + y_s$.

$\MdL := \{ y:I\rightarrow \MdR: y\; \text{l�st (H) auf}\; I \}$.
\end{satz}

\begin{satz} %19.3
\begin{liste}
\item $\MdL$ ist ein reeller Vektorraum, $\dim \MdL = m$.
\item F�r $y_1, \ldots, y_k \in \MdL$ sind �quivalent:
\begin{liste}
\item $y_1, \ldots, y_k$ sind linear unabh�ngig in $\MdL$;
\item $\forall x \in I$ sind $(y_j(x), y_j'(x), \ldots, y_j^{(m-1)}(x))\quad (j=1,\ldots k)$ linear unabh�ngig in $\MdR^m$;
\item $\exists x \in I:(y_j(x), y_j'(x), \ldots, y_j^{(m-1)}(x))\quad (j=1,\ldots, k)$ sind linear unabh�ngig in $\MdR^m$.
\end{liste}
\end{liste}
\end{satz}

\begin{definition} 
Seien $y_1, \ldots, y_m \in \MdL$. $y_1, \ldots, y_m$ hei�t ein
\begriff{L�sungssystem} (LS) von (H) und
\[W(x) := \left|\begin{array}{ccc}
y_1(x) & \ldots & y_m(x)\\y_1'(x) & \ldots & y_m'(x)\\ \vdots & & \vdots\\
y_1^{(m-1)}(x) & \ldots & y_m^{(m-1)}(x)\end{array}\right|\] hei�t
\begriff{Wronskideterminante}.

Sind $y_1, \ldots, y_m$ linear unabh�ngig in $\MdL$, so hei�t $y_1, \ldots, y_m$ ein \begriff{Fundamentalsystem} (FS) von (H).
\end{definition}

\begin{satz} %19.4
Sei $y_1, \ldots y_m$ ein L�sungssystem von (H).
\begin{liste} 
\item $W(x) = W(\xi)e^{-\int_{\xi}^{x}a_{m-1}(t)dt} \  (x, \xi \in I)$
\item $y_1, \ldots y_m$ ist ein Fundamentalsystem von (H) $\equizu W(x) \neq 0 \, \forall x \in I \equizu \exists \xi \in I: W(\xi) \neq 0$
\end{liste}
\end{satz}

\begin{satz}[Reduktionsverfahren von d'Alembert ($m=2$)]
Sei $y_1$ eine L�sung von $(*)\quad y''+a_1(x)y'+a_0(x)y=0$ und $y_1(x) \neq 0 \, \forall x \in I$. Sei $z$ eine L�sung von $z'=-(a_1(x) + \frac{2y_1'(x)}{y_1(x)})z$, $z \neq 0$ und $y_2(x):=y_1(x)\int z(x)dx$.

Dann ist $y_1, y_2$ ein Fundamentalsystem von (*).
\end{satz}

\begin{beweis}
Nachrechnen: $y_2$ l�st (*).
$W(x)=\left|\begin{array}{cc} y_1 & y_2 \\ y_1' & y_2'\end{array}\right| =
\left|\begin{array}{cc} y_1 & y_1\int zdx \\ y_1' & y_1'\int zdx + y_1 z
\end{array}\right| =\\ y_1y_1'\int zdx+y_1^2z-y_1y_1'\int zdx = 
\underbrace{y_1^2}_{>0}z \folgtnach{19.4} y_1, y_2$ sind linear unabh�ngig in
$\MdL$.

\end{beweis}

\begin{beispiel}
$(**)\quad y''+\frac{2x}{1-x^2}y'-\frac{2}{1-x^2}y = 0\quad (I=(1,\infty)); \, y_1(x)=x$

$z'=-(\frac{2x}{1-x^2}+\frac{2}{x})z = - \frac{2x^2+2(1-x^2)}{x(1-x^2)}z = \frac{2}{x(x^2-1)}z\quad (***)$

$\int \frac{2}{x(x^2-1)}dx = \log (1-\frac{1}{x^2})$

� 7 $\folgt$ allgemeine L�sung von (***): $z(x) = c e^{\log (1-\frac{1}{x^2})} = c(1-\frac{1}{x^2)} \quad (c \in \MdR)$

$z(x) = 1 - \frac{1}{x^2} \folgt \int z(x) dx = x + \frac{1}{x} \folgt y_2(x) = x(x+\frac{1}{x}) = 1 + x^2$

Fundamentalsystem: $y_1, y_2$. Allgemeine L�sung von (**): $y(x) = c_1x+c_2(1+x^2)\quad (c_1, c_2 \in \MdR)$
\end{beispiel}

\begin{satz} %19.6
Sei $y_1, \ldots, y_m$ ein FS von (H). $W$ sei die Wronskideterminante von $y_1, \ldots, y_m$ und f�r $k=1, \ldots, m$ sei $W_k(x)$ die Determinante, die aus $W(x)$ entsteht, indem man in $W(x)$ die $k$-te Spalte ersetzt durch $(0, \ldots, 0, b(x))^T$. Dann ist
$$y_s:=\sum_{k=1}^{m}y_k\int\frac{W_k}{W}dx$$
eine spezielle L�sung von $L_y = b(x)$.
\end{satz}

\begin{beweis}
�16, �18
\end{beweis}

\begin{beispiel}
$y''+\frac{2x}{1-x^2}y'-\frac{2}{1-x^2}y = x^2-1$ \\\\
FS der homogenen Gleichung: $x, x^2+1$ \\
$W(x) = \left|\begin{array}{cc}x & x^2+1 \\ 1 & 2x\end{array}\right| = 2x^2 - (x^2+1) = x^2 - 1$ \\
$W_1(x) = \left|\begin{array}{cc}0 & x^2+1 \\ x^2-1 & 2x\end{array}\right| = -(x^2+1)(x^2-1) \folgt \frac{W_1(x)}{W(x)} = -(x^2+1)$ \\
$\folgt \int\frac{W_1}{W}dx = -\frac{1}{3}x^3 - x$ \\
$W_2(x) = \left|\begin{array}{cc}x & 0 \\ 1 & x^2-1\end{array}\right| = x(x^2-1) \folgt \frac{W_2(x)}{W(x)} = x \folgt \int\frac{W_2}{W}dx = \frac{1}{2}x^2 \\
\folgt y_s(x) = -\frac{1}{3}x^4 - x^2 + (x^2+1)\frac{1}{2}x^2 = \frac{1}{6}x^4 - \frac{1}{2}x^2$. \\\\
Allgemeine L�sung der inhomogenen Gleichung: \\
$y(x) = c_1x + c_2(x^2+1)+\frac{1}{6}x^4+\frac{1}{2}x^2 (c_1,c_2 \in \MdR)$
\end{beispiel}

\end{document}
