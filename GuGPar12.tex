\documentclass[a4paper, 10pt]{report}

\usepackage{GuGStyle}

\setcounter{chapter}{1}
\setcounter{section}{1}
\setcounter{Satz}{0}

\begin{document}

\section{Bäume}

\begin{Def}
Ein \emp{Baum}\index{Baum} ist ein zusammenhängender Graph ohne Kreise (der Länge $\ge 1$)
\fboxsep20pt
% Knoten
\fbox{\xygraph{
!{<0cm,0cm>;<1cm,0cm>:<0cm,1cm>::}
!{(0,0) }*+{\bullet}="a"
}}
% Baum mit 2 Knoten
\fbox{\xygraph{
!{<0cm,0cm>;<1cm,0cm>:<0cm,1cm>::}
!{(0,0) }*+{\bullet}="a"
!{(1,0) }*+{\bullet}="b"
"a"-"b"
}}
% Baum mit 3 Knoten
\fbox{\xygraph{
!{<0cm,0cm>;<1cm,0cm>:<0cm,1cm>::}
!{(0,0) }*+{\bullet}="a"
!{(1,0) }*+{\bullet}="b"
!{(2,0) }*+{\bullet}="c"
"a"-"b"
"b"-"c"
}}
% Baum (Stern)
\fbox{\xygraph{
!{<0cm,0cm>;<1cm,0cm>:<0cm,1cm>::}
!{(1,-1) }*+{\bullet}="a"
!{(1,0) }*+{\bullet}="b"
!{(0,-2) }*+{\bullet}="c"
!{(2,-2) }*+{\bullet}="d"
"a"-"b"
"a"-"c"
"a"-"d"
}}
\end{Def}

\begin{Prop}
Ein Graph ist genau dann ein Baum, wenn es zu je 2 Ecken $x,y \in E(\Gamma)$
genau einen stachelfreien Weg von $x$ nach $y$ in $\Gamma$ gibt.
\end{Prop}
\begin{Bew}
\begin{itemize}
  \item[$\Rightarrow:$] Seien $x,y \in E(\Gamma), \; w = (k_1, \ldots, k_n)
  \textrm{ und } w' = (k_1', \ldots, k_n')$ stachelfreie Wege von $x$ nach $y$.\\
  Ist $k_n \not= k_m'$, so ist $\tilde{w} = (k_1, \ldots, k_n, k_m', \ldots,
  k_n')$ ein stachelfreier geschlossener Weg, enthält also einen Kreis
  $\Rightarrow$ Widerspruch.\\
  
  % TODO da oben noch ein Bildchen

  $\Rightarrow k_n = k_m'$. Induktion über $n$ ergibt Behauptung.
\end{itemize}

\end{Bew}

\end{document}