\documentclass[a4paper, 10pt]{report}

\usepackage{GuGStyle}

\setcounter{chapter}{1}
\setcounter{section}{1}
\setcounter{Satz}{0}

\begin{document}

\section{Bäume}

\begin{Def}
Ein \emp{Baum}\index{Baum} ist ein zusammenhängender Graph ohne Kreise (der Länge $\ge 1$)
\fboxsep20pt
% Knoten
\fbox{\xygraph{
!{<0cm,0cm>;<1cm,0cm>:<0cm,1cm>::}
!{(0,0) }*+{\bullet}="a"
}}
% Baum mit 2 Knoten
\fbox{\xygraph{
!{<0cm,0cm>;<1cm,0cm>:<0cm,1cm>::}
!{(0,0) }*+{\bullet}="a"
!{(1,0) }*+{\bullet}="b"
"a"-"b"
}}
% Baum mit 3 Knoten
\fbox{\xygraph{
!{<0cm,0cm>;<1cm,0cm>:<0cm,1cm>::}
!{(0,0) }*+{\bullet}="a"
!{(1,0) }*+{\bullet}="b"
!{(2,0) }*+{\bullet}="c"
"a"-"b"
"b"-"c"
}}
% Baum (Stern)
\fbox{\xygraph{
!{<0cm,0cm>;<1cm,0cm>:<0cm,1cm>::}
!{(1,-1) }*+{\bullet}="a"
!{(1,0) }*+{\bullet}="b"
!{(0,-2) }*+{\bullet}="c"
!{(2,-2) }*+{\bullet}="d"
"a"-"b"
"a"-"c"
"a"-"d"
}}
\end{Def}

\begin{Prop}
Ein Graph ist genau dann ein Baum, wenn es zu je 2 Ecken $x,y \in E(\Gamma)$
genau einen stachelfreien Weg von $x$ nach $y$ in $\Gamma$ gibt.
\end{Prop}
\begin{Bew}
\begin{itemize}
  \item[$\Rightarrow:$] Seien $x,y \in E(\Gamma), \; w = (k_1, \ldots, k_n)
  \textrm{ und } w' = (k_1', \ldots, k_n')$ stachelfreie Wege von $x$ nach $y$.\\
  Ist $k_n \not= k_m'$, so ist $\tilde{w} = (k_1, \ldots, k_n, k_m', \ldots,
  k_n')$ ein stachelfreier geschlossener Weg, enthält also einen Kreis
  $\Rightarrow$ Widerspruch.\\
  
  % TODO da oben noch ein Bildchen

  $\Rightarrow k_n = k_m'$. Induktion über $n$ ergibt Behauptung.
\end{itemize}

\end{Bew}

% 2006-10-26

\begin{DefBem}
Sei $\Gamma$ ein Graph, $x \in E(\Gamma)$
\begin{enumerate}
  \item Sei $K_x \defeqr \{k \in K(\Gamma): i(k)=x\}$,\\
  $v(x) \defeqr \#K_x$ heißt \emp{Ordnung}\index{Ecke!Ordnung} (oder Valenz,
  Index, $\ldots$) von $x$.\\

  \fbox{ \xygraph{
  !{<0cm,0cm>;<1cm,0cm>:<0cm,1cm>::}
  !{(0,0) }*+{\bullet_{x}}="a"
  !{(1,0) }*+{\bullet}="b"
  "a":@/^0.5cm/"b" ^{k_1}
  "a":@/_0.5cm/"b" _{k_2}
  }}
  $\Rightarrow v(x)=2$

  \item $x$ heißt Endpunkt, wenn $v(x) = 1$.
  \item $\Gamma \setminus x$ sei der Graph mit $E(\Gamma \setminus x) =
  E(\Gamma) \setminus \{x\}$ und $K(\Gamma \setminus x) = K(\Gamma) \setminus
  (K_x \cup \overline{K_x})$\\
  $\Gamma \setminus x$ ist ein Teilgraph von $\Gamma$ (Entfernen des ``Sterns''
  um $x$)\\
  
  
  $\Gamma=$
  \fbox{ \xygraph{
  !{<0cm,0cm>;<1cm,0cm>:<0cm,1cm>::}
  !{(0,0) }*+{\bullet_{x}}="a"
  !{(1,0) }*+{\bullet}="b"
  !{(2,1) }*+{\bullet}="c"
  !{(2,-1) }*+{\bullet}="d"
  "a":"b"
  "b":"c"
  "b":"d"
  "c":"a"
  }}
  $\Rightarrow \Gamma \setminus x = $
  \fbox{ \xygraph{
  !{<0cm,0cm>;<1cm,0cm>:<0cm,1cm>::}
  !{(1,0) }*+{\bullet}="b"
  !{(2,1) }*+{\bullet}="c"
  !{(2,-1) }*+{\bullet}="d"
  "b":"c"
  "b":"d"
  }}
  

  \item Ist $x$ ein Endpunkt, so gilt:
  \begin{enumerate}
    \item[(1)] $\Gamma$ zusammenhängend $\Leftrightarrow \Gamma \setminus x$
    zusammenhängend.
    \item[(2)] Jeder Kreis von $\Gamma$ ist in $\Gamma \setminus x$ enthalten.
    \item[(3)] $\Gamma$ ist Baum $\Leftrightarrow \Gamma \setminus x$ ist Baum.
  \end{enumerate}
  \begin{Bew}
  (1) und (2) sind offensichtlich, (3) folgt daraus.
  \end{Bew}
\end{enumerate}
\end{DefBem}


\begin{Prop}
\begin{enumerate}
  \item Sei $f: \Gamma \to \Gamma'$ ein Isomorphismus von Graphen.\\
  Dann gilt für alle $x \in E(\Gamma): v(x) = v(f_E(x))$.
  \item Sei $\Gamma$ ein Graph, $\textrm{Endpunkte}(\Gamma) \defeqr ...$,
  $\Gamma' \defeqr \Gamma \setminus \textrm{Endpunkte}(\Gamma)$.\\
  Jeder Automorphismus von $\Gamma$ induziert einen Automorphismus von
  $\Gamma'$.
  \item Ist $\Gamma$ ein Baum von endlichem Durchmesser $n$, so gibt es falls
  $n$ gerade/ungerade eine Ecke $x \in E(\Gamma)$/geometrische Kante $\kappa =
  (k, \bar{k})$ mit $f(x) = x$/$f(\kappa) = \kappa$ für jeden Automorphismus $f$
  von $\Gamma$.
  
  % TODO Die Fallunterscheidung überarbeiten
  
\end{enumerate}
\end{Prop}

\begin{Bew}
\begin{enumerate}
  \item \label{Bew2.4a}
  $f_K$ induziert Bijektion $K_x \to K'_{f_E(x)}$
  \item folgt aus \ref{Bew2.4a}
  \item
  \begin{itemize}
    \item[n$=$0]
    \fbox{\xygraph{
    !{<0cm,0cm>;<1mm,0cm>:<0cm,1cm>::}
    !{(0,0) }*+{\bullet}="a"
    }}
    
    \item[n$=$1]
    \fbox{\xygraph{
    !{<0cm,0cm>;<1mm,0cm>:<0cm,1cm>::}
    !{(0,0) }*+{\bullet}="a"
    !{(10,0) }*+{\bullet}="b"
    "a"-"b"
    }}
    
    \item[n$=$2]
    \fbox{\xygraph{
    !{<0cm,0cm>;<1mm,0cm>:<0cm,1cm>::}
    !{(0,0) }*+{\bullet}="a"
    !{(10,0) }*+{\bullet}="b"
    !{(20,0) }*+{\bullet}="c"
    "a"-"b"
    "b"-"c"
    }}\\
    \begin{Beh}
      $\Gamma'$ ist Baum vom Durchmesser $n-2$.
    \end{Beh}

    Daraus folgt mit Induktion über $n$ die Aussage der Proposition.
    
    \begin{Bew}
      Sei $w = (k_1, \ldots, k_m)$ stachelfreier Weg in $\Gamma'$, $x=i(w), \; y
      = t(w)$.
      Dann ist $m = d(x,y)$.\\
      In $\Gamma$ gilt: $v(x) \ge 2, \; v(y) \ge 2$, also gibt es Kanten $k_1',
      \; k_m'$ mit $i(k_1')=x, \; k_1' \not= k_1$ und $i(k_m') = y, \; k_m'
      \not= \overline{k_m}$.\\
      
      \fbox{\xygraph{
      !{<0cm,0cm>;<1mm,0cm>:<0cm,1cm>::}
      !{(1,1) }*+{\bullet}="s"
      !{(11,0) }*+{\bullet_{x}}="x"
      !{(21,0) }*+{\bullet}="a"
      !{(22,0) }*+{}="a2"
      !{(23,0) }*+{}="a3"
      !{(24,0) }*+{}="a4"
      !{(25,0) }*+{}="a5"
      !{(26,0) }*+{}="a6"
      !{(27,0) }*+{}="a7"
      !{(28,0) }*+{}="a8"
      !{(29,0) }*+{}="a9"
      !{(30,0) }*+{\bullet}="b"
      !{(40,0) }*+{\bullet_{y}}="c"
      !{(50,1) }*+{\bullet}="d"
      "x":"s" _{k_1'}
      "x":"a" ^{k_1}
      "a"-"a4"
      "a5"-"a6"
      "a7"-"b"
      "b":"c" ^{k_m}
      "c":"d" _{k_m'}
      }}\\
      Dann ist $w' = (\overline{k_1'}, k_1, \ldots, k_m, k_m')$ stachelfreier
      Weg in $\Gamma$.\\
      $\Rightarrow m+2 \le n \Rightarrow m \le n-2$.\\
      Sei umgekehrt $w = (k_1, \ldots, k_n)$ stachelfreier weg in $\Gamma$.\\
      Für $i=2, \ldots, n$ ist $v(i(k_i)) \ge 2$\\
      $\Rightarrow (k_2, \ldots, k_{n-1})$ ist stachelfreier Weg in $\Gamma'$\\
      $\Rightarrow d(\Gamma') \ge 2$.
    \end{Bew}
  \end{itemize}
\end{enumerate}
\end{Bew}

\begin{nnBsp}

\fbox{\xygraph{
!{<0cm,0cm>;<1mm,0cm>:<0cm,1cm>::}
!{(3,0) }*+{}="a3"
!{(4,0) }*+{}="a4"
!{(5,0) }*+{}="a5"
!{(6,0) }*+{}="a6"
!{(7,0) }*+{}="a7"
!{(8,0) }*+{}="a8"
!{(9,0) }*+{}="a9"
!{(10,0) }*+{\bullet}="b"
!{(20,0) }*+{\bullet}="c"
!{(30,0) }*+{\bullet}="d"
!{(33,0) }*+{}="e3"
!{(34,0) }*+{}="e4"
!{(35,0) }*+{}="e5"
!{(36,0) }*+{}="e6"
!{(37,0) }*+{}="e7"
"a3"-"a4"
"a5"-"a6"
"a7"-"b"
"b"-"c"
"c"-"d"
"d"-"e3"
"e4"-"e5"
"e6"-"e7"
}}
ist Baum.
Translation ist Automorphismus ohne Fixpunkt und Fixkante.
\end{nnBsp}


\begin{Folg}
\label{2.15}
Jeder endliche Baum besteht aus
\fbox{\xygraph{
!{<0cm,0cm>;<1mm,0cm>:<0cm,1cm>::}
!{(0,0) }*+{\bullet}="a"
}}
durch wiederholtes Anhängen von ``Endpunkten''.
\end{Folg}


\begin{DefBem}
Sei $\Gamma$ ein Graph.
\begin{enumerate}
  \item Ein Teilbaum $T \subseteq \Gamma$ heißt
  \emp{aufspannend}\index{Baum!auspannender} (oder \emp{Gerüst}\index{Gerüst}),
  wenn $E(T)=E(\Gamma)$.
  \item Jeder zusammenhängende Graph hat einen aufspannenden Teilbaum.
  \begin{Bew}
  Sei $T \subset \Gamma$ Teilbaum.\\
  Ist $E(T) \not= E(\Gamma)$, so gibt es eine Kante $k \in K(\Gamma)$ mit $i(k)
  \in E(\Gamma)$ und $t(k) \not\in E(\Gamma)$.\\
  Dann ist $T' = T \cup \{k, \bar{k}, t(k)\}$ Teilbaum von $\Gamma$.\\
  Ist $T$ endlich, so erhalten wir mit Induktion Teilbaum $T$ mit $E(T) =
  E(\Gamma)$.\\
  Falls nicht betrachte die Menge $\mathcal{T}$ der Teilbäume von $\Gamma$.\\
  $\mathcal{T} \not= \emptyset$ \chk\\ 
  Ist $T_1 \subseteq T_2 \subseteq \ldots$ Kette von Teilbäumen, so ist
  $\bigcup_i T_i \in \mathcal{T}$ \chk\\
  Zorn sagt: $\mathcal{T}$ enthält maximales Element $T$.\\
  Angenommen $E(T) \not= E(\Gamma)$. Dann gibt es nach obiger Konstruktion
  Teilbaum $T'$ mit $T \subsetneq T'$ \WSpr zu $T$ maximal
  \end{Bew}
\end{enumerate}
\end{DefBem}


\begin{BemDef}
\label{2.7}
Sei $\Gamma$ ein endlicher zusammenhängender Graph, $e(\Gamma) \defeqr
\#E(\Gamma), \; k(\Gamma) \defeqr \frac{1}{2}\#K(\Gamma)$.
Dann gilt:
\begin{enumerate}
  \item $g(\Gamma) \defeqr k(\Gamma) - e(\Gamma) + 1 \ge 0$.
  \item \label{2.7b}
  $g(\Gamma) = 0 \Leftrightarrow \Gamma$ ist Baum.
  \item $g(\Gamma)$ heißt \emp{Geschlecht}\index{Geschlecht} von $\Gamma$ (oder
  zyklomatische Zahl, Betti-Zahl).
\end{enumerate}
\end{BemDef}

\begin{Bew}
Ist $\Gamma$ ein Baum, so ist $g(\Gamma) = 0$ wegen Folgerung \ref{2.15}.\\
Ist $\Gamma$ kein Baum, so sei $T$ Gerüst von $\Gamma$, also $e(T) = e(\Gamma),
\; k(\Gamma) > k(T) \Rightarrow$ Behauptung
\end{Bew}


\begin{DefBem}
Sei $\Gamma$ ein Graph, $Z$ ein zusammenhängender Teilgraph.\\
$\Gamma/Z$ sei der folgende Graph:\\
$E(\Gamma/Z) = (E(\Gamma) \setminus E(Z)) \cup \{Z\}$\\
$K(\Gamma/Z) = K(\Gamma) \setminus K(Z)$\\
$I_Z(k) = (i_Z(k), t_Z(k))$ mit 
$i_Z = \begin{cases}i(k), & i(k) \not\in E(Z)\\ Z, & i(k) \in E(Z)
\end{cases}$ und 
$t_Z = \begin{cases}t(k), & t(k) \not\in E(Z)\\ Z, & t(k) \in E(Z)
\end{cases}$ für $k \in K(\Gamma/Z)$.
\end{DefBem}


\begin{nnBsp} 
$Z$ ist Gerüst von $\Gamma$.\\
\fbox{ \xygraph{
!{<0cm,0cm>;<1cm,0cm>:<0cm,1cm>::}
!{(0,0) }*+[blue]{\bullet_{a}}="a"
!{(1,1) }*+[blue]{\bullet_{b}}="b"
!{(2.5,0.5) }*+[blue]{\bullet_{c}}="c"
!{(3,-1)}*+[blue]{\bullet_{d}}="d"
"a"-@[blue]"b" "a"-"d"
"b"-@[blue]"c"
"b"-@[blue]"d"
} }
$\Rightarrow$
\fbox{ \xygraph{
!{<0cm,0cm>;<1cm,0cm>:<0cm,1cm>::}
!{(0,0) }*+[blue]{\bullet_{z}}="a"
"a"-@(ld,rd)"a"
} }
\end{nnBsp}

\begin{Bem}
\label{2.9}
Ist $\Gamma$ endlicher zusammenhängender Graph, $Z$ Teilgraph, so gilt:
\begin{enumerate}
  \item $g(\Gamma/Z) \le g(\Gamma)$.
  \item Falls $Z$ Teilbaum ist, so ist $g(\Gamma/Z) = g(\Gamma)$.
\end{enumerate}
\end{Bem}


\begin{Prop}
Sei $\Gamma$ ein zusammenhängender Graph. Z ein Teilgraph, dessen
Zusammenhangskomponenten Bäume sind. Dann gilt:\\
$\Gamma/Z$ ist Baum $\Leftrightarrow \Gamma$ ist Baum.
\end{Prop}

\begin{Bew}
Ist $\Gamma$ endlich, so folgt die Behauptung aus Bemerkung \ref{2.9} und
Bemerkung \ref*{2.7}\ref{2.7b}.\\
Ist $\Gamma$ unendlich, so sei $\Gamma'$ endlicher zusammenhängender Teilgraph
von $\Gamma$.\\
Dann ist $\Gamma' \cap Z$ ein ``Wald''.
Also ist $\Gamma'$ Baum $\Leftrightarrow \Gamma'/(\Gamma' \cap Z)$ Baum.\\
Schöpfe $\Gamma$ aus durch endliche Teilgraphen.
\end{Bew}

\end{document}