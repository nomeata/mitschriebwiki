\documentclass[a4paper, 10pt]{report}

\usepackage{GuGStyle}

\setcounter{chapter}{2}
\setcounter{section}{2}
\setcounter{Satz}{0}

\begin{document}

% Auf meiner Kopie fehlt der linke Rand. Das sollte jemand checken.

\section{Cayley-Graphen von Gruppen}

\begin{Bem}
\begin{enumerate}
\item[a)] Für jeden Graphen $\Gamma$ ist $\Aut(\Gamma)$ eine Gruppe.

\item[b)] $\Aut(\Gamma)$ ist Untergruppe von $\Sym(E(\Gamma)) \times \Sym(K(\Gamma))$

\item[c)] Ist $\Gamma$ kombinatorischer Graph, so ist $\Aut(\Gamma) \subseteq \Sym(E(\Gamma))$.
\end{enumerate}

\begin{Bew}
Klar.
\end{Bew}
\end{Bem}

\begin{nnBsp}
\begin{enumerate}
\item[1)] $\Gamma$ = \fbox{\xygraph{
!{<0cm,0cm>;<1cm,0cm>:<0cm,1cm>::}
!{(0,0) }*+{\bullet_1}="a"
!{(1,0) }*+{\bullet_2}="b"
"a":"b"_{k}
}}

$\Aut(\Gamma) = \{ id, \sigma \} \cong \mathbb{Z}/2\mathbb{Z}$, $\sigma(1) = 2, \sigma(2) = 1$, $\sigma(k) = \bar{k}$.

\item[2)] $\Gamma$ = \fbox{\xygraph{
!{<0cm,0cm>;<1cm,0cm>:<0cm,1cm>::}
!{(0,0) }*+{\bullet_1}="a"
!{(0.5,1) }*+{\bullet_2}="b"
!{(1,0) }*+{\bullet_3}="c"
"a"-"b"
"b"-"c"
"c"-"a"
}}
$\Aut(\Gamma) \cong S_3$

\item[3)] $\Gamma$ = \fbox{\xygraph{
!{<0cm,0cm>;<1cm,0cm>:<0cm,0.7cm>::}
!{(0,0) }*+{\bullet}="a"
!{(2,0) }*+{\bullet}="b"
!{(1,0.8) }*+{\bullet}="c"
!{(1,2) }*+{\bullet}="d"
"a"-"b"
"a"-"c"
"a"-"d"
"b"-"c"
"b"-"d"
"c"-"d"
}}
$\Aut(\Gamma) \cong S_4$

\item[4)] $\Gamma$ = \fbox{\xygraph{
!{<0cm,0cm>;<1cm,0cm>:<0cm,1cm>::}
!{(0,0) }*+{\bullet_1}="a"
!{(1,1) }*+{\bullet_2}="b"
!{(2,1) }*+{\bullet_3}="c"
!{(3,0) }*+{\bullet_4}="d"
!{(2,-1) }*+{\bullet_5}="e"
!{(1,-1) }*+{\bullet_6}="f"
"a"-"b"
"b"-"c"
"c"-"d"
"d"-"e"
"e"-"f"
"f"-"a"
}}
$\Aut(\Gamma) \cong D_6$

Drehungen: 6; Spiegelungen: 3 mit Achse durch 2 Ecken, 3 mit Achse durch Knotenmittelpunkte.

\end{enumerate}
\end{nnBsp}

\begin{DefBem}
Sei $G$ eine Gruppe, $S \subseteq G$.

Definiere $E(\Gamma(G,S)) := G$, $K(\Gamma(G,S)) := G \times S \times \{-1,1\}$.

Für $k = (g,s,\varepsilon) \in K(\Gamma(G,S))$ sei $\bar{k} = (g,s,-\varepsilon)$ und $i(k) = \begin{cases}g,  t(k) = g \cdot s & \text{ falls } \varepsilon = 1\\ gs,  t(k) = g & \text{ falls } \varepsilon = -1\end{cases}$.

Somit ist $\Gamma(G,S)$ ein Graph, der \emp{Cayley-Graph}\index{Cayley-Graph} von $G$ bzgl. $S$.
\end{DefBem}

\begin{nnBsp}
\begin{enumerate}
\item[1)] $G$ beliebig, $S = \emptyset$ $\Rightarrow$ $\Gamma(G,S) = \emptyset$

\item[2)] $G = \mathbb{Z}$, $S = \{ 1 \}$ $\Rightarrow$ \fbox{\xygraph{
!{<0cm,0cm>;<1cm,0cm>:<0cm,1cm>::}
!{(-3,0) }*+{}="-3"
!{(-2,0) }*+{\bullet_{-2}}="-2"
!{(-1,0) }*+{\bullet_{-1}}="-1"
!{(0,0) }*+{\bullet_0}="0"
!{(1,0) }*+{\bullet_1}="1"
!{(2,0) }*+{\bullet_2}="2"
!{(3,0) }*+{\bullet_3}="3"
!{(4,0) }*+{}="4"
"-3"-@{--}"-2"
"-2"-"-1"
"-1"-"0"
"0"-"1"
"1"-"2"
"2"-"3"
"3"-@{--}"4"
}}

$S = \{ 2 \}$ $\Rightarrow$ \fbox{\xygraph{
!{<0cm,0cm>;<1cm,0cm>:<0cm,1cm>::}
!{(-4,0) }*+{\cdots}="-4"
!{(-3,0) }*+{\bullet_{-3}}="-3"
!{(-2,0) }*+{\bullet_{-2}}="-2"
!{(-1,0) }*+{\bullet_{-1}}="-1"
!{(0,0) }*+{\bullet_0}="0"
!{(1,0) }*+{\bullet_1}="1"
!{(2,0) }*+{\bullet_2}="2"
!{(3,0) }*+{\bullet_3}="3"
!{(4,0) }*+{\cdots}="4"
"-4"-@/^0.5cm/"-2"
"-3"-@/_0.5cm/"-1"
"-2"-@/^0.5cm/"0"
"-1"-@/_0.5cm/"1"
"0"-@/^0.5cm/"2"
"1"-@/_0.5cm/"3"
"2"-@/^0.5cm/"4"
}}

$S = \{ 1, -1 \}$ $\Rightarrow$ \fbox{\xygraph{
!{<0cm,0cm>;<1cm,0cm>:<0cm,1cm>::}
!{(-3,0) }*+{\cdots}="-3"
!{(-2,0) }*+{\bullet_{-2}}="-2"
!{(-1,0) }*+{\bullet_{-1}}="-1"
!{(0,0) }*+{\bullet_0}="0"
!{(1,0) }*+{\bullet_1}="1"
!{(2,0) }*+{\bullet_2}="2"
!{(3,0) }*+{\bullet_3}="3"
!{(4,0) }*+{\cdots}="4"
"-3"-@/^0.5cm/"-2"
"-3"-@/_0.5cm/"-2"
"-2"-@/^0.5cm/"-1"
"-2"-@/_0.5cm/"-1"
"-1"-@/^0.5cm/"0"
"-1"-@/_0.5cm/"0"
"0"-@/^0.5cm/"1"
"0"-@/_0.5cm/"1"
"1"-@/^0.5cm/"2"
"1"-@/_0.5cm/"2"
"2"-@/^0.5cm/"3"
"2"-@/_0.5cm/"3"
"3"-@/^0.5cm/"4"
"3"-@/_0.5cm/"4"
}}

\item[3)] $G = \mathbb{Z}/n \mathbb{Z}$, $S = \{\bar{1}\}$ $\Rightarrow$
\fbox{\xygraph{
!{<0cm,0cm>;<1cm,0cm>:<0cm,1cm>::}
!{(0,0) }*+{\bullet_{\bar{0}}}="0"
!{(1,1) }*+{\bullet_{\bar{1}}}="1"
!{(2,1) }*+{\bullet_{\bar{2}}}="2"
!{(3,0) }*+{\bullet_{\bar{3}}}="3"
!{(2,-1) }*+{}="4"
!{(1,-1) }*+{\bullet_{n-1}}="5"
"0"-"1"
"1"-"2"
"2"-"3"
"3"-@{--}"4"
"4"-@{--}"5"
"5"-"0"
}}

\item[4)] $G = S_3$, $S = \{ \underbrace{(1 2 3)}_{\tau}, \underbrace{(1 2)}_{\sigma} \}$\\
$S_3 = \{ id, \tau, \tau^2, \sigma, \sigma \tau, \sigma \tau^2 \}$
\fbox{\xygraph{
!{<0cm,0cm>;<1cm,0cm>:<0cm,1cm>::}
!{(2,3) }*+{\bullet^{id}}="id"
!{(2,2) }*+{\bullet^{\sigma}}="s"
!{(1,1) }*+{\bullet_{\sigma\tau}}="st"
!{(0,0) }*+{\bullet_{\tau^2}}="t2"
!{(3,1) }*+{\bullet_{\sigma\tau^2}}="st2"
!{(4,0) }*+{\bullet_{\tau}}="t"
"id"-@/^0.2cm/"s"
"id"-@/_0.2cm/"s"
"s"-"st"
"st"-"st2"
"st2"-"s"
"t2"-@/^0.2cm/"st"
"t2"-@/_0.2cm/"st"
"st2"-@/^0.2cm/"t"
"st2"-@/_0.2cm/"t"
"id"-@/^0.4cm/"t"
"t"-@/^0.4cm/"t2"
"t2"-@/^0.4cm/"id"
}}

\end{enumerate}
\end{nnBsp}


\begin{Bem}
\label{3.3}
\begin{enumerate}
  \item \label{3.3a}
  $\Gamma(G,S)$ ist zusammenhängend $\Leftrightarrow S$ erzeugt $G$.
  \item Für jede Ecke $g \in G = E(\Gamma(G,S))$ ist $v(g) = 2 |S|$.
  \item $\Gamma(G,S)$ enthält Schleifen $\Leftrightarrow 1_G \in S$.
  \item $\Gamma(G,S)$ enthält Doppelkanten $\Leftrightarrow \exists \; s \in S:
  s \not=1, \; s^{-1} \in S$.\\
  \fbox{\xygraph{
  !{<0cm,0cm>;<1cm,0cm>:<0cm,1cm>::}
  !{(0,-1) }*+{\bullet_{g}}="g"
  !{(1,-1) }*+{\bullet_{gs}}="gs"
  "g":@/^0.5cm/"gs" ^s
  "gs":@/^0.5cm/"g" ^{s^{-1}}
  }}
  \item $\Gamma(G,S)$ enthält keine Dreifachkanten.
\end{enumerate}
\end{Bem}

\begin{Bew} 
\begin{enumerate}
  \item
  \begin{itemize}
    \item[$\Rightarrow$] Sei $g \in G$ beliebig, $w=(k_1, \ldots, k_n)$ Weg in
    $\Gamma(G,S)$ von $1$ nach $g$.\\
    Sei $k_i=(g_i, k_i, \varepsilon_i)$.
    Dann ist $g_1 = 1, g_2 = s_1^{\varepsilon_1},\; g_2 =
    s_1^{\varepsilon_1}s_2^{\varepsilon_2}, \ldots,\; g_n = s_1^{\varepsilon_1}
    \cdot \ldots \cdot s_{n-1}^{\varepsilon_{n-1}}$ und $g = t(k_n)=g_n
    s_n^{\varepsilon_n} = s_1^{\varepsilon_1} \cdot \ldots \cdot
    s_n^{\varepsilon_n} \Rightarrow g \in \langle S \rangle$.
    \item[$\Leftarrow$] Gleiche Überlegung rückwärts.
  \end{itemize}
  \stepcounter{enumi}
  \stepcounter{enumi}
  \item Sind $k_1, k_2$ Kanten mit $i(k_1) = i(k_2) = g$ und $t(k_1)=t(k_2) =
  gs$ für ein $s \in S$.\\
  Also ist $\OE \; k_1 = (g,s,1)$ und $k_2=(g,s',\varepsilon) \Rightarrow g s
  \overset{(*)}{=} g (s')^{\varepsilon} \Rightarrow s=s', \varepsilon=1$ oder
  $s'=s^{-1}, \varepsilon=-1 \WSpr$.
  \item Wäre $k_3 = (g,s'', \varepsilon'')$ noch eine Kante mit $i(k_3)=g, \;
  t(k_3)=gs$, so wäre wegen $(*)$ $gs = g(s'')^{\varepsilon''} \Rightarrow k_3 =
  k_1$ oder $k_3 = k_2$.
\end{enumerate}
\end{Bew}

\begin{Bem}
Sei $G$ eine Gruppe, $S \subseteq G$.\\
Dann operiert $G$ von \emp{links} auf $\Gamma(G,S)$ und diese Aktion ist treu.\\
\textbf{Genauer:} Für $g \in G$ sei $\varphi_G: \Gamma(G,S) \to \Gamma(G,S)$
gegeben durch $\varphi_g(g') = g g', \; \varphi_g(g', s, \varepsilon) = (g g', s,
\varepsilon)$\\
Endpunkt $t(\varphi_g(g', s, \varepsilon))=gg's, \; \varphi_g(t(g', s,
\varepsilon)) = \varphi_g(g's) = gg's$\\
Dann gilt:
\begin{enumerate}
  \item[(i)] $\varphi_g$ ist Automorphismus von $\Gamma(G,S)$
  \item[(ii)] $\varphi: G \to \Aut(\Gamma(G,S)), \; g \mapsto \varphi_g$ ist
  Gruppenhomomorphismus.
  \item[(iii)] $\varphi$ ist injektiv.
\end{enumerate}
\end{Bem}

% Do 02.11.2006

\begin{nnBsp}
% TODO Jede Menge Bildchen
\begin{enumerate}
  \item[(1)] $\Z$ operiert auf $\Gamma(\Z, \{1\})$ durch Translation auf
  $\Gamma(\Z, \{2\})$
  \item[(2)] $\Z/n\Z$ operiert auf $\Gamma(\Z/n\Z,\{\bar{1})$ durch Drehung.
  \item[(3)] $S_3$
\end{enumerate}
\end{nnBsp}


\begin{Prop} 
Sie $G$ eine Gruppe, $S \subseteq G,\; G_S \defeqr \langle S \rangle$ die von $S$
erzeugte Untergruppe von $G$.
\begin{enumerate}
  \item \label{3.5a}
  Die Zusammenhangskomponenten von $\Gamma(G,S)$ entsprechen bijektiv den
  Linksnebenklassen $g \cdot G_S, \; g \in G$.
  \item Sei $S' \subseteq S, \; H \defeqr \langle S' \rangle \subseteq G_S$
  Untergruppe, $\Gamma_H(G,S)$ der Graph, der aus $\Gamma(G,S)$ durch
  Kontraktion von $\Gamma(G,S')$ entsteht (d.h. jede Zusammenhangskomponente
  wird einzeln kontrahiert).
  Dann operiert $G$ auf $\Gamma_H(G,S)$.
  Es ist $E(\Gamma_H(G,S))=G/H$ (Menge der Linksnebenklasse).
\end{enumerate}
\end{Prop}

\begin{Bew}
\begin{enumerate}
  \item Die Zusammenhangskomponente $C_0$ von $\Gamma(G,S)$, die $1$ enthält,
  ist isomorph zu $\Gamma(G_S,S)$ (siehe \ref*{3.3}~\ref{3.3a} + Beweis).\\
  % TODO Bildchen
  Sei $C$ beliebige Zusammenhangskomponente, $g \in G$ eine Ecke von $C$.
  Dann ist $\varphi_g(1)=g$, also $\varphi_g(C_0) = C \Rightarrow E(C) = 
  (\varphi_g)_E(G_S) = g \cdot G_S$.
  \item Es ist $E(\Gamma_H(G,S))=G/H$ folgt aus \ref{3.5a}.\\
  \item Weiter ist $K(\Gamma_H(G,S)) = G \times (S \setminus S') \times
  \{\pm 1\}$ mit $i(g,s,1) = g \cdot h$ und $t(g,s,1)=g \cdot s \cdot
  H$.\\
  $g \in G$ operiert wie folgt: 
  % TODO ich werde aus meinen 
\end{enumerate}
\end{Bew}


\begin{Bem}
Sei $\Gamma$ ein Graph, $\bar{\Gamma}$ der Graph mit $E(\bar{\Gamma}) =
E(\Gamma), \; K(\bar{\Gamma})=\{(x,y) \in E(\Gamma) \times E(\Gamma): \exists\;k
\in K(\Gamma) \textrm{ mit }i(k)=x, t(k)=y\}$ mit $\overline{(x,y)} = (y,x),\;
i(x,y)=x,\;t(x,y)=y$.
\begin{enumerate}
  \item $p: \Gamma \to \bar{\Gamma}, p_0=\id, \; p_K=(k)=(i(k),t(k))$ ist
  Morphismus von Graphen.
  \item Es gibt einen eindeutig bestimmten Gruppenhomomorphismus $\rho:
  \Aut(\Gamma) \to \Aut(\Gamma')$, sodass für alle $\gamma \in \Aut(\Gamma)$ das
  folgende Diagramm kommutiert.
  % TODO Diagramm
  \begin{Bew}
  Definiere $\bar{\gamma} \in \Aut(\bar{\Gamma})$ durch $\bar{\gamma}_E =
  \gamma_E$ und $\bar{\gamma}_K$ wie folgt:
  Sei $(x,y) \in K(\bar{\Gamma}), k \in K(\Gamma)$ mit $i(k)=x, t(k)=y, \; 
  \bar{\gamma}(x,y)=(i(\gamma(k)),t(\gamma(k)))$. Setze $\rho(\gamma) \defeqr
  \bar{\gamma}$.
  % TODO Bildchen
  \end{Bew}
  \item $\Kern(\rho) = \{ \gamma \in \Aut(\Gamma): \gamma_E = \id, \gamma(l)=l
  \textrm{ für alle Schleifen } l \in K(\Gamma)\}$.
  \item $\bar{\bar{\Gamma}}$ sei $\bar{\Gamma}$ ohne Schleifen.\\
  Es gibt einen eindeutigen Gruppenhomomorphismus $\bar{\bar{\rho}}:
  \Aut(\Gamma) \to \Aut(\bar{\bar{\Gamma}})$ mit $\bar{\bar{\rho}}(\gamma)_E =
  \gamma_E$ und $\bar{\bar{\rho}}(\gamma)_K(k) = \rho(\gamma)_K(k)$ für alle $k
  \in K(\bar{\bar{\Gamma}})$.\\
  $\Kern(\bar{\bar{\rho}}) = \{ \gamma \in \Aut(\Gamma): \gamma_E = \id\}$.
\end{enumerate}
\end{Bem}


\begin{DefFolg} 
Sei $G$ eine Gruppe, $\Gamma$ ein Graph.
\begin{enumerate}
  \item Eine \emp{Aktion}\index{Aktion} von $G$ auf $\Gamma$ (oder $G$ operiert
  auf $\Gamma$) ist ein Gruppenhomomorphismus $\rho: G \to \Aut(\Gamma)$.
  \item Jede Aktion von $G$ auf $\Gamma$ induziert eine Aktion auf
  $\bar{\Gamma}$ und auf $\bar{\bar{\Gamma}}$.
  \item Eine Aktion $\rho: G \to \Aut(\Gamma)$ heißt \emp{treu}\index{Aktion!treu}
  (oder \emp{effektiv}\index{Aktion!effektiv}) wenn $\Kern(\rho) = \{1\}$,
  andernfalls heißt $\Kern(\rho)$
  \emp{Ineffektivitätskern}\index{Ineffektivitätskern} der Aktion $\rho$.
  \item Ist $\Gamma = \Gamma(G,S)$ eine Cayley-Graph, so operiert $G$ treu auf
  $\Gamma, \bar{\Gamma} \textrm{ und }\bar{\bar{\Gamma}}$.
\end{enumerate}
\end{DefFolg}

\end{document}