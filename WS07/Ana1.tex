\documentclass[12pt]{scrreprt}

\usepackage{schnaubelt}
\usepackage[utf8]{inputenc}
\usepackage{ngerman, url}
\usepackage{hyperref}

\author{Die Mitarbeiter von \url{http://mitschriebwiki.nomeata.de/}}
\title{Analysis I}
\makeindex

\begin{document}
\maketitle

\tableofcontents

\setcounter{chapter}{-1}
\chapter{Vorbemerkungen}
...

\chapter{Reelle und komplexe Zahlen}
\label{cha:zahlen}
Wir definieren die reellen Zahlen "`axiomatisch"', d.h.: Man legt in einer Definition die 
Eigenschaften der reellen Zahlen fest, die im folgenden verwendet werden dürfen.
Ausblick: \set{R} ist ein "`ordnungsvollständiger, geordneter Körper"'.

\section{Geordnete Körper}
\label{sec:zahlen.geordnete-koerper}
\begin{dfn*}
Sei $M$ eine nichtleere Menge. Eine Abbildung $\ast: M\times M \ra M, (x, y) 
\mapsto x \ast y$ heißt Verknüpfung auf $M$.
\end{dfn*}

\begin{dfn}
\label{dfn:zahlen.koerper}
Seien $K$ eine Menge, $0 \in K$, $1 \in K$ mit $0 \ne 1$, und "`$+$"' und "`$\cdot$"' Verknüpfungen auf 
$K$. Dann heißt $(K, 0, 1, +, \cdot)$ ein \emph{Körper}, wenn die folgenden Eigenschaften für alle 
$x$, $y$, $z \in K$ gelten:

\begin{enumerate}
\item Assoziativgesetze:
\[(x + y) + z = x + (y + z)\]
\[(x \cdot y) \cdot z = x \cdot (y \cdot z)\]

\item neutrale Elemente:
\[x + 0 = x, x \cdot 1 = x\]

\item inverse Elemente:
\begin{itemize}
  \item Zu jedem $x \in K$ existiert ein $a \in K$ mit $x + a = 0$.
  \item Zu jedem $x \in K \backslash \{0\}$ existiert ein $b \in K$ mit $x \cdot b = 1$.
\end{itemize}

\item Kommutativgesetze:
\[x + y = y + x,\ x \cdot y = y \cdot x\]

\item Distributivgesetz:
\[(x + y) \cdot z = (x \cdot z) + (y \cdot z)\]
\end{enumerate}

Man schreibt oft $K$ anstelle $(K, 0, 1, +, \cdot)$.
\end{dfn}

\begin{bsp*}
\begin{enumerate}
\item \set{Q} mit den üblichen 0, 1, $+$, $\cdot$ ist ein Körper.
\item \set{Z} ist kein Körper, da es kein $b \in$ \set{Z} gibt mit $2b = 1$.
\item Weitere Beispiele in linearer Algebra und Analysis I.
\end{enumerate}
\end{bsp*}

...
% ...
\subsection*{Eigenschaften von \set{R} und sup, inf}
\begin{satz}\label{satz:zahlen.sup-schranke}
\begin{enumerate}
\item ...\label{satz:zahlen.sup-schranke.a}
\item ...\label{satz:zahlen.sup-schranke.b}
\end{enumerate}
\end{satz}
\begin{proof}

\end{proof}

\begin{satz}\label{satz:zahlen.min-in-nat}

\end{satz}
\begin{proof}

\end{proof}

\begin{satz}\label{satz:zahlen.arch-ord}
\begin{enumerate}
\item ...\label{satz:zahlen.arch-ord.a}
\item ...\label{satz:zahlen.arch-ord.b}
\item ...\label{satz:zahlen.arch-ord.c}
\end{enumerate}
\end{satz}
\begin{proof}

\end{proof}

\begin{dfn*}

\end{dfn*}
\begin{dfn}\label{dfn:zahlen.maechtigigkeit}

\end{dfn}

\begin{bsp*}

\end{bsp*}
\paragraph{Beachte.}...

\begin{satz}\label{satz:zahlen.maechtigkeit-NQ}
\begin{enumerate}
\item ...\label{satz:zahlen.maechtigkeit-NQ.a}
\item ...\label{satz:zahlen.maechtigkeit-NQ.b}
\end{enumerate}
\end{satz}
\begin{proof}
\begin{enumerate}
\item ...
\item ...
\end{enumerate}
\end{proof}

\begin{dfn*}

\end{dfn*}

\begin{bsp*}

\end{bsp*}

\begin{satz}\label{satz:zahlen.sup-intervalle}
\begin{enumerate}
\item ...\label{satz:zahlen.sup-intervalle.a}
\item ...\label{satz:zahlen.sup-intervalle.b}
\end{enumerate}
\end{satz}
\begin{proof}
\begin{enumerate}
\item ...
\item ...
\end{enumerate}
\end{proof}

\subsection*{Potenzen mit rationalen Exponenten}
...

\begin{lem}\label{lem:zahlen.wurzel}

\end{lem}
\begin{proof}
\begin{enumerate}
\item ...
\item ...
\item ...
\end{enumerate}
\end{proof}

\paragraph{Folgerung.}...

\begin{dfn}\label{dfn:zahlen.wurzel}

\end{dfn}

\begin{satz}\label{satz:zahlen.potenzen}
\begin{enumerate}
\item ...\label{satz:zahlen.potenzen.a}
\item ...\label{satz:zahlen.potenzen.b}
\item ...\label{satz:zahlen.potenzen.c}
\item ...\label{satz:zahlen.potenzen.d}
\end{enumerate}
\end{satz}
\begin{proof}

\end{proof}

\section{Komplexe Zahlen}
\label{sec:zahlen.komplex}
...

\paragraph{Ansatz.}...

\begin{dfn*}

\end{dfn*}

\begin{dfn}\label{dfn:zahlen.komplex}

\end{dfn}

\begin{bsp*}

\end{bsp*}

\begin{satz}\label{satz:zahlen.kreg}
\begin{enumerate}
\item ...\label{satz:zahlen.kreg.a}
\item ...\label{satz:zahlen.kreg.b}
\item ...\label{satz:zahlen.kreg.c}
\item ...\label{satz:zahlen.kreg.d}
\item ...\label{satz:zahlen.kreg.e}
\item ...\label{satz:zahlen.kreg.f}
\item ...\label{satz:zahlen.kreg.g}
\item ...\label{satz:zahlen.kreg.h}
\end{enumerate}
\end{satz}
\begin{proof}
\begin{enumerate}
\item ...
\item ...
\item ...
\item ...
\item ...
\item ...
\item ...
\item ...
\end{enumerate}
\end{proof}

\chapter{Konvergenz von Folgen}
...

\chapter{Reihen}
...
\section{Konvergenzkriterien}
...

\section{Einige Vertiefungen/Vermischtes}
...

\subsection*{Umordnung von Reihen}
\begin{bsp}\label{bsp:reihen.umordn-altreihe}
Nach Bsp.~\ref{bsp:reihen.alt-reihe} konvergiert $\sum_{k=1}^\infty (-1)^{k+1} \frac{1}{k} = 1 - \frac{1}{2} + \frac{1}{3} - \frac{1}{4} \dotsb$.
Definiere rekursiv eine "`Umordnung"' $(b_k)_{k\ge 1}$ von $a_k = (-1)^{k+1}\frac{1}{k}$, \nat{k}.\\

\noindent Setze: $m = 1$: $b_1 \da 1$, $b_2 \da -\frac{1}{2}$ \folgt $b_1 + b_2 \ge \frac{1}{4}$\\
$m = 2$: $b_3 \da \frac{1}{3}$, $b_4 \da \frac{1}{5}$, $b_5 \da -\frac{1}{4} \folgt b_3 + b_4 + b_5 \ge \frac{1}{2}$\\

\noindent Definiert seien $b_{n_m} = -\frac{1}{2m}$ für ein \nat{m} mit $m\ge 2$, sowie 
\[b_{n_{m-1}+1} = \frac{1}{2 l_{m+1} +1}, \dotsc, b_{n_{m}-1} = \frac{1}{2l_m -1}\] 
für ein \nat{l_m}. Da $\sum_{k\ge l_m} \frac{1}{2k+1}$ divergiert (Übung) finden wir ein \nat{j}
\[b_{n_m +1} = \frac{1}{2l_m +1}, \dotsc, b_{n_m + j} = \frac{1}{2l+j},\] 
sodass: $b_{n_m+1} + \dotsb + b_{n_m + j} \ge \frac{1}{4} + \frac{1}{2m+2}.$\\

\noindent Setze $n_{m+1} = n_m + j +1$ und $b_{n_{m+1}} = -\frac{1}{2m+2}$
\folgt erhalten rekursiv $(b_k)_{k\ge 1}$ mit \[\sum_{k=1}^{n_m +1} b_n \ge (m+1)\frac{1}{4} \ra \infty,\quad (m\ra\infty)\]
\end{bsp}

\paragraph{Fazit.} $\sum_{k\ge 0}a_k$ divergiert, \emph{obwohl} die Reihe $\sum_{k=1}^\infty a_k$ mit
den gleichen Summanden konvergiert! Also: Hier gilt kein "`unendliches Kommutativgesetz"'.

\begin{dfn}\label{dfn:reihen.umordnung}
Sei $\sum_{k\ge 0}a_k$ eine Reihe und $\varphi: \set{N}_0 \ra \set{N}_0$ 
eine Bijektion. Setze $b_k = a_{\varphi(k)}$ für $k \in \set{N}_0$. Die Reihe $\sum_k b_k$
heißt Umordnung von $\sum_k a_k$.
\end{dfn}

\begin{satz}\label{satz:reihen.umordn-konv}
Sei $\sum_k a_k$ eine absolut konvergente Reihe. Dann konvergiert \emph{jede} Umordnung von $\sum_k a_k$
gegen den Wert $\sum^\infty_k a_k$.
\end{satz}
\begin{proof}
Sei $\ep > 0$ gegeben. Da $\sum \abs{a_k}$ konvergiert, gilt:
\begin{equation}\label{eqn:reihen.umordn-konv.a}\tag{$*$}
\exists N_\ep \in \set{N}: \forall n\ge N_\ep: \sum^n_{J = N_\ep+1} \abs{a_j} \le \ep\quad \text{nach 
Satz~\ref{satz:reihen.cauchy-krit}}
\end{equation}

\noindent Sei $\varphi: \set{N}_0 \ra \set{N}_0$ bijektiv. Sei $M_\ep = \max \left\{\varphi^{-1}(0), \dotsc, \varphi^{-1}(N_\ep)\right\}
\folgt \left\{0, \dotsc, N_\ep\right\} \subseteq \left\{\varphi(0), \dotsc, \varphi(M_\ep)\right\}$.

\bigskip 

\noindent Seien $n \ge N_\ep$, $m\ge M_\ep$. Setze \[D_{m,n} = \sum_{j=0}^m a_{\varphi(j)} + \sum_{j=0}^n (-a_j).\]
Als Summanden treten in $D_{m,n}$ nur $\pm a_k$ auf mit $k > N_\ep$. (alle $a_k$ mit $k \le N_\ep$ treten doppelt
auf und kürzen sich).

\[\folgt \abs{D_{m,n}} \le \sum_{k = N_\ep +1}^\infty \abs{a_k} \nach{\le}{(\ref{eqn:reihen.umordn-konv.a})} \ep
\quad \forall n\ge N_\ep, m\ge M_\ep\]

Da $\sum_{j=0}^\infty a_j$ existiert, folgt mit \ninf{} und Satz \ref{satz:konv.grenzw-ordn}, dass:
\[\exists\lim_{\ninf}\abs{D_{m,n}} = \abs{\sum_{j=0}^m a_{\varphi(j)} - \sum_{j=0}^\infty a_j} \le \ep, \forall m\ge M_\ep\]
Das ist die Behauptung.
\end{proof}

\subsection*{Cauchyprodukte}
Frage: Wie multipliziert man Reihen?

\begin{equation}\label{eqn:reihen.mult-rwert}
\left(\sum_{j=0}^\infty a_j\right)\left(\sum_{k=0}^\infty a_k\right) = \lim_{\ninf}\left(\sum_{j=0}^n a_j\right) 
\cdot \lim_{\ninf}\left(\sum_{k=0}^n a_k\right)
\end{equation}
...

Schema für Summanden $a_jb_k$:
...

Setze $Q_n = \{0, \dotsc, n\}^n$, $D_n = \{(j, k) \in Q_n, k+j \le n\}$.
Summiere $A_nB_n$ "`diagonal"', das heißt bilde zuerst

\begin{equation}\label{eqn:reihen.diag-summe}
c_n = \sum_{l=0}^n a_lb_{n-l}, n \in \set{N}
\end{equation}

$c_n = $ "`Summe über $a_jb_k$ mit $j+k = n$.

\begin{satz}\label{satz:reihen.cauchy-prod}
Seien $\sum_k a_k$, $\sum_k b_k$ absolut konvergente Reihen. Seien $c_n (n\in\set{N})$ in (\ref{eqn:reihen.diag-summe}) definiert.
Dann konvergiert $\sum_{n\ge0} c_n$ absolut und es gilt:

\begin{equation}\label{eqn:reihen.cauchy-prod}
\left(\sum_{j=0}^\infty a_j\right)\left(\sum_{k=0}^\infty a_k\right) = \sum_{n=0}^\infty c_n = \sum_{n=0}^\infty \sum_{j=0}^n a_jb_{n-j}
\end{equation}
\end{satz}

\begin{bem*}
Satz ist (im Allgemeinen) falsch für konvergente, nicht absolut konvergente Reihen (siehe Übung).
\end{bem*}

\begin{proof}
Seien $A_n$, $B_n$ aus (\ref{eqn:reihen.mult-rwert}), $A_n^* = \sum_{j=0}^n \abs{a_j}$, $B_n^* = \sum_{k=0}^n \abs{b_k}$,
$C_n = \sum_{j=0}^n c_j$. Nach Vorraussetzung $\exists A^* = \sum_{j=0}^\infty \abs{a_j}, B^* = \sum_{k=0}^\infty \abs{b_k}$.
Dann:

\begin{align*}
\abs{A_nB_n - C_n} = \Bigg\lvert\sum_{(j,k)\in Q_n\backslash D_n} a_jb_k\Bigg\lvert &\le 
	\sum_{(j,k)\in Q_n\backslash Q_{(\frac{n}{2})}} \abs{a_j}\abs{b_k} \\
	&= \sum_{(j,k)\in Q_n} \abs{a_j}\abs{b_k} - \sum_{(j,k)\in Q_{(\frac{n}{2})}} \abs{a_j}\abs{b_k}\\
	&= A_n^*B_n^* - A_{(\frac{n}{2})}^*B_{(\frac{n}{2})}^*
	%& \ra A^*B^*\text{ nach Satz~\ref{satz:konv.greg}} &\ra A^*B^*\quad(\ninf) %TODO
\end{align*}
\[\folgt \exists\lim_{\ninf} \abs{A_nB_n - C_n} = 0\]

Da $A_nB_n \ra AB (\ninf)$, folgt $\exists\sum_{n=0}^\infty C_n - AB \folgt$ (\ref{eqn:reihen.cauchy-prod}).
Ferner: \[\sum_{n=0}^N \abs{c_n} \nach{\le}{(\ref{eqn:reihen.diag-summe})}
\sum_{n=0}^N\sum_{j=0}^n \abs{a_j}\abs{b_{n-j}} \le %s.o.
A_N^*B_N^* \le A^*B^*\] für alle \nat{N}. Nach Satz \ref{satz:reihen.beschr-gwert} folgt die absolute Konvergenz von $\sum c_n$
\end{proof}

\begin{bsp}[Exponentialreihe]\label{bsp:reihen.expon-reihe}
Sei $z$, \kmplx{w}, $\exp(z) \da \sum_{k=0}^\infty \frac{z^k}{k!}$.
Die Reihe konvergiert absolut nach Bsp.~\ref{bsp:reihen.quot-krit} ($\forall \kmplx{z}$).
Beachte: $\exp(0) = 1$, $\exp(1) = \mathrm{e}$ (Bsp.~\ref{bsp:reihen.wurzel-krit})\\

\noindent\emph{Behauptung:}
\begin{enumerate}
\item $\exp(z+w) = \exp(z)\exp(w)$\label{bsp:reihen.expon-reihe.a}
\item $\exp(z) \neq 0, \exp(-z) = \frac{1}{\exp(z)}$\label{bsp:reihen.expon-reihe.b}
\item Sei $p\in\set{Q}\colon\exp(p) = \mathrm{e}^p$\label{bsp:reihen.expon-reihe.c}
\end{enumerate}
\end{bsp}
\begin{proof}
\begin{enumerate}
\item \[\exp(z)\exp(w) \nach{=}{Satz \ref{satz:reihen.cauchy-prod}} \sum_{n=0}^\infty\sum_{j=0}^\infty \frac{z^j}{j!} \frac{w^{n-j}}{(n-j)!} \frac{n!}{n!}
= \sum_{n=0}^\infty \frac{1}{n!} \underbrace{\sum_{j=0}^n \binom{n}{j}z^j w^{n-j}}_{\text{= Bsp.~\ref{bsp:vor.binom}: $(z+w)^n$}} 
= \exp(z+w)\]
\item $1 = \exp(0) = \exp(z-z) \nach{=}{a)} \exp(z)\exp(-z) \folgt$ b)
\item Sei $p = \frac{m}{n}$, $m\in\set{Z}$. \nat{n}. Dann gilt für $m > 0$ 
\[\exp(p)^n = \underbrace{\exp(p)\cdot\dotsm\cdot\exp(p)}_{\text{$n$-mal}} \nach{=}{a)} \exp(\underbrace{np}_m)
= \exp(\underbrace{1+\dotsb+1}_{\text{$m$-mal}}) = \exp(1)^m =\mathrm{e}^m\]
$\folgt \exp(p) = \mathrm{e}^{\frac{m}{n}}$. Fall $m < 0$ mit b).
\end{enumerate}
\end{proof}

\section{Potenzreihe}
\label{sec:reihen.potenzreihe}
\begin{dfn}\label{dfn:reihen.potenzreihe}
Es sei $(a_k)_{k\ge 0}$ gegeben. Für \kmplx{z} heißt $\sum_{k\ge 0} a_kz^k$ \emph{Potenzreihe}.
\end{dfn}

\begin{bem*}
Sei $D$ die Menge der \kmplx{z}, sodass die Potenzreihe konvergiert, dann ist 
$f: D \ra \set{C}$, $f(z) = \sum_{k=0}^\infty a_kz^k$ eine Funktion. Es gilt stets
$0\in D$, $f(0)=a_0$. (Man setzt $0^0 \da 1$)
\end{bem*}

\begin{dfn}\label{dfn:reihen.konvradius}
Der \emph{Konvergenzradius} $\varrho$ von $\sum a_kz^k$ ist gegeben durch:

\[\varrho = 
\begin{cases}
	\frac{1}{\lim\limits_{\kinf}\sqrt[k]{\abs{a_k}}},	& \text{wenn } \left(\sqrt[k]{\abs{a_k}}\right) \text{beschränkt und keine NF},\\
	0,											& \text{wenn } \left(\sqrt[k]{\abs{a_k}}\right) \text{unbeschränkt},\\
	\infty,										& \text{wenn } \left(\sqrt[k]{\abs{a_k}}\right) \text{NF}.
\end{cases}\]
\end{dfn}

\begin{thm}\label{thm:reihen.konvradius}
Sei $\varrho$ der Konvergenzradius von $\sum_{k \ge 0} a_kz^k$. Dann gelten:
\begin{enumerate}
\item $0 < \varrho < \infty$, dann konvergiert $\sum a_kz^k$ absolut für $\abs{z} < \varrho$
und divergiert für $\abs{z} > \varrho$, wobei \kmplx{z}.\label{thm:reihen.konvradius.a}
\item Wenn $\varrho = 0$, dann divergiert $\sum a_kz^k$ für alle $z\in\set{C}\backslash\{0\}$\label{thm:reihen.konvradius.b}
\item Wenn $\varrho = \infty$, dann konvergiert $\sum a_kz^k$ absolut für alle \kmplx{z}\label{thm:reihen.konvradius.c}
\end{enumerate}
Also: $\varrho = \sup \left\{r \ge 0\colon \sum a_kz^k\text{ konvergiert }\forall \kmplx{z}\text{ mit }\abs{z} \le r\right\}$ 
(dabei ist $\sup \set{R}_+ \da \infty$).
\end{thm}
\begin{proof}
Es gilt $\sqrt[k]{\abs{a_kz^k}} = \left(\abs{a_k}\abs{z}^k\right)^{\frac{1}{k}} = \abs{z}\sqrt[k]{\abs{a_k}} \ad b_k$
\begin{enumerate}
\item $\ls\limits_{\kinf} b_k \nach{=}{\ref{satz:konv.limsupinf-reg.e}} \abs{z}\ls\limits_{\kinf}\sqrt[k]{\abs{a_k}}$.
Nach Wurzelkriterium: \[\folgt \begin{cases}
	\abs{z} < \varrho \gdw \ls b_k < 1 \folgt \sum a_kz^k\text{ konvergiert absolut}\\
	\abs{z} > \varrho \gdw \ls b_k > 1 \folgt \sum a_kz^k\text{ divergiert}
\end{cases}\]
% FIXME: erst c, dann b
\item $\ls_{\kinf} b_k = \lim_{\kinf} b_k = 0 \folgt \sum a_kz^k$ konvergiert absolut 
$\forall \kmplx{z}$ nach Wurzelkriterium
\item Falls $\abs{z} \neq 0$, dann ist $(b_k)$ unbeschränkt $\folgt \left(b_k^k\right)$ ist unbeschränkt
$\folgt (a_kz^k)$ ist keine NF. Nach Kor.~\ref{kor:reihen.konv-nf} $\folgt \sum a_kz^k$ divergiert
\end{enumerate}
\end{proof}

\begin{bsp}\label{bsp:reihen.konvradius}
\begin{enumerate}
\item Polynome $p(z) = a_0 + a_1z + \dotsb + a_nz^n$ (\kmplx{z}), wobei $a_1, \dotsc, a_n$ gegeben.
Setze $a_j = 0$ für $j > n$ $\folgt \varrho = \infty \folgt$ konvergiert $\forall \kmplx{z}$

\item $\exp(z) = \sum_{k=0}^\infty \frac{z^k}{k!}$ konvergiert $\forall \kmplx{z}$ nach Bsp.~\ref{bsp:reihen.quot-krit}.
Nach Thm.~\ref{thm:reihen.konvradius} gilt: 
\begin{equation}\label{eqn:reihen.wurzelkrit-exp}
0 = \lim_{\kinf} \sqrt[k]{\frac{1}{k!}} = \lim_{\kinf} \frac{1}{\sqrt[k]{k!}}
\end{equation}
da $\varrho = \infty$ und $a_k = \frac{1}{k!}$

\item Geometrische Reihe $\sum_{k \ge 0} z^k$. Hier ist $a_k = 1 \folgt \varrho = 1$. Genauer: Bsp.~\ref{bsp:reihen.reihe}
liefert $\exists \sum_{k=0}^\infty z^k = \frac{1}{1-z}$ für $\abs{z} < 1$. Bsp.~\ref{bsp:reihen.konv-nf} $\folgt$ Divergenz
wenn $\abs{z} \ge 1$.

\item Sei $a_k = k!$. Nach (\ref{eqn:reihen.wurzelkrit-exp}) $\forall \nat{n}\,\exists \nat{K_n}\colon \frac{1}{\sqrt[k]{k!}} \le \frac{1}{n} 
(\forall k \ge K_n)$ ...

\item Betrachte $\sum_{k \ge 1} \frac{1}{k} (2z)^k$, d.\,h. $a_k = \frac{2^k}{k}$. 
Damit $\sqrt[k]{\abs{a_k}} = \frac{2}{\sqrt[k]{k}} \ra 2$ (\kinf, Üb.)
$\folgt \varrho = \frac{1}{2}$. Also absolute Konvergenz für $\abs{z} < \frac{1}{2}$, Divergenz für $\abs{z} > \frac{1}{2}$.
Hier gilt Konvergenz für $z = -\frac{1}{2}$, Divergenz für $z = \frac{1}{2}$ (nach Bsp.~\ref{bsp:reihen.alt-reihe} und 
\ref{bsp:reihen.reihe})
\end{enumerate}
\end{bsp}

\begin{bem*}
Im Fall $\abs{z} = \varrho \in (0, \infty)$ ist keine allgemeine Aussage möglich.
\end{bem*}

\begin{satz}\label{bsp:reihen.potreihe-reg}
Es seien $\sum a_kz^k, \sum b_kz^k$ Potenzreihen mit Konvergenzradius $\varrho_a, \varrho_b > 0$ und $\alpha, \beta \in \set{C}$.
Dann gelten für \kmplx{z} mit $\abs{z} < \min \{\varrho_a, \varrho_b\}$ (wobei $\min \{x, \infty\} = x$ für $\reell{x}$)
\begin{enumerate}
\item $\exists \sum\limits_{k=0}^\infty (\alpha a_k + \beta b_k) z^k = \alpha \sum\limits_{k=0}^\infty a_kz^k + \beta \sum\limits_{k=0}^\infty b_kz^k$\label{bsp:reihen.potreihe-reg.a}

\item $\exists \sum\limits_{n=0}^\infty \biggl(\sum\limits_{j=0}^n a_jb_{n-j}\biggr)z^n = \left(\sum\limits_{k=0}^\infty a_kz^k\right)\left(\sum\limits_{k=0}^\infty b_kz^k\right)$\label{bsp:reihen.potreihe-reg.b}
\end{enumerate}
\end{satz}
\begin{proof}
\begin{enumerate}
\item Thm.~\ref{thm:reihen.konvradius} und Satz \ref{satz:reihen.addit-reihenwerte}
\item Thm.~\ref{thm:reihen.konvradius} und Satz \ref{satz:reihen.cauchy-prod}, wobei in (\ref{eqn:reihen.diag-summe}) gilt:
\[c_n = \sum_{j=0}^n a_jz^jb_{n-j}z^{n-j} = z^n \sum_{j=0}^n a_j b_{n-j}\]
\end{enumerate}
\end{proof}

\begin{bsp}[Sinus und Cosinus]\label{bsp:reihen.sin-cos}
Für \kmplx{z} konvergieren absolut:
\[\sin(z) \da \sum_{k=0}^\infty \frac{(-1)^k}{(2k+1)!} z^{2k+1},\quad
\cos(z) \da \sum_{k=0}^\infty \frac{(-1)^k}{(2k)!} z^{2k}\]

\noindent Das sind Potenzreihen mit Koeffizienten
\[\sin\colon a_n = \begin{cases}
\frac{(-1)^k}{(2k+1)!}, & n= 2k+1 \text{ ungerade}\\
0, & \text{$n$ gerade}
\end{cases},\quad
\cos\colon a_n = \begin{cases}
0, & \text{$n$ ungerade}\\
\frac{(-1)^k}{(2k)!}, & n= 2k \text{ gerade}
\end{cases}\]
\end{bsp}
\begin{proof}
Zeige $\varrho = \infty$. \[\sin: \sqrt[k]{\abs{a_k}} = 
\begin{cases}
0, & n\text{ gerade}\\
\frac{1}{\sqrt[n]{n!}}, & n\text{ ungerade}
\end{cases} \tonach{(\ref{eqn:reihen.wurzelkrit-exp})} 0,\quad\ninf\]
cos genauso.
\end{proof}

\noindent Aus Reihen folgt:
\begin{equation}\label{eqn:reihen.reell-sincos}
\forall \reell{x}\colon \cos x, \sin x \in\set{R}
\end{equation}
\begin{equation}\label{eqn:reihen.sincos-reg}
\forall \kmplx{z}\colon \cos(-z) = \cos z,\quad \sin(-z) = -\sin z
\end{equation}

\begin{satz}
Sei \kmplx{z}. Dann gelten:
\[\text{Euler: }\exp(iz) = \cos(z) + i\sin(z), \quad (\cos z)^2 + (\sin z)^2 = 1\] % FIXME, Euler nur erste Formel
\begin{equation}\label{eqn:reihen....}
\cos z = \frac{1}{2}(\exp(iz) + \exp(-iz)),\quad \sin z = \frac{1}{2i}(\exp(iz) - \exp(-iz))
\end{equation}
Für \reell{x} folgt mit (\ref{eqn:reihen.reell-sincos}) $\Re \exp(ix) = \cos x$, $\Im \exp(iz) = \sin x$, 
$\abs{\exp(iz)} = 1$, $\abs{\cos x}$, $\abs{\sin x} \le 1$.
\end{satz}
\begin{proof}
Es gilt: $\exp(iz) = $
\end{proof}

\begin{kor}
...
\end{kor}

\chapter{Stetige Funktionen}
\label{cha:fnkt}
Ab jetzt wird (fast) immer in $\set{R}$ gerrechnet, insbesondere $B(x, r) = (x-r, x+r)$, $\overline{B}(x, r) = [x-r, x+r]$.
Stets sei $D \neq \emptyset$.

\section{Grenzwerte stetiger Funktionen}
\label{sec:fnkt.grenzw-stetigk}
\begin{dfn}\label{dfn:fnkt.abschluss}
Sei $D \subseteq \set{R}$. Dann heißt die Menge $\abg{D} \da \{x\in\set{R}\colon \exists x_n\in D$ (\nat{n})
mit $ x_n \ra x$, $\ninf\}$ der \emph{Abschluss} von $D$. $D$ heißt \emph{abgeschlossen} (abg.) falls $D = \abg{D}$.
\end{dfn}

\begin{bem*}
Es gilt $D \subseteq \abg{D}$ (Betrachte für $x\in D$ die Folge $(x_n)_{n\ge 1} = (x)_{n\ge 1}$)
\end{bem*}

\begin{bsp*}
Sei $D = (0, 1]$, dann ist $\abg{D} = [0, 1]$
\end{bsp*}
\begin{proof}
Es gilt $0\in\abg{D}$, da $\frac{1}{n}\in D$, $\frac{1}{n} \ra 0$ \nat{n} \folgt $[0, 1] \subseteq \abg{D}$. 
Umgekehrt: Sei ...
\end{proof}

\paragraph{Ebenso:}
\begin{enumerate}
\item $\abg{\set{R}\backslash\{0\}} = \set{R}$
\item Abgeschlossene Intervalle im Sinne von Def.~\ref{dfn:zahlen.intervalle} sind abgeschlossen im Sinne 
von Def.~\ref{dfn:fnkt.abschluss}, Bsp: $\abg{[0, 1]} = [0, 1]$.
\end{enumerate}

\begin{dfn}\label{dfn:fnkt.grenzw-fnkt}
Sei $D \subseteq \set{R}$, $x_0 \in \abg{D}$, \reell{y_0}. Eine Funktion $f: D \ra \set{R}$ \emph{konvergiert}
gegen den \emph{Grenzwert} $y_0$, wenn für \emph{jede} Folge $(x_n)_{n \ge 1} \subseteq D$ mit $x_n \ra x_0$ (\ninf) gilt:
$f(x_n) \ra y_0$ (\ninf). Man schreibt dann $y_0 = \lim_{x\ra x_0} f(x)$ oder $f(x) \ra y_0$ für $x \ra x_0$.
Wenn man zusätzlich $x_n < x_0$, bzw. $x_n > x_0$ ($\forall \nat{n}$) fordert, dann spricht man vom \emph{links-}, 
bzw. \emph{rechtsseitigen Grenzwert} und schreibt $y_0 = \lim_{x\ra x_0^-} f(x)$, bzw. $y_0 = \lim_{x\ra x_0^+} f(x)$.
\end{dfn}

\begin{bsp}\label{bsp:fnkt.grenzw-fnkt}
\begin{enumerate}
\item Sei $D = \set{R}$, $f(x) = x^2 +3$, \reell{x_0}. Sei \reell{x_n}, $x_n \ra x_0$. Dann $f(x_n) = x_n^2 + 3 \ra x_0^2 + 3$
(\ninf) nach Satz~\ref{satz:konv.greg} \folgt $\lim_{x\ra x_0} f(x) = x_0^2 + 3$
\item Sei $M \subseteq \set{R}$. Setze \[\charfnkt{M}(x) = 
\begin{cases}
1, & x\in M\\
0, & x\in \set{R}\backslash M
\end{cases} \tag{charakteristische Funktion}\]
...
\item Sei $D = \set{R}\backslash\{0\}$, $f(x) = \frac{1}{x}$, $x\in D$. Dann: $\lim_{x\ra 0} f(x)$ existiert nicht, 
da $\frac{1}{n} \ra 0$, aber $f(\frac{1}{n}) = n$ divergiert (\ninf).
\end{enumerate}
\end{bsp}

\begin{satz}[$\ep$-$\delta$-Charakterisierung]\label{satz:fnkt.ep-delta}
Sei $D \subseteq \set{R}$, $x_0\in\abg{D}$, $f: D \ra \set{R}$, \reell{y_0}. Dann sind äquivalent:
\begin{enumerate}
\item $\exists \lim_{x\ra x_0} f(x) = y_0$ \label{satz:fnkt.ep-delta.a}
\item $\forall\ep > 0\,\exists\delta_\ep > 0\,\forall x\in D\cap \overline{B}(x_0, \delta_\ep)$ gilt: $\abs{f(x) - y_0} \le \ep$ \label{satz:fnkt.ep-delta.b}
% FIXME: Bild
\end{enumerate}
\end{satz}
\begin{proof}
\begin{enumerate}
\item Es gelte \ref{satz:fnkt.ep-delta.b}). Sei $x_n\in D$ (\nat{n}) mit $x_n \ra x_0$ beliebig gegeben (\ninf). Sei $\ep > 0$. Wähle $\delta_\ep > 0$
aus \ref{satz:fnkt.ep-delta.b}). Dann $\exists N_\ep \in\set{N}$ mit $\abs{x_n - x_0} \le \delta_\ep$ für alle $n \ge N_\ep$. \ref{satz:fnkt.ep-delta.b}) liefert:
$\abs{f(x_n) - y_0} \le \ep$ ($\forall n \ge N_\ep$) \folgt $f(x_n) \ra y_0$, \ninf{} \folgt \ref{satz:fnkt.ep-delta.a})
\item Es gelte \ref{satz:fnkt.ep-delta.a}). Annahme: \ref{satz:fnkt.ep-delta.b}) sei falsch. Daraus folgt mit $\delta = \frac{1}{n}$:
$\exists \ep_\delta > 0\,\forall n\in\set{N}\,\exists x_n \in D$ mit $\abs{x_0 - x_n} \le \frac{1}{n}$ und $\abs{f(x) - y_0} > \ep_0$, 
d.\,h. $x_n \ra x_0$ (Satz~\ref{satz:konv.grenzw-ordn}) und $f(x_n) \not\ra y_0$ (\ninf) ...\ref{satz:fnkt.ep-delta.a}) \folgt \ref{satz:fnkt.ep-delta.b} %FIXME: Lightning
\end{enumerate}
\end{proof}

\begin{satz}\label{satz:fnkt.grenzw-reg}
Es seien $D \subseteq \set{R}$, $x_0\in\abg{D}$, $f, g: D\ra\set{R}$, $y_0, z_0\in\set{R}$, sodass $\exists \lim_{x\ra x_0}f(x) = y_0$,
$\exists \lim_{x\ra x_0}g(x) = z_0$. Dann gelten:
\begin{enumerate}
\item $\exists \lim\limits_{x\ra x_0} (f(x) + g(x)) = y_0 + z_0$
\item $\exists \lim\limits_{x\ra x_0} f(x)g(x) = y_0z_0$
\item $\exists \lim\limits_{x\ra x_0} \abs{f(x)} = \abs{y_0}$
\item Sei zusätzlich $y_0 \neq 0$. Dann $\exists r > 0$, sodass $\abs{f(x)} \ge \frac{\abs{y_0}}{2} > 0$ 
für alle $x\in D$ mit $\abs{x - x_0} \le r$. Ferner $\exists \lim\limits_{x\ra x_0} \frac{1}{f(x)} = \frac{1}{y_0}$
\item Sei zusätzlich $f(x) \le g(x)$ für alle $x\in D$. Dann gilt $x_0 \le z_0$. (Entsprechendes gilt für $\lim\limits_{x\ra {x_0}^\pm}$)
\end{enumerate}
\end{satz}
\begin{proof}
\begin{enumerate} % FIXME: Nur c und d
\item ...
\item ...
\end{enumerate}

\noindent a), b), e) gehen genauso mit Satz~\ref{satz:konv.greg} und Satz~\ref{satz:konv.grenzw-ordn}.
\end{proof}

\subsection*{Uneigentliche Grenzwerte}
\begin{dfn*}
Erweiterte Zahlengerade $\overline{\set{R}} = \set{R} \cup \{-\infty, +\infty\}$ (man schreibt
oft $\infty$ statt $+\infty$). Ordnung: $-\infty < x < +\infty$ ($\forall \reell{x}$), $\abs{\pm\infty} \da +\infty$
\end{dfn*}

\begin{dfn}\label{dfn:fnkt.uneig-grenzw}
Man schreibt $\lim\limits_{\ninf} x_n = +\infty\; (-\infty)$ für \reell{x_n}, \nat{n}, falls:
\[\forall \nat{K}\,\exists\nat{N_K}\,\forall n\ge N_K\colon x_n \ge K\; (x_n \le -K)\]
Damit $n^2 \ra \infty$, $-n^3 \ra -\infty$ (\ninf). \emph{Beachte:} $\left((-1)^n\right)$ divergiert nach wie vor.
\end{dfn}

\begin{bem}\label{bem:fnkt.uneig-grenzw}
\begin{enumerate}
\item Wenn $x_n\ra\infty$ oder $x_n\ra -\infty$, dann $\frac{1}{x_n}\ra 0$ (\ninf). (Beachte, nach 
Def.~\ref{dfn:fnkt.uneig-grenzw} gilt: $x_n \neq 0$, $n \ge N_1$)\label{bem:fnkt.uneig-grenzw.a}
\item Wenn $x_n \ra 0$ und ein \nat{n_0} existiert mit $x_n > 0$ für alle $n \ge n_0$, dann geht $\frac{1}{x_n}\ra +\infty$\label{bem:fnkt.uneig-grenzw.b}
\item Wenn $x_n \ra 0$, $x_n < 0$ ($\forall n \ge n_0$), dann $\frac{1}{x_n}\ra -\infty$\label{bem:fnkt.uneig-grenzw.c}
\end{enumerate}
\end{bem}
\begin{proof}
\begin{enumerate}
\item Sei $x_n\ra +\infty$ oder $x_n\ra -\infty$ (\ninf). Nach Def.~\ref{dfn:fnkt.uneig-grenzw} gilt
\[\forall \nat{K}\,\exists\nat{N_K}\,\forall n\ge N_K\colon \abs{x_n} \ge K \gdw 0 < \frac{1}{\abs{x_n}} \le \frac{1}{K} \ad \ep,\]
d.\,h. $\frac{1}{x_n} \ra 0$, \ninf.
\end{enumerate}

b), c) zeigt man ähnlich.
\end{proof}

\noindent In Anbetracht von \ref{bem:fnkt.uneig-grenzw}.\ref{bem:fnkt.uneig-grenzw.a}) %FIXME 4.7a) anstatt 4.7.1)
schreibt man 
\begin{equation}\label{eqn:fnkt.x-div-infty}
\frac{x}{\pm\infty} = 0,\;\reell{x}
\end{equation}
(damit gilt ...)

\begin{bsp*}
...
\end{bsp*}

\section{Eigenschaften stetiger Funktionen}
...


% Nach den Weihnachtsferien

\begin{dfn}\label{dfn:fnkt.} %Def. 4.26

\end{dfn}

\begin{bsp*}
\begin{enumerate}
\item ...
\item ...
\end{enumerate}
\end{bsp*}

\begin{bem}\label{}

\end{bem}
\begin{proof}

\end{proof}

\begin{thm}\label{}

\end{thm}
\begin{proof}

\end{proof}

\begin{bsp}\label{}

\end{bsp}

\section{Exponentialfunktion und ihre Verwandtschaft}
\label{}
...

\begin{dfn}\label{}

\end{dfn}

\begin{dfn}\label{}

\end{dfn}

\begin{bem}\label{}
...
\begin{enumerate}
\item ...
\item ...
\item ...
\item ...
\item ...
\item ...
\end{enumerate}
\end{bem}

\subsection*{Trigonometrische Funktionen}
...

\begin{satz}\label{}

\end{satz}

\begin{dfn*}

\end{dfn*}

...

\begin{dfn}\label{}

\end{dfn}

\begin{dfn}\label{}

\end{dfn}

\begin{bsp}\label{}

\end{bsp}

\chapter{Differentialrechnung}
\label{cha:diff}

\section{Rechenregeln}
\label{}

\begin{dfn}\label{}

\end{dfn}

\begin{bem*}
...
\begin{enumerate}
\item ...
\item ...
\item ...
\item ...
\end{enumerate}
\end{bem*}

\begin{bsp}\label{}
\begin{enumerate}
\item ...
\item ...
\item ...
\end{enumerate}
\end{bsp}

\begin{satz}\label{}

\end{satz}
\begin{proof}

\end{proof}

\begin{satz}\label{}
...
\begin{enumerate}
\item ...
\item ...
\item ...
\end{enumerate}
\end{satz}
\begin{proof}
\begin{enumerate}
\item ...
\item ...
\item ...
\end{enumerate}
\end{proof}

\begin{kor}\label{}

\end{kor}

\begin{satz}\label{}

\end{satz}
\begin{proof}

\end{proof}

\begin{satz}\label{}

\end{satz}
\begin{bem*}

\end{bem*}
\begin{proof}

\end{proof}

\begin{bsp}\label{}
\begin{enumerate}
\item ...
\item ...
\end{enumerate}
\end{bsp}

\begin{thm}\label{}

\end{thm}
\begin{proof}
\begin{enumerate}
\item ...
\item ...
\end{enumerate}
\end{proof}

\begin{bsp}\label{}
\begin{enumerate}
\item ...
\item ...
\item ...
\item ...
\item ...
\end{enumerate}
\end{bsp}

\begin{bsp}\label{}

\end{bsp}

\begin{dfn}\label{}

\end{dfn}
\begin{bem*}

\end{bem*}

\section{Qualitative Eigenschaften differenzierbarer Funktionen}
\label{}

\begin{dfn}\label{}

\end{dfn}

\begin{satz}\label{}
\begin{enumerate}
\item ...
\item ...
\item ...
\end{enumerate}
\end{satz}
\begin{proof}

\end{proof}
\begin{bem*}

\end{bem*}
\begin{bsp*}

\end{bsp*}
\begin{proof}

\end{proof}

\begin{thm}\label{}

\end{thm}
\begin{proof}

\end{proof}

\begin{satz}\label{}

\end{satz}
\begin{proof}

\end{proof}

\begin{dfn}\label{}

\end{dfn}

\begin{bem}\label{}
\begin{enumerate}
\item ...
\item ...
\item ...
\end{enumerate}
\end{bem}

\begin{kor}\label{}

\end{kor}
\begin{proof}

\end{proof}

\begin{satz}\label{}
\begin{enumerate}
\item ...
\item ...
\end{enumerate}
\end{satz}
\begin{bem*}

\end{bem*}
\begin{proof}

\end{proof}

\begin{bsp}\label{}

\end{bsp}
\begin{proof}

\end{proof}

\begin{kor}\label{}
\begin{enumerate}
\item ...
\item ...
\end{enumerate}
\end{kor}
\begin{bem*}

\end{bem*}
\begin{proof}
\begin{enumerate}
\item ...
\item ...
\end{enumerate}
\end{proof}

\begin{dfn}\label{}

\end{dfn}
\begin{bem*}

\end{bem*}

\begin{satz}\label{}

\end{satz}
\begin{bsp}\label{}
\begin{enumerate}
\item ...
\end{enumerate}
\end{bsp}
\begin{proof}

\end{proof}

\begin{bsp}\label{}
\begin{enumerate}
\item ...
\end{enumerate}
\end{bsp}
\begin{proof}

\end{proof}

\begin{thm}\label{}
\begin{enumerate}
\item ...
\item ...
\end{enumerate}
\end{thm}
\begin{proof}

\end{proof}

\begin{bsp}\label{}
\begin{enumerate}
\item ...
\item ...
\item ...
\item ...
\end{enumerate}
\end{bsp}

\section{Der Satz von Taylor}
\label{}

\begin{thm}\label{}

\end{thm}
\begin{proof}

\end{proof}

\begin{dfn}\label{}

\end{dfn}

\begin{bem}\label{}
\begin{enumerate}
\item ...
\item ...
\end{enumerate}
\end{bem}

\begin{thm}\label{}
\begin{enumerate}
\item ...
\item ...
\item ...
\end{enumerate}
\end{thm}
\begin{bsp*}

\end{bsp*}
\begin{proof}

\end{proof}

\begin{dfn}\label{}

\end{dfn}

\begin{bsp}\label{}
\begin{enumerate}
\item ...
\item ...
\item ...
\end{enumerate}
\end{bsp}

\subsection*{Newton-Verfahren}

\begin{thm}\label{}

\end{thm}
\begin{proof}

\end{proof}

\begin{bsp}\label{}

\end{bsp}

\chapter{Integralrechnung}
\label{cha:int}

\section{Riemann-Integral}
\label{}

\begin{dfn}\label{}

\end{dfn}

\begin{lem}\label{}
\begin{enumerate}
\item ...
\item ...
\end{enumerate}
\end{lem}
\begin{proof}

\end{proof}

\begin{bsp}\label{}

\end{bsp}

\begin{bem}\label{}

\end{bem}

\begin{satz}\label{}

\end{satz}
\begin{proof}

\end{proof}

\begin{dfn*}

\end{dfn*}

\begin{satz}\label{}
\begin{enumerate}
\item ...
\item ...
\item ...
\item ...
\end{enumerate}
\end{satz}
\begin{proof}
\begin{enumerate}
\item ...
\item ...
\item ...
\item ...
\end{enumerate}
\end{proof}

\section{Hauptsatz der Differential- und Integralrechnung}
\label{}

\begin{dfn}\label{}

\end{dfn}

\begin{lem}\label{}

\end{lem}
\begin{proof}

\end{proof}

\begin{thm}\label{}
\begin{enumerate}
\item ...
\item ...
\end{enumerate}
\end{thm}
\begin{proof}
\begin{enumerate}
\item ...
\item ...
\end{enumerate}
\end{proof}

\begin{bem*}

\end{bem*}

\begin{bsp}\label{}
\begin{enumerate}
\item ...
\item ...
\end{enumerate}
\end{bsp}

\begin{bsp*}
\begin{enumerate}
\item ...
\item ...
\item ...
\item ...
\item ...
\end{enumerate}
\end{bsp*}

\begin{satz}\label{}

\end{satz}
\begin{proof}

\end{proof}

\begin{bsp}\label{}
\begin{enumerate}
\item ...
\item ...
\item ...
\item ...
\end{enumerate}
\end{bsp}

\begin{satz}\label{}

\end{satz}
\begin{proof}

\end{proof}

\begin{bsp}\label{}
\begin{enumerate}
\item ...
\item ...
\item ...
\item ...
\end{enumerate}
\end{bsp}

\subsection*{6.15 Integration rationaler Funktionen}
\label{}

\begin{enumerate}
\item ...
\item ...
\item ...
\item \begin{enumerate}
\item ...
\item ...
\item ...
\end{enumerate}
\end{enumerate}

\begin{bsp}\label{}
\begin{enumerate}
\item ...
\item ...
\end{enumerate}
\end{bsp}

\section{Skalare Differentialgleichungen erster Ordnung}
\label{}

\begin{bsp*}

\end{bsp*}

\begin{satz}\label{}

\end{satz}
\begin{proof}

\end{proof}

\begin{bsp}\label{}
\begin{enumerate}
\item ...
\item ...
\item ...
\end{enumerate}
\end{bsp}

\section{Uneigentliche Riemann-Integrale}
\label{}

\begin{dfn}\label{}
\begin{enumerate}
\item ...
\item ...
\end{enumerate}
\end{dfn}

\begin{bem}\label{}
\begin{enumerate}
\item ...
\item ...
\end{enumerate}
\end{bem}

\begin{bsp}\label{}
\begin{enumerate}
\item ...
\item ...
\item ...
\item ...
\end{enumerate}
\end{bsp}

\begin{satz}\label{}
\begin{enumerate}
\item ...
\item ...
\end{enumerate}
\end{satz}
\begin{bem*}

\end{bem*}
\begin{proof}
\begin{enumerate}
\item ...
\item ...
\end{enumerate}
\end{proof}

\begin{bsp}\label{}
\begin{enumerate}
\item ...
\item ...
\item ...
\end{enumerate}
\end{bsp}

\begin{bsp}\label{}

\end{bsp}

\begin{bsp}\label{}

\end{bsp}

\paragraph{Trapezregel.} ...


\end{document}