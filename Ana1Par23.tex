\documentclass{article}
\newcounter{chapter}
\usepackage{ana}
\usepackage{slashed}
\usepackage{wasysym}


\author{Pascal Maillard, Wenzel Jakob und Joachim Breitner}
\title{Das Riemann-Integral}
\setcounter{chapter}{23}

\setlength{\parindent}{0pt}
\setlength{\parskip}{2ex}

\begin{document}

\theoremstyle{nonumberbreak}
\newtheorem{merkregel}{Merkregel}

\maketitle

In diesem Paragraphen gilt stets: $a,b \in \MdR,\ a<b,\ I = [a,b]$ und $f: I \to \MdR$ sei \emph{beschränkt}. $m := \inf f(I),\ M := \sup f(I)$.

% das Sütterlin Z
\def\Z{\ensuremath{\mathfrak{Z}}}

\begin{definition}
$Z = \{x_0,x_1,\ldots,x_n\} \subseteq I$ heißt eine \begriff{Zerlegung} von $I :\equizu a = x_0 < x_1 < \ldots < x_n = b.$\\
$I_j := [x_{j-1};x_j],\ |I_j| = x_j-x_{j-1},\ m_j := \inf f(I_j),\ M_j := \sup f(I_j)\ (j = 1,\ldots,n)$

Dann gilt: $m \le m_j \le M_j \le M\ (j = 1,\ldots,n),\ \sum_{j=1}^{n}{|I_j|} = b-a\ (=|I|)$
\begin{tabbing}
$s_f(Z) $ \=$:= \sum_{j=1}^{n}{m_j |I_j|}$ \=heißt die \begriff{Untersumme} von $f$ bzgl. $Z$.\\
$S_f(Z) $\>$:= \sum_{j=1}^{n}{M_j |I_j|}$ \>heißt die \begriff{Obersumme} von $f$ bzgl. $Z$.
\end{tabbing}

$m \le m_j \le M_j \le M \folgt m|I_j| \le m_j|I_j| \le M_j|I_j| \le M|I_j|$\\
Durch Summation erhält man: $m(b-a) \le s_f(Z) \le S_f(Z) \le M(b-a)$.

$\Z := \{Z: Z$ ist eine Zerlegung von $I\}.$ Sind $Z_1,Z_2 \in \Z \folgt Z_1 \cup Z_2 \in \Z$. Gilt $Z_1 \subseteq Z_2$, so heißt $Z_2$ eine \begriff{Verfeinerung} von $Z_1$.
\end{definition}

\begin{satz}[Zerlegungs-Verfeinerungen]
Seien $Z_1,Z_2 \in \Z$.
\begin{liste}
\item Ist $Z_1 \subseteq Z_2 \folgt s_f(Z_1) \le s_f(Z_2),\ S_f(Z_2) \le S_f(Z_1)$
\item $s_f(Z_1) \le S_f(Z_2)$
\end{liste}
\end{satz}

\begin{beweise}
\item Übung (es genügt zu betrachten: $Z_2 = Z_1 \cup \{t_0\},\ t_0 \notin Z_1$)
\item $Z := Z_1 \subseteq Z_2$. Dann: $s_f(Z_1) \overset{(1)}{\le} s_f(Z) \le S_f(Z) \overset{(1)}{\le} S_f(Z_2).$
\end{beweise}

\def\dx{\text{d}x}
\def\dt{\text{d}t}
% symbole für oberes und unteres Integral, die Maße sind mehr oder weniger
% durch Trial-and-Error bestimmt worden. Wichtig ist hierbei, dass sie sowohl
% bei den großen wie auch bei den kleinen Integralen einigermaßen passen
\def\uint{\declareslashed{}{\text{-}}{0}{-.7}{\int} \ensuremath{\slashed{\int}}}
\def\oint{\declareslashed{}{\text{-}}{.15}{.7}{\int} \ensuremath{\slashed{\int}}}

\newcommand{\intab}[1]{\ensuremath{\int_a^b{#1\dx}}}
\newcommand{\uintab}[1]{\ensuremath{\uint_a^b{#1\dx}}}
\newcommand{\ointab}[1]{\ensuremath{\oint_a^b{#1\dx}}}

\begin{definition}
$$\uintab{f} := \uintab{f(x)} := \sup\{s_f(Z): Z \in \Z\}\text{ heißt \begriff{unteres Integral} von }f$$
$$\ointab{f} := \ointab{f(x)} := \inf\{S_f(Z): Z \in \Z\}\text{ heißt \begriff{oberes Integral} von }f$$
\end{definition}

Sei $Z \in \Z$. Dann: $m(b-a) \le s_f(Z) \le \uintab{f} \overset{\text{31.1(2)}}{\le} S_f(Z) \le M(b-a) \folgt m(b-a) \le \uintab{f} \le \ointab{f} \le M(b-a)$

\begin{definition}
$f$ heißt (Riemann-)\begriff{integrierbar} über $[a,b] :\equizu \uintab{f} = \ointab{f}$. In diesem Fall heißt
$$\intab{f} := \intab{f(x)} := \ointab{f}\ (=\uintab{f})$$
das (Riemann-)\begriff{Integral} von $f$ über $[a,b]$.

$R[a,b] := \{g: [a,b] \to \MdR: g$ ist auf $[a,b]$ beschränkt und integrierbar über $[a,b]\}$
\end{definition}

\begin{beispiele}
\item Sei $c \in \MdR$ und $f(x) = c\ \forall x \in [a,b]$. Sei $Z = \{x_0,\ldots,x_n\} \in \Z;\ m_j = M_j = c\ (j = 1,\ldots,n) \folgt s_f(Z) = S-F(Z) = \sum_{j=1}^{n}{|I_j} = c(b-a) \folgt f \in R[a,b]$ und $\intab{c} = c(b-a)$.
\item $$f(x) := \begin{cases}
1,& x \in [a,b] \cap \MdQ\\
0,& x \in [a,b]\ \backslash\ \MdQ \end{cases}$$

Sei $Z = \{x_0,\ldots,x_n\} \in \Z,\ m_j = 0,\ M_j = 1\ (j=1,\ldots,n)\\
\folgt s_f(Z) = 0,\ S_f(Z) = \sum_{j=1}^{n}{|I_j|} = b-a.$

$\folgt \uintab{f} = 0 \ne b-a = \ointab{f} \folgt f \notin R[a,b].$

\item $[a,b]=[0,1]$, $f(x)=x$. Sei $n \in\MdN$ und $Z=\{x_0, \cdots, x_n\}$, wobei $x_j:=j*\frac{1}{n}\ (j=0,..,n).$  $m_j, M_j, I_j$ wie immer. Dann: $|I_j|=\frac{1}{n}$.
$$m_j=f(x_j-1)=(j-1)\frac{1}{n}.\ s_f(Z)=\sum_{j=1}^{n}(j-1)\frac{1}{n}=\frac{1}{n^2}(0+1+\cdots+(n-1))=\frac{1}{n^2}\frac{(n-1)n}{2}=\frac{n-1}{2n}$$
$$M_j=f(x_j)=\frac{j}{n}.\ S_f(Z)=\sum^{n}_{j=1}j\frac{1}{n^2}=\frac{1}{n^2}(1+\cdots+n)=\frac{1}{n^2}\frac{n(n+1)}{2}=\frac{n+1}{2n}$$
$$\frac{n-1}{2n}=s_f(Z)\le\uint^1_0x\dx\le\oint^1_0x\dx\le S_f(Z)=\frac{n+1}{2n} \folgt f\in R[0,1] \text{ und } \int_0^1x\dx=\frac{1}{2}$$
\end{beispiele}

\begin{satz}[Rechenregeln für Integrale]
Es seien $f,g \in R[a, b]$
\begin{liste}
\item Ist $f\le g$ auf $[a, b] \folgt \int_a^bf\dx \le \int_a^bg\dx$
\item Sind $\alpha, \beta \in \MdR \folgt \alpha f + \beta g \in R[a, b]$ und $\int_a^b(\alpha f + \beta g)dx = \alpha\int_a^bf\dx + \beta\int_a^bg\dx$
\end{liste}
\end{satz}

\begin{beweise}
\item "Ubung.
\item "Ubung: $\alpha f \in R[a, b]$ und $\int_a^b(\alpha f)\dx=\alpha\int_a^bf\dx$.\\Zu zeigen: $f+g \in R[a, b]$ und $\int_a^b(f+g)\dx = \int_a^bf\dx + \int_a^bg\dx$.
Sei $z=\{x_0, \cdots, x_n\} \in \Z, m_j, M_j, I_j$ wie immer. $\widetilde{m_j}:=\inf g(I_j),\ \widetilde{\widetilde{m_j}}:=\inf (f+g)(I_j)$. $x\in I_j:\ (f+g)(x)=f(x)+g(x) \ge m_j + \widetilde{m_j} \folgt \widetilde{\widetilde{m_j}}\ge m_j + \widetilde{m_j} \folgt \widetilde{\widetilde{m_j}}|I_j| \ge m_j|I_j| + \widetilde{m_j}|I_j| \folgtnach{Summation} S_{f+g}(Z) \ge S_f(Z) + S_g(Z) \folgt S_f(Z) + S_g(Z) \le \uint_a^b(f+g)dx \ \forall z \in \Z\ (*)$. Sei $\ep>0: \exists\ Z_1, Z_2 \in \Z: S_f(Z_1) > \uint_a^bf\dx - \ep=\int_a^bf\dx - \ep,\ s_g(Z_2)>\int_a^bg\dx - \ep,\ Z:=Z_1 \cup Z_2 \in \Z$. $\underbrace{\int_a^bf\dx + \int_a^bg\dx}_{=:A} - 2\ep < S_f(Z_1) + S_g(Z_2) \overset{23.1}{\le} S_f(Z) + S_g(Z) \overset{(*)}{\le} \int_a^b (f+g)dx$. Also: $A-2\ep \le \int_a^b(f+g)dx\ \forall \ep>0 \folgtwegen{\ep \to 0+} A\le\uint_a^b(f+g)dx$. Analog: $\oint_a^b(f+g)dx\le A \folgt A=\uint_a^b(f+g)dx = \oint_a^b(f+g)dx$
\end{beweise}

\begin{satz}[Riemannsches Integrabilit"atskriterium]
$f \in R[a,b] \equizu \forall \ep>0\ \exists Z \in \Z: S_f(Z)-s_f(Z) < \ep$.
\end{satz}
\begin{beweis}
\glqq$\Leftarrow$\grqq: Sei $\ep>0.$ Voraussetzung $\folgt \exists Z \in \Z: S_f(Z)<s_f(Z)+\ep \folgt \oint_a^bf\dx \le S_f(Z) < s_f(z) + \ep\le\uint_a^bf\dx + \ep$. Also: $\oint_a^bf\dx < \uint_a^bf\dx\ \forall \ep>0\folgtwegen{\ep\to 0+}\oint_a^bf\dx \le \uint_a^bf\dx (\le \oint_a^bf\dx)\folgt f\in R[a, b]$.\\
\glqq$\folgt$\grqq: $S:=\int_a^bf\dx$. Sei $\ep>0.\ \exists Z_1, Z_2 \in \Z: s_f(Z_1) > \uint_a^bf\dx - \frac{\ep}{2}=S-\frac{\ep}{2}.\ S_f(Z_2)<S+\frac{\ep}{2}.\ Z:=Z_1 \cup Z_2 \in \Z$. $S_f(Z)-s_f(Z) \overset{23.1}{\le}S_f(Z_2)-s_f(Z_1)<S+\frac{\ep}{2}-(S-\frac{\ep}{2})=\ep$.
\end{beweis}

\begin{satz}[Integratibilität monotoner und stetiger Funktionen]
\begin{liste}
\item Ist f auf $[a,b]$ monoton $\folgt f \in R[a,b]$.
\item $C[a,b] \subseteq R[a,b]$.
\end{liste}
\end{satz}

\begin{beweise}
\item f sei wachsend auf $[a,b]$. Sei $n\in\MdN$ und $Z=\{x_0,\cdots,x_n\}$ sei die \begriff{"aquidistante Zerlegung} von $[a,b]$ mit $n+1$ Teilpunkten. $x_j = a+j\frac{b-a}{n}\ (j=0,\cdots,n)$, dann: $|I_j|=\frac{b-a}{n}$. $m_j, M_j$ wie immer: $S_f(Z)-s_f(z)=\sum^n_{j=1}(\underbrace{M_j}_{=f(x_j)}-\underbrace{m_j}_{f(x_j-1)})|I_j|=\sum_{j=1}^n(f(x_j)_-f(x_j-1))\frac{b-a}{n}=\frac{b-a}{n}(f(x_1)-f(x_0)+f(x_2)-f(x_1)+\cdots+f(x_n)-f(x_n-1))=\frac{b-a}{n}(f(x_n)-f(x_0))=\frac{b-a}{n}(f(b)-f(a))=:\alpha_n$. Sei $\ep>0$, dann: $\exists n\in\MdN: \alpha_n <\ep\folgtnach{23.3}$Behauptung.
\item Sei $f \in C[a, b]$ und $\ep>0$. $\exists \delta>0: (*)\ |f(t)-f(s)|<\frac{\ep}{b-a}\ \forall t,s\in[a,b]$ mit $|t-s|<\delta$. Sei $Z=\{x_0,\cdots,x_n\}\in\Z\ m_j,\ M_j,\ |I_J|$ seien wie immer; $z$ sei so gew"ahlt, da"s $|I_j|<\delta\ (j=1,\cdots,n)$. Betrachte $I_j:$ 18.3$\folgt\ \exists s_j, t_j \in I_j: m_j=f(s_j),\ M_j=f(t_j)$. $|t_j-s_j|<\delta \folgtwegen{(*)}\underbrace{f(t_j)-f(s_j)}_{=M_j-m_j}<\frac{\ep}{b-a}\folgt S_f(Z)-s_f(Z)=\sum^n_{j=1}(\underbrace{M_j-m_j}_{\le\frac{\ep}{b-a}})|I_j|<\frac{\ep}{b-a}\sum^n_{j=1}|I_j|=\ep\folgtnach{23.3}f\in R[a,b]$
\end{beweise}

\begin{definition}
Sei $J\subseteq\MdR$ ein Intervall und $G,g: J\to\MdR$ seien Funktionen. $G$ hei"st eine \begriff{Stammfunktion} (SF) von $g$ auf $J$ :$\equizu$ G ist differenzierbar auf $J$ und $G'=g$ auf $J$.\\
\end{definition}
\textbf{Beachte:}
\begin{liste}
\item Sind $G_1$ und $G_2$ Stammfunktionen von $g$ auf $J \folgtnach{21.7}\ \exists c \in\MdR: G_1=G_2+c$ auf $J$.
\item Sei $I=[a,b]$. Es gibt Funktionen, die auf $[a,b]$ Stammfunktionen besitzen, aber "uber $[a,b]$ nicht integrierbar sind.
\begin{beispiel}
$$F(x) := \begin{cases}
x^{\frac{3}{2}} \sin\frac{1}{x},&x\in(0,1]\\
0,& x=0\end{cases}$$
Bekannt: (§22): $F$ ist auf $[0,1]$ differenzierbar und $f:=F'$ ist auf [0,1] \emph{nicht} beschr"ankt. Also: $f \notin R[0,1]$, besitzt aber auf $[0,1]$ die Stammfunktion $F$.
\end{beispiel}
\item Sei $I=[a,b]$. Es gibt Funktionen in $R[a,b]$, die auf $[a,b]$ keine Stammfunktionen besitzen.
\begin{beispiel}
Sei $[a,b]=[-1,1]$, $f(x):=\begin{cases}
1&x\in[-0,1]\\
0&x\in[-1,0)\end{cases}$. $f$ ist monoton auf $[-1,1]\folgtnach{23.4}f\in R[-1,1]$. Annahme: $f$ besitzt auf $[-1, 1]$ die Stammfunktion $F$. Auf $[0,1]:F'(x)=f(x)=1=(x)'\folgtnach{21.7}\ \exists c_1 \in \MdR: F(x)=x+c_1\ \forall x \in [0,1]$. Auf $[-1, 0)$: $F'(x)=f(x)=0 \folgtnach{21.7}\ \exists c_2 \in \MdR: F(x)=c_2\ \forall x\in [-1,0)$. $\displaystyle\lim_{x\to 0+}F(x)=c_1,\ \displaystyle\lim_{x\to 0-}F(x)=c_2$. $F$ stetig in $x=0\folgt c_1=c_2$.$\displaystyle\lim_{x \to 0+}\frac{F(x)-F(0)}{x-0}=\displaystyle\lim_{x\to 0+}\frac{x+c_1-c_1}{x}=1$, $\displaystyle\lim_{x\to 0-}\frac{F(x)-F(0)}{x-0}=\frac{c_2-c_1}{x}=0$, Widerspruch zur Differenzierbarkeit von $F$ in $x_0=0$.
\end{beispiel}
\end{liste}

\begin{satz}[1. Hauptsatz der Differential- und Integralrechnung]
Es sei $f\in R[a,b]$ und $f$ besitze auf $[a,b]$ die Stammfunktion $F$. Dann: 
$$\int_a^bf(x)dx = F(b) - F(a) =: F(x)|_a^b =: [F(x)]_a^b$$
\end{satz}

\begin{beweis}
Sei $Z=\{x_0,\ldots,x_n\} \in \Z; m_j, M_j, I_j$ sei wie gehabt. Sei $j\in\{1,\ldots,n\}$. MWS $\folgt \exists \xi_j \in I_j: F(x_j)-F(x_{j+1}) = F'(\xi_j)(x_j-x_{j+1}) = f(\xi_j)\cdot|I_j| \folgt \sum_{j=1}^nf(\xi_j)|I_j| = \sum _{j=1}^n(F(x_j) - F(x_{j+1}) = F(b) - F(a)$ \\
$m_j|I_j|\le f(\xi_j)|I_j| \le M_j|I_j| \folgtnach{Summation} s_f(Z)\le F(b) - F(a) \le S_f(Z) \ \forall Z\in\Z \folgt \underbrace{\uint_a^bfdx}_{=\int_a^bfdx} \le F(b) - F(a) \le \underbrace{\oint_a^bfdx}_{=\int_a^bfdx} \folgt F(b) - F(a) = \int_a^bfdx$
\end{beweis}

\begin{beispiele}
\item $\sum_0^{\frac{\pi}{2}}\cos xdx$, $\cos x$ ist stetig auf $[0,\frac{\pi}{2}]$, also integrierbar. $F(x) = \sin x$ ist eine Stammfunktion von $\cos x$ $\folgt \int_0^{\frac{\pi}{2}}\cos xdx = \sin x|_0^{\frac{\pi}{2}} = 1$.
\item $\int_0^1\frac{1}{1+x^2}dx = \arctan x|_0^1 = \arctan 1 - \arctan 0 = \frac{\pi}{4}$
\end{beispiele}

\begin{beispiele}
\item Sei $\MdQ\cap[0,1] = \{q_1, q_2, \ldots \}$, $f_n(x) = \begin{cases} 1, & x\in\{q_1,\ldots,q_n\} \\ 0, & x\in[0,1]\backslash\{q_1,\ldots,q_n\} \end{cases}$, $(n\in\MdN)$. $(f_n)$ konvergiert auf $[0,1]$ \emph{punktweise} gegen $f(x) = \begin{cases} 1, & x\in \MdQ \cap [0,1] \\ 0, & x \in [0,1]\backslash \MdQ \end{cases}$. Bekannt: $f \notin R[0,1]$. In 23.10 werden wir sehen: $f_n\in R[0,1] \ \forall n\in\MdN$.
\item Für $x\in[0,1]$, $n\in\MdN$, $n\ge 3$ sei $f_n$ wie in der Zeichnung:
$$f_n(x) = \begin{cases} n^2 x,& x\in[0,\frac{1}{n}] \\ 2n - n^2 x, & x\in(\frac{1}{n},\frac{2}{n}] \\ 0, & x \in (\frac{2}{n},1]\end{cases}$$
$f_n \in C[0,1] \folgt f_n \in R[0,1]$. zur Übung: $\int_0^1f_ndx = 1 \forall n\in\MdN$. $(f_n)$ konvergiert auf $[0,1]$ \emph{punktweise} gegen $f(x)= 0$. \\
Aber: $\lim_{n\to\infty}\int_0^1f_ndx = 1 \ne 0 = \int_0^1fdx = \int_0^1(\lim_{n\to\infty} f_n(x)) dx $
\end{beispiele}

\begin{satz}[Integrierbarkeit gleichmäßig konvergierender Funktionsfolgen]
$(f_n)$ sei eine Folge in $R[a,b]$ und $(f_n)$ konvergiert auf $[a,b]$ \emph{gleichmäßig} gegen $f:[a,b]\to\MdR$. Dann ist $f\in R[a,b]$ und 
$$\lim_{n\to\infty}\int_a^bf_n(x)dx  = \int_a^bfdx = \int_a^b(\lim_{n\to\infty} f_n)dx$$

$(f_n)$ sei eine Folge in $R[a,b]$ und $\sum_{n=1}^{\infty}f_n$ konvergiert auf $[a,b]$ \emph{gleichmäßig} gegen $f:[a,b]\to\MdR$. Dann ist $f\in R[a,b]$ und 
$$\sum_{n=1}^\infty \int_a^bf_n(x)dx = \int_a^b \sum_{n=1}^\infty f_n(x)dx $$
\end{satz}

\begin{beweis}
1. Zu $\ep=1 \ \exists m \in \MdN$: $f_m-1<f<f_m+1$ auf $[a,b]$. $f_n$ beschränkt auf $[a,b]$. \\
2. $A_n := \int_a^bf_ndx$ $(n\in\MdN)$. Sei $\ep>0$. $\exists n_0\in\MdN: f_n-\ep<f<f_n+\ep$ auf $[a,b] \ \forall n\ge n_0 \folgt$ für $n\ge n_0$ folgt (wie im Beweis von 23.2(1)): 
$$\underbrace{\uint_a^b(f_n-\ep)dx}_{=A_n-\ep(b-a)} \le \underbrace{\uint_a^bfdx}_{=: A} \le \underbrace{\oint_a^bfxds}_{=: B} \le \underbrace{\oint_a^b(f_n+\ep)dx}_{=A_n+\ep(b-a)}$$
$\folgt |A_n - A| \le \ep(b-a)$, $|A_n -B|\le \ep (b-a)$ \\ 
$\forall n\in n_0 \folgt A_n \to A, A_n \to B \ (n\to\infty) \folgt A = B $ \\ 
$\folgt f\in R[a,b]$ und $A_n \to \int_a^bfdx$
\end{beweis}

\begin{beispiel}
$$g(x) = \begin{cases} 0, & x=0 \\ 1, & x \in (0,1] \end{cases}$$
$g$ ist monoton $\folgt g \in R[0,1]$.
$$f(x) = \begin{cases} 1, & x=0 \\ 0, & x \in [0,1]\backslash\MdQ \\ \frac{1}{q}, & x = \frac{p}{q}, p,q \in \MdN \text{ teilerfremd} \end{cases}$$
Übungsblatt: $f\in R[0,1]$
$$(g\circ f)(x) = \begin{cases}1, &x\in Q\cap[0,1] \\ 0, & x \in [0,1]\backslash \MdQ \end{cases} \notin R[0,1]$$
\end{beispiel}

\begin{satz}[Integration von verketteten Funktionen]
Es sei $f\in R[a,b]$, $D := f([a,b])$ und $h: D \to R$ sei Lipschitzstetig auf $D$. Dann: $h\circ f \in R[a,b]$
\end{satz}

\begin{beweis}
$g:= h\circ f$. $\exists L>0$. $|h(t) - h(s)| \le L|t-s| \ \forall t,s \in D$. O.B.d.A: $L>0$. Sei $Z = \{x_0, \ldots, x_n\} \in \Z$, $m_j, M_j, I_j$ seien wie gehabt. $\tilde m_j := \inf g(I_j)$, $\tilde M_j := \sup g(I_j)$. Seien $x,y \in I_j$, etwa $f(x) \le f(y)$: $g(x) - g(y) \le |g(x) - g(<y|= |h(f(x)) - h(f(y))| \le L|f(x)-f(y)| = L(f(y)-f(x)) \le L(Mj-mj) =: c_j \folgt g(x) \le g(y) + c \ \forall x,y \in I_j \folgt \tilde M_j \le g(y)+c_j \ \forall y\in I_j \folgt \tilde M_j - c_j \le g(y) \ \forall y \in I_j \folgt \tilde M_j - c_j \le \tilde m_j \folgt \tilde M_j - \tilde m_j \le c_j = L(M_j - m_j) \folgt S_g(Z) - s_g(Z) = \sum_{j=1}^n(\tilde M_j - \tilde m_j)|I_j| \le L\sum_{j=1}^n(M_j - m_j)|I_j| = L(S_f(Z) - s_f(Z)) \ \forall z\in\Z \folgtnach{23.3} g\in R[a,b]$
\end{beweis}

\begin{satz}[Weitere Rechenregeln für Integrale]
Es seien $f,g \in R[a,b]$.
\begin{liste}
\item $|f| \in R[a,b]$ und $|\int_a^bfdx| \le \int_a^b|f|dx$ (\begriff{Dreiecksungleichung für Integrale})
\item $fg \in R[a,b]$
\item Ist $g(x) \ne 0 \ \forall x\in[a,b]$ und $\frac{1}{g}$ beschränkt auf $[a,b] \folgt \frac{1}{g} \in R[a,b]$
\end{liste}
\end{satz}

\begin{beweis}
\begin{liste}
\item $D:= f([a,b])$, $h(t) := |t| \ (t\in D)$. Dann: $|f|=h\circ f$. Für $t,s \in D$: $|h(t) - h(s)| = | |t| - |a| | \stackrel{\text{§1}}{\le} |t-s| \folgtnach{23.7} |f| \in R[a,b]$ \\
$ \pm f \le |f|$ auf $[a,b]$. 23.2 $\folgt \pm \int_a^bfdx \le \int_a^b|f|dx \folgt |\int_a^bfdx| \le \int_a^b|f|dx$
\item 1. $D:=f([a,b])$, $h(t) := t^2 \ (t\in D)$. Dann: $f^2 = h \circ f$.\\
$\exists \gamma >0: |f(x)|\le \gamma  \ \forall x\in[a,b] \folgt |t| < \gamma \ \forall t\in D$ Für $t, s \in D$: $|h(t)-h(s)| = |t^2 - s^2| = |t+s||t-s| \le (|t| + |s|)\cdot |t-s|\le 2\gamma |t-s| \folgtnach{23.7} f^2\in R[a,b]$\\
2. $f+g, f-g \in R[a,b] \folgt (f+g)^2,(f-g)^2 \in R[a,b] \folgt \frac{1}{4}\left( (f-g)^2 - (f-g)^2 \right) \in R[a,b] \folgt f\cdot g \in R[a,b]$
\item $D :=  g([a,b])$, $h(t) := \frac{1}{t}$ $(t\in D)$. Dann: $\frac{1}{g} = h \circ g$. \\
$\exists \gamma > 0: \frac{1}{|g(x)|} \le \gamma \ \forall x\in[a,b] \folgt \frac{1}{|t|} \le \gamma \ \forall t\in D$. Für $t,s \in D$: $|h(t) - h(s)| = |\frac{1}{t} - \frac{1}{s}| = \frac{|t-s|}{|t||s|}\le \gamma ^2 |t-s| \folgtnach{23.7} \frac{1}{g} \in R[a,b]$

\end{liste}
\end{beweis}

\begin{satz}[Aufteilung eines Integrals]
$f:[a,b] \to \MdR$ sei beschränkt und $c \in (a,b).$ Dann gilt:
$$f \in R[a,b] \equizu f \in R[a,c]\text{ und } f \in R[c,b].$$
In diesem Fall ist:
$$\intab{f} = \int_a^c{f\dx} + \int_c^b{f\dx}$$
\end{satz}

\def\hin{\item["`$\Rightarrow$"':]}
\def\zurueck{\item["`$\Leftarrow$"':]}

\begin{beweis}
\begin{description}
\hin Sei $\ep > 0.$ Aus 23.3 folgt: $\exists Z_1 \in \Z: S_f(Z_1) - s_f(Z_1) < \ep.$

$Z := Z_1 \cup \{c\} \in \Z.$ Sei $Z = \{x_0,\ldots,x_k,x_{k+1},\ldots,x_n\}$ mit $x_k = c.\ Z_0 := \{x_0,\ldots,x_k\}$ ist eine Zerlegung von $[a,c].\ M_j,\ m_j,\ I_j$ seien wie immer. Dann gilt:

$S_f(Z_0) - s_f(Z_0) = \sum_{j=1}^k{(M_j-m_j) |I_j|} \le \sum_{j=1}^n{(M_j-m_j) |I_j|} = S_f(Z)-s_f(Z) \le S_f(Z_1) - s_f(Z_1) < \ep \folgtnach{23.3} f \in R[a,c].$ Analog: $f \in R[c,b].$

\zurueck $S:=\int_a^c{f\dx} + \int_c^b{f\dx}.$ Sei $\ep > 0$ Dann gibt es Zerlegungen $Z_1$ von $[a,c]$ und $Z_2$ von $[c,b]: s_f(Z_1) = \uint_a^c{f\dx} - \ep = \int_a^c{f\dx},\ s_f(Z_2) > \int_b^c{f\dx} - \ep.$

$Z:=Z_1 \cup Z_2 \folgt Z \in \Z$ und $\uintab{f} \ge s_f(Z) = s_f(Z_1) + s_f(Z_2) > S-2 \ep.$

Also: $S-2\ep < \uintab{f}\ \forall \ep > 0 \folgtwegen{\ep \to 0+} S \le \uintab{f}.$

Analog: $\ointab{f} \le S \folgt f \in R[a,b],\ \intab{f} = S.$
\end{description}
\end{beweis}

\begin{satz}[Integral und Unstetigkeitsstellen]
$f,g: [a,b] \to \MdR$ seien Funktionen.
\begin{liste}
\item Ist $f$ beschränkt auf $[a,b]$ und $A:=\{x \in [a,b]: f$ ist in $x$ \emph{nicht} stetig$\}$ \emph{endlich}, dann gilt: $f \in R[a,b]$.
\item Ist $f \in R[a,b]$ und $A:=\{x \in [a,b]: f(x) \ne g(x)\}$ \emph{endlich}, dann gilt: $g \in R[a,b]$ und $\intab{g} = \intab{f}$.
\end{liste}
\end{satz}

\begin{beweise}
\item $\exists \gamma \ge 0: |f(x)| \le \gamma\ \forall x \in [a,b].$ Es genügt zu betrachten: $A:=\{t_0\}$ (wegen 23.9). O.B.d.A.: $t_0 = a$ oder $t_0 = b.$ Etwa: $t_0 = a$.

Sei $\ep > 0.$ Wähle $\alpha \in (a,b)$ mit $2\gamma(\alpha-a) < \ep/2.$

$f \in C[\alpha,b] \folgt f \in R[\alpha,b] \folgtnach{23.3}$ Es gibt eine Zerlegung $Z_1$ von $[\alpha,b]$ mit: $S_f(Z_1) - s_f(Z_1) < \ep/2.\ Z:=Z_1 \cup \{a\} \folgt Z \in \Z$ und $S_f(Z) - s_f(Z) = \underbrace{\sup f([a,\alpha]) - \inf f([a,\alpha]))(\alpha-1)}_{\le 2 \gamma} + \underbrace{S_f(Z_1)-s_f(Z_1)}_{< \ep/2} < 2 \gamma(\alpha-a) + \ep/2 < \ep/2 + \ep/2 = \ep.$

\item Klar: g ist beschränkt. $h := g-f.$ Dann: $h(x) = 0\ \forall x \in [a,b]\backslash A \folgt h \in C([a,b] \backslash A) \folgtnach{(1)} h \in R[a,b] \folgt g = h+f \in R[a,b].$

Noch zu zeigen: $\intab{h} = 0.\ \varphi := |h|.$ Aus 23.8 folgt: $\varphi \in R[a,b]$ und $|\intab{h}| \le \intab{\varphi}.$

Sei $Z := \{x_0,\ldots,x_n\} \in \Z,\ m_j := \inf \varphi(I_j),\ \varphi(x) = 0\ \forall x \in [a,b] \backslash A,\ \varphi(x) > 0\ \forall x \in A \folgt m_j = 0\ (j = 1,\ldots,n) \folgt s_f(Z) = 0 \folgt \uintab{\varphi} = \intab{\varphi} = 0 \folgt \intab{h} = 0.$
\end{beweise}

\begin{satz}[Mittelwertsatz der Integralrechnung]
Es seien $f,g \in R[a,b],\ g \ge 0$ (oder $g \le 0$) auf $[a,b],\ m:=\inf f([a,b]),\ M:=\sup f([a,b])$
\begin{liste}
\item $\exists \mu \in [m,M]: \intab{fg} = \mu \intab{g}$
\item Ist $f \in C[a,b] \folgt \exists \xi \in [a,b]: \intab{f} = f(\xi)(b-a)$
\end{liste}
\end{satz}

\begin{beweise}
\item $\alpha := \intab{g},\ \beta := \intab{fg}.\ m \le f \le M$ auf $[a,b] \folgt mg \le fg \le Mg$ auf $[a,b] \folgt m\alpha \le \beta \le M\alpha.$

Es ist $\alpha \ge 0.$ O.B.d.A.: $\alpha > 0.$ Dann gilt: $m \le \frac{\beta}{\alpha} \le M,\ \mu := \frac{\beta}{\alpha}.$

\item Setze in (1) $g \equiv 1 \folgt \intab{f} = \mu(b-a)\ (\mu \in [m,M]).$ Aus 18.1 folgt: $\exists \xi \in [a,b]: \mu = f(\xi).$
\end{beweise}

\paragraph{Der Riemannsche Zugang zum Integral}

\textit{Bemerkung: Wir haben bisher tatsächlich die \emph{Darbouxschen} Integrale betrachtet. Hier wird nun die ursprüngliche Definition von Riemann vorgestellt.}

$f:[a,b] \to \MdR$ sei beschränkt. Sei $Z:=\{x_0,\ldots,x_n\} \in \Z.\ m_j,M_j,I_j$ seien wie immer.

Wählt man in jedem $I_j$ einen Punkt $\xi_j$, so heißt $\xi := (\xi_1,\xi_2,\ldots,\xi_n)$ ein zu $Z$ passender \begriff{Zwischenvektor} und $\sigma_f(Z,\xi) := \sum_{j=1}^{n}{f(\xi_j) |I_j|}$ eine \begriff{Riemannsche Zwischensumme}.

$m_j = \le f(\xi_j) \le M_j\ (j=1,\ldots,n) \folgt s_f(Z) \le \sigma_f(Z,\xi) \le S_f(Z)$

\begin{satz}[Äquivalenz der Riemannschen und Darbouxschen Integrale]
$f: [a,b] \to \MdR$ sei beschränkt. Dann gilt: $f \in R[a,b]$ genau dann, wenn es ein $S \in \MdR$ gibt mit:
$$\forall \ep > 0\ \exists Z \in \Z: |\sigma_f(Z,\xi)-S| < \ep\text{ für jedes zu Z passende } \xi.\ (*)$$
In diesem Fall gilt:
$$S = \intab{f}.$$
\end{satz}

\begin{beweis}
\begin{description}
\hin $S := \intab{f}.$ Sei $\ep > 0.$ Wie im Beweis von 23.3: $\exists Z \in \Z: s_f(Z) > S-\ep,\ S_f(Z) < S+\ep.$

Sei $\xi$ passend zu $Z \folgt S-\ep < s_f(Z) \le \sigma_f(Z,\xi) \le S_f(Z) < S+\ep \folgt |\sigma_f(Z,\xi)-S| < \ep.$

\zurueck Sei $\ep > 0.$ Nach Voraussetzung gibt es ein $Z \in \Z$ so, dass (*) gilt. Sei $Z := \{x_0,\ldots,x_n\},\ m_j,\ M_j,\ I_j$ wie immer. Sei $j \in \{1,\ldots,n\}: \exists \xi_j,\eta_j \in I_j: f(\xi_j) > M_j - \ep,\ f(\eta_j) < m_j + \ep,\ \xi := (\xi_1,\ldots,\xi_n),\ \eta = (\eta_1,\ldots,\eta_n)$ sind passend zu $Z$.

$A := \sigma_f(Z,\xi),\ B := \sigma_f(Z,\eta).\ A = \sum_{j=1}^n{f(\xi_j) |I_j|} > \sum_{j=1}^n{(M_j-\ep) |I_j|} = S_f(Z) - \ep(b-a) \folgt S_f(Z) < A + \ep(b-a).\quad$(i)

Analog: $-s_f(Z) < \ep(b-a) - B.\quad$(ii)

Dann gilt: $S_f(Z) - s_f(Z) < A-B+2\ep(b-a) = A-S+S-B+2\ep(b-a) \le |A-S|+|B-S|+2\ep(b-a) \overset{\text{(*)}}{<} 2\ep+2\ep(b-a) = \ep(2 + 2(b-a)) \folgtnach{23.3} f \in R[a,b].$

$\intab{f} = \ointab{f} \le S_f(Z) \overset{\text{(i)}}{<} A + \ep(b-a) = A - S + S + \ep(b-a) \le |A-S| + S + \ep(b-a) \overset{\text{(*)}}{<} \ep + S + \ep(b-a).$

Also: $\intab{f} < S + \ep(1 + (b-a))\ \forall \ep > 0 \folgtwegen{\ep \to 0+} \intab{f} \le S.$ Analog folgt mit (ii): $S \le \intab{f}.$
\end{description}
\end{beweis}

\begin{definition}
Sei $f \in R[a,b].\ \int_c^cf(x)\dx:=0$ und $\int_b^af(x)\dx=:-\int_a^bf(x)\dx$
\end{definition}
\begin{bemerkung}
$\int_a^bf(x)\dx=\int_a^bf(t)\dt$.
\end{bemerkung}

\begin{satz}[2. Hauptsatz der Differential- und Integralrechnung]
Sei $f \in R[a,b]$ und $F:[a,b]\to\MdR$ sei definiert durch $F(x):=\int_a^xf(t)\dt$.
\begin{liste}
\item $F$ ist auf $[a,b]$ Lipschitzstetig, insbesondere $F \in C[a,b]$
\item Ist $f$ in $x_0 \in [a,b]$ stetig $\folgt F$ ist in $x_0$ differenzierbar und $F'(x_0)=f(x_0)$
\item Ist $f \in C[a,b] \folgt F \in C^1[a,b]$ und $F'=f$ auf $[a,b]$
\end{liste}
\end{satz}

\begin{beweise}
\item $L:=\sup\{|f(x)| : x \in [a,b]\}$. Sei $x,y \in [a,b]$, etwa $x\le y$. $F(y)=\int_a^yf(t)\dt\gleichnach{23.9}\int_a^xf(t)\dt + \int_x^yf(t)\dt=F(x)+\int_x^yf(t)\dt\folgt F(y)-F(x)=\int_x^yf(t)\dt\folgt |F(y)-F(x)|=|\int_x^yf(t)\dt|\overset{23.8}{\le}\int_x^y\underbrace{|f(t)|}_{\le L}\dt\le\int_x^yL\dt=L(y-x)=L|y-x|$
\item Sei $x_0 \in [a,b)$. Wir zeigen: $(*) \displaystyle\lim_{h\to 0+}\frac{F(x_0+h)-F(x_0)}{h}=f(x_0)$ (analog zeigt man f"ur $x_0 \in (a,b]\ :\ \displaystyle\lim_{h\to0-}\frac{F(x_0+h)-F(x_0)}{h}=f(x_0)$) Sei also $x_0 \in [a,b)$, $h>0$ und $x_0+h<b$. $g(h):=|\frac{F(x_0+h)-F(x_0)}{h}-f(x_0)|$. Zu zeigen: $g(h)\to 0\ (h\to0+)$. Es ist $\frac{F(x_0+h)-F(x_0)}{h}\gleichnach{s.o.}\frac{1}{h}\int_{x_0}^{x_0+h}f(t)\dt$, $\ \frac{1}{h}\int_{x_0}^{x_0+h}f(x_0)\dt=\frac{1}{h}f(x_0)h=f(x_0)\folgt g(h)=\frac{1}{h}|\int_{x_0}^{x_0+h}(f(t)-f(x_0))\dt|\overset{23.8}{\le}\frac{1}{h}\int_{x_0}^{x_0+h}|f(t)-f(x_0)|\dt;\ s(h):=\sup\{|f(t)-f(x_0)|\ :\ t \in [x_0,x_0+h]\}\folgt g(h)\le\frac{1}{h}\int_{x_0}^{x_0+h}s(h)\dt=\frac{1}{h}s(h)h=s(h)$. Also: $0\le g(h)\le s(h)$. $f$ stetig in $x_0 \folgt f(t)\to f(x_0)\ (t \to x_0) \folgt s(h)\to 0\ (h\to 0+) \folgt g(h)\to 0\ (h\to 0+)\folgt (*)$
\item folgt aus (2)
\end{beweise}

\begin{satz}[Anwendung des 2. Hauptsatzes auf stetige Funktionen]
Sei $J \subseteq \MdR$ ein beliebiges Intervall, $f \in C(J)$ und $\xi \in J$ (fest). $F:J\to\MdR$ sei definiert durch $F(x):=\int_{\xi}^xf(t)\dt$. Dann ist $F\in C^1(J)$ und $F'=f$ auf $J$. 
\end{satz}


\begin{beweis}
Seien $a,b \in J$, $a<b$ und $I:=[a,b]$. Es gen"ugt zu zeigen: $F$ ist differenzierbar auf $I$ und $F'=f$ auf $I$. $G(x):=\int_a^xf(t)dt\ (x\in I)$. Sei $\xi\le a$ (analoger Beweis f"ur $\xi\ge b$ und $\xi \in (a,b)$. F"ur $x \in [a,b]:\ F(x)=\int_{\xi}^x\cdots = \int_{\xi}^a\cdots + \int_a^x\cdots=F(a)+G(x)\folgtnach{23.13} F$ ist differenzierbar auf $I$ und $F'=G'=f$ auf $I$.
\end{beweis}

\begin{definition}
Im folgenden seien $I,J\subseteq\MdR$ beliebige Intervalle.
\begin{liste}
\item Sei $g:I\to\MdR$ und $x_0\in I$. $g(x)|_{x=x_0}:=g(x_0).$
\item Ist $f \in R[a,b]$, so hei"st $\int_a^bf(x)\dx$ auch ein \begriff{bestimmtes Integral}.
\item Besitzt $G:I\to\MdR$ auf $I$ eine Stammfunktion, so schreibt man f"ur eine solche auch $\int g(x)\dx$ (\begriff{unbestimmtes Integral}). "Gleichungen" der Form $\int g(x)\dx=h(x)$ gelten bis auf additive Konstanten! \textbf{Beispiel}: $\int e^x\dx=e^x, \int e^x\dx=e^x + 7$. $\int g(x)\dx=h(x)$ auf $I$ bedeutet: h ist eine Stammfunktion von $g$ auf $I$.
\end{liste}
\end{definition}

\begin{satz}[Partielle Integration]
\begin{liste}
\item Es seien $f,g \in R[a,b]$ und $F,G$ seien Stammfunktionen von $f$ bzw. $g$ auf $[a,b]$. Dann: $$\int_a^bFg\dx=F(x)G(x)|_a^b-\int_a^bfG\dx$$
\item Sind $f,g \in C^1[a,b] \folgt$ $$\int_a^b f'g\dx=f(x)g(x)|_a^b-\int_a^bfg'\dx$$
\item Sind $f,g \in C^1(I) \folgt$ auf $I$ gilt: $$\int f'g\dx=f(x)g(x)-\int fg'\dx$$
\end{liste}
\end{satz}

\begin{beweise}
\item $(FG)'=F'G+FG'=fG+Fg\folgt \int_a^bFg\dx+\int_a^bfG\dx=\int_a^b(FG)'\dx\gleichnach{23.5}F(x)G(x)|_a^b$
\item folgt aus (1)
\item $(fg)'=f'g+fg'\folgt fg=\int(f'g+fg')\dx$
\end{beweise}

\begin{beispiele}
\item $\int \log x\dx=\int\underbrace{1}_{f'}\underbrace{\log x}_{g}\dx=x\log x-\int x\frac{1}{x}dx=x\log x - x$ auf $(0, \infty)$.
\item $\int \sin^2 x\dx = \int\underbrace{\sin x}_{f'}\underbrace{\sin x}_{g}\dx = -\cos x\sin x - \int{-\cos^2 x\dx}=-\cos x\sin x + \int(1-\sin^2 x)\dx=-\cos x\sin x + x-\int\sin^2 x\dx$

$\folgt \int\sin^2\dx =\frac{1}{2}(x-\cos x \sin x)$ auf $\MdR$.
\item $\int \underbrace{x}_{f'}\underbrace{e^x}_{g}\dx=\frac{1}{2}x^2e^x-\int\frac{1}{2}x^2e^x\dx$ \emph{komplizierter!}\\
$\int\underbrace{x}_{f}\underbrace{e^x}_{g'}=xe^x-\int{e^x\dx}=xe^x-e^x$
\end{beispiele}

\begin{satz}[Substitutionsregeln]
Sei $f\in C(I)$ und $g \in C^1(J)$ und $g(J)\subseteq I$.
\begin{liste}
\item Ist $J=[\alpha, \beta]\folgt$ $$\int_{\alpha}^{\beta}f(g(t))g'(t)\dt=\int_{g(\alpha)}^{g(\beta)}f(t)\dt$$
\item Auf $J$ gilt: $$\int f(g(t))g'(t)\dt=\int f(x)\dx|_{x=g(t)}$$
\item $g$ sei auf $J$ streng monoton $\folgt$ auf $I$ gilt: $$\int f(x)\dx=\int f(g(t))g'(t)\dt|_{t=g^{-1}(x)}$$.
\end{liste}
\end{satz}

\begin{merkregel}
Ist $y=y(x)$ differenzierbar, so schreibt man f"ur $y'$ auch $\frac{dy}{dx}$. In 23.16 substituiere $x=g(t)$ (fasse also $x$ als Funktion von $t$ auf $\folgt g'(t)=\frac{dx}{dt}$ \glqq$\folgt dx=g'(t)dt$\grqq.
\end{merkregel}

\begin{beweise}
\item[(2)]
Sei $F$ eine Stammfunktion von $f$ auf $I$. $G(t):=F(g(t))\ (t\in J)$. $G'(t)=F'(g(t))g'(t)=f(g(t))g'(t)\ (t\in J)\folgt G$ ist eine Stammfunktion von $(f\circ g)g'$ auf $J \folgt$ (2)
\item[(1)]$\int_{\alpha}^{\beta}f(g(t))g'(t)\dt\overset{23.5}{=}G(\beta)-G(\alpha)=F(g(\beta))-F(g(\alpha))\overset{23.5}{=}\int_{g(\alpha)}^{g(\beta)}f(x)\dx$.
\item[(3)]$\int f(g(t))g'(t)\dt|_{t=g^{-1}(x)}=G(g^{-1}(x))=F(g(g^{-1}(x)))=F(x)$
\end{beweise}

\begin{beispiele}
\item $\int_0^1\sqrt{1-x^2}\dx$ (Substitution $x=\sin t,\ t=0\folgt x=0,\ t=\frac{\pi}{2}\folgt x=1,\dx=\cos t\dt$). $\int_0^1\sqrt{1-x^2}\dx=\int_0^{\frac{\pi}{2}}\sqrt{1-\sin^2 t}\cos t\dt=\int_0^\frac{\pi}{2}|\cos t|\cos t\dt=\int_0^{\frac{\pi}{2}}\cos^2t\dt=\int_0^{\frac{\pi}{2}}(1-\sin^2t)\dt=t-\frac{1}{2}(t-\cos t\sin t)|_0^{\frac{\pi}{2}}=\frac{\pi}{4}$.
\item $\int\frac{1}{x\log x}\dx$ (Substitution $x=e^t,\ t=\log x,\dt=\frac{1}{x}\dx$). $\int\frac{1}{x\log x}=\int\frac{1}{t}\dt=\log t=\log(\log(x))$ auf $(1,\infty)$.
\end{beispiele}

\begin{definition}
\begin{liste}
\item Seien $p$ und $q$ Polynome und $q \neq 0.$ Dann heißt $\frac{p}{q}$ eine \begriff{rationale Funktion}.

$\frac{p}{q}$ hat eine Darstellung der Form $\frac{p}{q} = p_1 + \frac{p_2}{q}$, wobei $p_1,p_2$ Polynome und $\frac{p_2}{q}$ \begriff{echt gebrochen rational}, d.h.: $\text{Grad } p_2 < \text{Grad } q$.

\item Seien $b,c \in \MdR$. Dann heißt das Polynom $x^2+bx+c$ \begriff{unzerlegbar} über $\MdR :\equizu 4c-b^2 > 0\quad(\equizu x^2+bx+c \ne 0\ \forall x \in \MdR)$

\item Ein \begriff{Partialbruch} ist eine rationale Funktion der Form $$\frac{A}{(x-x_0)^k}$$wobei $A,x_0 \in \MdR,\ k \in \MdN$, oder $$\frac{Ax+B}{(x^2+bx+c)^k}$$wobei $A,B,b,c \in \MdR,\ k \in \MdN$ und $x^2+bx+c$ unzerlegbar über \MdR.
\end{liste}
\end{definition}

\newcommand{\fint}[2]{\int{\frac{#1}{#2}\dx}}

\begin{satz}[Integration von rationalen Funktionen]
Es seien $b,c,x_0 \in \MdR,\ m \in \MdN,\ m > 1,\ p(x) := x^2+bx+c$ und $D := 4c-b^2 > 0$
\begin{liste}
\item $\displaystyle{\fint{1}{x-x_0} = \log{|x-x_0|}}$
\item $\displaystyle{\fint{1}{(x-x_0)^m} = \frac{-1}{m-1} \cdot \frac{1}{(x-x_0)^{m-1}}}$
\item $\displaystyle{\fint{1}{p(x)} = \frac{2}{\sqrt{D}} \arctan{\left( \frac{2x+b}{\sqrt{D}} \right)}}$
\item $\displaystyle{\fint{1}{p(x)^m} = \frac{1}{(m-1)D} \cdot \frac{2x+b}{p(x)^{m-1}} + \frac{4m-6}{(m-1)D} \fint{1}{p(x)^{m-1}}}$
\item $\displaystyle{\fint{x}{p(x)} = \frac{1}{2} \log{(p(x))} - \frac{b}{2} \fint{1}{p(x)}}$
\item $\displaystyle{\fint{x}{p(x)^m} = \frac{-1}{2(m-1)} \cdot \frac{1}{p(x)^{m-1}} - \frac{b}{2} \fint{1}{p(x)^m}}$
\end{liste}
\end{satz}

\begin{beweise}
\item klar
\item klar
\item  $p(x) = x^2+bx+c = x^2+bx+\frac{b^2}{4} + c - \frac{b^2}{4} = (x+\frac{b}{2})^2 + \frac{D}{4} = \frac{D}{4}(\frac{4}{D}(x+\frac{b}{2})^2 + 1) = \frac{D}{4}((\frac{2x+b}{\sqrt{D}})^2 + 1)= \frac{D}{4}(t^2+1),$ wobei $t = \frac{2x+b}{\sqrt{D}},$ also $x = \frac{\sqrt{D}t - b}{2}$

$\folgt \int\frac{1}{p(x)}\dx =$ (Substitution $t=\frac{2x+b}{\sqrt{D}},\  dx=\frac{\sqrt{D}}{2}\dt$) $\frac{4}{D} \int\frac{1}{t^2+1} \cdot \frac{\sqrt{D}}{2}\dt = \frac{2}{\sqrt{D}} \int\frac{1}{1+t^2}\dt = \frac{2}{\sqrt{D}} \arctan t = \frac{2}{\sqrt{D}} \arctan(\frac{2x+b}{\sqrt{D}})$

\item Übung, partielle Integration
\item $\int\frac{x}{p(x)}\dx = \frac{1}{2} \int\frac{2x+b-b}{p(x)}\dx = \frac{1}{2}\int \underbrace{\frac{p'(x)}{p(x)}}_{\log(p(x))}\dx - \frac{b}{2} \int\frac{1}{p(x)}\dx$

\item Übung, partielle Integration
\end{beweise}

\begin{definition}
\begin{liste}
\item Sei $Z = \{x_0,\ldots,x_n\} \in \Z,\ I_j = [x_{j-1},x_j]\ (j=1,\ldots,n)$

$|Z| := \max \{|I_j| : j=1,\ldots,n\}$ heißt das \begriff{Feinheitsmaß} von $Z$.

\item $\Z^* := \{(Z,\xi) : Z \in \Z,\ \xi$ ist passend zu $Z \}$. Eine Folge $((Z_n,\xi^{(n)}))$ in $\Z^*$ heißt eine \begriff{Nullfolge} $:\equizu |Z_n| \to 0\ (n \to \infty)$
\end{liste}
\end{definition}

\begin{satz}[Folgen von Zerlegungen mit $|Z_n| \to 0$]
$f: [a,b] \to \MdR$ sei beschränkt; sei $\gamma \ge 0$ mit: $|f(x)| \le \gamma\ \forall x \in [a,b].$
\begin{liste}
\item Sind $Z_1,Z_2 \in \Z$ und $Z_1 \subseteq Z_2$ und enthält $Z_2$ genau $p$ Teilpunkte mehr als $Z_1$, dann gilt: $$s_f(Z_2) \le s_f(Z_1) + 2p\gamma|Z_1|\text{ und}$$
$$S_f(Z_2) \ge S_f(Z_1) - 2p\gamma|Z_1|.$$

\item $\forall \ep > 0\ \exists \delta > 0\ \forall Z \in \Z$ mit $|Z| < \delta$: $$s_f(Z) > \uintab{f} - \ep,\ S_f(Z) < \ointab{f} + \ep.$$

\item Ist $(Z_n)$ eine Folge in $\Z$ mit $|Z_n| \to 0$, dann gilt: $$s_f(Z_n) \to \uintab{f},\ S_f(Z_n) \to \ointab{f}.$$
\end{liste}
\end{satz}

\begin{beweise}
\item Übung, es genügt den Fall $p=1$ zu betrachten.
\item Beweis nur für Untersummen. Sei $\ep > 0.\ \exists Z_1 \in \Z: s_f(Z_1) > \uintab{f} - \frac{\ep}{2};\ Z_1$ habe $p$ Teilpunkte. $\delta := \frac{\ep}{4\gamma p}.$

Sei $Z \in \Z$ und $|Z| < \delta.\ Z_2 := Z \cup Z_1 \in \Z;\ Z_2$ hat höchstens $p$ Teilpunkte mehr als $Z \folgt s_f(Z) = \underbrace{s_f(Z) - s_f(Z_2)}_{\underset{\text{(1)}}{>} -2p\gamma|Z|} + \underbrace{s_f(Z_2)}_{\ge s_f(Z_1)} > -2p\gamma|Z| + s_f(Z_1) > -\underbrace{2\gamma p \delta}_{=\frac{\ep}{2}} + \uintab{f} - \frac{\ep}{2} = \uintab{f} - \ep.$

\item Nur für Untersummen. $A := \uintab{f},\ s_n := s_f(Z_n).$ Sei $\ep > 0.$ Aus (2) folgt dann: $\exists \delta > 0: s_f(Z) > A-\ep\ \forall Z \in \Z$ mit $|Z| < \delta.\ \exists n_0 \in \MdN: |Z_n| < \delta\ \forall n \ge n_0.$ Also: $s_n \to A\quad(n \to \infty).$

\end{beweise}

\begin{beispiel}
$$a_n := \sum_{j=1}^n{\frac{\sqrt{j}}{n^{3/2}}}.\text{ Behauptung}: a_n \to \frac{2}{3}$$
\begin{beweis}
$$a_n = \sum_{j=1}^n{\underbrace{\sqrt{\frac{j}{n}}}_{= f(\frac{j}{n})} \frac{1}{n}},\ f(x) = \sqrt{x},\ x \in [0,1].$$

$Z_n = \{0,\frac{1}{n},\ldots,\frac{n}{n}\} \folgt a_n = S_f(Z_n) \underset{\text{23.18(3)}}{\overset{n \to \infty}{\to}} \uint_0^1\sqrt{x}\dx = \int_0^1 \sqrt{x}\dx = \frac{2}{3}$
\end{beweis}
\end{beispiel}

\begin{satz}[Riemannsche Definition des Integrals mit Nullfolgen]
$f: [a,b] \to \MdR$ sei beschränkt. $f \in R[a,b] \equizu \exists S \in \MdR: \sigma_f(Z_n,\xi^{(n)}) \to S\ (n \to \infty)$ für jede Nullfolge $((Z_n,\xi^{(n)})) \in \Z^*.$ In diesem Fall gilt: $S = \intab{f}$.
\end{satz}

\begin{beweis}
\begin{description}
\hin $S := \intab{f}$. Sei $((Z_n,\xi^{(n)})) \in \Z^*$ eine Nullfolge. Dann: $$\underbrace{s_f(Z_n)}_{\overset{\text{23.18}}{\to}S}  \le \sigma_f(Z_n,\xi^{(n)}) \le \underbrace{S_f(Z_n)}_{\overset{\text{23.18}}{\to}S}\ \forall n \in \MdN.$$

$\folgt \sigma_f(Z_n,\xi^{(n)}) \to S\ (n \to \infty)$.

\zurueck Sei $\ep > 0$ und $(Z_n)$ eine Folge in $\Z$ mit $|Z_n| \to 0.$ Wie im Beweis von 23.12: $\forall n \in \MdN\ \exists \xi^{(n)},\eta^{(n)}$ passend zu $Z_n$ mit: $$S_f(Z_n) - \ep < \sigma_f(Z_n,\xi^{(n)});\ \sigma(Z_n,\eta^{(n)}) < s_f(Z_n) + \ep$$

Aus 23.18(3) folgt für $n \to \infty:\ \ointab{f} - \ep \le S \le \uintab{f} + \ep\ \forall \ep > 0 \folgtwegen{\ep \to 0+} \ointab{f} \le S \le \uintab{f} \folgt f \in R[a,b]$ und $\intab{f} = S$.
\end{description}
\end{beweis}

\begin{beispiel}
\textit{Bemerkung: Dies ist ein Beispiel zum nächsten Satz, nicht zum vorherigen.}

$f_n(x) = \frac{1}{n} \sin (nx)\ (n \in \MdN,\ x \in [0,\pi]);\ |f_n(x)| = \frac{1}{n} |\sin(nx)| \le \frac{1}{n}\ \forall x \in [0,\pi].$

$\folgt (f_n)$ konvergiert gleichmäßig auf $[0,\pi]$ gegen $f \equiv 0$.

$f_n'(x) = \cos(nx),\ f_n'(\pi) = \cos(n\pi) = (-1)^n.$ Das heißt: $(f_n')$ konvergiert auf $[0,\pi]$ \emph{nicht} punktweise.
\end{beispiel}

\begin{satz}[Gleichmäßige Konvergenz der Stammfunktion]
$(f_n)$ sei eine Folge in $C^1[a,b],\ x_0 \in [a,b]$ und es gelte:
\begin{itemize}
\item[(i)] $(f_n(x_0))$ konvergiert
\item[(ii)] $(f_n')$ konvergiert gleichmäßig auf $[a,b]$ gegen $g:[a,b] \to \MdR.$
\end{itemize}

Dann konvergiert $(f_n)$ gleichmäßig auf $[a,b]$ und für $f(x) := \lim_{n\to\infty} f_n(x)\ (x \in [a,b])$ gilt: $f \in C^1[a,b]$ und $f'=g$ auf $[a,b].$

Also: $(\lim_{n\to\infty} f_n(x))' = f'(x) = g(x) = \lim_{n\to\infty} f_n'(x)\ \forall x \in [a,b].$
\end{satz}

\newcommand{\gehtwegen}[1]{\overset{#1}{\to}}
\newcommand{\gehtnach}[1]{\overset{\text{#1}}{\to}}

\begin{beweis}
O.B.d.A.: $x_0=a$ und $f_n(a) \to 0\ (n \to \infty).\ f(x):=\int_a^x g(t)\dt\ (x \in [a,b]).$ Aus 19.2 folgt: $g \in C[a,b].$

Damit wegen 23.13: $f \in C^1[a,b]$ und $f'=g$ auf $[a,b].$

Sei $x \in [a,b]: f_n(x) - \underbrace{f_n(a)}_{\to 0} \gleichnach{23.5} \int_a^x f_n'(t)\dt \gehtnach{23.6} \int_a^x g(t)\dt = f(x).$

$\folgt (f_n)$ konvergiert punktweise gegen $f$.

Für $x \in [a,b]: |f_n(x)-f(x)| = |f_n(x)-f_n(a)-f(x)+f_n(a)| = |\int_a^x (f_n'(t)-g(t))\dt + f_n(a)| \le \int_a^x |f_n'-g|\dt + |f_n(a)| \le \int_a^b |f_n'-g|\dt + |f_n(a)| =: c_n$

Wegen Voraussetzung (ii) konvergiert $(|f_n'-g|)$ auf $[a,b]$ gleichmäßig gegen $0$. Wegen 23.6 folgt damit: $\int_a^b |f_n'-g| \dt \to 0\ (n \to \infty) \folgt c_n \to 0\ (n \to \infty) \folgt (f_n)$ konvergiert gleichmäßig auf $[a,b]$ gegen $f$.
\end{beweis}

Wir können nun den Satz 21.9 beweisen.

\begin{beweis}
Sei $a<b$ und $[a,b] \subseteq I.\ f_n(x) := \sum_{k=0}^n a_k x^k,\ f_n'(x) = \sum_{k=1}^n ka_kx^{k-1},\ g(x) := \sum_{k=1}^\infty ka_kx^{k-1}$

Aus 19.1 folgt: $(f_n)$ und $(f_n')$ konvergieren auf $[a,b]$ gleichmäßig gegen $f$ bzw. $g$. Wegen unserem neuen Satz 23.20 nun ist $f$ auf $[a,b]$ differenzierbar und $f'=g$ auf $[a,b]$. $[a,b] \subseteq I$ beliebig $\folgt$ Beh.
\end{beweis}

\end{document}
