\documentclass[a4paper,twoside,DIV15,BCOR12mm]{scrbook}

\usepackage{mathe}
\usepackage{saetze-kuehnlein}
\usepackage{skull}
\usepackage{faktor}
\usepackage{enumerate}

\newcommand{\otm}{\stackrel{\circ}{\subset}} % Offene Teilmenge
\newcommand{\At}{\mathcal A}
\renewcommand{\da}{\coloneqq}
\newcommand{\ad}{\eqqcolon}
\newcommand{\V}{\mathcal V}
\newcommand{\kan}{\text{kan}}
\DeclareMathOperator{\cut}{Cut}
\DeclareMathOperator{\inj}{Inj}
\DeclareMathOperator{\dist}{Dist}
\DeclareMathOperator{\vol}{vol}

%\renewcommand*{\othersectionlevelsformat}[3]{\S\,#3\autodot\enskip}

\author{Die Mitarbeiter von \url{http://mitschriebwiki.nomeata.de/}}
\title{Topologie}
\makeindex

\begin{document}
\maketitle
 
%\renewcommand{\thechapter}{\Roman{chapter}}
%\chapter{Inhaltsverzeichnis}
\stepcounter{chapter}
%\renewcommand{\tocname}{bla}
\addcontentsline{toc}{chapter}{\protect\numberline {\thechapter}Inhaltsverzeichnis}
\tableofcontents

 % Vorwort

\chapter{Vorwort}
%\addcontentsline{toc}{chapter}{Vorwort}

\section*{Über dieses Skriptum}
Dies ist ein Mitschrieb der Vorlesung \glqq Topologie\grqq\ von Herrn Dr. Kühnlein im
Wintersemester 07/08 an der Universität Karlsruhe (TH).
%Die Mitschriebe der Vorlesung werden mit ausdrücklicher Genehmigung von Herrn Dr. Leuzinger hier veröffentlicht.
Herr Dr. Kühnlein ist für  den
Inhalt nicht verantwortlich.

\section*{Wer}
Gestartet wurde das Projekt von Joachim Breitner.


\section*{Wo}
Alle Kapitel inklusive \LaTeX-Quellen können unter \url{http://mitschriebwiki.nomeata.de} abgerufen werden.
Dort ist ein von Joachim Breitner programmiertes \emph{Wiki}, basierend auf \url{http://latexki.nomeata.de} installiert. 
Das heißt, jeder kann Fehler nachbessern und sich an der Entwicklung
beteiligen. Auf Wunsch ist auch ein Zugang über \emph{Subversion} möglich.

\setcounter{chapter}{0}
%\renewcommand{\thechapter}{\arabic{chapter}}
\renewcommand{\thechapter}{\Roman{chapter}}

% Fixme
%\renewcommand{\thesection}{§\Roman{chapter}.\arabic{section}}

\chapter{Einstieg}

\section{Historischer Überblick}

\emph{In diesem Mitschrieb ausgelassen}

\section{Ein paar topologische Argumente}

\subsection{Fundamentalsatz der Algebra}
\begin{satz}
Gegeben ist ein nichtkonstantes Polynom $f\in \MdC[X]$. Dann hat $f$ eine Nullstelle in $\MdC$.
\end{satz}

\begin{beweis}[Skizze] Sei $f=a_0 + a_1X + a_2X^2 + \cdots + a_dX^d$, mit $a_d \ne 0$, $d \ge 1$. Ist $a_0 = 0$, so ist $x=0$ Nullstelle. Wir nehmen also an: $a_0 \ne 0$.

Betrachte $f(R\cdot S^1)$, $R\cdot S^1 = \{ z \in \MdC \mid |z|=R\}$. Es ist:
\[
f(X) = a_d \cdot X^d \cdot (1+\frac{a_{d-1}}{a_dX} + \frac{a_{d-1}}{a_dX^2} + \cdots + \frac{a_0}{a_dX^d})
\]
Für große $R$ und $z\in R\cdot S^1$ ist also $f(z) \approx a_dz^d$, und $f(R\cdot S^1)$ ist in etwa ein $d$-mal umlaufener Kreis mit Radius $|a_d|\cdot R^d$.

Für kleine $r$ dagegen\dots

Beim „Zusammenziehen“ des Kreises $R\cdot S^1$ zum Kreis $r\cdot S^1$ muss irgendwann eine Nullstelle von $f$ getroffen, denn $0$ liegt im „Inneren“ von $f(R\cdot S^1)$, aber nicht im Inneren von $f(r\cdot S^1)$.
\end{beweis}

\subsection{Torus und Sphäre}
Gibt es eine surjektive stetige Abbildung von $\mathbb T$ nach $S^2$?

Dabei ist $\mathbb T$ der Torus, und $S^1$ die Kugeloberfläche. Die Antwort ist Nein, anschaulich gezeigt.

\subsection{Eulers Polyederformel}

Sei $P$ ein konvexes Polyeder in $\MdR^3$. Dann gilt: $E - K + F = 2$, wobei $E$ die Anzahl der Ecken, $K$ die Anzahl der Kanten und $F$ die Anzahl der Flächen ist.

\begin{beweis}[Idee]
Lege das Polyeder in die Einheitskugel, montiere ein Ventil und blase auf, bis das Polyeder in die Einheitsphäre stößt. Dies unterteilt $S^2$ in Flächen, Kanten und Ecken.

Die Unterteilung zu zwei Polyedern besitzen eine gemeinsame Verfeinerung. Mögliche Schritte dabei sind neue Kanten und neue Ecken einfügen. Diese Schritte verändern jedoch die Summe in der Eulerschen Formel nicht, das heißt je zwei Polyeder liefern die gleiche Summe.
\end{beweis}

\subsection{Satz vom Igel}

\begin{satz}
Jeder stetig gekämmte Igel besitzt mindestens einen Glatzpunkt.

Präzisierung: Es gibt kein stetiges Vektorfeld in der $S^2$ ohne Nullstelle.
\end{satz}

\subsection{}

Reelle Divisionsalgebren sind $\MdR$-Vektorräume, auf denen eine $\MdR$-bilieare Multiplikation mit 1-Element gegeben ist, die Distributivgesetzte gelten und dass jedes Element $x\ne 0$ ein Inverses besitzt.

Beispiele sind etwa $\MdR$, $\MdC$, $\mathbb H$, $\mathbb O$ (Cayley-Oktaven). Das sind alle endlichdimensionalen Divisionsalgebren, denn: Wen nich auf $\MdR^n$ die Struktur einer reellen Divisionsalgebra habe, dann bekomme ich eine Verknüpfung auf $S^{n-1}\subseteq \MdR^n$:
\[ (x,y) \mapsto \frac{x\cdot y}{\|x\cdot y\|} \]
Diese Verknüpfung ist fast so gut wie eine Gruppenstruktur auf $S^{n-1}$. Daraus, und das müssen Sie mir heute einfach glauben, folgt, dass es ein Vektorfeld ohne Nullstelle auf $S^{n-1}$ gibt. Dann muss $n-1\in\{0,1,3,7\}$ sein, es gibt also nur 1,2,4 und 8-dimensionale Divisionsalgebren.

\subsection{Brouwers Fixpunktsatz}
\begin{satz}
Wenn $f:[0,1]^n \to [0,1]^n$ stetig ist, so hat $f$ einen Fixpunkt: $f(x) = x$
\end{satz}


\chapter{Satz um Satz (hüpft der Has)}
\listtheorems{satz,wichtigedefinition}

\renewcommand{\indexname}{Stichwortverzeichnis}
\addtocounter{chapter}{1}
\addcontentsline{toc}{chapter}{\protect\numberline {\thechapter}Stichwortverzeichnis}
\printindex
\end{document}
