\documentclass[a4paper,12pt,index=toc]{scrbook}

\usepackage{ifxetex}

\ifxetex %xelatex-Zeugs:
  \usepackage{fontspec}
  \usepackage{polyglossia}
%\setdefaultlanguage[spelling=new,latesthyphen=true,babelshorthands=true]{german}
  \setdefaultlanguage[spelling=new,latesthyphen=true]{german}

  \setmainfont[Mapping=tex-text,Numbers=OldStyle,Ligatures=Rare]{Linux Libertine O}
  \setsansfont[Mapping=tex-text,Numbers=OldStyle,Ligatures=Rare]{Linux Biolinum O}

\else % für pdflatex...
  \usepackage[utf8x]{inputenc}
  \usepackage[german]{babel}
  \usepackage[T1]{fontenc}
  \usepackage{etoolbox} % in polyglossia enthalten ...
\fi

\usepackage{csquotes}
\usepackage{hyperref,amsmath,amsfonts,amssymb,bbm,enumitem}
\usepackage[thmmarks,amsmath,thref,hyperref]{ntheorem}
%\usepackage{thmtools}
\usepackage{cleveref}

%\def\filename{alggeo.sublinks}
%\newcommand{\readfromfile}[1]{\input{#1}}
%% coole Datei mit Formaten einlesen und dann neue erstellen:
%% Existenz überprüfen:
%\newread\foo
%\immediate\openin\foo=\filename\relax
%\ifeof\foo\relax%überprüfen ob schon vorhanden
%\else\readfromfile\filename\relax\fi
%\immediate\closein\foo
%\newwrite\blah
%\immediate\openout\blah=\filename\relax
%%% und am Ende wieder zu machen:
%\AtEndDocument{\immediate\write\blah{\string\endinput}\relax \immediate\closeout\blah}
\usepackage{coolthms}

\usepackage{tikz}
\usetikzlibrary{matrix,arrows,calc}

\deffootnote[1em]{1em}{1em}{\textsuperscript{\thefootnotemark}\ }
\setlength{\parindent}{0pt}
\setlength{\parskip}{3pt}

%% Standardformat
\theorempreskipamount0pt plus 2pt\relax
\theorempostskipamount0pt plus 2pt\relax
\theoremheaderfont{\hspace*{-2.5em}\scshape}%
\theorembodyfont{\normalfont}%
\theoremseparator{:}%
\theoremindent3em\relax%
\theoremnumbering{arabic}%
\theoremsymbol{}%
\theoremstyle{keinenummern} % ab hier ohne Nummern
\newtheorem{ziel}{Ziel}
\newtheorem{motivation}{Motivation}
\newtheorem{zeige}{Zeige}
\newtheorem{q}{Frage}

% "Standardumgebungen" alle in coolthms.sty enthalten...

% Warnungen :)
\usepackage{manfnt,environ,ragged2e}
\NewEnviron{w}{\begin{center}\begin{tikzpicture}%
\node[text width=0.8\textwidth,rounded corners,fill=orange,inner sep=1ex]{%
  \hbox{\hspace{1em}\makebox[\width][t]{\raisebox{1.3ex}{\lhdbend}}\hspace{1.2em}%
  \begin{minipage}[c]{.85\textwidth}\smallskip\FlushLeft\BODY\medskip\end{minipage}%
}};%\draw[->] (0,0) -- (current page.east);%
\end{tikzpicture}\end{center}\par}

\usepackage{textcomp}
\newcommand{\sect}[1]{% nicht zu breit:
\begin{tikzpicture}%
\node[right,text width=\textwidth-10pt,rounded corners,fill=blue!15,inner sep=5pt]{%
  \hbox{%
  \begin{minipage}[c]{.96\textwidth}\section{#1}\end{minipage}%
  \hfill\makebox[\width][t]{\raisebox{-5pt}{\raisebox{4pt}{\textleaf}}}%
}};\end{tikzpicture}%
\nopagebreak[4]\markright{\thesection. #1}}

\DeclareMathAlphabet{\mathpzc}{OT1}{pzc}{m}{it}

\def\C{\mathbb{C}}
\def\CC{\mathcal{C}}
\def\A{\mathbb{A}}
\def\V{\mathfrak{V}}
\def\VV{\mathcal{V}}
\def\I{\mathfrak{I}}
\def\II{\mathcal{I}}
\def\O{\mathcal{O}}
\def\P{\mathbb{P}}
\newcommand{\D}{\mathfrak{D}}
\newcommand{\DD}{\mathcal{D}} % Dehomogenierung und D_{p}
\renewcommand{\d}{\mathrm{d}}
\renewcommand{\H}{\mathcal{H}} % Homogenisierung
\def\B{\mathcal{B}}
\def\S{\mathcal{S}}
\newcommand{\J}{\mathcal{J}}
\newcommand{\K}{\mathcal{K}}
\newcommand{\AffVar}{\mathpzc{AffVar}}
\newcommand{\RedAlg}{\mathpzc{RedAlg}}
\newcommand{\Grp}{\mathpzc{Grp}}
\newcommand{\Var}{\mathpzc{Var}}
\newcommand{\Cat}{\mathpzc{Cat}}
\newcommand{\Set}{\mathpzc{Set}}
\newcommand{\Top}{\mathpzc{Top}}
\newcommand{\Ob}{\mathpzc{Ob}}
\newcommand{\Ar}{\mathpzc{Ar}}
\newcommand{\Cl}{\mathpzc{Cl}}
\newcommand{\F}{\mathcal{F}}
\newcommand{\FF}{\mathbb{F}}
\def\G{\mathcal{G}}
\newcommand{\g}{\mathfrak{g}}
\renewcommand{\L}{\mathcal{L}}
\def\l{\mathfrak{l}}
\def\ll{\mathpzc{l}}
\def\T{\mathcal{T}}
\def\M{\mathcal{M}}
\def\m{\mathfrak{m}}
\def\U{\mathfrak{U}}
\newcommand{\Hom}{\mathrm{Hom}}
\newcommand{\Rat}{\mathrm{Rat}}
\newcommand{\Mor}{\mathrm{Mor}}
\newcommand{\Kern}{\operatorname{Kern}}
\newcommand{\Bild}{\operatorname{Bild}}
\newcommand{\Rang}{\operatorname{Rang}}
\newcommand{\hoehe}{\operatorname{ht}}
\newcommand{\Spec}{\operatorname{Spec}}
\newcommand{\Sing}{\operatorname{Sing}}
\newcommand{\Reg}{\operatorname{Reg}}
\newcommand{\Div}{\operatorname{Div}}
\newcommand{\Divh}{\Div_{\mathrm{H}}}
\renewcommand{\div}{\operatorname{div}}
\newcommand{\ord}{\operatorname{ord}}
\newcommand{\Abb}{\mathrm{Abb}}
\newcommand{\Func}{\mathrm{Func}}
\newcommand{\Nat}{\mathrm{Nat}}
\newcommand{\id}{\mathrm{id}}
\newcommand{\dom}{\operatorname{dom}}
\newcommand{\Def}{\operatorname{Def}}
\newcommand{\cod}{\operatorname{cod}}
\newcommand{\Cone}{\operatorname{Cone}}
\newcommand{\trdeg}{\operatorname{trdeg}}
\newcommand{\op}{\mathrm{op}}
\newcommand{\Eq}{\mathrm{Eq}}
\renewcommand{\phi}{\varphi}
\newcommand{\ep}{\varepsilon}
\newcommand{\da}{:=}
\newcommand{\leer}{\ensuremath{\varnothing}}
\newcommand{\Quot}{\operatorname{Quot}}
\newcommand{\GL}{\operatorname{GL}}
\newcommand{\sign}{\operatorname{sign}}
\newcommand{\card}[1]{|#1|}
\newcommand{\restrict}[1]{|_{#1}}
\let\olddotsc\dotsc % für Bew. zu 2.7.3 und taylor-entwicklung (3.3), 3.3.4, 3.3.10
\renewcommand{\dotsc}{\ensuremath{\!...}}
% punktloses mit ``quer''
\newcommand{\iq}{\Bar{\imath}}

% alle tilden sollen wide sein.
\newcommand{\schlange}[1]{\widetilde{#1}}

% Graßmann-Dinger:
\let\grassmann\bigwedge
\def\bigwedge{\begingroup\textstyle\grassmann\endgroup}

\newcommand{\set}[1]{\ensuremath{\mathbb{#1}}}
\newcommand{\Q}{\set{Q}}
\newcommand{\N}{\set{N}}
\newcommand{\R}{\set{R}}
\newcommand{\Z}{\set{Z}}

%coole tikzpfeile :)
\usepackage{tikzpfeile}

\newcommand{\dach}{\widehat}
\newcommand{\oldbar}[1]{\bar{#1}}
\def\Bar#1{\ensuremath\overline{#1}}

\usepackage{quotienten}

%Listenformate:
\setenumerate[1]{leftmargin=*,labelindent=\parindent,label=(\alph*)}
\setenumerate[2]{leftmargin=*,labelindent=\parindent,label=(\roman*)}
%seperate listen im Beweise
\newlist{prooflist}{enumerate}{2}
\setlist[prooflist]{leftmargin=*,labelindent=\parindent,label=(\arabic*)}
\setlist[prooflist,2]{label=(\roman*)}

% index
\usepackage{makeidx}
\makeindex
%\let\saveemph\emph
%\def\emph#1{\saveemph{#1}\index{#1}}

%Polynomringe
\newcommand{\polyx}[1][n]{\ensuremath%
  [X_{1},\dotsc,X_{#1}]}
% im projektiven wollen wir mit 0 anfangen; mehrere optionale Argumente sind leider nicht ohne weiteres möglich...
\newcommand{\ppolyx}[1][n]{\ensuremath%
  [X_{0},\dotsc,X_{#1}]}

% sectionformat:
\renewcommand{\thesection}{\arabic{section}}
\renewcommand{\thechapter}{{\scshape\roman{chapter}}}
% Kapitälchen in Kopfzeilen
\setkomafont{pageheadfoot}{\normalfont}

\author{JProf. Dr. Gabriela Weitze-Schmithüsen}
\publishers{\small Jonathan Zachhuber, Jens Babutzka und Michael Fütterer \\ Karlsruher Institut für Technologie}
\subject{Skript zur Vorlesung}
\title{Algebraische Geometrie I}
\titlehead{\includegraphics[scale=0.6]{kitlogo}}
\date{Wintersemester 2010/2011}

\begin{document}

\maketitle
\tableofcontents

% 18.10.10
%\setcounter{chapter}{-1}
\chapter*{Motivation}

\begin{ziel}
Untersuche Nullstellenmengen von Polynomen: Für eine Menge von Polynomen
\[p_{1},\dotsc,p_{r}\in k\polyx\]
über einem Körper $k$ möchte man die Menge der Nullstellen
\[\{x=(x_{1},\dotsc,x_{n})\mid p_{i}(x)=0\text{ für alle }i\}\]
analysieren.
\end{ziel}

\begin{nbsp}
\begin{enumerate}
\item Betrachte $ax^{2}+by^{2}=1\iff x^{2}+y^{2}-1=0$ über $k=\R$. Das liefert eine Ellipse, für $a=b=1$ einen Kreis.
%Bild...
\item Betrachte $x^{2}+y^{2}=z^{2}$.
%Bild...
\item Betrachte (b) mit $x=1$: Dann ist $1+y^{2}=z^{2}\iff 1=z^{2}-y^{2}$, also eine Hyperbel.
%Bild
\item Bei linearen Gleichungen sehen wir mit Hilfe der linearen Algebra, das wir affine Unterräume erhalten.
\item Die Lösungsmengen sind abhängig vom Körper, z.B. sehen wir für $k=\Z/3\Z$ hat das Polynom $X^{3}-X$ als Lösungsmenge ganz $k$.
\end{enumerate}
\end{nbsp}

% Rest kommt noch ;)

\chapter{Die Kategorie der affinen Varietäten}\label{kap1}
\sect{Affine Varietäten und Verschwindungsideale}

Sei $k$ stets ein Körper.

\begin{dfn}
Eine Teilmenge $V\subseteq k^{n}$ heißt \emph{affine Varietät\index{affine Varietät}}, wenn
es eine Menge von Polynomen $\F\subseteq k\polyx$
mit \[V=\V(\F):=\{x=(x_{1},\dotsc,x_{n})\in k^{n}\mid f(x)=0\;\forall f\in\F\}\] gibt.
\end{dfn}

\begin{bsp}\label{bsp1.2}
\begin{enumerate}
\item $k^{n}=\V(\{0\})$
\item $\leer=\V(\{1\})$
\end{enumerate}
\end{bsp}

\begin{q}Wie eindeutig ist das $\F$? Zum Beispiel liefern Produkte und Summen von Polynomen keine neuen Nullstellen.
\end{q}

\begin{nbsp}\begin{enumerate}
\item $\V(x^{2}+y^{2}+z^{2}-1) \supsetneq \V(x^{2}+y^{2}+z^{2}-1, z-\frac{1}{2})$
\item $\V(x^{2}+y^{2}+z^{2}-1, z-\frac{1}{2}) = \V(x^{2}+y^{2}+z^{2}-1, z-\frac{1}{2}, x^{2}+y^{2}+z^{2}-1 + z-\frac{1}{2}).$
\end{enumerate}\end{nbsp}


\begin{bem}\label{1.1.3} Seien $\F_{1},\F_{2}\subseteq k\polyx$. Dann gilt:
\begin{enumerate}
\item\Label{1.1.3a} $\F_{1}\subseteq\F_{2}\implies \V(\F_{1})\supseteq \V(\F_{2})$
\item\Label{1.1.3b} Sei $(\F)$ das von $\F$ erzeugte Ideal. Dann gilt: $\V(\F) = \V((\F))$.
\item\Label{1.1.3c} Sei
\[\sqrt{(\F)}:=\{p\in k\polyx\mid\exists\, d\geq 1\text{ mit }p^{d}\in(\F)\}\]
das \emph{Radikalideal\index{Radikalideal}} von $(\F)$ ($\sqrt{(\F)}$ ist ein Ideal!). Dann gilt: $\V(\sqrt{(\F)})=\V(\F)$.
\item\Label{1.1.3d} Zu jeder affinen Varietät $V\subseteq k^{n}$ gibt es endlich viele Polynome $f_{1},\dotsc,f_{r}$ mit \[V=\V(\{f_{1},\dotsc,f_{r}\})=:\V(f_{1},\dotsc,f_{r}).\]
\end{enumerate}
\end{bem}

\begin{proof}\mbox{} %damit der link nicht überspring...
\ref{1.1.3a} und \ref{1.1.3b} folgen direkt aus der Definition.
\begin{enumerate}%\addtocounter{enumi}{2}
\item[\ref{1.1.3c}] Offenbar ist $\sqrt{(\F)}\supseteq(\F)$, also gilt nach \ref{1.1.3a} und \ref{1.1.3b}: 
\[\V(\sqrt{(\F)})\subseteq \V((\F))=\V(\F).\]

Für die andere Richtung: Sei $x=(x_{1},\dotsc,x_{n})\in \V(\F)$ und $f\in\sqrt{(\F)}$. Nach Definition existiert $d\in\N$, so dass $f^{d}\in(\F)$. Also gilt $f^{d}(x)=0$ und damit $f(x)=0$, demnach ist $x\in \V(\sqrt{(\F)})$.
\item[\ref{1.1.3d}] Nach \ref{1.1.3b} gilt $\V(\F)=\V((\F))$ und nach Hilberts Basissatz wird $(\F)$ von endlich vielen Elementen erzeugt.
\end{enumerate}
\end{proof}

\begin{db}
Sei $V\subseteq k^{n}$. Wir nennen
\[\I(V):=\{f\in k\polyx\mid f(x)=0\;\forall x\in V\}\]
das \emph{Verschwindungsideal von $V$\index{Verschwindungsideal}}. Es ist ein Ideal.
\end{db}

\begin{bem}\label{1.1.5} Es gilt:
\begin{enumerate}
\item\Label{1.1.5a} Seien $V_{1},V_{2}\subseteq k^{n}$ mit $V_{1}\subseteq V_{2}$. Dann ist $\I(V_{1})\supseteq \I(V_{2})$.
\item\Label{1.1.5b} Sei $V\subseteq k^{n}$. Dann ist $\I(V)$ ein Radikalideal.
\item\Label{1.1.5c} Sei $V$ eine affine Varietät. Dann gilt $V=\V(\I(V))$.

Es gilt sogar: $\I(V)$ ist das größte Ideal mit dieser Eigenschaft, d.h. für jedes Ideal $J\subseteq k\polyx$ mit $\V(J)=V$ folgt schon $J\subseteq \I(V)$.
\item\Label{1.1.5d} Seien $V_{1},V_{2}$ affine Varietäten. Dann gilt:
\[V_{1}=V_{2}\iff \I(V_{1})=\I(V_{2})\text{ und } V_{1}\subseteq V_{2}\iff \I(V_{1})\supseteq \I(V_{2}).\]
\end{enumerate}
\end{bem}

\begin{proof} Die Aussagen \ref{1.1.5a} und \ref{1.1.5b} folgen sofort aus der Definition.
\begin{enumerate}
\item[\ref{1.1.5c}] Sei zuerst $x\in V$. Dann gilt für alle $f\in \I(V)$: $f(x)=0$, also gilt $x\in \V(\I(V))$.

Sei nun $V$ eine affine Varietät. Dann gilt $V=\V(\F)$ für eine geeignete Menge $\F$. Es gilt $\F\subseteq \I(V)$, also ist
\[V=\V(\F)\supseteq \V(\I(V)).\]

Der Rest der Behauptung folgt aus der Definition des Verschwindungsideals.
\item[\ref{1.1.5d}] Die eine Richtung ist klar, bzw. folgt aus \ref{1.1.5a}. Für die andere Richtung überlegt man sich, dass nach \ref{1.1.5c} $V_{1}=\V(\I(V_{1}))=\V(\I(V_{2}))=V_{2}$ gilt.  Die Aussage für die Inklusionen folgt analog.
\end{enumerate}
\end{proof}

\begin{q}
Wir haben gesehen, dass die Zuordnung
\[V\mapsto \I(V)\]
injektiv ist. Ist sie auch surjektiv?
\end{q}

\sect{Zariski-Topologie}

\begin{db}\label{1.2.1}Sei $n\in\N$. Die affinen Varietäten im $k^{n}$ bilden die abgeschlossenen Mengen einer Topologie auf dem $k^{n}$. Diese heißt \emph{Zariski-Topologie\index{Zariski-Topologie}}.\end{db}
\begin{proof}\begin{prooflist}
\item $\leer$ und $k^{n}$ sind affine Varietäten nach \cref{bsp1.2}.
\item Seien $V_{1},V_{2}$ affine Varietäten, d.h. $V_{1}=\V(I_{1})$ und $V_{2}=\V(I_{2})$, wobei $I_{1},I_{2}$ Ideale in $k\polyx$ sind. Dann gilt:
\[V_{1}\cap V_{2}=\V(I_{1}\cup I_{2}) = \V(I_{1}+I_{2}).\]
Das gleiche Argument funktioniert für beliebige Familien $V_{\lambda}$ mit $\lambda\in\Lambda$ und $\Lambda$ Indexmenge, d.h. $\displaystyle\bigcap_{\lambda\in\Lambda}V_{\lambda}$ ist wieder eine affine Varietät.
% 20.10.2010
\item Seien $V_{1},V_{2}$ affine Varietäten mit $I_{1}=\I(V_{1})$ und $I_{2}=\I(V_{2})$.
\begin{zeige} $\V(I_{1}\cdot I_{2})\subseteq V_{1}\cup V_{2}\subseteq \V(I_{1}\cap I_{2})\subseteq \V(I_{1}\cdot I_{2})$.
\begin{prooflist}
\item $V_{1}\cup V_{2}\subseteq \V(I_{1}\cap I_{2})\subseteq \V(I_{1}\cdot I_{2})$ ist nach Definition klar.
\item Sei $x\in \V(I_{1}\cdot I_{2})$. Angenommen $x\notin V_{1}$. Dann existiert $g\in I_{1}$ mit $g(x)\neq 0$. Sei $f\in I_{2}$. Dann ist $f\cdot g\in I_{1}\cdot I_{2}$, folglich ist 
\[\bigl(f\cdot g\bigr)(x) = f(x)\cdot g(x) = 0\]
 und da $g(x)\neq 0$ ist $f(x)=0$, also $x\in \V(I_{2})=V_{2}$.
\end{prooflist}\end{zeige}
\end{prooflist}\end{proof}

\begin{dfn}\label{1.2.2}
Wir schreiben $\A^{n}(k)$ für $k^{n}$ mit der Zariski-Topologie.
\end{dfn}

\begin{bem}\label{1.2.3}
Sei $M\subseteq\A^{n}(k)$. Der Abschluss $\Bar{M}$ von $M$ bezüglich der Zariski-Topologie ist $\Bar{M}=\V(\I(M))$.
\end{bem}
\begin{proof}
$\Bar{M}\subseteq \V(\I(M))$ folgt sofort aus der Definition.

Für die andere Inklusion überlegt man sich: Sei $\V(J)\supseteq M$ eine abgeschlossene Obermenge von $M$. Dann ist $\I(M)\supseteq\I(\V(J))$ und damit
\[\V(\I(M))\subseteq\V(\I(\V(J)))=\V(J).\]
\end{proof}

\begin{bsp}\label{1.2.4}
Sei $n=1$. Dann gilt: $X\subseteq\A^{1}(k)$ ist genau dann abgeschlossen, wenn $X$ eine endliche Teilmenge oder ganz $\A^{1}(k)$ ist.
\end{bsp}

\begin{w}\enquote{Kleine Umgebungen} sind riesig groß!\end{w}

\begin{bem}\label{1.2.5}
\begin{enumerate}
\item\Label{1.2.5a} Wenn $k$ endlich ist, entspricht die Zariski-Topologie auf $\A^{1}(k)$ der diskreten Topologie.
\item\Label{1.2.5b} Wenn $k$ unendlich ist, ist die Zariski-Topologie nicht hausdorffsch.
\end{enumerate}\end{bem}
\begin{proof}\begin{enumerate}
\item[\ref{1.2.5a}] Punkte sind abgeschlossen, denn sei $x=(x_{1},\dotsc,x_{k})\in k^{n}$, dann ist
\[\{x\}=\V(X_{1}-x_{1},\dotsc,X_{n}-x_{n}).\]
Damit sind auch endliche Vereinigungen von Punkten abgeschlossen und somit schon \textit{alle} Teilmengen von $\A^{1}(k)$.
\item[\ref{1.2.5b}] {\scshape Erinnerung:} Ein topologischer Raum $X$ heißt hausdorffsch, wenn für alle Punkte $x,y\in X$ offene Umgebungen $U_{x}\ni x$ und $U_{y}\ni y$ mit $U_{x}\cap U_{y}=\leer$ existieren.

Für $n=1$ folgt die Behauptung also aus \cref{1.2.4}.

Den Fall $n\geq 2$ führen wir zurück auf den Fall $n=1$:

Seien $x,y\in\A^{n}(k)$, $U_{x},U_{y}$ offene Umgebungen von $x$ bzw. $y$. Wir setzen
\[V_{1}:=\A^{n}(k)\setminus U_{x} =: \V(I_{1})\text{ und }V_{2}:=\A^{n}(k)\setminus U_{y}=:\V(I_{2})\]
für entsprechende Ideale $I_{1}$ und $I_{2}$. Ohne  Einschränkung wählen wir $x$ und $y$ in $W:=\V(X_{2},\dotsc,X_{n})$, also auf der \enquote{$X_{1}$-Achse}. Dann gilt für alle Polynome $f\in I_{1}$ und $g\in I_{2}$, dass sie auf $W$ nicht verschwinden, da $x$ bzw. $y$ in den Komplementen von $V_{1}$ bzw. $V_{2}$ liegen.

Dann besteht $V_{1}\cap W$ aus nur endlich vielen Punkten, da diese Nullstellen von $f(X_{1},0,\dotsc,0)\neq 0$ sein müssen. Gleiches gilt für $V_{2}\cap W$. Also schneiden sich ihre Komplemente $U_{x}$ und $U_{y}$ sogar schon in $W$, da $\card{W}=\card{k}=\infty$.
\end{enumerate}\end{proof}

\begin{de}\label{1.2.6} Seien $X,X_{1},X_{2}$ topologische Räume.
\begin{enumerate}
\item Sei $Y\subseteq X$. Definiere auf $Y$ die \emph{Spurtopologie\index{Spurtopologie}} durch
\[U\subseteq Y\text{ offen}\: :\Longleftrightarrow\:\exists\;V\subseteq X\text{ offen mit }U=V\cap Y.\]
\item Sei $X\times Y$ das kartesische Produkt (als Mengen) und seien
\begin{align*}p_{1}\colon X_{1}\times X_{2}\ra X_{1},\quad(x_{1},x_{2})\mapsto x_{1},\\p_{2}\colon X_{1}\times X_{2}\ra X_{2},\quad(x_{1},x_{2})\mapsto x_{2},\end{align*}
die zugehörigen Projektionen. Die \emph{Produkttopologie\index{Produkttopologie}} auf $X_{1}\times X_{2}$ ist die gröbste Topologie (d.h. möglichst wenig offene Mengen), so dass $p_{1}$ und $p_{2}$ stetig sind.

Daher ist $U\subseteq X_{1}\times X_{2}$ genau dann offen, wenn $U$ beliebige Vereinigung endlicher Schnitte von Urbildern offener Mengen in $X_{1}$ bzw. $X_{2}$ unter $p_{1}$ bzw. $p_{2}$ ist.

\item $X$ heißt \emph{reduzibel\index{reduzibel}}, wenn es abgeschlossene echte Teilmengen \[A,B\subsetneq X\text{ mit }X=A\cup B\text{ gibt.}\]

\item $X$ heißt \emph{irreduzibel\index{irreduzibel}}, wenn $X$ nicht reduzibel ist.

\item Eine maximale irreduzible Teilmenge von $X$ heißt \emph{irreduzible Komponente\index{irreduzible Komponente}}.
\end{enumerate}\end{de}

\begin{bsp}\label{1.2.7}
Sei $X$ hausdorffsch und $M\subseteq X$. Dann gilt:
\[M\text{ ist irreduzibel (bzgl. Spurtopologie)}\iff \card{M}\leq 1,\]
$M$ ist also einelementig oder leer.

\textit{Denn}: Liegen $x\neq y$ in $M$, so finden wir (offene) Umgebungen $U_{x}$ und $U_{y}$ mit leerem Schnitt, können also
\[M=(M\setminus U_{x})\cup(M\setminus U_{y})\]
schreiben und sehen so, dass $M$ reduzibel ist. $\leer$ ist nie irreduzibel.
\end{bsp}

\begin{bsp}\label{1.2.8} Sei $k$ ein Körper mit unendlich vielen Elementen.
\begin{enumerate}
\item\Label{1.2.8a} $\A^{1}(k)$ ist irreduzibel, da echte abgeschlossene Teilmengen endlich sind.
\item\label{1.2.8b} $\V(X\cdot Y) = \V(X)\cup\V(Y)$ ist reduzibel mit irreduziblen Komponenten $\V(X)$ und $\V(Y)$.
\end{enumerate}\end{bsp}

\begin{bem}\label{1.2.9}
Sei $V\subseteq\A^{n}(k)$ eine affine Varietät. Dann gilt:
\[V\text{ ist irreduzibel}\iff\I(V)\text{ ist ein Primideal}.\]
\end{bem}
\begin{proof}
Sei zuerst $V$ irreduzibel. Seien $f,g\in k\polyx$ mit $f\cdot g\in\I(V)$. 

Angenommen $f\notin\I(V)$, dann gilt auch $\V(f)\not\supseteq\V(\I(V))=V$.

Andererseits gilt nach \cref{1.2.1}: $\V(f)\cup\V(g)=\V(f\cdot g)\supseteq V$, also
\[V=(V\cap \V(f))\cup(V\cap\V(g)),\]
wobei $V\cap\V(f)$ und $V\cap\V(g)$ in $V$ abgeschlossen sind. Außerdem ist $V\neq V\cap\V(f)$, also gilt, da $V$ irreduzibel ist, $V=V\cap\V(g)$. Daraus folgt $V\subseteq\V(g)$ und damit ist $g\in\I(V)$ und $\I(V)$ ist somit ein Primideal.

Sei nun $\I(V)$ ein Primideal. Seien $V_{1},V_{2}$ Varietäten mit $V=V_{1}\cup V_{2}$ und $I_{1}:=\I(V_{1})$, $I_{2}:=\I(V_{2})$.

Angenommen $V\neq V_{1}$, d.h. $V\supsetneq V_{1}$, dann ist auch $\I(V)\subsetneq\I(V_{1})=I_{1}$.

Andererseits ist $V=V_{1}\cup V_{2}=\V(I_{1}\cdot I_{2})$, also ist $I_{1}\cdot I_{2}\subseteq\I(V)$. Das impliziert aber $I_{2}\subseteq\I(V)$, da $\I(V)$ ein Primideal ist und $I_{1}\not\subseteq\I(V)$. Daher gilt 
\[V_{2}=\V(I_{2})\supseteq\V(\I(V))=V,\] 
also ist schon $V=V_{2}$ und damit ist $V$ irreduzibel.
\end{proof}

\begin{prop}\label{1.2.10} Sei $X$ ein topologischer Raum. Dann ist jede irreduzible Teilmenge in einer irreduziblen Komponente enthalten.\end{prop}
% 25.10.2010
\begin{proof} Verwende das Lemma von Zorn:

{\scshape Erinnerung:} Hat in einer halbgeordneten Menge $\M$ jede Kette (d.h. totalgeordnete Teilmenge) eine obere Schranke, dann hat $\M$ mindestens ein maximales Element.

Seien also $X'\subseteq X$ eine irreduzible Teilmenge, 
\[\M:=\{Y\subseteq X\mid Y\text{ irreduzibel, }Y\supseteq X'\}\]
und $\{Y_{\lambda}\}_{\lambda\in\Lambda}$ eine Familie aus $\M$, die totalgeordnet ist. Sei $\displaystyle Y:=\bigcup_{\lambda\in\Lambda}Y_{\lambda}$.

{\scshape Zeige:} $Y\in\M$, d.h. $Y$ ist irreduzibel.

Wir nehmen an, es gäbe abgeschlossene Mengen $A,B\subsetneq X$, so dass $Y\cap A$ und $Y\cap B$ echte Teilmengen von $Y$ sind, für die $Y=(Y\cap A)\cup(Y\cap B)$ gilt. Insbesondere gilt dann:
\[(X\setminus A)\cap Y\neq\leer\neq (X\setminus B)\cap Y.\]
Folglich existieren ein $\lambda_{1}$ mit $(X\setminus A)\cap Y_{\lambda_{1}}\neq\leer$ und ein $\lambda_{2}$ mit $(X\setminus B)\cap Y_{\lambda_{2}}\neq\leer$. Da die $Y_{\lambda_{i}}$ Teil einer Kette sind, können wir ohne Einschränkung $Y_{\lambda_{1}}\subseteq Y_{\lambda_{2}}$ annehmen. Damit ist aber auch $(X\setminus A)\cap Y_{\lambda_{2}}\neq\leer$ und wir finden eine echte Zerlegung
\[Y_{\lambda_{2}}=(Y_{\lambda_{2}}\cap A)\cup(Y_{\lambda_{2}}\cap B),\]
was im Widerspruch zur Irreduzibilität von $Y_{\lambda_{2}}$ steht. Folglich hat jede Kette eine obere Schranke und nach dem Lemma von Zorn hat $\M$ somit ein maximales Element.
\end{proof}

\pagebreak[3]
\begin{satz}\label{satz1} Sei $V$ eine affine Varietät. Dann gilt:
\begin{enumerate}
\item\Label{s1a} $V$ ist eine endliche Vereinigung von irreduziblen affinen Varietäten.
\item\Label{s1b} $V$ hat nur endlich viele irreduzible Komponenten $V_{1},\dotsc,V_{r}$. Insbesondere ist die Zerlegung
\[V=V_{1}\cup\dotsm\cup V_{r}\]
eindeutig.\end{enumerate}\end{satz}
\begin{proof}\begin{enumerate}
\item[\ref{s1a}] Seien 
\begin{align*}&\B:=\{V\subseteq k^{n}\mid V\text{ ist affine Varietät und erfüllt \textit{nicht} \ref{s1a}}\},\\
&\J:=\{\I(V)\subseteq k\polyx\mid V\in\B\}.\end{align*}
Wir nehmen an, $\B$ wäre nicht leer. Dann ist auch $\J$ nicht leer. Da $k\polyx$ noethersch ist, finden wir in $\J$ ein maximales Element $I_{0}=\I(V_{0})$. Damit ist $V_{0}$ ein minimales Element in $\B$. Dann ist $V_{0}$ aber \textit{nicht} irreduzibel, also existieren affine Varietäten $V_{1},V_{2}\subsetneq V$ mit $V=V_{1}\cup V_{2}$. Insbesondere sind diese aber nicht in $\B$, da $V_{0}$ minimal gewählt war, lassen sich also als Vereinigung endlich vieler irreduzibler Varietäten schreiben. Dann geht das aber auch für $V_{0}$ und das ist ein Widerspruch, da $V_{0}\in\B$.
\item[\ref{s1b}] Mit Hilfe von \cref{1.2.10} und dem \ref{s1a}-Teil sehen wir, dass wir \[V=V_{1}\cup\dotsm\cup V_{r}\] schreiben können, wobei die $V_{i}$ irreduzible Komponenten sind. Wir zeigen noch die Eindeutigkeit dieser Zerlegung: Sei $W$ eine irreduzible Komponente von $V$. Wir schreiben
\[W=(W\cap V_{1})\cup\dotsm\cup(W\cap V_{r})\]
und sehen, da $W$ irreduzibel ist, dass es ein $i$ mit $W=W\cap V_{i}$ gibt. Also gilt $W\subseteq V_{i}$ und damit schon $W=V_{i}$, da $W$ als irreduzible \textit{Komponente} eine \textit{maximale} irreduzible Teilmenge von $V$ ist.
\end{enumerate}\end{proof}

\sect{Der Hilbertsche Nullstellensatz}

\begin{motivation}
  Bisher haben wir die Mengen
  \[ \VV_n = \{ V\subseteq k^n \mid V \text{ affine Varietät }\}\text{ und }
     \J_n = \{ I\subseteq k\polyx \mid I \text{ Radikalideal }\} \]
  und zwischen ihnen die Abbildungen
  \begin{align*}
    \V\colon& \J_n \ra \VV_n, \quad I\mapsto\V(I)\\
    \I\colon& \VV_n \ra \J_m, \quad V\mapsto\I(V)
  \end{align*}
  betrachtet. Wir haben gesehen, dass $\V\circ\I=\id$ gilt. Gilt auch $\I\circ\V=\id$?
\end{motivation}

\begin{nbsp}
  Wenn $k$ nicht algebraisch abgeschlossen ist, muss das nicht gelten: sei $k=\R$ und $I=(X^2+1)$. Dann ist $\V(I)=\leer$, aber
  $\I(\V(I)) = k\polyx$.
\end{nbsp}

Wir werden sehen, dass $\I\circ\V=\id$ gilt, falls $k$ algebraisch abgeschlossen ist. Die einzige \enquote{Obstruktion} ist,
dass $\V(I)=\leer$ gilt, obwohl $I\neq k\polyx$ ist.

\begin{satz}[Hilbertscher Nullstellensatz]\label{satz2}\label{HNS}
  \begin{enumerate}
  \item\Label{s2a} {\bf Algebraische Form:}\Label{satz2a} Sei $\m$ ein maximales Ideal in \penalty-1000\relax $k\polyx$. Dann ist
    $\Quotient{k\polyx}{\m}$ eine endliche algebraische Körpererweiterung von $k$.
  \item {\bf Schwacher Hilbertscher Nullstellensatz:}\Label{satz2b} Ist $k$ ein algebraisch abgeschlossener Körper und
    $I\subsetneq k\polyx$ ein echtes Ideal, dann ist $\V(I)\neq\leer$.
  \item {\bf Starker Hilbertscher Nullstellensatz:}\Label{satz2c} Ist $k$ ein algebraisch abgeschlossener Körper und $I\subseteq
    k\polyx$ ein Ideal, dann ist $\I(\V(I))=\sqrt{I}$.
  \end{enumerate}
\end{satz}

Die Aussage wird in mehreren Schritten im Rest dieses Abschnitts bewiesen.

Wir nennen $x_1:=\Bar{X}_1,\dotsc,x_n:=\Bar{X}_n$. Ohne Einschränkung können wir nach einer eventuellen Umsortierung
der Variablen annehmen, dass $x_1,\dotsc,x_l$ algebraisch unabhängig über $k$ sind und $x_{l+1},\dotsc,x_n$ algebraisch über
$k(x_1,\dotsc,x_l)$. Also:
\[ k \subseteq S = k(x_1,\dotsc,x_l) \subseteq L=\Quotient{k\polyx}{\m}, \]
dabei ist $k(x_1,\dotsc,x_l)\cong\Quot(k\polyx[l])$ und $L$ ist endlich erzeugt als $S$-Modul.

\begin{lem}\label{1.3.1}
  Seien $R$, $S$, $T$ Ringe mit $R\subseteq S\subseteq T$, sodass gilt:
  \begin{itemize}
  \item $R$ ist noethersch
  \item $T$ ist endlich erzeugt als $R$-Algebra; seien $x_1,\dotsc,x_n$ solche Erzeuger
  \item $T$ ist endlich erzeugt als $S$-Modul; seien $w_1,\dotsc,w_m$ solche Erzeuger
  \end{itemize}
  Dann ist $S$ als $R$-Algebra endlich erzeugt.
\end{lem}
\begin{proof}
  Wir schreiben 
  \[ x_i=\displaystyle\sum_{j=1}^m a_{ij}w_j \text{ mit } a_{ij}\in S, \qquad
     w_iw_j=\displaystyle\sum_{l=1}^m b_{ijl}w_l \text { mit } b_{ijl}\in S. \]
  Sei $S_0$ die $R$-Unteralgebra von $S$, die von allen $a_{ij}$ und $b_{ijl}$ erzeugt wird. Nach dem Hilbertschen Basissatz ist
  $S_0$ ein noetherscher Ring. $T$ wird von den $w_i$ auch als $S_0$-Modul erzeugt und ist noethersch als $S_0$-Modul, da er ein
  endlich erzeugter Modul über einem noetherschen Ring ist. $S$ ist ein $S_0$-Untermodul von $T$. Damit ist $S$ endlich erzeugt
  als $S_0$-Modul, also auch als $R$-Algebra.
\end{proof}

Insbesondere ist in der Situation im Beweis des \hyperref[HNS]{Hilbertschen Nullstellensatzes} $k(x_1,\dotsc,x_l)$ als
$k$-Algebra endlich erzeugt.

\begin{lem}\label{1.3.2}
  Es sei $k$ ein Körper und $r\ge1$. Dann ist $k(X_1,\dotsc,X_r)$ nicht endlich erzeugt als $k$-Algebra.
\end{lem}
\begin{proof}
  Wir nehmen an, wir hätten endlich viele Erzeuger $h_1,\dotsc,h_l$ und schreiben $h_i=\frac{f_i}{g_i}$ mit $f_i,g_i\in
  k\polyx[r]$. Dann wählen wir ein Primpolynom $p$, das keines der $g_i$ teilt und schreiben
  \[ \frac1p = \sum_{i=1}^N a_i h_{i_1}\dotsm h_{i_{m_i}} \quad (a_i\in k,\ m_{i}\in\{1,\dotsc,l\}). \]
  Sei $H=g_1\dotsm g_l$. Multipliziert man obige Gleichung mit $H$ durch, bekommt man \[\frac{H}{p}\in k\polyx[r],\] also
  ist $p$ ein Teiler von $H$. Aber $p$ war teilerfremd zu allen $g_i$ gewählt. Das ist ein Widerspruch.
\end{proof}

\begin{prop}\label{1.3.3}
  Die \hyperref[s2a]{algebraische Version des Hilbertschen Nullstellensatzes} stimmt.
\end{prop}
\begin{proof}
  Das liegt daran, dass $\Quotient{k\polyx}{\m}$ nach \cref{1.3.1} als $k$-Algebra endlich erzeugt ist und nach \cref{1.3.2} folglich keine transzendenten Elemente enthält, also eine algebraische Körpererweiterung ist.
\end{proof}

%27.10.10

Um die Implikation $I\subsetneq k\polyx \implies \V(I)\neq\leer$ zu zeigen, betrachten wir für einen Punkt $p=(x_1,\dotsc,x_n)\in
k^n$ den Einsetzungshomomorphismus \[ \phi_p\colon k\polyx \ra k, \quad f\mapsto f(p) \] (das ist ein
$k$-Algebrenhomomorphismus). Dieser steigt genau dann auf $A=\Quotient{k\polyx}{I}$ ab, wenn $p\in\V(I)$ gilt. Außerdem ist jeder
$k$-Algebrenhomomorphismus $\phi\colon A\ra k$ von dieser Art: wähle \[p=(\phi(\Bar{X_1}),\dotsc,\phi(\Bar{X_n})).\] Ist $I=\m$ ein
maximales Ideal, dann ist $A=\Quotient{k\polyx}{\m}$ eine endliche algebraische Körpererweiterung von $k$. In diesem Fall
existiert genau dann ein $k$-Algebrenhomomorphismus von $A$ nach $k$, wenn $A=k$ gilt.

\begin{lem}\label{1.3.4}
  Aus der \hyperref[satz2a]{algebraischen Version} folgt der \hyperref[satz2b]{schwache Hilbertsche Nullstellensatz}.
\end{lem}
\begin{proof}
  Sei $\m$ ein maximales Ideal mit $I\subseteq\m$. Da $k$ algebraisch abgeschlossen ist, ist also, nach \cref{satz2a},
  $L=\Quotient{k\polyx}{\m}\cong k$. Sei $\phi\colon L\ra K$ ein Isomorphismus und \[p=(\phi(\Bar{X_1}),\dotsc,\phi(\Bar{X_n}))\in
  k^n.\] Für $f\in\m$ gilt dann $f(p)=\phi(f(\Bar{X}_1,\dotsc,\Bar{X}_n))=\phi(\Bar{f(X_1,\dotsc,X_n)}) = \phi(0)=0$. 
  
Also ist $p\in\V(I)$ und damit $\V(I)\neq\leer$.
\end{proof}

\begin{lem}[Schluss von Rabinowitsch]
  Aus dem \hyperref[satz2b]{schwachen Hilbertsche Nullstellensatz} folgt der \hyperref[satz2c]{starke Hilbertsche Nullstellensatz}.
\end{lem}
\begin{proof}
Zu zeigen ist $\I(\V(I))=\sqrt{I}$. Die Inklusion \enquote{$\supseteq$} ist klar. Für die andere Inklusion nehmen wir uns ein
$g\in\I(\V(I))$ und zeigen: es gibt ein $d\ge1$ mit $g^d\in I$.

Wir betrachten $k^{n+1}$ und definieren \[J=(I,gX_{n+1}-1).\] Dann ist $\V(J)=\leer$, denn wäre
$p=(x_1,\dotsc,x_{n+1})\in\V(J)$, dann wäre $(x_1,\dotsc,x_n)\in\V(I)$, also $g(x_1,\dotsc,x_n)=0$, aber damit wäre
\[\bigl(gX_{n+1}-1\bigr)(p)=-1\neq0,\] was ein Widerspruch ist.

Nach \cref{satz2b} muss dann also $J=k\polyx[n+1]$ gelten. Also gibt es \[b_1,\dotsc,b_{n+1}\in k\polyx[n+1],\] sodass
\[ 1=b_1f_1 + \dotsm + b_nf_n + b_{n+1}(gX_{n+1}-1) \] gilt. Wir verwenden den $k$-Algebrenhomomorphismus
\[ \phi\colon k\polyx[n+1] \ra k(X_1,\dotsc,X_n),\quad X_i\mapsto 
%X_i \text{ für } i\in\{1,\dotsc,n\},\ X_{n+1}\mapsto\frac1g \]
\begin{cases}X_{i},&\text{ für }i\in\{1,\dotsc,n\},\\
\frac{1}{g},&\text{ für }i=n+1.\end{cases}\]
und erhalten so
\[ 1 = \phi(1) = \phi(b_1)f_1 + \dotsm + \phi(b_n)f_n + 0. \]
Nun schreiben wir $\phi(b_i)=\frac{\schlange{b_i}}{g}a_i$ für $\schlange{b_i}\in k\polyx$. Durchmultiplizieren mit $g^d$ für genügend
großes $d$ ergibt dann $g^d\in I$.
\end{proof}

\begin{kor}\label{1.3.6}
  Ist $k$ algebraisch abgeschlossen, dann entsprechen die affinen Varietäten in $\A^n(k)$ bijektiv den Radikalidealen in
  $k\polyx$ via $V\mapsto\I(V)$.
\end{kor}

\sect{Morphismen zwischen affinen Varietäten}

\begin{ziel}In diesem Abschnitt sollen Morphismen definiert werden. Die Idee dabei ist, Abbildungen zu betrachten, die von Polynomen herkommen.
\end{ziel}
\begin{dfn}\label{1.4.1}
\begin{enumerate}
\item\Label{1.4.1a} Seien $V\subseteq k^{n}, W\subseteq k^{m}$ affine Varietäten. Wir nennen eine Abbildung $f\colon V\ra W$ einen \emph{Morphismus\index{Morphismus}}, wenn es Polynome $f_1,\dotsc,f_m \in k\polyx$ mit 
\[f(p)=(f_1(p),\dotsc,f_m(p)) \in k^{m}\text{ gibt.}\]
\item\Label{1.4.1b} Ein Morphismus $f\colon V\ra W$ heißt \emph{Isomorphismus\index{Isomorphismus}}, wenn es einen Morphismus $g\colon W\ra V$ mit $g\circ f=\id_V$ und $f\circ g=\id_W$ gibt.
\item\Label{1.4.1c} Gibt es einen Isomorphismus zwischen $V$ und $W$, so heißen $V$ und $W$ \emph{isomorph\index{isomorph}}.
\end{enumerate}
\end{dfn}
\begin{bem}\label{1.4.2} Die affinen Varietäten über $k$ bilden zusammen mit den Morphismen eine Kategorie: $\AffVar_k$.
Dabei sind die Objekte gerade die affinen Varietäten und die Morphismen zwischen zwei affinen Varietäten sind wie in \cref{1.4.1} gegeben. Man beachte, dass die Verkettung von Polynomen wieder ein Polynom ist, d.h. die Verkettung zweier Morphismen ist auch wieder ein Morphismus.
\end{bem}

\begin{bsp}\label{1.4.3}
\begin{enumerate}
\item\Label{1.4.3a}  Projektionen bzw. Einbettungen $\A^{n}(k)\ra \A^{m}(k)$ 
\[\begin{cases} (x_1,\dotsc,x_n) \mapsto (x_1,\dotsc,x_m),&\text{ falls }n\ge m, \\ (x_1,\dotsc,x_n)\mapsto (x_1,\dotsc,x_n,0,\dotsc,0),&\text{ falls }n<m, \end{cases}\] sind Morphismen.
\item\Label{1.4.3b} Jedes $f \in k\polyx$ definiert ein Morphismus $\A^{n}(k)\ra \A^{n}(k)$.
\item\Label{1.4.3c} Seien $V=\A^1(k)$, $W=\V(Y^2-X^3) \subseteq \A^2(k)$. Die Abbildung 
\[f\colon V\ra W, \qquad x\mapsto (x^2,x^3)\]
definiert einen Morphismus.
%Bild : Neilsche Parabel
Hierbei ist f bijektiv mit Umkehrabbildung 
\[g(x,y)=\begin{cases} y/x,&\text{ falls }x\neq 0, \\ 0,&\text{ falls }x=0. \end{cases}\]
Ist $k$ unendlich, so ist $g$ kein Morphismus. Also sind bijektive Morphismen im Allgemeinen keine Isomorphismen.
\item\Label{1.4.3d} Sei char$(k)=p>0$. Dann heißt $f\colon \A^n(k)\ra \A^n(k)$ gegeben durch 
\[f(x_1,\dotsc,x_n)=(x_1^p,\dotsc,x_n^p)\] 
\emph{Frobeniusmorphismus\index{Frobeniusmorphismus}}. Bekanntermaßen sind die Nullstellen des Polynoms $X^{p}-X$ genau die Elemente aus $\FF_p$. Damit sind die Fixpunkte des Frobeniusmorphismus gerade die Punkte mit Koordinaten in $\FF_p$.
\end{enumerate}
\end{bsp}

\begin{bem}\label{1.4.4}
\begin{enumerate}
\item\Label{1.4.4a} Morphismen sind stetig bezüglich der Zariski-Topologie.
\item\Label{1.4.4b} Sind $V\subseteq \A^n(k)$, $W \subseteq \A^m(k)$ affine Varietäten, dann lässt sich jeder Morphismus $f\colon V\ra W$ zu einem Morphismus $f^n\colon \A^n(k)\ra A^m(k)$ fortsetzen.
\end{enumerate}
\end{bem}

\begin{proof}
\begin{enumerate}
\item[\ref{1.4.4a}] Sei $f\colon V\ra W$ ein Morphismus und $Z=\V(J)\subseteq W$ abgeschlossen. Setze $I=\{g\circ f \mid g\in J\}$. Dann ist $f^{-1}(Z)=\V(I)$, denn: 
\[x\in f^{-1}(Z)\iff f(x)\in Z \iff g\bigl(f(x)\bigr)=0\text{ für alle }g\in J.\]
Also ist $f^{-1}(Z)$ abgeschlossen und $f$ damit stetig.
\item[\ref{1.4.4b}] Das folgt direkt aus \cref{1.4.1}, da Polynome global definierbar sind.
\end{enumerate}
\end{proof}

\begin{w}
Im Allgemeinen gibt es viel mehr stetige Abbildungen als Morphismen. 

Beispielsweise ist jede bijektive Abbildung von $k$ nach $k$   stetig, wie wir in \cref{1.2.4} gesehen haben.

\end{w}\par

\begin{dfn}\label{1.4.5}
Sei $V\subseteq \A^n(k)$ eine affine Varietät. Dann heißt 
\[k[V]=\{f\colon V\ra k \mid f\text{ ist Morphismus\}}\]
 der \emph{affine Koordinatenring\index{affiner Koordinatenring}}.
\end{dfn}

\begin{bem}
\begin{enumerate}
\item\Label{1.4.6a} Der Koordinatenring $k[V]$ ist eine reduzierte $k$-Algebra, das heißt aus $f^n=0$ folgt $f=0$.
\item\Label{1.4.6b} Es gilt $k[V] \cong \Quotient{k\polyx}{\I(V)}$.
\end{enumerate}
\end{bem}

\begin{proof}
\begin{enumerate}
\item[\ref{1.4.6a}] folgt aus \ref{1.4.6b}, wobei Addition, Multiplikation und Skalarmultiplikation komponentenweise definiert sind. Die Reduziertsheitsaussage ist klar, da $\I(V)$ ein Radikalideal ist.
\item[\ref{1.4.6b}] Definiere
\[\phi\colon  k\polyx\ra k[V],  \qquad p\mapsto p\restrict{V}.\]
Dann ist $\phi$ nach Definiton der Morphismen surjektiv, vgl. \cref{1.4.1}. Weiter gilt:
\[p \in \Kern(\phi) \iff p\restrict{V} \equiv 0 \iff \forall x\in V\colon p(x)=0 \iff p\in \I(V).\] 
Mit dem Homomorphiesatz folgt die Behauptung. 
\end{enumerate}
\end{proof}

\begin{bem}\label{1.4.7} Seien $V\subseteq \A^n(k)$, $W\subseteq \A^m(k)$ affine Varietäten. Dann gilt:
\begin{enumerate}
\item\Label{1.4.7a} Jeder Morphismus $f\colon V\ra W$ induziert einen $k$-Algebrenhomomorphismus
\[f^\sharp\colon  k[W]\ra k[V], \qquad g\mapsto g\circ f.\]
\item\Label{1.4.7b} Jeder $k$-Algebrenhomomorphismus zwischen $k[W]$ und $k[V]$ wird von einem solchen $f$ induziert.
\item\Label{1.4.7c} Die Abbildung 
\[\sharp\colon \Mor(V,W)\ra \Hom_k(k[W],k[V]), \qquad f\mapsto f^{\sharp}\]
ist sogar bijektiv.
\end{enumerate}
\end{bem}

\begin{proof}
\begin{enumerate}
\item[\ref{1.4.7a}] Die Algebrenhomomorphismuseigenschaften zu überprüfen ist elementar.
\item[\ref{1.4.7b}] Sei $\phi\colon  k[W]\ra k[V]$ ein $k$-Algebrenhomomorphismus und \[k[W]\ni p_i\colon  W\ra k\] sei die Projektion auf die $i$-te Koordinate.
Definiere nun $f\colon V\ra W$ durch \[x=(x_1,\dotsc,x_n)\mapsto \bigl(\phi(p_1)(x),\dotsc,\phi(p_m)(x)\bigr).\]
Wir zeigen, dass $f$ wohldefiniert ist und dass $f^{\sharp}=\phi$ ist.

Zuerst sehen wir, dass die $p_i$'s $k[W]$ als $k$-Algebra erzeugen  und nach der Definition von $f$ ist 
\[f^{\sharp}(p_i)=p_i\circ f=\phi(p_i).\]

Es bleibt also nur noch die Wohldefiniertheit zu zeigen, d.h. $f$ bildet wirklich nach $W=\V(\I(W))$ ab.

Wir zeigen also: Für alle $g \in \I(W)$ und $x\in V$ gilt $g\bigl(f(x)\bigr)=0$. 

Sei also $g\in\I(W)$. Dann ist $g$ als Element in $k[W]$ schon $0$ und damit ist auch $\phi(g)=0$ in $k[V]$. 

Nun definieren wir uns $\dach{\phi}$ indem wir $X_{i}\in k\polyx[m]$ auf ein Urbild von $\phi(p_{i})$ in $k\polyx$ schicken. Das dürfen wir, da der Polynomring frei ist und daher kommutiert das folgende Diagramm:
\[\begin{tikzpicture}
\matrix (m) [matrix of math nodes, row sep=3em, column sep=5em, text height=1.5ex, text depth=0.25ex]
{k\polyx[m]&k\polyx\\
\displaystyle k[W]=\Quotient{k[X_1,\dotsc,X_m]}{\I(W)}&\displaystyle k[V]=\Quotient{k\polyx}{\I(V)}\\};
\path[->,font=\scriptsize]
(m-1-1) edge node [auto] {$\dach\phi$} (m-1-2) 
(m-1-2) edge node [auto] {$\pi_{V}$} (m-2-2)
(m-1-1) edge node [auto] {$\pi_{W}$} (m-2-1) 
(m-2-1) edge node [auto] {$\phi$} (m-2-2);
\end{tikzpicture}\] 
%\[\begin{tikzpicture}
%\matrix (m) [matrix of math nodes, row sep={4em,between origins}, column sep={5em,between origins}, text height=1.5ex, text depth=0.25ex]
% {%
%  X_{i}     &[-2em]       &[+1.5em]       &[-2em]\dach{\phi}(p_{i})\in\pi_{V}^{-1}\bigl(\phi(p_{i})\bigr) \\[-2em]
%        &[-2em] k\polyx[m]     &[+1.5em] k\polyx&[-2em]                                             \\[+1.5em]
%        &[-2em] \displaystyle k[W]=\Quotient{k[X_1,\dotsc,X_m]}{\I(W)}     &[+1.5em] \displaystyle k[V]=\Quotient{k\polyx}{\I(V)} &[-2em]                                             \\[-2em]
% p_{i}   &[-2em]       &[+1.5em]       &[-2em] \phi(p_{i})           \\
% };
%  \path[->,font=\scriptsize]
% (m-2-2) edge node [auto] {$\eta_{V}$} (m-2-3) 
% (m-2-3) edge node [auto] {$f^{**}$} (m-3-3)
% (m-2-2) edge node [auto] {$f$} (m-3-2) 
% (m-3-2) edge node [auto] {$\eta_{W}$} (m-3-3);
%  \path[|->,font=\scriptsize]
% ($ (m-1-1)! 2/5 !(m-4-1) $) edge ($ (m-1-1)! 3/5 !(m-4-1) $)
% ($ (m-1-1)! 2/5 !(m-1-4) $) edge ($ (m-1-1)! 3/5 !(m-1-4) $)
% ($ (m-4-1)! 2/5 !(m-4-4) $) edge ($ (m-4-1)! 3/5 !(m-4-4) $)
% ($ (m-1-4)! 2/5 !(m-4-4) $) edge ($ (m-1-4)! 3/5 !(m-4-4) $);
%\end{tikzpicture}\]
Also folgt die Behauptung, da 
\[g\bigl(\phi(p_1)(x),\dotsc,\phi(p_m)(x)\bigr)=\dach{\phi}(g)(x)=0\] 
für $g \in \I(W)$, da $\dach{\phi}(g)\in\I(V)$ liegt, da das Diagramm kommutiert.
\item[\ref{1.4.7c}] Die Surjektivität der Abbildung wurde gerade gezeigt, also bleibt nur die Injektivität zu zeigen.
Sei also $f_1^{\sharp}=f_2^{\sharp}$, d.h. $g\circ f_1=g\circ f_2$ für alle $g\in k[W]$.
Insbesondere gilt das für die Projektionen auf die einzelnen Koordinaten. Damit sind $f_{1}$ und $f_{2}$ in allen Komponenten gleich und es folgt $f_1(x)=f_2(x)$ für alle $x\in V$. 
\end{enumerate}
\end{proof}

\begin{satz}\label{satz3} Wir bezeichnen die Kategorie der endlich erzeugten, reduzierten $k$-Algebren mit $\RedAlg_k$.
\begin{enumerate}
\item\Label{satz3a} Die Zuordnung $V\mapsto k[V]$ induziert einen kontravarianten Funktor $\Phi$.
\item\Label{satz3b} Ist $K=k$ algebraisch abgeschlossen, so induziert $\Phi$ eine Äquivalenz der Kategorien $\AffVar_k$ und $\RedAlg_k$.
\end{enumerate}
\end{satz}

\begin{proof}
\begin{enumerate}
\item[\ref{satz3a}] Wir definieren $\Phi$ auf den Morphismen durch 
\[\Mor(V,W)\ra \Hom_k(k[W],k[V]), \qquad g\mapsto g^{\sharp}.\]
Dann gilt $\Phi(g_1\circ g_2)=(g_1\circ g_2)^{\sharp}=g_2^\sharp\circ g_1^\sharp$ und $\Phi(\id)=\id$.
\item[\ref{satz3b}] Nun zeigen wir, dass $\Phi$ eine natürliche Äquivalenz definiert. Genauer: 
\begin{itemize}
\item $\Phi$ ist ein volltreuer Funktor (d.h. induziert Bijektion auf den Morphismen)
\item Für jede endlich erzeugte, reduzierte $K$-Algebra $A$ gibt es eine affine Varietät $V$ mit $\Phi(V)\cong A$. 
\end{itemize}
Alternativ könnte man auch zeigen: Es existiert ein Funktor 
\[\Psi\colon  \RedAlg_k\ra \AffVar_k\] mit $\Psi\circ \Phi\cong \id_{\AffVar_k}$ und $\Phi\circ \Psi\cong \id_{\RedAlg_k}$.

Zuerst stellen wir fest: Nach \cref{1.4.7} ist $\Phi$ volltreu. 

Sei nun $A$ eine endlich erzeuge $K$-Algebra mit Erzeugern $a_1,\dotsc,a_n$.
Definiere den Homomorphismus 
\[\phi\colon  K\polyx\ra A\text{ durch }X_i\mapsto a_i.\]
Aus der Reduziertheit von $A$ folgt, dass
$I=\Kern(\phi)$ ein Radikalideal ist. Setze nun $V=\V(I)$. Da $K$ algebraisch abgeschlossen ist, gilt $\I(V)=\sqrt{(I)}=I$. Insgesamt gilt mit dem Homomorphiesatz:
\[K[V]\cong\Quotient{K\polyx}{\I(V)}=\Quotient{K\polyx}{I}\cong A.\] 
\end{enumerate}
\end{proof}

\begin{bsp}\label{1.4.8} In diesem Beispiel sei wieder $K$ algebraisch abgeschlossen.
\begin{enumerate}
\item\Label{1.4.8a} Ist $V=\A^n(K)$, so ist $\I(V)=(0)$ und damit $K[V]\cong K\polyx$.
\item\Label{1.4.8b} Im anderen Extremfall $V=\leer$ gilt $\I(V)=K\polyx$, d.h. $K[V]\cong \{0\}$.
\item\Label{1.4.8c} Ist $V$ ein Punkt, d.h. $V=\{(x_1,\dotsc,x_n)\}$, dann ist $\I(V)=(X_1-x_1,\dotsc,X_n-x_n)$. Also gilt dann $K[V]\cong K$. Man weiß ja auch, dass die maximalen Ideale in einem algebraisch abgeschlossenen Körper gerade den Punkten entsprechen.
\item\Label{1.4.8d} Sei $V$ eine Hyperebene, d.h. $\I(V)=(a_1X_1+\dotsc+a_nX_n+c)$ mit mindestens einem $a_i\neq 0$.
Dann ist $K[V]\cong K\polyx[n-1]$ und $V\cong \A^{n-1}$.
\item\Label{1.4.8e} Wir betrachten die Neilsche Parabel aus \cref{1.4.3}. Wir haben also 
\[V=\V(Y^2-X^3)\text{ sowie }I=(Y^2-X^3),\text{ also }\V(I)=\{(t^2,t^3) \mid t\in K\}.\] 
Folglich gilt
\[K[V]\cong \Quotient{K[X,Y]}{(Y^2-X^3)}.\] 
Man beachte: $\A^1(k)$ und $V$ sind birational, also gilt $K(V)\cong K(\A^1(K))$. 

Konkreter:
Die Äquivalenzklasse von $X^3$ ist ein Quadrat in $K(V)$, also ist auch die Klasse von $X$ ein Quadrat.
Wie sieht eine Wurzel von der Klasse von $X$ aus? 

Betrachte  dazu
\[r_1\colon  \begin{cases} \A^1(K)\ra&\!\!\! V \\ \hfill t\mapsto &\!\!\!(t^2,t^3) \end{cases}\quad \text{ und } \quad r_2\colon  \begin{cases} \hfill V\ra&\!\!\! \A^1(K) \\ (x,y) \mapsto &\!\!\! y/x\end{cases}\]
Dann sind $r_1$ und $r_2$ \hyperref[1.6.5]{birationale Abbildungen} und induzieren 
\begin{align*}\alpha_1\colon  K(V)&\ra K(\A^1(K))=K(t),\\
\alpha_2\colon  K(\A^1(K))&\ra K(V),\end{align*}
wobei $\alpha_1(X)=t^2$ und $\alpha_2(t)=Y/X$. Also gilt in $K(V)$: $Y^2/X^2=X$.

Im Koordinatenring $K[V]$ gibt es hingegen keine Wurzel aus der Klasse von $X$. Denn ansonsten gäbe es $P \in K[X,Y]$ mit $P^2-X \in (Y^2-X^3)$. Also gäbe es ein 
\[f \in K[X,Y]\text{ mit }P^2(X,Y)=X+f(X,Y)(Y^2-X^3).\]
Insbesondere wäre dann
 \[P^2(X,0)=X-f(X,0)\cdot X^3=X(1-f(X,0)X^2)\text{ in } K[X],\]
 Also wäre $P^2(X,0)$ durch $X$ aber nicht durch $X^2$ teilbar, was unmöglich ist.
 
Daher ist $K[V]$ hier nicht isomorph zu $K[X]$.
\end{enumerate}
\end{bsp}

\sect{Die Garbe der regulären Funktionen}

In diesem Abschnitt wollen wir Morphismen \enquote{lokal}, d.h. auf offenen Mengen definieren.

In diesem Abschnitt sei stets $K$ ein algebraisch abgeschlossener Körper.

\begin{db}\label{1.5.1}
  Sei $f\in K\polyx$ und \[ \D(f) := \{ x\in\A^n(K) \mid f(x)\neq0 \} = \A^n(K)\setminus\V(f). \] Das ist eine offene Teilmenge des
  $\A^n(K)$. Die Menge $\{\D(f) \mid f\in K\polyx\}$ bildet eine Basis der Zariski-Topologie.
\end{db}
\begin{proof}
  Sei $U\subseteq\A^n(K)$ eine offene Menge, $V=\A^n(K)\setminus U$ und $I=\I(V)$. Dann gilt für alle $f\in I$ und $p\in \D(f)$,
  dass $f(p)\neq0$, also $p\not\in V=\V(I)$ und damit $p\in U$. Also ist $U$ die Vereinigung aller $\D(f)$ für $f\in I$.
\end{proof}

\begin{bem}\label{1.5.2}
  Sei $V$ eine affine Varietät und $h\in K\polyx$. Dann gilt
  \begin{align*}
    \V(h)\cap V=\leer &\iff \I(V)+(h)=K\polyx \\ &\iff 1=gh+f\text{ für passende }g\in K\polyx,\ f\in\I(V)
    \\ &\iff\Bar{1}=\Bar{g}\Bar{h}\text{ in } K[V] \\ &\iff \Bar{h}\text{ ist in }K[V]\text{ invertierbar}.
  \end{align*}
\end{bem}

\begin{dfn}\label{1.5.3}
  Sei $V\subseteq\A^n(K)$ eine affine Varietät und $U\subseteq V$ offen (bzgl. der Spurtopologie).
  \begin{enumerate}
  \item\Label{1.5.3a} Sei $p\in U$. Eine Funktion $r\colon U\ra\A^1(K)$ heißt \emph{regulär in $p$\index{regulär}}, wenn es eine Umgebung $U_p$ von $p$ gibt
    und $f_p,g_p\in K[V]$ mit $g_p(x)\neq0$ für alle $x\in U_p$, sodass\[r(x)=\frac{f_p(x)}{g_p(x)}\text{ für alle }x\in U_p\text{ gilt.}\] Die Funktion $r$ heißt \emph{regulär\index{regulär}}, wenn sie für alle $p\in U$ regulär ist.
  \item\Label{1.5.3b} Die Menge \[\O_V(U) = \{ r\colon U\ra \A^1(K) \mid r \text{ regulär}\}\] heißt \emph{$K$-Algebra der regulären
      Funktionen} oder \emph{regulärer Ring\index{regulärer Ring}}.
  \end{enumerate}
\end{dfn}

\begin{nbsp}
  Für $V=\A^1(K)$ und $U=V\setminus\{0\}$ ist z.B. $\frac1x\in\O_V(U)$.
\end{nbsp}

\begin{bem}\label{1.5.4}
  Sei $V\subseteq\A^n(K)$ eine affine Varietät und $U\subseteq V$ offen. Dann ist $\O_V(U)$ eine $K$-Algebra, die $K[V]$
  umfasst.
\end{bem}

\begin{bem}[Restriktion der regulären Funktionen]\label{1.5.5}
  Für offene Mengen $U,U'\subseteq V$ mit $U'\subseteq U$ ist die Restriktionsabbildung
  \[ \rho_{UU'}\colon \O_V(U)\ra\O_V(U'),\quad f\mapsto f\restrict{U'} \]
  ein $K$-Algebrenhomomorphismus mit folgenden Eigenschaften:
  \begin{enumerate}
  \item\Label{1.5.5a} Für in $V$ offene Mengen $U''\subseteq U'\subseteq U$ gilt $\rho_{UU''}=\rho_{U'U''}\circ\rho_{UU'}$.
  \item\Label{1.5.5b} Ist $U$ offen in $V$ und $(U_i)_{i\in I}$ eine offene Überdeckung von $U$, so gilt
    \begin{enumerate}
    \item\Label{1.5.5bi} Für alle $f\in\O_V(U)$ gilt $f=0$ genau dann wenn $\rho_{UU_i}(f)=0$ für alle $i\in I$.
    \item\Label{1.5.5bii} Ist eine Familie $(f_i)_{i\in I}$ mit $f_i\in\O_V(U_i)$ gegeben, die 
    \[\rho_{U_iU_i\cap U_j}(f_i)=\rho_{U_jU_i\cap U_j}(f_j)\]
      für alle $i,j\in I$ erfüllt, dann gibt es ein $f\in\O_V(U)$ mit
      $\rho_{UU_i}(f)=f_i$ für alle $i\in I$
      (\enquote{Verklebeeigenschaft}).
    \end{enumerate}
  \end{enumerate}
\end{bem}
\begin{proof}
  \begin{enumerate}
  \item[\ref{1.5.5a}] Folgt aus den Eigenschaften von Einschränkungen von Funktionen.
  \item[\ref{1.5.5b}]
    \begin{enumerate}
    \item[\ref{1.5.5bi}] Klar.
    \item[\ref{1.5.5bii}] Definiere $f(x)=f_i(x)$, falls $x\in U_i$. Dann ist $f$ wohldefiniert und eine reguläre Funktion, denn
      auf jedem $U_i$  stimmt $f$ mit $f_i$ überein und $f_i$ ist auf $U_i$ regulär.
    \end{enumerate}
  \end{enumerate}
\end{proof}

\begin{dfn}\label{1.5.6}
  \begin{enumerate}
  \item Sei $X$ ein topologischer Raum. Für jede offene Menge $U\subseteq X$ sei eine $K$-Algebra $\O(U)$ gegeben. Weiter sei
    für jede Inklusion von offenen Mengen $U'\subseteq U$ ein $K$-Algebren-Homomorphismus $\rho_{UU'}\colon\O(U)\ra\O(U')$
    gegeben, sodass die Eigenschaften \ref{1.5.5a} und \ref{1.5.5b} aus \cref{1.5.5} gelten. Dann heißt
    $\{\O(U),\rho_{UU'}\}_{U'\subseteq U\text{ offen}}$ eine \emph{Garbe von $K$-Algebren\index{Garbe}} und $\rho_{UU'}$ heißt
    \emph{Restriktionshomomorphismus\index{Restriktionshomomorphismus}}.
  \item Garben von Vektorräumen, Ringen, Mengen etc. sind analog definiert.
  \item Aus der Bedingung \ref{1.5.5bi} in \cref{1.5.5} folgt die Eindeutigkeit von $f$ in \ref{1.5.5bii}.
  \item Sei $X$ ein topologischer Raum und $\O_X$ eine Garbe von Ringen auf $X$. Dann heißt $(X,\O_X)$ ein \emph{geringter Raum\index{geringter Raum}}
    und $\O_X$ heißt \emph{Strukturgarbe\index{Strukturgarbe}} auf $X$.
  \end{enumerate}
\end{dfn}

\begin{bem}\label{1.5.7}
  Sei $V$ eine affine Varietät über $K$. Die regulären Ringe $\O_V(U)$ aus \cref{1.5.3} bilden eine Garbe von $K$-Algebren.
\end{bem}

\begin{bsp}\label{1.5.8}
  Weitere Beispiele für Garben:
  \begin{itemize}
  \item topologischer Raum mit Garbe der stetigen Funktionen,
  \item differenzierbare Mannigfaltigkeiten mit $C^k$-Funktionen,
  \item Riemannsche Flächen mit Garbe der holomorphen Funktionen.
  \end{itemize}
\end{bsp}

\begin{ziel} Wir wollen reguläre Funktionen möglichst global definieren.
Dazu zeigen wir zunächst, dass endlich viele der $U_p$ reichen
und dass $\O_V(V)=K[V]$.
\end{ziel}

\begin{bem}\label{1.5.9}
  Sei $V$ eine affine Varietät.
  \begin{enumerate}
  \item\Label{1.5.9i} $V$ ist ein \emph{noetherscher topologischer Raum\index{noetherscher topologischer Raum}}, d.h. jede absteigende Kette abgeschlossener Mengen
    $V_1\supseteq V_2\supseteq V_3\supseteq\dotso$ wird stationär.
  \item\Label{1.5.9ii} Offene Mengen in $U$ sind \emph{quasikompakt\index{quasikompakt}}, d.h. jede offene Überdeckung besitzt eine endliche
    Teilüberdeckung.
  \item\Label{1.5.9iii} Für $g\in K[V]$ sei $\D(g)=\{x\in V\mid g(x)\neq0\}$. Die Mengen $\D(g)$ bilden eine Basis der Topologie
    von $V$. Es gilt sogar, dass jede offene Menge in $V$ Vereinigung von endlich vielen der $\D(g)$ ist.
  \end{enumerate}
\end{bem}
\begin{proof}
  \begin{enumerate}
  \item[\ref{1.5.9i}] Das gilt, weil $K[V]$ noethersch als Ring ist.
  \item[\ref{1.5.9ii}] Wenn das nicht so wäre, dann gäbe es in einer offenen Überdeckung \[U\subseteq\bigcup_{i\in I}U_i\] eine
    Folge $i_1,i_2,i_3,\dotsc$, so dass man kein $U_{i_j}$ weglassen kann. Damit wäre \[U_{i_1}\subseteq U_{i_1}\cup U_{i_2}
    \subseteq U_{i_1}\cup U_{i_2}\cup U_{i_3}\subseteq \dotsm\] eine aufsteigende Kette offener Mengen, die nie stationär
    wird. Dies ist ein Widerspruch zu \ref{1.5.9i}.
  \item[\ref{1.5.9iii}] Dass die $\D(g)$ eine Basis bilden, sieht man wie in \cref{1.5.1}. Es reichen endlich viele, weil
    jedes Ideal in $K[V]$ endlich erzeugt ist.
  \end{enumerate}
\end{proof}

%8.11.10

\begin{prop}\label{1.5.10}
  Sei $V$ eine affine Varietät. Dann gibt es für $g\in K[V]$ und eine reguläre Funktion $r\in\O_V(\D(g))$ ein $f\in K[V]$ und ein
  $d\in\N_0$, sodass $r=\frac{f}{g^d}$.
\end{prop}
\begin{proof}
  \begin{prooflist}
  \item\Label{1.5.10i} Ohne Einschränkung kann in \cref{1.5.3} $U_p=\D(g_p)$ gewählt werden, denn: Wir finden zunächst $f_{p},g_{p}\in K[V]$, so dass
    $r(x)=\frac{f_p(x)}{g_p(x)}$ für alle $x\in U_p$. Wir wählen ein $\schlange{g_p}\in K[V]$ mit $p\in \D(\schlange{g_p})\subseteq U_{p}\subseteq \D(g_{p})$. Damit ist $\schlange{g_p}\in\sqrt{(g_p)}$, es gibt also ein $h\in K[V]$ und ein $d\in\N$
    mit $\schlange{g_p}^{d}=h\cdot g_p$. Wir wählen $\dach{f_{p}}=f_p\cdot h$ und $\dach{g_p}=\schlange{g_p}^d$ und haben
    dann \[r(x)=\frac{f_p(x)}{g_p(x)} = \frac{f_p(x)h(x)}{g_p(x)h(x)} = \frac{\dach{f_p}(x)}{\dach{g_p}(x)}\]
    auf $\D(\dach{g_p})=\D(\schlange{g_p})\subseteq \D(g_p)$.
  \item Nach \cref{1.5.9ii} genügen endlich viele der $\D(g_p)$, also können wir \[\D(g)=\D(g_1)\cup\dotsm\cup
    \D(g_n)\] schreiben und haben nach \ref{1.5.10i} $r=\frac{f_i}{g_i}$ auf $\D(g_i)$ mit $f_i,g_i\in K[V]$.

     Wir basteln nun $\schlange{f_i}$, $\schlange{g_i}\in K[V]$, so dass $r=\frac{\schlange{f}_i}{\schlange{g}_i}$ auf $\D(\schlange{g}_i)=\D(g_i)$ und
      $\schlange{f}_i\schlange{g}_j=\schlange{f}_j\schlange{g}_i$ in $K[V]$ gilt.

    Es gilt $f_i(x)g_j(x)-f_j(x)g_i(x)=0$ für alle $x\in \D(g_i)\cap \D(g_j)=\D(g_ig_j)$. Also gilt
    \[g_i(x)g_j(x)\cdot\bigl(f_i(x)g_j(x)-f_j(x)g_i(x)\bigr)=0\] für alle $x\in V$, da $V=\D(g_ig_j)\cup\V(g_ig_j)$. Wir wählen
    $\schlange{f_i}=f_ig_i$ und $\schlange{g_i}=g_i^2$. Dann ist \[r(x)=\frac{f_i(x)g_i(x)}{g_i^2(x)} =
    \frac{\schlange{f_i}(x)}{\schlange{g_i}(x)}\] auf $\D(g_i)=\D(\schlange{g_i})$ und $\schlange{f_i}\schlange{g_j}-\schlange{f_j}\schlange{g_i}=0$ auf
    ganz $V$.

    %10.11.10
    Es gilt \[\D(g)=\bigcup_{i=1}^n \D(g_i),\text{ also }\V(g)=\bigcap_{i=1}^n\V(g_i).\] Nach dem \hyperref[HNS]{Hilbertschen Nullstellensatz}
    gilt also $g\in\sqrt{(g_1,\dotsc,g_n)}$, d.h. es gibt ein $d\in\N$ mit $g^d=h_1g_1+\dotso+h_ng_n$ mit
    $h_i\in K[V]$. Damit ist für alle $j$:
    \[ f_jg^d = \sum_{i=1}^nh_ig_if_j = \sum_{i=1}^n h_ig_jf_i = g_j\sum_{i=1}^nh_if_i. \]
    Wir setzen $f=\displaystyle\sum_{i=1}^nh_if_i$ und haben dann für alle $j$ \[r(x)=\frac{f_j(x)}{g_j(x)}=\frac{f(x)}{g^d(x)}\]
    auf $\D(g_j)$.
  \end{prooflist}
\end{proof}

\begin{kor}\label{1.5.11}
  \begin{enumerate}
  \item\Label{1.5.11i} $\O_V(\D(g))\cong K[V][\frac1g] = K[V]_{\{g^d\mid d\in\N_0\}}$
  \item\Label{1.5.11ii} Ist $V$ irreduzibel, so gilt $\O_V(\D(g))=\{\frac{f}{g^d}\in\Quot(K[V])\}$.
  \item\Label{1.5.11iii} $\O_V(V)=K[V]$
  \end{enumerate}
\end{kor}
\begin{proof}
  \begin{enumerate}
  \item[\ref{1.5.11i}] Die Abbildung
    \[K[V][\textstyle\frac1g]\ra\O_V(\D(g)),\quad \displaystyle\frac{f}{g^d}\mapsto\left(x\mapsto\frac{f(x)}{g^d(x)}\right) \]
    ist ein wohldefinierter $K$-Algebren-Homomorphismus, denn $\frac{f_1}{g^{d_1}}=\frac{f_2}{g^{d_2}}$ gilt genau dann, wenn es
    ein $\schlange{d}\in\N_0$ gibt, sodass \[g^{\schlange{d}}(f_1g^{d_2}-f_2g^{d_1})=0\] ist. Da $g(x)\neq0$ auf $\D(g)$ gilt, ist das
    äquivalent zu \[\frac{f_1(x)}{g^{d_1}(x)}=\frac{f_2(x)}{g^{d_2}(x)}\text{ für alle }x\in \D(g).\] Außerdem ist die Abbildung
    offenbar injektiv und wegen \cref{1.5.10} auch surjektiv.
  \item[\ref{1.5.11ii}] Das folgt aus \ref{1.5.11i}, denn hier ist $K[V]$ nullteilerfrei.
  \item[\ref{1.5.11iii}] Hier wählen wir $g=1$, sodass $\D(g)=V$ und nach \ref{1.5.11i} $\O_V\cong K[V]$ gilt.
  \end{enumerate}
\end{proof}

\begin{prop}\label{1.5.12}
  Seien $V\subseteq\A^n(K)$, $W\subseteq\A^m(K)$ affine Varietäten. Genau dann ist eine Abbildung $f\colon V\ra W$ ein
  Morphismus, wenn $f$ stetig ist und reguläre Funktionen erhält, d.h. für alle offenen $U\subseteq W$ und $r\in\O_W(U)$ gilt
  $r\circ f\in\O_V(f^{-1}(U))$.
\end{prop}
\begin{proof}
  Sei zuerst $f$ ein Morphismus. Dann ist $f$ stetig nach \cref{1.4.4}. Sei $r\in\O_W(U)$. Dann lässt sich $r$ lokal
  schreiben als \[r(x)=\frac{g_p(x)}{h_p(x)}\text{ mit }g_p,h_p\in K[W].\] Weil $g_p\circ f=f^\sharp(g_p)$ und $h_p\circ
  f=f^\sharp(h_p)$ in $K[V]$ liegen, ist $r\circ f(x)=\frac{g_p\circ f(x)}{h_p\circ f(x)}$ eine lokale Darstellung von
  $r\circ f$. Also ist $r\circ f$ eine reguläre Funktion.

  Sei nun umgekehrt $f$ eine stetige Abbildung, die reguläre Funktionen erhält. Für die Koordinatenfunktionen \[p_i\colon
  \A^m(K)\ra \A^1(K),\ (x_1,\dotsc,x_m)\mapsto x_i,\] wollen wir $p_i\circ f\in K[V]$ zeigen. Dies gilt, denn $p_i\in K[W] =
  \O_W(W)$, also nach Voraussetzung $p_i\circ f\in\O_V(V)=K[V]$ mit \cref{1.5.11iii}.
\end{proof}

\begin{bem}\label{1.5.13}
  Sei $V$ eine affine Varietät.
  \begin{enumerate}
  \item\Label{1.5.13a} Sei $U\subseteq V$ offen, $r\in\O_V(U)$ und $A=\{p\in U\mid r(p)=0\}$. Dann ist $A$ abgeschlossen in $U$.
  \item\Label{1.5.13b} (\enquote{Identitätssatz}) Sind $r_1\in\O_V(U_1)$, $r_2\in\O_V(U_2)$ und gibt es ein $\schlange{U}\subseteq
    U_1\cap U_2$, das in $U_1\cap U_2$ dicht liegt und so, dass $r_1\restrict{\schlange{U}}=r_2\restrict{\schlange{U}}$, dann gilt
    $r_1\restrict{U_1\cap U_2}=r_2\restrict{U_1\cap U_2}$.
  \end{enumerate}
\end{bem}
\begin{proof}
  \begin{enumerate}
  \item[\ref{1.5.13a}] Sei $q\in U\setminus A$. Wir schreiben $r=\frac{f}{g}$ auf einem $\D(g)\subseteq U$. Dann ist $r(q)\neq0$,
    also $f(q)\neq0$. Insbesondere ist $r(x)\neq0$ für alle $x\in \D(f)\cap \D(g)$. Also ist $\D(f)\cap \D(g)$ eine Umgebung von $q$
    in $U\setminus A$.
  \item[\ref{1.5.13b}] Setze $r=r_1\restrict{U_1\cap U_2}=r_2\restrict{U_1\cap U_2}\in \O_V(U_1\cap U_2)$. Es ist $r=0$ auf
    $\schlange{U}$. Da die Nullstellenmenge nach \ref{1.5.13a} abgeschlossen ist und dicht liegt, folgt $r=0$ auf $U_1\cap U_2$.
  \end{enumerate}
\end{proof}

\begin{db}\label{1.5.14}
  \begin{enumerate}
  \item\Label{1.5.14i} Eine \emph{quasi-affine Varietät\index{quasi-affine Varietät}} $W$ ist eine Zariski-offene Teilmenge einer affinen Varietät.
  \item\Label{1.5.14ii} Eine Abbildung $f\colon W_1\ra W_2$ zwischen quasi-affinen Varietäten heißt \emph{Morphismus\index{Morphismus}} oder \emph{reguläre Abbildung\index{reguläre Abbildung}}, wenn $f$
    stetig ist und reguläre Funktionen erhält. Die quasi-affinen Varietäten bilden zusammen mit den Morphismen eine Kategorie.
  \item\Label{1.5.14iii} Eine quasi-affine Varietät $W$ heißt \emph{affin (als abstrakte Varietät)\index{affin (als abstrakte Varietät)}}, wenn es einen Isomorphismus
    $f\colon W\ra V$ gibt, sodass $V=\V(I)$ eine affine Varietät in einem $\A^n(K)$ ist.
    \item\Label{1.5.14iv} Seien $W_{1}\subseteq\A^{n}(K)$, $W_{2}\subseteq\A^{m}(K)$ quasi-affine Varietäten und $f\colon W_{1}\ra W_{2}$. Dann ist $f$ genau dann eine reguläre Abbildung, wenn es reguläre Funktionen 
    \[f_{1},\dotsc,f_{m}\in\O_{\Bar{W_{1}}}\text{ mit }f=(f_{1},\dotsc,f_{m})\]
    gibt, denn $f$ ist genau dann ein Morphismus, wenn $f$ stetig ist und reguläre Funktionen erhält. Analog zu \cref{1.5.12} lässt sich $f$ also lokal als Quotient von regulären Funktionen schreiben. 
    \item Sei $V$ eine affine Varietät und $U\subseteq V$ offen. Dann ist
    \[\O_{V}(U)=\O_{\Bar{U}}(U)=:\O(U).\]
    Denn: Nach Definition ist jedes Element aus $\O_{V}(U)$ lokal auf einer offenen Teilmenge $\dach{U}$ von $V$ von der Form $\frac{g}{h}$ mit $g,h\in K[V]$. Insbesondere stimmt das dann auch auf $\dach{U}\cap\Bar{U}$. Umgekehrt können wir, wenn $r=\frac{g}{h}$ lokal auf $\schlange{U}\subseteq\Bar{U}$ für $g,h\in K[V]$ gilt, ein offenes $\dach{U}\subseteq V$ mit $\dach{U}\cap\Bar{U}=\schlange{U}$ wählen.
  \end{enumerate}
\end{db}

\begin{bsp}\label{1.5.15}
  \begin{enumerate}
  \item\Label{1.5.15i} Sei $f\in K\polyx$. Dann ist $\D(f)$ affin als abstrakte Varietät, denn: Definiere $\schlange{f}\in
    K\polyx[n+1]$ durch $\schlange{f}=f\cdot X_{n+1}-1$ und $I=(\schlange{f})$. Dann ist $\V(I)=\{(x_1,\dotsc,x_{n+1})\in\A^{n+1}(K)
    \mid f(x_1,\dotsc,x_n)\cdot x_{n+1}=1\}$. Definiere weiter
    \begin{align*}
      r_1&\colon \D(f)\ra\V(I),\quad (x_1,\dotsc,x_n)\mapsto\left(x_1,\dotsc,x_n,\frac{1}{f(x_1,\dotsc,x_n)}\right),\\
      r_2&\colon \V(I)\ra \D(f),\quad (x_1,\dotsc,x_{n+1})\mapsto(x_1,\dotsc,x_n).
    \end{align*}
    Dann sind $r_1$ und $r_2$ wohldefinierte reguläre Abbildungen und zueinander invers.
  \item\Label{1.5.15ii} $\A^1(K)\setminus\{0\}=\D(X)$ ist affin als abstrakte Varietät.
  \item\Label{1.5.15iii} $\A^2(K)\setminus\{(0,0)\}$ ist nicht affin als abstrakte Varietät; siehe Übungsblatt.
  \item\Label{1.5.15iv} Für $\GL_n(K) = \{A\in K^{n\times n}\mid \det{A}\neq0\} \subseteq K^{n^2}$ gilt
    $\GL_n(K)=\D(\det(\cdot))$. Mit \ref{1.5.15i} kann $\GL_n(K)$ als affine Varietät in $\A^{n^2+1}(K)$ aufgefasst werden,
    nämlich als $\{(A,y)\in K^{n^2+1} \mid y\cdot\det{A}-1=0\}$.
  \end{enumerate}
\end{bsp}

\sect{Rationale Abbildungen}
In diesem Abschnitt wollen wir alle regulären Funktionen, die auf einem in $V$ dichten $U$ definiert sind, gleichzeitig betrachten.

In diesem Abschnitt sei stets $K$ algebraisch abgeschlossen und $V$ eine affine Varietät.

\begin{db}\label{1.6.1}
\begin{enumerate}
\item\Label{1.6.1a} Eine \emph{rationale Funktion\index{rationale Funktion}} auf $V$ ist eine Äquivalenzklasse von Paaren $(U,f)$, wobei $U$ eine dichte, offene Teilmenge von $V$ mit $f\in \O_V(U)$ ist. Hierbei sei 
\[(U,f)\sim(U',f') \iff f\restrict{U\cap U'}=f'\restrict{U\cap U'}.\]
\item\Label{1.6.1b} In jeder Äquivalenzklasse gibt es ein maximales Element (bzgl. Inklusion auf $U$) $(U,f)$. Dieses $U$ heißt \emph{Definitionsbereich\index{Definitionsbereich}} und $V\setminus U$ heißt \emph{Polstellenmenge\index{Polstellenmenge}}.
\item\Label{1.6.1c} Die rationalen Funktionen auf $V$ bilden eine $K$-Algebra. Schreibweise: $\Rat(V)$.
\item\Label{1.6.1d} Ist $V$ irreduzibel, so ist $\Rat(V)$ isomorph zu $\Quot(K[V])=K(V)$.
\end{enumerate}
\end{db}

\begin{proof}
\begin{enumerate}
\item[\ref{1.6.1a}] Zu zeigen: $\sim$ definiert eine Äquivalenzrelation. Es bleibt nur die Transitivität zu zeigen.
Seien $(U_1,f_1)\sim (U_2,f_2)$ und $(U_2,f_2)\sim (U_3,f_3)$. Dann stimmen $f_1$ und $f_3$ auf $U_1\cap U_2\cap U_3$ überein. Da $U_1\cap U_2\cap U_3$ dicht in $V$ ist, folgt mit dem Identitätssatz \cref{1.5.13}:
\[f_1\restrict{U_1\cap U_3}=f_3\restrict{U_1\cap U_3}\]
\item[\ref{1.6.1b}] Ist $(U,f)\sim(U',f')$, so kann auf $U\cup U'$ die reguläre Funktion $\dach{f}$ definiert werden, indem man 
\[\dach{f}(x)=\begin{cases} f(x),&\text{falls } x\in U, \\ f'(x),&\text{falls } x\in U',\end{cases}\] setzt.

Setze also $U_{max}=U\cup U'$, wobei $U'$ alle offenen Mengen durchläuft für die es ein $f'\in \O(U')$ gibt mit $(U',f')\sim (U,f)$.
\item[\ref{1.6.1c}] Summen und Produkte von rationalen Funktionen sind wieder rational und das ist mit der Äquivalenzrelation verträglich. Dabei ist zu beachten, dass man die Definitionsbereiche schneiden muss.
\item[\ref{1.6.1d}] Definiere den $K$-Algebrenhomomorphismus $\alpha$ durch
\[\alpha\colon K(V)\ra \Rat(V), \quad \frac{g}{h}\mapsto\relax \left[\left(\D(h),\frac{g}{h}\right)\right].\] 
Sei $(U,r)$ mit $r\in \O(U)$ eine rationale Funktion. Da $r$ regulär ist, gibt es $U'\subseteq U$ mit $r\restrict{U'}=\frac{g}{h}$, wobei
$g,h \in K[V]$ und $h\in \O(U')$. Da $V$ irreduzibel ist, ist $U'$ bereits dicht. Also gilt $\alpha(\frac{g}{h})=[(U,r)]$ und $\alpha$ ist surjektiv.

Die Injektivität ist klar, da $K(V)$ ein Körper ist.
\end{enumerate}
\end{proof}

\begin{dfn}\label{1.6.2}
Sei $V$ irreduzibel. Der Körper $K(V)\cong \Rat(V)$ heißt \emph{Funktionenkörper\index{Funktionenkörper}}.
\end{dfn}

\begin{w}
Ist $V$ irreduzibel, so repräsentiert der gekürzte Bruch\par $\frac{g}{h} \in K(V)$ die rationale Funktion $r:=[(\D(h),\frac{g}{h})]$.

Aber $\D(h)$ kann eine \textit{echte} Teilmenge des Definitionsbereichs $\Def(r)$ sein. 

Ist jedoch $K[V]$ faktoriell, so gilt $\D(h)=\Def(r)$ (vgl. dazu auch Übungsblätter 4 und 5).
\end{w}

\begin{prop}\label{1.6.3}
Seien $V,W$ irreduzible, affine Varietäten, $f \colon V\ra W$ ein Morphismus und $f^{\sharp} \colon K[W]\ra K[V]$ der induzierte $K$-Algebrenhomomorphismus. Dann gilt:
  \begin{enumerate}
  \item\Label{1.6.3a} $f^{\sharp}$ kann genau dann zu einem Körperhomomorphismus von $K(W)\ra K(V)$ fortgesetzt werden, wenn $f^{\sharp}$ injektiv ist.
  \item\Label{1.6.3b} $f^{\sharp}$ ist genau dann injektiv, wenn $f(V)$ dicht in $W$ ist.
  \end{enumerate}
\end{prop}

\begin{proof}
  \begin{enumerate}
  \item[\ref{1.6.3a}] Die Notwendigkeit der Injektivität von $f^{\sharp}$ ist klar. Ist umgekehrt $f^{\sharp}$ injektiv, so definiert man $f^{\sharp}(\frac{a}{b})=\frac{f^{\sharp}(a)}{f^{\sharp}(b)}$. Die Injektivität impliziert die Wohldefiniertheit und man sieht, dass man auf diese Art einen     Körperhomomorphismus erhält.
  \item[\ref{1.6.3b}]  Für $f^{\sharp}$ gilt: Für jede Untervarietät $Z \subseteq V$ ist ${f^{\sharp}}^{-1}(\I(Z))=\I(f(Z))$. Denn:
\begin{align*}g \in {f^{\sharp}}^{-1}(\I(Z))& \iff g\circ f \in \I(Z) \iff g \circ f(x)=0 \;\forall x \in Z\\
& \iff g(y)=0 \;\forall y \in f(Z).\end{align*}
Nun gilt: $f^{\sharp}$ ist injektiv $\iff 0={f^{\sharp}}^{-1}(0)={f^{\sharp}}^{-1}(\I(V))=\I(f(V))=\I(\Bar{f(V)})$
\[ \iff \Bar{f(V)}=W.\]  
  \end{enumerate}
\end{proof}

\begin{dfn}\label{1.6.4}
Ein Morphismus $f \colon V\ra W$ mit $\Bar{f(V)}=W$ heißt \emph{dominant\index{dominant}}.
\end{dfn}

\begin{db}\label{1.6.5} Seien $V\subseteq \A^n(K)$, $W\subseteq \A^m(K)$ affine Varietäten.
  \begin{enumerate}
  \item\label{1.6.5a} Eine \emph{rationale Abbildung\index{rationale Abbildung}} $f \colon V\ppf W$ ist eine Äquivalenzklasse $(U,f_U)$, wobei $U$ offen und dicht in $V$ und $f_U \colon U\ra W$ reguläre Abbildungen sind. Dabei ist
  \[(U,f_U)\sim (U',f'_U) \iff f_U\restrict{U\cap U'}=f'_{U'}\restrict{U\cap U'}.\]
  \item\label{1.6.5b} Rationale Abbildungen nach $\A^1(K)$ sind rationale Funktionen.
  \item\label{1.6.5c} Jede rationale Abbildung $r \colon V\ppf W$ hat einen maximalen Definitionsbereich $\Def(r)$, der offen ist.
  \item\label{1.6.5d} Sind $V$ und $W$ irreduzibel, so ist die Komposition von dominanten rationalen Abbildungen wieder eine dominante rationale Abbildung, d.h. für $U$ dicht in $V$ ist \[\Bar{r(U)}=W.\]
  \item\label{1.6.5e} Sind $V$ und $W$ irreduzibel, so induziert jede dominante rationale Abbildung \[f \colon V\ppf W\] einen Körperhomomorphismus $f^{\sharp} \colon K(W)\ra K(V)$ mit \[f^{\sharp}(g)=g\circ f.\]
  \item\label{1.6.5f} Eine rationale Abbildung $f \colon V\ppf W$ heißt birational, wenn es eine rationale Abbildung $g \colon W\ppf V$ gibt, so dass $f\circ g$ und $g\circ f$ definiert sind und jeweils die Identität ergeben.
  \end{enumerate}
\end{db}

\begin{bsp}\label{1.6.6}
  \begin{enumerate}
  \item\Label{1.6.6a} Seien $V_1=\V(X\cdot Y)$, $V_2=\A^1(K)$ und $V_3=\A^1(K)$. Weiter seien folgende rationale Abbildungen gegeben:
  \begin{align*}f\colon&V_1\ppf V_2, \quad (x,y)\mapsto x,\\
  g\colon&V_2\ppf V_3, \quad x\mapsto \tfrac{1}{x},\end{align*}

  Hier ist $\Def(g\circ f)\subseteq \V(Y)$, also insbesondere nicht dicht in $V_1$.
  \item\Label{1.6.6b} Die Abbildung $\sigma \colon \A^2(K)\ra \A^2(K),$  $(x,y)\mapsto (\frac{1}{x},\frac{1}{y})$ ist auf $\A^2(K)\setminus \V(X\cdot Y)$ definiert, also eine rationale Abbildung.

  Beachte: $\sigma\circ \sigma=\id{\restrict{\A^2(K)\setminus\V(X\cdot Y)}}$, folglich ist $\sigma$ sogar birational.

  \item\label{1.6.6c} Seien $V_1=\A^1(K)$, $V_2=\V(Y^2-X^3)$ (vgl. \cref{1.4.8e}). Betrachte die rationalen Abbildungen 
  \[r_1 \colon V_1\ra V_2, \quad t\mapsto (t^2,t^3),\]
  \[r_2 \colon V_2\supseteq \D(X)\ra V_1, \quad (x,y)\mapsto \frac{y}{x},\]

  Dann erhalten wir:\begin{itemize}
     \item $r_2\circ r_1=\id_{U_1}$ mit $U_1=\Def(r_2\circ r_1)=\A^1(K)\setminus \{0\}$.
     \item $r_1\circ r_2=\id_{U_2}$ mit $U_2=\Def(r_1\circ r_2)=V_2\setminus \{(0,0)\}$.
     \end{itemize}
  Also ist $\V(Y^2-X^3)$ birational zu $\A^1(K)$, aber sie sind nicht isomorph.
  \end{enumerate}
\end{bsp}  

\begin{nbem}
Irreduzible affine Varietäten bilden zusammen mit den dominanten rationalen Abbildungen  eine Kategorie.
\end{nbem}
 
\begin{satz}\label{satz4} Sei $K$ algebraisch abgeschlossen. Dann ist die Kategorie der endlich erzeugten Körper\-erweiterungen $\Quotient{L}{K}$ mit $K$-Algebrenhomomorphismen äquivalent zur Kategorie der irreduziblen affinen Varietäten mit dominanten rationalen Abbildungen.
\end{satz}

\begin{proof} Wir definieren
\[\Phi\colon\begin{cases}V\mapsto K(V)\\f\mapsto f^{\sharp}\end{cases}\]
und zeigen, dass das eine Äquivalenz von Kategorien liefert. Dazu zeigen wir: 
  \begin{itemize}
  \item Für jede endlich erzeugte Körpererweiterung $\Quotient{L}{K}$ gibt es eine irreduzible affine Varietät $V$ mit $K(V)\cong L$.
  \item $\Phi$ ist volltreu, d.h. induziert eine Bijektion auf den Morphismenmengen.
  \end{itemize}
Zum ersten Punkt: Wir wählen endlich viele Erzeuger $x_1,\dotsc,x_n$ der Körpererweiterung $\Quotient{L}{K}$, d.h. $L=K(x_1,\dotsc,x_n)$. Wir definieren $A=K[x_1,\dotsc,x_n]$ als die von $x_1,\dotsc,x_n$ erzeugte $K$-Algebra in $L$. Dann ist $A$ eine endlich erzeugte, reduzierte, nullteilerfreie
$K$-Algebra. Deshalb gibt es nach \cref{satz3} eine affine Varietät $V$ mit $K[V]\cong A$. Da $A$ nullteilerfrei ist, ist $V$ irreduzibel und es gilt $K(V)\cong \Quot(A)\cong L$.

Zum zweiten Punkt zeigen wir: Sind $V,W$ affine Varietäten und $\alpha \colon K(W)\ra K(V)$ ein $K$-Algebrenhomomorphismus, so gibt es eine rationale Abbildung \[f=f_{\alpha}\colon V\ppf W\text{ mit }\alpha=f^{\sharp}.\]
%
Für den Beweis dieser Aussage seien $g_1,\dotsc,g_m$ Erzeuger von $K[W]$. 

Dann ist $\alpha(g_i)\in K(V)$ und $\bigcup_{i=1}^{m} \Def(\alpha(g_i))$ ist offen und dicht in $V$. Wähle 
\[U:=\D(g)\subseteq \bigcap_{i=1}^{m} \Def(\alpha(g_i))\text{ mit }g \in K[V].\]
Nach \cref{1.5.15} ist  $\D(g)$ affin, also existiert eine affine Varietät $Z$ und ein Isomorphismus $\Psi \colon Z\ra \D(g)$.
%kommutatives Diagramm
So erhalten wir
\[\Psi^{\sharp}\circ \alpha \colon K[W] \ra \O(Z)=K[Z]\]
und das induziert $\schlange{f} \colon Z\ra W$.

Damit ist $\dach{f}:=\schlange{f}\circ \Psi^{-1}$ eine reguläre Abbildung von $U$ nach $W$. Diese induziert eine rationale Abbildung $f \colon V\ppf W$, da $U$ dicht in $V$ liegt. Da $\Psi^{\sharp}\circ \alpha$ injektiv ist, ist $f$ dominant nach \cref{1.6.3}.

Nun bleibt nur noch zu zeigen, dass $\Phi$ treu ist. Seien dazu $r_1$, $r_2 \colon V\ppf W$ rationale Abbildungen und $r_1^{\sharp}=r_2^{\sharp}=\alpha$. Wir wählen $U$ offen, affin und so klein, dass $r_1$ und $r_2$ auf $U$ definiert sind.
Dann induzieren $r_1$ und $r_2$ aber als Abbildungen von $U\cong Z$ nach $W$ aufgefasst das selbe 
\[\alpha \colon K[W] \ra \O(U)\cong K[Z].\] 
Nun können wir \cref{satz3} anwenden und sehen $r_1\restrict{U}=r_2\restrict{U}$ und da $U$ dicht in $V$ liegt, folgt $r_1=r_2$.
\end{proof}

\sect{Spektrum eines Rings}

\addtocounter{thmcnt}{-1}
\begin{erinnerung}[aus Algebra {\scshape ii}]\label{1.7.0}
  Seien $R$, $S$ Ringe und $f\colon R\ra S$ ein Ringhomomorphismus.
  \begin{enumerate}
  \item Ist $J\subseteq S$ ein Ideal, ist $I=f^{-1}(J)\subseteq R$ ein Ideal. Ist dabei $J$ ein Primideal, so ist auch $I$
    ein Primideal.
  \item Ist $f$ surjektiv und $I\subseteq R$ ein Ideal, dann ist $J=f(I)\subseteq S$ ein Ideal. Ist dabei $I$ ein Primideal und
    $\Kern f\subseteq I$, dann ist $J$ ein Primideal.
  \end{enumerate}
  Insgesamt gilt also: Primideale in $S$ entsprechen Primidealen in $R$, die den Kern von $f$ enthalten (falls $f$ surjektiv
  ist).
\end{erinnerung}

\newlength\breite
\settowidth\breite{Radikalideale in $K\polyx$,}
\begin{eb}\label{1.7.1}
  Sei $V\subseteq\A^n(K)$ eine affine Varietät ($K$ algebraisch abgeschlossen). Dann haben wir Bijektionen
  \begin{enumerate}
  \item\Label{1.7.1.1}\mbox{$\{\text{Untervarietäten von }V\}$\begin{minipage}[t]{.5\textwidth}$ \lra \left\{%
  \begin{minipage}[c]{\breite}\begin{center}%
    Radikalideale in $K\polyx$,\par die $\I(V)$ enthalten%
  \end{center}\end{minipage}%
  \right\}$\par $\lra\; \{\text{Radikalideale in }K[V]\}$\end{minipage}}
  \item\Label{1.7.1.2} $\displaystyle \{\text{irreduzible Untervarietäten von }V\} \lra \{\text{Primideale in
    }K[V]\}$
  \item\Label{1.7.1.3} $\displaystyle\{\text{Punkte in }V\} \lra \{\text{maximale Ideale in }K[V]\}$
  \end{enumerate}
\end{eb}
\begin{proof}
  \begin{enumerate}
  \item[\ref{1.7.1.1}] Das folgt aus \cref{1.1.3}, \cref{1.1.5}, \cref{HNS} und \cref{1.7.0}.
  \item[\ref{1.7.1.2}] Das folgt aus \cref{1.2.9} und \cref{1.7.0}.
  \item[\ref{1.7.1.3}] Die eine Zuordnung ist 
  \[p=(x_1,\dotsc,x_n)\mapsto \m_p=\I(\{p\})=(X_1-x_1,\dotsc,X_n-x_n).\] 
  Für die andere sei $\m$ ein maximales Ideal in $K[V]$, dann gibt es einen Isomorphismus 
  \[\alpha\colon\Quotient{K\polyx}{\m}\ra K\]
   und wir wählen $p=(\alpha(X_1),\dotsc,\alpha(X_n))$.
  \end{enumerate}
\end{proof}

\begin{db}\label{1.7.2}
  Sei $R$ ein kommutativer Ring mit Eins.
  \begin{enumerate}
  \item\Label{1.7.2i} Die Menge $\Spec R := \{\wp\subseteq R \mid \wp\text{ ist Primideal}\}$ heißt \emph{Spektrum von $R$\index{Spektrum}}.
  \item\Label{1.7.2ii} Für eine Teilmenge $S\subseteq R$ heißt $\V(S) := \{\wp\in\Spec R \mid \wp\supseteq S\}$
    \emph{Verschwindungsmenge von $S$\index{Verschwindungsmenge}} und es gilt $\V(S)=\V((S))$.
  \item\Label{1.7.2iii} Die Mengen $\V(I)$ für Ideale $I\subseteq R$ bilden die abgeschlossenen Mengen einer Topologie auf
    $\Spec R$, diese heißt \emph{Zariski-Topologie\index{Zariski-Topologie}}.
  \item\Label{1.7.2iv} Für $Z\subseteq\Spec R$ heißt $\I(Z):=\displaystyle\bigcap_{\wp\in Z}\wp$ das \emph{Verschwindungsideal von $Z$\index{Verschwindungsideal}}.
  \end{enumerate}
\end{db}
\begin{proof}
  \begin{enumerate}
  \item[\ref{1.7.2ii}] Die Inklusion \enquote{$\supseteq$} ist klar. Für die umgekehrte Inklusion sei $\wp\in\V(S)$, also
    $\wp\supseteq S$. Da $\wp$ ein Ideal ist, folgt $\wp\supseteq (S)$, also $\wp\in\V((S))$.
  \item[\ref{1.7.2iii}] Das zeigt man wie in \cref{1.2.1}, insbesondere gilt
    \[\displaystyle\bigcap_{\lambda\in\Lambda}\V(I_\lambda) =
    \V\Bigl(\bigcup_{\lambda\in\Lambda}I_\lambda\Bigr)=\V\Bigr(\sum_{\lambda\in\Lambda}I_\lambda\Bigr)\] und
    $\V(I_1)\cup\V(I_2)=\V(I_1\cap I_2)=\V(I_1\cdot I_2)$.
  \end{enumerate}
\end{proof}

\begin{db}\label{1.7.3}
  \begin{enumerate}
  \item\Label{1.7.3i} Elemente aus $R$ können als Funktionen aufgefasst werden:
    \[ r\colon \Spec R \ra \!\!\!\bigcup_{\wp\in\Spec R}\!\!\! \Quot\bigl(\Quotient{R}{\wp}\bigr),\quad
    p\mapsto\Bar{r}\in\Quot\bigl(\Quotient{R}{\wp}\bigr) \]
  \item\Label{1.7.3ii} Ein Primideal $\wp$ ist genau dann Nullstelle von $r\in R$, wenn $\Bar{r}=0$ in $\Quotient{R}{\wp}$, also wenn
    $r\in \wp$ gilt.
  \end{enumerate}
\end{db}

\begin{bsp}\label{1.7.4}
  \begin{enumerate}
  \item\Label{1.7.4i} Sei $R=K[V]$ für eine affine Varietät $V$ und einen algebraisch abgeschlossenen Körper $K$. Dann enthält
    $\Spec R$ je einen Punkt für jede irreduzible Untervarietät von $V$. Ist $\m$ ein maximales Ideal in $\Spec R$, dann ist
    $\Quot\bigl(\Quotient{R}{\m}\bigr)\cong K$.
  \item\Label{1.7.4ii} Für $R=\Z$ ist $\Spec R = \{(p)\mid p\in\P\}\cup\{(0)\}$, kann also als $\P\cup\{0\}$ aufgefasst werden.
  \end{enumerate}
\end{bsp}

%22.11.10
\begin{bem}\label{1.7.5}
  \begin{enumerate}
  \item\Label{1.7.5i} Für $Z\subseteq\Spec R$ ist $\V(\I(Z))=\Bar{Z}$ der Abschluss von $Z$.
  \item\Label{1.7.5ii} Für $f\in R$ sei $\D(f)=\Spec R\setminus\V(f)$ = $\{\wp\in\Spec R\mid f\notin \wp\}$. Die Mengen $\D(f)$
    bilden eine Basis der Topologie auf $\Spec R$.
  \item\Label{1.7.5iii} Sei $V\subseteq\Spec R$ nichtleer. Dann ist $V$ genau dann irreduzibel, wenn $\I(V)$ ein Primideal ist.
  \end{enumerate}
\end{bem}
\begin{proof}
  \begin{enumerate}
  \item[\ref{1.7.5i}] Es gilt:\vspace*{-6pt}
  \[\V(\I(Z))=\{\wp\in\Spec R\mid \wp\supseteq\I(Z)\} = \{\wp\in\Spec R\mid \wp\supseteq\bigcap_{q\in Z}q\}\supseteq Z.\]
    Daraus folgt, dass der Abschluss von $Z$ in $\V(\I(Z))$ liegt. Für die andere Inklusion sei
    $\Bar{Z}=\V(J)=\{q\in\Spec R\mid q\supseteq J\}$. Da $Z\subseteq\Bar{Z}$, gilt für alle $\wp\in Z$, dass $\wp\supseteq J$. Sei
    nun $q\in\V(\I(Z))$. Dann gilt \[q\supseteq\bigcap_{\wp\in Z}\wp\supseteq J,\] also $q\in\V(J)$.
  \item[\ref{1.7.5ii}] Das geht wie in \cref{1.5.9iii}.
  \item[\ref{1.7.5iii}] Das geht wie in \cref{1.2.9}.
  \end{enumerate}
\end{proof}

\begin{bem}\label{1.7.6}
  \begin{enumerate}
  \item\Label{1.7.6i} Für $\wp\in\Spec R$ gilt $\Bar{\{\wp\}}=\{q\in\Spec R\mid q\supseteq \wp\}=\V(\wp)$.
  \item\Label{1.7.6ii} Für $\wp\in\Spec R$ ist $\{\wp\}$ genau dann abgeschlossen, wenn $\wp$ ein maximales Ideal ist.
  \item\Label{1.7.6iii} Sei $R$ nullteilerfrei. Dann gilt $\Bar{\{(0)\}}=\Spec R$.
  \end{enumerate}
\end{bem}

\begin{bem}\label{1.7.7}
  Sei $X$ ein topologischer Raum und $x\in X$ mit $\Bar{\{x\}}=X$. Dann heißt $x$ \emph{generischer Punkt\index{generischer Punkt}}.
\end{bem}

\begin{bsp}\label{1.7.8}
  Sei $V$ eine affine Varietät.
  \begin{enumerate}
  \item Die abgeschlossenen Punkte in $\Spec K[V]$ entsprechen bijektiv den Punkten der affinen Varietät $V$.
  \item Ist $V$ irreduzibel, dann ist $\{(0)\}$ ein generischer Punkt. Dazu gehört gerade $V$ als Untervarietät von $V$.
  \end{enumerate}
\end{bsp}

%24.11.10

\begin{bem}\label{1.7.9}
  Sei $\alpha\colon R\ra R'$ ein Ringhomomorphismus.
  \begin{enumerate}
  \item\Label{1.7.9i} $\alpha$ induziert eine stetige Abbildung \[ f_\alpha\colon \Spec R'\ra\Spec R,\quad
    \wp\mapsto\alpha^{-1}(\wp). \]
  \item\Label{1.7.9ii} Ist $I\subseteq R$ ein Ideal, so ist $f_\alpha^{-1}(\V(I))=\V(\alpha(I))$.
  \end{enumerate}
\end{bem}
\begin{proof}
  \begin{enumerate}
  \item[\ref{1.7.9i}] Nach \cref{1.7.0} ist $f_\alpha$ wohldefiniert. Die Stetigkeit folgt aus \ref{1.7.9ii}.
  \item[\ref{1.7.9ii}]
    %\begin{multline*}
      $\wp\in f_\alpha^{-1}(\V(I)) \iff f_\alpha(\wp)\in\V(I) \iff f_\alpha(\wp)\supseteq I$% \\
      \[\iff \alpha^{-1}(\wp)\supseteq I \iff \wp\supseteq\alpha(I) \iff \wp\in\V(\alpha(I))\]
%    \end{multline*}
  \end{enumerate}
\end{proof}

\begin{prop}\label{1.7.10}
  Sei $K$ algebraisch abgeschlossen und $V\subseteq\A^n(K)$ eine affine Varietät. Dann ist die Abbildung
  \[ \m\colon V\ra \Spec K[V], \quad x\mapsto \m_x=\{f\in K[V]\mid f(x)=0\} \]
  injektiv und stetig.
\end{prop}
\begin{proof}
  Die Injektivität folgt aus \cref{1.7.1}. Für den Beweis der Stetigkeit sei $I\subseteq K[V]$ ein Ideal und
  $Z:=\V(I)\subseteq\Spec K[V]$. Für die affine
  Varietät $Z$ gilt
  \[x\in\V(I) \iff f(x)=0 \text{ für alle }f\in I \iff f\in \m_x \text{ für alle }f\in I \iff I\subseteq \m_x. \]
  Damit gilt $\m^{-1}(Z)=\{x\in V\mid \m_x\in Z\}=\{x\in V\mid \m_x\supseteq I\}=\V(I)\subseteq V$. Also sind Urbilder
  abgeschlossener Mengen abgeschlossen.
\end{proof}

\begin{prop}\label{1.7.11}
  Sei $R$ ein kommutativer Ring mit Eins, $I\subseteq R$ ein Ideal und $V\subseteq\Spec R$ eine abgeschlossene Menge. Dann
  gelten:
  \begin{enumerate}
  \item\Label{1.7.11i} $\V(\I(V))=V$
  \item\Label{1.7.11ii} $\I(\V(I))=\sqrt{I}$
  \end{enumerate}
\end{prop}
\begin{proof}
  \begin{enumerate}
  \item[\ref{1.7.11i}] Aus \cref{1.7.5i} folgt $\V(\I(V))=\Bar{V}=V$.
  \item[\ref{1.7.11ii}] Nach \cref{1.7.12} gilt $\displaystyle\I(\V(I))=\!\!\bigcap_{\wp\in\V(I)}\!\!\wp=\bigcap_{\wp\supseteq I}\wp=\sqrt{I}$.
  \end{enumerate}
\end{proof}

\begin{lem}[Lemma von Krull]\label{1.7.12}
  Sei $R$ ein kommutativer Ring mit Eins.
  \begin{enumerate}
  \item\Label{1.7.12i} Sei $S\subseteq R$ ein multiplikatives System und $I\subseteq R$ ein Ideal, das disjunkt zu $S$ ist. Dann
    gibt es ein Primideal $\wp\subseteq R$, das $I$ enthält und ebenfalls zu $S$ disjunkt ist.
  \item\Label{1.7.12ii} Es gilt: $\displaystyle\!\!\!\bigcap_{\substack{\wp\in\Spec R\\\wp\supseteq I}}\!\!\!\wp = \sqrt{I}$.
  \end{enumerate}
\end{lem}
\begin{proof}
  \begin{enumerate}
  \item[\ref{1.7.12ii}] Die Inklusion \enquote{$\supseteq$} gilt, weil der Schnitt von Radikalidealen wieder ein Radikalideal
    ist. Die andere Inklusion folgt aus \ref{1.7.12i}: Sei $a\in R\setminus\sqrt{I}$ und $S:=\{a^n\mid n\in\N_0\}$. Dann ist $S$
    ein multiplikatives System und $I\cap S=\leer$. Nach \ref{1.7.12i} gibt es dann ein Primideal $\wp$ mit $\wp\supseteq I$ und
    $\wp\cap S=\leer$. Damit fliegt $a$ im Schnitt raus.
  \item[\ref{1.7.12i}] Betrachte den kanonischen Ringhomomorphismus $\phi\colon R\ra R_S$. Sei $I'=\langle\phi(I)\rangle$ das von $\phi(I)$
    erzeugte Ideal in $R_S$, also $I'=\{\frac{f}{a}\in R_S\mid f\in I,\ a\in S\}$.

    Zunächst überlegen wir uns, dass $I'\neq R_S$, also $1\notin I'$. Denn wäre $1\in I'$, dann gäbe es $f\in I$ und $a\in S$
    mit $1=\frac{f}{a}$ in $R_S$, d.h. es gäbe ein
    \[t\in S\text{ mit }t(f-a)=0.\] 
    Dann wäre $tf=ta$ sowohl in $I$ als auch in $S$---ein Widerspruch zur Disjunktheit!

    Somit ist $I'$ ein echtes Ideal und damit in einem maximalen Ideal $\m'$ enthalten. Dann ist $\wp:=\phi^{-1}(\m')$ ein
    Primideal, enthält $I$ und hat leeren Schnitt mit $S$, denn sonst wäre für $s\in S\cap I$ das Bild $\phi(s)$ eine Einheit in
    $R_S$.
  \end{enumerate}
\end{proof}

\begin{ziel}
  Wir suchen eine Strukturgarbe $\O_X$ auf $X=\Spec R$ mit $\O_X(\D(f))\cong R_f$.
\end{ziel}

Ab jetzt verwenden wir Folgende \textsc{Schreibweisen}:
\begin{itemize}
\item Für $f\in R$ und $S:=\{f^n\mid n\in\N_0\}$ schreiben wir $R_f:=R_S$.
\item Für ein Primideal $\wp\subseteq R$ und $S:=R\setminus\wp$ schreiben wir $R_\wp:=R_S$.
\item Es gilt $a=0$ in $R_S$ genau dann, wenn es ein $t\in S$ gibt mit $ta=0$.
\end{itemize}

\textsc{Vorüberlegung}: In $\Spec R$ ist $\D(g)\subseteq \D(f)$ äquivalent zu $\V(g)\supseteq\V(f)$. In diesem Fall gilt
$g\in\sqrt{(f)}$, d.h. es gibt ein $m\in\N$ und $h\in R$ mit $g^m=fh$. Wir erhalten so eine Abbildung
\[ \rho_{\D(g)}^{\D(f)}\colon R_f\ra (R_f)_h=R_{fh}=R_{g^m}=R_g. \]

Wir setzen $\B=\{\D(g)\mid g\in R\}$, was eine Basis der Topologie auf $\Spec R$ ist.

\begin{prop}\label{1.7.13}
  Sei $R$ ein noetherscher kommutativer Ring mit Eins.
  Auf $\Spec R$ gibt es eine eindeutige Garbe $\F$ von Ringen mit $\F(\D(f))\cong R_f$ und für $\D(g)\subseteq
  \D(f)$ ist die Restriktionsabbildung $\rho_{\D(g)}^{\D(f)}$ wie in der Vorüberlegung definiert.
\end{prop}
\begin{proof}
  Wir definieren $\F(\D(f))=R_f$ und zeigen
  \begin{enumerate}
  \item\Label{1.7.13a} $\F$ erfüllt auf $\B$ die Garbeneigenschaften, d.h.
    \begin{enumerate}[label=(\textsc{g}\arabic*)]
    \item\label{G1} $\rho_{U''}^U=\rho_{U''}^{U'}\circ\rho_{U'}^U$ und $\rho_U^U=\id_{\F(U)}$ für
      $U,U',U''\in\B$ mit $U''\subseteq U'\subseteq U$.
    \item\label{G2} Ist $U=\bigcup_{i\in I}U_i$ (mit $U,U_i\in\B$), $f_i\in\F(U_i)$ und gilt für alle
      $U'\in\B$ mit $U'\subseteq U_i\cap U_j$, dass $\rho_{U'}^{U_i}=\rho_{U'}^{U_j}$, dann gibt es genau ein
      $f\in U$ mit $\rho_{U_i}^U(f)=f_i$.
    \end{enumerate}
  \item\Label{1.7.13b} $\F$ lässt sich eindeutig zu einer Garbe fortsetzen.
  \end{enumerate}
  \begin{enumerate}
  \item[zu] \ref{1.7.13a}: Die erste Garbeneigenschaft ist klar. 
  
  Für die zweite sei ohne Einschränkung $I=\{1,\dotsc,m\}$ endlich
    (denn wie in \cref{1.5.9} sind offene Mengen quasikompakt). Sei $U=\D(f)$ und $U_i=\D(f_i)$. Wegen
    \[\D(f)=\D(f_1)\cup\dotsm\cup \D(f_m)\] ist $\V(f)=\V(f_1,\dotsc,f_m)$, also: 
    \begin{equation}\label{1.7.13s}\text{Es gibt ein }k\in\N\text{ mit }f^k\in(f_1,\dotsc,f_m).\tag{$*$}\end{equation}
%
    \textsc{Eindeutigkeit}: Sei $g\in R_f$ mit $g=0$ in $R_{f_i}$ für alle $i\in\{1,\dotsc,m\}$. Dann gibt es ein $M\in\N$ mit
    $gf_i^M=0$ für alle $i\in I$. Für genügend großes $N$ gilt 
    \[(f_1,\dotsc,f_m)^N\subseteq(f_1^M,\dotsc,f_m^M).\] 
    Mit \cref{1.7.13s}
    folgt $f^{kN}\in(f_1,\dotsc,f_m)^N\subseteq(f_1^M,\dotsc,f_m^M)$, also gibt es $a_i$ mit
    \[f^{kN}=a_1f_1^M+\dotsm+a_mf_m^M.\]
    Dann gilt $f^{kN}g=a_1f_1^Mg+\dotsm+a_mf_m^Mg=0$, also gilt $g=0$ in $R_f$.

    \textsc{Existenz}: Gegeben $g_i\in R_{f_i}$, sodass $g_i=g_j$ in $R_{f_if_j}$, suchen wir ein $g\in R_f$, sodass $g=g_i$ in
    allen $R_{f_i}$ gilt. Da $g_i=g_j$ in $R_{f_if_j}$, gibt es ein genügend großes $N$, für das \[g_if_i^Nf_j^N=g_jf_i^Nf_j^N\] gilt,
    und das ohne Einschränkung nicht von $i$ und $j$ abhängt. Insbesondere gibt es wie gerade eben ein $w\in\N$ mit
    \[f^w\in(f_1^N,\dotsc,f_m^N)\text{ und }f^w=\sum_{i=1}^ma_if_i^N.\] 
    Wir definieren 
    \[g=\frac{1}{f^w}\sum_{i=1}^ma_if_i^Ng_i\in R_f.\] 
    Dann gilt in $R_{f_i}$
    \[ gf_j^N=\frac{1}{f^w}\sum_{i=1}^ma_if_i^Nf_j^Ng_j=\frac{1}{f^w}f^wg_jf_j^N=g_jf_j^N, \]
    also $g=g_j$ in $R_{f_j}$.
    % 29.11.10
  \item [zu] \ref{1.7.13b}: Das folgt aus dem folgenden \cref{1.7.14}.
  \end{enumerate}
\end{proof}

\begin{lem}\label{1.7.14}
  Sei $X$ ein topologischer Raum und $\B$ eine Basis. Eine \emph{$\B$-Garbe\index{$\B$-Garbe}} besteht aus einem Ring
  $\F(U)$ für jedes $U\in\B$ und Restriktionsabbildungen $\rho_{U'}^U$ für alle $U,U'\in\B$ mit
  $U'\subseteq U$, sodass die Garbeneigenschaften \ref{G1} und \ref{G2} aus \cref{1.7.13} erfüllt sind.

  Jede $\B$-Garbe lässt sich eindeutig zu einer Garbe auf $X$ fortsetzen.
\end{lem}
\begin{proof}
  Sei $U\subseteq X$ eine beliebige offene Menge. Wir definieren
  \[ \F_U = \varprojlim_{\substack{\schlange{U}\subseteq U\\\schlange{U}\in\B}}\F(\schlange{U}) =
     \Bigl\{f_{\schlange{U}}\in\prod_{\substack{\schlange{U}\subseteq U\\\schlange{U}\in\B}}\F(\schlange{U}) \ \Big|\ 
     \rho_{\dach{U}}^{\schlange{U}}(f_{\schlange{U}})=f_{\dach{U}}\text{ für }
     \dach{U},\schlange{U}\in\B \text{ mit } \dach{U}\subseteq\schlange{U}\subseteq U \Bigr\}. \]
     Mit den Restriktionsabbildungen
     \[\rho_{U'}^U\colon (f_{\schlange{U}})_{\substack{\schlange{U}\subseteq U\\\schlange{U}\in\B}} \mapsto
     (f_{\schlange{U}})_{\substack{\schlange{U}\subseteq U'\\\schlange{U}\in\B}} \text{ (wobei $U'\subseteq U$)}\] 
     ist $\F$ eine
     Garbe. Außerdem stimmt für $U\in\B$ das neue $\F(U)$ mit dem ursprünglichen $\F(U)$ überein via
     dem Isomorphismus 
     \[(f_{\schlange{U}})_{\substack{\schlange{U}\subseteq U\\\schlange{U}\in\B}} \mapsto f_U.\]
     Schließlich ist $\F$ eindeutig bestimmt, denn für einen weiteren Kanditaten $\G$ stimmen $\F$
     und $\G$ auf jeder Basismenge überein und damit nach \ref{G2} auch auf allen offenen Mengen.
\end{proof}

\begin{bem*}
  In \cref{1.7.13} muss $R$ nicht noethersch sein! Es gilt für beliebige Ringe $R$ und $f\in R$, dass $\D(f)$ quasikompakt ist
  (siehe \cref{1.7.15}).
\end{bem*}

\begin{bem}\label{1.7.15}
  Sei $R$ ein beliebiger kommutativer Ring mit Eins und $f$, $f_i$ ($i\in I$) aus $R$.
  \begin{enumerate}
  \item\Label{1.7.15i} Es gilt $\displaystyle \D(f)=\bigcup_{i\in I}\D(f_i)$ genau dann, wenn $\sqrt{(f)}=\sqrt{(f_i\mid i\in
      I)}$. 
      
      Insbesondere gilt $\displaystyle\Spec R=\bigcup_{i\in I}\D(f_i)$ genau dann, wenn $1\in(f_i\mid i\in I)$.
  \item\Label{1.7.15ii} $\D(f)$ ist quasikompakt.
  \item\Label{1.7.15iii} Wenn $R$ noethersch ist, ist $\Spec R$ ein noetherscher topologischer Raum.
  \end{enumerate}
\end{bem}

    \begin{w}
      Die Gegenrichtung stimmt nicht! Insbesondere ist $\Spec R$ nicht immer ein noetherscher topologischer Raum.
    \end{w}

\begin{proof}
  \begin{enumerate}
  \item[\ref{1.7.15i}] Es gilt 
  \[\displaystyle \D(f)=\bigcup_{i\in I}\D(f_i) \iff \V(f)=\bigcap_{i\in I}\V(f_i) \iff
    \sqrt{(f)}=\sqrt{(f_i\mid i\in I)}.\]
%
     Für den zweiten Teil verwende $\D(1)=\Spec R$.
  \item[\ref{1.7.15ii}] Sei $\D(f)=\bigcup_{i\in I} U_i$ und ohne Einschränkung $U_i=\D(f_i)$ für $f_i\in R$, da die $\D(g)$ eine
    Basis der Topologie bilden. Nach \ref{1.7.15i} gibt es ein $m\in\N$ und $i_1,\dotsc,i_n\in I$ mit
    $f^m\in(f_{i_1},\dotsc,f_{i_n})$. Damit folgt \[ \D(f)=\D(f^m)\subseteq\bigcup_{j=1}^n \D(f_{i_j}) \subseteq\bigcup_{i\in I}
    \D(f_i) = \D(f), \] also $\D(f)=\D(f_{i_1})\cup\dotso\cup \D(f_{i_n})$. Es genügen also endlich viele, um $\D(f)$ zu überdecken.
  \item[\ref{1.7.15iii}] Wenn wir eine absteigende Kette $V_1\supsetneq V_2\supsetneq V_3\supsetneq\dotso$ von abgeschlossenen
    Mengen hätten, die nicht stationär wird, wäre $\I(V_1)\subsetneq\I(V_2)\subsetneq\I(V_3)\subsetneq\dotso$ eine aufsteigende
    Kette von Idealen, die ebenfalls nicht stationär wird, im Widerspruch dazu, dass $R$ noethersch ist.
  \end{enumerate}
\end{proof}

\begin{dfn}\label{1.7.16}
  \begin{enumerate}
  \item\Label{1.7.16i} Sei $R$ ein beliebiger kommutativer Ring mit Eins, $X=\Spec R$ und $O_X$ die Garbe aus
    \cref{1.7.13}. Der geringte Raum $(X,\O_X)$ heißt \emph{affines Schema zu $R$\index{affines Schema}}.
  \item\Label{1.7.16ii} Ist $R$ noethersch, so heißt $(X,\O_X)$ ein \emph{noethersches affines Schema\index{noethersches affines Schema}}.
  \end{enumerate}
\end{dfn}

\begin{prop}\label{1.7.17}
  Sei $V$ eine affine Varietät über einem algebraisch abgeschlossenen Körper $K$ und $R=K[V]$. Das affine Schema zu $R$ ist
  noethersch und für die stetige Injektion \[\m\colon V\inj\Spec R,\quad x\mapsto \m_x=\{f\in K[V]\mid f(x)=0\} \] gilt
  $\m_*\O_V\cong\O_{\Spec R}$.
\end{prop}
Hierbei ist $\O_V$ die Garbe der regulären Funktionen auf $V$, $\m_*\O_V$ die Bildgarbe auf $\Spec R$, definiert durch
$\m_*\O_V(U)=\O_V(\m^{-1}(U))$ (siehe auch Aufgabe 5 auf Übungsblatt 4); und $\m_*\O_V\cong\O_X$ heißt: für jedes offene $U$
existiert ein Isomorphismus \[\theta_U\colon \m_*\O_V(U)\ra\O_X(U)\] und diese Isomorphismen sind mit den Restriktionsabbildungen
verträglich, d.h. für $U'\subseteq U$ ist das Diagramm
\[\begin{tikzpicture}
\matrix (m) [matrix of math nodes, row sep=3em, column sep=5em, text height=1.5ex, text depth=0.25ex]
{ \m_*\O_V(U)  & \O_X(U) \\
  \m_*\O_V(U') & \O_X(U') \\};
\path[->,font=\scriptsize]
(m-1-1) edge node [auto] {$\theta_U$} (m-1-2) 
(m-1-2) edge node [auto] {$\rho_{U'}^U$} (m-2-2)
(m-1-1) edge node [auto] {$\rho_{U'}^U$} (m-2-1) 
(m-2-1) edge node [auto] {$\theta_{U'}$} (m-2-2);
\end{tikzpicture}\]
kommutativ.
\begin{proof}
  Wir zeigen, dass die Garben auf der Basis übereinstimmen. Dann folgt die Behauptung aus \cref{1.7.14}. Sei
  $U=\D(f)=\{\wp\in\Spec R\mid \wp\not\ni f\}$. Aus \cref{1.7.13} folgt einerseits $\O_X(\D(f))=R_f$. Andererseits gilt 
  \[\m^{-1}(U) = \{x\in V \mid \m_x\not\ni f\} = \{x\in V\mid f(x)\neq0\} = \D(f)\subseteq V,\]
   also $\m_*\O_V(U)=\O_V(\m^{-1}(U))\cong R_f$ wegen
  \cref{1.5.11}. Außerdem passen die Restriktionsabbildungen zusammen.
\end{proof}


\chapter{Projektive Varietäten}\label{kap2}

Wir hatten bereits gesehen: Ein Manko an $\A^n(K)$ ist, dass sich Geraden nicht immer schneiden. Daher wollen wir $\A^n(K)$ zum
projektiven Raum $\P^n(K)$ vergrößern. Die Punkte darin sollen die Geraden in $\A^{n+1}(K)$ sein.

%\sect{Der Projektive Raum für Schüler}

%29.11.10
\sect{Der Projektive Raum \texorpdfstring{$\P^{n}(k)$}{P\textasciicircum\relax n(k)}}

Sei $k$ immer ein Körper und $n\in\N_{0}$.

\begin{dfn}\label{2.2.1}
Der \emph{$n$-dimensionale projektive Raum\index{projektiver Raum}} über $k$ besteht aus allen Ursprungs\-geraden in $k^{n+1}$:
\[ \P^{n}(k):=\displaystyle\Quotient{k^{n+1}\!\setminus\!\{0\}}{\sim}\:, \] wobei
$(x_{1},\dotsc,x_{n+1})\sim(y_{1},\dotsc,y_{n+1}):\Longleftrightarrow\exists\;\lambda\in k^{\times}\colon y_{i}=\lambda x_{i}\;\forall i$.
Für die Äquivalenzklasse von $(x_{0},\dotsc,x_{n})$ schreiben wir
\[(x_{0}:x_{1}:\dotsm:x_{n})\]
und nennen sie die \emph{homogenen Koordinaten des Punktes\index{homogenen Koordinaten}}.
\end{dfn}

\begin{bsp}\label{2.2.2}\begin{enumerate}
\item Für $n=0$ ist $\P^{0}(k) = \{(1)\} =: \{\infty\}$.
\item Für $n=1$ ist \[\P^{1}(k)=\Quotient{\{(x_{0}:x_{1})\mid(x_{0},x_{1})\neq(0,0)\}}{\sim}=\{(1,t)\mid t\in k\}\cup\{(0,1)\}.\] \vspace*{-6pt}Wir sehen:\vspace*{-12pt}
\[\mu\colon\P^{1}(k)\ra k\cup\{\infty\},\quad(x_{0}:x_{1})\mapsto\begin{cases}\frac{x_{1}}{x_{0}},&x_{0}\neq0,\\\infty,&x_{0}=0,\\\end{cases}\]
ist eine Bijektion und ordnet jeder Geraden ihre Steigung zu.

{\scshape Spezialfall:}\begin{itemize}\item Für $k=\R$ ist $\P^{1}(\R)=\Quotient{\S^{1}}{\{\pm1\}}$, entspricht also einer Kreislinie.
\item Für $k=\C$ ist $\P^{1}(\C)=\C\cup\{\infty\}$. Das lässt sich (z.B. mit Hilfe der riemannschen Zahlenkugel) mit der $\S^{2}$ identifizieren.
\end{itemize}
\item Es gilt $\P^{n}(\R)=\Quotient{\S^{n}(\R)}{\{\pm1\}}$ und $\P^{n}(\C)=\Quotient{\S^{n}(\C)}{\{c\in\C\mid |c|=1\}}$. Bezüglich der Quotiententopologie, die von $k^{n+1}$ mit der euklidischen Topologie herkommt, sind das kompakte Hausdorffräume.
\end{enumerate}
\begin{w}Das ist nicht die Zariski-Topologie!\end{w}
%1.12.10
\begin{enumerate}[resume]
\item Für $k=\FF_{p}$ ist
\[\card{\P^{n}(\FF_{p})}=\frac{p^{n+1}-1}{p-1}=1+\dotsm+p^{n}.\]
\end{enumerate}\end{bsp}

\begin{db}\label{2.2.3}
Für $i\in\{0,\dotsc,n\}$ sei \[\U_{i}:=\{x=(x_{0}:\dotsm:x_{n})\in\P^{n}(k)\mid x_{i}\neq 0\}.\] Das ist wohldefiniert(!) und es gilt:
\begin{enumerate}
\item\Label{2.2.3a} $\displaystyle\P^{n}(k)=\bigcup_{i=0}^{n}\U_{i}$
\item\Label{2.2.3b}\Label{phii} Sei
\[\phi_{i}\colon\A^{n}(k)\ra \P^{n}(k),\quad(y_{1},\dotsc,y_{n})\mapsto(y_{1}:\dotsm:y_{i}:1:y_{i+1}:\dotsm:y_{n}).\]
Das ist eine Einbettung mit Bild $\U_{i}$.
Die Abbildung
\[U_{i}\ra\A^{n}(k),\quad(x_{0}:\dotsm:x_{n})\mapsto\left(\frac{x_{0}}{x_{i}},\dotsc,\frac{x_{i-1}}{x_{i}},\frac{x_{i+1}}{x_{i}},\dotsc,\frac{x_{n}}{x_{i}}\right)\]
ist wohldefiniert und bijektiv und ihre Umkehrabbildung ist $\phi_{i}$.
\item\Label{2.2.3c}
Die Abbildung $\P^{n}(k)\setminus \U_{i}\ra\P^{n-1}(k),$
\begin{align*}(x_{0}:\dotsm:x_{n})\mapsto&(x_{0}:x_{1}:\dotsm:x_{i-1}:x_{i+1}:\dotsm:x_{n})\\&=:(x_{0}:\dotsm:\dach{x_{i}}:\dotsm:x_{n})\end{align*}
ist wohldefiniert und bijektiv.
\item\Label{2.2.3d} Damit erhalten wir eine Bijektion:
\[\P^{n}(k)\longleftrightarrow\A^{n}(k)\cup\A^{n-1}(k)\cup\dotsm\A^{1}(k)\cup\{\infty\}.\]
\end{enumerate}\end{db}
\begin{proof}\begin{enumerate}
\item[\ref{2.2.3a}] ist klar nach Definition der $\U_{i}$.
\item[\ref{2.2.3b}] Man rechnet einfach nach, dass die Abbildungen zueinander invers sind.
\item[\ref{2.2.3c}] Da $x_{i}=0$ gibt es ein $j\neq i$ mit $x_{j}\neq0$, also ist die Abbildung wohldefiniert. Wir geben wieder eine Umkehrabbildung an:
\begin{align*}\P^{n-1}(k)&\ra\P^{n}(k)\setminus \U_{i},\\(y_{0}:\dotsm:y_{n-1})&\mapsto(y_{0}:\dotsm:y_{i-1}:0:y_{i}:\dotsm:y_{n-1}).\end{align*}
\item[\ref{2.2.3d}] Das folgt induktiv aus \ref{2.2.3b} und \ref{2.2.3c}.
\end{enumerate}\end{proof}

\sect{Projektive Varietäten}

Auch in diesem Abschnitt sei $k$ immer ein Körper, $n\in\N_{0}$.

Ein unmittelbares Problem im Projektiven ist das Folgende: Sei $f\in k\ppolyx$. Dann ist die dazugehörige Polynomfunktion
\[\P^{n}(k)\ra k,\quad x=(x_{0}:\dotsm:x_{n})\mapsto f(x_{0},\dotsc,x_{n})\]
nur wohldefiniert, falls $f(\lambda x_{0},\dotsc,\lambda x_{n}) = f(x_{0},\dotsc,x_{n})$ für alle $\lambda\in k^{\times}$ ist. Das ist aber fast nie der Fall.

Um Varietäten zu definieren genügt es aber, dass \enquote{$f(x)=0$} wohldefiniert ist, das heißt
\[\forall\;\lambda\in k^{\times}\colon  f(x_{0},\dotsc,x_{n})=0\iff f(\lambda x_{0},\dotsc,\lambda x_{n})=0.\]
Das tun gerade die homogenen Polynome.

\begin{erinnerung}\label{2.3.1}\begin{enumerate}
\item\Label{2.3.1a} $f\in k\ppolyx$ heißt \emph{homogen vom Grad $d$\index{homogen vom Grad $d$}}, wenn
\[f=\sum_{i=0}^{l}a_{i}X_{0}^{r_{i,0}}\dotsm X_{n}^{r_{i,n}}\text{ mit }r_{i,0}+\dotsm+r_{i,n}=d\;\;\forall i.\]
\item\Label{2.3.1b} Falls $k$ unendlich ist, so gilt:
\[f\text{ ist homogen vom Grad }d\iff f(\lambda x_{0},\dotsc,\lambda x_{n}) = \lambda^{d}\cdot f(x_{0},\dotsc,x_{n})\;\forall\lambda\in k^{\times}.\]
\item\Label{2.3.1c} Seien $k=\FF_{3}$ und $f(X)=X^{3}+X$. Dann ist $f$ nicht homogen, aber es gilt
\[\forall\lambda\in k^{\times}\colon  f(\lambda x)=\lambda^{3}x^{3}+\lambda x=\lambda(x^{3}+x)=\lambda f(x),\]
da $\lambda^{3}=\lambda$ im $\FF_{3}$.
\end{enumerate}\end{erinnerung}
\begin{proof}\begin{enumerate}
\item[\ref{2.3.1b}] Für homogene Polynome gilt die Eigenschaft nach Definition. Sei also $f$ ein Polynom, das die reche Seite der Äquivalenz erfülle. Wir zerlegen $f$ in seine homogenen Komponenten, schreiben also
\[f=\sum_{i=0}^{l}f_{i}\text{ mit }\deg{f_{i}}=i\text{ und }f_{i}\text{ homogen.}\]
Nun setzen wir $g_{(x_{0},\olddotsc,x_{n})}(\lambda):=f(\lambda x_{0},\dotsc,\lambda x_{n})$, fassen also den Ausdruck als Polynom in $\lambda$ auf. Da die $f_{i}$ homogen sind und wegen der Vorraussetzung an $f$, gilt:\vspace*{-10pt}
\[\sum_{i=0}^{l}\lambda^{i}f_{i}(x) = \sum_{i=0}^{l}f_{i}(\lambda x) = f(\lambda x)=\lambda^{d}f(x) = \lambda^{d}\biggl(\sum_{i=0}^{l}f_{i}(x)\biggr).\]
Da $k$ unendlich ist, stimmen die Ausdrücke für alle $x$ als Polynome in $\lambda$ überein, es gilt also $f_{i}=0$ für $i\neq d$. Daher ist $f=f_{d}$ und damit homogen vom Grad $d$.
\end{enumerate}\end{proof}

\begin{dfn}\label{2.3.2}
$V\subseteq\P^{n}(k)$ heißt \emph{projektive Varietät\index{projektive Varietät}}, wenn es in $k\ppolyx$ eine Menge $\F$ von homogenen Polynomen gibt, so dass
\[V=\V(\F):=\{x\in\P^{n}(k)\mid f(x)=0\;\forall f\in\F\}.\]
\end{dfn}

\begin{bsp}\label{2.3.3}\begin{enumerate}
\item Die Menge $H_{i}:=\V(X_{i})=\P^{n}(k)\setminus \U_{i}$ ist eine projektive Varietät und nach \cref{2.2.3c} bijektiv zu $\P^{n-1}$.

Allgemein nennen wir $H=\V(f)$ mit $f\in k\ppolyx$ eine \emph{Hyperfläche\index{Hyperfläche}} und wenn $f$ sogar linear ist, nennen wir $H$ eine \emph{Hyperebene\index{Hyperebene}}.
\item Es gilt $\V(X_{0},\dotsc,X_{n})=\leer$.
\item Betrachte $V=\V(X_{0}X_{2}-X_{1}^{2})\subseteq\P^{2}(k)$.
\begin{itemize}
\item Zuerst bilden wir $V\cap \U_{0}$ in den $\A^{2}$ durch
\[(x_{0}:x_{1}:x_{2})\mapsto\left(\frac{x_{1}}{x_{0}},\frac{x_{2}}{x_{0}}\right)\]
ab. Ein Punkt $(x_{0}:x_{1}:x_{2})\in\P^{2}(k)$ liegt genau dann in $V$, wenn er \[x_{0}x_{2}-x_{1}^{2}=0\] erfüllt und für einen Punkt aus $V\cap \U_{0}$ ist diese Bedingung äquivalent zu $\frac{x_{2}}{x_{0}}-\bigl(\frac{x_{1}}{x_{0}}\bigr)^{2}=0$. Das heißt ein Punkt $(x,y)\in\A^{2}$ liegt genau dann im Bild von $V\cap \U_{0}$, wenn $y-x^{2}=0$. Diese Punkte liegen also alle auf einer Parabel.
\item Nun bilden wir $V\cap \U_{1}$ nach $\A^{2}$ ab. Analog betrachten wir die Abbildung
\[(x_{0}:x_{1}:x_{2})\mapsto\left(\frac{x_{0}}{x_{1}},\frac{x_{2}}{x_{1}}\right)\]
und sehen mit dem gleichem Argument, dass für die betroffenen Punkte \[\frac{x_{0}x_{2}}{x_{1}^{2}}-1=0\] gilt, also im $\A^{2}$ entsprechend: $xy-1=0$. Diese Punkte liegen also alle auf einer Hyperbel.
\end{itemize}\end{enumerate}\end{bsp}

\begin{erinnerung}\label{2.3.4}\begin{enumerate}
\item\Label{2.3.4a} Der Polynomring $S:=k\ppolyx$ ist ein \emph{graduierter Ring\index{graduierter Ring}} mit Graduierung 
\[S_{d}:=\{f\in k\ppolyx\mid f\text{ homogen von Grad }d\},\]
\mbox{d.h.:\hspace*{-1em}\begin{minipage}[t]{.85\textwidth}\begin{itemize}
\item $\displaystyle S=\bigoplus_{d\geq 0}S_{d}$; die Elemente in $S_{d}$ bezeichnen wir als \emph{homogene Elemente\index{homogene Elemente}}.\vspace*{-6pt}
\item Es gilt: $S_{d}\cdot S_{l}\subseteq S_{d+l}$.
\end{itemize}\end{minipage}}

$S$ ist sogar eine \emph{graduierte $k$-Algebra\index{graduierte $k$-Algebra}}, d.h. $k=S_{0}$ und die $S_{d}$ sind $k$-Vektorräume.
\item\Label{2.3.4b} Ein Ideal $I$ heißt \emph{homogen\index{homogen}}, wenn es von homogenen Elementen erzeugt wird.
\item\Label{2.3.4c} Die Summe, das Produkt, der Durchschnitt und die Radikalideale von homogenen Idealen sind wieder homogen.
\item\Label{2.3.4d} Es gilt: $I$ ist genau dann homogen, wenn
\vspace*{-10pt}
\[\forall f\in I\colon  f=\sum_{i=0}^{d}f_{i}\text{ mit }f_{i}\in S_{i}\implies f_{i}\in I.\]
\end{enumerate}\end{erinnerung}
\begin{proof}Als Beispiel beweisen wir, dass das Radikalideal eines homogenen Ideals wieder homogen ist. Für die restlichen Beweise verweisen wir auf Algebra {\scshape ii}.

Sei also $I$ ein homogenes Ideal und $f\in\sqrt{I}$. Seien $f_{d}\in S_{d}$ mit $f=\sum_{d=0}^{n}f_{d}$. Dann gibt es ein $m\in\N$, so dass $f^{m}\in I$ und es wir sehen:
\[f^{m}=f_{n}^{m}+\text{Terme von kleinerem Grad.}\]
$f_{n}^{m}$ ist auch ein homogenes Element, also gilt nach \ref{2.3.4d}: $f_{n}^{m}\in I$. Daher ist $f_{n}\in\sqrt{I}$ und rekursiv auch alle $f_{d}$. Damit ist, wieder nach \ref{2.3.4d}, $\sqrt{I}$ ein homogenes Ideal.
\end{proof}

\begin{db}\label{2.3.5}\begin{enumerate}
\item\Label{2.3.5a} Sei $V\subseteq\P^{n}(k)$. Dann nennen wir \[\I(V)=\langle\{f\in k\ppolyx\mid f\text{ homogen},f(x)=0\;\forall x\in V\}\rangle\] das \emph{Verschwindungsideal von $V$\index{Verschwindungsideal}}.
\item\Label{2.3.5b} Für ein homogenes Ideal $I$ heißt
\[\V(I):=\{x\in\P^{n}(k)\mid f(x)=0\text{ für alle homogenen }f\in I\}\]
\emph{Nullstellenmenge von $I$\index{Nullstellenmenge}}. Das ist eine projektive Varietät.
\item\Label{2.3.5c} $\I(V)$ ist ein Radikalideal.
\item\Label{2.3.5d} Seien $I_{1},I_{2}$ homogene Ideale, $V_{1},V_{2}$ projektive Varietäten. Dann gilt:
\begin{itemize}
\item $I_{1}\subseteq I_{2}\implies\V(I_{1})\supseteq\V(I_{2})$,
\item $V_{1}\subseteq V_{2}\implies\I(V_{1})\supseteq\I(V_{2})$, sowie
\item $V_{1}=V_{2}\iff\I(V_{1})=\I(V_{2})$.
\end{itemize}\end{enumerate}\end{db}
\begin{proof}\begin{enumerate}
\item[\ref{2.3.5c}] Sei $I:=\I(V)$ homogen. Dann ist nach \cref{2.3.4c} auch $\sqrt{I}$ homogen. Sei $f\in\sqrt{I}$ ein homogenes Element. Dann ist $f\in I$ nach dem gleichen Argument, wie im Affinen (\cref{1.1.3c}).
\item[\ref{2.3.5d}] Hier wirken die Argumente aus \cref{kap1}, \cref{1.1.3} und \cref{1.1.5}.
\end{enumerate}\end{proof}

\begin{prop}[Projektiver Nullstellensatz]\label{2.3.6}
Seien $K$ ein algebraisch abgeschlossener Körper und $I$ ein homogenes Ideal in $K\ppolyx$. Dann gilt:
\begin{enumerate}
\item\Label{2.3.6a} $\I(\V(I))=\sqrt{I}$, falls $\sqrt{I}\neq(X_{0},\dotsc,X_{n})$,
\item\Label{2.3.6b} $\V(I)=\leer\iff I=K\ppolyx$ oder $\sqrt{I}=(X_{0},\dotsc,X_{n})$.
\end{enumerate}\end{prop}

Der \hyperref[beweisprojhns]{Beweis} kommt später.

% 6.12.10
\begin{db}\label{2.3.8}\begin{enumerate}
\item\Label{2.3.8a} Die projektiven Varietäten im $\P^{n}(k)$ bilden die abgeschlossenen Mengen einer Topologie auf $\P^{n}(k)$. Diese heißt \emph{Zariski-Topologie\index{Zariski-Topologie}}.
\item Für $M\subseteq\P^{n}(k)$ ist $\V(\I(M))=\Bar{M}$.
\item Sei $V$ eine projektive Varietät. Dann gilt:
\[V\text{ ist irreduzibel}\iff \I(V)\text{ ist ein Primideal}.\]
\item Jede projektive Varietät besitzt eine eindeutige Zerlegung in endlich viele irreduzible Komponenten.
\end{enumerate}\end{db}
\begin{proof} Genau wie im Affinen. Siehe dazu \cref{1.2.1}, \cref{1.2.3}, \cref{1.2.9} und \cref{satz1} aus \cref{kap1}.
\end{proof}

\begin{q} Wie sieht das Bild einer affinen Varietät $V=\V(I)$ in $\P^{n}$ aus?\end{q}

\begin{dl}\label{2.3.9} Seien
\begin{align*}\H\colon k\polyx&\ra k\ppolyx,\\
 f=\sum_{i=0}^{d}f_{i}&\mapsto\sum_{i=0}^{d}f_{i}X_{0}^{d-i}=:F(X_{0},\dotsc,X_{n}),\\
\intertext{wobei die $f_{i}$ homogen mit $\deg f_{i}=i$ sind, die \emph{Homogenisierung\index{Homogenisierung}} und}
\DD\colon k\ppolyx&\ra k\polyx,\\
F(X_{0},\dotsc,X_{n})&\mapsto F(1,X_{1},\dotsc,X_{n})=:f(X_{1},\dotsc,X_{n})\end{align*}
die \emph{Dehomogenisierung\index{Dehomogenisierung}}. Dann gilt:
\begin{enumerate}
\item\Label{2.3.9a} $\DD\circ \H=\id$ und für homogenes $F$ gilt $X_{0}^{e}\cdot \H\circ \DD(F)=F$ für ein $e\in\N$.
\item\Label{2.3.9b} Sei $\DD(F)=f$ oder $F=\H(f)$. Dann ist, nach Definition, $F$ homogen und es gilt $\deg F=d=\deg f$. Außerdem gilt:\vspace*{-6pt}
\[\forall x_{0},\dotsc,x_{n}\in k, x_{0}\neq 0\colon  F(x_{0},\dotsc,x_{n})=x_{0}^{d}\cdot f(\tfrac{x_{1}}{x_{0}},\dotsc,\tfrac{x_{n}}{x_{0}}).\]
Insbesondere ist $F(1,x_{1},\dotsc,x_{n}) = f(x_{1},\dotsc,x_{n})$.
\item\Label{2.3.9c} $\D$ ist ein $k$-Algebrenhomomorphismus und es gilt:
\begin{align*}
\H(f\cdot g)&=\H(f)\cdot \H(g)\\
X_{0}^{j}\cdot\H(f+g)&=\H(f)+X_{0}^{\deg f-\deg g}\cdot \H(g).\end{align*}
Insbesondere liegt $\H(f+g)$ nicht notwendigerweise in dem Ideal $(\H(f),\H(g))$.
\end{enumerate}\end{dl}
%
\begin{proof} Sei $\displaystyle f=\sum_{i=0}^{d}f_{i}$ wie in der Definition.
\begin{enumerate}\vspace*{-6pt}
\item[\ref{2.3.9a}] Es ist $\displaystyle \DD(\H(f))=\DD\biggl(\sum_{i=0}^{d}f_{i}X_{0}^{d-i}\biggr)=\sum_{i=0}^{d}f_{i}=f$. 

Sei nun $F$ homogen vom Grad $d$. Schreibe $F$ als 
\[F=\sum_{i=0}^{d}f_{i}X_{0}^{d-i}.\]\vspace*{-6pt}
Dann hat jedes $f_{i}$ Grad $i$ und sei $\displaystyle e=\deg\sum_{i=0}^{d}f_{i}$. Insbesondere ist $e\leq d$. Damit:
\[\H(\DD(F))=\H\biggl(\sum_{i=0}^{d}f_{i}\biggr)=\H\biggl(\sum_{i=0}^{e}f_{i}\biggr)=\sum_{i=0}^{e}f_{i}X_{0}^{e-i}=\sum_{i=0}^{d}f_{i}X_{0}^{e-i}.\]
Also ist $F=X_{0}^{d-e}\cdot \H(\DD(F))$.
\item[\ref{2.3.9b}] Sei $F$ homogen vom Grad $d$ und $f:=\DD(F)$. Seien $x_{0},\dotsc,x_{n}\in k$ mit $x_{0}\neq 0$. Dann gilt:\vspace*{-6pt}
\[F(x_{0},\dotsc,x_{n})=x_{0}^{d}\cdot F\bigl(1,\tfrac{x_{1}}{x_{0}},\dotsc,\tfrac{x_{n}}{x_{1}}\bigr)=x_{0}^{d}\cdot f\bigl(\tfrac{x_{1}}{x_{0}},\dotsc,\tfrac{x_{n}}{x_{1}}\bigr).\]
\item[\ref{2.3.9c}] $\DD$ ist die Auswertungsabbildung für $X_{0}=1$ und mit Skalarmultiplikation verträglich, damit also ein $k$-Algebrenhomomorphismus.

Seien $\displaystyle f=\sum_{i=0}^{d}f_{i}$ und $\displaystyle g=\sum_{i=0}^{e}g_{i}$, mit $f_{i}$ bzw. $g_{i}$ homogen vom Grad $i$. Dann ist:\vspace*{-12pt}
\begin{align*}
\H(f)\cdot \H(g)&=\biggl(\sum_{i=0}^{d}f_{i}X_{0}^{d-i}\biggr)\biggl(\sum_{i=0}^{e}g_{i}X_{0}^{e-i}\biggr)
%=\sum_{i=0}^{d}\sum_{j=0}^{e}f_{i}g_{j}X_{0}^{d+e-(i+j)}
=\sum_{k=0}^{d+e}\sum_{i=0}^{k}f_{i}g_{k-i}X_{0}^{d+e-k}\\
\H(f\cdot g)&=\H\biggl(\sum_{i=0}^{d}\sum_{j=0}^{e}f_{i}g_{j}\biggr)=\H\biggl(\sum_{k=0}^{d+e}\sum_{j=0}^{k}f_{i}g_{k-i}\biggr)=\sum_{k=0}^{d+e}\sum_{i=0}^{k}f_{i}g_{k-i}X_{0}^{d+e-k}\end{align*}
da $f_{i}g_{k-i}$ Grad $k$ hat. Also ist $\H(fg)=\H(f)\H(g)$.

Sei, ohne Einschränkung, $e\leq d$. Dann gilt für die Summe von $f$ und $g$:\vspace*{-6pt}
\[f+g=\sum_{i=0}^{d}(f_{i}+g_{i})=\sum_{i=0}^{d-j}(f_{i}+g_{i}),\]\vspace*{-6pt}
für ein $j\in\N_{0}$, denn eventuell heben sich Summanden weg. Es gilt also\vspace*{-4pt}
\[\H(f+g)=\sum_{i=0}^{d-j}(f_{i}+g_{i})X_{0}^{d-i-j}\]
\vspace*{-10pt}
und damit\vspace*{-6pt}
\[X_{0}^{j}\H(f+g)
%=\sum_{i=0}^{d}(f_{i}+g_{i})X_{0}^{d-i}
=\H(f)+\sum_{i=0}^{d}g_{i}X_{0}^{d-i}=\H(f)+\sum_{i=0}^{e}g_{i}X_{0}^{d-i}=\H(f)+X_{0}^{d-e}\cdot \H(g),\]%\vspace*{-6pt}
da $g_{i}=0$ für $i>e=\deg g$. Das zeigt die Behauptung.
\end{enumerate}\end{proof}

\begin{ndfn}\hypertarget{2.3.8.5}{
  Für ein Ideal $I\subseteq k\polyx$ bezeichnen wir im Folgenden mit \[I^{*}=(\H(I))=(\H(f)\mid f\in I)\] das von
  den Homogenierungen aller Polynome aus $I$ erzeugte Ideal.}
\end{ndfn}

\begin{lem}\label{2.3.10}
  Seien $I\subseteq k\polyx$ ein Ideal und $I^*$ wie eben, sowie $\phi_{0}\colon\A^{n}(k)\inj\P^{n}(k)$ wie in \cref{phii}.
  Seien $\V(I)\subseteq\A^{n}(K)$ und $\V(I^{*})\subseteq\P^{n}(K)$
  die zugehörigen Varietäten. Dann gilt $\phi_{o}(\V(I))=\U_{0}\cap\V(I^{*})$.
\end{lem}
\begin{proof}
  Sei $x=(x_{1},\dotsc,x_{n})\in\A^{n}(K)$. Dann gilt
  \begin{align*}
    x\in\V(I) &\iff \forall f\in I\colon  f(x)=0 \\ &\iff \forall f\in I\colon  F(1:x_1:\dotsm:x_n)=0 \text{ für } F=\H(f) \\
    &\iff \forall f\in I\colon  F(\phi_{0}(x))=0 \\ &\iff \phi_{0}(x)\in\V(I^{*})\cap \U_{0}.
  \end{align*}
  Daraus folgt $\phi_{0}(\V(I))=\U_{0}\cap\V(I^{*})$.
\end{proof}

\begin{prop}\label{2.3.11prop}
  Sei $\phi_{0}\colon \A^{n}(K)\inj\P^{n}(K)$ wie vorher. Dann ist $\phi_{0}$ ein Homöomorphismus auf $\U_{0}$.
\end{prop}
\begin{proof}
  Wir zeigen, dass Bilder und Urbilder abgeschlossener Mengen abgeschlossen ist.
  Eine abgeschlossene Teilmenge von $\U_{0}$ hat die Form $\V(J)\cap \U_{0}$ mit einem homogenen Ideal $J$. Für eine solche sei
  $I=(\DD(F)\mid F\in J, F\text{ homogen})$ und $I^{*}=(\H(I))$ wie vorher.

  Dann gilt $J\subseteq I^{*}$, denn: Sei $F\in J$ homogen und $f=\DD(F)\in I$ seine Dehomogenisierung. Dann gilt
  $F=X_0^e\H(\DD(f))\in I^*$, also $\V(J)\supseteq\V(I^*)$.

  Wir zeigen $\phi_0^{-1}(\V(J))=\V(I)$, also $\phi_{0}(\V(I))=\V(J)\cap \U_{0}$. \cref{2.3.10} sagt 
  \[\phi_0(\V(I))=\V(I^*)\cap \U_0,\] 
  also wissen wir $\phi_0(\V(I))\subseteq\V(J)\cap \U_0$. Seien umgekehrt 
 \[z=(1:x_1:\dotsm:x_n)\in\V(J)\cap \U_0,\quad x=(x_1,\dotsc,x_n)\]
  und $f\in I$; sei dabei ohne Einschränkung $f$ ein Erzeuger, also $f=\DD(F)$ für ein $F\in J$. Es gilt
  dann $f(x_1,\dotsc,x_n)=F(1:x_1:\dotsm:x_n)=0$, also $x\in\V(I)$. Damit ist $\phi_0$ stetig. Wegen \cref{2.3.10} ist
  \[\phi_0(\V(I))=\V(I^*)\cap \U_0\]
   eine abgeschlossene Teilmenge von $\U_0$, also ist $\phi_0^{-1}$ auch stetig.
\end{proof}

%Lemma 3.7, das ist jetzt verschoben
\begin{lem}\label{2.3.7}
Sei wieder $K$ algebraisch abgeschlossen, $I\subseteq K\polyx$ ein Radikalideal, $\phi_0$ und $I^{*}$ wie \hyperlink{2.3.8.5}{oben}. Dann ist
$\V(I^*)$ die kleinste Varietät in $\P^n(K)$, die $\phi_0(\V(I))$ enthält.
\end{lem}
\begin{proof}
  Wir zeigen: $\Bar{\phi_0(\V(I))}=\V(I^*)$. Aus \cref{2.3.10} folgt $\phi_0(\V(I))\subseteq\V(I^*)$. Sei nun $J$ ein
  homogenes Ideal mit $\V(J)\supseteq\phi_0(\V(I))$. Wir zeigen $J\subseteq I^*$. Sei dazu $F\in J$ homogen und $f=\DD(F)$ seine
  Dehomogenisierung. Dann gilt für alle $x\in\V(I)$: 
  \[F(\phi_0(x))=f(x)=0,\text{ also }f\in\I(\V(I))=\sqrt{I}=I.\] 
  Damit folgt
  $\H(f)\in I^*$, also $F=X_0^e\cdot\H(\DD(F))\in I^*$.
\end{proof}

\begin{kor}\label{2.3.12}
  Sei $K$ algebraisch abgeschlossen, $f\in K\polyx$ und $F=\H(f)$. Dann gilt $\Bar{\phi_0(\V(f))}=\V(F)$.
\end{kor}
\begin{proof}
  Das folgt aus \cref{2.3.7}: sei $I=(f)$, dann ist $I^*=(F)$.
\end{proof}

%8.12.11

\begin{dfn}\label{2.3.11dfn}
  Sei $V\subseteq\P^n(k)$ eine projektive Varietät. Die Menge
  \[ \schlange{V}=\{(x_0,\dotsc,x_n)\in\A^{n+1}(k)\mid(x_0:\dotsm:x_n)\in V\}\cup\{(0,\dotsc,0)\} \subseteq\A^{n+1}(k) \]
  heißt \emph{affiner Kegel zu $V$\index{affiner Kegel}}.
\end{dfn}

\begin{nbsp}
  Für $\V(X^2+Y^2-Z^2)\subseteq\P^2(\R)$ ist das wirklich ein Kegel.
\end{nbsp}

\begin{prop}\label{2.3.12prop}
  Sei $V\subseteq\P^n(k)$ eine projektive Varietät.
  \begin{enumerate}
  \item\Label{2.3.12a} Der affine Kegel $\schlange{V}$ über $V$ ist eine affine Varietät in $\A^{n+1}(k)$. 
  
  Genauer: ist
    $V=\V^{\mathrm{proj}}(\F)$ für eine Menge $\F\subseteq k\ppolyx$, die aus homogenen nichtkonstanten Polynomen besteht, dann
    gilt $\schlange{V}=\V^{\mathrm{aff}}(\F)$ in $\A^{n+1}(k)$.
  \item\Label{2.3.12b} Sei $k$ unendlich und $V$ nichtleer. Dann gilt $\I^{\mathrm{aff}}(V)=\I^{\mathrm{proj}}(V)$.
  \end{enumerate}
\end{prop}
\begin{proof}
  \begin{enumerate}
  \item[\ref{2.3.12a}] Sei $V=\V(\F)$. Falls $\F$ ein vom Nullpolynom verschiedenes konstantes Polynom enthält, ist $V=\leer$
    und damit $\schlange{V}=\{(0,\dotsc,0)\}$, was eine affine Varietät ist. Sonst sei $x=(x_0,\dotsc,x_n)\in\A^{n+1}(K)$. Ist $x$
    der Nullpunkt, so liegt $x$ sowohl in $\schlange{V}$ als auch in $\V^{\mathrm{aff}}(\F)$, da ein nichtkonstantes homogenes
    Polynom immer $(0,\dotsc,0)$ als Nullstelle hat. Ist $x\neq0$, so gilt
    \[ x\in\V^{\mathrm{aff}}(\F) \iff \forall f\in\F\colon f(x_0,\dotsc,x_n)=0 \iff
    (x_0,\dotsc,x_n)\in\V^{\mathrm{proj}}(\F). \]
    Also gilt $\V^{\mathrm{aff}}(\F)=\schlange{V}$.
  \item[\ref{2.3.12b}] Wir zeigen $\I(V)=\I(\schlange{V})$, falls $V\neq\leer$.

    Sei zunächst $f\in k\ppolyx$ ein homogenes Polynom. Es gilt
    \begin{align*}
      f\in\I(V) &\iff \forall (x_0:\dotsm:x_n)\in V\colon f(x_0:\dotsm:x_n)=0 \\ &\iff \forall x\in\schlange{V}\setminus\{0\}\colon
      f(x)=0 \\ &\iff f\in\I(\schlange{V}).
    \end{align*}
    Wir benutzen hierbei, dass für ein $f\in\I(V)$, das homogen und nicht konstant ist, $f(0)=0$ gilt (falls $V\neq\leer$).

    Es bleibt noch zu zeigen, dass $\I(\schlange{V})$ ein homogenes Ideal ist. Dazu zeigen wir: ist $f\in\I(\schlange{V})$ mit
    \[ f=\sum_{i=0}^d f_i, \qquad \text{(}f_i\text{ homogen von Grad }i\text{)} \] dann gilt $f_i\in\I(\schlange{V})$ für alle
    $i\in\{0,\dotsc,d\}$. Da mit einem $x\in\schlange{V}$ auch $\lambda x\in\schlange{V}$ für alle $\lambda\in k$ gilt, haben wir für
    alle $x\in\schlange{V}$: \[ 0=f(x)=f(\lambda x)=\sum_{i=0}^d f_i(\lambda x) = \sum_{i=0}^d \lambda^i f_{i}(x). \] Wenn wir das als
    Polynom in $\lambda$ auffassen, folgt $f_i(x)=0$ für alle $i$, da $k$ unendlich ist. Also ist $f_i\in\I(\schlange{V})$ für alle $i$.
  \end{enumerate}
\end{proof}

Wir können nun den projektiven Nullstellensatz beweisen.
\begin{proof}[Beweis von \cref{2.3.6}]\label{beweisprojhns}
  Sei $K$ ein algebraisch abgeschlossener Körper und $I$ ein homogenes Ideal in $K\ppolyx$.
  \begin{enumerate}
  \item[\ref{2.3.6b}] Sei $I\neq K\ppolyx$ und $V=\V^{\mathrm{proj}}(I)=\V(I^{\mathrm{homogen}})\subseteq\P^n(K)$. Ist dann
    $V=\leer$, gilt, nach \cref{2.3.12b}, $\{(0,\dotsc,0)\}=\schlange{V}=\V^{\mathrm{aff}}(I^{\mathrm{homogen}})=\V^{\mathrm{aff}}(I)$ und damit nach dem
    \hyperref[satz2c]{affinen Hilbertschen Nullstellensatz} $\sqrt{I}=\I(\V^{\mathrm{aff}}(I))=(X_0,\dotsc,X_n)$. Die umgekehrte Implikation ist
    klar.
  \item[\ref{2.3.6a}] Wenn $I=K\ppolyx$, stimmt die Aussage. Sei also $I\neq K\ppolyx$ und $\sqrt{I}\neq(X_0,\dotsc,X_n)$. Nach
    \ref{2.3.6b} gilt dann $\V^{\mathrm{proj}}(I)\neq\leer$, also, wieder nach \cref{2.3.12} und dem \hyperref[satz2c]{affinen Hilbertschen Nullstellensatz}, \[\I(\V(I))=\I(\schlange{V})=\I(\V^{\mathrm{aff}}(I))=\sqrt{I}.\]
  \end{enumerate}
\end{proof}

\begin{dfn}\label{2.3.13}
Sei $V\subseteq\P^{n}(k)$ eine projektive Varietät. Dann heißt
\[k[V]:=\Quotient{k\ppolyx}{\I(V)}\]
\emph{homogener Koordinatenring\index{homogener Koordinatenring}}. $k[V]$ ist eine graduierte $k$-Algebra mit Graduierung
\[k[V]_{d}=\Quotient{k\ppolyx_{d}}{\I(V)\cap k\ppolyx_{d}}.\]
\end{dfn}

\sect{Quasi-projektive Varietäten}

\begin{dfn}\label{2.4.1}
$W\subseteq\P^{n}(k)$ heißt \emph{quasi-projektive Varietät\index{quasi-projektive Varietät}}, wenn $W$ offene Teilmenge einer projektiven Varietät ist.
\end{dfn}

\begin{bem}\label{2.4.2}
Sei $W\subseteq\P^n(k)$ eine quasi-projektive Varietät. Dann gilt:
\begin{enumerate}
\item\Label{2.4.2a} Jede offene Teilmenge $\dach{W}\subseteq W$ ist Vereinigung affiner Varietäten, also
\[\dach{W}=\bigcup_{\lambda\in\Lambda}U_{\lambda}\text{ mit }U_{\lambda}\subseteq \U_{i}=\{(x_{0}:\dotsm:x_{n})\in\P^{n}(k)\mid x_{i}\neq 0\},\]
wobei die $U_{\lambda}$ offen und die $\varphi_{i}^{-1}(U_{\lambda})$ affin als abstrakte Varietät sind.
\item\Label{2.4.2b} $W$ ist quasikompakt. 
\end{enumerate}\end{bem}
\begin{proof}\begin{enumerate}
\item[\ref{2.4.2a}] Wir identifizieren, via $\varphi_{i}$, die $\U_{i}$ mit $\A^{n}(k)$ und schreiben
\[\dach{W}=\bigcup_{i=0}^{n}\dach{W}\cap \U_{i}.\]
Nun gilt die Aussage für die $\dach{W}\cap \U_{i}\subseteq\A^{n}(k)$ nach \cref{1.5.9iii} und \cref{1.5.15i} aus \cref{kap1}, da die $\D(f)$ eine Basis der Topologie bilden und affin als abstrakte Varietäten sind.
\item[\ref{2.4.2b}] Sei $\displaystyle W=\bigcup_{\lambda\in\Lambda}U_{\lambda}$ eine offene Überdeckung. Dann gilt auch\vspace*{-12pt}
\[W=\bigcup_{i=0}^{n}\bigcup_{\lambda\in\Lambda}U_{\lambda}\cap \U_{i}.\]
Nach \cref{1.5.9ii} aus \cref{kap1} genügen aber für jedes $i$ endlich viele offene Mengen, also genügen auch insgesamt endlich viele.
\end{enumerate}\end{proof}

\sect{Reguläre Funktionen}

Seien, wie bisher, $k$ ein Körper, $n\in\N_{0}$, $\U_{i}=\{(x_{0}:\dotsm:x_{n})\in\P^{n}(k)\mid x_{i}\neq 0\}$ und $\varphi_{i}\colon\A^{n}(k)\inj\P^{n}(k)$, $(x_{0},\dotsc,\dach{x_{i}},\dotsc,x_{n})\mapsto(x_{0}:\dotsm:x_{i-1}:1:x_{i+1}:\dotsm:x_{n})$.

\begin{nbem} Seien $F$, $G$ homogene Polynome mit $\deg F = \deg G$. Dann ist $\frac{F}{G}$ wohldefiniert auf $\D(G)$.
\end{nbem}

\begin{dfn}\label{2.5.1}
Seien $W\subseteq\P^{n}(k)$ eine quasi-projektive Varietät und $r\colon W\ra k$ eine Abbildung.\begin{enumerate}
\item Wir nennen $r$ \emph{regulär in $p\in W$\index{regulär}}, wenn es eine offene Umgebung $U_{p}\subseteq W$ von $p$ und homogene Polynome $G$ und $H$ mit $\deg G=\deg H$ und $G(z)\neq 0$ für alle $z\in U_{p}$ gibt, so dass
\[r=\frac{H}{G}\text{ auf }U_{p}.\]
\item Wir nennen $r$ \emph{regulär\index{regulär}}, wenn $r$ regulär in jedem $p\in W$ ist.
\end{enumerate}\end{dfn}

\begin{bem}\label{2.5.2}
Sei $r\colon W\ra k$ eine Abbildung. Dann ist $r$ genau dann regulär, $r\restrict{W\cap \U_{i}}$ regulär für jedes $i$ ist, d.h. $r\circ\varphi_{i}\restrict{\varphi_{i}^{-1}(W)}$ ist reguläre Funktion der quasi-affinen Varietät $\varphi_{i}^{-1}(W)$, definiert wie in \cref{kap1}, \cref{1.5.3a}.
\end{bem}
\begin{proof} Sei zunächst $r$ regulär. Dann gibt es für jedes $p\in\varphi_{i}^{-1}(W)$ eine offenes $U_{\varphi_{i}(p)}$ mit homogenen Polynomen $G$ und $H$, so dass dort $r=\frac{G}{H}$ gilt. Dann gilt
\[r\circ\varphi_{i}=\frac{G\circ\varphi_{i}}{H\circ\varphi_{i}}=\frac{g}{h},\]
wobei $g$ und $h$ die Dehomogenisierungen bzgl. $X_{i}$ von $G$ bzw. $H$ sind, also ist $r\circ\varphi_{i}$ regulär im Affinen.

Sei nun $q\in W$, ohne Einschränkung sei $q$ schon in $\U_{0}$, also gibt es $p\in\varphi_{0}^{-1}(W)$ mit $\varphi_{0}(p)=q$. Nach Vorraussetzung gibt es $g$ und $h$, so dass, auf einer Umgebung $U_{p}$ von $p$, $r\circ\varphi_{0}=\frac{g}{h}$ gilt. Dann ist dort aber schon
\[r=\frac{G}{H}\cdot\frac{X_{0}^{D-\deg G}}{X_{0}^{D-\deg H}},\]
wobei $D:=\max\{\deg G,\deg H\}$ und $G$ und $H$ die Homogenisierungen von $g$ bzw. $h$ sind. Damit ist $r$ regulär.
\end{proof}

\begin{db}\label{2.5.3}
Sei $W\subseteq\P^{n}(k)$ eine quasi-projektive Varietät. Für eine offene Teilmenge $U\subseteq W$ definieren wir 
\[\O_{W}(U):=\{r\colon U\ra k\mid r\text{ ist regulär}\}.\]
\begin{enumerate}
\item $\O_{W}(U)$ ist eine $k$-Algebra.
\item $U\mapsto\O_{W}(U)$ ist eine Garbe von $k$-Algebren (dabei: $\leer\mapsto\O_{W}(\leer)=\{0\}$).
\end{enumerate}\end{db}

%13.12.10
\begin{w}
{\bf Ab jetzt:} Sei $K$ ein algebraisch abgeschlossener Körper.
\end{w}

\begin{satz}\label{satz5}
Sei $V\subseteq\P^{n}(K)$ eine projektive Varietät. Dann gilt:
\begin{enumerate}
\item\Label{s5a} $V$ zusammenhängend $\implies\O_{V}(V)=K$.
\item\Label{s5b} Es sei $K[V]=\Quotient{K\ppolyx}{\I(V)}$ der homogene Koordinatenring und $F\in K[V]$ homogen mit $\deg F\geq 1$. Dann gilt:
\[\O(\D(F))\cong(K[V]_{F})_{0}:=\{\frac{G}{F^{k}}\mid G\in K[V]\text{ homogen}, \deg G = k\cdot\deg F\}.\]
Wir nennen das \emph{homogene Lokalisierung\index{homogene Lokalisierung}}.
\end{enumerate}\end{satz}
\begin{proof}\begin{enumerate}
\item[\ref{s5b}] Wir definieren eine Abbildung
\[\Psi\colon(K[V]_{F})_{0}\ra\O(\D(f)),\quad\frac{G}{F^{k}}\mapsto\left(z\mapsto\frac{G(z)}{F^{k}(z)}\right).\]
$\Psi$ ist wohldefiniert und injektiv, denn
\[\frac{G_{1}(z)}{F^{n}(z)}=\frac{G_{2}(z)}{F^{m}(z)}\;\forall z\in\D(F)\implies G_{1}\cdot F^{m}=G_{2}\cdot F^{n}\text{ auf }\D(F).\]
Dann ist aber schon $(G_{1}\cdot F^{m}-G_{2}\cdot F^{n})\cdot F=0$ auf ganz $V$, also $\frac{G_{1}}{F^{n}}=\frac{G_{2}}{F^{m}}$.

Wir zeigen nun: $\Psi$ ist auch surjektiv.

Sei dazu $r\in\O(\D(F))$. Falls $\U_{i}\cap V\neq\leer$ ist, nach \cref{2.5.2}, $r\circ\varphi_{i}$ auf $\D(F)\cap \U_{i}=\D(f_{i})$ regulär, wobei $f_{i}$ die Dehomogenisierung von $F$ bzgl. $X_{i}$ ist. Jetzt sind wir im Affinen und nach \cref{kap1}, \cref{1.5.11i}, gibt es dort ein $g_{i}\in K[X_{0},\dotsc,\dach{X_{i}},\dotsc,X_{n}]$ und $k_{i}\in\N_{0}$, so dass
\[r\circ\varphi_{i}=\frac{g_{i}}{f_{i}^{k_{i}}}.\]
Diesen Ausdruck können wir wieder homogenisieren und eventuell mit einer $X_{i}$-Potenz multiplizieren und erhalten so
\[r\restrict{\U_{i}}=\frac{G_{i}}{F_{i}^{k_{i}}\cdot X_{i}^{e_{i}}},\text{ mit }G_{i}\in K\ppolyx\text{ und }e_{i}\in\N_{0}.\]

Da $\D(F)=\D(F^{k})$ können wir eventuell zu einer Potenz von $F$ übergehen. Sei also ohne Einschränkung $k_{i}=1$. Also ist
\[r=\frac{G_{i}}{X_{i}^{e_{i}}\cdot F}\text{ auf }\D(F)\cap \U_{i}.\]
Insbesondere gilt $\frac{G_{i}}{X_{i}^{e_{i}}F}=\frac{G_{j}}{X_{j}^{e_{j}}F}$ auf $\D(F)\cap\D(X_{i})\cap\D(X_{j})$ und damit bereits
\begin{equation}\label{s5s}G_{i}\cdot X_{j}^{e_{j}}\cdot F\cdot X_{j}\cdot X_{i}=G_{j}\cdot X_{i}^{e_{i}}\cdot F\cdot X_{j}\cdot X_{i}\text{ auf ganz }V,\tag{$*$}\end{equation}
denn außerhalb von $\D(X_{i})\cap\D(X_{j})\cap\D(F)$ ist der Ausdruck konstant $0$.

Da $F$ homogen mit $\deg F\geq 1$ ist, ist $F\in(X_{0},\dotsc,X_{n})$. Wir finden sogar ein $m\in\N$, so dass $F^{m}\in(X_{0}^{e_{0}+1},\dotsc,X_{n}^{e_{n}+1})$, denn es gilt $\deg F^{m}=m\cdot\deg F$, wir können also
\[F^{m}=\sum_{i}a_{\alpha^{(i)}}X_{0}^{\alpha_{0}^{(i)}}\dotsm X_{n}^{\alpha_{n}^{(i)}}\text{ mit }\alpha_{0}^{(i)}+\dotsm+\alpha_{n}^{(i)}=m\cdot\deg F\]
schreiben und dabei $m$ so groß wählen, dass
\[m\cdot\deg F\geq\sum_{i=0}^{n}(e_{i}+1).\]
Es gibt also ein $j$ mit $\alpha_{j}^{(i)}\geq e_{j+1}$ und damit wird $F^{m}$ von $X_{j}^{e_{j}+1}$ geteilt und liegt, wie behauptet, in dem Ideal.

Damit liegt $F^{m+1}$ in $(F\cdot X_{0}^{e_{0}+1},\dotsc,F\cdot X_{n}^{e_{n}+1})$, also finden wir $h_{i}\in K\ppolyx$, so dass
\[F^{m+1}=\sum_{i=0}^{n}h_{i}\cdot F\cdot X_{i}^{e_{i}+1}\]
gilt. Wir setzen $\displaystyle G:=\sum_{i=0}^{n}h_{i}G_{i}X_{i}$ und mit Hilfe von \cref{s5s} lässt sich
\[X_{j}\cdot F^{m+1}\cdot G_{j}=\sum_{i=0}^{n}X_{j}h_{i}FX_{i}^{e_{i}+1}G_{j}=\sum_{i=0}^{n}X_{i}h_{i}FX_{j}^{e_{j}+1}G_{i}=F\cdot G\cdot X_{j}^{e_{j}+1}\]
einsehen. Somit gilt, auf $\D(F)\cap \U_{j}$, gerade \[\frac{G}{F^{m+1}}=\frac{G_{j}}{X_{j}^{e_{j}}\cdot F}=r\restrict{\U_{j}}.\] Also ist, nach \cref{2.5.2}, $\Psi\bigl(\frac{G}{F^{m+1}}\bigr)=r$, damit ist $\Psi$ surjektiv und die Isomorphie ist gezeigt.

\item[\ref{s5a}] Ohne Einschränkung genügt es die Aussage nur für irreduzible $V$ zu zeigen, denn: Sei
\[V=\bigcup_{i=1}^{r}V_{i}\]
die Zerlegung in irreduzible Komponenten. Wenn $f\in\O_{V}$ auf jedem $V_{i}$ konstant ist, so ist es, da $V$ als zusammenhängend vorausgesetzt war, schon auf ganz $V$ konstant.

Sei also $V$ irreduzibel. Dann ist $\I(V)$ ein Primideal und $K[V]$ nullteilerfrei und demnach $L:=\Quot(K[V])$ ein Körper.

Sei $r\in\O_{V}$. Definiere
\[f_{i}:=r\restrict{\U_{i}}\in\O_{V}(\U_{i})=(K[V]_{X_{i}})_{0}.\]
Also ist $f_{i}=\frac{G_{i}}{X_{i}^{d_{i}}}$ mit $G_{i}$ homogen vom Grad $d_{i}$. Wir können nun $f_{i}$ als Element von $L$ auffassen.

\hyperlink{s5bzz}{{\scshape Zeige:}} $f_{i}$ ist ganz über $K[V]$, d.h. $f_{i}^{m}+a_{m-1}f_{i}^{m-1}+\dotsm+a_{0}=0$ mit $a_{j}\in K[V]$.

Dann können wir das nämlich mit $X_{i}^{d_{i}m}$ multiplizieren und erhalten
\[G_{i}^{m}+\sum_{j=0}^{m-1}a_{j}G_{i}^{j}X_{i}^{d_{i}(m-j)}=0.\]
Dabei hat $G_{i}^{m}$ Grad $d_{i}\cdot m$ und auch $G_{i}^{j}X_{i}^{d_{i}(m-j)}$ Grad $d_{i}\cdot j+d_{i}(m-j)=d_{i}\cdot m$.
\begin{w}Wieso OE?\end{w}
Ohne Einschränkung haben also alle $a_{i}$ Grad $0$ und sind damit in $K$. Damit ist $f_{i}$ ganz über $K$ und da $K$ algebraisch abgeschlossen ist, liegt es somit schon in $K$, ist also konstant.

Damit ist auch $r\restrict{\U_{i}}$ konstant für jedes $i$ und somit auch $r$.

\hypertarget{s5bzz}{Es bleibt zu zeigen: $f_{i}$ ist ganz über $K[V]$.}

Zunächst sieht man, dass die $\frac{G_{i}}{X_{i}^{d_{i}}}$ alle über $L$ das selbe Element definieren, denn
\[\frac{G_{i}}{X_{i}^{d_{i}}}=\frac{G_{j}}{X_{j}^{d_{j}}}\iff X_{j}^{d_{j}}G_{i}-G_{j}X_{i}^{d_{i}}=0\]
auf $\D(X_{i})\cap\D(X_{j})$ und das liegt dicht in $V$, denn $V$ ist irreduzibel. Damit sind die Ausdrücke nach \cref{kap1}, \cref{1.5.13b}, schon auf ganz $V$ gleich.

Wir setzen also $f:=f_{i}$ in $L$ und zeigen, dass $f$ ganz über $K[V]$ ist.

Sei dazu $d:=d_{0}+\dotsm+d_{n}$, wobei $d_{i}:=\deg G_{i}$. Wir zeigen noch:
\begin{enumerate}
\item\label{s5bi} $K[V]_{d}\cdot f^{t}\subseteq K[V]_{d}\;\forall t\in\N$, denn sind $\alpha_{i}\in\N$ mit $\alpha_{0}+\dotsm+\alpha_{n}=d$, so ist
\[X_{0}^{\alpha_{0}}\dotsm X_{n}^{\alpha_{n}}\cdot f\in K[V]_{d}\cdot f\]
und sogar in $K[V]$, denn $f=\frac{G_{i}}{X_{i}^{d_{i}}}$ und es gibt sicherlich ein $i$ mit $\alpha_{i}\geq d_{i}$. Damit sehen wir
\[X_{i}^{\alpha_{i}}\cdot f=X_{i}^{\alpha_{i}}\cdot\frac{g_{i}}{X_{i}^{d_{i}}}=X_{i}^{\alpha_{i}-d_{i}}\cdot g_{i}\]
und $X_{i}^{\alpha_{i}-d_{i}}g_{i}$ hat gerade Grad $\alpha_{i}$, insgesamt hat der Ausdruck also wieder Grad $d$, wir erhalten also
\[K[V]_{d}\cdot f\subseteq K[V]_{d}\]
und iterativ $K[V]_{d}f^{t}\subseteq K[V]_{d}$.
\item\label{s5bii} $K[V][f]\subseteq\frac{1}{X_{0}^{d}}\cdot K[V]$, denn aus \ref{s5bi} folgt, dass insbesondere $X_{0}^{d}f^{t}\in K[V]$, also für alle $t\in\N$
\[f^{t}\in\frac{1}{X_{0}^{d}}\cdot K[V].\]
\item\label{s5biii} Daraus folgt: $f$ ist ganz über $K[V]$, denn $\frac{1}{X_{0}^{d}}K[V]$ ist ein endlich erzeugter $K[V]$-Modul und aus Algebra {\scshape ii} wissen wir, dass dann auch $K[V][f]$ als Untermodul endlich erzeugt ist und damit $f$ ganz über $K[V]$ ist.
\end{enumerate}
\end{enumerate}\end{proof}


\sect{Morphismen}

\begin{db}\label{2.6.1}
Seien $W\subseteq \P^n(k)$, $W'\subseteq \P^m(k)$ quasi-projektive Varietäten.
  \begin{enumerate}
  \item\Label{2.6.1a} Eine Abbildung $f\colon W\ra W'$ heißt \emph{Morphismus\index{Morphismus}}, wenn es ein $z \in W$, eine offene Umgebung $U_z \subseteq W$ von $z$ und Polynome $F_0,\dotsc,F_n \in k\ppolyx$ vom gleichen Grad gibt, so dass für alle $y \in U_z$ gilt:
\[f(y)=(F_0(y),\dotsc,F_n(y))\]
  \item\Label{2.6.1b} Betrachte $f \colon \P^n(k)\supseteq W \ra W' \subseteq \P^m(k)$. Das liefert eine Abbildung 
  \[\A^n(k)\supseteq f^{-1}(W'\cap \U_j) \cap \U_i \ra W' \cap \U_j\subseteq\A^{m}(k).\] 

Jetzt gilt:
$f$ ist genau dann ein Morphismus, wenn für alle $i,j$ mit 
\[U_{ij}:=f^{-1}(W'\cap \U_j) \cap \U_i \neq \leer\]
 die Abbildung $f\restrict{U_{ij}} \colon U_{ij} \ra W'\cap \U_j$ ein Morphismus von quasi-affinen Varietäten ist (siehe \cref{1.5.14} in \cref{kap1}).
  \item\Label{2.6.1c} Morphismen $W\ra \A^1(k)$ entsprechen bijektiv den regulären Funktionen.
  \item\Label{2.6.1d} Morphismen sind stetig.
  \item\Label{2.6.1e} Die quasi-projektiven Varietäten bilden mit den Morphismen eine Kategorie. Diese heißt $\Var_k$.
  \end{enumerate}
\end{db}

\begin{proof}
  \begin{enumerate}
  \item[\ref{2.6.1b}] Sei zunächst $f$ ein Morphismus. Dann kann man $f$ lokal schreiben als (ohne Einschränkung: $i=0$)
\begin{align*}f(x)&=(F_0(1:x_1:\dotsm:x_n):\dotsm:F_m(1:x_1:\dotsm:x_n))\\
&=(f_0(x):\dotsm :f_m(x))=\left(\frac {f_0(x)}{f_j(x)},\dotsc,\frac{f_{j-1}(x)}{f_j(x)},\frac{f_{j+1}(x)}{f_j(x)},\dotsc,\frac {f_m(x)}{f_j(x)}\right)\end{align*}

wobei $f_j$ die Dehomogenisierung von $F_j$ nach $X_0$ bezeichnet.

Für die Rückrichtung weiß man, dass, nach \cref{1.5.14iv}, $f$ auf $U_{ij}$ gegeben ist durch 
\[f(y)=(r_1(y),\dotsc,r_m(y))\]
wobei $r_i=\frac{f_i}{g_i}$. Wir setzen $z=(y_1:\dotsm:y_{i-1}:1:y_i:\dotsm :y_n)$. Damit:
\begin{align*} f(z)&=\left(1:\frac{f_1(y)}{g_1(y)}:\dotsm :\frac{f_m(y)}{g_m(y)}\right)\\&=(G_1(z)\cdot\dotsm\cdot G_m(z)X_0^{e_0}:F_1(z)X_0^{e_1}: \dotsm :F_m(z)X_0^{e_m})\end{align*}
wobei $F_i$ und $G_i$ die Homogenisierungen von $f_i$ bzw. $g_i$ bzgl. $X_0$ sind und die $e_i$ so gewählt seien, dass am Ende alle Polynome denselben Grad haben.  
  \item[\ref{2.6.1c}] Zu zeigen:
\[\{f\colon W\ra \U_0\subseteq \P^1(k) \mid f \text{ ist Morphismus}\} \lra \O(W).\]
Wir identifizieren $\U_0$ mit $k$ durch die Bijektion $\Psi \colon (x_0,x_1)\mapsto \frac{x_1}{x_0} \in k$. Als Zuordnung findet sich dann
\[f\mapsto r:=\Psi \circ f, \quad r\mapsto f=\Psi^{-1}\circ r\]
Außerdem sehen wir: Ist $f$ ein Morphismus, so gilt lokal: $f(z)=(F(z):G(z))$. Also gilt lokal: \[r(z)=\frac{G(z)}{F(z)},\] also ist $r$ regulär.

Ist umgekehrt $r$ regulär, so gilt lokal: $r(z)=\frac{G(z)}{F(z)}$. Damit haben wir 
\[f(z)=\Psi^{-1}(r(z))=(1,\frac{G(z)}{F(z)})=(F(z):G(z))\]
und $f$ ist ein Morphismus.

  \item[\ref{2.6.1d}] {\scshape Zeige:} Jeder Morphismus $f\colon W\ra W'$ ist stetig.

 {\scshape Idee:} Führe Stetigkeit zurück auf die affine Situation.
Wir wissen, es gilt:
\[\bigcup_{i=0}^{n}\U_i=\P^n,\ \bigcup_{i=0}^{m}\U_j=\P^m\text{ und } f\colon \P^n \supseteq W\ra W'\subseteq \P^m.\]
Betrachte also
\[ f_{ij}=f\restrict{U_{ij}} \colon U_{ij}=f^{-1}(W'\cap \U_j)\cap\U_{i}\ra W'\cap \U_j=:W_j'\subseteq \U_j\cong \A^m\]
Nach \ref{2.6.1b} sind die $f_{ij}$ Morphismen im affinen Sinn und damit insbesondere stetig (siehe \cref{2.3.11prop} und \cref{kap1}, \cref{1.4.4}).

Wir zeigen nun, dass $U_{ij}$ offen in $W$ ist: $f$ ist lokal (auf einer offenen Umgebung $U_z$ von $z$) gegeben durch 
\[f(w)=(F_0(w):\dotsm :F_m(w)).\]
Also ist 
\[U_{ij}\cap U_z=\{w \in U_z\cap \U_i \mid f(w) \in \U_j\}=\{w \in U_z\cap \U_i \mid F_j(w)\neq 0\}\]
offen. Damit ist aber auch $U_{ij}$ offen.

Insgesamt sehen wir: $f$ ist auf jedem $U_{ij}$ stetig, und deshalb auch insgesamt stetig.
  \item[\ref{2.6.1e}] ist klar.
  \end{enumerate}  
\end{proof}

\settowidth\breite{$f\text{ ist stetig und für offenes }U\subseteq W', g\in \O(U)$nnnn}
\begin{kor}\label{2.6.2}
  \begin{enumerate}
  \item\Label{2.6.2a} Für eine Abbildung $f\colon W\ra W'$ zwischen quasi-projektiven Varietäten gilt:
%
\[f\text{ ist ein Morphismus}\iff\! %
\begin{minipage}[c]{\breite}\begin{center}
$f$ ist stetig und für offenes $U\subseteq W', g\in \O(U)$ gilt:\\
$g\circ f \in \O(f^{-1}(U)).$%
\end{center}\end{minipage}
\]
  \item\Label{2.6.2b} Für affine Varietäten stimmen die Definitonen von Morphismen überein, d.h. 
  \begin{align*}W_1\subseteq \A^n(k)& \inj \U_0\subseteq \P^n(k)\\
  W_2\subseteq \A^m(k)& \inj \U_0\subseteq \P^m(k)\end{align*}
und $f$ ist genau dann ein Morphismus im Sinn von \cref{2.6.1}, wenn $f$ auch ein Morphismus im Sinn von \cref{1.4.1} aus \cref{kap1} ist.
  \end{enumerate}
\end{kor}

\begin{proof}
\begin{enumerate}
  \item[\ref{2.6.2a}] Die Richtung von links nach rechts folgt aus \cref{2.6.1d} (Stetigkeit), \cref{2.6.1c} (reguläre Abbildungen entsprechen Morphismen nach $\A^1$) sowie \cref{2.6.1e} (Verkettung von Morphismen funktioniert).

Zur Rückrichtung: Nach den Voraussetzungen ist $f_{ij}$ stetig und zieht reguläre Funktionen zu regulären Funktionen zurück.
Nach \cref{1.5.12} aus \cref{kap1} ist $f_{ij}$ ein Morphismus und nach \cref{2.6.1b} ist damit auch $f$ ein Morphismus.
  \item[\ref{2.6.2b}] Aus \ref{2.6.2a} folgt: Ist $f$ ein Morphismus in der projektiven Welt, so ist $f$ stetig und respektiert die Strukturgarbe. Wieder nach \cref{1.5.12} aus \cref{kap1} ist $f$ ein Morphismus im affinen Sinn.
\end{enumerate}
\end{proof}

\begin{bsp}\label{2.6.3}
\begin{enumerate}
  \item\Label{2.6.3a} Sei $k$ ein unendlicher Körper und der Morphismus $f$ wie folgt gegeben:
\[f\colon \P^2\setminus \{(0:0:1)\}\ra \P^1, \quad (x_0:x_1:x_2)\mapsto (x_0:x_1)\]
Dann lässt sich $f$ in $(0:0:1)$ nicht stetig fortsetzen.

Denn: Wir nehmen an es gibt  $(r:s)\neq (0:0)$ mit $f(0:0:1)=(r:s)$. Dann müsste für $(\schlange{r}:\schlange{s})\neq (r:s)$ die Menge $f^{-1}(\{(\schlange{r}:\schlange{s})\})$ abgeschlossen in $\P^2$ sein.

Aber: $f^{-1}(\{(\schlange{r}:\schlange{s})\})=\{(\lambda \schlange{r}:\lambda \schlange{s}:1) \mid \lambda \in k\setminus \{0\}\}\cup\{(\schlange{r}:\schlange{s}:0)\}$. Also:
\[f^{-1}(\{(\schlange{r}:\schlange{s})\})\cap\U_2 \lra \{(\lambda \schlange{r},\lambda \schlange{s}) \mid \lambda \in k\setminus \{0\}\} \subsetneqq \{(\lambda \schlange{r},\lambda \schlange{s}) \mid \lambda \in k\} \cong \A^1\]
Folglich enthält $\{(\lambda \schlange{r}, \lambda \schlange{s}) \mid \lambda \in k\setminus \{0\}\}$ unendlich viele Elemente und ist nicht isomorph zum Bild von $\A^1$ und damit nicht abgeschlossen in $\A^2$. Also kann $f$ nicht stetig auf ganz $\P^2$ sein.
  \item\Label{2.6.3b} Sei $E:=\V(X_0X_2^2-X_1^3+X_1X_0^{2}) \subseteq \P^2(k)$. Also gilt: 
  \[E\cap \U_0=\{(1:x_1:x_2) \in \U_0 \mid x_2^2-x_1^3+x_1=0\} \lra \{(x,y)\in \A^2 \mid y^2=x^3-x\}\]
Weiter ist $E\setminus (E\cap \U_0)=\{(0:0:1)\}$. Nun lässt sich 
\[f \colon E\cap\U_0\ra \P^1, \quad (x_0:x_1:x_2)\mapsto (x_0:x_1)\] auf $E$ in $(0:0:1)$ fortsetzen durch 

\[f(x_0:x_1:x_2)=\begin{cases} (x_0:x_1),&\text{falls } (x_0:x_1:x_2)\neq (0:0:1),\\ (x_1^2:x_2^2+x_1x_0),&\text{falls } (x_0:x_1:x_2)\neq (1:0:0). \end{cases}\] 
Das ist wohldefiniert, denn für $(x_0:x_1:x_2) \notin \{(0:0:1),(1:0:0)\}$ gilt:
\[(x_0:x_1)=(x_0(x_2^2+x_1x_0):x_1(x_2^2+x_1x_0))=(x_1^3:x_1x_2^2+x_0x_1^2)=(x_1^2:x_2^2+x_0x_1)\]
Dabei wurde benutzt, dass der Punkt auf $E$ liegt und $x_2^2+x_1x_0 \neq 0$ sowie $x_1 \neq 0$.

Jetzt erhält man einen Morphismus $E\ra \P^1$ \enquote{vom Grad $2$}.
  \item\Label{2.6.3c} Betrachte
\[f\colon \P^1\ra \V(X_0X_2-X_1^2)\subseteq \P^2, \quad (x_0:x_1)\mapsto (x_0^2:x_0x_1:x_1^2)\]
$f$ ist ein Isomorphismus mit Umkehrabbildung
\[g\colon\V(X_0X_2-X_1^2)\ra \P^1, \quad (x_0:x_1:x_2)\mapsto \begin{cases} (x_0:x_1), &x_0 \neq 0,\\(x_1:x_2), &x_1 \neq 0, \end{cases} \]
Die Koordinatenringe dazu sind: $K[\P^1]=K[X_0,X_1]$ sowie

\[K[W]:=K[\V(X_0X_2-X_1^2)]=\Quotient{K[X_0,X_1,X_2]}{(X_0X_2-X_1^2)}.\]
Wir sehen: $K[W]$ ist nicht faktoriell, denn $\Bar{X_1}^2=\Bar{X_0}\Bar{X_2}$.

Insbesondere sind $K[\P^1]$ und $K[W]$ nicht isomorph.

Folglich sind auch die affinen Kegel nicht isomorph, d.h $\A^2$ und \[\V(X_0X_2-X_1^2)\subseteq \A^3\] sind nicht isomorph.
%%Bild Höhenlinien
  \item\Label{2.6.3d} Sei $A=\begin{pmatrix} a&b\\c&d \end{pmatrix}$ mit det($A)\neq 0$. Dann ist die Abbildung 
\[\P^1\ra \P^1, \quad (x_0:x_1)\mapsto (cx_1+dx_0:ax_1+bx_0)\]
ein Automorphismus des $\P^1$.
\end{enumerate}
\end{bsp}

\begin{dfn}\label{2.6.4}
Eine quasi-projektive Varietät $W\subseteq \P^n$ heißt \emph{affin\index{affin}}, wenn $W$ isomorph zu einer affinen Varietät in einem $\A^m \isom \U_0\subseteq \P^m$ ist.
\end{dfn}

\begin{db}\label{2.6.5}
 Sei $K$ algebraisch abgeschlossen und $W\subseteq \P^n(K)$ eine quasi-projektive Varietät.
\begin{enumerate}
  \item\Label{2.6.5a} Eine \emph{rationale Funktion\index{rationale Funktion}} auf $W$ ist eine Äquivalenzklasse von Paaren $(U,f)$, wobei $U$ offen und dicht in $W$ ist, $f\in \O(U)$ und
  \[(U,f)\sim (U',f') \iff f\restrict{U\cap U'}=f'\restrict{U\cap U'}\]
  \item\Label{2.6.5b} Ist $W$ irreduzibel, so ist 
  \[K(W):=\Rat(W):=\{f\colon W \ppf K \mid f \text{ ist rationale Funktion}\}\]
  ein Körper. Dieser heißt \emph{Funktionenkörper\index{Funktionenkörper}}. 
  \item\Label{2.6.5c} Ist $W$ irreduzibel, so gilt für jede dichte, offene, affine Teilmenge $U$ von $W$:
\[K(W)\cong \Quot(\O(U))\]
  \item\Label{2.6.5d} Ist $W=V$ eine irreduzible projektive Varietät, so gilt:
\[K(V)\cong \Quot^{\mathrm{homogen}}(K[V]):=(K[V]_s)_0\] 
mit $S:=\{f\in K[V] \mid f \text{ ist homogen}, f\neq 0 \}$. $K(V)$ ist der homogene Quotientenkörper, also:
\[(K[V]_s)_0=\Bigl\{\frac{f}{g} \in \Quot(K[V]) \mid f,g \text{ homogen}, \deg f=\deg g\Bigr\}\]
\end{enumerate}
\end{db}

\begin{proof}
\begin{enumerate} 
  \item[\ref{2.6.5c}]  Sei $U\subseteq W$ wie in der Behauptung gegeben. Betrachte
  \[\Rat(W)\ra \Rat(U), \quad [(\schlange{U},f)]\mapsto\relax [(U\cap \schlange{U},f\restrict{U\cap \schlange{U}})].\]
Diese Abbildung ist surjektiv, da $U$ dicht in $W$ ist. Außerdem ist sie nach Definiton der Äquivalenzrelation auch injektiv. Jetzt folgt aus \cref{1.6.1} in \cref{kap1}:
\[\Rat(W)\cong \Rat(U)=\Quot(\O(U)).\]
  \item[\ref{2.6.5b}] Dass $K(W)$ ein Körper ist, folgt aus \ref{2.6.5c}.
  \item[\ref{2.6.5d}] Die Abbildung 
\[\Quot^{\mathrm{homogen}}(K[V])\ra \Rat(V), \quad \frac{f}{g} \mapsto \Bigl[\D(g),x\mapsto \frac{f(x)}{g(x)}\Bigr]\]
ist ein Isomorphismus.
\end{enumerate}
\end{proof}
% 22.12.10
\settowidth\breite{irreduzible quasi-projektive}
\newlength\zweitebreite
\settowidth\zweitebreite{mit $K$-Algebrenhomomorphismens}
\begin{db}\label{2.6.6} Seien $W_1$, $W_2$ quasi-projektive Varietäten.
  \begin{enumerate}
  \item\Label{2.6.6a} Eine \emph{rationale Abbildung\index{rationale Abbildung}} $f\colon W_1\ppf W_2$ ist eine Äquivalenzklasse von Paaren $(U,f_U)$, wobei $U\subseteq W_1$ offen und dicht, $f_U\colon U\ra W_2$ ein Morphismus ist und
   \[(U,f_{\phantom{\schlange{U}}\hspace*{-.6em}U})\sim (\schlange{U},f_{\schlange{U}}) \iff f_{\phantom{\schlange{U}}\hspace*{-.6em}U}\restrict{U\cap \schlange{U}}=f_{\schlange{U}}\restrict{U\cap \schlange{U}}\]
  \item\Label{2.6.6b} $f$ heißt \emph{dominant\index{dominant}}, wenn das Bild in $W_2$ für einen Repräsentanten $(U,f_U)$ (und damit für alle) dicht ist. 
  \item\Label{2.6.6c} Die Zuordnung $W\mapsto K(W)=\Rat(W)$  definiert kontravariante Äquivalenz von Kategorien.
    \[\left\{%
    \begin{minipage}[c]{\breite}\begin{center}%
      irreduzible quasi-projektive\par Varietäten mit dominanten\par rationalen Abbildungen%
    \end{center}\end{minipage}%
    \right\} \lra \left\{%
    \begin{minipage}[c]{\zweitebreite}\begin{center}%
      endlich erzeugte\par Körpererweiterungen $\Quotient{L}{K}$\par mit $K$-Algebrenhomomorphismen
    \end{center}\end{minipage}%
    \right\}\] 
  \end{enumerate}
\end{db}

\begin{proof}
\begin{enumerate}
  \item[\ref{2.6.6c}] folgt aus dem Affinen (vgl. \cref{satz4}) bzw. \cref{2.6.5}.

Denn: Für ein $r\colon W_1\ppf W_2$ erhalten wir durch die Wahl eines Repräsentanten, affines $U_2\subseteq W_2$ und affines $U_1\subseteq r^{-1}(U_2)\subseteq W_1$ und durch Einschränken von $r$ eine Abbildung $U_1\ra U_2$.

Jeder Körper wird bis auf Isomorphie getroffen, denn: Aus der affinen Situation (vgl. \cref{satz4}) wissen wir, dass es für jede endliche Körpererweiterung $\Quotient{L}{K}$ eine affine Varietät $U$ mit $K(U)\cong L$ gibt.

Fasse nun $U$ als quasi-projektive Varietät auf. Der Funktor aus \cref{satz4} induziert ein Isomorphismus auf den Morphismenmengen.
\[\Phi(r\colon W_1\ppf W_2)=\begin{cases} K(W_2)\ra K(W_1)\\g\mapsto g\circ r \end{cases}\] Nach der Überlegung zu Beginn des Beweises entsprechen die rationalen Abbildungen zwischen $W_1$ und $W_2$ bijektiv den rationalen Abbildungen zwischen den entsprechenden 
affinen Varietäten und diese entsprechen nach \cref{satz4} den Morphismen zwischen $K(U_2)$ und $K(U_1)$. Nach \cref{2.6.5} ist $K(U_{2})$ isomorph zu $K(W_2)$ und genauso ist $K(U_1)\cong K(W_1)$. 
\end{enumerate}
\end{proof}


\sect{Graßmann-Varietäten}

Sei $G(d,n):=\{U\subseteq K^{n}\mid U\text{ ist Untervektorraum vom $K^{n}$ von Dimension }d\}$.

\begin{nbsp}\begin{itemize}[leftmargin=*,labelindent=\parindent]
\item $d=1$: $G(1,n)$ entspricht dem $\P^{n}(k)$.
\item $d=n$: $G(n,n)$ ist einelementig.
\end{itemize}\end{nbsp}

\begin{ziel} Wir wollen versuchen $G(d,n)$ mit der Struktur einer projektiven Varietät zu versehen. Dafür brauchen wir eine \enquote{natürliche} Einbettung in einen $\P^{D}(k)$.
\end{ziel}

\begin{db}\label{2.7.1}
Sei $1\leq d\leq n$, $V=K^{n}$, $\bigwedge^{d}V$ die \emph{$d$-te äußere Potenz\index{aussere@äußere Potenz}} und
\[\P(V):=\Quotient{V\setminus\{0\}}{\sim},\text{ wobei }w_{1}\sim w_{2}\; :\Longleftrightarrow\;\exists\lambda\in K^{\times}\text{ mit }w_{2}=\lambda w_{1}.\]
Dann ist die Abbildung
\[\Psi\colon G(d,n)\ra\P(\bigwedge^{d}V),\quad U=\langle u_{1},\dotsc,u_{d}\rangle\mapsto\relax [u_{1}\wedge\dotsm\wedge u_{d}]\]
wohldefiniert und injektiv.
\end{db}

\begin{nerinnerung}\begin{itemize}[leftmargin=*,labelindent=\parindent]
\item $\bigwedge^{d}V$ hat die Basis $\{e_{i_{1}}\wedge\dotsm\wedge e_{i_{d}}\mid 1\leq i_{1}<\dotsm<i_{d}\leq n\}$.
Insbesondere ist $\dim\bigwedge^{d}V=\binom{n}{d}$.
%
\item $\wedge$ ist multilinear und alternierend, insbesondere gilt $v_{1}\wedge\dotsm\wedge v_{d}=0$, wenn $v_{i}=v_{j}$ für $i\neq j$. Genauer:
\[v_{1}\wedge\dotsm\wedge v_{d}=0\iff v_{1},\dotsc, v_{d}\text{ sind linear abhängig.}\]
\end{itemize}\end{nerinnerung}

\begin{proof}[Beweis von \cref{2.7.1}]
Sei $\schlange{v}_{1},\dotsm,\schlange{v}_{d}$ eine andere Basis von $U$. Dann gibt es
\[\schlange{v}_{i}=\sum_{j=1}^{d}a_{ij}v_{j}\]
mit $A:=(a_{ij})\in\GL_{d}(K)$. Damit gilt aber
\begin{align*}
\schlange{v}_{1}\wedge\dotsm\wedge\schlange{v}_{d}&=\sum_{j=1}^{d}a_{1j}v_{j}\wedge\dotsm\wedge\sum_{j=1}^{d}a_{dj}v_{j}=\sum_{\sigma\in S_{d}}a_{1\sigma(1)}v_{\sigma(1)}\wedge\dotsm\wedge a_{d\sigma(d)}v_{\sigma(d)}\\
&=\biggl(\sum_{\sigma\in S_{d}}(-1)^{\sign\sigma}\cdot\prod_{j=1}^{d}a_{j}\sigma(j)\biggr)\cdot v_{1}\wedge\dotsm\wedge v_{d}=\det A\cdot v_{1}\wedge\dotsm\wedge v_{d}
\end{align*}
und da $\det A\neq 0$ ist, sind die beiden Punkte äquivalent, die Abbildung ist also wohldefiniert.

Für die Injektivität überlegen wir uns, dass $U=\{v\in V\mid v\wedge v_{1}\wedge\dotsm\wedge v_{d}=0\}$ ist, wobei $v_{1},\dotsc, v_{d}$ eine Basis von $U$ ist. Das ist so, denn es gilt:
\begin{align*}
v\wedge v_{1}\wedge\dotsm\wedge v_{d}=0&\iff v,v_{1},\dotsc,v_{d}\text{ sind linear abhängig}\\
&\iff v\in\langle v_{1},\dotsc,v_{d}\rangle=U,
\end{align*}
da $v_{1},\dotsc,v_{d}$ als Basis linear unabhängig sind.
\end{proof}

\begin{db}\label{2.7.2}
Sei $[w]\in\P(\bigwedge^{d}(V))$. Dann gilt
\[[w]\in\Bild\Psi\iff\exists\; v_{1},\dotsc,v_{d}\in V\text{ linear unabhängig mit }w=v_{1}\wedge\dotsm\wedge v_{d}.\]
In diesem Fall heißt $w$ \emph{total zerlegbar\index{total zerlegbar}}. Insbesondere gilt, wenn $w\wedge v_{i}=0$ für alle $i\in\{1,\dotsc,d\}$, für die lineare Abbildung
\[V\ra\bigwedge^{d+1}V,\quad \varphi_{w}\colon v\mapsto w\wedge v,\]
dass $\dim(\Kern\varphi_{w})\geq d$.
\end{db}

\begin{lem}\label{2.7.3}
Sei $d\geq 2$ und $w\in\bigwedge^{d}V$. Dann gilt:
\begin{enumerate}
\item\Label{2.7.3a} $v\in\Kern\varphi_{w}\iff\exists\; w'\in\bigwedge^{d-1}V\text{ mit }w=v\wedge w'$,
\item\Label{2.7.3b} Seien $v_{1},\dotsc,v_{k}\in V$ linear unabhängig. Dann gilt:
\[v_{1},\dotsc,v_{k}\in\Kern\varphi_{w}\iff\exists\; w'\in\bigwedge^{d-k}V\text{ mit }w=v_{1}\wedge\dotsm\wedge v_{k}\wedge w'.\]
\item\Label{2.7.3c} $\dim(\Kern\varphi_{w})\leq d$,
\item\Label{2.7.3d} $\dim(\Kern\varphi_{w})=d\iff w$ ist total zerlegbar.
\end{enumerate}\end{lem}

\begin{proof}
Die Aussagen \ref{2.7.3a}, \ref{2.7.3c} und \ref{2.7.3d} folgen sofort aus \ref{2.7.3b}. Es genügt also das zu zeigen. Die eine Implikation ist nach Definition von $\varphi_{w}$ sofort klar. 

Seien also $v_{1},\dotsc,v_{k}\in\Kern\varphi_{w}$. Diese ergänzen wir zu einer Basis $\{e_{1},\dotsc,e_{n}\}$ von $V$ mit 
\[e_{1}=v_{1},\dotsc,e_{k}=v_{k}.\]
Für $w$ finden wir also
\[w=\!\!\!\!\!\!\!\!\!\!\sum_{\substack{1\leq i_{1}<\dotsm<i_{d}\leq n\\\iq=(i_{1},\olddotsc,i_{d})}}\!\!\!\!\!\!\!\!\! \lambda_{\iq}e_{i_{1}}\wedge\dotsm\wedge e_{i_{d}}.\]
Für $j\in\{1,\dotsc,k\}$ gilt dann $e_{j}\in\Kern\varphi_{w}$, also:
\[0=w\wedge e_{j}=\sum_{\iq}\lambda_{\iq}e_{i_{1}}\wedge\dotsm\wedge e_{i_{d}}\wedge e_{j}=\!\!\!\!\sum_{\substack{\iq\\ j\notin\{i_{1},\olddotsc,i_{d}\}}}\!\!\!\!\!\lambda_{\iq}e_{i_{1}}\wedge\dotsm\wedge e_{i_{d}}\wedge e_{j}\]
%und das ist eine Linearkombination von linear unabhängigen Vektoren. Also ist $\lambda_{\iq}=0$, wenn $\iq=(i_{1},\dotsc,i_{d})$ und $j\notin\{i_{1},\dotsc,i_{d}\}$, also $\{1,\dotsc,k\}\setminus\{i_{1},\dotsc,i_{d}\}$. Damit gilt
und das ist eine Linearkombination von linear unabhängigen Vektoren. Also ist $\lambda_{\iq}=0$ für die
$\iq=(i_{1},\dotsc,i_{d})$, für die es ein solches $j$ gibt, also wenn $\{1,\dotsc,k\}\setminus\{i_{1},\dotsc,i_{d}\}\neq\leer$ gilt. Damit $\lambda_{\iq}\neq 0$ gelten kann, muss also $\{1,\dotsc,k\}\subseteq\{i_{1},\dotsc,i_{d}\}$ gelten, also sind nur solche Summanden in unserer Darstellung von $w$ relevant. Damit haben wir
\[w=\sum_{\iq}\lambda_{\iq}e_{1}\wedge\dotsm\wedge e_{k}\wedge e_{i_{k+1}}\wedge\dotsm\wedge e_{i_{d}}=(e_{1}\wedge\dotsm\wedge e_{k})\wedge\biggl(\sum_{\iq}\lambda_{\iq}e_{i_{k+1}}\wedge\dotsm\wedge e_{i_{d}}\biggr)\]
und $\displaystyle\biggl(\sum_{\iq}\lambda_{\iq}e_{i_{k+1}}\wedge\dotsm\wedge e_{i_{d}}\biggr)$ liegt in $\bigwedge^{d-k}V$, ist also eine geeignete Wahl für $w'$.
\end{proof}

\begin{prop}\label{2.7.4}
Das Bild von $\Psi$ ist in $\P(\bigwedge^{d}V)$ abgeschlossen, d.h. $\Bild\Psi$ ist eine projektive Varietät.
\end{prop}

\begin{proof}
Wir wählen eine Basis $\B:=\{e_{1},\dotsc,e_{n}\}$ von $V$. Dann finden wir
\[\S_{d}:=\{e_{i_{1}}\wedge\dotsm\wedge e_{i_{d}}\mid 1\leq i_{1}<\dotsm<i_{d}\leq n\}\]
als zugehörige Basis von $\bigwedge^{d}V$. Sei wieder, für $w\in\bigwedge^{d}V$,
\[\varphi_{w}\colon V\ra\bigwedge^{d+1}V,\quad v\mapsto w\wedge v,\]
und $\L_{w}:=(l_{ij}(w))_{i,j}$ die Abbildungsmatrix von $\varphi_{w}$ bzgl. $\B$ und $\S_{d+1}$. Außerdem sind die $l_{ij}$ linear in $w$.

Aus \cref{2.7.2} und \cref{2.7.3} folgt, dass
\[w\in\Bild\Psi\!\iff\! \dim(\Kern\varphi_{w})\geq d\iff\Rang\varphi_{w}\leq n-d\iff\det(l_{ij}(w))_{I,J}=0,\]
für alle $\card{I}=\card{J}=n-d+1$, also alle $n-d+1$-Minoren der Matrix $0$ sind (d.h. wenn beliebige $n-d+1$ Zeilen bzw. Spalten linear abhängig sind).

Diese sind homogene Polynome in den Koordinaten von $w$ (bzgl. $\S_{d}$) von Grad $n-d+1$. $\Bild\Psi$ ist Nullstelle von diesen und damit projektive Varietät.
\end{proof}

%10.1.11
\chapter{Geometrische Eigenschaften}
\label{kap3}

In diesem Kapitel sei stets $K$ ein algebraisch abgeschlossener Körper.

\sect{Lokale Ringe zu Punkten}

Gegeben die Strukturgarbe zu einer Varietät wollen wir \enquote{Funktionskeime} in einem Punkt betrachten.

\begin{dfn}\label{3.1.1}
  Sei $W$ eine quasi-projektive Varietät und $p\in W$.
  \begin{enumerate}
  \item\Label{3.1.1i} Es sei
    \[ \O_{W,p} = \Quotient{\{ (U,f) \mid p\in U\subseteq W\text{ offen, } f\in\O_{W}(U) \}}{\sim}, \]
    wobei $(U_1,f_1)\sim(U_2,f_2)$ bedeuten soll, dass es eine offene Menge $U_3\subseteq U_1\cap U_2$ mit $p\in U_3$ gibt,
    sodass $f_1\restrict{U_3}=f_2\restrict{U_3}$ gilt. Das ist eine $K$-Algebra und heißt \emph{lokaler Ring von $W$ in $p$\index{lokaler Ring}}.
  \item\Label{3.1.1ii} Die Elemente in $\O_{W,p}$ heißen \emph{Funktionskeime\index{Funktionskeime}}. 
  
  Für die Äquivalenzklasse eines $(U,f)$ schreiben
    wir auch $f_p$.
  \end{enumerate}
\end{dfn}

\begin{bsp}\label{3.1.2}
  Sei $W=\A^1(K)$ und $x=0$. Wir wissen, dass die nichtleeren offenen Teilmengen von $\A^1(K)$ alle von der Form
  $\A^1(K)\setminus\{x_1,\dotsc,x_n\}$ sind und reguläre Funktionen auf einer solchen Menge sind Quotienten von Polynomen
  $\frac{f}{g}$ mit $g(x)\neq0$ für $x\neq x_i$, $i=1,\dotsc,n$. Also ist $\O_{\A^1(K),0}\cong K[X]_{(X)}$.
\end{bsp}

\begin{bem}\label{3.1.3}
  \begin{enumerate}
  \item\Label{3.1.3i} Die Abbildung \[ \phi_{p}\colon \O_{W,p}\ra K,\quad f_p=[(U,f)]\mapsto f_p(p):=f(p), \]
    ist ein wohldefinierter $K$-Algebrenhomomorphismus und heißt \emph{Auswertung\index{Auswertung}}.
  \item\Label{3.1.3ii} $\O_{W,p}$ ist ein lokaler Ring mit maximalem Ideal \[ \m_p = \{ [(U,f)]\in\O_{W,p}\mid f(p)=0\}. \]
  \end{enumerate}
\end{bem}
\begin{proof}
  \begin{enumerate}
  \item[\ref{3.1.3i}] Die Wohldefiniertheit folgt direkt aus der Definition der Äquivalenzrelation.
  \item[\ref{3.1.3ii}] Die Abbildung $\phi_p$ ist surjektiv, denn die konstanten Abbildungen definieren Elemente in $\O_{W,p}$,
    und $\Kern \phi_p=\m_p$. Damit gilt $\Quotient{\O_{W,p}}{\m_p}\cong K$, also ist $\m_p$ ein maximales Ideal. Es bleibt noch zu
    zeigen, dass es kein weiteres maximales Ideal gibt. Sei dazu $I\subseteq\O_{W,p}$ ein Ideal mit $I\not\subseteq \m_p$. Dann
    gibt es ein $[(U,f)]\in I$ mit $[(U,f)]\notin \m_p$, also $f(p)\neq0$. Wir definieren $\schlange{U}=\D(f)\ni p$ und
    $g=\frac1f\in\O_W(\schlange{U})$. Dann gilt $[(U,f)]\cdot[(\schlange{U},g)]=[(\schlange{U},1)]=1$, also ist $[(U,f)]\in I$
    invertierbar und damit $I=\O_{W,p}$.
  \end{enumerate}
\end{proof}

\begin{bem}\label{3.1.4}
  Die Abbildung \[ \psi_p\colon \O_W(U)\ra\O_{W,p},\quad f\mapsto[(U,f)]=f_p, \] heißt \emph{kanonische Keim-Abbildung\index{kanonische Keim-Abbildung}}.

  Sei $W=W_1\cup\dotsm\cup W_k$ die Zerlegung von $W$ in irreduzible Komponenten. Wir wissen: ist $U\subseteq W$ offen und $p\in
  U$, so ist $U\cap W_i$ dicht in $W_i$, falls $p\in W_i$.

  Falls $U\cap W_j=\leer$ für alle $W_j$ mit $p\notin W_j$, dann ist $\psi_p\colon\O_W(U)\ra\O_{W,p}$ injektiv. Denn wenn
  $\psi_p(f)=0$ ist, gibt es ein offenes $\schlange{U}\subseteq U$ mit $f\restrict{\schlange{U}}=0$. Da $\schlange{U}$ dicht in $U$ ist,
  gilt $f\restrict{U}=0$, denn \enquote{$=0$} ist eine abgeschlossene Bedingung.
\end{bem}

\begin{prop}\label{3.1.5}
  Ist $V$ eine affine Varietät, so gilt \[ \O_{V,p}\cong K[V]_{\m_p^V}. \]
  Hierbei ist $\m_p^V = \{ f\in K[V] \mid f(p)=0\}$.
\end{prop}
\begin{proof}
  Wir definieren
  \[ \phi\colon K[V]_{\m_p^V}\ra\O_{V,p},\quad\frac{f}{g}\mapsto\Bigl[\Bigl(\D(g),x\smapsto\frac{f(x)}{g(x)}\Bigr)\Bigr]. \]

  $\phi$ ist wohldefiniert: denn $\phi$ wird induziert von
  \[ \psi_p\colon \O_V(V)=K[V]\ra\O_{V,p},\quad f\mapsto[(V,x\smapsto f(x))], \]
  denn für $g\notin \m_p^V$ gilt $g(p)\neq0$, also ist $\psi_p(g)$ eine Einheit in $\O_{V,p}$.

  $\phi$ ist injektiv: Sei $\phi(\frac{f}{g})=0$, dann gibt es ein $U\subseteq\D(g)$ mit $\frac{f(x)}{g(x)}=0$ für alle $x\in
  U$. Für $h\in K[V]$ mit $p\in\D(h)\subseteq U$ gilt dann $h(p)\neq0$, also $h\in \m_p^V$. Also gilt $h(x)f(x)=0$ für alle
  $x\in V$, also $f=0$ in $K[V]_{\m_p^V}$.

  $\phi$ ist surjektiv: Sei $[(U,f)]\in\O_{V,p}$. Ohne Einschränkung ist \[U=\D(h)\text{ für ein }h\in K[V].\] Es gilt dann
  $f\in\O_V(U)=\O_V(\D(h))=K[V]_h$, also ist $f=\frac{g}{h^k}$ für ein $g\in K[V]$ und $k\in\N_0$. Damit ist
  $[(U,f)]=\phi(\frac{g}{h^k})$.  
\end{proof}

\begin{kor}\label{3.1.6}
  Sei $W$ eine quasi-projektive Varietät und $x\in W$. Sei weiter $V_0\subseteq W$ offen und affin mit $p\in V_0$. Dann gilt
  \[ \O_{W,p} \cong \O_{V_0,p} \cong K[V_0]_{\m_p^V}. \]
\end{kor}

\begin{prop}\label{3.1.7}
  Seien $W_1$, $W_2$ quasi-projektive Varietäten. Seien weiter $p\in W_1$, $q\in W_2$ und $\O_{W_1,p}\cong\O_{W_2,q}$. Dann gibt es offene
  Umgebungen $U_1$ und $U_2$ von $p$ bzw. $q$ mit $U_1\cong U_2$, d.h. \enquote{der lokale Ring kennt die Umgebung eines Punktes}.
\end{prop}
\begin{proof}
  Sei $\phi\colon\O_{W_1,p}\ra\O_{W_2,p}$ ein Isomorphismus.
  \begin{prooflist}
  \item Wir wählen $U_1$ offen und affin in $W_1$ mit $p\in U_1$ und so, dass $U_1$ nur die irreduziblen Komponenten von $W_1$
    schneidet, die $p$ enthalten. Entsprechend wählen wir ein $\schlange{U_2}$ für $q$. Wir haben dann
    \[ \begin{tikzpicture}[bij/.style={below,sloped,inner sep=1pt}]
      \matrix (m) [matrix of math nodes, row sep=3em,
      column sep=3em, text height=1.5ex, text depth=0.25ex]
      { \O(U_1) & \O_{W_1,p} & \O_{W_2,q} & \O(\schlange{U_2}). \\};
      \path[right hook->] (m-1-1) edge node[auto] {$\psi_p$} (m-1-2);
      \path[->] (m-1-2) edge node [bij] {$\sim$} node[above] {$\phi$} (m-1-3);
      \path[left hook->] (m-1-4) edge node[above] {$\psi_q$} (m-1-3);
    \end{tikzpicture} \]
    Wir hätten gerne, dass $\phi\circ\psi_p(\O(U_1))\subseteq\psi_q(\O(\schlange{U_2}))$ gilt (sogar \enquote{$=$}).
  \item Da $U_1$ affin ist, gilt $\O(U_1)=K[U_1]$. Seien $f_1,\dotsc,f_k$ Erzeuger von $K[U_1]$. Wir betrachten
    $\phi((f_1)_p),\dotsc,\phi((f_k)_p)$ in $\O_{W_2,q}\cong K[\schlange{U_2}]_{\m_q^{\schlange{U_2}}}$: es ist
    \[ \phi((f_i)_p) = \frac{g_i}{h_i} \text{ mit } g_i,h_i\in K[U_2] \text{ und } h_i\notin \m_q^{\schlange{U_2}}. \]
    Wir wählen nun eine offene und affine Teilmenge $U_2\subseteq\schlange{U_2}\cap\D(h_1)\cap\dotso\cap\D(h_n)$ mit $q\in U_2$. Es
    gilt dann $\displaystyle\frac{g_i}{h_i}\in\O(U_2)$ und
    $\displaystyle\psi_q\Bigl(\frac{g_i}{h_i}\Bigr)=\phi(\psi_p(f_i))$. Wir haben dann
    \[ \begin{tikzpicture}[bij/.style={below,sloped,inner sep=1pt}]
      \matrix (m) [matrix of math nodes, row sep=3em,
      column sep=3em, text height=1.5ex, text depth=0.25ex]
      { \O(U_1) & \O_{W_1,p} & \O(U_2)\subseteq\O_{W_2,p}. \\};
      \path[right hook->] (m-1-1) edge node[auto] {$\psi_p$} (m-1-2);
      \path[->] (m-1-2) edge node[above] {$\phi$} node [bij] {$\sim$} (m-1-3);
    \end{tikzpicture} \]
    Insbesondere definiert ein injektiver Homomorphismus 
    \[K[U_1]\cong\O(U_1)\inj\O(U_2)\cong K[U_2]\] 
    einen surjektiven Morphismus $U_2\surj U_1$ von affinen Varietäten.
    % 12.1.11
  \item Wir haben jetzt $p\in U_1\subseteq W_1$, $q\in U_2\subseteq W_2$ und ein kommutatives Diagramm
    \[\begin{tikzpicture}[bij/.style={above,sloped,inner sep=0.5pt}]
      \matrix (m) [matrix of math nodes, row sep=3em, column sep=5em, text height=1.5ex, text depth=0.25ex]
      {K[U_1] & K[U_2] \\ \O_{W_1,p} & \O_{W_2,q},\\};
      \path[->]
      (m-1-1) edge node [below] {$\phi$} (m-1-2) 
      (m-2-1) edge node [bij] {$\sim$} node [below] {$\phi$} (m-2-2);
      \path[right hook->]
      (m-1-1) edge node [auto] {$\psi_p$} (m-2-1)
      (m-1-2) edge node [auto] {$\psi_q$} (m-2-2);
    \end{tikzpicture}\]
    sodass $\phi\circ\psi_p(K[U_1])\subseteq K[U_2]$. Dadurch erhalten wir einen Morphismus 
    \[f\colon U_2\ra U_1\text{ mit }f(q)=p,\] $f^\sharp = \phi\colon K[U_1]\ra K[U_2]$ und $f_p^\sharp = \phi\colon\O_{W_1,p}\ra\O_{W_2,p}$. Analog
    erhalten wir einen Morphismus $g\colon \schlange{U_1}\ra\schlange{U_2}$ (es ist ohne Einschränkung $\schlange{U_1}\subseteq U_1$,
    $\schlange{U_2}\subseteq U_2$) mit $g(p)=q$, $g^\sharp=\phi^{-1}\colon K[\schlange{U_2}] \ra K[\schlange{U_1}]$ und
    $g_p^\sharp=\phi^{-1}\colon \O_{W_2,q}\ra\O_{W_1,p}$.

    Wir setzen jetzt $\dach{U_1}=\schlange{U_1}$ und $\dach{U_2}=f^{-1}(\schlange{U_{1}})\subseteq U_2$. Wir haben dann
    Abbildungen \[ f\circ g\colon \dach{U_1}\ra U_1,\qquad g\circ f\colon\dach{U_2}\ra \schlange{U_2}. \]
    Auf den lokalen Ringen induziert $g\circ f$ die Identität, also ist $g\circ f$ die Einbettung $\dach{U_2}\inj
    U_2$. Analog ist $f\circ g$ die Einbettung $\dach{U_1}\inj\schlange{U_1}$. Daraus folgt 
    \[f\circ g(\dach{U_1})\subseteq \dach{U_1},\text{ also }g(\dach{U_1})\subseteq
    f^{-1}(\dach{U_1})=\dach{U_2},\] 
    und analog $f(\dach{U_2})\subseteq\dach{U_1}$. Hier gilt
    dann $f\circ g=\id_{\dach{U_1}}$ und $g\circ f=\id_{\dach{U_2}}$.

    Also sind $f$ und $g$ Isomorphismen.
  \end{prooflist}
\end{proof}

\begin{bem}\label{3.1.8}
  \begin{enumerate}
  \item\Label{3.1.8i} Morphismen von Varietäten induzieren Homomorphismen zwischen den lokalen Ringen: Sei $\phi\colon W_1\ra
    W_2$ ein Morphismus von quasi-projektiven Varietäten und $x\in W_1$. Dann ist
    \[ \phi_x^\sharp\colon \O_{W_2,\phi(x)}\ra\O_{W_1,x},\quad [(U,f)] \mapsto\relax [(\phi^{-1}(U),f\circ \phi)], \]
    ein wohldefinierter $K$-Algebrenhomomorphismus mit $\phi_x^\sharp(\m_{\phi(x)})\subseteq \m_x$.
  \item\Label{3.1.8ii} Es gilt $\id_x^\sharp=\id_{\O_{V,x}}$ und
    $(\phi_2\circ\phi_1)^\sharp_x={\phi_1}^\sharp_x\circ{\phi_2}^\sharp_x$. Wir erhalten also einen Funktor von der Kategorie der
    punktierten quasi-projektiven Varietäten in die Kategorie der lokalen Ringe.
  \end{enumerate}
\end{bem}

\sect{Dimension von Varietäten}\label{3.2}

{\scshape Wünsche:} \begin{itemize}
 \item $\dim \A^n=n$
 \item $\dim \V(XZ,YZ)\subseteq \A^3=2$ (\enquote{Ebene})
 \item $\dim \V(X^2+Y^2-Z^2)\subseteq \A^3=2$ 
 \item $\dim \V(X^2+Y^2-1)\subseteq \A^2=1$ (\enquote{Kreis})
\end{itemize}

\begin{dfn}\label{3.2.1}
Sei $X$ ein topologischer Raum. Dann heißt 
\[\dim(X):=\sup \{d\in \N_0 \mid \exists \text{ Kette } \leer\neq X_0 \subsetneq  \dotsm \subsetneq X_d \text{ mit $X_i$ abg. und irred. in $X$}\}\]
die \emph{Krulldimension\index{Krulldimension}} von $X$. 
\end{dfn}

\settowidth\breite{Insbesondere ist die Krulldimension von $\C^n$ mit euklidischer Topologie $0$.FFGH}
\begin{w}
Für Mannigfaltigkeiten stimmen Krulldimension und Dimension nicht über\-ein, da in Hausdorffräumen Punkte die einzigen irreduziblen, nichtleeren Teil\-mengen sind (vgl. \cref{1.2.7} in \cref{kap1}).

Insbesondere ist die Krulldimension von $\C^n$ mit euklidischer Topologie $0$.
\end{w}
% 
\[\text{Ab jetzt: Mit Dimension ist stets die Krulldimension gemeint!}\]

\begin{bem}\label{3.2.2}
\begin{enumerate}
  \item\Label{3.2.2a} Ist $X$ ein topologischer Raum, $Y\subseteq X$ mit Spurtopologie versehen, so ist $\dim(Y) \le \dim(X)$.
  \item\Label{3.2.2b} Ist $X=\bigcup\limits_{i=1}^{n} X_i$ mit $X_i$ abgeschlossene Teilmengen von $X$, so gilt: 
  \[\dim(X)=\sup_{i} \dim(X_i).\]
%Verbesserung: i
\end{enumerate}
\end{bem}

\begin{proof}
  \begin{enumerate}
  \item[\ref{3.2.2a}] Sei $\leer \neq y_0 \subsetneq \dotsm \subsetneq y_k$ eine Kette von abgeschlossenen, irreduziblen Teilmengen in $Y$.

Dann ist auch $\leer \neq \Bar{y_0} \subsetneq \dotsm \subsetneq \Bar{y_n}$ Kette von abgeschlossenen, irreduziblen Teilmengen in $X$. Denn:
  \begin{itemize}
     \item $\Bar{y_i}$ ist irreduzibel nach Übungsblatt 2,
     \item $y_i=\Bar{y_i} \cap Y$, da $y_i$ in $Y$ abgeschlossen ist. Also $\Bar{y_i}\neq \Bar{y_{i+1}}$.
   \end{itemize}
Damit ist $\dim Y\le \dim X$.  
  \item[\ref{3.2.2b}] Nach \ref{3.2.2a} gilt \enquote{$\le$}.

Wähle Kette $\leer \neq A_0 \subsetneq \dotsm \subsetneq A_k$ in $X$ von irreduziblen, abgeschlossenen Teilmengen.

$A_k=\bigcup\limits_{i=1}^{n} (A_k \cap X_i)$ und die Irreduzibilität von $A_k$ impliziert nun $A_k=A_k \cap X_i$ für ein $i$, d.h. $A_k\subseteq X_i$.

Nach \ref{3.2.2a} folgt nun $k \le \dim A_k \le \dim X_i$ und damit die Behauptung.
 \end{enumerate}
\end{proof}

\begin{erinnerung}\label{3.2.3}
Sei $R$ ein kommutativer Ring mit $1$.
  \begin{enumerate}
  \item\Label{3.2.3a} Für ein Primideal $\wp \subseteq R$ definiert man die \emph{Höhe von $\wp$\index{Höhe}} durch
  \[ \hoehe(\wp)=\sup \{n\in \N_0 \mid \wp_0 \subsetneq \dotsm \subsetneq \wp_n:=\wp \text{ Kette von Primidealen}\}.\]
  \item\Label{3.2.3b} Man definiert die \emph{Krulldimension von $R$\index{Krulldimension}} durch
  \[\dim R=\sup\{\hoehe(\wp) \mid \wp \subseteq R \text { ist Primideal}\}.\] 
  \end{enumerate}
\end{erinnerung}

\begin{prop}\label{3.2.4}
Sei $V$ eine affine Varietät. Dann ist $\dim V=\dim K[V]$. 
\end{prop}
\begin{proof}
Erinnerung an \cref{1.7.1} aus \cref{kap1}:
\[\displaystyle \{\text{nichtleere irreduzible Untervarietäten von }V\} \lra \{\text{Primidealen in}K[V]\}\]
durch $W\mapsto \I(W)$ bzw. $\wp \mapsto \V(\wp)$. Diese Zuordnungen sind inklusionsumkehrend, also entsprechen aufsteigende Primidealketten absteigenden Ketten von irreduziblen Varietäten.
\end{proof}

\begin{erinnerung}[aus Algebra {\scshape ii}]\label{3.2.5}
Sei $K$ ein Körper, $A$ eine endlich erzeugte nullteilerfreie $K$-Algebra. Dann gilt:
\begin{itemize}[leftmargin=*,labelindent=\parindent]
  \item $\dim K\polyx=n$
  \item Ist $\phi \colon K\polyx \surj A$ surjektiver $K$-Algebrenhomomorphismus, so gilt: \[\dim A+\dim(\Kern(\phi))=n.\]
  \item Alle maximalen Primidealketten in $A$ haben dieselbe Länge. Diese ist $\dim A$. 

Insbesondere: Alle maximalen Primideale haben die Höhe $\dim(A)$.
  \item Allgemein gilt: Ist $R$ lokaler Ring mit maximalen Ideal $\m$, so ist $\hoehe\m=\dim R$.

Für einen beliebigen Ring $R$ gilt: $\hoehe\m=\dim R_\m$.
\end{itemize}
\end{erinnerung}

\begin{kor}\label{3.2.6}
\begin{enumerate}
  \item $\dim \A^n=n$
  \item Ist $V$ irreduzible affine Varietät im $\A^n$, so gilt 
\[\dim V+\dim \I(V)=n.\]
  \item Für $x \in V$ ist
\[\dim \O_{V,x}=\dim K[V]_{\m_x^V}=\hoehe(\m_x^V)=\dim K[V]=\dim V.\]
\end{enumerate}
\end{kor}

\begin{prop}\label{3.2.7} 
Sei $W$ eine quasi-projektive Varietät, $x\in W$, $V_0$ eine affine, offene Umgebung von $x$. Dann nennen wir 
\[\dim \O_{W,x}=:\dim_x W\]
 \emph{lokale Dimension von $x$ in $W$\index{lokale Dimension}}.
Es gilt:
  \begin{enumerate}
  \item\Label{3.2.7a} $\dim \O_{W,x}=\dim \O_{V_0,x}=\hoehe(\m_x^{V_0})$
  \item\Label{3.2.7b} Ist $W$ irreduzibel, so gilt für alle $x,y \in W$: $\dim_x W=\dim_y W=\dim W$.

  Außerdem: Ist $\leer \neq U$ offen und affin in $W$, so ist $\dim U=\dim W$.
  \item\Label{3.2.7c} $\dim_x W=\dim \O_{W,x}=\max\{\dim Z \mid Z \text{ irreduzible Komponente von W},x\in Z\}$.
   %%Bild ?
  \end{enumerate}
\end{prop}

\begin{proof}
  \begin{enumerate}
  \item[\ref{3.2.7a}] folgt mit \cref{3.2.5} direkt aus $\O_{W,x}\cong \O_{V_0,x}\cong K[V_0]_{\m_x^V}$.
  \item[\ref{3.2.7b}] Wir zeigen zunächst: $\dim_x W=\dim_y W \; \forall x,y \in W$.
  
  Seien $U_1,U_2$ offene, affine Umgebungen von $x$ bzw. $y$. Da $W$ irreduzibel ist, ist $U_1\cap U_2\neq \leer$.  

  Sei $z\in U_1\cap U_2$. Dann folgt mit \cref{3.2.6}: 
  \begin{align*}\dim_x W&=\dim \O_{W,x}=\dim \O_{U_1,x}=\dim \O_{U_1,z}\\&=\dim \O_{U_2,z}=\dim \O_{U_2,y}=\dim \O_{W,y}=\dim_y W\end{align*}
%
Nun zeigen wir: $\dim W=d:=\dim \O_{W,x}=\dim U$ für ein beliebiges, offenes, affines $U\neq \leer$ in $W$.

Wir sehen: $\dim W\ge d=\dim U$, da $U\subseteq W$. 

Angenommen, es gäbe eine Kette $\{x\}=W_0 \subsetneq W_1 \subsetneq \dotsc \subsetneq W_k$ von irreduziblen, abgeschlossenen Teilmengen mit $k>d$. Dann wählen wir $U$ offen und affin mit $x\in U$ und sehen, dass 
\[U\cap W_0=\{x\} \subsetneq U\cap W_1 \subsetneq \dotsm \subsetneq U\cap W_k\subseteq U\]
eine Kette von abgeschlossen, irreduziblen Teilmengen von $U$ ist, denn $\Bar{W_i\cap U}=W_i$, da $W_i$ irreduzibel ist. Damit wäre $d=\dim U\ge k>d$, was einen Widerspruch ergibt.
  \item[\ref{3.2.7c}] Wir können ohne Einschränkung annehmen, dass $W$ affin ist. Dann folgt mit \cref{3.2.5}:
\[\O_{W,x}=\dim K[W]_{\m_x^W}=\hoehe \m_x^W=\sup \{k \mid 0\neq \wp_0 \subsetneq \dotsm \subsetneq \wp_k=\m_x^W\}\]
Ohne Einschränkung sei $\wp_0$ minimales Primideal und die Kette maximal. Also entspricht $\wp_0$ einer irreduziblen Kompenente $Z$ mit $x\in Z$. Wieder mit \cref{3.2.5} ist $\dim Z=k$.
  \end{enumerate}
\end{proof}

\begin{dfn}\label{3.2.8}
  \begin{enumerate}
  \item\Label{3.2.8a} Eine quasi-projektive Varietät von Dimension $1$ heißt \emph{Kurve\index{Kurve}}.
  \item\Label{3.2.8b} Eine quasi-projektive Varietät von Dimension $2$ heißt \emph{Fläche\index{Fläche}}.
  \end{enumerate}
\end{dfn}

\begin{bsp}\label{3.2.9}
  \begin{enumerate}
  \item Nach \cref{3.2.7} ist $\dim \P^n=n$.
  \item Betrachte eine Hyperfläche $\V(f)\subseteq \A^n(K)$ mit $f\in K\polyx$, $\deg f\ge 1$.

\[\text{Dann gilt: }\dim \V(f)=n-1.\]
Denn: Nach \cref{3.2.2} kann man $f$ ohne Einschränkung als irreduzibel voraussetzen.
Damit ist $\dim \V(f)=\dim K[V]=n-\hoehe(f)$.

Angenommen es gäbe ein Primideal $\wp$ mit $0\subsetneq \wp \subsetneq (f)$. Wähle nun $h\in \wp$ mit minimalem Grad. Da $f$ irreduzibel ist, folgt $h \in (f)$.

Folglich: $h=h'f$ und $h\neq f$, also $h' \in \wp$. Aber der Grad von $h'$ ist kleiner als der von $h$---ein Widerspruch!

Insgesamt: $\hoehe (f)=1$ und die Behauptung gilt.
   \item Betrachte die Varietät $V=\V(XZ,YZ)=\V(Z)\cup \V(X,Y)$ (Ebene mit senktrechter Gerade durch den $0$-Punkt).

Es ist $\V(Z)\cong \A^2$, d.h $\dim \V(Z)=2$, und $\V(X,Y)\cong \A^1$, d.h $\dim \V(X,Y)=1$. 

Nach \cref{3.2.2} ist also $\dim V=2$.
   \item Es ist $\dim \V(X^2+Y^2-Z^2)=2$, da Hyperfläche.
  \item Genauso ist $\dim \V(X^2+Y^2-1)=1$, da Hyperfläche.

   Die Wünsche, die zu Beginn des \hyperref[3.2]{Abschnitts \cref{3.2}} geäußert wurden, sind also erfüllt.
  \end{enumerate}
\end{bsp}

%17.1.11
\sect{Der Tangentialraum}

\begin{nbsp}
\begin{enumerate}
\item Sei $V=\V(Y^{2}-X^{3}+X)$ eine elliptische Kurve, $P=(0,0)$. Die Tangente an $V$ in $P$ ist die $Y$-Achse, also $\V(X)$.
% hier bräuchte man ein schönes Bild von einer elliptischen Kurve...
\item Sei $V=\V(Y^{2})$ der Newton-Knoten. Dann gibt es in $P=(0,0)$ zwei \enquote{Tangenten}. Aber wie könnte der Tangentialraum aussehen?
% hübsches Bild vom Newtonknoten
\item Sei $V=\V(Y^{2}-X^{3})$ die Neilsche Parabel. Dann ist die $X$-Achse \enquote{doppelte Tangente} in $P=(0,0)$, aber auch hier ist der Tangentialraum nicht intuitiv klar.
% neilsche Parabel....
\end{enumerate}\end{nbsp}

\begin{erinnerung}\label{3.3.1}
Sei $f\in K\polyx$ und $p=(p_{1},\dotsc,p_{n})\in K^{n}$. Dann haben wir die \emph{Taylor-Entwicklung\index{Taylor-Entwicklung}}
\[f=\!\!\!\!\sum_{\substack{\alpha=(k_{1},\olddotsc,k_{n})\\k_{i}\geq 0}}\!\!\frac{1}{(k_{1}+\dotsm+k_{n})!}\frac{\partial^{k_{1}}}{\partial X_{n}}\dotsm\frac{\partial^{k_{n}}}{\partial X_{n}}f(p)(X_{1}-p_{1})^{k_{1}}\dotsm(X_{n}-p_{n})^{k_{n}}.\]
Insbesondere gilt dann:
\[f=f(p)+\sum_{i=1}^{n}\frac{\partial f}{\partial X_{i}}(p)\cdot(X_{i}-p_{i})+\text{Restterme},\]
wobei die Restterme vom Grad mindestens $2$ sind, also in $\m_{p}^{2}$, wobei
\[\m_{p}:=(X_{1}-p_{1},\dotsc,X_{n}-p_{n}).\]
\end{erinnerung}

\begin{dfn}\label{3.3.2}
Sei $p\in K$ mit $p=(p_{1},\dotsc,p_{n})$.
\begin{enumerate}
\item\Label{3.3.2a} Für $f\in K\polyx$ sei
\[f_{p}^{(1)}:=f^{(1)}:=\sum_{i=1}^{n}\frac{\partial f}{\partial X_{i}}(p)\cdot X_{i}=:\DD_{p}(f).\]
\item\Label{3.3.2b} Sei $\leer\neq V\subseteq\A^{n}(K)$ eine affine Varietät, $p\in V$ und $I:=\I(V)$. Seien zusätzlich
\[\II_{p}:=\langle f_{p}^{(1)}\mid f\in I\rangle\text{ und }\T_{p}:=\T_{V,p}:=\V(\II_{p})\subseteq\A^{n}(K).\]
Dann heißt $\T_{p}$ \emph{Tangentialraum an $V$ in $p$\index{Tangentialraum an $V$ in $p$}}.
\end{enumerate}\end{dfn}

\begin{bsp}\label{3.3.3}
Seien $f=X^{2}+Y^{2}-1$, $V=\V(f)$, $I=(f)$ und $p=(a,b)\in V$.
% Bild...
Dann ist
\[f_{p}^{(1)}=\frac{\partial f}{\partial X}(p)\cdot X+\frac{\partial f}{\partial Y}(p)\cdot Y=2aX + 2bY,\]
also $\II_{p}=\langle 2aX+2bY \rangle$ und $\T_{V,p}=\V(\II_{p})=\{(x,y)\in K^{2}\mid bY=-aX\}$, denn es gilt
\[h\cdot f_{p}^{(1)}=\sum_{i=1}^{n}\frac{\partial(h\cdot f)}{\partial X_{i}}(p)\cdot X_{i}=\sum_{i=1}^{n}\frac{\partial h}{\partial X_{i}}(p)\cdot f(p)\cdot X_{i}+\sum_{i=1}^{n}\frac{\partial f}{\partial X_{i}}(p)\cdot h(p)\cdot X_{i}=h(p)\cdot f_{p}^{1},\]
denn $f(p)=0$, da $p\in V$, und damit ist $\II_{p}$ auch nicht größer.
\end{bsp}

\begin{bem}\label{3.3.4}
\begin{enumerate}
\item\Label{3.3.4a} Die Taylor-Entwicklung liefert
\[f=f(p)+f_{p}^{(1)}-f_{p}^{(1)}(p)+\text{Restterme},\]
wobei die Restterme in $\m_{p}^{2}$ liegen.
\item\Label{3.3.4b} $\DD_{p}(f+g)=(f+g)_{p}^{(1)}=f_{p}^{(1)}+g_{p}^{(1)}=\DD_{p}(f)+\DD_{p}(g)$

$\DD_{p}(f\cdot g)=(f\cdot g)_{p}^{(1)}=f(p)\cdot g_{p}^{(1)}+g(p)\cdot f_{p}^{(1)}=f(p)\cdot \DD_{p}(g)+g(p)\cdot\DD_{p}(f)$, vgl. auch \cref{3.3.3}.
\item\Label{3.3.4c} Sei $V\subseteq\A^{n}(K)$ eine affine Varietät, $I=\I(V)=\langle f_{1},\dotsc,f_{n}\rangle$ und $p\in V$. Dann ist
\begin{itemize}
\item $\II_{p}=\langle f_{1}^{(1)},\dotsc,f_{r}^{(1)}\rangle$ analog zu \cref{3.3.3},
\item $\T_{V,p}=\V(\II_{p})$ ein linearer Unterraum von $\A^{n}(K)=K^{n}$:
\begin{align*}\T_{V,p}&=\{x=(x_{1},\dotsc,x_{n})\in\A^{n}(K)\mid \forall f\in I:\sum_{j=1}^{n}\frac{\partial f}{\partial X_{j}}(p)\cdot X_{j}=0\}\\
&=\Kern\left(\frac{\partial f_{i}}{\partial X_{i}}(p)\right)_{\substack{i=1,\olddotsc,r\\ j=1,\olddotsc,n}},\end{align*}
also der Kern der Jacobi-Matrix.
\end{itemize}
\end{enumerate}\end{bem}

\begin{bsp}\label{3.3.5}
Damit können wir jetzt ganz viele Tangentialräume bestimmen. Sei $p=(0,0)$.
\begin{enumerate}
\item Sei $V=\V(Y^{2}-X^{3}+X)$, dann ist $\T_{V,p}=\V(2\cdot 0\cdot Y - 3\cdot 0^{2}\cdot X + 1\cdot X)=\V(X)$.
\item Sei $V=\V(Y^{2}-X^{3}+X^{2})$, dann ist $\T_{V,p}=\V(0)=\A^{2}(K)$.
\item Sei $V=\V(Y^{2}-X^{3})$. Dann ist $\T_{V,p}=\V(0)=\A^{2}(K)$.
\item Sei $V=\V(X^{2}+Y^{2}-Z^{2})$, dann ist 
\[\T_{V,p}=\V(0)=\A^{3}(K)\text{ und }\T_{V,(0,1,1)}=\V(2Y-2Z).\]
% hier könnte man ein Bild malen....
\end{enumerate}\end{bsp}

\begin{w} Bisher hängt unsere Definition von $\T_{V,p}$ von der Einbettung von $V$ in den $\A^{n}(K)$ ab.\end{w}

\begin{bem}\label{3.3.6}
Sei $\varphi\colon\A^{n}\supseteq V\ra W\subseteq\A^{m}$ ein Morphismus zwischen affinen Varietäten. Dann induziert $\varphi$ in natürlicher Weise eine $K$-lineare Abbildung
\[\operatorname{d}_{p}\varphi\colon\V(\II_{p})=\T_{V,p}\ra\T_{W,\varphi(p)}=\V(\II_{\varphi(p)}).\]
\end{bem}

\begin{proof}
Gegeben seien also $\varphi\colon V\ra W$, $I:=\I(V)$ und $J:=\I(W)$.

Dann wählen wir eine Fortsetzung $\dach{\varphi}\colon\A^{n}\ra\A^{m}$ und 
\[\dach{\varphi}=(\dach{\varphi}_{1},\dotsc,\dach{\varphi}_{m})\text{ mit }\dach{\varphi}_{i}\in K\polyx.\] 
So erhalten wir
\[\dach{\varphi}^{\sharp}\colon K[Y_{1},\dotsc,Y_{m}]\ra K\polyx\]
mit $\dach{\varphi}^{\sharp}(Y_{i})=\dach{\varphi}_{i}$ und $\dach{\varphi}^{\sharp}(J)\subseteq I$. Wir definieren den $K$-Algebrenhomomorphismus
\[\alpha\colon K[Y_{1},\dotsc,Y_{m}]\ra K\polyx\text{ durch }\alpha(Y_{i})=\DD_{p}(\dach{\varphi}^{\sharp}(Y_{i}))=\DD_{p}(\dach{\varphi}_{i})=\dach{\varphi}_{i}\mbox{}^{(1)}_{p}.\]
Nun genügt es $\alpha(\II_{\varphi(p)})\subseteq\II_{p}$ zu zeigen, denn dann induziert $\alpha$ das gewünschte $\operatorname{d}_{p}\varphi$.

Seien also $g\in J$ und $\Bar{g}:=\DD_{p}(g)\in\II_{\varphi(p)}$. Dann gilt:
\begin{align*}
\alpha(\Bar{g})&=\sum_{j=1}^{m}\frac{\partial g}{\partial \varphi_{j}}(\varphi(p))\cdot \DD_{p}(\dach{\varphi}_{j})=\sum_{i=1}^{n}X_{i}\biggl(\sum_{j=1}^{m}\frac{\partial g}{\partial \varphi_{j}}(\varphi(p))\cdot\frac{\partial\dach{\varphi}_{j}}{\partial X_{i}}(p)\biggr)\\
&=\sum_{i=1}^{n}X_{i}\frac{\partial (g\circ \dach{\varphi})}{\partial X_{i}}(p)=\DD_{p}(g\circ\dach{\varphi})\in\II_{p},
\end{align*}
denn $g\circ\dach{\varphi}$ liegt in $I$.
\end{proof}

%19.1.11
\begin{db}\label{3.3.7}
\begin{enumerate}
\item\Label{3.3.7a} Wir nennen $\DD:\O_{V,p}\ra K$ eine \emph{Derivation an $p$\index{Derivation}}, wenn $\DD$ $K$-linear ist und 
\[\DD(f\cdot g)=g(p)\cdot\DD(f)+f(p)\cdot\DD(g).\]
\item\Label{3.3.7b} $\DD$ ist bereits durch $\DD\restrict{\m_{p}}$ für das maximale Ideal $\m_{p}=\{f\mid f(p)=0\}$ festgelegt, denn für $k$ konstant ist $\DD(k)=0$, also gilt, für beliebiges $f\in\O_{V,p}$, $\schlange{f}:=f-f(p)$ ist in $\m_{p}$ und $\DD(\schlange{f})=\DD(f)$.
\item\Label{3.3.7c} Sei $f\in\m_{p}^{2}$. Dann ist $\DD(f)=0$, denn wir können, ohne Einschränkung, $f=g\cdot h$, mit $g,h\in\m_{p}$ wählen und sehen
\[\DD(f)=\DD(gh)=g(p)\DD(h)+h(p)\DD(g)=0,\]
da $g(p)=h(p)=0$. $\DD$ definiert also eine lineare Abbildung
\[\l\colon\Quotient{\m_{p}}{\m_{p}^{2}}\ra K.\]
\item\Label{3.3.7d} Jedes solche $\l$ kommt von einer Derivation her. Dazu definieren wir \[\DD(f_{p}):=\l(f_{p}-f_{p}(p)).\]
%
Wir müssen also zeigen, dass $\DD$ eine Derivation an $p$ ist, also dass:
\[\DD(f_{p}\cdot g_{p})=\l(f_{p}\cdot g_{p}-f_{p}(p)\cdot g_{p}(p))\]
gilt. Dazu verwenden wir, dass $(f_{p}-f_{p}(p))\cdot(g_{p}-g_{p}(p))\in\m_{p}^{2}$, also
\begin{align*}
0&=\l\bigl((f_{p}-f_{p}(p))(g_{p}-g_{p}(p))\bigr)\\
&=\l\bigr(f_{p}g_{p}-f_{p}(p)g_{p}(p)-f_{p}(p)g_{p}-g_{p}(p)f_{p}+2f_{p}(p)g_{p}(p)\bigl)
\end{align*}
 und sehen damit:
\begin{align*}
\l\bigl(f_{p}g_{p}-f_{p}(p)g_{p}(p)\bigr)&=\l\bigl(f_{p}(p)g_{p}-f_{p}(p)g_{p}(p)\bigr) + \l\bigl(g_{p}(p)f_{p}-f_{p}(p)g_{p}(p)\bigr)\\
&=f_{p}(p)\DD(g_{p})+g_{p}(p)\DD(f_{p}).
\end{align*}
\end{enumerate}\end{db}

Die Derivationen an $p$ auf $\O_{V,p}$ können demnach mit dem Dualraum $(\Quotient{\m_{p}}{\m_{p}^{2}})^{\vee}$ identifiziert werden. Wir erhalten so einen Isomorphismus von $K$-Vektorräumen. Genauer:

\begin{prop}\label{3.3.8}
\begin{enumerate}
\item\Label{3.3.8a} Jedes $a=(a_{1},\dotsc,a_{n})\in\A^{n}(K)$ definiert eine Derivation
\begin{align*}\DD_{a}\colon\O_{\A^{n},p}\cong K\polyx_{\m_{p}}&\ra K,\\
f&\mapsto\sum_{i=1}^{n}\frac{\partial f}{\partial X_{i}}(p)\cdot a_{i}=f_{p}^{1}(a)=\DD_{p}(f)(a),
\end{align*}
dabei war $\m_{p}=(X_{1}-p_{1},\dotsc,X_{n}-p_{n})$ für $p=(p_{1},\dotsc,p_{n})$.
\item\Label{3.3.8b} Falls $a\in\T_{V,p}=\V(\II_{p})$, steigt $\DD_{a}$ zu einer Derivation $\tau_{a}\colon\O_{V,p}\ra K$ ab, via des von der Einbettung $i\colon V\inj\A^{n}$ induzierten Morphismus $i^{\sharp}\colon\O_{\A^{n},p}\surj\O_{V,p}$. 
\end{enumerate}\end{prop}

\begin{proof}\begin{enumerate}
\item[\ref{3.3.8a}] Die Derivationseigenschaft kommt von der entsprechenden Eigenschaft der $\frac{\partial}{\partial X_{i}}$.
\item[\ref{3.3.8b}] Ohne Einschränkung nehmen wir $V$ als affin an. Zunächst gilt für alle $f\in\I(V)$, dass $\DD_{a}(f)=\DD_{p}(f)(a)=0$, da $f_{p}^{(1)}\in\II_{p}$ und $a\in\T_{V,p}$. 

Sei nun $\bigl(\frac{f}{g}\bigr)\mbox{}_{p}\in\O_{\A^{n},p}$ mit $i^{\sharp}\bigl(\bigl(\frac{f}{g}\bigr)\mbox{}_{p}\bigr)=0,$ das heißt $g(p)\neq 0$ und $\frac{f}{g}$ ist $0$ in $K[V]$, also gibt es ein $h$ mit $h(p)\neq 0$, so dass $h\cdot f\in\I(V)$. Insbesondere ist $\DD_{a}(hf)=0$ und $f(p)=0$, da $0=\bigl(hf\bigr)(p)=h(p)f(p)$. Damit gilt:
\[0=\DD_{a}(hf)=\DD_{a}(h) f(p)+\DD_{a}(f) h(p)\]
und da $h(p)\neq 0$, muss schon $\DD_{a}(f)=0$ gelten. Damit ist auch
\[\DD_{a}\left(\frac{f}{g}\right)=\frac{g(p)\cdot\DD_{a}(f)-f(p)\cdot\DD_{a}(g)}{g(p)^{2}}=0\]
und damit ist $\tau_{a}$ wohldefiniert.
\end{enumerate}\end{proof}

\begin{defprop}\label{3.3.9}
Sei $V$ eine affine Varietät und $p=(p_{1},\dotsc,p_{n})\in V$.
\begin{enumerate}
\item\Label{3.3.9a} $\DD_{a}\colon\O_{V,p}\ra K$ induziert einen Vektorraumhomomorphismus
\[\tau_{a}\colon\Quotient{\m_{p}}{\m_{p}^{2}}\ra K,\]
also ein Element $\tau_{a}\in\bigl(\Quotient{\m_{p}}{\m_{p}^{2}}\bigr)^{\vee}$.
\item\Label{3.3.9b} Die Abbildung
\[\alpha_{p}\colon \T_{V,p}\ra\bigl(\Quotient{\m_{p}}{\m_{p}^{2}}\bigr)^{\vee},\quad a\mapsto\tau_{a}\]
ist ein Isomorphismus von Vektorräumen.
\end{enumerate}\end{defprop}

\begin{proof}\begin{enumerate}
\item[\ref{3.3.9a}] folgt sofort aus \cref{3.3.7c} mit \cref{3.3.8b}.
\item[\ref{3.3.9b}] Wir definieren die Umkehrabbildung durch
\[\beta_{p}\colon\bigl(\Quotient{\m_{p}}{\m_{p}^{2}}\bigr)^{\vee}\ra\T_{V,p},\quad\l\mapsto\bigl(\l(\Bar{X_{1}-p_{1}}),\dotsc,\l(\Bar{X_{n}-p_{n}})\bigr).\]
Wir zeigen zuerst, dass $\beta$ wohldefiniert ist, also $\beta_{p}(\l)=:a\in\T_{V,p}=\V(\II_{p})$.

Sei dazu $f\in\I(V)$, also $f_{p}^{(1)}\in\II_{p}$. Dann gilt nach Definition
\begin{align*}
f_{p}^{(1)}(a)&=\sum_{i=1}^{n}\frac{\partial f}{\partial X_{i}}(p)\cdot\l(\Bar{X_{i}-p_{i}})\\
&=\l\biggl(\sum_{i=1}^{n}\Bar{\frac{\partial f}{\partial X_{i}}(p)\cdot(X_{i}-p_{i})}\biggr)=\l\bigl(\Bar{f_{p}^{(1)}-f_{p}^{(1)}(p)}\bigr).
\end{align*}
In \cref{3.3.4a} hatten wir gesehen, dass
\[f-f(p)-(f_{p}^{(1)}-f_{p}^{(1)}(p))\in\m_{p}^{2}\]
liegt, wobei hier natürlich $f(p)=0$ ist. In $K[V]$ ist damit sogar $f_{p}^{(1)}-f_{p}^{(1)}(p)\in\m_{p}^{2}$, also gilt 
\[\text{in }\Quotient{\m_{p}}{\m_{p}^{2}}\text{ ist }\Bar{f_{p}^{(1)}-f_{p}^{(1)}(p)}=0,\]
also ist $f_{p}^{(1)}(a)=0$ und damit ist $a\in\V(\II_{p})=\T_{V,p}$.

Um einzusehen, dass es sich wirklich um die Umkehrabbildung handelt, erinnern wir uns daran, dass, für $a=(a_{1},\dotsc,a_{n})$,
\[\tau_{a}\colon \Quotient{\m_{p}}{\m_{p}^{2}}\ni f\mapsto\DD_{p}(f)(a)=\sum_{i=1}^{n}\frac{\partial f}{\partial X_{i}}(p)\cdot a_{i}\in K\]
gilt. Damit erhalten wir für $\beta_{p}(\alpha_{p}(a))$:
\[\beta_{p}(\tau_{a})=\biggl(\sum_{i=1}^{n}\frac{\partial (\Bar{X_{1}-p_{1}})}{\partial X_{i}}(p)\cdot a_{1},\dotsc,\sum_{i=1}^{n}\frac{\partial (\Bar{X_{n}-p_{n}})}{\partial X_{i}}(p)\cdot a_{n}\biggr)\]
und da $\frac{\partial (X_{j}-p_{j})}{\partial X_{i}}$ genau dann $1$, wenn $i=j$ und ansonsten $0$ ist, ergibt das gerade wieder $a$.

Nun überlegen wir uns noch, dass $\l$ durch
\[\alpha_{p}(\beta_{p}(\l))=\alpha_{p}(\l(\Bar{X_{1}-p_{1}}),\dotsc,\l(\Bar{X_{n}-p_{n}})),\]
nach gleicher Argumentation wie oben, gerade auf die Abbildung
\begin{align*}
f\mapsto \DD_{p}(f)(\l(\Bar{X_{1}-p_{1}}),\dotsc,\l(\Bar{X_{n}-p_{n}}))&=\sum_{i=1}^{n}\frac{\partial f}{\partial X_{i}}(p)\cdot \l(\Bar{X_{i}-p_{i}})\\
&=\l(\Bar{f_{p}^{(1)}-f_{p}^{(1)}(p)}),
\end{align*}
geschickt wird. In $\Quotient{\m_{p}}{\m_{p}^{2}}$ ist aber, wie oben gesehen,
\[\Bar{f}=\Bar{f_{p}^{(1)}-f_{p}^{(1)}(p)}\]
und damit gilt tatsächlich
\[\alpha_{p}(\beta_{p}(\l))=\bigl(f\mapsto \l(f)\bigr)=\l,\]
$\beta_{p}$ ist also, wie behauptet, die Umkehrabbildung.

Somit sind die Vektorräume isomorph.
\end{enumerate}\end{proof}

\begin{db}\label{3.3.10}
Sei $V$ eine quasi-projektive Varietät, $p\in V$.
\begin{enumerate}
\item\Label{3.3.10a} $\displaystyle \bigl(\Quotient{\m_{p}}{\m_{p}^{2}}\bigr)^{\vee}$ heißt \emph{(Zariski-)Tangentialraum\index{Tangentialraum}}.
\item\Label{3.3.10b} Der Punkt $p$ heißt \emph{regulär\index{regulär}} (bzw. \emph{nicht-singulär\index{nicht-singulär}}), wenn $\dim\T_{V,p}=\dim_{p} V$ ist.

Andernfalls heißt $p$ \emph{singulär\index{singulärer Punkt}}.
\item\Label{3.3.10c} Sei $U\subseteq\A^{n}$ eine offene und affine Umgebung von $p$ in $V$ und $\I(U)=\langle f_{1},\dotsc,f_{n}\rangle$. Dann gilt
\[p\text{ ist regulär }\iff \Rang\left(\frac{\partial f_{i}}{\partial X_{j}}\right)_{\substack{i=1,\olddotsc,r\\ j=1,\olddotsc,n}}=n-\dim_{p}V,\]
denn nach \cref{3.3.4c} ist $\T_{p}$ gerade der Kern dieser Matrix.

Diese Äquivalenz nennt man das \emph{Jacobi-Kriterium\index{Jacobi-Kriterium}}.
\end{enumerate}\end{db}

\begin{bsp}\label{3.3.11}\begin{enumerate}
\item\Label{3.3.11a} Sei $V=\V(X^{2}+Y^{2}-Z^{2})$. Dann ist $\dim V=\dim_{p}V$ für alle Punkte $v\in V$. Für die Jacobi-Matrix gilt:
\[\J=\left(\!\!\begin{array}{r}2x\\2y\\-2z\end{array}\!\right)\]
und nach \cref{3.3.10c} ist $p$ genau dann regulär, wenn \[\Rang\J=3-2=1\] gilt. Also ist $p=(0,0,0)$ der einzige singuläre Punkt.
\item\Label{3.3.11b} Sei $V=\V(X^{2}-Y^{2})=\V((X-Y)(X+Y))\subseteq\A^{3}$. Anschaulich sind das zwei Ebenen, die senkrecht aufeinander stehen. Ihr Schnitt ist die $Z$-Achse. Analog zu \ref{3.3.10a} finden wir
\[\J=\left(\!\!\begin{array}{c}\phantom{-}2x\\-2y\\\phantom{-}0\end{array}\!\right)\]
und demnach ist hier $p$ genau dann singulär, wenn $p=(0,0,z)$ mit $z\in K$, also ist die Menge der singulären Punkte gerade die $Z$-Achse.
\item\Label{3.3.11c} Sei $V=\V(f)\subseteq\A^{n}$ eine Hyperfläche, also $\deg f\geq 1$. Dann ist
\[\J_{p}=\left(\!\!\begin{array}{c}\frac{\partial f}{\partial X_{1}}(p)\\\vdots\\\frac{\partial f}{\partial X_{n}}(p)\end{array}\!\right)\]
und $p$ ist genau dann regulär, wenn $\Rang\J=n-(n-1)=1$, d.h.
\[p\text{ ist singulär }\iff\frac{\partial f}{\partial X_{1}}(p)=\dotsm=\frac{\partial f}{\partial X_{n}}(p)=0.\]
\end{enumerate}\end{bsp}

\sect{Der singuläre Ort einer Varietät}

\begin{dfn}\label{3.4.1} Sei $W$ quasi-projektive Varietät. Dann heißt 
\[\Sing W=\{p \in W \mid p \text{ singulär}\}\] der \emph{singuläre Ort\index{singulärer Ort}}.
\[\Reg W=\{p \in W \mid p \text{ regulär}\}=W\setminus \Sing W\] 
heißt \emph{regulärer Ort\index{regulärer Ort}}.
\end{dfn}

\begin{db}\label{3.4.2}
  \begin{enumerate}
  \item\Label{3.4.2a} Ein noetherscher, lokaler Ring $R$ mit maximalem Ideal $\m$ heißt \emph{regulär\index{regulär}}, wenn 
  $\dim_k \Quotient{\m}{\m^2}=\dim R$. Dabei ist $k=\Quotient{R}{\m}$.
  \item\Label{3.4.2b} Sei $W$ quasi-projektive Varietät, $p \in W$. Dann gilt:
  \[p \text{ ist regulär} \iff \O_{W,p} \text{ ist regulär}.\]
  \end{enumerate}
\end{db}

\begin{lem}\label{3.4.3} Sei $(R,\m)$ ein lokaler, noetherscher Ring.
  \begin{enumerate}
  \item\Label{3.4.3a} Ist $R$ regulär, so ist $R$ auch nullteilerfrei.
  \item\Label{3.4.3b} Es gilt: $\dim \Quotient{\m}{\m^2} \ge \dim R$
  \end{enumerate}
\end{lem}
\begin{proof}siehe \cref{3.5.4}.
\end{proof}

\begin{satz}\label{satz6} Sei $W$ eine quasi-projektive Varietät.
  \begin{enumerate}
   \item\label{satz6a} Ist $p\in W$, so ist $\dim \T_{W,p} \ge \dim_p W$ und $\Rang \J_p \le n-\dim_p W$, wobei $\J_p$ die Jacobi-Matrix zu $V$ in $p$ ist und $V\subseteq \A^n$ eine offene, affine Umgebung von $p$ ist.
  \item\label{satz6b} Sind $W_1\neq W_2$ irreduzible Komponenten von $W$, $p\in W_1\cap W_2$, dann ist $p$ singulär.
   \item\label{satz6c} $\Sing W$ ist abgeschlossen.
   \item\label{satz6d} Es ist $\Sing W\neq W$.
  \end{enumerate}
\end{satz}

\begin{proof}
  \begin{enumerate}
  \item[\ref{satz6a}] folgt aus \cref{3.4.3b} sowie $\T_{W,p}\cong \Kern \J_p$.
  \item[\ref{satz6b}] Da wir eine lokale Eigenschaft untersuchen können wir ohne Einschränkung $W$ als affin annehmen. Sei nun $p\in W_1\cap W_2$.

  Dann ist $\m_p^W\supseteq \I(W_1),\I(W_2)$ in $K[W]$.

  Sind $W_1$, $W_2$ irreduzible Komponenten, so sind $\I(W_1)$, $\I(W_2)$ minimale Primideale. Folglich sind die Bilder von $\I(W_1)$ und $\I(W_2)$ minimale Primideale in $\O_{W,p}$ bzw. $K[W]_{\m_p^W}$ und verschieden (vgl. Aufgabe 2, Übungsblatt 6).

  Damit ist $(0)$ kein Primideal in $\O_{W,p}$. Also ist $\O_{W,p}$ nicht nullteilerfrei und nach \cref{3.4.3a} ist $\O_{W,p}$ nicht regulär.
  \item[\ref{satz6c}] Sei $W=W_1\cup \dotsm \cup W_k$ die Zerlegung in irreduzible Komponenten.

  Ist $p\in W_i$ und $p\notin W_j$ für $j\neq i$, so ist $p$ genau dann singulär in $W_i$, wenn $p$ singulär in $W\cap W_i$ ist.

 Denn: Ist $p\in W_i\cap W_j$ $(i\neq j)$, so ist $p \in \Sing W$ und damit
\[\Sing W=\bigcap\limits_{i=1}^{k} \Sing W_i \cup \bigcap\limits_{j\neq i}^{}(W_i\cap W_j).\]
Da $  \bigcap\limits_{j\neq i}^{}(W_i\cap W_j)$ abgeschlossen ist genügt es die Aussage für irreduzible Varietäten zu zeigen. 

Sei also $W$ irreduzibel und affin. Dann gilt:
\[\Sing W=\{p\in W \mid \Rang \J_p < n-\dim W\}=\{p\in W \mid \det M_p=0\;\forall M_{p}\},\]
wobei die $M_{p}$ die $(n-\dim W)\times (n-\dim W)$-Minoren von $\J_p$ sind. Diese Menge ist abgeschlossen.
  \item[\ref{satz6d}] Ist $W=W_1\cup \dotsm \cup W_k$ die Zerlegung in irreduzible Komponenten und $\Reg W_i$ nicht leer, so ist $\Reg W_i$ dicht in $W_i$.

Also gibt es $p\in W_i$, so dass $p\notin W_j$, für alle $i\neq j$. Damit ist $p\in \Reg W$ und man kann ohne Einschränkung annehmen, dass $W$ irreduzibel ist.

Sei jetzt also $V=W$ affin und irreduzibel. Betrachte zuerst denn Spezialfall, dass $W=V=\V(f)$ eine Hyperfläche ist, wobei $f\in K\polyx$ quadratfrei sein soll.

Nach \cref{3.3.11} gilt:\[\Sing V=\left\{p\in V\;\middle|\; \frac{\partial f}{\partial X_i}(p)=\dotsm=\frac{\partial f}{\partial X_n}(p)=0\right\}\]
Wäre $\Sing V=V$, dann wäre $\frac{\partial f}{\partial X_i}\in \I(V)=\langle f\rangle$ für $i=1,\dotsc,n$.

Hat $K$ Charakteristik $0$, so muss $f$ bereits konstant sein.

Hat $K$ Charakteristik $p$, so muss jeder Exponent, der auftritt bereits durch $p$ teilbar sein. Also ist $f=g^p$ für ein $g\in K\polyx$, was im Widerspruch zur Quadratfreiheit von $f$ steht.

Nun der allgemeine Fall:

Nach \cref{3.4.4} gibt es eine offene, dichte Teilmenge $U\subseteq V$, die isomorph ($\phi$ bezeichne den Isomorphismus) zu einer offenen, dichten Teilmenge $U'$ in einer Hyperfläche $\V(f)$ ist.

Nach dem Spezialfall ist $\Reg \V(F)\neq \leer$. Damit ist $U'\cap \Reg \V(f)\neq \leer$, da $U'$ dicht ist und $\Reg \V(f)$ offen ist. Sei $p'\in U'\cap \Reg \V(f)$.

Dann ist $p'$ regulär, d.h. $\O_{U',p'}$ ist regulär. Also ist auch $\phi(p') \in U$ regulär, da $\O_{U,p}$ regulär ist.
  \end{enumerate}
\end{proof}
\begin{lem}\label{3.4.4} Jede irreduzible quasi-projektive Varietät $W$ von Dimension $d$ ist birational äquivalent zu einer Hyperfläche im $\A^{d+1}(K)$.
\end{lem}
\begin{nerinnerung} 
\begin{enumerate}
  \item Sei $A$ eine nullteilerfreie, endlich erzeugte $K$-Algebra.
  \begin{itemize}
       \item Die Noethernormalisierung impliziert, dass $A$ ganz über dem Polynomring $K\polyx[d]$ ist.
       \item Die Dimension bleibt unter ganzen Ringerweiterungen erhalten. Also gilt $d=\dim A$ und der Transzendenzgrad von $\Quotient{\Quot A}{K}=d=\dim A$.
 
       Damit gilt für eine irreduzible Varietät $V$ die Formel $\trdeg K(V)=\dim V$.
  \end{itemize}
   \item\Label{3.4erinn} Sei $\Quotient{L}{K(X_1,\dotsc,X_d)}$ endliche Körpererweiterung, $E:=K(X_1,\dotsc,X_d)$.

Hat $K$ Charakteristik $0$, so ist $\Quotient{L}{K}$ separabel, d.h. $L=E(\alpha)$ für ein $\alpha \in L$ nach dem Satz vom primitiven Element.

Hat $K$ Charakteristik $p$, so impliziert die algebraische Abgeschlossenheit von $K$, dass $\Quotient{L}{K}$ separabel erzeugt ist, d.h. es gibt eine Transzendenzbasis $\{\schlange{X_1},\dotsc,\schlange{X_d}\}$, so dass $\Quotient{L}{K(\schlange{X_1},\dotsc,\schlange{X_d})}$ separabel ist (siehe Bosch, Abschnitt 7.3, Satz 7).

Im Folgenden sei eine Transzendenzbasis mit dieser Eigenschaft gewählt.

\end{enumerate}
\end{nerinnerung}

\begin{proof}[Beweis von \cref{3.4.4}] Seien $L=K(W)$ und $\{X_1,\dotsc,X_d\}$ eine Transzendenzbasis wie in \ref{3.4erinn}. Also gibt es ein primitives Element $y\in L=K(W)$ von $\Quotient{L}{K(X_1,\dotsc,X_d)}$, d.h.
\[L=K(X_1,\dotsc,X_d,y)=K(X_1,\dotsc,X_d)[y]\]
Sei $p(Y)=Y^n+a_{n-1}Y^{n-1}+\dotsm+a_0$ das Minimalpolynom von $y$ über $K(X_1,\dotsc,X_d)$ und $a_0=:\frac{f_i}{g_i}$ mit $f_i, g_i \in K\polyx[d]$.

Dann setzen wir 
\[\displaystyle g:=\prod\limits_{i=1}^{n-1} g_i\text{ und damit }h:=gY+ga_{n-1}Y^{n-1}+\dotsm+ga_0 \in K[X_1\dotsc,X_d,Y],\] 
denn die $ga_{i}$ sind bereits in $K\polyx[d]$.

Nun definieren wir $H:=\V(h)\subseteq \A^{d+1}$ und sehen, dass
\[K(H)=\Quot\left(\Quotient{K[X_1,\dotsc,X_d,Y]}{(h)}\right)=L=K(W).\]
Also gibt es, nach \cref{satz4} und \cref{2.6.6c}, eine birationale Abbildung $H\ppf W$.
\end{proof}

\sect{Reguläre Ringe und Krullscher Höhensatz}

\begin{ziel} Sei $I=(f_1,\dotsc,f_k)$, $\hoehe(I):=\{\inf \hoehe(\wp) \mid \wp \supseteq I, \wp$ ist Prmideal\}.

Was kann man über $\hoehe I$ sagen? 
\end{ziel}

\begin{nbem}
Ein Primideal $\wp$ heißt \emph{minimales Primoberideal\index{minimales Primoberideal}} von einem Ideal $I$, wenn $\wp\supseteq I$ und $\wp$ minimal mit dieser Eigenschaft ist. 

Zorns Lemma zeigt, dass es zu jedem $x\in R\setminus R^{\times}$ ein minimales Primoberideal von $(x)$ gibt.
\begin{enumerate}
  \item\label{khil} (Krullsches Hauptideallemma, vgl. Brodmann {\scshape iii}.10.15)
  
   Ist $R$ ein noetherscher Ring, $x\in R\setminus R^{\times}$ und $\wp$ ein minimales Primoberideal von $(x)$, dann gilt $\hoehe \wp \le 1$.
  \item Jeder noetherscher Ring hat nur endlich viele minimale Primideale (vgl. auch Satz~1, Brodmann {\scshape ii}, 9.17 (iii)).
  \item\label{pivl} (Primidealvermeidungslemma, vgl. Brodmann {\scshape iii}.11.10)

  Sind $\wp_1,\dotsc,\wp_n$ Primideale in $R$ und $I$, $J$ Ideale, mit $I\nsubseteq \wp_i$ und $I\nsubseteq J$, dann gilt auch $I\nsubseteq J\cup \wp_1\dotsm\cup \wp_n$. 
\end{enumerate} 
\end{nbem}

\begin{lem}\label{3.5.1}
Sei $R$ noethersch, $q \in \Spec R$, $x\in q$, $\hoehe q \ge l \ge 1$. Dann exisitert eine Primidealkette
\[x\in \wp_0'\subsetneq\dotsm \subsetneq \wp_{l-1}'=q.\]
\end{lem}
\begin{proof} Wir machen vollständige Induktion über $l$:

Für $l=1$ wählen wir $\wp_0'=q$.

Sei nun $l\ge 2$. Da $\hoehe q\le l$ gilt, gibt es eine Primidealkette $\wp_0\subsetneq\dotsm\subsetneq \wp_l:=q$. Ist $x \in \wp_{l-1}$, so kann man die Induktionsvorraussetzung anwenden und erhält die Behauptung.

Ab jetzt sei also $x\notin \wp_{l-1}$. Dann ist $I:=(x) + \wp_{l-2}\subseteq q$. Sei $s$ ein minimales Primoberideal von $I$ mit $I\subseteq s\subseteq q$.

Es gilt $\wp_{l-2}\subsetneq \wp_{l-1}\subsetneq q$, also $\Bar{0}\subsetneq\Bar{\wp_{l-1}}\subsetneq \Bar{q}$ in $\Quotient{R}{\wp_{l-2}}$ und damit ist $\hoehe \Bar{q}\le 2$, und somit ist, nach dem \hyperref[khil]{Krullschen Hauptideallemma}, $\Bar{q}$ \textit{kein} minimales Oberprimideal von $(\Bar{x})$. Damit ist aber auch $q$ kein minimales Primoberideal von $(x)+\wp_{l-2}=I$, also ist $s\neq q$.

Nun können wir also die Induktionsvorraussetzung auf $s$ anwenden, es gibt somit eine Primidealkette $x\in \wp_0'\subsetneq \dotsm\subsetneq \wp_{l-2}$ in $s$. Wenn wir diese durch $q$ um 1 verlängern sind wir fertig.
\end{proof}

\begin{kor}\label{3.5.2}
Sei $(R,\m)$ ein lokaler, noetherscher Ring und $x\in \m$. Dann ist \[\dim \Quotient{R}{(x)}\ge R-1.\]

Vermeidet $x$ die Primideale $\wp_i$ von $R$, d.h gilt $x\notin \wp_i$ für alle minimalen Primideale $\wp_i$, dann gilt $\dim \Quotient{R}{(x)}=R-1$.
\end{kor}

\begin{prop}\label{3.5.3}(Krullscher Höhensatz)
Sei $R$ ein noetherscher Ring, $I=\langle x_1,\dotsc,x_k\rangle$ ein echtes Ideal in R. Dann gilt: $\hoehe \wp\ge k$ für jedes minimale Primideal von $I$. Insbesondere ist $\hoehe I\ge k$.
\end{prop}
\begin{proof} Wir machen Induktion über $k$:

Der Fall $k=1$ folgt aus dem \hyperref[khil]{Krullschen Hauptideallemma}.

Sei also $k\ge 2$: Sei $\wp$ ein minimales Primoberideal von $I$ und $\wp_0\subsetneq \dotsm \subsetneq \wp_l:=\wp$ eine Primidealkette.

Nach \cref{3.5.1} finden wir eine Primidealkette mit $x_{k}\in\wp_0'\subsetneq \dotsm \subsetneq \wp_{l-1}':=\wp$. In $\Quotient{R}{(x_{k})}$ gilt nun $\Bar{\wp_0'}\subsetneq \dotsm \subsetneq \Bar{\wp_{l-1}'}=\Bar{\wp}$ und $\Bar{\wp}$ ist ein minimales Primoberideal von $I=\langle\Bar{x_1},\dotsc,\Bar{x_{k-1}}\rangle$. Mit der Induktionsvoraussetzung folgt nun: 
\[l-1\leq \hoehe \Bar{\wp}\leq k-1.\]
Also ist $l\leq k$ und das zeigt die Behauptung.
\end{proof}

\settowidth\breite{$\{\m_1,\dotsc,\m_n\}$ ist minimalesmmm}
\begin{lem}\label{3.5.4} Sei $(R,\m)$ ein noetherscher, lokaler Ring und $k=\Quotient{R}{\m}$.
\begin{enumerate}
  \item\Label{3.5.4a} Für $\m_1,\dotsc,\m_n$ gilt: 
  \[\begin{minipage}[c]{\breite}\begin{center}%
    $\{\m_1,\dotsc,\m_n\}$ ist minimales\\
    Erzeugendensystem von $\m$%
  \end{center}\end{minipage}%
    \!\!\!\!\!\iff \{\m_1,\dotsc,\m_n\} \text{ ist Basis von }\Quotient{\m}{\m^2}.\]
  \item\Label{3.5.4b} Alle minialen Erzeugendensysteme von $\m$ haben dieselbe Anzahl: $\dim_k \Quotient{\m}{\m^2}$.
  \item\Label{3.5.4c} Es ist $\dim_k \Quotient{\m}{\m^2}\ge \dim R$.
\end{enumerate}
\end{lem}
\begin{proof}
\begin{enumerate}
  \item[\ref{3.5.4a}] Sei zuerst $ \langle\m_1,\dotsc,\m_n\rangle =\m$. Betrachte die Projektionen $\pi_1\colon R\ra k=\Quotient{R}{\m}$, $\pi_2\colon \m\ra \Quotient{\m}{\m^2}$. Dann ist $\Bar{r\m_1}:=\pi_1(r)\pi_2(\m)$ wohldefiniert und $\{\Bar{\m_1},\dotsc,\Bar{\m_n}\}$ ist ein Erzeugendensystem von $\Quotient{\m}{\m^2}$.

  Sei jetzt $\langle\Bar{\m_1},\dotsc,\Bar{\m_n}\rangle=\Quotient{\m}{\m^2}$. Definiere $N=\langle\m_1,\dotsc,\m_n\rangle$. Dann ist $\m=N+\m^2$ und damit $N=\m$ nach dem Nakayama-Lemma aus Algebra 2.
  \item[\ref{3.5.4b}] folgt aus \ref{3.5.4a}.
  \item[\ref{3.5.4c}] Da $R$ lokal ist, folgt $\dim R=\hoehe \m$. Außerdem folgt, mit \cref{3.5.3} und \ref{3.5.4b},
  \[\hoehe \m\le \card{\{\text{minimales Erzeugendensystem von }\m\}}=\dim_{k} \Quotient{\m}{\m^2}.\]
\end{enumerate} 
\end{proof}

\begin{prop}\label{3.5.5}
Ist $(R,\m)$ regulär, so ist $R$ auch nullteilerfrei.
\end{prop}
\begin{proof} Erinnerung aus \cref{3.4.2}: 
\begin{align*}
(R,\m) \text{ ist regulär }& \iff \dim_k \Quotient{\m}{\m^2}=\dim R\\
&\iff \m\text{ kann durch } \dim R\text{ viele Elemente erzeugt werden.}
\end{align*}
Wir beweisen die Aussage nun via Induktion über $d:=\dim R=\dim_{k}\Quotient{\m}{\m^2}$.

Ist $d=0$, so kann $\m$ von $0$ Elementen erzeugt werden, d.h. $\m=\{0\}$ und damit ist $R$ ein Körper.

Sei nun $d\ge 1$ und $\wp_1,\dotsc,\wp_r$ die minimalen Primideale von $R$ (das sind nur endlich viele, vgl. die Bemerkung zu Beginn des Abschnitts.)

Da $\dim R=d>0$ ist, kann $\m$ nicht minimal sein. Also ist $\m \nsubseteq \wp_i$ für alle $i=1,\dots,r$. Außerdem gilt: $\m \nsubseteq \m^2$, da $\dim_{k} \Quotient{\m}{\m^2}>0$.

Mit dem \hyperref[pivl]{Primidealvermeidungslemma} ist folglich $\m\nsubseteq \m^2\cup \wp_1\cup\dotsm \cup \wp_r$.

Wähle nun $x\in \m\setminus (\m^2\cup \wp_1\dotsm\cup \wp_r)$ und ergänze $\Bar{x}$ zu einer Basis $\{\Bar{x},\Bar{x_2},\dotsm,\Bar{x_d}\}$ von $\Quotient{\m}{\m^2}$.

$\pi\colon R\ra \Quotient{R}{(x)}$ bezeichne die kanonische Projektion. Nach \cref{3.5.2} gilt \[\dim \Quotient{R}{(x)}=d-1.\] Ferner gilt für das maximale Ideal $\pi(\m)$ in $\Quotient{R}{(x)}$, nach \cref{3.5.4},
 \[\dim_{k} \Quotient{\pi(\m) }{\pi(\m)^2} \le |\{\text{ Erzeuger von }\pi(\m)\}|\le d-1.\]

Auch ist $\dim \Quotient{\pi(\m)}{\pi(\m)^2} \ge \dim\Quotient{R}{(x)}$, wieder nach \cref{3.5.4}.

Also gilt oben Gleichheit, d.h. $\Quotient{R}{(x)}$ ist regulär. Nach Induktionsvorrausetzung ist damit $\Quotient{R}{(x)}$ nullteilerfrei, d.h. $(x)$ ist Primideal in $R$.

Damit gibt es ein $\wp_i$ mit $(x)\supsetneq \wp_i$. Sei nun $b\in \wp_i$. Also ist $b=a\cdot x$ mit $a\in R$. Da $\wp_i$ Primideal und $x\notin \wp_i$ ist, muss $a\in \wp_i$ gelten. Also ist \[\wp_i=\wp_i x\text{ und damit }\wp_i=\wp_i \m.\]
Nun liefert das Nakayama-Lemma $\wp_i=0$, d.h $R$ ist nullteilerfrei.
\end{proof}

% 31.1.11
\chapter{Nicht-singuläre Kurven}

\sect{Divisoren}

Sei $\CC$ immer eine reguläre, projektive, zusammenhängende Kurve über einem algebraisch abgeschlossen Körper $K$.

\begin{ziel}
Wir möchten die Divisorengruppe basteln und damit das Geschlecht definieren.
\end{ziel}

\begin{dfn}\label{4.1.1}
\begin{enumerate}
\item\Label{4.1.1a} Ein \emph{Divisor\index{Divisor}} auf $\CC$ ist eine formale Summe
\[D=\sum_{i=1}^{k}n_{i}P_{i}\]
mit $k\in\N$, $n_{i}\in\Z$ und $P_{i}\in \CC$. Die Menge
\[\Div\CC:=\{D\mid D\text{ ist Divisor auf }\CC\}\]
ist mit \enquote{$+$} die freie abelsche Gruppe über $\CC$. Sie heißt \emph{Divisorengruppe\index{Divisorengruppe}}.
\item\Label{4.1.1b} Für $\displaystyle D=\sum_{i=1}^{k}n_{i}P_{i}\in\Div\CC$ heißt $\displaystyle\deg D:=\sum_{i=1}^{k}n_{i}$ der \emph{Grad von $D$\index{Grad eines Divisors}}.

$\deg\colon\Div\CC\ra\Z$ ist also ein Gruppenhomomorphismus.
\item\Label{4.1.1c} $D=\sum n_{i}P_{i}$ heißt \emph{effektiv\index{effektiv}}, wenn alle $n_{i}\geq 0$ sind. Wir schreiben dann $D\geq 0$.
\end{enumerate}\end{dfn}

\begin{bem}\label{4.1.2}
\begin{enumerate}
\item\Label{4.1.2a} Divisoren können allgemein auch für irreduzible Varietäten höherer Dimension definiert werden, also als endliche Summen von \enquote{Primdivisoren}, d.h. irreduzible Untervarietäten von Kodimension $1$.
\item\Label{4.1.2b} Im Spezialfall Kurven gilt für die zugehörigen  lokalen Ringe: $\O_{\CC,P}$ ist ein noetherscher lokaler Ring von Dimension $1$ und $\dim_{K}\Quotient{\m_{P}}{\m_{P}^{2}}=1$. Nach Algebra {\scshape ii} gilt also:
\begin{itemize}
\item $(\O_{\CC,P},\m_{P})$ ist ein diskreter Bewertungsring.
\item $(\O_{\CC,P},\m_{P})$ ist ein Hauptidealring.
\item Für alle $x\in\O_{\CC,P}\setminus\{0\}$ gibt es $u\in\O_{\CC,P}^{\times}$ und $n\in\N_{0}$, so dass $x=ut^{n}$, wobei $t$ ein Erzeuger von $\m_{P}$ ist. So ein $t$ nennen wir auch \emph{Uniformisierende\index{Uniformisierende}}.
\item $\nu_{P}\colon K(\CC)^{\times}\ra\Z$, $\frac{f}{g}\mapsto n_{1}-n_{2}$, wobei $f=u_{1}t^{n_{1}}$ und $g=u_{2}t^{n_{2}}$, ist eine diskrete Bewertung.
\end{itemize}\end{enumerate}\end{bem}

\begin{db}\label{4.1.3}
Sei $f\in K(\CC)^{\times}$. Dann gilt:
\begin{enumerate}
\item\Label{4.1.3a} $\ord_{P}f:=\nu_{P}(f)$, mit $\nu_{P}$ wie in \cref{4.1.2b}, heißt \emph{Ordnung von $f$ in $P$\index{Ordnung}}.
\item\Label{4.1.3b} $\displaystyle\div f:=\sum_{P\in\CC}\ord_{P}(f)\cdot P$ heißt \emph{Divisor zu $f$\index{Divisor}}.
\item\Label{4.1.3c} Wir nennen $D\in\Div\CC$ \emph{Hauptdivisor\index{Hauptdivisor}}, wenn es $f\in K(\CC)^{\times}$ gibt, so dass $D=\div f$.
\item\Label{4.1.3d} Die Hauptdivisoren bilden eine Untergruppe $\Divh\CC$ von $\Div\CC$.
\item\Label{4.1.3e} $\displaystyle\Cl(\CC):=\Quotient{\Div\CC}{\Divh\CC}$ heißt \emph{Divisorenklassengruppe\index{Divisorenklassengruppe}}.
\item\Label{4.1.3f} $D,D'\in\Div\CC$ heißen \emph{linear äquivalent\index{linear äquivalent}}, wenn $D-D'\in\Divh\CC$ liegt.

In dem Fall schreiben wir $D\sim D'$ oder $D\cong D'$.
\end{enumerate}\end{db}

\begin{proof}\begin{enumerate}
\item[\ref{4.1.3b}] Wir müssen noch zeigen, dass die Summe wirklich endlich ist. Da $f\neq 0$ gilt:
\begin{align*}
\ord_{P}f\neq 0&\iff\ord_{P}f>0\text{ oder }-\ord_{P}f=\ord_{P}\tfrac{1}{f}>0\\
&\iff P\in\V(f)\text{ oder }P\in\V\bigl(\tfrac{1}{f}\bigr).
\end{align*}
Aber $\V(f)$ und $\V\bigl(\tfrac{1}{f}\bigr)$ sind abgeschlossene echte Teilmenge einer Varietät von Dimension $1$ und damit endlich. Damit gibt es auch nur endlich viele Summanden, die nicht $0$ sind.
\item[\ref{4.1.3d}] Es gilt $\div(f\cdot g)=\div f + \div g$, $\div(1)=0$ und $\div\frac{1}{f}=-\div f$, da $\nu_{P}$ ein Gruppenhomomorphismus ist.
\end{enumerate}\end{proof}

\begin{bsp}\label{4.1.4}
Sei $\CC=\P^{1}$. Dann können wir $f\in K(\CC)^{\times}$ als
\[f=\frac{\displaystyle\prod_{i=0}^{n}(X-a_{i})}{\displaystyle\prod_{j=0}^{m}(X-b_{j})}\]
schreiben und sehen, dass dann für $a\in\CC\setminus\{\infty\}$
\[\ord_{a}f=\card{\{i\in\{1,\dotsc,n\}\mid a_{i}=a\}}-\card{\{j\in\{1,\dotsc,m\}\mid b_{j}=a\}}\]
gilt. Den Punkt $a=\infty$ fassen wir als $(0:1)$ auf und setzen ihn in das homogenisierte Polynom
\[\H(f)=\frac{\displaystyle\prod_{i=0}^{n}(X-a_{i}X_{0})X_{0}^{M-n}}{\displaystyle\prod_{j=0}^{m}(X-b_{j}X_{0})X_{0}^{M-m}},\]
wobei $M:=\max\{m,n\}$ ist, ein und sehen damit, dass
\[\ord_{\infty}f=(M-n)-(M-m)=m-n.\]
Insgesamt sehen wir also:
\begin{align*}
\deg(\div f)=\sum_{a\in\P^{1}}\ord_{a}f&=\card{\{a\mid \exists\, i: a=a_{i}\}}-\card{\{a\mid \exists\, j: a=b_{j}\}}+m-n\\
&=n-m+m-n=0.
\end{align*}
Umgekehrt kann zu jedem Divisor $D$ von Grad $0$ so ein $f$ gefunden werden, mit dem $D=\div f$ ist. Wir erhalten also:
\[\Divh\P^{1}=\{D\in\Div\P^{1}\mid\deg D=0\}=\Kern(\deg).\]
Also ist $\Cl(\P^{1})\cong\Bild(\deg)=\Z$.
\end{bsp}

\begin{db}\label{4.1.5}
Sei $f\in K(\CC)^{\times}$ und $P\in\CC$.
\begin{enumerate}
\item $\ord_{P}f = 0\iff f\in\O_{P}^{\times}\iff f$ ist in $P$ definiert und $f(P)\neq 0$.
\item $\ord_{P}f>0\iff f\in\m_{P}\subseteq\O_{\CC,P}$, d.h. $f$ ist in $P$ definiert und $f(P)=0$.
\item \mbox{$\ord_{P}f<0 \iff f$} kann nicht in $P$ fortgesetzt werden, also ist $\frac{1}{f}$ in $P$ definiert und es gilt $\frac{1}{f}(P)=0$.

In diesem Fall heißt $P$ \emph{Polstelle\index{Polstelle}}.
\item Sei $t$ die Uniformisierende, also $\m_{P}=\langle t\rangle$. Dann gibt es ein $u\in\O_{P}^{\times}$, so dass
\[f=ut^{\ord_{P}f}.\]
\end{enumerate}\end{db}

\begin{prop}\label{4.1.6}
Sei $\CC$ eine reguläre Kurve (nicht notwendigerweise projektiv), $P\in\CC$, sowie $X$ eine projektive Varietät und $f\colon \CC\setminus\{P\}\ra X$ ein Morphismus. Dann existiert ein Morphismus
\[\Bar{f}\colon \CC\ra X,\quad \Bar{f}\restrict{\CC\setminus\{P\}}=f,\]
der $f$ fortsetzt.
\end{prop}

\begin{proof}
Sei $X\subseteq\P^{n}$. Dann ist, ohne Einschränkung, $X\not\subseteq\V(X_{i})$ für alle $i\in\{0,\dotsc,n\}$, denn ansonsten wählen wir $n$ einfach kleiner.

Sei $\U_{i}=\P^{n}\setminus\V(X_{i})$. Dann gilt für $\displaystyle U:=\bigcap_{i=0}^{n}\U_{i}$, dass
\[W:=f^{-1}(U)=\bigcap_{i=0}^{n}f^{-1}(\U_{i})\neq\leer\]
ist und, da offen, damit dicht in $\CC$ liegt, da $f$ als Morphismus stetig ist.

Sei außerdem $h_{ij}=\frac{x_{i}}{x_{j}}\circ f$. Dann ist $h_{ij}$ eine reguläre Funktion auf $W\setminus\{P\}$. Insbesondere definiert jedes $h_{ij}$ ein Element im Funktionenkörper, wir können also $h_{ij}\in K(\CC)^{\times}$ auffassen.

Sei $r_{i}:=\ord_{P}h_{i0}$. Wähle $k$ mit $r_{k}$ minimal. Dann ist
\[\ord_{p}(h_{ik})=\ord_{p}\left(\frac{h_{i0}}{h_{k0}}\right)=r_{i}-r_{k}\geq 0\]
und nach \cref{4.1.5} liegt $P$ somit im Definitionsbereich von $h_{ik}$, d.h. es gibt eine Umgebung $\schlange{W}$ von $P$ mit $h_{ik}\in\O_{\schlange{W}}$.

Insbesondere ist auf $\schlange{W}$, nach gleicher Argumentation, $\ord_{P}h_{kk}=0$, also $h_{kk}(P)\neq 0$.

Nun definieren wir $\Bar{f}$ durch
\[\Bar{f}(x)=\begin{cases}f(x),&x\neq p,\\(h_{0k}(p):\dotsm:h_{nk}(p)),&x=p.\end{cases}\]
Dann ist $\Bar{f}$ ein Morphismus, denn für $x\in\schlange{W}$ gilt
\begin{align*}
f(x)&=\bigl(f_{0}(x):\dotsm f_{n}(x)\bigr)=\bigl(\bigl(x_{0}\circ f\bigr)(x):\dotsm:\bigl(x_{n}\circ f\bigr)(x)\bigr)\\
&=\bigl(\bigl(\tfrac{x_{0}}{x_{k}}\circ f\bigr)(x):\dotsm:\bigl(\tfrac{x_{n}}{x_{k}}\circ f\bigr)(x)\bigr)=\bigl(h_{0k}(x):\dotsm:h_{nk}(x)\bigr).
\end{align*}
Damit ist aber auch $\Bar{f}(p)\in X$, denn ist $X$ ist abgeschlossen.
\end{proof}

Damit gilt auch:
\begin{kor}\label{4.1.7}
Jede rationale Abbildung $\CC\ra X$ für eine projektive Varietät $X$ lässt sich zu einem Morphismus von $\CC$ nach $X$ fortsetzen.
\end{kor}

%2.2.11
\begin{kor}\label{4.1.8}
Sind zwei zusammenhängende reguläre projektive Kurven $\CC_{1}$ und $\CC_{2}$ birational, so sind sie bereits isomorph.
\end{kor}

\begin{proof}
Seien $\Psi_{1}\colon\CC_{1}\ppf\CC_{2}$ und $\Psi_{2}\colon\CC_{2}\ppf\CC_{1}$ mit $\Psi_{1}\circ\Psi_{2}=\id$ und $\Psi_{2}\circ\Psi_{1}=\id$. Dann lassen diese sich nach \cref{4.1.7} zu $\Bar{\Psi_{1}}$, bzw. $\Bar{\Psi_{2}}$ fortsetzen. Damit gilt, jeweils auf einer dichten Teilmenge, $\Bar{\Psi_{1}}\circ\Bar{\Psi_{2}}=\id$ und $\Bar{\Psi_{2}}\circ\Bar{\Psi_{1}}=\id$ und damit, da die Kurven zusammenhängend sind, schon jeweils auf der ganzen Kurve.
\end{proof}

\sect{Verzweigungsindizes} 

In diesem Abschnitte seien $\CC_{1}$ und $\CC_{2}$ reguläre projektive zusammenhängende Kurven und $f\colon\CC_{1}\surj\CC_{2}$ ein surjektiver Morphismus.

\begin{db}\label{4.2.1}
\begin{enumerate}
\item\Label{4.2.1a} Sei $Q\in\CC_{2}$, $P\in f^{-1}(Q)$ und $t$ die Uniformisierende in $Q$, also $\m_{Q}=\langle t\rangle$. Dann heißt
\[e_{P}:=e_{P}(f):=\ord_{P}(t\circ f)=\nu_{P}(t\circ f)\]
\emph{Verzweigungsgrad von $f$ in $P$\index{Verzweigungsgrad}}.
\item\Label{4.2.1b} Wir definieren $f^{*}\colon\Div\CC_{2}\ra\Div\CC_{1}$ durch den Gruppenhomomorphismus
\[Q\mapsto\sum_{P\in f^{-1}(Q)}\!e_{P}\cdot P.\]
\item\Label{4.2.1c} Es gilt $f^{*}(\div g)=\div(g\circ f).$
\item\Label{4.2.1d} $f^{*}$ steigt zu einem Homomorphismus von $\Cl(\CC_{2})$ nach $\Cl(\CC_{1})$ ab.
\end{enumerate}\end{db}

\begin{proof}\begin{enumerate}
\item[\ref{4.2.1a}] Wir zeigen, dass $e_{P}$ nicht von der Wahl von $t$ abhängt: Sei dazu $t'=ut$ mit $u\in\O_{Q}^{\times}$ auch Uniformisierende. Dann gilt
\[\ord_{P}(t'\circ f)=\ord_{P}\bigl((u\circ f)\cdot(t\circ f)\bigr)=0+\ord_{P}(t\circ f),\]
da mit $u$ auch $u\circ f$ eine Einheit ist.
\item[\ref{4.2.1b}] Die Summe ist endlich, denn $f^{-1}(Q)$ ist eine abgeschlossene echte Teilmenge von $\CC_{1}$ und damit endlich.
\item[\ref{4.2.1c}] Es gilt, nach Definition,
\[f^{*}(\div g)=\sum_{P\in\CC_{2}}\ord_{P}(g)\cdot f^{*}(P)=\sum_{P\in\CC_{2}}\ord_{P}(g)\sum_{Q\in f^{-1}(P)}e_{Q}(f)\cdot Q\]
und
\[\div(g\circ f)=\sum_{Q\in\CC_{1}}\ord_{Q}(g\circ f)\cdot Q=\sum_{P\in\CC_{2}}\sum_{Q\in f^{-1}(P)}\ord_{Q}(g\circ f)\cdot Q,\]
da $\CC_{1}$ gerade die Vereinigung der Urbilder von $f$ ist. Es genügt also zu zeigen, dass für $P\in\CC_{2}$ und $Q\in f^{-1}(P)$
\[\ord_{Q}(g\circ f)=\ord_{P}(g)\cdot e_{Q}(f)\]
ist. Es sei also $q:=\ord_{Q}(g\circ f)$. Dann finden wir eine Uniformisierende $t_{Q}\in\O_{\CC_{1},Q}$ und $u_{1}\in\O_{\CC_{1},Q}^{\times}$, so dass $g\circ f=u_{1}\cdot t_{Q}^{q}$. Genauso finden wir für $r:=\ord_{P}(g)$ eine Uniformisierende $t_{P}$ und $u_{2}\in\O_{\CC_{2},P}$ mit $g=u_{2}\cdot t_{P}^{r}$. Außerdem haben wir $s:=e_{Q}(f)=\ord_{Q}(t_{P}\circ f)$, also $u_{3}\in\O_{\CC_{1},Q}^{\times}$ mit $t_{P}\circ f=u_{3}\cdot t_{Q}^{s}$. Nun gilt, mit Hilfe des Einsetzungshomomorphismus,
\begin{align*}
u_{1}\cdot t_{Q}^{q}&=g\circ f = \bigl(u_{2}\cdot t_{P}^{r}\bigr)\circ f=\bigl(u_{2}\circ f\bigr)\cdot \bigl(t_{P}\circ f)^{r}\\
&=\bigl(u_{2}\circ f\bigr)\cdot \bigl(u_{3}\cdot t_{Q}^{s}\bigr)^{r}=\bigl(u_{2}\circ f\bigr)\cdot u_{3}^{r}\cdot t_{Q}^{rs}.
\end{align*}
Da aber auch $u_{2}\circ f$ und $u_{3}^{r}$ Einheiten in $\O_{\CC_{1},Q}$ sind, haben die Ausdrücke die selbe Bewertung und damit ist $q=rs$, wie behauptet.
\item[\ref{4.2.1d}] folgt aus \ref{4.2.1c}, da Hauptdivisoren auf Hauptdivisoren abgebildet werden.
\end{enumerate}\end{proof}

\begin{bsp}\label{4.2.2}
Sei $K=\C$, $\CC=\P^{1}(\C)$ und $f\colon X\mapsto X^{3}$.
\begin{itemize}
\item Für $P=0$ ist $t=X$, also $t\circ f=X^{3}$ und damit $e_{0}=\ord_{0}(t\circ f)=3$.
\item Für $P=a\in\C^{\times}$ ist $t=X-a^{3}$ und damit, mit einer dritten Einheitswurzel $\zeta$,
\[e_{a}=\ord_{a}(X^{3}-a^{3})=\ord_{a}\bigl((X-a)(X-\zeta a)(X-\zeta^{2}a)\bigr)=1,\]
da nur $(X-a)$ keine Einheit ist.
\item Für $P=\infty$ ist $t=\frac{1}{X}$ und damit ist
\[e_{\infty}=\ord_{\infty}\bigl(\tfrac{1}{X^{3}}\bigr)=3=-\ord_{\infty}f.\]
\end{itemize}\end{bsp}

\begin{dfn}\label{4.2.3}
So ein Morphismus $f$ induziert $f^{\sharp}\colon K(\CC_{2})\inj K(\CC_{1})$, wir können also $K(\CC_{1})$ als Körpererweiterung von $K(\CC_{2})$ auffassen. Wir definieren
\[\deg f:=[K(\CC_{1}):K(\CC_{2})].\]
\end{dfn}

\begin{nbem}
Da $\trdeg_{K}(\CC_{1})=\trdeg_{K}(\CC_{2})=1$ ist diese Körpererweiterung algebraisch.
\end{nbem}

\begin{satz}\label{satz7}
Seien $\CC_{1}$ und $\CC_{2}$ zusammenhängende reguläre projektive Kurven und $f\colon\CC_{1}\surj\CC_{2}$ ein surjektiver Morphismus, dann gilt:
\begin{enumerate}
\item\Label{s7a} Für $Q\in\CC_{2}$ ist $\!\!\displaystyle\sum_{P\in f^{-1}(Q)}\!\!\!e_{P}(f)=n:=\deg(f)$.
\item\Label{s7b} Für jeden Divisor $D$ auf $\CC_{2}$ gilt $\deg(f^{*}D)=\deg(D)\cdot\deg(f).$
\end{enumerate}\end{satz}

\begin{nbem}
Die Aussage $\ref{s7b}$ folgt direkt aus $\ref{s7a}$, denn sei $D:=\sum n_{i}P_{i}$, dann ist
\[\deg(f^{*}D)=\deg\biggl(\sum_{i=1}^{k}n_{i}\sum_{P\in f^{-1}(Q)}\!e_{P}P\biggr)=\sum_{i=1}^{k}n_{i}\sum_{P\in f^{-1}(Q)}\!e_{P}=\deg(D)\deg(f).\]
\end{nbem}

Der \hyperlink{bews7}{{\scshape Beweis}} von \ref{s7a} kommt später.

\begin{kor}\label{4.2.4}
Sei $\CC$ eine projektive, zusammenhängende, reguläre Kurve. Dann gilt:
\begin{enumerate}
\item\Label{4.2.4a} Alle Hauptdivisoren auf $\CC$ haben Grad $0$.
\item\Label{4.2.4b} Die Abbildung $\deg\colon\Cl(\CC)\ra\Z$, $[D]\mapsto\deg D$ ist wohldefiniert.
\end{enumerate}\end{kor}

\begin{proof}
\begin{enumerate}
\item[\ref{4.2.4a}] Sei $f\in K(C)^{\times}$. Dann lässt $f$ sich nach \cref{4.1.7} zu einem Morphismus von $\CC$ nach $\P^{1}$ fortsetzen. Dann gilt
\[\deg(\div f)=\sum_{P\in\CC}\ord_{P}(f)=\!\!\!\sum_{P\in f^{-1}(0)}\!\!\ord_{P}(f)+\!\!\!\sum_{P\in f^{-1}(\infty)}\!\!\ord_{P}(f),\]
da nur Null- und Polstellen von Null verschieden Ordnung haben.

Wie in \cref{4.2.2} wählen wir $t=X$ als Uniformisierende in $0$ und $t=\frac{1}{X}$ als Uniformisierende in $\infty$. Damit ist, für $P\in f^{-1}(0)$,
\[e_{P}=\ord_{P}(X\circ f)=\ord_{P}(f)\]
und, für $P\in f^{-1}(\infty)$,
\[e_{P}=\ord_{P}\bigl(\tfrac{1}{x}\circ f\bigr)=\ord_{P}\bigl(\tfrac{1}{f}\bigr)=-\ord_{P}(f).\]
Insgesamt erhalten wir so, nach \hyperref[4.2.1b]{Definition von $f^{*}$} und mit Hilfe von \cref{s7b},
\[\deg(f)=\!\!\!\!\!\!\sum_{P\in f^{-1}(0)}\!\!\!\!\!e_{P}-\!\!\!\!\!\!\!\!\sum_{P\in f^{-1}(\infty)}\!\!\!\!\!\!e_{P}=\deg\bigl(f^{*}\bigl((0)-(\infty)\bigr)\bigr)=\deg(f)\cdot \deg\bigl((0)-(\infty)\bigr)=0,\]
wobei $(0)$ bzw. $(\infty)$ die Divisoren sind, bei denen $n_{0}=1$ bzw. $n_{\infty}=1$ und alle anderen $n_{P}=0$ sind.
\item[\ref{4.2.4b}] folgt sofort aus \ref{4.2.4a} mit \cref{4.1.3e}.
\end{enumerate}
\end{proof}

\settowidth\breite{Für jedes Primideal $\wp\neq 0$ ist $R_{\wp}$}
\begin{nerinnerung}
Aus Algebra {\scshape ii} wissen wir, dass für einen nullteilerfreien Ring $R$, der kein Körper ist, gilt:
\begin{align*}
R\text{ ist ein Dedekindring }&\iff R\text{ ist $1$-dimensional und normal}\\
&\iff\begin{minipage}[c]{\breite}\begin{center}%
Für jedes Primideal $\wp\neq 0$ ist $R_{\wp}$\\ein diskreter Bewertungsring.%
\end{center}\end{minipage}
\end{align*}
\end{nerinnerung}

\begin{bem}\label{4.2.5}
Sei $\CC$ eine affine Varietät. Dann ist $\CC$ genau dann eine zusammenhängende reguläre Kurve, wenn $K[\CC]$ ein Dedekindring ist.
\end{bem}

%\begin{proof}

\begin{bem}\label{4.2.10}
Sei $V$ irreduzible projektive Varietät in $\P^n$.
\begin{enumerate}
  \item\Label{4.2.10a} Sind $P_1,\dotsc,P_{N+1}$ endlich viele Punkte, so liegen sie in einer offenen, affinen Teilmenge $U$  von $V$.
  \item\Label{4.2.10b} Es gibt ein $v\in K(V)$ mit $v\notin \O_{N+1}, v\in \O_i$ für  $i\in \{1,\dotsc,N\}$.
\end{enumerate}
\end{bem}
\begin{proof}
\begin{enumerate}
  \item[\ref{4.2.10a}] Nach LA gibt es ein lineares $F \in K[X_0,\dotsc,X_n]$ mit $F(p_i)\neq 0\; (\forall i)$. 

Nach Koordinatenwechsel ist $F=X_0$ und $\U_0=\P^n\setminus \V(F)$. Also sind \[P_1,\dotsc,P_{N+1}\in U:=\U_0\cap V\] und dies ist eine affine Varietät.
  \item[\ref{4.2.10b}] Sei $U$ wie in \ref{4.2.10a} mit $P_1,\dotsc,P_{N+1}\in U$. Wähle $h\in \O(U)=K[U]$ mit \[h(P_1),\dotsc,h(P_N)\neq 0\text{ mit }h(P_{N+1})=0.\] Das geht nach dem \hyperref[pivl]{Primidealvermeidungslemma}. Nun tut $\frac{1}{h}$ das Gewünschte.
 \end{enumerate}
\end{proof}

\begin{lem}\label{4.2.6}
Sei $f\colon\CC_{1}\surj\CC_{2}$ ein surjektiver Morphismus zwischen projektiven regulären zusammenhängenden Kurven. Dann gilt:

%\begin{center}
\hspace*{\stretch{1}}Wenn $V\subseteq\CC_{2}$ offen und affin ist, dann ist $f^{-1}(V)$ offen und affin in $\CC_{1}$.\hspace*{\stretch{1}}
%\end{center}
\end{lem}

\begin{proof}\begin{prooflist}
\item Zuerst konstruieren wir ein potentielles $f^{-1}(V)=:\schlange{V}$. Dazu betrachten wir
\[B:=K[V]\inj K(\CC_{2}) \inj[$f^{*}$] K(\CC_{1})\]
und bezeichnen den ganzen Abschluss von $B$ in $K[\CC_{1}]$ mit $A$. Aus Algebra {\scshape ii} wissen wir, dass $A$ dann ein endlich-erzeugter $B$-Modul ist (Algebra {\scshape ii}, Satz 15, bzw. Shafarevich {\scshape ii}.5, Thm. 4). $A$ ist also eine endlich-erzeugte $K$-Algebra und nullteilerfrei, da $A\subseteq K(\CC_{1})$. Es gibt also ein affines $\schlange{V}$ mit $A=K[\schlange{V}]$ und
\[K(\schlange{V})=\Quot(A)=K(\CC_{1}).\]
Nach \cref{satz4} ist $\schlange{V}$ somit birational zu $\CC_{1}$. Außerdem ist $\schlange{V}$ eine reguläre irreduzible Kurve, da $A=K[\schlange{V}]$ ein Dedekindring ist. Nach \cref{4.1.8} ist der Abschluss $\Bar{\schlange{V}}$ isomorph zu $\CC_{1}$, wir können $\schlange{V}$ also als Teilmenge von $\CC_{1}$ auffassen.
\item Zeige nun: $\schlange{V}=f^{-1}(V)$

Angenommen es gäbe $P_0\in \CC_1\setminus \schlange{V}$ mit $f(P_0)=Q\in V$. Seien $P_1,\dotsc,P_k$ alle Urbilder von $Q=f(P_0)$, die auch in $\schlange{V}$ liegen. 
%
Nach \cref{4.2.10b} kann man nun ein $v\in K(\CC_1)$ mit $v\notin \O_{P_0}$ und $v\in \O_{P_i} \; \forall i\in \{1,\dotsc,k\}$ wählen.
\begin{enumerate}
\item Wir zeigen: Man kann ohne Einschränkung annehmen, dass $v$ keine Polstelle in $\schlange{V}$ hat.

Ist nämlich $x\in \schlange{V}$ eine Polstelle, so setzt man $y=f(x)+Q$. Wir wählen nun $h\in B=K[V]$ mit $h(Q)\neq 0$, $h(y)=0$, d.h. $h\in \m_y^v\setminus \m_Q^v$. 

Für $v':=v\dotsc (h\circ f)$gilt damit:
\[\ord_x v'=\ord_x (v)+\ord_x (h\circ f)\ge \ord_ x (v) +1,\]
da $f(x)=y$ Nullstelle in $h$ ist.

Außerdem gilt $\ord_{P_i} v'=\ord_{P_i} v+0$ und es sind keine neuen Pole in $\schlange{V}$ entstanden, da $h$ auf ganz $V$ regulär ist. Durch mehrmaliges Anwenden dieses Verfahrens kann man alle Polstellen entfernen.
\item \hypertarget{ganzpolstelle}{Somit ist $v$ nun aus $A=K[V]$ und damit ganz über $B$.}

 Also gibt es $b_0,\dotsc,b_{n-1} \in B$ mit \[v^n+b_{n-1}v^{n-1}+\dotsm+b_0=0,\]d.h. $v=-b_{n-1}-\frac{b_{n-2}}{v}-\dotsm-\frac{b_0}{v^{n-1}}$.

Da $v\notin \O_{P_0}$ gilt, ist die linke Seite nicht in $P_0$ definiert, aber es ist $\frac{1}{v}\in \O_{P_0}$. Demnach ist die rechte Seite in $P_0$ definiert, da $b_i\circ f$ auf ganz $f^{-1}(V)$ regulär ist, was ein Widerspruch ergibt.
\end{enumerate}
\end{prooflist}\end{proof}

\begin{w} Ab jetzt sei stets $V$ eine affine Umgebung $Q$, also ist nach \cref{4.2.6} $\schlange{V}=f^{-1}(V)$ affin.

Außerdem sei $B=K[V]$, $A=K[\schlange{V}]$ ist dann der ganze Abschluss von $B$ in $K(\CC_1)$.
\end{w}

\begin{lem}\label{4.2.8} Sei $\schlange{\O}:=\displaystyle \bigcap_{i=1}^{k} \O_{P_i} \subseteq K(\CC_1)$. Dann gilt:
  \begin{enumerate}
  \item\Label{4.2.8a} $\schlange{\O}$ ist Hauptidealring.
  \item\Label{4.2.8b} Es gibt $t_1,\dotsc,t_k \in \schlange{\O}$ mit $\ord_{P_i} (t_j)=\delta_{ij}$.
  \item\Label{4.2.8c} Jedes $v\in \schlange{\O}$ lässt sich eindeutig schreiben als \[v=u\cdot t_1^{e_1}\dotsm t_k^{e_k}\] mit $u\in \schlange{\O}^{\times}$, $e_i=\nu_{P_i} (v)$.
  \end{enumerate}
\end{lem}
\begin{proof} \begin{enumerate}
  \item[\ref{4.2.8b}] Sei $\m_i=\m_{P_i}^{\schlange{V}}\subseteq A$ und $\schlange{t_i}$ Uniformisierende, d.h. $\m_{P_i}=(\schlange{t_i})\subseteq \O_{P_i}$.
Wie im Beweis von \cref{4.2.6} kann $\schlange{t_i}$ ohne Einschränkung als regulär vorrausgesetzt werden.

Um die $t_i$ zu konstruieren, wählen wir zuerst $g_i\in A=K[\schlange{V}]$ mit: 
\[g_i(P_i)\neq 0 \text{ und } g_i(P_j)=0\; \text{ wobei }i,j\in \{1,\dotsc,k\}.\] 
Sei $t_1=\schlange{t_1}+\displaystyle \sum_{j=2}^{k} \alpha_j \cdot g_j^2$ mit $\alpha_j \in K$, $\displaystyle\alpha_j \neq \frac
{-\schlange{t_1}(P_j)}{(g_j(P_j))^2}$.

Dann ist $t_1(P_j)=\schlange{t_1}(P_j)+\alpha_j(g_j(P_j))^2\neq 0$, $t_1(P_1)=0$ und 
\[\schlange{t_1}+\displaystyle \sum_{j=2}^{k} \alpha_j \cdot g_j^2 \in \m_{P_1} \setminus \m_{P_1}^2,\]
 da die $g_j\in \m_{P_1}^2$ und $\schlange{t_1} \in \m_{P_1}\setminus \m_{P_1}^2$ sind. 

Also ist $v_{P_1}(t_1)=1$, d.h. $t_1$ tut Gewünschtes. Analog konstruiert man $t_2,\dotsc,t_k$.
  \item[\ref{4.2.8c}] folgt aus \ref{4.2.8b}, denn wir setzen 
  \[u=\frac{v}{t_1^{\ord P_1(v)}\dotsm t_k^{\ord P_k(v)}}\] und sehen, dass $\ord P_i(u)=0$ (für jedes $i$) ist, also ist $u\in \schlange{\O}$.
  \item[\ref{4.2.8a}] folgt aus \ref{4.2.8c}: Sei $I$ ein Ideal in $\schlange{\O}$, dann ist \[I=(t_1^{e_1}\dotsm t_k^{e_k}),\] wobei $e_i=\inf\{\ord_{P_i} (v) \mid v\in I\}$.
\end{enumerate}
\end{proof}

\begin{lem}\label{4.2.9} Es gilt:
  \begin{enumerate}
  \item\label{4.2.9a} $\schlange{\O}=A\cdot \O_Q=\{a\cdot (h\circ f) \mid a\in A_i, h\in \O_Q\}$
  \item\label{4.2.9b} $\schlange{\O}$ ist ein freier $\O_Q$-Modul vom $\Rang n=\deg f$. Dabei fasst man wiederum $\O_Q$ via $f^*$ als Teilring von $\schlange{\O}$ auf.
  \end{enumerate}
\end{lem}

\begin{proof}
\begin{enumerate}
\item[\ref{4.2.9a}] Seien $w\in \schlange{\O}$, $x_1,\dotsc,x_r$ die Polstellen von $w$ und $y_1,\dots,y_r$ ihre Bilder. Seien weiterhin $l_i=\ord_{x_i} (w)$ und $-n=\min\{l_1,\dotsc,l_r\}$.

Wir wählen $h'\in B$ mit $h'(y_i)=0$ und $h'(Q)\neq 0$ und setzen $h=(h')^N$. 

Dann ist $a:=w\cdot (h\circ f)\in A$ und $h\in \O_Q^{\times}$. Folglich ist $w=a\cdot (h\circ f)^{-1}$.
\item[\ref{4.2.9b}] $A$ ist der ganze Abschluss von $B$ in $K(\CC_1)$ und damit endlich erzeugt als $B$-Modul (vgl. \cref{4.2.6}). Nach \ref{4.2.9a} ist $\schlange{\O}$ endlich erzeugt als $\O_Q$-Modul.

Weiter ist $\O$ torsionsfrei, d.h. $\schlange{\O}$ ist ein freier $\O_Q$-Modul (Hauptsatz über Moduln von Hauptidealringen, siehe z.B. Bosch).

Ferner ist $\Rang_{\O_Q} (\schlange{\O}) \le \dim_{K(\CC_2)} K(\CC_1)$, da $K(\CC_2)=\Quot (\O_Q)$ gilt.

Sei $\{\alpha_1,\dotsc,\alpha_n\}$ eine Basis von $\Quotient{K(\CC_1)}{K(\CC_2)}$, und $l$ die maximale Polstellenordnung in den $P_i$'s. Dann sind $\alpha_1\cdot t^l,\dotsc,\alpha_n\cdot t^l$ linear unabhängig und in $\schlange{\O}$. 
Damit ist $\Rang (\schlange{\O})\ge n$, d.h. $\schlange{\O}$ ist frei vom Rang $n$. 
\end{enumerate}
\end{proof}
\begin{proof}[Beweis von \cref{s7a}] \hypertarget{bews7}{Zu} zeigen ist: \[n=\deg (f)=\sum_{i=1}^{k} e_{P_i} = \sum_{i=1}^{k} \ord_{P_i} (t\circ f),\] wobei $(t)=\m_Q$.

Da $t\circ f\in \schlange{\O}$ liefert \cref{4.2.8}: $t\circ f= u\cdot t_1^{e_{P_1}}\dotsm t_k^{e_{P_k}}$, wobei $u \in \schlange{\O}$. Mit dem Chinesischen Restsatz folgt
\[\Quotient{\schlange{\O}}{(t\circ f)}\cong \bigoplus_{i=1}^{k} \Quotient{\schlange{\O}}{(t_i^{e_{P_i}})}\cong \bigoplus_{i=1}^{k} K^{e_{P_i}}.\]
Also ist $\dim_K \Quotient{\schlange{\O}}{(t\circ f)}=\displaystyle \sum_{i=1}^{k} e_{P_i}$.

Andererseits ist $\O \cong \O_Q^n\ra (\Quotient{\O_a}{(t)})^n\cong K^n$. 

Damit gilt $\Quotient{\schlange{\O}}{(t\cdot \schlange{\O})}=\Quotient{\schlange{\O}}{(t\circ f)}\cong K^n$ und damit ist $\dim_K \Quotient{\O}{(t\circ f)}=n$.
\end{proof}
%7.2.11
\sect{Das Geschlecht einer Kurve}

Sei $\CC$ immer eine nicht-singuläre, zusammenhängende, projektive Kurve.

\begin{db}\label{4.3.1}
\begin{enumerate}
\item\Label{4.3.1a} Sei $D$ ein Divisor. Dann nennen wir
\[\L(D):=\{f\in K(\CC)^{\times}\mid D+\div(f)\text{ ist \hyperref[4.1.1c]{effektiv}}\}\cup\{0\}\]
den \emph{Riemann-Roch-Raum von $D$\index{Riemann-Roch-Raum}}. $\L(D)$ ist ein $K$-Vektorraum, da für $P\in\CC$ immer
\[\ord_{P}(f+g)\geq\min\{\ord_{P}(f),\ord_{P}(g)\}\text{ gilt.}\]
\item\Label{4.3.1b} Wir setzen $\ll(D):=\dim_{K}\L(D)$.
\item\Label{4.3.1c} Für einen Divisor $D=\sum n_{P} P$ nennen wir $\{P\in\CC\mid n_{P}\neq 0\}$ den \emph{Träger von $D$\index{Träger}}.
\end{enumerate}\end{db}

\begin{bem}\label{4.3.2}
\begin{enumerate}
\item\Label{4.3.2a} Es gilt $\L(0)=K$ und $\ll(0)=1$.
\item\Label{4.3.2b} Ist $\deg D<0$, so gilt schon $\L(D)=\{0\}$ und $\ll(D)=0$.
\item\Label{4.3.2c} Ist $D\sim D'$, so gilt $\ll(D)=\ll(D')$.

Insbesondere ist $\ll$ somit auf $\Cl(\CC)$ wohldefiniert.
\end{enumerate}\end{bem}

\begin{proof}\begin{enumerate}
\item[\ref{4.3.2a}] Nach \cref{s5a} sind die regulären Funktionen alle konstant.
\item[\ref{4.3.2b}] Nach \cref{4.2.4} ist $\deg(\div (f))=0$ und damit ist der Grad von $D+\div(f)$ kleiner als $0$ und somit ist der Divisor für kein $f$ effektiv.
\item[\ref{4.3.2c}] Sei $g\in K(\CC)^{\times}$ mit $D'=D+\div g$. Dann gilt
\[\div f+D'\geq 0\iff 0\leq \div f+\div g+D=\div(fg)+D,\]
also erhalten wir einen Vektorraumisomorphismus durch
\[\L(D')\ra\L(D),\quad f\mapsto fg,\]
und damit sind die Dimensionen der beiden Räume gleich.
\end{enumerate}\end{proof}

\begin{satz}[Riemann]\label{satz8}\begin{enumerate}
\item\Label{s8a} Ist $D\in\Div\CC$ mit $\deg D\geq-1$, so gilt:
\[\ll(D)\leq\deg D+1.\]
\item\Label{s8b} Es gibt ein $\gamma\in\N$, so dass für alle $D\in\Div\CC$
\[\deg D+1-\gamma\leq\ll(D).\]
\end{enumerate}\end{satz}

\begin{dfn}\label{4.3.3}
Das kleinste $\gamma$ für das \cref{s8b} erfüllt ist, nennen wir das \emph{Geschlecht von $\CC$\index{Geschlecht}}. Wir schreiben auch $\g(\CC)$ oder $\g$.
\end{dfn}

\begin{bem}\label{4.3.4}
Ist $\CC\cong\CC'$, so ist $\g(\CC)=\g(\CC')$, da schon die Divisorengruppen gleich sind.
\end{bem}

\begin{lem}\label{4.3.5}
Seien $D\in\Div(\CC)$ und $P_{0}\in\CC$. Dann ist
\[\L(D)\subseteq\L(D+P_{0})\text{ und }\ll(D+P_{0})\leq\ll(D)+1.\]
\end{lem}

\begin{proof}
Sei $D:=\sum\limits_{P\in\CC}n_{P}P$.

Dass $\L(D)\subseteq\L(D+P_{0})$ ist klar, denn für $f\in\L(D)$ gilt $\ord_{P_{0}}(f)\geq-n_{0}:=-n_{P_{0}}$ und damit liegt $f$ insbesondere in $\L(D+P_{0})$. Wir sehen sogar, dass für 
\[f\in\L(D+P_{0})\setminus\L(D)\text{ dann }\ord_{P_{0}}(f)=-(n_{0}+1)\] 
gelten muss.

Um die zweite Aussage einzusehen, sei $f_{1},\dotsc,f_{r}$ eine Basis von $\L(D+P_{0})$, wobei $f_{1},\dotsc,f_{s}\notin\L(D)$ und $f_{s+1},\dotsc,f_{r}\in\L(D)$. Sei $t$ ein Erzeuger von $\m_{P_{0}}$. Dann finden wir für $i\in\{1,\dotsc,s\}$ jeweils $u_{i}\in\O_{\CC,P_{0}}^{\times}$ mit 
\[f_{i}=u_{i}t^{-(n_{0}+1)}.\]
Damit definieren wir $g_{i}:=u_{i}(P_{0})\cdot f_{1}-u_{1}(P_{0})\cdot f_{i}$ und sehen, dass damit
\[g_{i} = t^{-(n_{0}+1)}\cdot\bigl(u_{i}(P_{0})\cdot u_{1}-u_{1}(P_{0})\cdot u_{i}\bigr),\text{ wobei }u_{i}(P_{0})\cdot u_{1}-u_{1}(P_{0})\cdot u_{i}\in\m_{P_{0}},\]
da der Ausdruck in $P_{0}$ verschwindet. Damit ist aber $\ord_{P_{0}}(g_{i})\geq-n_{0}$, also sind die $g_{i}$ schon aus $\L(D)$.

Nun sind aber $g_{2},\dotsc,g_{s},f_{s+1},\dotsc,f_{r}$ linear unabhängig und in $\L(D)$, es gilt also, wie behauptet,
\[\ll(D)\geq r-1=\ll(D+P_{0})-1.\]
\end{proof}

\begin{proof}[Beweis von \cref{satz8}]
\begin{enumerate}
\item[\ref{s8a}] Wir zeigen $\ll(D)\leq \deg(D)+1$ durch vollständige Induktion über $\deg D=:d$.

Für $\deg D=-1$ gilt nach \cref{4.3.2b} $\ll(D)=0$, wir beschränken uns demnach auf $\deg D\in\N_{0}$.

Sei also zuerst $d=0$ und seien $f,g\in\L(D)$. Dann ist $\div(f)+D\geq 0$ und es gilt sogar $\div(f)+D=0$, da beide von Grad $0$ sind. Gleiches gilt für $g$ und damit erhalten wir
\[\div f=-D=\div g.\]
Damit ist auch $\div\bigl(\frac{f}{g}\bigr)=0$ und $\frac{f}{g}$ somit, nach \cref{s5a}, konstant, da $\frac{f}{g}$ hier schon regulär ist.

Sei nun $d\geq 1$. Wir schreiben $D=\sum n_{i}P_{i}$ und wählen ein $P_{i}$ mit $n_{i}>0$. Für $D':=D-P_{i}$ ist dann $\deg D'=\deg D-1$ und nach \cref{4.3.5} gilt, zusammen mit der Induktionsvoraussetzung,
\[\ll(D)=\ll(D'+P_{i})\leq\ll(D')+1\leq d+1.\]
\item[\ref{s8b}] Wir setzen $s(D):=\deg D+1-\ll(D)$ und zeigen, dass es ein $\gamma\in\N$ mit $s(D)\leq\gamma$, für alle $D\in\Div\CC$, gibt.

Wir erinnern uns, dass nach \cref{4.3.2c} und \cref{4.2.4b}
\hypertarget{bews8e1}{\[D\sim D'\implies s(D)=s(D')\]}
gilt. Außerdem überlegen wir uns, dass es für $D=\sum n_{P}P$ und $D'=\sum n'_{P}P$ mit $D'\leq D$ Punkte $P_{1},\dotsc,P_{k}$ mit $n'_{P_{i}}\leq n_{P_{i}}$ gibt, an allen anderen Stellen sind sie gleich. \cref{4.3.5} liefert dann iterativ, dass 
\hypertarget{bews8e3}{\[\ll(D)\leq\ll(D')+\sum_{i=1}^{k}(n_{P_{i}}-n'_{P_{i}})=\ll(D')+\deg D-\deg D'.\]}
Das bedeutet aber gerade, dass \hypertarget{bews8e2}{hier $s(D')\leq s(D)$ ist.} Wir zeigen nun:
\begin{prooflist}
\item\label{bews8i} Für alle $D\in\Div\CC$ gibt es $D'\in\Div\CC$ mit $D\sim D'$, so dass $D'\leq k\cdot N$, wobei $k\in\N$ und $N$ für $f\in K(\CC)\setminus K$ der \emph{Nullstellendivisor\index{Nullstellendivisor}} $f^{*}0$ ist.
\item\label{bews8ii} Es gibt ein $\gamma\in\N$, so dass für alle $k\in\N$ damit $s(k\cdot N)\leq\gamma$ gilt.
\end{prooflist}
Dann folgt die Behauptung, denn für alle $D\in\Div\CC$ gibt es nach \ref{bews8i} und \hyperlink{bews8e1}{obiger Überlegung} ein $D'$ mit $s(D)=s(D')$ und nach der \hyperlink{bews8e2}{anderen Überlegung} gilt mit \ref{bews8ii} schon
\[s(D')\leq s(k\cdot N)\leq\gamma.\]
Wir zeigen zuerst Behauptung \ref{bews8i}: Dazu sei wieder $D:=\sum n_{P}P$. Wir suchen also ein $g\in K(\CC)\setminus K$ mit $D+\div g\leq k\cdot N$.

Insbesondere heißt das für $g$, dass alle Nullstellen von $g$ im Träger von $N$ liegen sollten und dass für $n_{P}>0$ für ein $P$, das nicht im Träger von $N$ liegt, \[\ord_{P}(g)\leq-n_{P}\] gelten muss. Dabei ist $P$ genau dann im Träger von $N$, wenn $\ord_{P}(f)>0$ ist, also $f$ da eine Nullstelle hat.

Seien $P_{1},\dotsc,P_{r}$ die Punkte in $\CC$ mit $n_{P_{i}}>0$ und $\ord_{P_{i}}(f)\leq 0$. Sei
\[h_{i}:=\frac{1}{f}-\frac{1}{f}(P_{i})\quad\text{für }i\in(1,\dotsc,r).\]
Dann ist $\ord_{P_{i}}(h_{i})\geq 1$ und für alle $P_{j}$ mit $\ord_{P_{j}}(f)\leq 0$ ist $\ord_{P_{j}}(h_{i})\geq 0$. Da die Ordnung von einer Bewertung herkommt, gilt auch
\[\ord_{P_{i}}(h_{i}^{-n_{P_{i}}})\leq-n_{P_{i}}\text{ und }\ord_{P_{j}}(h_{i}^{-n_{P_{i}}})\leq 0\]
für die entsprechenden Punkte. Die $h_{i}^{-n_{P_{i}}}$ haben also sicherlich keine Nullstellen außerhalb des Trägers von $N$,
\[g:=\prod_{i=1}^{r}h_{i}^{n_{P_{i}}}\]
erfüllt also alle unsere Wünsche. Nun können wir unser $k$ so wählen, dass für $P$ in dem Träger von $N$ immer 
\[n_{P}+\ord_{P}(g)\leq k\cdot e_{P}(f)\]
gilt, da es sich dabei nur um endlich viele Punkte handelt. Und damit gilt, wie behauptet
\[D+\div g\leq k\cdot N.\]

Nun zeigen wir noch Behauptung \ref{bews8ii}, also dass es ein $\gamma\in\N$ gibt, so dass
\[s(k\cdot N)=\deg(k\cdot N)+1-\ll(k\cdot N)\leq\gamma.\]
Sei also $f\in K(\CC)^{\times}$ und $g_{1},\dotsc,g_{r}$ eine Basis von $K(\CC)$ über $K(f)=K(\frac{1}{f})$, also $r=\deg f$. Ohne Einschränkung können wir die $g_{i}$ ganz über $K[\frac{1}{f}]$ wählen.

Dann gilt nach ähnlicher Argumentation wie \hyperlink{ganzpolstelle}{im Beweis zu} \cref{4.2.6}: Wenn $P$ eine Polstelle von $g_{i}$ ist, so ist $P$ schon eine Polstelle von $\frac{1}{f}$. Damit ist $P$ aber eine Nullstelle von $f$ und liegt somit im Träger von $N$. Wir finden also ein $\gamma_{0}$, so dass, für $i\in\{1,\dotsc,r\}$,
\[\div g_{i}+\gamma_{0}\cdot N\geq 0\]
gilt. Damit zeigen wir nun, dass $\ll(k\cdot N)\geq \deg(k\cdot N)-r\cdot(\gamma_{0}-1)$, wir also \[\gamma:=r(\gamma_{0}-1)+1\] 
finden. Sei dazu $h_{ik}:=\frac{g_{i}}{f^{j}}$, für $j\in\{0,\dotsc,k\}$. Damit gilt
\[\div h_{ij} + (k+\gamma_{0})\cdot N=\div g_{i}-j\cdot \div f +k\cdot N +\gamma_{0}\cdot N\leq (k-j)\cdot N\geq 0,\]
da $\div f\leq N$ und $\div g_{i}+\gamma_{0}\cdot N\geq 0$ ist. Damit liegen die $h_{ij}$ alle in $\L((k+\gamma_{0})\cdot N)$ und, da die $h_{ij}$ über $K$ linear unabhängig sind, ist
\[\ll((k+\gamma_{0})\cdot N)\geq r\cdot (k+1).\]
Wenn wir \cref{4.3.5} iterativ anwenden, sehen wir, \hyperlink{bews8e3}{ähnlich wie oben}, dass
\[\ll(k\cdot N)\geq\ll((k+\gamma_{0})\cdot N)-\deg(\gamma_{0}\cdot N)\geq r\cdot (k+1)-\gamma_{0}r=kr-r\cdot(\gamma_{0}-1),\]
da, nach \cref{s7a}, $\deg N=\deg f=r$ ist und damit folgt auch schon die Behauptung, denn dann ist auch $kr=\deg(k\cdot N)$.
\end{enumerate}
\end{proof}

\sect{Der Satz von Riemann-Roch}

$\CC$ sei stets eine zusammenhängende, reguläre, projektive Kurve, $K$ algebraisch abgeschlossen.

\begin{db}\label{4.4.1} Sei $\Omega_{\CC}=\Omega_{\Quotient{K(\CC)}{K}}$ der $K(\CC)$-Vektorraum der $K$-Differentiale von $K(\CC)$ (siehe auch Algebra {\scshape ii}).
\begin{enumerate}
 \item\Label{4.4.1a} $\Omega_{\CC}$ heißt auch Vektorraum der \emph{rationalen Differentiale\index{rationales Differential}}.
 \item\Label{4.4.1b} Es ist $\dim_{K(\CC)} \Omega_{\CC}=1$.
\end{enumerate}
\end{db}
\begin{proof} \begin{enumerate}
\item[\ref{4.4.1b}] $\CC$ ist birational zu einer Hyperfläche $\V(F)\subseteq \A^2(K)$ nach \cref{3.4.4} aus \cref{kap3}. Damit ist $K(\CC)=K(\Bar{X},\Bar{Y})$ und $F(\Bar{X},\Bar{Y})=0$. 

Außerdem wird $\Omega_{\CC}$ von $\d\Bar{X}, \d\Bar{Y}$ erzeugt und die notwendige Bedingung 
\[\d F(\Bar{X},\Bar{Y})=0\text{ erzwingt }\frac{\d F}{\d X} \d \Bar{X}+\frac{\d F}{\d Y} \d \Bar{Y}=0.\]
Der Lösungsraum dieses linearen Gleichungssystems ist eindimensional. \end{enumerate}
\end{proof}
\begin{db}\label{4.4.2}  Sei $\omega \in \Omega_{\CC}$, $\omega\neq 0$. 
  \begin{enumerate}
  \item\Label{4.4.2a} Sei $P\in \CC$, $t_P$ uniformisierend. Dann ist $\omega=f\cdot\d t_P$ und \[\ord_P (\omega):=\ord_P (f)\] ist wohldefiniert.
  \item Man definiert $\div \omega:= \sum\limits_{P\in \CC}^{} \ord_P (\omega) \cdot P$.
  \item $\K$ heißt \emph{kanonischer Divisor\index{kanonischer Divisor}}, wenn $\K=\div \omega$ für ein $\omega\in \Omega_{\CC}$ gilt.
  \item Je zwei kanonische Divisoren sind linear äquivalent.
  \end{enumerate}
\end{db}
\begin{w} Für den Beweis von \ref{4.4.2a} muss man die Unabhängigkeit der Ordnung von $t_P$ zeigen. Das ist schwierig!

Die restlichen Aussagen folgen dann aus \ref{4.4.2a}.
\end{w}
\begin{satz}[Riemann-Roch]
Sei $\K$ ein kanonischer Divisor auf $\CC$ und $D\in \Div(\CC)$. Dann gilt: \[\ll(D)-\ll(\K - D)=\deg D+1-\g.\]
\end{satz}

\begin{kor}\label{4.4.3} \begin{enumerate}
  \item\Label{4.4.3a} Ist $\K$ ein kanonischer Divisor sowie $D=0$, so ist $\deg D=0$, $\ll(D)=1$ nach \cref{4.3.2a}. Also ist in diesem Fall $\ll(\K)=\g$.
  \item\Label{4.4.3b} Ist $D=\K$ ein kanonischer Divisor, so ist $\ll(\K)=\g$ nach \ref{4.4.3a}. Also gilt hier \[\deg (\K)=\g-1+\ll(D)-\ll(0)=2\g-2.\]
 \end{enumerate}
\end{kor}
\begin{bsp}\label{4.4.4}
Sei $\CC=\P^1(K)$. Dann ist $K(\CC)=K(X)$ und $\Omega_{\CC}=K(X)\d X$, $\omega=\d X$ ist uniformisierend.

Wir suchen den kanonischen Divisor $\K=\div \omega$, d.h. wir müssen für jeden Punkt $P$ die Ordnung $\ord_P(\omega)$ bestimmen (vgl. \cref{4.4.2}).

Sei dazu zunächst $a\in K$. Dann ist $X-a$ uniformisierend. Es gilt also $\d X=\d (X-a)$. Also ist $\ord_a (\omega)=0$.

Nun sei $a=\infty$. Dann ist $\frac{1}{X}$ uniformisierend und mit der Leibnizregel sieht man: \[\d\frac{1}{X}=-\frac{1}{X^2} \d X.\]
Also ist $\ord_{\infty}(\omega)=-2$ und damit ist $\K=-2\cdot \infty$ der kanonische Divisor zu $\omega$. 

Insbesondere ist $\deg \K=-2$. Setze $D=\K$. Nun liefert \cref{4.4.3b}: \[\g(\P^1(K))=0.\] 
\end{bsp}

\chapter{Liste der Sätze}

\theoremlisttype{allname}
\listtheorems{satz}

%\section{Definitionen}
%\listtheorems{dfn}
\renewcommand\indexname{Stichwortverzeichnis}
\printindex

\end{document}
