\documentclass[12pt]{book}   %12pt --> art10

\oddsidemargin=0cm
\evensidemargin=0cm
\topmargin=-2cm
\textwidth=18cm
\textheight=25cm
\mathsurround=1.5pt
\parskip=5pt
\parindent=0pt
%\setcounter{page}{0}

%\input pictex
%\input prepictex
%\input postpictex

%\usepackage{color}
\usepackage{a4}
\usepackage{amsmath,amssymb,amsfonts}
\usepackage[latin1]{inputenc}
\usepackage{fontenc}[german]
\usepackage[all]{xy}
\usepackage{german}
\selectlanguage{german}
%\renewcommand{\thefootnote}{\fnsymbol{footnote}}
%\pagestyle{empty}

\usepackage{hyperref}

\newtheorem{alles}{alles}[section]
\newtheorem{bem}[alles]{Bemerkung}
\newtheorem{satz}[alles]{Satz}
\newtheorem{hilfs}[alles]{Hilfssatz}
\newtheorem{defini}[alles]{Definition}
\newtheorem{bsp}[alles]{Beispiel}
\newtheorem{fazit}[alles]{Fazit}
\newtheorem{Folgerung}[alles]{Folgerung}

%\makeindex

\begin{document}
Das Ziel diese Skripts ist es, einen ersten Einblick in die Topologie zu geben.
An keiner Stelle wird versucht, Ergebnisse bis in die letzten Winkel und 
Spitzen zu treiben; wir erlauben uns auch bisweilen, nicht geringstm\"ogliche 
Voraussetzungen in Aussagen zu machen, sondern hoffen, durch eine 
Beschr\"ankung auf einfachere Situationen bisweilen den Inhalt der S\"atze
(von denen es ohenhin nicht so viele gibt) deutlich zu machen. Die Vorlesung
ist nicht f\"ur Spezialisten gedacht - das verbietet sich schon angesichts des
Dozenten, der ja auch selbst kein Spezialist ist. Hiermit sei seiner Hoffnung 
der Ausdruck verliehen, dass die subjektive Stoffauswahl nicht zu sehr zu 
Lasten der Allgemeinheit geht und ein Verst\"andnis trotz allem zustande
kommen kann.

Jedenfalls werden in dieser Vorlesung nicht alle erlaubten Implikationen 
zwischen allen m\"oglichen Aussagen vorgef\"uhrt werden. 

Ich habe auf eine umfangreiche Illustration verzichtet, zum einen weil dies in 
der Vorlesung passieren soll, zum 
anderen, weil es vielleicht auch f\"ur Leser eine instruktive \"Ubung ist, sich
selbst ein Bild von dem zu machen, wovon die Rede ist. Der begriffliche 
Apparat ist das pr\"azise Werkzeug, die Bilder sind ja \glqq nur\grqq\ ein 
Hilfsmittel, das uns helfen soll zu sehen, wo die Werkzeuge angesetzt werden 
k\"onnen. Au\ss erdem sind manche Bilder sehr irref\"uhrend, zumal wenn es um
Sachverhalte geht, die sich definitiv nicht mehr in unserem Anschauungsraum
abspielen k\"onnen.



\bigskip

Nun zum Inhalt selbst.

\chapter{Einstieg}

\section{Kontext}




Die Topologie ist eine mathematische Grundlagendisziplin die sich verst\"arkt
seit dem Ende des 19.\ Jahrhunderts eigenst\"andig entwickelt hat. Vorher waren
einige topologische Ideen im Zusammenhang mit geometrischen und analytischen 
Fragestellungen entstanden. Um Topologie handelt es sich zun\"achst immer dann,
wenn geometrische Objekte deformiert werden und solche Eigenschaften der 
Objekte in den Vordergrund treten die sich dabei nicht \"andern. 

Topologisch ist eine Kugel dasselbe wie ein W\"urfel - geometrisch zwar 
v\"ollig unterschiedlich, aber doch gibt es einige Gemeinsamkeiten. 
Es w\"are vielleicht einmal interessant zu verfolgen, ob der Kubismus am Ende
des 19.\ Jhdts.\ und die topologische Frage nach \glqq simplizialen 
Zerlegungen\grqq\ geometrischer Objekte sich gegenseitig beeinflusst haben\dots

Der Begriff der N\"ahe spielt in der Topologie eine gewisse Rolle, mehr als der
Begriff des Abstands, der f\"ur die Geometrie immerhin namensgebend war.
Die topologischen 
Mechanismen, die so entwickelt wurden, wurden nach und nach von ihren 
geometrischen Eltern entfernt; daf\"ur sind die Eltern ja da: sich 
\"uberfl\"ussig zu machen. Und so konnten topologische Ideen sich auch auf 
andere Bereiche der Mathematik ausdehnen und diese geometrisch durchdringen.

Auch au\ss erhalb der Mathematik ist die Topologie l\"angst keine unbekannte
mehr. So gab es in der ersten H\"alfte des 20.\ Jhdts.\ die topologische 
Psychologie von Kurt Lewin, die allerdings nur die Terminologie von der 
Topologie \"ubernahm, und nicht etwa mithilfe topologischer Argumente neue
Einsichten produzierte. Etwas anders sieht es nat\"urlich mit den 
\glqq richtigen\grqq Naturwissenschaften aus. In der Physik taucht
die Topologie zum Beispiel in der Form von Modulr\"aumen in der Stringtheorie 
auf, und in der Molekularchemie kann man zum Beispiel Chiralit\"at als
topologisches Ph\"anomen verstehen.



\section{Beispiele - was macht die Topologie?}

\begin{bsp} \label{Lasso}{\bf Nullstellenfang mit dem Lasso}

{\rm Es sei $f:\mathbb C\longrightarrow \mathbb C$ eine nichtkonstante 
Polynomabbildung, d.h.\ $f(z) = \sum_{i=0}^d a_iz^i$ mit $d>0$ und $a_d\neq 0.$

Dann hat $f$ eine Nullstelle in $\mathbb C.$ Das kann man zum Beispiel so 
plausibel machen:

Wenn $a_0=0$ gilt, dann ist $z=0$ eine Nullstelle. Wenn $a_0\neq 0$, dann 
brauchen wir ein Argument. Wir betrachten den Kreis vom Radius $R$ um den 
Nullpunkt: $RS^1 = \{z\in \mathbb C \mid |z|=R\}.$ Aus der Gleichung
$$f(z) = a_dz^d \cdot (1 + \frac{a_{d-1}}{a_d z} + \dots + 
\frac{a_{0}}{a_d z^d})$$ 
folgt, dass das Bild von $RS^1$ unter $f$ jedenfalls f\"ur gro\ss es $R$ im 
Wesentlichen der $d$-fach 
durchlaufene Kreis vom Radius $|a_d|R^d$ ist. Im Inneren dieser Schlaufe liegen
f\"ur gro\ss es $R$ sowohl die 0 als auch $a_0.$ Wenn man nun den Radius 
kleiner macht, so wird diese Schlaufe f\"ur $R\searrow 0$ zu einer Schlaufe um 
$a_0$
zusammengezogen -- das ist die Stetigkeit von $f.$ F\"ur kleines $R$ liegt 
insbesondere $0$ nicht im Inneren der Schlaufe. Das aber hei\ss t, dass beim
Prozess des Zusammenziehens die Schlaufe irgendwann mindestens einmal die 0
trifft. Dann hat man eine Nullstelle von $f$ gefunden.

Einen anders gelagerten und pr\"azisen topologischen Beweis des 
Fundamentalsatzes werden wir in \ref{Fundamentalsatz} f\"uhren.
}
\end{bsp}

In diesem Argument -- das man streng durchziehen kann -- wird ein topologisches
Ph\"anomen benutzt, um den Fundamentalsatz der Algebra zu beweisen. Das 
Zusammenziehen der Kurve durch Variation des Parameters $R$ werden wir sp\"ater
allgemeiner als Spezialfall einer Homotopie verstehen.

\begin{bsp}{\bf Fahrradpanne}

{\rm Es gibt keine stetige Bijektion von einem Torus $T$ 
(\glqq Fahrradschlauch\grqq ) auf eine Kugeloberfl\"ache $S.$ 


Denn:  Auf dem Torus gibt es eine geschlossene Kurve $\gamma$, die ihn nicht 
in zwei 
Teile zerlegt. Ihr Bild unter einer stetigen Bijektion von $T$ nach $S$ 
w\"urde dann $S$ auch nicht in zwei Teile zerlegen, da das stetige Bild des
Komplements $T\smallsetminus \gamma$ gleich 
$S\smallsetminus{\rm Bild\ von\ }\gamma$ zusamenh\"angend sein m\"usste, aber
das stimmt f\"ur keine geschlossene Kurve auf $S$.}
\end{bsp}

Auch hier sieht man ein topologisches Prinzip am Werk. Es ist oft sehr schwer
zu zeigen, dass es zwischen zwei topologischen R\"aumen (siehe sp\"ater) keine
stetige Bijektion gibt. Dass ich keine solche finde sagt ja noch nicht wirklich
etwas aus\dots

In der linearen Algebra wei\ss\  man sehr genau, wann es 
zwischen zwei Vektorr\"aumen einen Isomorphismus gibt, das h\"angt ja nur an 
der Dimension. \"Ahnlich versucht man in der Topologie, zu topologischen 
R\"aumen zugeordnete Strukturen zu finden, die nur vom Isomorphietyp 
abh\"angen, und deren Isomorphieklassen man besser versteht als die der 
topologischen R\"aume.

\begin{bsp} {\bf Eulers \footnote{Leonhard Euler, 1707-1783}Polyederformel}

{\rm F\"ur die Anzahl $E$ der Ecken, $K$ der Kanten und $F$ der Fl\"achen eines
(konvexen) Polyeders gilt die Beziehung $E-K+F=2.$

Das kann man zum Beispiel einsehen, indem man das Polyeder zu einer Kugel 
aufbl\"ast, auf der man dann einen Graphen aufgemalt hat (Ecken und Kanten des 
Polyeders), und dann f\"ur je zwei solche zusammenh\"angenden Graphen zeigt, 
dass sie eine gemeinsame Verfeinerung haben. Beim Verfeinern \"andert sich
aber $E-K+F$ nicht, und so muss man nur noch f\"ur ein Polyeder die 
alternierende Summe auswerten, zum Beispiel f\"ur das Tetraeder, bei dem 
$E=F=4, K=6$ gilt.
}
\end{bsp}

\begin{bsp} {\bf Reelle Divisionsalgebren}

{\rm Eine reelle Divisionsalgebra ist ein $\mathbb R$-Vektorraum $A$ mit einer
bilinearen Multiplikation, f\"ur die es ein neutrales Element gibt und jedes
$a\in A\smallsetminus\{0\}$ invertierbar ist. 

Beispiele hier\"ur sind $\mathbb R,\mathbb C, \mathbb H$ 
(Hamilton\footnote{William Hamilton, 1788-1856}-Quaternionen) und -- wenn man 
die Assoziativit\"at wirklich nicht haben will -- $\mathbb O$ (die 
Cayley\footnote{Arthur Cayley, 1821-1895}-Oktaven). Die Dimensionen dieser 
Vektorr\"aume sind $1,2,4,8.$ Tats\"achlich ist es so, dass es keine weiteren
endlichdimensionalen reellen Divisionsalgebren gibt. Dies
hat letztlich einen topologischen Grund. 

Zun\"achst \"uberlegt man sich, dass die Struktur einer Divisionsalgebra auf 
$\mathbb R^n$ auf der $n-1$-dimensionalen Sph\"are eine Verkn\"upfung
induziert, die fast eine Gruppenstruktur ist.

Dann kann man im wesentlichen topologisch zeigen, dass solch eine Struktur
auf der Sp\"are nur f\"ur
$n\in\{1,2,4,8\}$ existieren kann. Solch eine Gruppenstruktur stellt n\"amlich 
topologische Bedingungen, die f\"ur die anderen Sph\"aren nicht erf\"ullt sind.
}\end{bsp}



Eng damit zusammen h\"angt der

\begin{bsp} {\bf Satz vom Igel\footnote{Frans Ferdinand Igel, ???}}

{\rm Dieser Satz sagt, dass jeder stetig gek\"ammte Igel mindestens einen 
Glatzpunkt besitzt. Die Richtigkeit dieses Satzes gr\"undet sich nicht darauf, 
dass es bisher noch niemanden gelingen ist, einen Igel zu k\"ammen. Sie hat
handfeste mathematische Gr\"unde, die in einer etwas pr\"aziseren Formulierung 
klarer werden:

Etwas weniger prosaisch besagt der Satz \glqq eigentlich\grqq, dass ein 
stetiges Vektorfeld auf der zweidimensionalen Sph\"are mindestens eine
Nullstelle besitzt.
}
\end{bsp}

\begin{bsp} {\bf Brouwers\footnote{Luitzen Egbertus Jan Brouwer, 1881-1966}
Fixpunktsatz}

{\rm Jede stetige Abbildung des $n$-dimensionalen Einheitsw\"urfels 
$W=[0,1]^n$ in sich selbst hat einen Fixpunkt.

F\"ur $n=1$ ist das im Wesentlichen der Zwischenwertsatz. Ist $f:[0,1]
\longrightarrow [0,1]$ stetig, so ist auch $g(x) := f(x)-x$ eine stetige 
Abbildung von $[0,1]$ nach $\mathbb R$, und es gilt $g(0)\geq 0, g(1)\leq 0.$

Also hat $g$ auf jeden Fall eine Nullstelle $x_0$, aber das hei\ss t dann 
$f(x_0) = x_0.$

F\"ur $n\geq 2$ ist der Beweis so einfach nicht m\"oglich, wir werden ihm 
eventuell auch nur f\"ur $n=2$ sp\"ater noch begegnen.
}
\end{bsp}

\section{Mengen, Abbildungen, usw.}

Wir werden f\"ur eine Menge $M$ mit ${\cal P}(M)$ immer die Potenzmenge 
bezeichnen: 
$${\cal P}(M) = \{A \mid A\subseteq M\}.$$
F\"ur eine Abbildung $f:M\longrightarrow N$ nennen wir das Urbild
$f^{-1}(n)$ eines Elements $n\in f(M)\subseteq N$ auch eine 
\index{Faser}{\it Faser} von $f.$

Eine Abbildung ist also injektiv, wenn alle Fasern  einelementig sind.

Ist $f$ surjektiv, so gibt es eine Abbildung $s:N\longrightarrow M$ mit
$f\circ s = {\rm Id}_N$ -- die identische Abbildung auf $N.$ Jede solche 
Abbildung $s$ hei\ss t ein \index{Schnitt}{\it Schnitt} zu $f$. Er w\"ahlt zu 
jedem $n\in N$ ein $s(n)\in f^{-1}(n)$ aus. Wenn man also $M$ als Vereinigung
der Fasern von $f$ \"uber den Blumentopf $N$ malt, so erh\"alt der Name Schnitt
eine gewisse Berechtigung.

Eine \index{Partition}{\it Partition} von $M$ ist eine Zerlegung von $M$ in 
disjunkte, nichtleere Mengen $M_i,i\in I,$ wobei $I$ eine Indexmenge ist:
$$M=\bigcup_{i\in I} M_i,\ \ \forall i\neq j: M_i\cap M_j = \emptyset, 
M_i\neq\emptyset.$$

Hand in Hand mit solchen Partitionen gehen \"Aquivalenzrelationen auf $M.$
Die Relation zur Partition $M_i,i\in I$ wird gegeben durch
$$m\sim \tilde m \iff \exists i\in I: m,\tilde m\in M_i.$$
Umgekehrt sind die \"Aquivalenzklassen zu einer \"Aquivalenzrelation $\sim $ 
eine Partition von $M.$ Die Menge der \"Aquivalenzklassen nennen wir auch den
\index{Faktorraum}{\it Faktorraum} $M/\sim$:
$$M/\sim = \{M_i\mid i\in I\}.$$
Die Abbildung $\pi_\sim: M\longrightarrow M/\sim, \pi_\sim(m):= [m] = $ 
\"Aquivalenzklasse von $m$ hei\ss t die {\it kanonische Projektion von } $M$ 
auf $M/\sim$.

Ist $\sim$ eine \"Aquivalenzrelation auf $M$ und $f:M\longrightarrow N$ eine
Abbildung, sodass
jede \"Aquivalenzklasse von $\sim$ in einer Faser von $f$ enthalten ist (d.h.
$f$ ist konstant auf den Klassen), so wird durch 

$$\tilde f:M/\sim \longrightarrow N, \tilde f([m]) := f(m),$$
eine Abbildung definiert, f\"ur die $f=\tilde f \circ \pi_\sim$ gilt. Das ist
die mengentheoretische Variante des Homomorphiesatzes.

\begin{bsp}{\bf Gruppenaktionen}

{\rm Ein auch in der Topologie wichtiges Beispiel, wie \"Aquivalenzrelationen 
bisweilen entstehen, ist das der \index{Gruppenoperation}{\it Operation} 
einer Gruppe $G$ auf der Menge $M$.

Solch eine Gruppenaktion ist eine Abbildung
$$\bullet:G\times M\longrightarrow M,$$
die die folgenden Bedingungen erf\"ullt:

$$\begin{array}{rl}\forall m\in M:& e_G\bullet m = m\\
\forall g,h\in G, m\in M:& g\bullet (h\bullet m) = (gh)\bullet m.\\
\end{array}$$
Hierbei ist $e_G$ das neutrale Element von $G$ und $gh$ ist das Produkt von 
$g$ und $h$ in $G$.

F\"ur jedes $g\in G$ ist die Abbildung 
$$\rho_g:M\longrightarrow M,\rho_g(m) := g\bullet m,$$
eine Bijektion von $M$ nach $M$, die Inverse ist $\rho_{g^{-1}},$ und 
$$g\mapsto \rho_g$$
ist ein Gruppenhomomorphismus von $G$ in die symmetrische Gruppe von $M$.

Die \index{Bahn}{\it Bahn} von $m\in M$ unter der Operation von $G$ ist 
$$G\bullet m := \{g\bullet m\mid g\in G\}.$$
Man kann leicht verifizieren, dass die Bahnen einer Gruppenoperation eine
Partition von $M$ bilden.

Umgekehrt ist es so, dass jede Partition $(M_i)_{i\in I}$ von $M$ von der 
nat\"urlichen Aktion einer geeigneten Untergruppe $G$ von ${\rm Sym}(M)$ 
herkommt. Hierzu w\"ahle man einfach
$$G:=\{\sigma\in {\rm Sym}(M) \mid \forall i\in I: \sigma(M_i)= M_i\}$$
und verifiziere was zu verifizieren ist.
}
\end{bsp}
\begin{bsp} {\bf projektive R\"aume}

{\rm Es seien $K$ ein K\"orper und $n$ eine nat\"urliche Zahl.

Auf $X:=K^{n+1}\smallsetminus\{0\}$ operiert die Gruppe $K^\times$ durch
die skalare Multiplikation
$$a\bullet v := a\cdot v.$$
Die Bahn von $v\in X$ unter dieser Operation ist $Kv\smallsetminus \{0\}.$
Da die $0$ ohnehin zu jeder Geraden durch den Ursprung geh\"ort, kann man
den Bahnenraum $X / K^\times$ mit der Menge aller Geraden durch den Ursprung
identifizieren. Dieser Raum hei\ss t der {\it $n$-dimensionale projektive Raum
\"uber $K$}, \index{projektiver Raum}  in Zeichen $\mathbb P^n(K).$

Speziell f\"ur $n=1$ gilt:
$$\mathbb P^1(K) = \{[{a\choose 1}] \mid a\in K \} \bigcup \{[{1\choose 0}]\}.$$
Oft identifiziert man den ersten gro\ss en Brocken hier mit $K,$ den 
hinzukommenden Punkt nennt man suggestiver Weise $\infty.$

Genauso haben wir f\"ur beliebiges $n$ eine Zerlegung
$$\mathbb P^n(K) = \{[ {v\choose 1}] \mid v\in K^n \} \bigcup 
\{[{w\choose 0}] \mid w\in K^n,w\neq 0\} = K^n \bigcup \mathbb P^{n-1}(K),$$
wobei die Auswahl des {\it affinen Teils} $K^n$ durch die Bedingung, dass die 
letzte Koordinate nicht null ist, relativ willk\"urlich ist. 
}
\end{bsp}


\begin{defini} {\bf Faserprodukte}

{\rm Es seien $A,B,S$ Mengen und 
$f_A:A\longrightarrow S,\ f_B:B\longrightarrow S$
zwei Abbildungen.

Weiter sei $F$ eine Menge mit Abbildungen $\pi_A,\pi_B$ von $F$ nach $A$ bzw.\ 
$B$, sodass $f_A\circ\pi_A = f_B\circ \pi_B.$

$F$ hei\ss t ein {\it Faserprodukt}\index{Faserprodukt} von $A$ und $B$ \"uber
$S$, wenn f\"ur jede Menge $M$ und jedes Paar von Abbildungen $g_A,g_B$ von
$M$ nach $A$ bzw.\ $B$ mit $f_A\circ g_A = f_B\circ g_B$ genau eine Abbildung
$h:M\longrightarrow F$ existiert, sodass 
$$g_A = \pi_A\circ h,\ \ g_B = \pi_B\circ h.$$
}

\end{defini}

Insbesondere impliziert das, dass es zwischen zwei Faserprodukten von $A$ und 
$B$ \"uber $S$ genau einen sinnvollen Isomorphismus gibt. Denn nach Definition
gibt es f\"ur ein zweites Faserprodukt $(\widetilde F,\widetilde{\pi_A},
\widetilde{\pi_B})$ genau eine Abbildung $h$ von $\widetilde F$ nach $F$ mit
$$\widetilde{\pi_A} = \pi_A\circ h,\ \ \widetilde{\pi_B} = \pi_B\circ h$$
und auch genau eine Abbildung $\tilde h:F\longrightarrow  \widetilde F$ mit
$$\pi_A = \widetilde{\pi_A} \circ \tilde h,\ \ \pi_B = \widetilde{\pi_B}\circ 
\tilde h.$$
Dann ist aber $h\circ\tilde h$ eine Abbildung von $F$ nach $F$ mit
$$\pi_A = pi_A\circ (h\circ\tilde h),\ \ \pi_B = \pi_B\circ (h\circ\tilde h),$$
was wegen der Eindeutigkeit aus der Definition zwangsl\"aufig
$$h\circ\tilde h ={\rm Id}_F$$ 
nach sich zieht. Analog gilt auch
$$\tilde h\circ h = {\rm Id}_{\widetilde F}.$$

{\bf Schreibweise:} F\"ur das Faserprodukt schreibt man meistens $A\times_SB,$
wobei in der Notation die Abbildungen $f_A$ unf $f_B$ unterdr\"uckt werden.

\medskip

Ein Faserprodukt existiert immer. Wir k\"onnen n\"amlich 
$$F:=\{(a,b)\in A\times B \mid f_A(a) = f_B(b)\} $$
w\"ahlen und f\"ur $\pi_A,\pi_B$ die Projektion auf den ersten beziehungsweise
zweiten Eintrag. 

Die Abbildung $h$ aus der Definition ist dann einfach $h(m) = (g_A(m), g_B(m)).$

Wir k\"onnen $F$ auch hinschreiben als 
$$F=\bigcup_{s\in S} \left(f_A^{-1}(s) \times f_B^{-1}(s)\right),$$
also als Vereinigung der Produkte der Fasern von $f_A$ und $f_B$ \"uber jeweils
demselben Element von $S.$ Das erkl\"art den Namen.

\begin{bsp} {\bf Spezialf\"alle}

{\rm 
\begin{itemize}
\item[a)] Wenn $S$ nur aus einem Element $s$ besteht, dann sind $f_A$ und $f_B$
konstant, und damit $A\times_SB = A\times B$ das mengentheoretische Produkt.
\item[b)] Wenn $A,B$ Teilmengen von $S$ sind und die Abbildungen $f_A,f_B$ 
einfach die Inklusionen, dann gilt
$$A\times_SB = \{(a,b)\in A\times B\mid a=b\} = \{(s,s)\mid s\in A\cap B\} 
\simeq A\cap B.$$
\end{itemize}
}
\end{bsp}

\section{Metrische R\"aume}

\begin{defini} {\bf Metrischer Raum} 

{\rm Ein {\it metrischer Raum}\index{metrischer Raum} ist eine Menge $X$ 
zusammen mit einer Abbildung $$d:X\times X\longrightarrow \mathbb R_{\geq 0},$$
sodass die folgenden Bedingungen erf\"ullt sind:
\begin{itemize}
\item $\forall x,y\in X: d(x,y) = d(y,x)$ (Symmetrie)
\item $\forall x\in X: d(x,x) = 0.$
\item $\forall x,y\in X: x\neq y\Rightarrow d(x,y) >0.$
\item $\forall x,y,z\in X: d(x,y) + d(y,z) \geq d(x,z).$ (Dreiecksungleichung)
\end{itemize}
Die Abbildung $d$ hei\ss t dabei die {\it Metrik}.

Penibler Weise sollte man einen metrischen Raum als Paar $(X,d)$ schreiben.
Meistens wird das micht gemacht, aber Sie kennen diese Art der Schlamperei ja
schon zur Gen\"uge\dots
}\end{defini}

\begin{bsp} \label{L_unendlich}{\bf LA und ANA lassen gr\"u\ss en}

{\rm 
\begin{itemize}
\item [a)] Ein reeller Vektorraum mit einem Skalarprodukt 
$\langle\cdot,\cdot\rangle$ wird bekanntlich mit 
$$d(v,w) := \sqrt{\langle v-w,v-w\rangle} = |v-w|$$
zu einem metrischen Raum
\item[b)] Jede Menge $X$ wird notfalls durch 
$$d(x,y) = \left\{\begin{array}{rl} 1, & {\rm falls}\ x\neq y,\\
                                    0, & {\rm falls}\ x=y,\\
\end{array}\right.$$
zu einem metrischen Raum. Diese Metrik hei\ss t die {\it diskrete Metrik}
\index{diskrete Metrik} auf $X.$
\item[c)] Es sei $X$ eine Menge und ${\cal B}(X)$ der Vektorraum der 
beschr\"ankten reellwertigen Funktionen auf $X.$ Dann wird $X$ verm\"oge
$$d(f,g) := \sup\{|f(x)-g(x)|\mid x\in X\}$$
zu einem metrischen Raum. 

Anstelle der Norm aus einem Skalarprodukt benutzt man hier also die sogenannte
Maximumsnorm 
$$|f|_\infty := \sup\{|f(x)|\mid x\in X\},$$
um eine Metrik zu konstruieren. Diese kommt nicht von einem Skalarprodukt her,
wenn $X$ mindestens 2 Elemente hat.

Allgemeiner sei f\"ur eine Menge $X$ und einen metrischen Raum $(Y,e)$ die 
Menge
${\cal B}(X,Y)$ definiert als die Menge aller beschr\"ankten Abbildungen von 
$X$ nach $Y$. Dabei hei\ss t $f$ beschr\"ankt, wenn ein $R\in \mathbb R$ 
existiert mit
$$\forall x_1,x_2\in X: e(f(x_1),f(x_2)) < R.$$
Dann wird ${\cal B}(X,Y)$ zu einem metrischen Raum verm\"oge
$$d(f,g) := \sup\{e(f(x),g(x))\mid x\in X\}.$$
\item[d)] Auf den rationalen Zahlen l\"asst sich f\"ur eine Primzahl $p$
auf folgende Art eine Metrik konstruieren:

Jede rationale Zahl $q\neq 0$ kann man schreiben als 
$p^{{\rm v}_p(q)} \cdot \frac ab,$ wobei $a,b\in \mathbb Z$ keine Vielfachen von 
$p$ sind. Dann ist $v_p(q)$ eindeutig bestimmt.

Wir setzen f\"ur zwei rationale Zahlen $x,y$
$$d_p(x,y) :=\left\{\begin{array}{rl}
0, & {\rm falls}\ x=y,\\
p^{-{\rm v}_p(x-y)}, & {\rm sonst}.\\ \end{array}\right.$$
Dies ist die sogenannte $p$-adische Metrik auf $\mathbb Q.$
\end{itemize}
}\end{bsp}

\begin{defini} {\bf Folgen und Grenzwerte}

{\rm 
\begin{itemize}
Es sei $(X,d)$ ein metrischer Raum. 
\item[a)]
Eine Folge $(x_n)_{n\in \mathbb N}$
in $X$ hei\ss t {\it konvergent gegen den Grenzwert} $y\in X,$ falls
$$\lim_{n\to\infty} d(x_n,y) = 0.$$
Nat\"urlich ist $y$ hierbei eindeutig bestimmt.
\item[b)] Eine Cauchyfolge\footnote{Augustin-Louis Cauchy, 1789-1857} in $X$ 
ist eine Folge $(x_n)_{n\in \mathbb N}$ mit
$$\forall \varepsilon>0 : \exists N\in \mathbb N: \forall m,n>N:
d(x_m,x_n) < \varepsilon.$$
Jede konvergente Folge ist eine Cauchyfolge.
\item[c)] $X$ hei\ss t {\it vollst\"andig}\index{vollst\"andig}, wenn jede
Cauchyfolge in $X$ einen Grenzwert in $X$ hat.
\end{itemize}
}
\end{defini}

\begin{bsp} {\bf Schatten der Vergangenheit}

{\rm 
Jeder endlichdimensionale euklidische Vektorraum ist vollst\"andig. 

Die Konvergenz einer Folge in ${\cal B}(X)$ mit der $\infty$-Norm ist einfach 
die gleichm\"a\ss ige Konvergenz im Sinne der Analysis. Insbesondere ist
${\cal B}(X)$ mit dieser Metrik vollst\"andig.

$\mathbb Q$ mit der $p$-adischen Metrik ist nicht vollst\"andig. Man kann einen
K\"orper $\mathbb Q_p$ konstruieren, der $\mathbb Q$ enth\"alt, auf dem eine
Metrik definiert ist, die die $p$-adische fortsetzt, und in dem sich jedes 
Element durch eine $p$-adische Cauchyfolge in $\mathbb Q$ approximieren 
l\"asst. Damit wird ein arithmetisch wichtiges Pendant zu den reellen Zahlen 
geschaffen, die sich ja auch konstruieren lassen als (archimedische) 
Cauchyfolgen in $\mathbb Q$ modulo Nullfolgen. 
}
\end{bsp}
\begin{defini} {\bf Isometrien}

{\rm Es seien $(X,d)$ und $(Y,e)$ zwei metrische R\"aume. Eine 
{\it abstandserhaltende Abbildung} von $X$ nach $Y$  ist eine Abbildung
$f:X\longrightarrow Y,$ f\"ur die gilt:
$$\forall x_1,x_2 \in X: d(x_1,x_2) = e(f(x_1),f(x_2)).$$
Solche Abbildungen sind immer injektiv. Eine surjektive abstandserhaltende
Abbildung hei\ss t eine {\it Isometrie}. 

Die Menge der Isometrien von $X$ nach $X$ ist eine Untergruppe der { 
symmetrischen Gruppe} 
${\rm Sym}(M)$ aller Bijektionen von $X$ nach $X$.
}\end{defini}

Aber eigentlich ist das momentan kein Begriff, der unsere Aufmerksamkeit zu 
stark in Anspruch nehmen sollte. 

\begin{defini} {\bf Kugeln}

{\rm Es seien $(X,d)$ ein metrischer Raum und $x\in X$ sowie $r>0$ eine reelle 
Zahl. Dann hei\ss t
$$B_r(x) := \{y\in X \mid d(x,y)<r\}$$
die {\it offene Kugel} vom Radius $r$ um den Mittelpunkt $x.$ 

{\bf Vorsicht:} Weder $x$ noch $r$ m\"ussen durch die Menge $B_r(x)$ eindeutig 
bestimmt sein.
Wenn zum Beispiel $X$ mit der diskreten Metrik ausgestattet ist, so ist 
$X=B_2(x) = B_3(y)$ f\"ur alle $x,y\in X.$
}
\end{defini}

\begin{defini} \label{stetig1}{\bf Stetigkeit}

{\rm 
Es seien $(X,d)$ und $(Y,e)$ zwei metrische R\"aume. Dann hei\ss t eine
Abbildung $f:X\longrightarrow Y$ {\rm stetig}, falls f\"ur jedes 
$x\in X$ und jedes $\varepsilon>0$ ein $\delta>0$ existiert, sodass
$$f(B_\delta(x)) \subseteq B_\varepsilon(f(x)).$$
In Worten: Jede offene Kugel um $f(x)$ enth\"alt das Bild einer offenen Kugel 
um $x$.
}
\end{defini}

So ist zum Beispiel jede Abbildung von $X$ nach $Y$ stetig, wenn auf $X$ die 
diskrete Metrik vorliegt. Denn dann ist ja $\{x\} = B_{\frac12}(x)$ im Urbild
jeder offenen Kugel um $f(x)$ enthalten.

\begin{bsp}{\bf Noch einmal die Analysis}

{\rm Es seien $X,Y$ metrische R\"aume. Dann bezeichnen wir mit 
${\cal C}(X,Y)$ die Menge aller stetigen Abbildungen von $X$ nach $Y,$ und 
mit ${\cal C}_0(X,Y)$ die Menge aller beschr\"ankten stetigen Abbildungen
von $X$ nach $Y.$ 

Wenn $Y$ vollst\"andig ist, dann ist auch ${\cal C}_0(X,Y)$ (als Teilraum von
${\cal B}(X,Y)$) vollst\"andig. 

Im Fall $Y=\mathbb R$ l\"asst man das $Y$ auch h\"aufig weg und schreibt nur 
${\cal C}(X)$ bzw.\ ${\cal C}_0(X).$
}\end{bsp}

\begin{hilfs}\label{Normen}{\bf Normen}

Es sei $V=\mathbb R^n$ mit dem Standardskalarprodukt versehen und $N$ eine
Norm auf $V,$ d.h. $N:V\longrightarrow \mathbb R_{\geq 0}$ erf\"ullt 
\begin{itemize}
\item $\forall v\in V:N(v) = 0 \iff v=0$ (Positivit\"at),
\item $\forall a\in \mathbb R,v\in V: N(av) = |a|N(v)$ (Homogenit\"at),
\item $\forall v,w\in V: N(v+w) \leq N(v) + N(w)$ (Dreicksungleichung).
\end{itemize}
Dann ist $N$ stetig bez\"uglich der Standardmetrik.
\end{hilfs}
{\it Beweis.} Das Urbild von $(-\varepsilon , \varepsilon)$ unter $N$ ist 
konvex, d.h. 
$$\forall v,w\in V: [N(v),N(w)\leq \varepsilon\Rightarrow 
\forall \lambda\in [0,1]: N(\lambda v + (1-\lambda) w) \leq \varepsilon].$$
Das folgt sofort aus den drei aufgelisteten Eigenschaften der Normabbildung. 

Wegen der Positivit\"at und der Homogenit\"at gibt es eine Konstante $c>0$ 
(abh\"an\-gig von $\varepsilon$), sodass die Vektoren 
$\pm c e_i,\ 1\leq i\leq n,$
in $N^{-1}(-\varepsilon , \varepsilon)$ liegen. Dabei ist $\{e_1,\dots ,e_n\}$ 
die Standardbasis von $\mathbb R^n.$

Es sei $v\in B_{c/\sqrt n}(0)\subseteq V,$ d.h. $v= \sum_{i=1}^n a_i c e_i, 
\sum_i a_i^2 < 1/n.$ Dann ist aber die Summe $\sum_i|a_i| < 1.$ F\"ur
$\alpha = \sum |a_i|$ ist also $v=\alpha\cdot \sum_i \frac{a_i}{\alpha}
ce_i$ das $\alpha$-fache einer Konvexkombination 
von $\pm ce_1,\dots \pm ce_n.$ Wegen  $|\alpha|<1$ liegt wegen der 
Homogenit\"at von $N$ auch $v$ im Urbild von $(-\varepsilon, \varepsilon)$, und
damit liegt die offene Kugel $B_{c\sqrt n}(0)$ im Urbild: $N$ ist stetig im 
Ursprung. 

Nun seien $x\in V$ beliebig und $\delta>0$ vorgegeben. Dann gibt es nach dem 
eben gesehenen ein $\varepsilon>0$ mit 
$$\forall y\in B_\varepsilon (0) : |N(y| < \delta.$$
F\"ur $y\in B_\varepsilon (0)$ gilt demnach wegen
$N(x) = N(x+y-y)\geq N(x+y) + N(y):$
$$-N(y) \leq  N(x+y) - N(x) \leq N(y),$$
und daher $N(B_\varepsilon(x)) \subseteq B_\delta(N(x)).$ 

Das zeigt die Stetigkeit von $N.$ \hfill{$\bigcirc$}

\begin{defini} {\bf Die Topologie eines metrischen Raums}

{\rm Es sei $(X,d)$ ein metrischer Raum. Eine Teilmenge $A\subseteq X$ hei\ss t
{\it offen} \index{offen}, falls f\"ur jedes $x\in A$ eine reelle Zahl $r>0$
existiert, sodass $B_r(x)\subseteq A$ gilt. 

Die Gesamtheit aller offenen Mengen in $X$ hei\ss t die {\it Topologie} von 
$(X,d).$
}

\end{defini}

\begin{bem} {\bf Eigenschaften}

{\rm Die offenen Mengen eines metrischen Raums haben die folgenden beiden
Eigenschaften: beliebige Vereinigungen und endliche Durchschnitte von offenen
Mengen sind wieder offen.

Eine Abbildung $f:X\longrightarrow Y$ zwischen metrischen R\"aumen ist genau 
dann stetig, wenn f\"ur jede offene Teilmenge $U\subseteq Y$ das Urbild
$f^{-1}(U)$ offen in $X$ ist.

Es kann sehr viele verschiedene Metriken auf $X$ geben, die zur selben 
Topologie f\"uhren. So stimmen zum Beispiel f\"ur zwei Normen auf $\mathbb R^n$
die zugeh\"origen Topologien \"uberein, was im Wesentlichen aus Hilfssatz 
\ref{Normen} folgt. Dieser Hilfssatz sagt n\"amlich, dass die Identit\"at auf
$\mathbb R^n$ stetig ist, wenn wir auf Seiten des Definitionsbereichs die
euklidische Standardl\"ange als Norm benutzen, und auf Bildseite die Norm $N.$
Auch in der anderen Richtung ist die Identit\"at stetig (das muss man noch 
beweisen!), und das impliziert die Gleichheit der zugeh\"origen Topologien.   
}
\end{bem}




\chapter{Topologische Grundbegriffe}

\section{Topologische R\"aume und ein paar Konstruktionen}

\begin{defini} {\bf Topologischer Raum}

{\rm Ein {\it topologischer Raum} \index{topologischer Raum} ist eine Menge 
$X$, f\"ur die eine Teilmenge 
${\cal T} \subseteq {\cal P} (X)$ mit folgenden Eigenschaften ausgew\"ahlt 
wurde:
\begin{itemize}
\item $\emptyset, X\in {\cal T}$
\item $\forall A,B\in {\cal T}: A\cap B\in {\cal T}$
\item $\forall {\cal S}\subseteq{\cal T}: \bigcup_{A\in {\cal S}}A \in {\cal T}.$
\end{itemize}
Hierbei hei\ss t $\cal T$ die Topologie auf $X$, und die Elemente von 
$\cal T$ sind die \index{offen} {\it offenen Mengen} des 
topologischen Raums $(X,{\cal T}).$ 

Die Mengen $X\smallsetminus A, A\in {\cal T},$ hei\ss en {\it abgeschlossene
Teilmengen}\index{abgeschlossen} von $X$.


}\end{defini}
\begin{bsp}{\bf Alte Bekannte}

{\rm Die offenen Mengen eines metrischen Raums $X$ bilden eine Topologie auf 
$X$.


Ist $X$ eine beliebige Menge, so ist die Potenzmenge ${\cal P}(X)$  eine
Topologie auf $X.$ Sie ist sogar die Topologie zu einer Metrik auf $X$, zur 
diskreten Metrik n\"amlich. Sie hei\ss t die {\it diskrete Topologie}).

Auch $\{\emptyset, X\}$ ist eine Topologie auf $X.$

Auf $\{0,1\}$ gibt es die Topologie $\{\emptyset, \{0\}, \{0,1\}\}.$
Diese kommt nicht von einer Metrik her.
}

\end{bsp}
\begin{defini}{\bf Inneres, Abschluss und der zu schmale Rand}

{\rm Es seien $X$ ein topologischer Raum und $A\subseteq X$ eine offene 
Teilmenge. Das {\it Innere} ${\stackrel\circ A}$ von $A$ ist definiert als die
Vereinigung
$$\stackrel\circ A := \bigcup_{U\subseteq_o A} U,$$
wobei das Symbol $\subseteq_o$ bedeutet, dass $U$ eine (in $X$) offene 
Teilmenge von $A$ ist.

$\stackrel\circ A$ ist offen.

Der {\it Abschluss} $\bar A$ von $A$ ist definiert als der Durchschnitt aller 
abgeschlossenen Teilmengen von $X$, die $A$ enthalten.

$\bar B$ ist abgeschlossen.

Der {\it Rand} von $A$ ist die Menge 
$\partial A:=\bar A\smallsetminus \stackrel\circ A.$

}
\end{defini}
\begin{defini} \label{dicht}{\bf Dichtheit, Diskretheit}

{\rm Es sei $X$ ein topologischer Raum. Eine Teilmenge $D\subseteq X$ hei\ss t 
{\it dicht}\index{dicht}, wenn ihr Abschluss ganz $X$ ist. 

Jede Teilmenge ist also dicht in ihrem Abschluss, wenn man diesen wie in 
\ref{Spurtopologie} als topologischen Raum betrachtet.

Eine Teilmenge $D\subseteq X$ hei\ss t \index{diskret}{\it diskret}, wenn
jeder Punkt $x\in X$ eine Umgebung besitzt, die mit $D$ endlichen Durchschnitt 
hat.

F\"ur metrische R\"aume hei\ss t das gerade, dass $D$ keinen H\"aufungspunkt 
besitzt. 
}
\end{defini}


\begin{defini}{\bf Umgebungen, Basis einer Topologie}

{\rm Es sei $(X,{\cal T})$ ein topologischer Raum.
\begin{itemize}
\item[a)] F\"ur $x\in X$ hei\ss t eine Teilmenge $A\subset X$ eine 
{\it Umgebung}\index{Umgebung} von $x,$ falls eine offene Teilmenge 
$U\subseteq X$ existiert mit $x\in U\subseteq A.$ Ist $A$ selbst schon offen, 
so hei\ss t es eine {\it offene Umgebung} von $x$ (falls $x\in A$).
\item[b)] Eine Teilmenge ${\cal B}\subseteq {\cal T}$ hei\ss t eine {\it Basis}
von ${\cal T},$ falls jedes Element von $\cal T$ sich schreiben l\"asst als
Vereinigung von Elementen aus ${\cal B}.$

(So sind zum Beispiel die offenen Kugeln $B_r(x)$ eine Basis der Topologie auf 
einem metrischen Raum.)

$\cal B$ hei\ss t eine {\it Subbasis} von $\cal T$, falls sich jedes 
$U\in {\cal T}$ als Vereinigung von endlichen Durchschnitten von Elementen aus
${\cal B}$ schreiben l\"asst.
\item[c)] F\"ur $x\in X$ hei\ss t eine Menge $\cal U$ von Umgebungen von $x$ 
eine {\it Umgebungsbasis} von $x,$ wenn jede Umgebung von $x$ ein Element von 
$\cal U$ als Teilmenge enth\"alt.
\end{itemize}}
\end{defini}
\begin{bem} {\bf Einsichtig}

{\rm Eine Teilmenge ${\cal B }\subseteq {\cal T}$ ist genau dann eine Basis 
der Topologie $\cal T$, wenn sie f\"ur jedes $x\in X$ eine Umgebungsbasis 
enth\"alt.

F\"ur jede Teilmenge $\cal B$ von ${\cal P}(X)$ gibt es genau eine Topologie,
die $\cal B$ als Subbasis besitzt. Sie ist die {\it von ${\cal B}$ erzeugte } 
Topologie, und besitzt 
$$\{U_1\cap \dots \cap U_n \mid n\in\mathbb N, U_i\in {\cal B}\}$$
als Basis.
}\end{bem}

\begin{defini} {\bf Feinheiten}

{\rm Wenn ${\cal T}_1,{\cal T}_2$ zwei Topologien auf einer Menge $X$ sind, so
hei\ss t ${\cal T}_1$ {\it feiner} als ${\cal T}_2,$ wenn ${\cal T}_2\subseteq
{\cal T}_1,$ also wenn ${\cal T}_1$ mehr offene Mengen besitzt als ${\cal T}_2.$

Die feinste Topologie auf $X$ ist also die diskrete, w\"ahrend 
$\{\emptyset, X\}$ die gr\"obste Topologie auf $X$ ist.

Zu je zwei Topologien gibt es eine gemeinsame Verfeinerung. Die gr\"obste 
gemeinsame Verfeinerung ist die Topologie, die die Vereinigung der beiden 
gegebenen als Subbasis besitzt.}
\end{defini}

\begin{defini} \label{Spurtopologie}{\bf Teilr\"aume und Produkte}

{\rm
\begin{itemize}
\item[a)] Es seien $X$ eine Menge und $(Y,{\cal S})$ ein Topologischer Raum. 
Weiter sei $f:X\longrightarrow Y$ eine Abbildung. F\"ur zwei Teilmegen
$A,B\subseteq Y$ gilt
$$f^{-1}(A\cup B) = f^{-1}(A)\cup f^{-1}(B), \ \ 
f^{-1}(A\cap B) = f^{-1}(A)\cap f^{-1}(B).$$
Das zeigt im wesentlichen bereits, dass 
$${\cal T}:= \{ f^{-1}(U) \mid U\in {\cal S}\}$$
eine Topologie auf $X$ ist. Man nennt sie die \index{Spurtopologie}
{\it Spurtopologie} auf $X$ 
(bez\"uglich $f$).

Damit k\"onnen wir unheimlich viele neue topologische R\"aume konstruieren.
(Tun Sie das!) 
\item[b)] Ist speziell $X\subseteq Y$ und $f$ die Einbettung dieser Teilmenge,
so nennt man $X$ (mit der Spurtopologie) einen {\it Teilraum} von $Y.$

Eine Teilmenge $A$ von $X$ ist genau dann offen bez\"uglich der Spurtopologie, 
wenn es eine offene Teilmenge $U$ von $Y$ gibt mit $A= U\cap X.$
\item[c)] Sind $X,Y$ zwei topologische R\"aume, so definieren wir auf 
$X\times Y$ die {\it Produkttopologie} \index{Produkttopologie}, indem wir 
als Basis die Produkte $U\times V$ f\"ur offene $U\subseteq X$ und
$V\subseteq Y$ verwenden. 
\end{itemize}
}
\end{defini}

\begin{defini} \label{Quotienten}{\bf Quotiententopologie}

{\rm Es sei $X$ ein topologischer Raum und $\equiv$ eine \"Aquivalenzrelation
auf $X.$ Dann wird auf dem Raum $X/\equiv$ der \"Aquivalenzklassen von $X$ 
eine Topologie eingef\"uhrt, indem man f\"ur offenes $U\subseteq X$ die
Menge 
$$\{[u]\mid u\in U\}$$
aller \"Aquivalenzklassen von Elementen aus $U$ zur offenen Menge erkl\"art
und die davon erzeugte Topologie verwendet. Diese Topologie hei\ss t die
{\it Quotiententopologie}\index{Quotiententopologie} auf $X/\equiv.$

Damit bekommen wir zum Beispiel eine Topologie auf dem projektiven Raum
$\mathbb P^n(\mathbb R)$ oder $\mathbb P^n(\mathbb C).$ Die Topologie auf 
$\mathbb P^1(\mathbb C)$ verdient hier historisch und didaktisch besondere
Aufmerksamkeit. Eine Teilmenge $A\subseteq\mathbb P^1(\mathbb C) = \mathbb C
\cup\{\infty\}$ ist genau dann offen, wenn $A\cap \mathbb C$ offen ist und wenn
zus\"atzlich im Fall $\infty\in A$ ein $R>0$ existiert mit 
$$\{z\in \mathbb C \mid |z|>R \} \subseteq A.$$

}

\end{defini}
\section{Wichtige Eigenschaften topologischer R\"aume}
\begin{defini} {\bf Kompaktheit}

{\rm Ein topologischer Raum $X$ hei\ss t \index{kompakt}{\it kompakt}, wenn
jede \"Uberdeckung $X=\bigcup_{i\in I}U_i$ von $X$ durch offene Mengen eine
endliche Teil\"uberdeckung enth\"alt: 
$$\exists n\in \mathbb N, i_1,\dots i_n\in I: X=\bigcup_{k=1}^n U_{i_k}.$$
Genauso hei\ss t eine Teilmenge von $X$ kompakt, wenn sie bez\"uglich der 
Spurtopologie (der Inklusion) kompakt ist.

Anstelle des Begriffs \glqq kompakt\grqq\ wird auch gelegentlich 
\glqq \"uberdeckungsendlich\grqq\ verwendet. Es ist nicht ganz einheitlich, ob 
zur Kompaktheit auch die Eigenschaft, hausdorff'sch zu sein (siehe sp\"ater), 
geh\"ort oder 
nicht. Wir wollen hier Kompaktheit so verstehen wie gesagt.
}
\end{defini}
\begin{bem} {\bf Kompakta in metrischen R\"aumen}

{\rm Ein kompakter metrischer Raum $X$ ist sicher beschr\"ankt, denn 
$$X=\bigcup_{n\in \mathbb N} B_n(x)$$
gilt f\"ur jedes $x\in X,$ und das ist eine offene \"Uberdeckung von $X.$

Eine kompakte Teilmenge $A$ eines metrischen Raums $X$ ist 
auch abgeschlossen. Ist n\"amlich $x\in X\smallsetminus A$ im Komplement von 
$A,$ so ist 
$$A\subseteq \bigcup_{n\in \mathbb N} \{y\in X\mid d(y,x)> 1/n \}$$
eine offene \"Uberdeckung von $A,$ und damit langen endlich viele dieser
Mengen, um $A$ zu \"uberdecken. Es ist also 
$$A\subseteq \{y\in X\mid d(y,x)> 1/n \}$$
f\"ur ein festes $n\in \mathbb N,$ und daher ist $B_{1/n}(x)$ in 
$X\smallsetminus A$ enthalten.

Ein etwas feineres Argument zeigt, dass ein kompakter metrischer Raum sogar
vollst\"andig ist.

Eine abgeschlossene Teilmenge $A$ eines kompakten Raums $X$ ist kompakt, denn
f\"ur jede \"Uberdeckung $\ddot U$ von $A$ durch offene Teilmengen von $X$ 
ist $\ddot U\cup\{X\smallsetminus A\}$ eine offene \"Uberdeckung von $X$, also
langen endlich viele davon, um $X$ zu \"uberdecken, und von diesen endlich
vielen kann man notfalls $X\smallsetminus A$ weglassen, um eine endliche 
Teil\"uberdeckung von $A$ zu erhalten.


}

\end{bem}

\begin{satz} {\bf \`a la Heine\footnote{Heinrich-Eduard Heine, 1821-1881}-Borel\footnote{Emile Borel, 1871-1956}}

Es sei $X$ ein vollst\"andiger metrischer Raum, in dem sich jede beschr\"ankte
Menge $A\subseteq X$ f\"ur jedes $\varepsilon>0$ durch endlich viele Mengen
von Durchmesser $\leq \varepsilon$ \"uberdecken l\"asst. 

Dann gilt:
 
Eine Teilmenge $A\subseteq X$ ist genau dann kompakt, wenn sie
abgeschlossen und beschr\"ankt ist.

\end{satz}

{\it Beweis:} In der einen Richtung haben wir es schon gesehen: ein Kompaktum 
in einem metrischen Raum ist abgeschlossen und beschr\"ankt.

Sei umgekehrt $A\subseteq X$ abgeschlossen und beschr\"ankt. Weiter sei 
$\ddot U$ eine offene \"Uberdeckung von $A.$ Nehmen wir an, es gebe in 
$\ddot U$ keine endliche Teil\"uberdeckung von $A.$

In $A$ gibt es eine abgeschlossene Teilmenge $A_1\subseteq A$ von Durchmesser
$\leq 1,$ die sich nicht durch endlich viele $U\in \ddot U$ \"uberdecken 
l\"asst, da $A$ ja nach Voraussetzung eine endliche Vereinigung von Mengen vom
Durchmesser $\leq 1$ ist.  

Wir w\"ahlen -- wieder unter Ausnutzung der Eigenschaft von $X$ -- sukzessive 
Teilmengen 
$$A\supseteq A_1\supseteq A_2\dots \supseteq A_k\supseteq \dots$$
derart, dass $A_k$ Durchmesser $\leq 1/2^k$ hat und sich nicht durch endlich 
viele $U\in \ddot U$ \"uberdecken l\"asst. 

F\"ur jedes $i\in \mathbb N$ w\"ahlen wir nun ein Element $x_i\in A_i$
(so etwas gibt es, nicht wahr?).

Dann ist $(x_i)_{i\in \mathbb N}$ eine Cauchy-Folge, denn 
$$d(x_i,x_k) \leq 1/2^{\max (i,k)}.$$

Also konvergiert die Folge gegen ein $x\in A,$ da $X$ vollst\"andig und $A$
abgeschlossen ist.

Dieses $x$ liegt also in einem $U\in\ddot U,$ und da $U$ offen ist, gibt es 
ein $\varepsilon>0,$ sodass $B_\varepsilon(x)\subseteq U$ gilt. Daher liegt f\"ur
gro\ss es $k$ auch $A_k$ ganz in $U$, was der Konstruktion der Teilmengen
$A_k$ widerspricht. 

Diese ist also nicht m\"oglich, un damit ist $A$ eben doch kompakt. 
\hfill{$\bigcirc$}


\begin{bem} \label{einige Kompakta} {\bf Beispielmaterial}

{\rm 

\begin{itemize}
\item[a)] Als Spezialfall erhalten wir den klassischen Satz von Heine-Borel,
der sagt, dass eine Teilmenge von $\mathbb R^n$ genau dann \"uberdeckungsendlich
ist, wenn sie abgeschlossen und beschr\"ankt ist. 

Heine hat diesen Satz 1872 
f\"ur Intervalle in $\mathbb R$ benutzt, um zu zeigen, dass eine stetige 
Funktion auf einem beschr\"ankten und abgeschlossenen Intervall gleichm\"a\ss 
ig stetig ist.

\item[b)]
Ein metrischer Raum, in dem diese \"Aquivalenz nicht gilt, ist zum Beispiel
der folgende:

Es sei $X:={\cal C}_0(\mathbb N)$ der Raum der beschr\"ankten Funktionen 
auf $\mathbb N$ (siehe \ref{L_unendlich}). 

F\"ur $n\in\mathbb N$ sei $\delta_n$ die Funktion auf $\mathbb N,$ die
auf $n$ den Wert 1 annimmt, und sonst den Wert 0. Die Menge
$$D:=\{ \delta_n \mid n\in \mathbb N\}$$
ist eine beschr\"ankte, abgeschlossene Teilmenge von $X.$ Aber kompakt ist sie 
nicht, denn in $B_{1/2}(\delta_n)$ liegt kein weiteres 
$\delta_k, k\in \mathbb N,$ und so ist
$$D = \bigcup_{n\in \mathbb N} B_{1/2}(\delta_n)$$
eine offene \"Uberdeckung von $D$ ohne endliche Teil\"uberdeckung.

In der Funktionalanalysis spielen \"ahnliche R\"aume eine wichtige Rolle, und
insbesondere die Frage, wann die abgeschlossene Einheitskugel in einem 
normierten Vektorraum kompakt ist.

\item[c)] Es gibt auch topologische R\"aume, in denen {\it jede Teilmenge} 
kompakt ist, egal ob offen, abgeschlossen, keins von beiden\dots

Als Beispiel hierf\"ur nehme ich eine (beliebige!) Menge $X$ und versehe sie
mit der {\it koendlichen} Topologie. Dies hei\ss t, dass neben der leeren Menge
genau die Mengen offen sind, deren Komplement in $X$ endlich ist.

Klar: hier ist alles kompakt. Denn f\"ur $A\subseteq X$ und offenes 
$U\neq\emptyset$ 
\"uberdeckt $U$ bereits alles bis auf endlich viele Elemente von $A.$

\item[d)] Eine wichtige Beispielklasse f\"ur kompakte R\"aume sind die
projektiven R\"aume $\mathbb P^n(\mathbb R)$ und $\mathbb P^n(\mathbb C)$.

Im Fall $n=1$ sieht man die Kompaktheit sehr sch\"on wie folgt: Ist $\dots U$ 
eine offene \"Uberdeckung von $X=\mathbb P^1(K)$ (mit $K=\mathbb R$ oder 
$\mathbb C$), so gibt es darin eine Menge $U_\infty\in \dots U,$ sodass
$\infty\in U_\infty.$ Das Komplement von $U_\infty$ ist nach Konstruktion der
opologie auf $X=K\cup\{\infty\}$ offen und beschr\"ankt (siehe 
\ref{Quotienten} d)) und daher kompakt wegen Heine-Borel. Also reichen endlich viele
weitere Elemente aus $\dots U$, um $K\smallsetminus U_\infty$ zu \"uberdecken.

\end{itemize}
}
\end{bem}

\begin{defini}\label{Zusammenhang} {\bf zusammenh\"angend}

{\rm Es sei $X$ ein topologischer Raum. Dann hei\ss t $X$ {\it 
zusammenh\"angend}\index{zusammenh\"angend}, wenn $\emptyset$ und $X$ die 
einzigen Teilmengen von $X$ sind, die sowohl offen als auch abgeschlossen sind.
Das ist \"aquivalent dazu, dass es keine Zerlegung von $X$ in zwei nichtleere,
disjunkte und offenen Teilmengen gibt.

Eine Teilmenge $A\subseteq X$ hei\ss t zusammenh\"angend, wenn sie bez\"uglich 
der
Teilraumtopologie zusammenh\"angend ist, also genau dann, wenn sie nicht 
in der Vereinigung zweier offener Teilmengen von $X$ liegt, deren Schnitte 
mit $A$ nichtleer und disjunkt sind.

Die Vereinigung zweier zusammenh\"angender Teilmengen mit nichtleerem
Durchschnitt ist wieder zusammenh\"angend.
}
\end{defini}

\begin{bsp} {\bf Intervalle}

{\rm Die zusammenh\"angenden Teilmengen von $\mathbb R$ sind gerade die 
Intervalle, egal ob offen oder abgeschlossen oder \dots. Dabei werden auch 
die leere Menge und einelementige Mengen als Intervalle gesehen.

Ist n\"amlich $A\subseteq \mathbb R$ zusammenh\"angend und sind 
$x<y$ beide in $A$, so liegt auch jeder Punkt $z$ zwischen $x$ und $y$ in $A$,
da sonst 
$$A = (A\cap(-\infty,z)) \bigcup (A\cap (z,\infty))$$
eine disjunkte, nichttriviale, offene Zerlegung von $A$ w\"are.

Ist umgekehrt $A$ ein Intervall, so sei $A=B\cup C$ eine nichttriviale
disjunkte Zerlegung. Ohne Einschr\"ankung gebe es ein $b_0\in B$ und ein 
$c_0\in C$ mit $b_0<c_0.$ 

Es sei $z:= \sup\{b\in B\mid b<c_0\}.$ Dies liegt in $A,$ und damit auch in 
$B$ oder $C.$ W\"are $z\in B$ und $B$ offen in $A$, so m\"usste es ein $r>0$ 
geben mit
$$\forall a\in A: |z-a| < r \Rightarrow z\in B.$$
Also kann $z$ nicht zu $B$ geh\"oren, wenn dies offen in $A$ ist, denn es gibt 
Elemente $c\in C, c>z,$ die beliebig nahe an $z$ dran liegen. 

W\"are $z\in C$ und $C$ offen in $A$, so g\"abe es ein $r>0$, sodass
$$\forall a\in A: |z-a| < r \Rightarrow z\in C.$$
Das wiederum geht nicht, denn $z$ ist das Supremum einer Teilmenge von $B.$

Also sind weder $B$ noch $C$ offen in $A$, und das zeigt, dass $A$ 
zusammenh\"angend ist.
}
\end{bsp}

\begin{defini} {\bf Zusammenhangskomponenten}

{\rm Es sei $X$ ein topologischer Raum. Wir nennen zwei Punkte $x,y$ in $X$
{\it \"aquivalent}, falls es eine zusammenh\"angende Teilmenge von $X$ gibt, 
die beide enth\"alt. Dies ist tats\"achlich eine \"Aquivalenzrelation:
\begin{itemize}
\item $x\simeq x$ ist klar f\"ur alle $x\in X,$ denn $\{x\}$ ist 
zusammenh\"angend. 
\item Symmetrie ist auch klar, nicht wahr?
\item Transitivit\"at: Es seien $x\simeq y$ und $y\simeq z$, dann gibt es 
zusammenh\"angende $A,B\subseteq X,$ sodass $x,y\in A$ und $y,x\in B$. 
Aber $A\cup B$ ist auch zusammenh\"angend, denn aus $A\cup B = U\cup V$ (offene
Zerlegung) folgt
$A=(A\cap U) \cup (A\cap V)$ und analog f\"ur $B$, wir h\"atten also 
disjunkte offene \"Uberdeckungen von $A$ und $B$, und damit folgt OBdA 
$A\subseteq U, A\cap V = \emptyset.$ Genauso ist auch $B$ in einer der beiden 
Mengen enthalten und hat mit der anderen leeren Schnitt. Aus 
$B\subseteq U$ folgt $V=\emptyset,$ w\"ahrend aus $B\subseteq V$ folgt, dass
$U$ und $V$ nicht disjunkt sind: beide enthalten $y.$
\end{itemize}
Die \"Aquivalenzklasse ovn $x$ hei\ss t die {\it Zusammenhangskomponente} von 
$x$. Diese ist weder zwangsl\"aufig offen noch zwangsl\"aufig abgeschlossen.
}
\end{defini}

\begin{defini}{\bf Hausdorff'sch}

{\rm Ein topologischer Raum $X$ hei\ss t \index{hausdorff'sch}
{\it hausdorff'sch}\footnote{Felix Hausdorff, 1868-1942}, wenn je zwei
Punkte $x\neq y$ in $X$ disjunkte Umgebungen haben.

Man sagt dann auch, $X$ erf\"ulle das Trennungsaxiom $T2:$ verschiedene Punkte
lassen sich durch Umgebungen trennen.}

\end{defini}
 
\begin{bem} {\bf Vererbung}

{\rm Wenn $X$ hausdorff'sch ist, so auch jeder Teilraum von $X.$

Jeder metrische Raum ist hausdorff'sch.

Das Produkt zweier Hausdorffr\"aume ist wieder hausdorff'sch.

Nicht hausdorff'sch ist beispielsweise ein Raum $X$ mit mindestens zwei 
Elementen und der Topologie $\{\emptyset, X\}.$
}
\end{bem}

\begin{bem}\label{fuer Liouville} {\bf Kompakta in Hausdorffr\"aumen}

{\rm Jedes Kompaktum $K$ in einem Hausdorffraum $X$ ist abgeschlossen. Denn:
Ist $x\in X\smallsetminus K$, so gibt es f\"ur jedes $k\in K$ disjunkte
offene Umgebungen $U_k$ von $k$ und $V_k$ von $x$. Es ist $\dots U := \{ U_k\mid k\in K\}$ eine offene \"Uberdeckung von $K,$ und wegen der Kompaktheit 
gibt es endlich viele $k_1,\dots ,k_n$ in $K,$ sodass 
$$K \subseteq \bigcup_{i=1}^n U_{k_i}.$$
Dazu disjunkt ist $\bigcap_{i=1}^n V_{k_i},$ aber das ist eine offene Umgebung
von $x$. Also liegt $x$ nicht im Abschluss von $K.$
}
\end{bem}


\section{Stetigkeit}

\begin{defini} {\bf Stetige Abbildungen}

{\rm Eine Abbildung $f:X\longrightarrow Y$ zwischen zwei topologischen R\"aumen
hei\ss t {\it stetig}\index{stetig}, falls f\"ur jede offene Teilmenge $U$
von $Y$ das Urbild $f^{-1}(U)$ in $X$ offen ist.

Wir hatten dies bei metrischen R\"aumen als \"aquivalent zur klassischen 
$\delta-\varepsilon$-Definition gesehen.

Wie bei metrischen R\"aumen werden wir mit ${\cal C}(X,Y)$ die Menge aller 
stetigen Abbildung zwischen den topologischen R\"aumen $X$ und $Y$ bezeichnen.

Eine stetige Abbildung, die bijektiv ist, und deren Umkehrabbildung auch
stetig ist, hei\ss t ein {\it Hom\"oomorphismus}. Zwei topologische R\"aume,
zwischen denen es einen Hom\"oomorphismus gibt, hei\ss en kreativer Weise
{\it hom\"oomorph}.
}

\begin{bem} {\bf Sysiphos\footnote{Ignatz Sysiphos, -683- -651}}

{\rm In der Topologie betrachtet man zwei hom\"oomorphe topologische R\"aume 
als
im Wesentlichen gleich. Eine Eigenschaft eines topologischen Raums $X$ 
hei\ss t eine {\it topologische Eigenschaft}, wenn jeder zu $X$ hom\"oomorphe
Raum diese Eigenschaft auch hat. Kompaktheit, Zusammenhang, Hausdorffizit\"at
sind solche Eigenschaften. Beschr\"anktheit oder Vollst\"andigkeit eines 
metrischen Raums ist keine topologische Eigenschaft.

Nat\"urlich m\"ochte man eine \"Ubersicht gewinnen,
wann zwei topologische R\"aume hom\"oomorph sind, oder welche 
Hom\"oomorphieklassen es insgesamt gibt. Das ist in dieser Allgemeinheit ein
aussichtsloses Unterfangen. Es gibt (mindestens) zwei M\"oglichkeiten, die
W\"unsche etwas abzuschw\"achen: man kann sich entweder auf etwas speziellere
topologische R\"aume einschr\"anken oder den Begriff des Hom\"oomorphismus
ersetzen.

Das erstere passiert zum Beispiel bei der Klassifikation der topologischen
Fl\"achen.

F\"ur das zweitere bietet sich der Begriff der Homotopie an.

Auf beides kommen wir sp\"ater noch zu sprechen.

Oft genug ist es sehr schwer nachzuweisen, dass zwei gegebene R\"aume nicht 
zueinander hom\"oomorph sind. Wenn ich keine bistetige Bijektion finde, sagt 
das vielleicht mehr \"uber mich aus als \"uber die R\"aume. Hier ist es 
manchmal hilfreich, topologischen R\"aumen besser greifbare Objekte aus anderen
Bereichen der Mathematik zuordnen zu k\"onnen, die f\"ur hom\"oomorphe R\"aume 
isomorph sind, und wo dies besser entschieden werden kann. Das ist eine 
Motivation daf\"ur, algebraische Topologie zu betreiben oder allgemeiner
eben Funktoren von der Kategorie der topologischen R\"aume in andere
Kategorien zu untersuchen. 
}
\end{bem}
\end{defini}

\begin{bem} {\bf Ringkampf}

{\rm 

\begin{itemize}

\item[a)] Die Identit\"at auf $X$ ist stets ein Hom\"oomorphismus (wenn man 
nicht zwei verschiedene Topologien benutzt\dots). Eine konstante Abbildung ist 
immer stetig. 

\item[b)]
Die Verkn\"upfung zweier stetiger Abbildungen $f:X\longrightarrow  Y, 
g:Y\longrightarrow Z$ ist wieder stetig. 

Insbesondere zeigt das, dass hom\"oomorph zu sein eine \"Aquivalenzrelation
auf jeder Menge von topologischen R\"aumen ist.

\item[c)]
Sind $f:X\longrightarrow Y$ und $g:X\longrightarrow Z$ stetig, so ist auch 
$f\times g:X\longrightarrow Y\times Z$ stetig bez\"uglich der Produkttopologie.
Diese ist die feinste Topologie auf $Y\times Z$ mit dieser Eigenschaft.

\item[d)]

${\cal C}(X)$ ist wieder der Raum  der stetigen reellwertigen Funktionen auf 
$X$ (wobei $\mathbb R$ bei so etwas immer mit der Standardtopologie versehen 
ist!). Dies ist wieder ein Ring (bez\"uglich der \"ublichen Verkn\"upfungen), 
denn die Addition und Multiplikation sind stetige Abbildungen von 
$\mathbb R^2$ nach $\mathbb R,$ und wir k\"onnen  b) und c) 
anwenden.
\end{itemize}
}

\end{bem}



\begin{hilfs}\label{Erhaltungssatz} {\bf Ein Erhaltungssatz}

Es sei $f:X\longrightarrow Y$ stetig. Dann gelten:
\begin{itemize}
\item[a)] Wenn $X$ kompakt ist, dann auch $f(X).$
\item[b)] Wenn $X$ zusammenh\"angend ist, dann auch $f(X).$
\item[c)] Wenn $Y$ hausdorff'sch ist und $f$ injektiv, dann ist $X$ 
hausdorff'sch.
\end{itemize}
\end{hilfs}

{\it Beweis.} 
\begin{itemize}
\item[a)] Es sei $\ddot V$ eine offene \"Uberdeckung von $f(X)$ in $V.$ Dann
ist $\ddot U:= \{f^{-1}(U)\mid U\in \ddot U\}$ eine offene \"Uberdeckung 
von $X.$ Da $X$ kompakt ist, gibt es endlich viele $V_1,\dots,V_n\in \ddot V,$
sodass bereits $\{f^{-1}(V_i)\mid 1\leq i\leq n\}$ eine \"Uberdeckung von $X$ 
ist. 

Aus $f(f^{-1}(V_i)) = f(X)\cap V_i$ folgt, dass $\{V_1,\dots ,V_n\}$ das Bild
von $f$ \"uberdecken.
\item[b)] Es sei $f(X) = A\cup B$ eine disjunkte Zerlegung von $f(X)$ in nicht 
leere Teilmengen. Wenn $A,B$ in der Spurtopologie offen w\"aren, dann g\"abe 
es offene Teilmengen $V,W$ von $Y$ mit $A=V\cap f(X), B=W\cap f(X).$

Mithin w\"are $f^{-1}(V),f^{-1}(W)$ eine offene \"Uberdeckung von $X$, die 
noch dazu disjunkt ist, da sich $V$ und $W$ nicht in $f(X)$ schneiden.

Andererseits w\"are diese Teilmengen von $X$ nicht leer (weil $A$ und $B$ nicht
leer sind), und das widerspricht der Definition von Zusammenhang.

\item[c)] Es seien $x_1\neq x_2$ Punkte in $X.$ Dann sind ihre Bilder in $Y$ 
verschieden, denn $f$ soll injektiv sein. Daher haben $f(x_1),f(x_2)$ in $Y$ 
disjunkte Umgebungen, und deren Urbilder sind disjunkte Umgebungen von $x_1$ 
und $x_2.$ \hfill{$\bigcirc$}
\end{itemize}

Die Umkehrungen gelten jeweils nat\"urlich nicht, wie einfache Gegenbeispiele 
lehren. Aber wir treffen bei n\"aherem Hinsehen 

\begin{Folgerung} {\bf alte Bekannte}

Es sei $f:X\longrightarrow \mathbb R$ stetig.

\begin{itemize}
\item[a)]
Wenn $X$ kompakt ist, dann nimmt $f$ ein Maximum und ein Minimum an.

Als Spezialfall hiervon erinnnern wir an \ref{Normen}: eine Norm $N$ auf dem 
$\mathbb R^n$ ist immer stetig bez\"uglich der Standardmetrik. Daher nimmt sie 
auf der (kompakten) Einheitssph\"are ein positives Minimum $m$ und ein Maximum
$M$ an, und das f\"uhrt wegen der Homogenit\"at der Norm zu
$$\forall x\in \mathbb R^n : m|x| \leq N(x) \leq M|x|.$$
Dies zeigt, dass $N$ und die Standardmetrik dieselbe Topologie liefern.
\item[b)]
Wenn $X\subseteq \mathbb R$ ein Intervall ist, dann ist auch $f(X)$ ein 
Intervall -- das ist der Zwischenwertsatz.\index{Zwischenwertsatz}
\phantom{.}\hfill{$\bigcirc$}
\end{itemize}
\end{Folgerung}

Insbesondere ist also ${\cal C}(X) = {\cal C}_0(X),$ wenn $X$ kompakt ist, und 
dies ist als Teilraum von ${\cal B}(X)$ ein metrischer Raum. Hier gibt es nun 
den wichtigen Satz 

\begin{satz} {\bf von Stone\footnote{Marshall Harvey Stone, 1903-1989}-Weierstra\ss\footnote{Karl Theodor Wilhelm Weierstra\ss, 1815-1897}}

Es sei $K$ ein kompakter topologischer Raum und ${\cal A}\subseteq {\cal C}(K)$
ein Teilring, der die konstanten Funktionen enth\"alt und folgende Bedingung 
erf\"ullt:

$$\forall x\neq y\in K:\exists f\in {\cal A}:f(x)=0, f(y)=1.$$

Dann ist $\cal A$ dicht in ${\cal C}(K).$

\end{satz}

{\bf Das hei\ss t:} Jede stetige Funktion auf $X$ l\"asst sich gleichm\"a\ss ig
durch eine Folge in ${\cal A}$ approximieren. 

Die Bedingung an ${\cal A}$ hat einen Namen: man sagt, ${\cal A} $ 
{\it trenne die Punkte} von $X.$
Insbesondere impliziert dies, dass $K$ hausdorff'sch ist.

Ein beliebter Spezialfall des
Satzes ist der eines Kompaktums $X\subseteq \mathbb R^n$, wobei man dann f\"ur
${\cal A}$ gerne den Ring der Polynomfunktionen (in $n$ Variablen) auf $X$ 
w\"ahlt. Klar: Schon die linearen Abbildungen langen, um Punkte zu trennen.

{\it Beweis} des Satzes. Hier folge ich den Grundz\"ugen der modernen Analysis
von Dieudonn\'e\footnote{Jean Alexandre Eug\`ene Dieudonn\'e, 1906-1992}.

Wir bezeichnen mit $\overline{\cal A}$ den Abschluss von $\cal A$ in 
${\cal C}(K).$

Wir f\"uhren den Beweis des Satzes in mehreren Schritten.

\begin{itemize}
\item[1.] Es gibt eine Folge von reellen Polynomen $u_n\in \mathbb R[X],$
die auf dem Intervall $[0,1]$ gleichm\"a\ss ig gegen die Wurzelfunktion 
konvergiert.

Um dies einzusehen setzen wir $u_1\equiv 0$ und definieren rekursiv
$$u_{n+1} (t) := u_n(t) + \frac12(t-u_n(t)^2), n\geq 1.$$
Dann ist $(u_n)$ punktweise monoton steigend (auf $[0,1]$ wohlgemerkt, nur dort
betrachten wir das) und beschr\"ankt. Punktweise gilt also (wegen des 
Monotoniekriteriums) $\lim_{n\to\infty} u_n(t) = \sqrt(t).$ Dann impliziert der 
Satz von Dini\footnote{Ulisse Dini, 1845-1918}, was in 1.\ behauptet wird.

\item[2.] F\"ur jedes $f\in {\cal A}$ geh\"ort $|f|$ zu $\overline{\cal A}.$

Denn f\"ur $a:= \max_{x\in K}|f(x)|$ ist $(u_n(f^2/a^2))$ (mit $u_n$ aus Punkt 
1.) eine Folge in $\cal A,$ die gegen $|f|$ konvergiert.

\item[3.] F\"ur $x\neq y\in K$ und $a,b\in \mathbb R$ gibt es $f\in {\cal A}$ 
mit $f(x)=a, f(y) = b.$

Denn: Es gibt ja nach Voraussetzung in ${\cal A}$ eine Funktion $g$ mit
$g(x)\neq g(y).$ Setze nun 
$$f:= \frac a{g(x)-g(y)}(g-g(y)) + \frac b{g(y)-g(x)}(g-g(x)).$$
NB: $x,y$ sind fest, die Variable versteckt sich hinter dem nackten $g.$

\item[4.] F\"ur jedes $f\in {\cal C}(K),$ jedes $x\in K$ und jedes 
$\varepsilon >0$ gibt es eine Funktion $g\in \overline{\cal A}$ derart, dass
$$g(x) = f(x),\ \ \forall y\in K: g(y)\leq f(y)+\varepsilon.$$

Zun\"achst gibt es wegen 3.\ f\"ur jedes $z\in K$ eine Hilfsfunktion 
$h_z\in {\cal A},$ sodass $h_z(x) = f(x),\ h_z(z)\leq f(z)+\frac\varepsilon 2.$
Da $h_z$ stetig ist, gibt es eine Umgebung $U_z$ von $z,$ in der 
$$h_z(y) \leq f(y)+\varepsilon, \ \ y\in U_z$$
gilt. Da $K$ kompakt ist, wird es von endlich vielen der $U_z$ \"uberdeckt, 
es gibt also $z_1,\dots , z_n\in K: K=\cup_i U_{z_i}.$

Setze nun $g(y):= inf_i h_{z_i}(y).$ 
Diese Funktion liegt wegen 
$\inf(p,q) = \frac12(p+q-|p-q|)$ in $\overline{\cal A}$ und hat die 
gew\"unschte Eigenschaft.

\item[5.] $\overline{\cal A} = {\cal C}(K).$

Es sei $f\in {\cal C}(K)$ und $\varepsilon >0.$ F\"ur jedes $x\in K$ gibt es
eine Funktion $h_x\in {\cal A}$ mit 
$$h_x(x) = f(x),\ \ \forall y\in K: h_x(y)\leq f(y)+\varepsilon.$$
Die Stetigkeit von $h_x$ zeigt, dass jedes $x\in K$ eine offene Umgebung 
$V_x$ hat mit
$$\forall y\in V_x: h_x(y)\geq f(y)-\varepsilon.$$
\"Ahnlich wie in 4.\ gibt es $x_1,\dots ,x_r\in K,$ sodass $K$ von den 
$V_x$ \"uberdeckt wird. Auch \"ahnlich wie eben liegt das Supremum $g$ der 
Funktionen $h_{x_i},1\leq i\leq r,$ in $\overline{\cal A},$ und es gilt
$$||f-g||_\infty \leq\varepsilon.$$
Das ist die Behauptung.\phantom{.}\hfill{$\bigcirc$}

\end{itemize}

\begin{defini} {\bf Wo ein Weg ist\dots}


{\rm Es sei $X$ ein topologischer Raum. Ein {\it Weg} ist eine stetige 
Abbildung eines kompakten reellen Intervalls $[a,b]$ mit $a<b$ nach $X.$

Sind $f:[a,b]\longrightarrow X$ und $[b,c]\longrightarrow X$ zwei Wege mit
$f(b) = g(b),$ so ist $g\ast f: [a,c]\longrightarrow X$ ein Weg, wenn wir
$$g\ast f(t) = \left\{ \begin{array}{ll}f(t), & t\in [a,b]\\
                                   g(t), & t\in [b,c]\\
\end{array}\right. $$
definieren. 

$X$ hei\ss t {\it wegzusammenh\"angend}, wenn es f\"ur alle $x,y\in X$
einen Weg $f$ mit $f(a) = x, f(b)=y$ gibt.

Ein wegzusammenh\"angender Raum ist zusammenh\"angend. Wegen 
\ref{Erhaltungssatz} ist ja das Bild eines Weges zusammenh\"angend, und 
wegen des letzten Satzes in \ref{Zusammenhang} ist die Verenigung der Bilder
der Wege, die bei einem festen $x\in X$ anfangen, auch zusammenh\"angend. Naja,
das zweite muss man sich vielleicht noch
einmal \"uberlegen, wir haben ja jetzt unendlich viele beteiligte Teilmengen. 
Das ist eine nette \"Ubung.}

\end{defini}



\begin{defini} {\bf Offenheit}

{\rm Eine Abbildung $f:X\longrightarrow Y$ zwischen zwei topologischen 
R\"aumen hei\ss t {\it offen}, wenn f\"ur jede offene Teilmenge $A\subseteq X$
das Bild $f(A)$ in $Y$ offen ist.

$f$ hei\ss t {\it offen in } $x\in X,$ falls jede Umgebung von $x$ unter $f$ 
auf eine Umgebung von $f(x)$ abgebildet wird.

Insbesondere ist ein Hom\"oomorphismus also eine stetige und offene Bijektion.
}
\end{defini}

\begin{bsp} \label{kte Wurzel}{\bf Vorbereitung}
 
{\rm Es sei $k>0$ ein nat\"urliche Zahl.

Auf der Menge der komplexen Zahlen ist die Abbildung $z\mapsto z^k$ eine offene 
und surjektive Abbildung, wie in der Beschreibung durch Polarkoordinaten 
ersichtlich ist.

Au\ss erdem zeigen die Polarkoordinaten, dass diese Abbildung auf 
$\mathbb C\smallsetminus\{0\}$ lokal injektiv ist. Das hei\ss t: jedes 
$z_0\neq 0$ hat eine offene Umgebung, auf der die Abbildung $z\mapsto z^k$ 
injektiv ist.

Das wiederum impliziert, dass es f\"ur jedes $z_0\in \mathbb 
C\smallsetminus\{0\}$ eine Umgebung gibt, auf der sich eine stetige und offene
$k$-te Wurzel $z\mapsto z^{1/k}$ definieren l\"asst. In Wirklichkeit ist
diese sogar reell differenzierbar.

Dies wird nun implizieren, dass komplexe nichtkonstante Polynome offen sind.
Das werden wir sp\"ater benutzen, um den Fundamentalsatz der Algebra zu 
beweisen.
}
\end{bsp}
 
\begin{hilfs} \label{Polynom-offen}{\bf komplexe Polynome}

Es sei $f:\mathbb C\longrightarrow \mathbb C$ eine polynomiale Abbildung,
die nicht konstant ist. Dann ist $f$ offen.
\end{hilfs}

{\it Beweis.}

Wir zeigen, dass das Bild einer Umgebung der $0$ unter $f$ eine Umgebung von 
$f(0)$ ist. Da Translationen in $\mathbb C$ Hom\"oomorphismen sind und aus 
Polynomen wieder Polynome machen, zeigt das, dass f\"ur jedes $z\in \mathbb C$
und jede Umgebung $U$ von $z$ die Menge $f(U)$ eine Umgebung von $f(z)$ ist,
und das ist gerade die Behauptung.

Hierbei d\"urfen wir uns auf den Fall zur\"uckziehen, dass $f(0)=0$ gilt.

Es sei also $f(z) = \sum_{i=1}^d a_iz^i.$

Wir bemerken zun\"achst, dass $f$ im Nullpunkt reell differenzierbar ist. 
Die Ableitung im Nullpunkt ist die $\mathbb R$-lineare Abbildung, die durch 
Multiplikation mit $a_1$ zustande kommt. Wegen der 
binomischen Formeln gilt hier ja
$$\lim_{|h|\to 0} \frac{f(z_0+h) - f(z_0) - a_1h }{|h|} = 0.$$

Wenn $a_1\neq 0$ gilt, dann ist die Ableitung ein Isomorphismus, und der 
Satz von der impliziten Funktion sagt, dass
es eine Umgebung $U$ von $0$ und eine Umgebung $V$ von $f(0)=0$ gibt, sodass
$f$ auf $U$ injektiv ist, $f(U) = V,$ und die lokale Umkehrabbildung zu $f$
auf $V$ differenzierbar. Das hei\ss t, dass auch $f^{-1}$ in $0$ stetig ist, 
$f$ also offen.

Es bleibt der Fall $a_1=0.$ Es sei $k=\min\{n\in \mathbb N \mid a_k\neq 0\}.$
Dann ist $k>1,$ da $a_0=a_1=0.$ Wir wollen zeigen, dass $f$ in einer Umgebung
der $0$ eine $k$-te Wurzel hat: $f(z) = g(z)^k,$ und dass $g$ offen gew\"ahlt 
werden kann. Dann sagt uns die Offenheit von $z\mapsto z^k,$ dass auch $f$ im
Nullpunkt offen ist, und wir sind fertig.

Dazu schreiben wir $f(z) = z^k\cdot h(z)$ mit 
$h(z) = \sum_{i=k}^d a_i z^{i-k}.$
Das Polynom $\tilde h$ hat also im Nullpunkt den Wert $a_k\neq 0.$
In einer Umgebung von $a_k$ gibt es wegen \ref{kte Wurzel} eine stetige, offene 
$k$-te Wurzel. Die $k$-te Wurzel $h(z)^{1/k}$ ist also in einer Umgebung der $0$
definiert, und bei n\"aherem Hinsehen sieht man, dass die Ableitung im 
Nullpunkt regul\"ar ist.

Daher gilt in einer Umgebung der $0:$ 
$$f(z) = z^k (h(z)^{1/k})^k = (z h(z)^{1/k})^k.$$ 
Das ist die $k$-te Potenz einer bei $0$ offenen Abbildung, und damit ist $f$ 
selbst im Ursprung offen.
\hfill{$\bigcirc$}

\begin{satz} \index{Liouville}\label{Liouville}{\bf \`a la Liouville}

Es seien $f:X\longrightarrow Y$ eine stetige und offene Abbildung, $X$ sei 
nichtleer und kompakt, $Y$ sei zusammenh\"angend und hausdorff'sch.

Dann ist $f$ surjektiv und insbesondere ist $Y$ auch kompakt.

\end{satz}

{\it Beweis.} Das Bild von $f$ ist offen nach Definition der Offenheit und
kompakt wegen \ref{Erhaltungssatz}. Als Kompaktum in $Y$ ist $f(X)$ 
abgeschlossen, siehe \ref{fuer Liouville}. Es ist mithin 
$Y= f(X) \cup (Y\smallsetminus f(X))$ eine Zerlegung von $Y$ als Vereinigung
zweier offener disjunkter Teilmengen. Da $f(X)$ nicht leer ist und $Y$
zusammenh\"angend ist, muss $Y\smallsetminus f(X)$ leer sein: $f$ ist 
surjektiv. \hfill{$\bigcirc$}

\begin{satz} \index{Fundamentalsatz der Algebra}\label{Fundamentalsatz} 
{\bf Fundamentalsatz der Algebra}

Es sei $f:\mathbb C\longrightarrow \mathbb C$ ein nichtkonstantes Polynom. 
Dann besitzt $f$ eine komplexe Nullstelle.

\end{satz}

{\it Beweis.} Nach Hilfssatz \ref{Polynom-offen} ist $f$ offen. Au\ss erdem 
gilt (siehe \ref{Lasso}), dass $|f(z)|$ mit $|z|$ gegen unendlich geht. 

Wir k\"onnen demnach $f$ zu einer stetigen Abbildung von 
$\mathbb P^1(\mathbb C)$ auf sich selbst fortsetzen, und man verifiziert, dass
auch die Fortsetzung offen ist. Also ist die Fortsetzung von $f$ surjektiv nach
Liouville, und es gibt ein $z\in\mathbb P^1(\mathbb C)$ mit $f(z) = 0.$
Da $z$ nicht $\infty$ sein kann (hier wird $f$ ja unendlich) ist $z\in 
\mathbb C$ wie behauptet. 
\phantom{Anfang}\hfill{$\bigcirc$}

\section{Topologische Mannigfaltigkeiten}

\begin{defini} {\bf Atlas}

{\rm Es sei $X$ ein topologischer Raum. Ein $n$-dimensionaler 
\index{Atlas}{\it Atlas} auf $X$ 
besteht aus einer offenen \"Uberdeckung $\ddot U$ von $X,$ sodass f\"ur jedes 
$U\in \ddot U$ ein Hom\"oomorphismus 
$$\varphi_U: U \longrightarrow Z(U)\subseteq \mathbb R^n$$
existiert, wobei $Z(U)$ in $\mathbb R^n$ offen ist.}

\end{defini}

Zum Beispiel besitzt jede offene Teilmenge $U$ des $\mathbb R^n$ einen Atlas; 
wir nehmen einfach $U$ selbst als \"Uberdeckung und die Identit\"at als
Kartenabbildung. 

{\bf Vorsicht:} Wir halten im Vor\"ubergehen fest, dass es nicht {\it a
priori} klar ist, dass die Dimension eines Atlas durch die Topologie auf $X$
festliegt. Das ist so, aber der Beweis ist nicht so offensichtlich.
Schlie\ss lich muss man so etwas zeigen, wie dass es f\"ur $m\neq n$ keine 
offene stetige Abbildung einer $m$-dimensionalen Kugel in eine 
$n$-dimensionale gibt.

\begin{defini} {\bf topologische Mannigfaltigkeit}

{\rm Ein topologischer Raum $X$ ist eine $n$-dimensionale {\it topologische
Mannigfaltigkeit}\index{topologische Mannigfaltigkeit}, wenn er hausdorff'sch
ist, mit einem Atlas
ausger\"ustet werden kann und eine abz\"ahlbare Basis der Topologie besitzt.
}

\end{defini}

\begin{bem} {\bf Abz\"ahlbarkeitsaxiome}

{\rm Die letzte Bedingung erm\"glicht einige Konstruktionen mit topologischen 
Mannigfaltigkeiten, die sich als sehr hilfreich erweisen. 
Sie impliziert zum Beispiel, dass jede offene \"Uberdeckung von $X$ eine
abz\"ahlbare Teil\"uberdeckung hat. 

Man nennt sie auch das {\it zweite Abz\"ahlbarkeitsaxiom}.


Der Name schreit nach einem Vorg\"anger: ein topologischer Raum erf\"ullt das 
{\it erste Abz\"ahlbarkeitsaxiom}, wenn jeder Punkt eine abz\"ahlbare 
Umgebungsbasis besitzt. Metrische R\"aume haben diese Eigenschaft zum Beispiel,
sie ist eine {\bf lokale} Bedingung, sagt sie doch nur etwas \"uber Umgebungen 
von einem jeden Punkt aus. Das werden wir im n\"achsten Hilfssatz einmal
austesten.

Wenn es eine Abz\"ahlbare Umgebungsbasis von $x$ gibt, so gibt es auch eine 
der Gestalt
$$U_1\supseteq U_2\supseteq U_3\dots$$
Das sieht man durch sukzessive Schnittbildung einer gegebenen abgez\"ahlten 
Umgebungsbasis.

Das zweite Abz\"ahlbarkeitsaxiom impliziert offensichtlich das erste.
}
\end{bem}

\begin{defini} {\bf schon wieder Folgen}

{\rm Eine Folge $(x_n)$ in einem topologischen Raum $X$ {\it konvergiert gegen}
$x\in X,$ falls in jeder Umgebung von $x$ alle bis auf endlich viele 
Folgenglieder liegen.

{\bf Vorsicht:} \underline{Der} Grenzwert ist im Allgemeinen nicht mehr 
eindeutig, also eigentlich der bestimmter Singular verboten. F\"ur die
Eindeutigkeit des Grenzwerts braucht man ein Trenungsaxiom, zum Beispiel ist 
hausdorff'sch hinreichend.

$X$ hei\ss t {\it folgenkompakt}, wenn
jede Folge in $X$ eine konvergente Teilfolge besitzt.
}
\end{defini}

\begin{hilfs} {\bf Folgen f\"ur die Folgenkompaktheit}

Es sei $X$ ein topologischer Raum.
\begin{itemize}
\item[a)] Ist $X$ kompakt und erf\"ullt das erste Abz\"ahlbarkeitsaxiom, so
ist $X$ folgenkompakt.
\item[b)] Ist $X$ folgenkompakt und metrisch, so ist $X$ auch kompakt.
\end{itemize}
\end{hilfs}

{\it Beweis.} 
\begin{itemize}
\item[a)] Es sei $(x_n)$ eine Folge in $X.$ Dann gibt es ein $x\in X$, sodass
in jeder Umgebung $U$ von $x$ f\"ur unendlich viele $n\in \mathbb N$ der Punkt
$x_n$ liegt. 

Anderenfalls lie\ss e sich f\"ur alle $x\in X$ eine Umgebung $U_x$ finden, die
nur endlich viele Folgenglieder enth\"alt, und weil 
$$X=\bigcup_{x\in X} U_x$$
eine endliche Teil\"uberdeckung hat, h\"atte man einen Widerspruch.

Nun haben wir so ein $x.$ Dieses besitzt eine abz\"ahlbare Umgebungsbasis
$$U_1\supseteq U_2 \supseteq U_3,\dots$$
und wir k\"onnen bequem eine Teilfolge $x_{n_k}$ w\"ahlen mit 
$$\forall k\in \mathbb N:n_{k+1} > n_{k} \ \ {\rm und }\ \ x_{n_k}\in U_k.$$
\item[b)] Es sei $\ddot U$ eine offene \"Uberdeckung des folgenkompakten 
metrischen Raums $X.$ 

F\"ur jedes $x\in X$ w\"ahlen wir ein $U_x\in \ddot U$ derart, dass eine der
beiden folgenden Bedingungen erf\"ullt ist:
$$B_1(x)\subseteq U_x\ \ {\rm  oder }\ \ 
\exists r(x)>0 : B_{r(x)}(x)\subseteq U_x, \forall U\in \ddot U: 
B_{2r(x)}(x)\not\subseteq U.$$
Jetzt nehmen wir an, dass $\ddot U$ keine endliche Teil\"uberdeckung besitze.
Wir starten mit einem beliebigen $x_1\in X$ und w\"ahlen 
$$x_2\in X\smallsetminus U_{x_1},\ 
x_3\in X\smallsetminus(U_{x_1}\cup U_{x_2}),\dots$$
Da $X$ folgenkompakt ist, gibt es eine Teilfolge $(x_{n_k})$, die gegen ein
$a\in X$ konvergiert. Wir w\"ahlen ein $r\in (0,1)$ derart, dass
$B_r(a)\subseteq U_a.$ Dann liegt $x_{n_k}$  f\"ur 
gro\ss es $k$ in $B_{r/5}(a),$ und es gilt 
$$d(x_{n_k},x_{n_{k+1}}) < 2r/5.$$
Andererseits zeigt
$$x_{n_k}\in B_{4r/5}(x_{n_k})\subseteq B_{r}(a)\subseteq U_a\in \ddot U,$$
dass $r(x_{n_k})\geq 2r/5,$ und damit auch 
$$d(x_{n_k},x_{n_{k+1}}) \geq 2r/5.$$
Dieser Widerspruch besiegelt das Schicksal unserer irrigen Annahme, $\ddot U$
habe keine endliche Teil\"uberdeckung. 

Also ist $X$ kompakt, da $\ddot U$ beliebig war.\hfill{$\bigcirc$}
\end{itemize}

\begin{bsp} {\bf Sch\"onheiten des Abendlandes}

{\rm Nach diesem Grundlagenexkurs kehren wir nun zu den topologischen 
Mannigfaltigkeiten zur\"uck. Wir kennen noch keine Beispiele. Oder doch?

\begin{itemize}
\item[a)] Jede offene, nichtleere Teilmenge von $\mathbb R^n$ ist eine
$n$-dimensionale topologische Mannigfaltigkeit. Hier muss man vor allem das
zweite Abz\"ahlbarkeitsaxiom testen. {\bf Tun Sie das!}

Jeder Hausdorffraum mit einem endlichen Atlas ist dann auch eine topologische 
Mannigfaltigkeit.

\item[b)] Es sei $K=\mathbb R$ oder $\mathbb C.$ Dann ist der projektive Raum
$\mathbb P^n(K)$ mit der fr\"uher eingef\"uhrten Quotiententopologie eine 
topologische Mannigfaltigkeit.

Denn er l\"asst sich \"uberdecken durch die offenen Mengen 
$$U_k:=\{(x_i)_{1\leq i\leq n+1} \mid x_k = 1\},\ 1\leq k\leq n+1,$$
und diese werden beim Quotientenbilden mit ihrem Bild auch topologisch 
identifiziert, liefern also einen endlichen Atlas von $\mathbb P^n(K).$

\item[c)] Keine topologische Mannigfaltigkeit ist zum Beispiel der folgende
Raum, obwohl er einen endlichen Atlas hat: Wir nehmen die Einheitssph\"are 
$S^1\subseteq \mathbb C$ und definieren $X=S^1/\simeq,$
wobei die \"Aquivalenzrelation $\simeq$ durch $x\simeq -x$ f\"ur $x\neq 
\pm1$ definiert ist. 

Ein offener Halbkreis wird hierbei injektiv nach $X$ abgebildet, und wir 
erhalten einen sch\"onen Atlas, von dem sogar zwei Karten gen\"ugen. Aber $X$ 
ist nicht hausdorff'sch, weil die Klassen von $\pm 1$ sich nicht durch offene
Umgebungen trennen lassen.
\end{itemize}
}
\end{bsp}





%\input Stichworte.tex

\end{document}


