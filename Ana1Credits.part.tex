\chapter{Credits für Analyis I} Abgetippt haben die folgenden Paragraphen:\\% no data in Ana1Begriffe.tex
% no data in Ana1Vorwort.tex
\textbf{§ 1: Reelle Zahlen}: Joachim Breitner\\
\textbf{§ 2: Natürliche Zahlen}: Joachim Breitner\\
\textbf{§ 3: Folgen, Abzählbarkeit}: Joachim Breitner\\
\textbf{§ 4: Wie Sie Wollen}: Joachim Breitner, Pascal Maillard\\
\textbf{§ 5: Wurzeln und rationale Exponenten}: Jonathan Picht, Joachim Breitner\\
\textbf{§ 6: Konvergente Folgen}: Joachim Breitner, Pascal Maillard\\
\textbf{§ 7: Wichtige Beispiele}: Joachim Breitner\\
\textbf{§ 8: Häufungswerte und Teilfolgen}: Joachim Breitner, Manuel Holtgrewe\\
\textbf{§ 9: Oberer und unterer Limes}: Joachim Breitner\\
\textbf{§ 10: Das Cauchy-Kriterium}: Joachim Breitner, Pascal Maillard\\
\textbf{§ 11: Unendliche Reihen}: Pascal Maillard\\
\textbf{§ 12: Konvergenzkriterien}: Joachim Breitner\\
\textbf{§ 13: Umordnungen und Produkte von Reihen}: Pascal Maillard\\
\textbf{§ 14: Potenzreihen}: Wenzel Jakob\\
\textbf{§ 15: $g$-adische Entwicklungen}: Joachim Breitner\\
\textbf{§ 16: Grenzwerte bei Funktionen}: Joachim Breitner\\
\textbf{§ 17: Stetigkeit}: Joachim Breitner\\
\textbf{§ 18: Eigenschaften stetiger Funktionen}: Wenzel Jakob, Joachim Breitner\\
\textbf{§ 19: Funktionsfolgen und -reihen}: Joachim Breitner und Wenzel Jakob\\
\textbf{§ 20: Gleichmäßige Stetigkeit}: Wenzel Jakob\\
\textbf{§ 21: Differenzierbarkeit}: Joachim Breitner, Pascal Maillard und Wenzel Jakob\\
\textbf{§ 22: Höhere Ableitungen}: Joachim Breitner, Pascal Maillard\\
\textbf{§ 23: Das Riemann-Integral}: Pascal Maillard, Wenzel Jakob und Joachim Breitner\\
\textbf{§ 24: Uneigentliche Integrale}: Pascal Maillard\\
\textbf{§ 25: Funktionen von beschränkter Variation}: Wenzel Jakob\\
\textbf{§ 26: Das Riemann-Stieltjes-Integral}: Pascal Maillard und Wenzel Jakob\\
