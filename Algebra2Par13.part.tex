\section{Flache Moduln}

\begin{Bem}
  Für jeden $R$-Modul $M$ ist die Zuordnung $M \mapsto M \ten[R] N$ ein Funktor
  \[
  \ten[R] N: \KatRMod \to \KatRMod
  \]
\end{Bem}

\begin{Bew} 
  Ist $\varphi: M \to M'$ $R$-linear, so setze $\varphi_N: M \ten[R] N \to M'
  \ten[R] N, x \ten y \mapsto \varphi(x) \ten y, \displaystyle
  \sum_{i=0}^na_i(x_i \ten y_i) \mapsto \sum_{i=0}^n a_i(\varphi(x_i) \ten y_i)$
\end{Bew}

\begin{Prop}
\label{1.12}
  Der Funktor $\ten[R] N$ ist rechtsexakt, d.h. ist $0 \to M'
  \overset{\varphi}{\to} M \overset{\Psi}{\to} M'' \to 0$ exakt, so ist $ M'
  \ten[R] N \overset{\varphi_N}{\to} M \ten[R] N \overset{\Psi_N}{\to} M'' \ten[R] N
  \to 0$ exakt.
\end{Prop}

\begin{nnBsp} 
  $R = \ZZ, N = \ZZ/2 \ZZ$\\
  $0 \to \ZZ \overset{ \cdot 2}{\to} \ZZ \to \ZZ/2
  \ZZ \to 0$
  \[
  \begin{matrix}
  \varphi_N:& \ZZ/2 \ZZ &\to& \ZZ / 2 \ZZ &(\cong \ZZ \ten[\ZZ] \ZZ / 2 \ZZ
  \text{ nach 1.9a})\\
  &x\ten\overline 1 & \mapsto & \varphi(x)\ten\overline 1&\\
  &1\ten\overline 1&\mapsto&2\ten\overline 1=0\
  \end{matrix}
  \]
  $\Rightarrow \varphi_N$ ist nicht injektiv.
\end{nnBsp}

\begin{Bew} 
  \textbf{1. Schritt:} $\B{\varphi_N} \subseteq \K{\Psi_N}$,
  denn: $\Psi_N(\varphi_N(x \ten y)) = \Psi_N(\varphi(x) \ten y) =
  \underset{=0}{\underbrace{\Psi(\varphi}}(x)) \ten y = 0$. Homomorphiesatz
  liefert ein $\bar{\Psi}: M \ten[R] N/\B{\varphi_N} \to M'' \ten[R]
  N$.\\
  \textbf{2. Schritt:} $\bar{\Psi}$ ist Isomorphismus.\\
  Dann ist $\bar{\Psi}$ und damit $\Psi_N$ surjektiv und $\K{\Psi_N} =
  \B{\varphi_N}$.\\
  Konstruiere Umkehrabbildung $\sigma: M'' \ten[R] N \to \bar{M} \defeqr M
  \ten[R] N/\B{\varphi_N}$. Wähle zu jedem $x'' \in M''$ ein Urbild
  $\chi(x'') \in \Psi^{-1}(x'') \subset M$.
  Definere $\tilde{\sigma}: M'' \times N \to \bar{M}$ durch $(x'', y) \mapsto
  \chi(x'') \ten y$\\
  $\tilde{\sigma}$ wohldefiniert:
  Sind $x_1,x_2 \in M$ mit $\Psi(x_1) = \Psi(x_2) = x''$, so ist $\underset{= \varphi(x')
  }{\underbrace{x_1 - x_2}} \in \B{\varphi} \Rightarrow \overline{x_1
  \ten y} - \overline{x_2 \ten y} = \underset{\in \B{\varphi_N}
  }{\underbrace{\overline{\varphi(x') \ten y}}} = 0$\\
  Rest klar!!
\end{Bew}

% ---

\begin{DefProp}
\label{1.13}
  Sei $N$ ein $R$-Modul.
  \begin{enumerate}
    \item $N$ hei\ss t \emp{flach}\index{R-Modul!flacher}, wenn der Funktor $\ten[R] N$ exakt ist,
    d.h. f\"ur jede kurze exakte Sequenz von $R$-Moduln 
    $0\to M'\to M\to M''\to 0$
    auch $0\to M'\ten[R] N\to M\ten[R] N\to M''\ten[R] N\to 0$ exakt ist.
    \item $N$ ist genau dann flach, wenn f\"ur jeden $R$-Modul $M$ und jeden Untermodul $M'$ von $M$
    die Abbildung $i:M'\ten[R] N\to M\ten[R] N$ injektiv ist.
    \item Jeder projektive $R$-Modul ist flach.
    \item Ist $R=K$ ein K\"orper, so ist jeder $R$-Modul flach.
    \item F\"ur jedes multiplikative Monoid $S$ ist $R_S$ flacher $R$-Modul.
  \end{enumerate}
\end{DefProp}

\begin{Bew}
\begin{enumerate}
\item[(b)] folgt aus \ref{1.12}
\item[(e)] Sei $M$ ein $R$-Modul, $M'\subseteq R$ Untermodul.
Nach \"U2A4 ist $M\ten[R] R_S \cong M_S$.\\
Zu zeigen: Die Abbildung $M'_S\to M_S, \frac{a}{s}\to \frac{a}{s}$ ist injektiv. \\
Sei also $a\in M'$ und $\frac{a}{s}=0$ in $M_S$, d.h. in $M$ gilt: $t\cdot a=0$ f\"ur ein $t\in S$.
$\Rightarrow t\cdot a = 0$ in $M'\Rightarrow \frac{a}{s}=0$ in $M'_S$.
\item[(d)] folgt aus (c), weil jeder $K$-Modul frei ist, also projektiv.
\item[(c)] Sei $N$ projektiv. Nach Prop. \ref{1.6} gibt es einen $R$-Modul
$N'$, sodass $N \oplus N'\defeql F$ frei ist.\\ \textbf{Beh. 1}: $F$ ist flach.\\
Dann sei $M$ $R$-Modul, $M'\subseteq M$ Untermodul; dann ist $F\ten[R] M'\to F\ten[R] M$ injektiv.\\
\textbf{Beh. 2}: Tensorprodukt vertauscht mit direkter Summe.\\
Dann ist
\[
\begin{array}{ccccc}
M'\ten[R] F & \cong M'\ten[R](N\oplus N')   &\cong (M'\ten[R] N)&\oplus &(M'\ten[R] N') \\
\downarrow &                                 &\downarrow               &   &\downarrow\\
M\ten[R] F &                               &\cong (M\ten[R] N)&\oplus &(M\ten[R] N')\\
\end{array}
\]
Die Abbildung $M'\ten[R] F\to M\ten[R] F$ bildet $M'\ten[R] N$ auf $M\ten[R] N$ ab,

$M'\ten[R] N\to M\ten[R] N$ ist also als Einschr\"ankung einer injektiven Abbildung selbst injektiv.\\
\textbf{Bew. 1}: Sei $\{e_i:i\in I\}$ Basis  von $F$, also $F=\bigoplus_{i\in I} R e_i\cong \bigoplus_{i\in I} R$.
Wegen Beh. 2 ist $M\ten[R] F\cong M\ten[R] \bigoplus_{i\in I}R\cong \bigoplus_{i\in I}(M\ten[R] R)=\bigoplus_{i\in I}M$.
Genauso: $M' \ten[R] F\cong \bigoplus_{i\in I}M'$.\\
Die Abbildung $M'\ten[R] F\to M\ten[R] F$ ist in jeder Komponente die Einbettung $M'\hookrightarrow M$, also injektiv.\\
\textbf{Bew. 2}: Sei $M=\bigoplus_{i\in I} M_i$, zu zeigen: $M\ten[R] N \cong \bigoplus_{i\in I}(M_i\ten[R] N)$.\\
Die Abbildung $M\times N \to \bigoplus_{i\in I} (M_i\ten[R] N), \left((x_i)_{i \in I}, y\right)\mapsto (x_i\ten y)_{i\in I}$ ist bilinear, induziert
also eine $R$-linare Abbildung $\varphi: M\ten[R] N\to \bigoplus_{i\in I}(M_i\ten[R] N)$.\\
Umgekehrt: F\"ur jedes $i\in I$ induziert $M_i\hookrightarrow M$ $\psi_i : M_i\ten[R] N\to M\ten[R] N$;
die $\psi_i$ induzieren $\psi:\bigoplus_{i\in I}(M_i\ten[R] N)\to M\ten[R] N$ (UAE der direkten Summe).
``Nachrechnen'': $\phi$ und $\psi$ sind zueinander invers. 
\end{enumerate}
\end{Bew}
