\documentclass{article}
\newcounter{chapter}
\setcounter{chapter}{12}
\usepackage{ana}
\def\gdw{\equizu}
\def\Arg{\text{Arg}}
\def\MdD{\mathbb{D}}
\def\Log{\text{Log}}
\def\Tr{\text{Tr}}
\def\abnC{\ensuremath{[a,b]\to\MdC}}
\def\wegint{\ensuremath{\int\limits_\gamma}}
\def\iint{\ensuremath{\int\limits}}
\def\ie{\rm i}
\def\Rand{\partial}
\def\Aut{\text{Aut}}
\theoremstyle{nonumberbreak}
\newtheorem{beachte}{Beachte}

\title{Das Schwarzsche Lemma} % Sorry, latexki blickt das nicht mit: ; $\text{Aut}(\MdD)$}
\author{Lars Volker} % Wer nennenswerte Änderungen macht, schreibt euch bei \author dazu

\begin{document}
\maketitle

$\MdD := \{ z\in\MdC: |z|<1\}$.

\begin{satz}[Schwarzsches Lemma]
Es sei $f\in H(\MdD), f(\MdD) \subseteq \MdD$ und $f(0) = 0$. \\
Dann: \\  \centerline{$|f(z)| \leq |z| \forall z \in \MdD$ und $|f'(0)|\leq 1$ $(*)$. } \\  \\ Ist $|f'(0)|=1$ oder $|f'(z_0)| = |z_0|$ für ein $z_0 \in \MdD \backslash \{0\}$, so ex. ein $\lambda \in \Rand \MdD$ mit: $f(z)=\lambda z$.
\end{satz}
\begin{beweis}
%FIXME: da muss statt dem ersten \neq ein nichtidentisch-null hin
O.b.d.A.: $f\not\equiv 0$. 11.8 $\Rightarrow \exists g \in H(\MdD): f(z) = zg(z)$. Sei $z \in \MdD$. Wähle $r>0$ so, dass $r<1$ und $|z| < r$. Dann: $|g(z)| \stackrel{\text{11.7}}{\leq} \max\limits_{|w|=r} |g(w)| = \max\limits_{|w|=r} \frac{|f(w)|}{|w|} \leq \frac{1}{r} $ $\stackrel{r \rightarrow 1}{\Rightarrow} |g(z)| \leq1$. Also $|g(z)| \leq 1 \forall z \in \MdD$. 
$f'(z) = g(z) + zg'(z) \Rightarrow f'(0) = g(0)$ Also gilt $(*)$.
Es sei $|f'(0)| = 1$ oder $|f(z_0)| = |z_0|$ für ein $z_0 \in \MdD \backslash \{0\} \Rightarrow  |g(0)| = 1$ oder $|g(z_0)| = 1 \Rightarrow |g|$ hat ein Maximum in $\MdD$. 11.6 $\Rightarrow$ g konstant $\Rightarrow \exists \lambda \in \MdC: g(z) = \lambda \forall z \in \MdD$. Dann: $f(z) = \lambda z$. Es ist $|\lambda| = |g(0)| = 1$ oder $|\lambda| = |g(z_0)| = 1 \Rightarrow \lambda \in \partial \MdD$.
\end{beweis}

\begin{definition}
Sei $a\in\MdD$ und $S_a \in H(\MdC \backslash \{ \frac{1}{\overline{a}}\})$ def. durch $S_a(z) := \frac{z-a}{1-\overline{a}z}$
\end{definition}

\begin{beachte}
$|\frac{1}{\overline{a}}| = \frac{1}{|a|} > 1$, also $\frac{1}{\overline{a}} \notin \overline{\MdD}$. $S_a(a) = 0$, $S_a(0) = -a$.
\end{beachte}

\begin{satz}
Sei $a \in \MdD$. Dann:
\begin{liste}
\item $S_a$ ist auf $\MdC \backslash \{\frac{1}{\overline{a}}\}$ injektiv.
\item $S_a^{-1} = S_{-a}$ auf $\overline{\MdD}$
\item $S_a(\partial \MdD) = \partial \MdD$
\item $S_a(\MdD) = \MdD$
\item Ist $\lambda \in \partial \MdD$, so ist $\lambda S_a \in \Aut(\MdD)$.
\end{liste}
\end{satz}

\begin{beweis}
\begin{liste}
\item[(1)] Nachrechnen.
\item[(2)] $w = S_a(z) = \frac{z-a}{1-\overline{a}z} \gdw z-a = w - \overline{a}zw \gdw z(1+\overline{a}w) = w+a \gdw z = \frac{w+a}{1+\overline{a}w} = S_{-a}(w)$
\item[(3)] Sei $|z| = 1$, also $z=e^{it} (t\in\MdR)$.$|S_a(z)| = |\frac{e^{it} - a}{1-\overline{a}e^{it}}| = |\frac{e^{it} -a}{e^{it}(e^{-it}-\overline{a})}| = \frac{|e^{it} -a|}{|e^{it}||\overline{e^{it}-a}|} = 1$. Also: $S_a(\partial \MdD) \subseteq \partial \MdD$, $\partial \MdD \stackrel{(2)}{=} S_a( \underbrace{S_{-a}(\partial \MdD)}_{\stackrel{\text{wie oben}}{\subseteq \partial \MdD}} ) \subseteq S_a(\partial \MdD)$.
\item[(4)] Sei $z \in \MdD$. $|S_a(z)| \stackrel{\text{11.7}}{\leq} \max\limits_{|w|=1}|S_a(w)| \stackrel{(3)}{=} 1 \Rightarrow S_a(\MdD) \subseteq \overline{\MdD}$
Sei $z \in \MdD$, $w:=S_a(z)$. Annahme: $|w|=1 \stackrel{(3)}{\Rightarrow} |z| = |S_{-a}(w)| = 1$ Wid.
Also $S_a(\MdD) \subseteq \MdD$. Genauso $S_{-a}(\MdD) \subseteq \MdD$. Dann $\MdD \stackrel{(2)}{=} S_a(S_{-a}(\MdD)) \subseteq S_a(\MdD)$
\item[(5)] folgt aus (1) und (4).
\end{liste}
\end{beweis}

\begin {satz}
 Sei $f \in H(\MdD)$, $f \in \mbox{Aut}(\MdD)$ und $f(0)=0 \equizu \exists \lambda \in \partial \MdD:\ f(z) = \lambda z$.
\end{satz}
\begin{beweis}
  "$\Leftarrow$": Klar \\
  "$\Rightarrow$": Dann $f^{-1} \in \mbox{Aut}(\MdD)$, $f^{-1}(0) = 0$. Sei $z \in \MdD$, $w := f(z)$; dann: \\
   $z = f^{-1}(w)$, $|z| = |f^{-1}(w)| \stackrel{12.1}{\leq} |w| = |f(z)|\stackrel{12.1}{\leq} |z|$ Also 
   $|f(z)| = |z|\  \forall z \in \MdD$. $12.1 \Rightarrow \exists \lambda \in \partial \MdD:\ f(z) = \lambda z\ \forall z \in \partial \MdD$
\end{beweis}

\begin{satz}
 $\mbox{Aut}(\MdD) = \{ \lambda S_a: \lambda \in \partial \MdD, a \in \MdD\}$
\end{satz}
\begin{beweis}
 "$\supseteq$": 12.2 (5)\\
 "$\subseteq$": Sei $f \in \mbox{Aut}(\MdD)$, $a := f^{-1}(0) \in \MdD$. 
 $g := f \circ S_a$; $g \in \mbox{Aut}(\MdD)$ und $g(0) = f(S_a(0)) = f(a) = 0$. 
 12.3 $\Rightarrow \exists \lambda \in \partial \MdD: g(z) = \lambda z$. Es ist $f = g \circ S_a = \lambda S_a$
\end{beweis}


%In diesem Paragraphen sei $G \subseteq \MdC$ stets ein \begriff{Gebiet}. Fast wörtlich wie in Analysis I zeigt man:
%
%\begin{satz}[Identitätssatz für Potenzreihen]
%$\sum\limits_{n=0}^{\infty}a_n(z-z_0)^n$ sei eine Potenzreihe mit Konvergenzradius $r>0$, \\
%es sei $f(z)=\sum\limits_{n=0}^{\infty}a_n(z-z_0)^n$ für $z \in U_r(z_0)$, es sei $(z_k)$ eine \\
%Folge in $\dot U_r(z_0)$ mit $z_k \to z_0$ und es gelte $f(z_k) = 0$  $\forall$ $k$ $\in$ $\MdN$. \\
%Dann: $a_n = 0$ $\forall$ $n$ $\in$ $\MdN_0$.
%\end{satz}
%
%\begin{satz}[Identitätssatz für holomorphe Funktionen]
%Es sei $f \in H(G)$, $z_0 \in G$, $(z_k)$ eine Folge in $G\backslash\{z_0\}$ mit $f(z_k) = 0$ $\forall$ $k$ $\in$ $\MdN$\\
%und $ z_k \to z_0$.\\
%Dann: $f = 0$ auf $G$.
%\end{satz}
%
%\begin{beweis}
%$\exists r > 0$: $U_r(z_0) \subseteq G$. 10.4 $\Rightarrow f(z) = \sum\limits_{n=0}^{\infty} \frac{f^{(n)}(z_0)}{n!}(z-z_0)^n$ $\forall$ $z$ $\in$ $U_r(z_0)$\\
%$\exists k_0 \in \MdN$: $z_k \in U_r(z_0)$ $\forall$ $k$ $\geq$ $k_0$. 11.1 $\Rightarrow f^{(n)}(z_0) = 0$ $\forall$ $n \in \MdN_0$\\
%$\Rightarrow z_0 \in A := \{z \in G: f^{(n)}(z) = 0$ $\forall$ $n$ $\in$ $\MdN_0\}$. $B:= G\backslash A$, $A \cap B = \emptyset$\\
%Sei $ a \in A$. $\exists \delta > 0: U_{\delta}(a) \subseteq G$. 10.4 $\Rightarrow f(z) = \sum\limits_{n=0}^{\infty} \frac{f^{(n)}(a)}{n!}(z-a)^n$ $\forall$ $z$ $\in$ $U_{\delta}(a)$\\
%$a \in A \Rightarrow f^{(n)}(a) = 0$ $\forall$ $n$ $\in$ $\MdN_0$ $\Rightarrow f \equiv 0$ auf $U_{\delta}(a)$\\
%$\Rightarrow f^{(n)} \equiv 0$ auf $U_{\delta}(a)$ $\forall$ $n$ $\in$ $\MdN_0$\\
%$\Rightarrow U_{(\delta)}(a) \subseteq A$. $A$ ist also offen. Sei $b \in B$. $\exists k \in \MdN_0: f^{(k)}(b) \neq 0$; \\
%$f^{(k)}$ stetig $\Rightarrow \exists \epsilon > 0: U_{\epsilon}(b) \subseteq G$ und $f^{(k)}(z) \neq 0$ $\forall$ $z \in U_{\epsilon}(b)$\\
%$\Rightarrow U{\epsilon}(b) \subseteq B$; d.h. $B$ ist offen. $G$ ist ein Gebiet $\Rightarrow B = \emptyset \Rightarrow G = A \Rightarrow$ Beh.
%\end{beweis}
%
%\textbf{Bezeichnung} \\
%für $f \in H(G)$: $Z(f) := \{z\in G: f(z) = 0\}$.
%
%\begin{folgerung}
%\begin{liste}
%\item Ist $f \in H(G)$, $f \not\equiv$ %%nicht identisch!!!!!%% 
%$0$ auf $G$ und $z_0 \in Z(f)$, so existiert ein $\epsilon > 0$: 
%\\$U_{\epsilon}(z_0) \subseteq G$, $f(z) \neq 0$ $\forall$ $z$ $\in$ $\dot U_{\epsilon}(z_0)$
%\item Ist $f \in H(G)$, $z_0 \in G$ und gilt: $f^{(n)}(z_0) = 0$ $\forall$ $n$ $\in$ $\MdN_0$, so ist $f = 0$ auf $G$. 	
%\end{liste}
%\end{folgerung}
%
%\begin{beweis}
%\begin{liste}
%\item folgt aus 11.2
%\item Verfahre wie im Beweis von 11.2
%\end{liste}
%\end{beweis}
%
%\begin{satz}
%Sei $G$ ein EG und $f \in H(G)$ mit $Z(f) = \emptyset$
%\begin{liste}
%\item $\exists h \in H(G)$: $e^h = f$ auf $G$
%\item Ist $n \in \MdN$, so existiert ein $g \in H(G)$: $g^n = f$ auf $G$
%\end{liste}
%\end{satz}
%
%\begin{beweis}
%\begin{liste}
%\item Es ist $\frac{f'}{f} \in H(G)$. $G$ ist ein EG $\Rightarrow \exists F \in H(G)$: $F'= \frac{f'}{f}$ auf $G$. $\phi := \frac{e^F}{f}$.\\
%Dann: $\phi \in H(G)$ und $f' = 0$ auf $G$. (nachrechnen!)\\
%$\exists c \in \MdC$: $e^F = c \cdot f$ auf $G$.\\
%Klar: $c \neq 0$. 7.1 $\Rightarrow \exists a \in \MdC$: $c = e^a \Rightarrow f = e^{F - a}$ auf $G$.
%\item Sei $h$ wie in (1), $g := e^{\frac{1}{n} h}$. Dann: $g^n = e^h = f$ auf $G$.
%\end{liste}
%\end{beweis}
%
%\begin{satz}
%Sei $D \subseteq \MdC$ offen.
%\begin{liste}
%\item Ist $F \in H(D)$, $0 \in D$, $F(0) = 0$ und $F'(0) \neq 0$, so gilt: $0 \in F(D)°$
%\item Ist $f \in H(D)$ \begriff{nicht} konstant, so ist $f(D)$ offen.
%\item \begriff{Satz von der Gebietstreue:}\\
%Ist $f \in H(G)$ \begriff{nicht} konstant, so ist $f(G)$ ein Gebiet.
%\end{liste}
%\end{satz}
%
%\begin{beweis}
%\begin{liste}
%\item $u := \Re F$, $v := \Im F$. 4.1 $\Rightarrow u_x(0) = v_y(0), u_y(0) = - v_x(0)$ \\
%und $F'(0) = u_x(0) + i v_x(0)$\\
%$\Rightarrow det 
%\left( \begin{array}{ccc}
%u_x(0) & u_y(0) \\
%v_x(0) & v_y(0) \\
%\end{array} \right)
%= u_x(0)^2 + v_x(0)^2 = |F'(0)|^2 \neq 0$\\
%Umkehrsatz (Analysis II) $\Rightarrow \exists U \subseteq D$: $0 \in U$, $U$ ist offen und $F(U)$ ist offen.
%$F(0) = 0 \Rightarrow 0 \in F(U) \Rightarrow \exists \delta > 0$: $U_{\delta}(0) \subseteq F(U) \subseteq F(D)$.
%\item Sei $w_0 \in f(D)$. z.z. $\exists \delta > 0$: $U_{\delta}(w_0) \subseteq f(D)$.\\
%O.B.d.A. $w_0 = 0$. $\exists z_0 \in D$: $f(z_0) = w_0 = 0$. O.B.d.A. $z_0 = 0$.\\
%Also: $f(0) = 0$. $\exists \varepsilon > 0$: $U_{\varepsilon}(z_0) \subseteq D$.\\
%10.4 $\Rightarrow f(z) = a_0 + a_1 z + a_2 z^2 + \dots$  $\forall z \in U_{\varepsilon}(0)$;\\
%$f(0) = 0 \Rightarrow a_0 = 0$. 11.3 $\Rightarrow \exists n \in \MdN$: $a_n \neq 0$\\
%$m:= min \{n \in \MdN : a_n \neq 0 \}$ ($\geq 1$)\\
%Dann: $f(z) = z^m (a_m + a_{m+1} z + a_{m+2} z^2 + \dots) = z^m \cdot g(z)$  $\forall z \in U_{\varepsilon}(0)$,\\
%wobei $g \in H(U_{\varepsilon}(0))$ und $g(0) = a_m \neq 0$.\\
%$g$ stetig $\Rightarrow \exists r \in (0,\varepsilon)$: $g(z) \neq 0$ $\forall z \in U_r(0)$\\
%$U_r(0)$ ist ein EG $\stackrel{11.4}{\Rightarrow} \exists h \in H(U_r(0))$: $h^m = g$ auf $U_r(0)$\\
%Def. $F \in H(U_r(0))$ durch $F(z) := z h(z)$. \\
%Dann: $F(0)=0$, $F'(z) = h(z) + zh'(z)$\\
%also $F'(0)^m = h(0)^m = g(0) \neq 0$, also $F'(0) \neq 0$.\\
%Weiter: $F^m = f$ auf $U_r(0)$. (1)$\Rightarrow \exists R > 0$: $U_R(0) \subseteq F(U_r(0))$.\\
%$\delta := \MdR^m$. Sei $w \in U_{\delta}(0)$. 1.5 $\Rightarrow \exists v \in \MdC$: $v^m = w$\\
%Dann: $|v|^m = |w| < \delta = \MdR^m \Rightarrow |v| <R \Rightarrow v \in U_R(0) \subseteq F(U_r(0))$\\
%$\Rightarrow \exists z \in U_r(0) \subseteq D$ mit: $F(z) = v$.\\
%$\Rightarrow w = v^m = F(z)^m = f(z) \in f(D)$\\
%Also: $U_{\delta}(0) \subseteq f(D)$
%\item 3.6 $\Rightarrow f(h)$ ist zusammenhängend $\stackrel{(2)}{=} f(G)$ ist ein Gebiet.
%
%\end{liste}
%\end{beweis}
%


\end{document} 
