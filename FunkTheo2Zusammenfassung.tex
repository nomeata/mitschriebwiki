\documentclass[11pt]{article}

\usepackage{latexki}

\lecturer{Dr. Herzog}
\title{Funktionentheorie 2 -- Zusammenfassung}
\semester{?}
\scriptstate{partial}

\usepackage[T1]{fontenc}
\usepackage[utf8]{inputenc}
\usepackage{ae} 
\usepackage[english,ngerman]{babel} 
\usepackage{epsf, amsmath, amssymb, graphicx, epsfig, hyperref, amsthm}
\usepackage{graphicx}
\usepackage{amsmath}
\usepackage{geometry}
\geometry{a4paper,tmargin=25mm,bmargin=30mm,lmargin=15mm,rmargin=20mm}
\usepackage{bbold}
\usepackage{extarrows}

\setlength{\parindent}{0pt} %Setzt den Erstzeileneinzug auf Null.
\setlength{\parskip}{\baselineskip} %Trennt Absätze durch eine zusätzliche Leerzeile.

\newcommand{\C}{\mathbb{C}}
\newcommand{\K}{\mathbb{K}} 
\newcommand{\R}{\mathbb{R}}
\newcommand{\Q}{\mathbb{Q}} 
\newcommand{\Z}{\mathbb{Z}} 
\newcommand{\N}{\mathbb{N}}
\newcommand{\D}{\mathbb{D}}
\newcommand{\T}{\mathbb{T}}

\begin{document}

\begin{center}
{\Large \textbf{Funktionentheorie II - Formelsammlung}} \\
\textit{Skript Dr. Herzog}
\end{center}

\underline{§ 1 Schreibweisen und Wiederholung}

\underline{§ 2 Der Satz von Montel}

\underline{§ 3 Der Riemann'sche Abbildungssatz}

\underline{§ 4 Automorphismen spezieller Gebiete} 

\underline{§ 5 Harmonische Funktionen}

\underline{§ 6 Konforme Äquivalenz von Ringgebieten}

\underline{§ 7 Das Schwarz'sche Spiegelungsprinzip}

\underline{§ 8 Der Satz von Bloch}

\underline{§ 9 Der kleine Satz von Picard}

\underline{§ 10 Schlichte Funktionen in $\D$}

\underline{§ 11 Zur Potenzreihendarstellung holomorpher Funktionen}

\underline{§ 12 Der Satz von Mittag-Leffler}

\underline{§ 14 Unendliche Produkte}

\underline{§ 15 Der Weierstraß'sche Produktsatz}

\underline{§ 16 Der Ring $H(\C)$}

\underline{§ 17 Die Jensen'sche Formel}

\underline{§ 18 Periodische Funktionen}

\underline{§ 19 Elliptische Funktionen}

\underline{§ 20 Der Fixpunktsatz von Earle-Hamilton in $\C$}

\underline{§ 21 Die Subordination}

\underline{§ 22 Verbindungen zur Funktionalanalysis und die Sätze von Montel und Vitali}

\newpage

\underline{§ 1 Schreibweisen und Wiederholung}

\textbf{Satz 1.1 (Cauchy'scher Integralsatz und Folgerungen)}

\textbf{Satz 1.2:} Ist $\Omega \subseteq \C$ offen und ist $f \in H(\Omega)$ lokal injektiv, so ist $f'(z) \neq 0$ ($z \in \Omega$).

\textbf{Definition 1.1:} Sei $\Omega \subseteq \C$ offen und $(f_n)$ eine Folge in $H(\Omega)$. $(f_n)$ konvergiert auf $\Omega$ lokal gleichmäßig \\
$:\Leftrightarrow$ $(f_n)$ konvergiert auf $\Omega$ kompakt $:\Leftrightarrow$ $(f_n)$ konvergiert auf jeder kompakten Teilmenge von $\Omega$ gleichmäßig.

\textbf{Satz 1.3 (Konvergenzsatz von Weierstraß)}

\textbf{Satz 1.4 (Hurwitz):} Sei $\Omega \subseteq \C$ ein Gebiet und $(f_n)$ eine Folge in $H(\Omega)$ und $(f_n)$ konvergiere lokal gleichmäßig auf $\Omega$ gegen $f \colon \Omega \to \C$.
\vspace{-0.6cm}
\begin{itemize}
\item[(i)] Ist $0 \notin f_n(\Omega)$ $(n \in \N)$, so ist entweder $0 \notin f(\Omega)$ oder $f=0$. \vspace{-0.2cm}
\item[(ii)] Sind alle $f_n$ auf $\Omega$ injektiv, so ist entweder $f$ injektiv oder $f$ ist konstant auf $\Omega$. 
\end{itemize}
\vspace{-0.3cm}

\underline{§ 2 Der Satz von Montel}

\textbf{Gleichmäßig beschränkt:} $\exists \, c \geq 0 \, \forall \, z \in K \, \forall \, f \in \mathcal{F} \colon |f(z)| \leq c$

\textbf{Gleichgradig stetig:} $\forall \, \varepsilon > 0 \, \exists \, \delta > 0 \, \forall \, f \in \mathcal{F} \colon z,w \in K, |z-w| < \delta \Rightarrow |f(z) - f(w)| < \varepsilon$

\textbf{Satz 2.1 (Arzelà-Ascoli):} Sei $K \subseteq \C$ kompakt und $\mathcal{F}$ eine Menge von Funktionen $f\colon K \to \C$. Ist $\mathcal{F}$ gleichmäßig beschränkt und gleichgradig stetig, so besitzt jede Folge $(f_n)$ in $\mathcal{F}$ eine gleichmäßig konvergente Teilfolge.

\textbf{Lokal gleichmäßig beschränkt:} $\forall \, z_0 \in \Omega \, \exists \, r > 0 \colon \overline{K(z_0, r)} \subseteq \Omega$ und $\{ f|_{\overline{K(z_0,r)}} \colon f \in \mathcal{F}\}$ ist gleichmäßig beschränkt. 

\textbf{Satz 2.2 (Montel):} Es sei $\Omega \subseteq \C$ offen und $\mathcal{F} \subseteq H(\Omega)$ lokal gleichmäßig beschränkt. Dann hat jede Folge $(f_n)$ in $\mathcal{F}$ eine auf $\Omega$ lokal gleichmäßig konvergente Teilfolge.

\textbf{Bemerkung:} 
\vspace{-0.6cm}
\begin{itemize}
\item[(1)] Eine Teilmenge $\mathcal{F}$ von $H(\Omega)$ mit der Eigenschaft: \glqq Jede Folge in $\mathcal{F}$ enthält eine lokal gleichmäßig konvergente Teilfolge\grqq{} heißt \textbf{normale Familie}. \vspace{-0.2cm}
\item[(2)] Ist $\Omega \subseteq \C$ offen, so existiert eine Folge kompakter Mengen $(K_n)$ mit $K_1 \subseteq K_2 \subseteq K_3 \subseteq \dots$ und $\bigcup_{n=1}^\infty K_n = \Omega$.
\end{itemize}
\vspace{-0.3cm}

\underline{§ 3 Der Riemann'sche Abbildungssatz}

\textbf{Definition:} Zwei Gebiete $\Omega_1, \Omega_2 \subseteq \C$ heißen \textbf{konform äquivalent} ($\Omega_1 \sim \Omega_2$) $:\Leftrightarrow$ $\exists \, f \in H(\Omega_1) \colon f \colon \Omega_1 \to \Omega_2$ ist bijektiv. In diesem Fall heißt $f$ eine \textbf{konforme Abbildung} von $\Omega_1$ nach $\Omega_2$ und es gilt $f^{-1} \in H(\Omega_2)$.

\textbf{Bemerkung:} $\sim$ ist eine Äquivalenzrelation auf der Menge der Gebiete in $\C$.

\textbf{Satz 3.1:} Sind $\Omega_1, \Omega_2 \subseteq \C$ Gebiete mit $\Omega_1 \sim \Omega_2$, so gilt: $\Omega_1$ ist einfach zusammenhängend $\Leftrightarrow$ $\Omega_2$ ist einfach zusammenhängend. 

\textbf{Bemerkung:} $\C$ ist einfach zsh. $f(z) = e^z (z \in \C)$ ist lokal injektiv und $f(\C) = \C \backslash \{0\}$ ist nicht einfach zsh. 

\textbf{Satz 3.2 (Riemann'scher Abbildungssatz):} Sei $\Omega \subseteq \C$ ein Gebiet. Dann gilt: $\Omega \sim \D \Leftrightarrow \Omega$ ist einfach zusammenhängend und $\Omega \neq \C$.

\textbf{Bez.:} Sei $\Omega \subseteq \C$ ein Gebiet. Wir sagen: $\Omega$ hat die Eigenschaft $(W) :\Leftrightarrow \, \forall \, f \in H(\Omega)$ mit $0 \notin f(\Omega) \, \exists \, g \in H(\Omega)$ mit $g^2 = f$. \\
Bekannt: $\Omega$ ist einfach zusammenhängend $\Rightarrow$ $\Omega$ hat die Eigenschaft $(W)$.

\textbf{Lemma 3.3:} $\Omega_1, \Omega_2$ seien Gebiete in $\C$ mit $\Omega_1 \sim \Omega_2$. Dann gilt: $\Omega_1$ hat die Eigenschaft $(W) \Leftrightarrow \Omega_2$ hat die Eigenschaft $(W)$.

\textbf{Lemma 3.4:} Sei $\Omega \subseteq \C$ ein Gebiet mit der Eigenschaft $(W)$ und es sei $\Omega \neq \C$. Dann existiert ein Gebiet $\tilde{\Omega} \subseteq \C$ mit $0 \in \tilde{\Omega} \subseteq \D$ und $\Omega \sim \tilde{\Omega}$.

\textbf{Lemma 3.5:} Sei $\Omega \subseteq \C$ ein Gebiet mit $0 \in \Omega \subseteq \D$, $\Omega \neq \D$ und mit der Eigenschaft $(W)$. Dann existiert ein $\varphi \in H(\Omega)$: $\varphi(0) = 0$, $\varphi$ ist injektiv, $\varphi(\Omega) \subseteq \D$ und $|\varphi'(0)| > 1$.

\textbf{Lemma 3.6:} Sei $\Omega \subseteq \C$ ein Gebiet mit der Eigenschaft $(W)$ und $0 \in \Omega \subseteq \D$. Weiter sei $\mathcal{F} := \{ \varphi \in H(\Omega)\colon \varphi(0) = 0, \varphi \text{ injektiv und } \varphi(\Omega) \subseteq \D \}$. Dann gilt: $\mathcal{F} \neq \emptyset$ und $\exists \, \Psi \in \mathcal{F} \colon |\varphi'(0)| \leq |\Psi'(0)|$ $(\varphi \in \mathcal{F})$.

\textbf{Satz 3.7 (Folgerung):} Sei $\Omega \subseteq \C$ ein Gebiet. Dann gilt: $\Omega$ ist einfach zusammenhängend $\Leftrightarrow \Omega$ hat die Eigenschaft $(W)$.

\textbf{Satz 3.8:} Sei $\Omega \subseteq \C$ ein Gebiet. Folgende Aussagen sind äquivalent:
\vspace{-0.6cm}
\begin{itemize}
\item[(1)] $\Omega$ ist einfach zusammenhängend. \vspace{-0.2cm}
\item[(2)] $\Omega = \C$ oder $\Omega \sim \D$. \vspace{-0.2cm}
\item[(3)] $\forall \, f \in H(\Omega) \, \exists \, F \in H(\Omega) \colon F' = f$ auf $\Omega$. \vspace{-0.2cm}
\item[(4)] $\int_\gamma f(z) \, dz = 0$ für jedes $f \in H(\Omega)$ und jeden ssd. geschlossenen Weg $\gamma$ mit $\gamma^* \subseteq \Omega$. \vspace{-0.2cm}
\item[(5)] $\forall f \in H(\Omega)$ mit $0 \notin f(\Omega) \, \exists \, g \in H(\Omega)\colon e^g = f$ \vspace{-0.2cm}
\item[(6)] $\forall f \in H(\Omega)$ mit $0 \notin f(\Omega) \, \exists \, g \in H(\Omega)\colon g^2 = f$ $(W)$
\end{itemize} 
\vspace{-0.3cm}

\underline{§ 4 Automorphismen spezieller Gebiete}

In diesem § sei stets $\Omega \subseteq \C$ ein Gebiet.

\textbf{Definition:} $f \in H(\Omega)$ heißt ein \textbf{Automorphismus} von $\Omega$ $:\Leftrightarrow f\colon \Omega \to \Omega$ ist bijektiv. 
\vspace{-0.6cm}
\begin{itemize}
\item Aut$(\Omega) := \{f \colon f \text{ ist Automorphismus von } \Omega\}$ \vspace{-0.2cm}
\item Es gilt stets: $id_\Omega \in \text{Aut}(\Omega)$
\end{itemize}
\vspace{-0.3cm}

\textbf{Satz 4.1 (Lemma von Schwarz):} Sei $f \in H(\D), f(0) = 0, |f(z)| \leq |1| \, (z \in \D)$. Dann gilt: 
\vspace{-0.6cm}
\begin{itemize}
\item[(i)] $|f(z)| \leq |z| \, (z \in \D)$ und $|f'(0)| \leq 1$. \vspace{-0.2cm}
\item[(ii)] Gilt zusätzlich $|f(z_0)| = |z_0|$ für ein $z_0 \in \D \backslash \{0\}$ oder $|f'(0)| = 1$, so existiert ein $c \in \partial \D$ mit $f(z) = cz \, (z \in \D)$.
\end{itemize}
\vspace{-0.3cm}

\textbf{Definition:} Sei $a \in \D$. Die Möbiustransformation $\varphi_a(z) := \frac{z-a}{1-\overline{a}z} \, (z \in \D)$ heißt ein \textbf{Blaschkefaktor}.
\vspace{-0.6cm}
\begin{itemize}
\item $\varphi_a \colon \D \to \D$ ist bijektiv \vspace{-0.2cm}
\item $\varphi_a^{-1} = \varphi_{-a}$ \vspace{-0.2cm}
\item $\{ c \cdot \varphi_a \colon a \in \D, |c| =1 \} \subseteq \text{Aut}(\D)$
\end{itemize}
\vspace{-0.3cm}

\textbf{Satz 4.2:} Es gilt $\{ c \cdot \varphi_a \colon a \in \D, |c| =1 \} = \text{Aut}(\D)$.

\textbf{Bez.:} $\D^x := \D \backslash \{0\}$, $\C^x := \C \backslash \{0\}$.

\textbf{Lemma 4.3:} Sei $f \in H(\C)$ und $g \in H(\C^x)$ definiert durch $g(z) = f(\frac{1}{z})$. Dann gilt: $0$ ist wesentliche Singularität von $g \Leftrightarrow f$ ist kein Polynom.

\textbf{Vorbemerkung:} Sei $\Omega \subseteq \C$ offen und $f \in H(\Omega \backslash \{ z_0 \})$ mit $z_0 \in \Omega$. $f$ habe in $z_0$ einen Pol $m$-ter Ordnung. Dann existieren $a_1, \dots , a_m \in \C$ mit $a_m \neq 0$ so, dass $h(z) := f(z) - \sum_{k=1}^m \frac{a_k}{(z - z_0)^k}$ in $z_0$ eine hebbare Singularität hat, also $h \in H(\Omega)$. Es gilt: \\
$f(z) = \frac{1}{(z - z_0)^m} \underbrace{[(z-z_0)^m \, h(z) + a_1 (z-z_0)^{m-1} + \dots + a_{m-1} (z - z_0) + a_m]}_{=: \, g(z)}$. \\
Wegen $g(z_0) = a_m \neq 0$ existiert ein $\delta > 0$ mit $g(z) \neq 0$ $(z \in K(z_0, \delta))$. Damit hat $\frac{1}{f} \colon K(z_0, \delta) \to \C$ in $z_0$ eine hebbare Singularität, also $\frac{1}{f} \in H(K(z_0, \delta))$ und $z_0$ ist eine $m$-fache Nullstelle von $\frac{1}{f}$.

\textbf{Lemma 4.4:} Sei $\Omega \subseteq \C$ ein Gebiet, $z_0 \in \Omega$ und $f\in H(\Omega \backslash \{z_0\})$ injektiv. Dann gilt:
\vspace{-0.6cm}
\begin{itemize}
\item[(i)] $z_0$ ist keine wesentliche Singularität von $f$. \vspace{-0.2cm}
\item[(ii)] Ist $z_0$ eine hebbare Singularität von $f$, so ist $f$ auf $\Omega$ injektiv. \vspace{-0.2cm}
\item[(iii)] Ist $z_0$ ein Pol von $f$, so ist $z_0$ ein einfacher Pol.
\end{itemize}
\vspace{-0.3cm}

\textbf{Satz 4.5:} Sei $f \in H(\C)$. Dann sind folgende Aussagen äquivalent:
\vspace{-0.6cm}
\begin{itemize}
\item[(i)] $\exists \, a \in \C^x \, \exists \, b \in \C \colon f(z) = az + b$ auf $\C$. \vspace{-0.2cm}
\item[(ii)] f ist injektiv. \vspace{-0.2cm}
\item[(iii)] $f \in \text{Aut}(\C)$.
\end{itemize}
\vspace{-0.3cm}

\textbf{Bemerkung:} Es gibt lokal injektive Funktionen in $H(\C)$, die nicht injektiv sind. Bsp $f(z) = e^z$.

\textbf{Satz 4.6:} Sei $f \in H(\C^x)$. Dann sind äquivalent:
\vspace{-0.6cm}
\begin{itemize}
\item[(i)] $\exists \, a \in \C^x \colon f(z) = az \, (z \in \C)$ oder $f(z) = \frac{a}{z} \, (z \in \C^x)$. \vspace{-0.2cm}
\item[(ii)] $f$ ist auf $\C^x$ injektiv und $0 \notin f(\C^x)$. \vspace{-0.2cm}
\item[(iii)] $f \in \text{Aut}(\C^x)$.
\end{itemize}
\vspace{-0.3cm}

Nun sei $\Omega \subseteq \C$ ein Gebiet und $\emptyset \neq A \subseteq \Omega$ diskret in $\Omega$ (d.h. $A$ hat keine Häufungspunkte in $\Omega$). Dann ist auch $\Omega \backslash A$ ein Gebiet. \\
Es sei Aut$_A(\Omega) := \{ f \in \text{Aut}(\Omega) \colon f(A) = A \}$. Wir wollen Aut$(\Omega \backslash A)$ untersuchen. 

\textbf{Beispiel:} siehe Seite 20

\textbf{Definition:} Sei $\Omega \subseteq \C$ ein Gebiet. Die Menge der isolierten Randpunkte von $\Omega$ sei definiert durch \\ 
iso $\partial \Omega := \{ a \in \partial \Omega\colon \, \exists \, \varepsilon > 0 \colon \partial \Omega \cap (K(a, \varepsilon)\backslash \{a\}) = \emptyset \}$
\vspace{-0.6cm}
\begin{itemize}
\item iso $\partial \D = \emptyset$ \vspace{-0.2cm}
\item $\partial \D^x = \partial \D \cup \{0\}$, iso $\partial \D^x = \{ 0 \}$ \vspace{-0.2cm}
\item $\partial \C^x = \{ 0 \} = $ iso $\partial \C^x$
\end{itemize}
\vspace{-0.3cm}

\textbf{Lemma 4.7:} Es sei $g \in H(\Omega)$, $g$ auf $\Omega \backslash A$ injektiv, $g(\Omega \backslash A) \subseteq \Omega$, $a \in A$ und $g(a) \in \partial \Omega$. Dann gilt: $g(a) \in$ iso $\partial \Omega$.

\textbf{Satz 4.8:} Es sei $\Omega$ beschränkt und iso $\partial \Omega = \emptyset$. Dann gilt: $\{ g|_{\Omega \backslash A} \colon g \in \text{Aut}_A(\Omega)\} = \text{Aut}(\Omega \backslash A)$.

\textbf{Satz 4.9 (Folgerung):} $f \in \text{Aut}(\D^x) \Leftrightarrow \, \exists \, c \in \partial \D \colon f(z) = cz \, (z \in \D^x)$.

\textbf{Bemerkung:}
\vspace{-0.6cm}
\begin{itemize}
\item[(1)] Satz 4.8 gilt im Allgemeinen nicht für unbeschränkte Gebiete. (siehe Beispiel Seite 23) \vspace{-0.2cm}
\item[(2)] Satz 4.8 gilt im Allgemeinen nicht, falls iso $\partial \Omega \neq \emptyset$. (siehe Beispiel Seite 23)
\end{itemize}
\vspace{-0.3cm}

\textbf{Definition:} $\Omega$ heißt starr $\Leftrightarrow$ Aut$(\Omega) = \{ id_\Omega\}$.

\textbf{Übung:} Seien $a,b \in \D^x, a \neq b$ und $\Omega = \D \backslash \{ 0,a,b \}$. Dann gilt: $\Omega$ ist nicht starr $\Leftrightarrow a= -b$ oder $2b = a + \overline{a}b^2$ oder $2a = b+\overline{b}a^2$ oder $[\,|a|=|b| \text{ und } a^2 + b^2 = ab\,(1+|a|^2)\,]$ \\
Bsp. $\D \backslash \{ 0, \frac{1}{2}, -\frac{1}{2}\}$ ist nicht starr, aber $\D \backslash \{ 0, \frac{1}{2}, \frac{3}{4}\}$ ist starr.


\textbf{Bez.:} $\mathbb{H} := \{ z \in \C \colon \text{Im}\, z > 0\}$ 

\textbf{Satz 4.10:} Sei $\Phi(z) = \frac{z-i}{z+i}$. Es gilt:
\vspace{-0.6cm}
\begin{itemize}
\item[(i)] $\Phi$ ist auf $\mathbb{H}$ injektiv und $\Phi(\mathbb{H}) = \D$ \vspace{-0.2cm}
\item[(ii)] Aut$(\mathbb{H}) = \{ \Phi^{-1} \circ f \circ \Phi \colon f \in \text{Aut}(\D) \}$
\end{itemize}
\vspace{-0.3cm}

\textbf{Satz 4.11:} Sei $\Omega$ beschränkt, $f \in$ Aut$(\Omega)$, $z_0 \in \Omega$ und $f(z_0) = z_0$. Dann gilt: $|f'(z_0)| = 1$.
\vspace{-0.6cm}
\begin{itemize}
\item Satz 4.11 gilt nicht, falls $\Omega$ unbeschränkt ist. \vspace{-0.2cm}
\item Bsp: $\Omega = \C, f(z) = 2z-1, z_0 = 1, f'(z_0) = 2$.
\end{itemize}
\vspace{-0.3cm}

\underline{§ 5 Harmonische Funktionen}

\textbf{Definition:} Sei $\Omega \subseteq \C$ offen und $u \in C^2(\Omega, \C)$ mit $\Delta u := u_{xx} + u_{yy} = 0$ auf $\Omega$, so heißt $u$ harmonisch in $\Omega$. Har$(\Omega) := \{ u \colon \Omega \to \C \colon u \text{ ist harmonisch in } \Omega \}$.

\textbf{Bemerkung:} 
\vspace{-0.6cm}
\begin{itemize}
\item[(1)] Ist $f \in H(\Omega)$, so sind $u = $ Re$f$ und $v = $ Im$f$ in $\Omega$ harmonisch (vgl. Cauchy-Riemann'sche Differentialgleichungen). Insbesondere sind $p_n(x,y) =$ Re $z^n$ und $q_n(x,y) = $ Im $z^n$ $(n \in \N_0, z = x+iy \in \C)$ harmonische Polynome (vom Grad $n$). \\
Ist nun $z_0 \in \Omega, f \in H(\Omega), K(z_0,r) \subseteq \Omega$ und $f(z) = \sum_{n=0}^\infty a_n (z-z_0)^n$ $(z \in K(z_0, r))$, so gilt: $f(z) = \sum_{n=0}^\infty \underbrace{(\alpha_n + i\beta_n)}_{= a_n} (\text{Re}(z-z_0)^n + i\text{Im}(z-z_0)^n) = \text{Re}f(z) + i \text{Im}f(z)$, und diese Reihen konvergieren absolut und gleichmäßig auf jeder kompakten Teilmenge von $K(z_0, r)$.  \vspace{-0.2cm}
\item[(2)] Die Funktion $\log |z| = \frac{1}{2} \log(x^2 + y^2)$ ist zunächst in $\C \backslash (- \infty, 0]$ harmonisch (wegen $\log |z| = \text{Re}(\log z)$). Sie ist sogar in $\C \backslash \{ 0 \}$ harmonisch. \\
Allerdings existiert kein $f \in H(\C \backslash \{ 0 \})$ mit $\log |z| = \text{Re}f(z)$ $(z \in \C \backslash \{ 0 \})$. Sonst wäre $f(z) = \log |z| + iv(z)$ also $f(z)-\log z \in i\R$ $(z \in \C \backslash (-\infty,0])$. Dann ist $f(z) - \log z$ konstant auf $\C \backslash (-\infty,0]$, also $f'(z) = \frac{1}{z}$ auf $\C \backslash (-\infty,0]$, somit ($f'$ stetig) auch $f'(z) = \frac{1}{z}$ auf $\C \backslash \{ 0 \}$. Wid.
\end{itemize}
\vspace{-0.3cm}

\textbf{Satz 5.1:} Es sei $f \in H(\Omega)$ und $0 \notin f(\Omega)$. Dann gilt $\log|f| \in \text{Har}(\Omega)$.

\textbf{Satz 5.2:} Es sei $\Omega \subseteq \C$ ein Gebiet. Dann gilt: $\Omega$ ist einfach zusammenhängend $\Leftrightarrow \forall \, u \in \text{Har}(\Omega) \, \exists \, f \in H(\Omega) \colon u = \text{Re }f$.

\textbf{Bemerkung:} Ist $\Omega$ ein einfach zsh. Gebiet, so existiert zu jedem $u \in$ Har$(\Omega)$ eine sog. zu $u$ \textbf{konjugiert harmonische Funktion} $v \in$ Har$(\Omega)$, sodass $f(z) = u(z) + iv(z)$ auf $\Omega$ holomorph ist.

\textbf{Definition:} Sei $\Omega \subseteq \C$ offen und $u \colon \Omega \to \R$ stetig. $u$ hat auf $\Omega$ die Mittelwerteigenschaft (MWE) $:\Leftrightarrow$ Für jede Kreisscheibe $\overline{K(z_0, r)} \subseteq \Omega$ gilt: $u(z_0) = \frac{1}{2\pi} \int_0^{2\pi} u(z_0 + re^{it}) \, dt$.

\textbf{Satz 5.3:} Sei $\Omega \subseteq \C$ offen und $u \in$ Har$(\Omega)$. Dann hat $u$ auf $\Omega$ die MWE.

\textbf{Lemma 5.4:} Sei $\Omega \subseteq \C$ ein Gebiet und $\emptyset \neq D \subseteq \Omega$. Gilt dann:
\vspace{-0.6cm}
\begin{itemize}
\item[(i)] $D$ ist offen. \vspace{-0.2cm}
\item[(ii)] Jeder Häufungspunkt $w$ von $D$ mit $w \in \Omega$ gehört zu $D$.
\end{itemize}
\vspace{-0.6cm}
So ist $\Omega = D$.

\textbf{Die Poisson'sche Integralformel:} siehe Seite 30

\textbf{Satz 5.5:} Sei $h \colon \partial \D \to \R$ stetig und $f \colon \overline{\D} \to \C$ definiert durch \\
$ f(z) =
  \begin{cases} 
\frac{1}{2\pi} \int_0^{2\pi} \frac{e^{it} + z}{e^{it} - z} \, h(e^{it}) \, dt &, |z| < 1 \\ 
h(z) &, |z| = 1.
\end{cases}
$ \\
Dann ist $f \in H(\D)$ und $u := \text{Re} f$ auf $\overline{\D}$ stetig. 

\textbf{Bemerkung:} 
\vspace{-0.6cm}
\begin{itemize}
\item[(i)] Insbesondere hat das sogenannte Dirichlet'sche Randwertproblem 
\vspace{-0.2cm}
\begin{align*}
\Delta u(x,y) &= u_{xx}(x,y) + u_{yy}(x,y) = 0 \quad (x^2 + y^2 < 1) \\
u(x,y) &= h(x,y) \quad (x^2+y^2=1)
\end{align*}
eine Lösung $u \in C(\overline{\D}) \cap C^2(\D)$
\vspace{-0.2cm}
\item[(ii)] Satz 5.5 kann durch Variablentransformation auf beliebige Kreisscheiben übertragen werden: \\
Ist $h \in \partial K(z_0, r) \to \R$ stetig, so existiert ein $f \in H(K(z_0, r))$ so, dass \\
$ u(z) :=
  \begin{cases} 
\text{Re} f(z) &, z \in K(z_0,r) \\ 
h(z) &, z \in \partial K(z_0,r).
\end{cases}
$ \\
auf $\overline{K(z_0,r)}$ stetig ist.
\end{itemize}
\vspace{-0.3cm}

\textbf{Satz 5.6 (Umkehrung von Satz 5.3):} Es sei $\Omega \subseteq \C$ offen und $u \colon \Omega \to \R$ sei stetig und habe die MWE. Dann ist $u \in$ Har$(\Omega)$.

\textbf{Satz 5.7:} Es sei $\Omega \subseteq \C$ ein Gebiet. 
\vspace{-0.6cm}
\begin{itemize}
\item[(i)] Ist $u \in$ Har$(\Omega)$ und hat $u$ in $\Omega$ ein globales Extremum, so ist $u$ konstant. (Für lokale Extrema siehe Satz 5.9) \vspace{-0.2cm}
\item[(ii)] Ist $\Omega$ beschränkt, $u \in C(\overline{\Omega}, \R) \cap \text{Har}(\Omega)$ und nicht konstant, so ist \\
$\min_{\xi \in \overline{\Omega} (\partial \Omega)} u(\xi) < u(z) < \max_{\xi \in \overline{\Omega} (\partial \Omega)} u(\xi)$ ($z \in \Omega$).
\end{itemize}
\vspace{-0.3cm}

\textbf{Satz 5.8 (Identitätssatz):} Sei $\Omega \subseteq \C$ ein Gebiet, $u \in$ Har$(\Omega)$, $z_0 \in \Omega$, $r > 0$, $K(z_0, r) \subseteq \Omega$ und $u = 0$ auf $K(z_0, r)$. Dann ist $u = 0$ auf $\Omega$.

\textbf{Satz 5.9:} Sei $\Omega \subseteq \C$ ein Gebiet und $u \in$ Har$(\Omega)$. Hat $u$ in $\Omega$ ein lokales Extremum, so ist $u$ konstant.

\newpage
\underline{§ 6 Konforme Äquivalenz von Ringgebieten}

Für $j = 1,2$ und $0 < r_j < R_j$ sei $A_j := A(r_j, R_j) := \{ z \in \C \colon r_j < |z| < R_j \}$. \\
Ziel: $A_1 \sim A_2 \Leftrightarrow \frac{R_1}{r_1} = \frac{R_2}{r_2}$

\textbf{Satz 6.1:} Ist $\frac{R_1}{r_1} = \frac{R_2}{r_2}$, so ist $A_1 \sim A_2$.

\textbf{Satz 6.2:} Sei $\Omega \subseteq \C$ ein Gebiet, $f \in H(\Omega)$, $0 \notin f(\Omega)$, $\alpha \geq 1$ und $|f(z)| = |z|^\alpha$ $(z \in \Omega)$. Dann gilt: 
\vspace{-0.6cm}
\begin{itemize}
\item[(1)] $\frac{f'(z)}{f(z)} = \frac{\alpha}{z} \, (z \in \Omega)$. (Beachte: $0 \notin \Omega$) \vspace{-0.2cm}
\item[(2)] Ist $R_1 > 1$ und $\Omega = A(1, R_1)$, so ist $\alpha \in \N$ und $f(z) = cz^\alpha$ für ein $c \in \partial \D$. \vspace{-0.2cm}
\item[(3)] Ist $R_1 > 1$, $\Omega = A(1, R_1)$ und $f$ injektiv, so ist $\alpha = 1$ (und $f(z) = cz$ für ein $c \in \partial \D$).
\end{itemize}
\vspace{-0.3cm}

\textbf{Satz 6.3:} Sei $1 < R_1 \leq R_2$, $A_j := A(1, R_j) \, (j = 1,2)$, $f \in H(A_1)$ injektiv und $f(A_1) = A_2$. Dann gilt:
\vspace{-0.6cm}
\begin{itemize}
\item[(1)] $R_1 = R_2$, also $A_1 = A_2$. \vspace{-0.2cm}
\item[(2)] $\exists \, c \in \C$ mit $|c| = 1$ und $f(z) = cz$ oder \\
$\exists \, c \in \C$ mit $|c| = R_1$ und $f(z) = \frac{c}{z}$.
\end{itemize}
\vspace{-0.3cm}

\textbf{Satz 6.4:} Für $j = 1,2$ sei $0 < r_j < R_j$ und $A_j = A(r_j, R_j)$. Es sei $f \in H(A_1)$ eine konforme Abbildung von $A_1$ auf $A_2$. Dann gilt:
\vspace{-0.6cm}
\begin{itemize}
\item[(1)] $\frac{R_1}{r_1} = \frac{R_2}{r_2}$ \vspace{-0.2cm}
\item[(2)] $\exists \, \lambda \in \C \colon |\lambda| = \frac{r_2}{r_1}$ und $f(z) = \lambda z$ oder \\
$\exists \, \lambda \in \C \colon |\lambda| = r_2 R_1$ und $f(z) = \frac{\lambda}{z}$.
\end{itemize}
\vspace{-0.3cm}

\textbf{Satz 6.5 (Folgerung):} Sei $0 < r < R$ und $A = A(r,R)$. Dann: $f \in$ Aut$(A) \Leftrightarrow \, \exists \, c \in \C \colon |c| = 1$ und $f(z) = cz$ oder $\exists \, c \in \C \colon |c| = rR$ und $f(z) = \frac{c}{z}$. 

\underline{§ 7 Das Schwarz'sche Spiegelungsprinzip}

Sei $\Omega \subseteq \C$ ein Gebiet und $\Omega^* := \{ \overline{z} \colon z \in \Omega \}$. Dann ist $\Omega^*$ ein Gebiet. Ist $f \colon \Omega \to \C$ eine Funktion, so bezeichne $f^*\colon \Omega^* \to \C$ die Funktion $f^*(z) = \overline{f(\overline{z})}$. Im Folgenden sei $\Omega$ stets ein Gebiet.

\textbf{Lemma 7.1:}
\vspace{-0.6cm}  
\begin{itemize}
\item[(1)] Sei $f \in H(\Omega)$. Dann ist $f^* \in H(\Omega)$ und $(f^*)'(z) = \overline{f'(\overline{z})}$ $(z \in \Omega^*)$ \vspace{-0.2cm}
\item[(2)] Sei $\Omega = \Omega^*$. Dann ist $\Omega \cap \R \neq \emptyset$ und $\Omega \cap \R$ ist die Vereinigung offener Intervalle in $\R$.
\end{itemize}
\vspace{-0.3cm}

\textbf{Bez.:} Im Fall $\Omega = \Omega^*$ sei 
\vspace{-0.6cm}
\begin{itemize}
\item $\Omega_+ := \{ z \in \Omega \colon \text{Im} z > 0 \}$, \vspace{-0.2cm}
\item $\Omega_- := \{ z \in \Omega \colon \text{Im} z < 0 \}$, \vspace{-0.2cm}
\item $\Omega_0 := \{ z \in \Omega \colon z \in \R \}$.
\end{itemize}
\vspace{-0.3cm}

Weiter sei $f \in C(\Omega_+ \cup \Omega_0), f \in H(\Omega_+)$ und $g \colon \Omega \to \C$ definiert durch \\
$ g(z) =
  \begin{cases} 
f(z) &, z \in \Omega_+ \cup \Omega_0 \\ 
\overline{f(\overline{z})} &, z \in \Omega_-.
\end{cases}
$ \\
Dann gilt $g|_{\Omega_+} \in H(\Omega_+)$ und wegen Lemma 7.1 $g|_{\Omega_-} \in H(\Omega_-)$

\newpage
\textbf{Satz 7.2 (Schwarz'sches Spiegelungsprinzip):} Unter obigen Setzungen sind äquivalent:
\vspace{-0.6cm}  
\begin{itemize}
\item[(1)] $g \in H(\Omega)$ \vspace{-0.2cm}
\item[(2)] $f(x) \in \R$ $(x \in \Omega_0)$.
\end{itemize}
\vspace{-0.3cm}  

\underline{§ 8 Der Satz von Bloch}

\textbf{Lemma 8.1:} Sei $f \in H(\D), M \geq 0, |f| \leq M$ auf $\D$, $f(0)=0$ und $f'(0) = 1$. Dann ist $M \geq 1$ und $K(0, \frac{1}{6M})\subseteq f(\D)$.

\textbf{Satz 8.2 (Folgerung):} Sei $R > 0, f \in H(K(0,R)), M \geq 0, |f| \leq M$ auf $K(0,R), f(0) = 0$ und $\mu := |f'(0)| > 0$. Mit $r := \frac{R^2 \mu^2}{6M}$ gilt $K(0,r) \subseteq f(K(0,R))$.

\textbf{Lemma 8.3:} Es sei $\Omega \subseteq \C$ ein konvexes Gebiet, $f \in H(\Omega), a \in \Omega$ und $|f'(z) - f'(a)| < |f'(a)|$ $(z \in \Omega)$. Dann ist $f$ auf $\Omega$ injektiv.

\textbf{Satz 8.4 (Satz von Bloch):} Sei $\Omega \subseteq \Omega$ offen, $\overline{\D} \subseteq \Omega, f \in H(\Omega), f(0) = 0$ und $f'(0) = 1$. Dann existiert ein $a \in \D$ und ein $\varphi > 0$ mit: $K(a, \varphi) \subseteq \D$, $f$ ist auf $K(a, \varphi)$ injektiv und $K(f(a), \frac{1}{72}) \subseteq f(K(a, \varphi)) \subseteq f(\D)$.

\textbf{Satz 8.5 (Folgerungen):} 
\vspace{-0.6cm}  
\begin{itemize}
\item[(1)] Sei $\Omega \subseteq \C$ offen, $R > 0, \overline{K(0,R)} \subseteq \Omega, f \in H(\Omega)$ und $f'(0) \neq 0$. Dann enthält $f(K(0,R))$ eine offene Kreisscheibe mit Radius $\frac{1}{72} \, R|f'(0)|$. \vspace{-0.2cm}
\item[(2)] Ist $g \in H(\C)$ nicht konstant, so enthält $g(\C)$ offene Kreisscheiben von jedem Radius. \vspace{-0.2cm}
\item[(3)] Aus Satz 8.5 (2) folgt insbesondere der Satz von Liouville.
\vspace{-0.3cm}  
\end{itemize}

\underline{§ 9 Der kleine Satz von Picard}

\textbf{Vorbemerkungen:}
\vspace{-0.6cm}  
\begin{itemize}
\item[(1)] Für $x \geq 1$ sei $\varphi(x) = \log(\sqrt{x+1} + \sqrt{x}) - \log(\sqrt{x} + \sqrt{x-1})$ \\
Übung: $\varphi$ ist auf $[1,\infty)$ fallend. Insbesondere gilt: $\varphi(x) \leq \varphi(1) = \log(\sqrt{2} + 1) < \log e = 1$ $(x \geq 1)$. \vspace{-0.2cm}
\item[(2)] Sei $R$ ein achsenparalleles Rechteck in $\C$ (abg.) mit Breite $< 1$ und Höhe $< \sqrt{3}$. \\
 Sei $z_0 \in R$. Dann existiert ein Eckpunkt $a$ von $R$ mit $|\text{Re} (a-z_0)|<\frac{1}{2}$ und $|\text{Im}(z-z_0)| < \frac{\sqrt{3}}{2}$, also $|a-z_0|^2 < \frac{1}{4} + \frac{3}{4} = 1$; d.h. $a \in K(z_0, 1)$.
\end{itemize}
\vspace{-0.3cm} 

\textbf{Lemma 9.1:} Sei $\Omega \subseteq \C$ ein einfach zsh. Gebiet, $f \in H(\Omega)$ und $0,1 \notin f(\Omega)$. Dann existiert ein $g \in H(\Omega)$ mit $f(z) = -e^{i \pi \cosh (2 g(z))} \, (z \in \Omega)$.

\textbf{Lemma 9.2:} $\Omega$ und $f$ seien wie in Lemma 9.1 und $g \in H(\Omega)$ erfülle $f = -e^{i \pi \cosh(2g)}$ auf $\Omega$. Dann gilt:
\vspace{-0.6cm}  
\begin{itemize}
\item[(1)] $g(\Omega)$ enthält keine offene Kreisscheibe vom Radius 1. \vspace{-0.2cm}
\item[(2)] Ist $\Omega = \C$, so sind $f$ und $g$ konstant. 
\end{itemize}
\vspace{-0.3cm} 

\textbf{Satz 9.3 (Kleiner Satz von Picard):} Sei $f \in H(\C), a,b \in \C, a \neq b$ und $a,b \notin f(\C)$. Dann ist $f$ konstant.

Ist also $f \in H(\C)$, so tritt genau eine der folgenden Möglichkeiten ein:
\vspace{-0.6cm}  
\begin{itemize}
\item[(1)] $f(\C) = \C$ \vspace{-0.2cm}
\item[(2)] $\exists c \in \C \colon f(\C) = \C \backslash \{ c \}$ \vspace{-0.2cm}
\item[(3)] $f$ ist konstant.
\end{itemize}
\vspace{-0.6cm} 
Jede Möglichkeit kann eintreten. 

\textbf{Erweiterung von Satz 9.3 auf meromorphe Funktionen:}
\vspace{-0.6cm}  
\begin{itemize}
\item[(1)] Sei $\Omega \subseteq \C$ offen, $A \subseteq \Omega$ diskret in $\Omega$. Ist $f \in H(\Omega \backslash A)$ und ist jedes $a \in A$ ein Pol von $f$, so heißt $f$ meromorph, in Zeichen $f \in M(\Omega)$ [$A = \emptyset$ ist zugelassen, also $H(\Omega) \subseteq M(\Omega)$] \vspace{-0.2cm}
\item[(2)] Vorbemerkung nach Lemma 4.3: Ist $\Omega \subseteq \C$ offen, $z_0 \in \Omega$, $f \in H(\Omega \backslash \{ z_0 \})$ und hat $f$ in $z_0$ einen Pol $m$-ter Ordnung, so existiert ein $\delta > 0$: $K(z_0,\delta) \subseteq \Omega$, $\frac{1}{f} \in H(K(z_0, \delta))$ und $\frac{1}{f}$ hat in $z_0$ eine Nullstelle $m$-ter Ordnung.
\end{itemize}
\vspace{-0.3cm} 

\textbf{Beispiel:} siehe Seite 59

\textbf{Satz 9.4 (Kleiner Satz von Picard für meromorphe Funktionen):} Sei $f \in M(\C)$, seien $a,b,c \in \C$ paarweise verschieden und $f$ nehme die Werte $a,b,c$ nicht an. Dann ist $f$ konstant. 

Sei $\Omega \subseteq \C$ offen, $z_0 \in \Omega$ und $f \in H(\Omega \backslash \{ z_0 \})$ habe in $z_0$ eine wesentliche Singularität. Es gilt (Satz von Casorati-Weierstraß) $\overline{f(K(z_0, \delta)\backslash \{ z_0 \} )} = \C$ für jedes $\delta > 0$ mit $K(z_0, \delta) \subseteq \Omega$.

\textbf{Satz 9.5 (Großer Satz von Picard):} $\exists \, c \in \C \colon \C \backslash \{ c \} \subseteq f(K(z_0, \delta) \backslash \{ z_0 \})$ für jedes $\delta > 0$ mit $K(z_0, \delta) \subseteq \Omega$. Weiter gilt $|f^{-1}(\{ w \}) \cap (K(z_0, \delta) \backslash \{ z_0 \})| = \infty$ $(w \in \C \backslash \{ c \})$.

\textbf{Anwendung auf ganze Funktionen:} Sei $f \in H(\C)$ nicht konstant. \\
1. Fall: $f$ ist ein Polynom $\xLongrightarrow[]{\text{FS der Algebra}}$ $f(\C) = \C$. \\
2. Fall: $f$ ist kein Polynom. Nach Lemma 4.3 ist dann $0$ wesentliche Singularität von $g(z) = f(\frac{1}{z})$ $(z \in \C^x)$. \\
Nach Satz 9.5: $\exists \, c \in \C \colon \C \backslash \{ c \} \subseteq g(\D^x) \subseteq g(\C^x) = f(\C^x) \subseteq f(\C)$. \\
Fazit: Satz 9.3 folgt aus Satz 9.5.

\textbf{Satz 9.6 (Anwendungen des kleinen Satzes von Picard):} Sei $f \in H(\C), p \in \C^x$
\vspace{-0.6cm}  
\begin{itemize}
\item[(1)] Ist $f(z + p) = f(z)$ $(z \in \C)$, so hat $f$ einen Fixpunkt. \vspace{-0.2cm}
\item[(2)] Ist $f$ nicht von der Form $f(z) = z + b$ für ein $b \in \C^x$, so hat $f \circ f$ einen Fixpunkt.
\end{itemize}
\vspace{-0.6cm} 

\vspace{-0.4cm}  
\begin{itemize}
\item Bsp. zu (1): $f(z) = e^z$ hat einen Fixpunkt $(p = 2 \pi i)$  \vspace{-0.2cm}
\item Bsp. zu (2): $f(z) = z + e^z$ hat keinen Fixpunkt, $f \circ f$ hat einen Fixpunkt
\end{itemize}
\vspace{-0.3cm} 

\textbf{Satz 9.7:} Es seien $f,g \in H(\C)$ und es gelte $1 = e^f + e^g$ auf $\C$. Dann sind $f$ und $g$ konstant. 

\textbf{Satz 9.8:} 
\vspace{-0.6cm}  
\begin{itemize}
\item[(1)] $f,g \in H(\C), h := e^f + e^g$. Dann gilt: $0 \notin h(\C)$ oder card $h^{-1} (\{ 0 \}) = \infty$  \vspace{-0.2cm}
\item[(2)] $f \in H(\C)$, $p$ Polynom mit grad $p \geq 1, g := pe^f$. Dann gilt $g(\C) = \C$.
\end{itemize}
\vspace{-0.3cm} 

\textbf{Satz 9.9 (Satz von Iyer, 1939):} Sei $\Omega$ ein einfach zsh. Gebiet in $\C$, $f,g \in H(\Omega)$ und $f^2 + g^2 = 1$ auf $\Omega$. Dann existiert ein $h \in H(\Omega)$: $f = \cos \circ \, h$ und $g = \sin \circ \, h$ auf $\Omega$.

\textbf{Satz 9.10 (Satz von Iyer, 1939):} Seien $f,g \in H(\C), n \in \N, n \geq 3$ und es gelte $f^n + g^n = 1$ auf $\C$. Dann sind $f$ und $g$ konstant. 

\textbf{Bemerkung:} Vgl. Satz von Fermat-Wiles: Ist $n \geq 3, x,y \in \Q$ und $x^n + y^n = 1$, so ist $x\cdot y = 0$.

\textbf{Satz 9.11 (Satz von Gross, 1966):} Seien $f,g \in H(\C)$, $n \in \N, n \geq 3$, $p$ ein Polynom vom Grad $\leq n-3, p \neq 0$ und es gelte $f^n + g^n = p$ auf $\C$. Dann sind $f,g$ und $p$ konstant. 

\underline{§ 10 Schlichte Funktionen in $\D$}

In der Funktionentheorie: schlicht $=$ injektiv \\
$S := \{ f \in H(\D) \colon f(0) = 0, f'(0) = 1, f \text{ schlicht} \}$. \\
Sei $f \in S$. Dann gilt $f'(z) \neq 0$ $(z \in \D)$ (vgl. Satz 1.2). \\
Potenzreihendarstellung von $f$: $f(z) = z + \sum_{k=2}^\infty a_k z^k$ $(z \in \D)$.

\textbf{Beispiel 10.1 (Die Koebefunktion):} $K(z) = \sum_{n=1}^\infty nz^n = z + 2z^2 + 3z^3 + \dots$ $(z \in \D)$. Es gilt:
\vspace{-0.6cm} 
\begin{itemize}
\item[(1)] $K(z) = \frac{z}{(1-z)^2} = \frac{1}{4} \left( \left( \frac{1+z}{1-z}\right)^2 - 1 \right)$ $(z \in \D)$ \vspace{-0.2cm}
\item[(2)] $K \in S$ \vspace{-0.2cm}
\item[(3)] $K(\D) = \C \backslash \{ t \in \R \colon t \leq - \frac{1}{4} \}$ 
\end{itemize} 
\vspace{-0.3cm} 

\textbf{Satz 10.2:} Sei $f \in S$.
\vspace{-0.6cm} 
\begin{itemize}
\item $f_1(z) = \overline{f(\overline{z})}$ \vspace{-0.2cm}
\item $f_2(z) = e^{-i \varphi} f(e^{i \varphi} z)$ $(\varphi \in \R)$ \vspace{-0.2cm}
\item $f_3(z) = \frac{1}{r} f(rz)$ $(0 < r < 1)$ \vspace{-0.2cm}
\item $f_4(z) = \frac{wf(z)}{w-f(z)}$ $(w \notin f(\D))$
\end{itemize}
\vspace{-0.6cm} 
Dann gilt: $f_i \in S$ $(i = 1, \dots , 4)$.

\textbf{Satz 10.3:}
\vspace{-0.6cm} 
\begin{itemize}
\item[(1)] Sei $\Omega$ ein einfach zsh. Gebiet in $\C$, $0 \in \Omega$, $z^2 \in \Omega$ $(z \in \Omega)$, $f \in H(\Omega)$, $f(0) = 0$, $0 \notin f(\Omega \backslash \{ 0 \})$ und $f'(0) \notin 0$. Dann existiert ein $g \in H(\Omega)$ mit $(g(z))^2 = f(z^2)$ $(z \in \Omega)$. \vspace{-0.2cm}
\item[(2)] Ist $f \in S$, so existiert ein $g \in S$ mit $(g(z))^2 = f(z^2)$ $(z \in \D)$.
\end{itemize} 
\vspace{-0.3cm} 

\textbf{Satz 10.4 (Satz von Bieberbach):} Ist $f \in S$ und $f(z) = z + \sum_{n=2}^\infty a_n z^n$. Dann gilt:
\vspace{-0.6cm} 
\begin{itemize}
\item[(1)] $|a_2|  \leq 2$.\vspace{-0.2cm}
\item[(2)] Ist $|a_2| = 2$, so ist $f$ eine Rotation der Koebefunktion $K$ (also $f(z) = e^{-i \varphi} K(e^{i \varphi}z)$ für ein $\varphi \in \R$). Insbesondere ist dann $|a_n| = n$ $(n \in \N)$.
\end{itemize} 
\vspace{-0.3cm}

\textbf{Satz (Bieberbach'sche Vermutung, 1916):} Ist $f \in S$, $f(z) = z + \sum_{n=2}^\infty a_n z^n$, so gilt: $|a_n| \leq n$ $(n \in \N)$.

\textbf{Geschichte dieser Vermutung:} siehe Seite 72

\textbf{Satz 10.5:} Sei $m \geq 2$, $a_m \neq 0$ und $f(z) = z + a_2 z^2 + \dots + a_m z^m$.
\vspace{-0.6cm} 
\begin{itemize}
\item[(1)] Ist $f \in S$, so ist $|a_m| \leq \frac{1}{m}$ \vspace{-0.2cm}
\item[(2)] Ist $f(z) = z + a_mz^m$, so gilt: $f \in S \Leftrightarrow |a_m| \leq \frac{1}{m}$.
\end{itemize} 
\vspace{-0.3cm}

\newpage
\textbf{Satz 10.6 (Koebe'scher $\frac{1}{4}$-Satz):} 
\vspace{-0.6cm} 
\begin{itemize}
\item[(1)] Ist $f \in S$, so gilt: $K(0, \frac{1}{4}) \subseteq f(\D)$ \vspace{-0.2cm}
\item[(2)] Ist $f \in S$ und $f(\D)$ konvex, so gilt $K(0, \frac{1}{2}) \subseteq f(\D)$ \vspace{-0.2cm}
\item[(3)] Ist $f \in H(\D)$, $F(z) := \frac{1}{z} + f(z)$ $(z \in \D^x)$, $F$ schlicht auf $\D^x$ und sind $w_1, w_2 \in \C \backslash F(\D^x)$, so gilt: $|w_1 - w_2| \leq 4$.
\end{itemize} 
\vspace{-0.3cm}

\textbf{Bemerkung:} 
\vspace{-0.6cm} 
\begin{itemize}
\item[(1)] In (1) ist $\frac{1}{4}$ bestmöglich (vgl. die Koebe'sche Funktion in Bsp. 10.1) \vspace{-0.2cm}
\item[(2)] Auf die Schlichtheit in Satz 10.6 (1) kann nicht verzichtet werden.  \vspace{-0.2cm}
\item[(3)] Sei $F(z) = \frac{1}{z} + z$ $(z \in \D^x)$. \\
Übung: $F$ ist schlicht auf $\D^x$ und $\C \backslash F(\D^x) = [-2,2]$. (Die Schrank $4$ in Satz 10.6 ist also bestmöglich.)
\end{itemize} 
\vspace{-0.3cm}

\textbf{Lemma 10.7:} Sei $f \in S$ und $0 \leq r < 1$. Dann: $|\frac{a f''(a)}{f'(a)} - \frac{2r^2}{1-r^2}| \leq \frac{4r}{1-r^2}$ $(a \in \D, |a| = r)$.

\textbf{Satz 10.8 (Koebe'scher Verzerrungssatz):} Sei $f \in S$ und $0 \leq r < 1$. Dann gilt: $\frac{1-r}{(1+r)^3} \leq |f'(z)| \leq \frac{1+r}{(1-r)^3}$ $(z \in \D, |z| \leq r)$.

\underline{§ 11 Zur Potenzreihendarstellung holomorpher Funktionen}

Sei $\Omega \subseteq \C$ offen, $f \in H(\Omega)$, $z_0 \in \Omega$ und $K(z_0,r) \subseteq \Omega$. \\
Bekannt: $f$ hat eine Potenzreihenentwicklung um $z_0$ mit Konvergenzradius $\geq r$: $f(z) = \sum_{n=0}^\infty a_n (z - z_0)^n$ $(|z-z_0| < r)$. Den Konvergenzradius bezeichnen wir mit $R(f,z_0)$. \\
Bekannt: $R(f,z_0) \geq \text{dist}(z_0, \partial \Omega) \, (:= \infty$, falls $\Omega = \C)$.

\textbf{Definition:} Sei $z_0 \in \C$, $r > 0$, $f \in H(K(z_0, r))$. $\alpha \in \partial K(z_0,r)$ heißt \textbf{regulärer Punkt} von $f :\Leftrightarrow \, \exists \, r_1 > 0 \, \exists \, g \in H(K(\alpha, r_1)) \, \forall \, z \in K(z_0,r) \cap K(\alpha, r_1) \colon g(z) = f(z)$. \\
In diesem Fall ist jedes $\beta \in (\partial K(z_0,r)) \cap K(\alpha, r_1)$ ebenfalls ein regulärer Punkt von $f$. \\
Ist $\alpha \in \partial K(z_0, r)$ kein regulärer Punkt von $f$, so heißt $\alpha$ \textbf{singulärer Punkt} von $f$. \\
Die Menge der singulären Punkte von $f$ ist eine abgeschlossene Menge: Ist $(\alpha_n)$ eine Folge singulärer Punkte in $\partial K(z_0,r)$ mit Grenzwert $\alpha_0$, so ist $\alpha_0$ auch singulär. 

\textbf{Beispiel:} siehe Seite 81-82

\textbf{Satz 11.1:} Sei $f \in H(\D)$, $f(z) = \sum_{n=0}^\infty a_n z^n$ und $R(f,0) = 1$. Dann hat $f$ einen singulären Punkt auf $\partial \D$.

\textbf{Satz 11.2 (Satz von Pringsheim):} Sei $f$ wie in Satz 11.1 mit $a_n \geq 0$ $(n \in \N_0)$. Dann ist $1$ ein singulärer Punkt von $f$.

\textbf{Beispiel:} $f(z) = \sum_{n=1}^\infty \frac{z^n}{n^2}$ hat Konvergenzradius $1$. Die Funktion $z \mapsto f(z)$ ist auf $\overline{\D}$ stetig. Dennoch ist nach Satz 11.2 der Punkt $1$ singulärer Punkt von $f$.

\newpage
\underline{§ 12 Der Satz von Mittag-Leffler}

\textbf{Vorbemerkungen:}
\vspace{-0.6cm}
\begin{itemize}
\item[(1)] $f \in M(\C) \Leftrightarrow \, \exists \, A \subseteq \C \colon A$ ist in $\C$ diskret und $f \in H(\C \backslash A)$ und jedes $a \in A$ ist ein Pol von $f$. In diesem Fall gilt, falls $A = \{ a_1, a_2, \dots \}$ unendlich ist: $|a_n| \to \infty$ $(n \to \infty)$.  \vspace{-0.2cm}
\item[(2)] Sei $\mathcal{P}$ die Menge aller Polynome mit komplexen Koeffizienten und $\mathcal{P}_0 := \{ p \in \mathcal{P} \colon p(0) = 0, p \not\equiv 0 \}$.   \vspace{-0.2cm}
\item[(3)] Ist $\Omega \subseteq \C$ offen, $z_0 \in \Omega$ ein Pol von $f \in H(\Omega \backslash \{ z_0 \})$. Dann existiert ein $p \in \mathcal{P}_0$, so dass $h(z) = p(\frac{1}{z-z_0})$ der Hauptteil der Laurententwicklung von $f$ um $z_0$ ist. \vspace{-0.2cm}
\item[(4)] Es seien $b_1, \dots , b_m \in \C$, $b_j \neq b_k$ $(j \neq k)$ und $p_1, \dots , p_m \in \mathcal{P}_0$. Sei $f(z) = \sum_{k=0}^m p_k (\frac{1}{z - b_k})$. Dann ist $f \in M(\C)$. Jedes $b_k$ ist ein Pol von $f$ und der Hauptteil ist jeweils $p_k (\frac{1}{z-b_k})$ $(k = 1, \dots , m)$.
\end{itemize}
\vspace{-0.3cm}

\textbf{Satz 12.1 (Satz von Mittag-Leffler):} Sei $(b_n)_{n=1}^\infty$ eine Folge in $\C$ mit $b_j \neq b_k$ $(j \neq k)$ und $|b_n| \to \infty$ $(n \to \infty)$. Weiter sei $(p_n)_{n=1}^\infty$ eine Folge in $\mathcal{P}_0$ und $h_n(z) = p_n(\frac{1}{z - b_n})$ $(n \in \N, z \neq b_n)$. Dann existiert eine Folge $(q_n)$ in $\mathcal{P}$ mit
\vspace{-0.6cm}
\begin{itemize}
\item[(1)] Die Reihe $\sum_{n=1}^\infty (h_n - q_n)$ konvergiert auf $\Omega := \C \backslash \{ b_k \colon k\in \N \}$ lokal gleichmäßig. \vspace{-0.2cm}
\item[(2)] Für $f(z) := \sum_{n=1}^\infty (h_n(z) - q_n(z))$ $(z \in \Omega)$ gilt: $f \in M(\C)$ und $f$ hat in $b_n$ einen Pol mit Hauptteil $h_n$ $(n \in \N)$, und $f$ hat keinen weiteren Pol.
\end{itemize}
\vspace{-0.3cm}

\textbf{Bemerkung:}
\vspace{-0.6cm}
\begin{itemize}
\item[(1)] $\sum_{n=1}^\infty h_n$ konvergiert im Allgemeinen nicht lokal gleichmäßig auf $\Omega$. \vspace{-0.2cm}
\item[(2)] $(b_n)$, $(h_n)$ seien wie in Satz 12.1. Ist $g \in M(\C)$ und hat $g$ genau die Pole $b_n$ mit Hauptteil $h_n$ $(n \in \N)$, so existiert ein $\varphi \in H(\C)$ mit $g = \sum_{n=1}^\infty (h_n - q_n) + \varphi$. \vspace{-0.2cm}
\item[(3)] Satz 12.1 gilt analog für eine beliebige offene Teilmenge $\Omega$ von $\C$ (mit $\{ b_n \colon n \in \N \}$ diskret in $\Omega$). Die Beweisidee ist analog. Der Beweis ist technisch aufwendiger. 
\end{itemize}
\vspace{-0.3cm}

\underline{§ 14 Unendliche Produkte}

\textbf{Definition:} Sei $(a_n)_{n=1}^\infty$ eine Folge in $\C$ und $P_n := \prod_{k = 1}^n (1 + a_k)$ $(n \in \N)$. $(P_n)_{n=1}^\infty$ heißt ein \textbf{unendliches Produkt} und wird mit $\prod_{k = 1}^\infty (1 + a_k)$ bezeichnet. \\
$\prod_{k = 1}^\infty (1 + a_k)$ heißt \textbf{konvergent} $: \Leftrightarrow$ $(P_n)$ ist konvergent. Dann wird $P := \lim_{n \to \infty} P_n$ auch mit $\prod_{k = 1}^\infty (1 + a_k)$ bezeichnet.

\textbf{Lemma 14.1:} $(a_n), (P_n)$ seien wie oben. Es sei $\prod_{n = 1}^\infty (1 + a_n)$ konvergent und $P = \prod_{n = 1}^\infty (1 + a_n) \neq 0$. Dann gilt: $a_n \to 0$ $(n \to \infty)$.

\textbf{Lemma 14.2:} $(a_n), (P_n)$ seien wie oben. $q_n := \prod_{k = 1}^n (1 + |a_k|)$ $(n \in \N)$. Dann gilt:
\vspace{-0.6cm}
\begin{itemize}
\item[(1)] $q_n \leq \exp(|a_1| + \dots + |a_n|)$ $(n \in \N)$. \vspace{-0.2cm}
\item[(2)] $|p_n - 1| \leq q_n - 1$ $(n \in \N)$.
\end{itemize} 
\vspace{-0.3cm}

\textbf{Definition:} Sei $\Omega \subseteq \C$ offen (bel.) und $(f_n)_{n=1}^\infty$ eine Folge von Funktionen $f_n \colon \Omega \to \C$. Sei $P_n(z) := \prod_{j=1}^n (1 + f_j(z))$ $(z \in \Omega, n \in \N)$. Dann: \\
$\prod_{j=1}^\infty (1 + f_j)$ heißt auf $\Omega$ punktweise/gleichmäßig/lokal gleichmäßig konvergent \\
$:\Leftrightarrow$ $(P_n)$ konvergiert auf $\Omega$ punktweise/gleichmäßig/lokal gleichmäßig.

\textbf{Definition:} Sei $\Omega \subseteq \C$ offen, $f \in H(\Omega)$ und $z_0 \in \Omega$. \\
ord$(f,z_0) :=
  \begin{cases} 
0 &, \text{ falls } f(z_0) \neq 0 \\ 
k &, \text{ falls } z_0 \text{ Nullstelle } k\text{-ter Ordnung von } f \text{ ist.}
\end{cases}
$

\textbf{Bemerkung:} Ist $g \in H(\Omega)$, so gilt ord$(fg, z_0) = $ ord$(f, z_0) + $ ord$(g, z_0)$.

\textbf{Satz 14.3:} Sei $\Omega \subseteq \C$ ein Gebiet, $(f_n)$ eine Folge in $H(\Omega)$ mit $f_n \not\equiv 0$ $(n \in \N)$ und $\sum_{n=1}^\infty |1-f_n|$ konvergiert auf $\Omega$ lokal gleichmäßig. Dann gilt:
\vspace{-0.6cm}
\begin{itemize}
\item[(1)] $\prod_{n=1}^\infty f_n$ konvergiert auf $\Omega$ lokal gleichmäßig und für $f(z) := \prod_{n=1}^\infty f_n(z)$ $(z \in \Omega)$ gilt: $f \in H(\Omega)$. \vspace{-0.2cm}
\item[(2)] Für $z_0 \in \Omega$ gilt:
\vspace{-0.2cm}
\begin{itemize}
\item[(a)] $f_k(z_0) = 0$ für höchstens endlich viele $k \in \N$.
\item[(b)] $f(z_0) = 0 \Leftrightarrow \, \exists \, k \in \N \colon f_k(z_0) = 0$.
\item[(c)] ord$(f,z_0) = \sum_{k=1}^\infty$ ord$(f_k, z_0)$. (Beachte (a).)
\end{itemize}
\end{itemize}
\vspace{-0.3cm}

\underline{§ 15 Der Weierstraß'sche Produktsatz}

\textbf{Satz 15.1:} Für $k \in \N$ sei $(z \in \C)$ \\
$g_k(z) := z + \frac{z^2}{2} + \dots + \frac{z^k}{k}$, \\
$f_k(z) = \exp(g_k(z))$, \\
$E_0(z) := 1-z$, $E_k(z) := (1-z)f_k(z) = (1-z) \exp (z + \frac{z^2}{2} + \dots + \frac{z^k}{k})$. Es gilt:
\vspace{-0.6cm}
\begin{itemize}
\item[(1)] $E_k \in H(\C)$ $(k \in \N_0)$ \\
$E_k(0) = 1$, $E_k(1) = 0$ $(k \in \N_0)$ \vspace{-0.2cm}
\item[(2)] Sei $k \in \N$ und $f_k(z) = \sum_{n=0}^\infty a_n z^n$ die Potenzreihenentwicklung von $f_k$ um $0$. Dann ist $a_n \geq 0$ $(n \in \N_0)$. \vspace{-0.2cm}
\item[(3)] $E_k'(z) = -z^kf_k(z)$ $(k \in \N, z \in \C)$ \vspace{-0.2cm}
\item[(4)] $|1-E_k(z)| \leq |z|^{k+1}$ $(k \in \N_0, z \in \overline{\D})$.
\end{itemize}
\vspace{-0.3cm}

\textbf{Vorbemerkung 15.2:} Sei $(z_n)$ eine Folge in $\C \backslash \{ 0 \}$ mit $|z_n| \to \infty$ $(n \to \infty)$. Sei $R > 0$. Dann existiert ein $n_0 = n_0(R) \in \N$ mit $|z_n| \geq 2R$ $(n \geq n_0)$, also $(\frac{R}{|z_n|})^n \leq (\frac{1}{2})^n$ $(n \geq n_0)$. Damit ist die Reihe $\sum_{n=1} (\frac{R}{||z_n})^n$ konvergent. \\
Fazit: Es gibt Folgen $(p_n)$ in $\N_0$ mit: $\sum_{n=1}^\infty (\frac{R}{|z_n|})^{p_n + 1}$ ist konvergent für jedes $R > 0$. Bsp: $p_n = n-1$.

\textbf{Beispiel:} siehe Seite 98-99

\textbf{Satz 15.3 (Produktsatz von Weierstraß):} Sei $(z_n)_{n = 1}^\infty$ eine Folge in $\C^x$ mit $|z_n| \to \infty$ $(n \to \infty)$ und $(p_n)$ eine Folge in $\N_0$ wie in Satz 15.2. Dann konvergiert das Produkt $f(z) := \prod_{n=1}^\infty E_{p_n} ( \frac{z}{z_n} )$ auf $\C$ lokal gleichmäßig und es gilt:
\vspace{-0.6cm}
\begin{itemize}
\item[(1)] $f \in H(\C)$ \vspace{-0.2cm}
\item[(2)] $f^{-1}(\{ 0 \}) = \{ z_1, z_2, \dots \}$ \vspace{-0.2cm}
\item[(3)] Ist $a \in \C$ und $|\{ n \in \N\colon z_n = a \}| = k$, so ist ord$(f,a) = k$.
\end{itemize}
\vspace{-0.3cm}

\textbf{Satz 15.4:} Sei $(\alpha_n)_{n=1}^\infty$ eine Folge in $\C$ mit $|\alpha_n| \to \infty$ $(n \to \infty)$ und $\alpha_j \neq \alpha_k$ für $j \neq k$. Weiter sei $(m_n)_{n = 1}^\infty$ eine Folge in $\N$. Dann existiert ein $f \in H(\C)$ mit:
\vspace{-0.6cm}
\begin{itemize}
\item[(1)] $f^{-1}(\{ 0 \}) = \{ \alpha_1, \alpha_2, \dots \}$ und \vspace{-0.2cm}
\item[(2)] ord$(f,\alpha_n) = m_n$ $(n \in \N)$.
\end{itemize}
\vspace{-0.3cm}

\textbf{Beispiel:} siehe Seite 101-102

\newpage
\textbf{Satz 15.5. (Eine Folgerung aus dem Produktsatz von Weierstraß):} Sei $f \in H(\C)$, $k :=$ ord$(f,0)$, $f^{-1}(\{ 0 \})\backslash \{ 0 \} = \{ \alpha_1, \alpha_2, \dots \}$ und $m_j := $ ord$(f, \alpha_j)$ $(j \in \N)$. Jede Nullstelle $\alpha_j$ sei in $(z_n)$ genau $m_j$-mal aufgeführt. Dann existiert eine Folge $(p_n)$ in $\N_0$ und ein $g \in H(\C)$ mit $f(z) = z^k \left( \prod_{n=1}^\infty E_{p_n} (\frac{z}{z_n}) \right) e^{g(z)}$ $(z \in \C)$. Das Produkt konvergiert dabei lokal gleichmäßig auf $\C$.

\textbf{Satz 15.6:} $f \in M(\C) \Leftrightarrow \, \exists \, g,h \in H(\C), h \neq 0 \colon f = \frac{g}{h}$.

\underline{§ 16 Der Ring $H(\C)$}

\textbf{Satz 16.1 (Ein Interpolationssatz):} Es sei $(\alpha_k)_{k = 1}^\infty$ eine Folge in $\C$ mit $\alpha_k \neq \alpha_j$ $(k \neq j)$ $|a_k| \to \infty$ $(k \to \infty)$, $(m_k)_{k=1}^\infty$ eine Folge in $\N_0$ und $w_{n,k} \in \C$ $(0 \leq n \leq m_k)$. Dann existiert ein $f \in H(\C)$ mit $f^{(n)}(\alpha_k) = n! \, w_{n,k}$ $(k \in \N, 0 \leq m \leq m_k)$.

Sei $\Omega \subseteq \C$ offen. Dann ist $H(\Omega)$ bezüglich $+$ und $\cdot$ ein Ring mit Einselement (kommutativ).

\textbf{Satz 16.2:} Ist $\Omega \subseteq \C$ ein Gebiet, so ist $H(\Omega)$ nullteilerfrei, d.h. $f \cdot g = 0$, $f,g \in H(\Omega) \Rightarrow f=0 \vee g = 0$.

\textbf{Bemerkung:} Ist $\Omega \subseteq \C$ offen und kein Gebiet, so ist $H(\Omega)$ nicht nullteilerfrei.

\textbf{Definition:} Sei $(R,+,\cdot)$ ein kommutativer Ring mit Einselement $\mathbb{1}$. Ist $S \subseteq R$ bzgl. $+$ und $\cdot$ ein Ring, so heißt $S$ Unterring von $R$. Ist $I \subseteq R$ ein Unterring von $R$ und gilt zusätzlich: $a \in I, b \in R \Rightarrow a \cdot b \in I$, so heißt $I$ ein \textbf{Ideal} in $R$.

\textbf{Bemerkung:} Sind $a_1, \dots, a_n \in R$, so bezeichnet $[a_1, \dots, a_n] := \left\lbrace \sum_{k=1}^n b_k a_k \colon b_1, \dots , b_n \in R \right\rbrace$ das sogenannte von $a_1, \dots, a_n$ erzeugte Ideal. Ist $I \subseteq R$ ein Ideal und existieren $a_1, \dots , a_n \in R$ mit $I = [a_1, \dots , a_n]$, so heißt $I$ endlich erzeugt. Existiert ein $a \in R$ mit $I=[a]$, so heißt $I$ ein Hauptideal von $R$. Ist jedes Ideal in $R$ ein Hauptideal, so heißt $R$ ein Hauptidealring. Es gilt stets $[1] = R$.

\textbf{Beispiel:} $H(\C)$ ist kein Hauptidealring.

Andererseits gilt aber: \\
\textbf{Satz 16.3:} Jedes endlich erzeugte Ideal in $H(\C)$ ist ein Hauptideal, d.h: Sind $g_1, \dots, g_n \in H(\C)$, so existieren Funktionen $g, f_1, \dots, f_n, h_1, \dots, h_n$ mit $g = \sum_{k=1}^n f_k g_k$ und $g_k = h_k g$ $(k = 1, \dots , n)$. (Genau dann ist $[g_1, \dots , g_n] = [g]$).

\textbf{Bemerkung:} Da $H(\C)$ kein Hauptidealring ist, folgt: Nicht jedes Ideal in $H(\C)$ ist endlich erzeugt.

\underline{§ 17 Die Jensen'sche Formel}

\textbf{Hilfssatz 17.1:} $\frac{1}{2\pi} \int_0^{2\pi} \log |1-e^{it}| \, dt = 0$.
\vspace{-0.6cm}
\begin{itemize}
\item Dieses Integral ist als Lebesgue-Integral gemeint, kann aber auch als uneigentliches Riemann-Integral aufgefasst werden. \vspace{-0.2cm}
\item Es folgt $\frac{1}{2\pi} \int_0^{2\pi} \log |1-e^{i(t - \varphi)}| \, dt = 0$ $(\varphi \in \R)$.
\end{itemize}
\vspace{-0.3cm}

\textbf{Satz 17.2 (Jensen'sche Formel):} Sei $\Omega = K(0,R)$, $f \in H(\Omega)$, $f(0) \neq 0$, $0 < r < R$ und $\alpha_1, \dots , \alpha_m$ die Nullstellen von $f$ in $\overline{K(0,r)}$, aufgezählt mit Vielfachheit. Dann gilt: \\
$|f(0)| \cdot \prod_{n=1}^m \frac{r}{|\alpha_n|} = \exp (\frac{1}{2 \pi} \int_0^{2\pi} \log |f(re^{it})| dt)$.

\textbf{Eine Anwendung auf die Nullstellenverteilung ganzer Funktionen:} siehe Seite 115

\underline{§ 18 Periodische Funktionen}

Im Folgenden sei stets $f \in M(\C)$, $A$ die Menge der Pole von $f$ und $f$ nicht konstant. Wir setzen $f(a) = \infty$ $(a \in A)$. Wegen $|f(z)| \to \infty$ für $z \to a \in A$ ist damit $f \colon \C \to \hat{\C}$ stetig.

\textbf{Definition:} $w \in \C$ heißt eine Periode von $f :\Leftrightarrow f(z+w) = f(z)$ $(z \in \C)$. \\
Per$(f) := \{ w \in \C \colon w \text{ ist eine Periode von } f \}$.
\vspace{-0.6cm}
\begin{itemize}
\item Stets ist $0 \in$ Per$(f)$. \vspace{-0.2cm}
\item $a \in A$, $w \in$ Per$(f) \Rightarrow a+w \in A$
\end{itemize}
\vspace{-0.3cm}

\textbf{Beispiel:}
\vspace{-0.6cm}
\begin{itemize}
\item $f(z) = z$; Per$(f) = \{ 0 \}$ \vspace{-0.2cm}
\item $f(z) = e^z$; Per$(f) = \{ 2k\pi i \colon k \in \Z \}$
\end{itemize}
\vspace{-0.3cm}

\textbf{Bez.:} Für $a,b \in \C$ sei \\
$\Z a := \{ ka \colon k \in \Z \}$ \\
$\Z a + \Z b = \{ ka + jb \colon k,j \in \Z \}$

\textbf{Lemma 18.1:} 
\vspace{-0.6cm}
\begin{itemize}
\item[(a)] $w_1, w_2 \in$ Per$(f) \Rightarrow \Z w_1 + \Z w_2 \subseteq$ Per$(f)$ \vspace{-0.2cm}
\item[(b)] Per$(f)$ ist diskret in $\C$. 
\end{itemize}
\vspace{-0.3cm}

\textbf{Bemerkung:} $\C$ ist ein reeller Vektorraum der Dimension $2$. \\
Für $w_1, w_2 \in \C^x$ gilt: Sind $w_1, w_2$ linear abhängig über $\R$, so existiert ein $\alpha \in \R$ mit $w_2 = \alpha w_1 \Rightarrow \frac{w_1}{w_2} \in \R \Rightarrow \text{Im}(\frac{w_1}{w_2}) = 0$. Ist umgekehrt Im$(\frac{w_1}{w_2}) = 0$, so existiert ein $\alpha \in \R$ mit $w_2 = \alpha w_1$; $w_1, w_2$ sind also linear abhängig. \\
Also: $w_1, w_2$ linear abhängig $\Leftrightarrow$ Im$(\frac{w_1}{w_2}) = 0$ $\Leftrightarrow$ Im$(\frac{w_2}{w_1}) = 0$. \\
Weiter gilt: Sind $z_1, z_2$ linear unabhängig über $\R$, so gilt: $|z_1 + z_2| < |z_1| + |z_2|$.

\textbf{Satz und Definition 18.2:} Für Per$(f)$ gibt es folgende 3 Möglichkeiten:
\vspace{-0.6cm}
\begin{itemize}
\item[(1)] Per$(f) = \{ 0 \}$. Dann heißt $f$ \textbf{unperiodisch}. \vspace{-0.2cm}
\item[(2)] $\exists \, w_1 \in \C^x \colon$ Per$(f) = \Z w_1$. Dann heißt $f$ \textbf{einfach periodisch} und $w_1$ heißt \textbf{primitive Periode} von $f$. \vspace{-0.2cm}
\item[(3)] $\exists \, w_1, w_2 \in \C^x$ linear unabhängig über $\R$: Per$(f) = \Z w_1 + \Z w_2$. Dann heißt $f$ \textbf{elliptisch} oder \textbf{doppelperiodisch} und $(w_1, w_2)$ heißt ein \textbf{primitives Periodenpaar} von $f$.
\end{itemize}
\vspace{-0.3cm}

\textbf{Bemerkung:} Im Fall (3) heißt $\Z w_1 + \Z w_2$ auch das Periodengitter der Funktion $f$. (siehe Beispiel Seite 121)

\underline{§ 19 Elliptische Funktionen}

Im Folgenden seien stets $w_1, w_2 \in \C^x$ linear unabhängig über $\R$ und $H := \Z w_1 + \Z w_2$.

\textbf{Definition:} $f \in M(\C)$ heißt \textbf{elliptisch} bzgl. $H$ $:\Leftrightarrow H \subseteq$ Per$(f)$. In diesem Fall schreiben wir $f \in \mathcal{K}(H)$.

\newpage
\textbf{Bemerkung:}
\vspace{-0.6cm}
\begin{itemize}
\item[(1)] Konstante Funktionen gehören stets zu $\mathcal{K}(H)$. Ist $f$ konstant, so ist Per$(f) = \C$. \vspace{-0.2cm}
\item[(2)] Ist $f \in M(\C)$ und $(\frac{w_1}{2}, \frac{w_2}{2})$ ein primitives Periodenpaar von $f$, also Per$(f) = \Z \frac{w_1}{2} + \Z \frac{w_2}{2}$, so gilt $H \subsetneqq$ Per$(f)$, insbesondere ist $f \in \mathcal{K}(H)$.
\end{itemize}
\vspace{-0.6cm}

Sei $u \in \C$ und $P_u := \{ u+\alpha_1 w_1 + \alpha_2 w_2 \colon \alpha_1, \alpha_2 \in [0,1) \}$. $P_u$ heißt halboffenes Periodenparallelogramm. \\
Im Folgenden sei stets $f \in \mathcal{K}(H)$, $A$ die Menge der Pole von $f$ und $N$ die Menge der Nullstellen von $f$.

\textbf{Bemerkung:}
\vspace{-0.6cm}
\begin{itemize}
\item[(1)] Kennt man $f$ auf einem $P_u$, so kennt man $f$ auf $\C$. \vspace{-0.2cm}
\item[(2)] Die Menge $A \cap P_u$ ist stets endlich. Ist $f \neq 0$, so ist $N \cap P_u$ stets endlich. Im Fall $f \neq 0$ kann $u$ stets so gewählt werden, dass $A \cap \partial P_u = \emptyset \, [= N \cap \partial P_u]$ gilt. \vspace{-0.2cm}
\item[(3)] Ist nun $u$ so gewählt, dass $A \cap \partial P_u = \emptyset$, und $\gamma_1, \dots , \gamma_4$ wie im Bild (Parameterdarstellung der Verbindungsstrecke) und $\gamma = \gamma_1 \oplus \gamma_2 \oplus \gamma_3 \oplus \gamma_4$, so gilt $\gamma^* = \partial P_u$, sowie \\
 $\int_{\gamma_1} f(z) \, dz = - \int_{\gamma_3} f(z) \, dz$ (da $w_1 \in$ Per$(f)$), \\
$\int_{\gamma_2} f(z) \, dz = - \int_{\gamma_4} f(z) \, dz$ (da $w_2 \in$ Per$(f)$). \\
Es folgt: $\int_\gamma f(z) \, dz = \sum_{k=1}^4 \int_{\gamma_k} f(z) \, dz = 0$.
\end{itemize}
\vspace{-0.3cm}

\textbf{Satz 19.1 (Erster Satz von Liouville):} Ist zusätzlich $f \in H(\C)$, so ist $f$ konstant.

\textbf{Satz 19.2 (Zweiter Satz von Liouville):} Für jedes $u \in \C$ gilt: $\sum_{a \in A \cap P_u}$ Res$(f,a) = 0$ (mit $\sum_\emptyset := 0$).

\textbf{Satz und Definition 19.3:} ord$(f) := $ Anzahl der Pole in einem $P_u$ (einschließlich Vielfachheit) von $f$. Es gilt: Ist $f$ nicht konstant, so ist ord$(f) \geq 2$.

\textbf{Bemerkung:} Ist also $f$ nicht konstant, so hat $f$ in jedem $P_u$ $(u \in \C)$ mindestens 2 Pole oder mindestesns einen Pol 2. Ordnung.

\textbf{Satz 19.4:} Es gilt: Ist $f \in \mathcal{K}(H)$, so ist $f' \in \mathcal{K}(H)$. Ist  $f \in \mathcal{K}(H)$ nicht konstant, so ist $\frac{f'}{f} \in \mathcal{K}(H)$. 

\textbf{Satz 19.5 (Dritter Satz von Liouville):} $f$ sei nicht konstant, $a_1, \dots , a_n$ seien die Nullstellen und $b_1, \dots , b_p$ seien die Polstellen von $f$ in einem $P_u$ (jeweils einschließlich Vielfalchheit). Dann gilt: $n = p$ ($= \text{ord}(f)$).

\textbf{Lemma 19.6:} Die Reihe $\sum_{w \in H\backslash \{ 0 \}} \frac{1}{|w|^3}$ ist konvergent.

\textbf{Satz und Definition 19.7:} Die Reihe $\frac{1}{z^2} + \sum_{w \in H\backslash \{ 0 \}} \left(\frac{1}{(z-w)^2} - \frac{1}{w^2}\right)$ konvergiert auf $\C \backslash H$ lokal gleichmäßig und für $f(z) := \frac{1}{z^2} + \sum_{w \in H\backslash \{ 0 \}} \left(\frac{1}{(z-w)^2} - \frac{1}{w^2}\right)$ gilt: 
\vspace{-0.6cm}
\begin{itemize}
\item[(1)] $f \in M(\C)$ \vspace{-0.2cm}
\item[(2)] $f\in \mathcal{K}(H)$ und ord$(f) = 2$. \vspace{-0.2cm}
\end{itemize}
\vspace{-0.3cm}
$f$ heißt die Weierstraß'sche $\rho$-Funktion zu $H$; $\rho(z) := f(z)$.

\textbf{Bemerkung:} Man kann zeigen: $\mathcal{K}(H)$ ist ein Körper (vgl. Satz 19.4)

\textbf{Satz 19.8:} Ist $f \in \mathcal{K}(H)$, so existieren rationale Funktionen $R_1, R_2$ mit $f = R_1(\rho) + \rho'\cdot R_2(\rho)$.

\newpage
\textbf{Satz (Die Differentialgleichung der Weierstraß'schen $\boldsymbol{\rho}$-Funktion:):} Es gilt \\
$(\rho'(z))^2 = 4(\rho(z))^3 - c_1 \rho(z) - c_2$ $(z \in \C \backslash H)$ mit $c_1 = 60 \sum_{w \in H \backslash \{ 0 \}} \frac{1}{w^4}$, $c_2 = 140 \sum_{w \in H \backslash \{ 0 \}} \frac{1}{w^6}$. \\
Hieraus erhält man die Differentialgleichung \\
$(\rho'(z))^2 = 4(\rho(z) - \rho(\frac{w_1}{2})) \cdot (\rho(z) - \rho(\frac{w_2}{2})) \cdot (\rho(z) - \rho(\frac{w_1 + w_2}{2}))$ $(z \in \C \backslash H)$. 

\underline{§ 20 Der Fixpunktsatz von Earle-Hamilton in $\C$}

Sei $\Omega \subseteq \C$ ein Gebiet und $M \subseteq \Omega$. Wir sagen: $M$ liegt strikt in $\Omega :\Leftrightarrow \, \exists \, \varepsilon > 0 \, \forall \, z \in M \colon K(z, \varepsilon) \subseteq \Omega$. 
\vspace{-0.6cm}
\begin{itemize}
\item $\Omega$ liegt strikt in $\Omega \Leftrightarrow \Omega = \C$.
\end{itemize}
\vspace{-0.3cm}

\textbf{Beispiel:} siehe Seite 130

\textbf{Lemma 20.1:} $\alpha \colon \Omega \to \R$ ist stetig.

Nun sei $\gamma \colon [0,1] \to \Omega$ ein ssd. Weg. $W(\gamma) := \int_0^1 \alpha (\gamma(t)) \, |\gamma'(t)|\, dt$. \\
Beachte: Nach Lemma 20.1 existiert dieses Integral.  \\
Es sei $d \colon \Omega \times \Omega \to \R$ definiert durch $d(z,w) := \inf \underbrace{\{ W(\gamma) \colon \gamma(0) = z, \, \gamma(1) = w \}}_{\neq \emptyset, \text{ da } \Omega \text{ Gebiet}}$. \\

\textbf{Bemerkung:} Ist $\Omega = \C$, so ist $\alpha(z) = 0$ $(z \in \C)$ (Satz von Liouville), also $d(z,w) = 0$ $(z,w \in \Omega \,(= \C))$.

\textbf{Definition:} Eine Abbildung $d \colon \Omega \times \Omega \to \R$ heißt \textbf{Halbmetrik}, wenn alle Eigenschaften einer Metrik bis auf $d(z,w) = 0 \Rightarrow z = w$ erfüllt sind.  \\
Ist $\Omega$ beschränkt, so ist $d$ eine Metrik (vgl. Lemma 20.2 und Beweis von Satz 20.3).

\textbf{Lemma 20.2:} $d$ ist eine Halbmetrik auf $\Omega$.

\textbf{Satz 20.3 (Fixpunktsatz von Earle-Hamilton in $\C$):} Sei $f \in H(\Omega)$, $f(\Omega) \subseteq \Omega$ und $f(\Omega)$ sei beschränkt und liege strikt in $\Omega$. Dann hat $f$ genau einen Fixpunkt. 
\vspace{-0.6cm}
\begin{itemize}
\item Beispiel: siehe Seite 137 \vspace{-0.2cm}
\item Im Fall $\Omega = \C$ folgt Satz 20.3 direkt aus dem Satz von Liouville. Ist $f \in H(\C)$ und $f(\C)$ beschränkt, so ist $f$ konstant, hat also genau einen Fixpunkt. \vspace{-0.2cm}
\item Satz 20.3 gilt analog (mit fast demselben Beweis) für holomorphe Abbildungen in (un-)endlich dimensionalen komplexen Vektorräumen.
\end{itemize}
\vspace{-0.3cm}

\underline{§ 21 Die Subordination}

\textbf{Definition:} 
\vspace{-0.6cm}
\begin{itemize}
\item[(1)] Für $r > 0$ sei $r \D := \{ rz \colon z \in \D \} = K(0,r)$ \vspace{-0.2cm}
\item[(2)] Seien $f,g \in H(\D)$. $g \ll f :\Leftrightarrow f$ ist auf $\D$ injektiv, $f(0) = g(0)$ und $g(\D) \subseteq f(\D)$.
\end{itemize}
\vspace{-0.3cm}

\textbf{Satz 21.1 (Prinzip der Subordination):} Seien $f,g \in H(\D)$ und $g \ll f$. Sei $h := f^{-1} \circ g$ (Beachte: $h \in H(\D)$ wegen $g(\D) \subseteq f(\D)$). Dann gilt: 
\vspace{-0.6cm}
\begin{itemize}
\item[(1)] $|h(z)| \leq |z|$ $(z \in \D)$ und $|h'(0)| \leq 1$. \vspace{-0.2cm}
\item[(2)] $g(r\D) \subseteq f(r \D)$ $(r \in (0,1])$.
\end{itemize}
\vspace{-0.3cm}

\textbf{Satz 21.2:} Sei $f \in H(\D)$ injektiv und $f(\D)$ konvex. Dann ist $f(r\D)$ konvex ($r \in (0,1]$).

\newpage
\textbf{Satz 21.3:} Seien $f,g \in H(\D)$, $g \ll f$ und $g(0)=0=f(0)$. Es seien $f(z)=\sum_{n=1}^\infty a_n z^n$, $g(z)=\sum_{n=1}^\infty b_n z^n$ $(z \in \D)$ die Potenzreihenentwicklungen von $f$ und $g$. Dann gilt:
\vspace{-0.6cm}
\begin{itemize}
\item[(1)] $|b_1| \leq |a_1|$ ($\Leftrightarrow |g'(0)| \leq |f'(0)|$) \vspace{-0.2cm}
\item[(2)] Ist $f(\D)$ konvex, so ist $|b_n| \leq |a_1|$ $(n \in \N)$.
\end{itemize}
\vspace{-0.3cm}

\textbf{Satz 21.4 (Satz von Borel):} Sei $f \in H(\D)$, $f(\D) \subseteq \{ z \in \C \colon \text{Re} z > 0 \}$, $f(0) = 1$ und $f(z) = 1 + \sum_{n=1}^\infty a_n z^n$ $(z \in \D)$ die Potenzreihe von $f$. Dann ist $|a_n| \leq 2$ $(n \in \N)$.

\textbf{Lemma 21.5:} 
\vspace{-0.6cm}
\begin{itemize}
\item[(1)] $\gamma(x) := -x \log x$ $(x \in (0,1))$. Dann gilt: $\gamma(x) \leq \gamma(\frac{1}{e}) = \frac{1}{e}$ $(x \in (0,1))$ \vspace{-0.2cm}
\item[(2)] Sei $g \in H(\D)$, $g(0) > 0$ und $|g|>1$ auf $\D$. Dann existiert ein $h \in H(\D)$: $g = e^h$, Re$h > 0$ auf $\D$ und $h(0) \in \R$.
\end{itemize}
\vspace{-0.3cm}

\textbf{Satz 21.6:} Sei $f \in H(\D)$ und $0 < |f| < 1$ auf $\D$. Dann gilt $|f'(0)| \leq \frac{2}{e}$.

\textbf{Bemerkung:} Die Schranke $\frac{2}{e}$ in Satz 21.6 ist bestmöglich. 

\underline{§ 22 Verbindungen zur Funktionalanalysis und die Sätze von Montel und Vitali}

Im Folgenden sei stets $\Omega \subseteq \C$ ein Gebiet. \\
Ist $K \subseteq \Omega$, $|K| = \infty$ kompakt, so sei $||f||_K := \max_{z \in K} |f(z)|$ $(f \in H(\Omega))$. \\
Übung: $||\cdot||_K$ ist eine Norm auf $H(\Omega)$. Allerdings ist $(H(\Omega), ||\cdot||_K)$ stets unvollständig. 

\textbf{Beispiel:} siehe Seite 145

\textbf{$\boldsymbol{H(\Omega)}$ als metrischer Raum (vgl. § 2):} \\
Es existiert eine Folge kompakter Mengen $K_1 \subseteq K_2 \subseteq K_3 \subseteq \dots $ mit $\bigcup_{n=1}^\infty K_n = \Omega$ und jede kompakte Teilmenge $K \subseteq \Omega$ liegt in einem $K_n$. \\
Setze $d(f,g) := \sum_{n=1}^\infty \frac{1}{2^n} \underbrace{\frac{||f-g||_{K_n}}{1+||f-g||_{K_n}}}_{\leq 1}$ $(f,g \in H(\Omega))$.

\textbf{Lemma 22.1:} Es gilt:
\vspace{-0.6cm}
\begin{itemize}
\item[(1)] $d$ ist eine Metrik auf $H(\Omega)$ \vspace{-0.2cm}
\item[(2)] $d(f+h,g+h) = d(f,g)$ $(f,g,h \in H(\Omega))$ ($d$ ist translationsinvariant) \vspace{-0.2cm}
\item[(3)] $f_k \to f$ in $(H(\Omega,d)) \Leftrightarrow (f_k)$ konvergiert lokal gleichmäßig gegen $f$. \vspace{-0.2cm}
\item[(4)] $+\colon H(\Omega) \times H(\Omega) \to H(\Omega)$, $\cdot \colon \C \times H(\Omega) \to H(\Omega)$ sind stetig.
\end{itemize}
\vspace{-0.3cm}

\textbf{Satz 22.2:} $(H(\Omega),d)$ ist vollständig.

\textbf{Bemerkung:} $H(\Omega)$ versehen mit der durch $d$ erzeugten Topologie ist ein sogenannter Fréchetraum (ein lokalkonvexer topologischer Vektorraum, dessen Topologie durch eine vollständige translationsinvariante Metrik erzeugt wird).

\newpage
Der Satz von Montel (Satz 2.2) kann nun folgendermaßen formuliert werden (als Kompaktheitskriterium):
\textbf{Satz 22.3:} Sei $\mathcal{F} \subseteq H(\Omega)$. Dann ist $\mathcal{F}$ genau dann eine kompakte Teilmenge des metrischen Raums $(H(\Omega),d)$, wenn gilt:
\vspace{-0.6cm}
\begin{itemize}
\item[(1)] $\mathcal{F}$ ist abgeschlossen. \vspace{-0.2cm}
\item[(2)] Es gibt eine Folge $(b_n)$ in $[0,\infty)$ mit $||f||_{K_n} \leq b_n$ $(n \in \N, f \in \mathcal{F})$.
\end{itemize}
\vspace{-0.3cm}

\textbf{Satz 22.4 (Satz von Vitali):} $(f_k)$ sei eine lokal gleichmäßig beschränkte Folge in $H(\Omega)$. Es sei $M \subseteq \Omega$ eine Menge mit Häufungspunkt in $\Omega$. Dann gilt: Konvergiert $(f_k(z))$ für jedes $z \in M$, so ist $(f_k)$ auf $\Omega$ lokal gleichmäßig konvergent.

\textbf{Spezielle Banachräume analytischer Funktionen:} Einige Unterräume von $H(\Omega)$ liefern bei geeigneter Norm klassische Banachräume. Wir betrachten den Fall $\Omega = \D$. \\
A$(\D) := \{ f \colon \overline{\D} \to \C \,|\, f \text{ ist stetig auf } \overline{\D} \text{ und holomorph auf } \D \}$ versehen mit der Maximumnorm $||f||_{\infty} = \max_{z \in \overline{\D}} |f(z)|$ ist ein Banachraum. \\
$H^\infty (\D) := \{ f \in H(\D) \colon f \text{ ist beschränkt auf } \D \}$ mit $||f||_\infty := \sup_{|z| < 1} |f(z)|$ ist ein Banachraum.

\textbf{Definition:} Sei $\Omega \subseteq \C$ offen und $u \colon \Omega \to \R$ stetig. $u$ heißt \textbf{subharmonisch} auf $\Omega$, falls für jede Kreisscheibe $\overline{K(z,r)} \subseteq \Omega$ gilt: $u(z) \leq \frac{1}{2\pi} \int_0^{2\pi} u(z + re^{it}) \, dt$ (vgl. MWE für harmonische Funktionen).

\textbf{Satz 22.5:} Ist $u \colon \Omega \to \R$ subharmonisch und ist $\varphi \colon \R \to \R$ wachsend und konvex, so ist auch $\varphi \circ u \colon \Omega \to \R$ subharmonisch. 

\textbf{Bemerkung:} Analog zeigt man: Ist $u\colon \Omega \to [0,\infty)$ subharmonisch und ist $\varphi \colon [0,\infty) \to \R$ stetig, wachsend und konvex, so ist $\varphi \circ u$ subharmonisch.

\textbf{Lemma 22.6:} Sei $\Omega \subseteq \C$ offen und $f \in H(\Omega)$. Dann ist $u \colon \Omega \to [0, \infty)$, $u(z) = |f(z)|^p$ für jedes $p \geq 1$ subharmonisch.

\textbf{Satz 22.7:} Sei $\Omega \subseteq \C$ offen, $K \subseteq \Omega$ kompakt, $v \colon K \to \R$ stetig und harmonisch auf $\dot{K}$ und $u \colon \Omega \to \R$ subharmonisch mit $u(z) \leq v(z)$ $(z \in \partial K)$. Dann ist $u(z) \leq v(z)$ $(z \in K)$

\textbf{Lemma 22.8:} Sei $u \colon \D \to \R$ subharmonisch und $m(r) := \frac{1}{2\pi} \int_0^{2\pi} u(re^{it}) \, dt$ $(r \in [0,1))$. Dann ist $m$ wachsend auf $[0,1)$.

\textbf{Satz:} Sei $p \in [1, \infty)$ fest, $f \in H(\D)$ und $M_p(f,r) := (\frac{1}{2\pi} \int_0^{2\pi} |f(re^{it})|^p \, dt)^{\frac{1}{p}}$ $(r \in [0,1))$. Dann ist $M_p(f, \cdot)$ wachsend auf $[0,1)$.

\textbf{Definition:} $H^p(\D) := \{ f \in H(\D)\colon M_p(f,\cdot) \text{ ist beschränkt auf } [0,1)\}$. Für $f \in H^p(\D)$ sei $||f||_p := \lim_{r \to 1-} M_p(f,r)$.

\textbf{Satz:} Es gilt: $H^p(\D)$ ist ein komplexer Vektorraum und $||\cdot||_p$ ist eine Norm auf $H^p(\D)$.
\vspace{-0.6cm}
\begin{itemize}
\item Die Räume $H^p(\D)$ $(p \in [1,\infty])$ heißen auch Hardy-Räume. \vspace{-0.2cm}
\item Es gilt: $(H^p(\D), ||\cdot||_p)$ ist vollständig $(p \in [1, \infty])$. \vspace{-0.2cm}
\item Weiter gilt: $A(\D) \subsetneqq H^\infty(\D) \subsetneqq H^p(\D) \subsetneqq H^s(\D)$ $(1 \leq s < p < \infty)$. \vspace{-0.2cm}
\item Beispiel: siehe Seite 155-156
\end{itemize}
\vspace{-0.6cm}

\end{document}
