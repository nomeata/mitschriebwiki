\documentclass{article}
\newcounter{chapter}
\setcounter{chapter}{26}
\usepackage{ana}

\title{Zwei Eindeutigkeitssätze}
\author{Joachim Breitner, Florian Mickler}
% Wer nennenswerte Änderungen macht, schreibt sich bei \author dazu


\begin{document}
\maketitle

Stets in diesem Paragraphen: $I=[a,b]\subseteq\MdR$, $x_0\in I$, $y_0\in\MdR$ und $f\in C(D,\MdR)$. Wir betrachten das Anfangswertproblem:
\[ \text{(A)} \quad 
\begin{cases}
y'=f(x,y) \\ y(x_0) = y_0
\end{cases}
\]

\begin{satz}[Satz von Nagumo]
Es gelte \[|f(x,y) - f(x,\tilde y)| \le \frac{|y-\tilde y|}{|x-x_0|} \ \forall (x,y),(x,\tilde y)\in D\text{ mit }x\ne x_0\,.\]
Dann hat (A) höchstens eine Lösung auf $I$.
\end{satz}

\begin{beweis}
Seien $y_1,y_2: I \to \MdR$ Lösungen von (A) auf $I$, $y:= y_1-y_2$. ($\folgt y(x_0)=0$)

$\lim_{x\to x_0} \frac{y(x)}{x-x_0} = \lim_{x\to x_0} \frac{y(x) - y(x_0)}{x-x_0} = y'(x_0) = y_1'(x_0) - y_2'(x_0) = f(x_0,y_1(x_0))  - f(x_0,y_2(x_0)) = 0$.

Definiere $h:i\to\MdR$ durch $h(x) := 
\begin{cases}
\frac{|y(x)|}{|x-x_0|}, & x\in I\setminus\{x_0\}\\
0, &x=x_0
\end{cases} \folgt h \in C(I,\MdR)
$. Voraussetzung $\folgt |f(t,y_1(t)) - f(t,y_2(t))| \le h(t) \ \forall t\in I$.
\begin{align*}
\forall x\in I: |y(x)| &=|y_1(x) - y_2(x)| \\
&\gleichnach{12.1} \left|\int_{x_0}^x (f(t,y_1(t)) - f(t,y_2(t))) dt \right| \\
&\le \left| \int_{x_0}^x |f(t,y_1(t)) - f(t,y_2(t))| dt \right| \\
&\le \left|\int_{x_0}^x h(t) dt\right|
\end{align*}
\textbf{Annahme:} $\exists x_1 \in I: y(x_1)\ne0$. Dann: $x_1\ne x_0$, etwa $x_0<x_1$; $h(x_1) >0$, $h(x_0) = 0$. $\exists \xi \in [x_0,x_1]: h(t) \le h(\xi) \ \forall t\in[x_0,x_1]$. Dann: $h(\xi) > 0 \folgt \xi \ne x_0$, also $x_0 < \xi$.
\begin{multline*}
\text{Dann: }h(\xi) = \frac{|y(\xi)|}{|\xi-x_0|} = \frac{|y(\xi)|}{\xi-x_0} 
\le \frac1{\xi-x_0} \left|\int_{x_0}^\xi h(t) dt \right|\\ = \frac{1}{\xi-x_0} \int_{x_0}^\xi h(t) dt < \frac1{\xi-x_0} \int_{x_0}^\xi h(\xi) dt = h(\xi)\text{, Widerspruch.}
\end{multline*}
\end{beweis}

\begin{satz}[Satz von Osgood]
Es sei $\phi:(0,\infty)\to\MdR$ stetig und $>0$ auf $(0,\infty)$, $t_0>1$ und das uneigentliche Integral $\int_0^{t_0}\frac{du}{\phi(u)}$ sei divergent.\\
Weiter gelte \[|f(x,y)-f(x,\tilde y)|\le\phi(|y-\tilde y|) \forall (x,y),(x,\tilde y) \in D\text{ mit } y\ne \tilde y.\] \\
Dann hat (A) auf I h"ochstens eine L"osung.
\end{satz}
\begin{bemerkung}
f gen"uge auf D einer LB bzgl. y mit der Lipschitz-Konstanten L: $\phi(u):=Lu$
\end{bemerkung}
\begin{beweis}
o.B.d.A: $x_0=a$. $\int_0^{t_0}\frac{du}{\phi(u)}$ div. \folgt  $\int_{\frac{1}{k}}^{t_0}\frac{du}{\phi(u)}\to\infty(k\to\infty)$.\\
Daher: o.B.d.A:   $\int_{\frac{1}{k}}^{t_0}\frac{du}{\phi(u)} > 2(b-a)\forall k\in\MdN$.\\

\textbf{(I):} Sei $k\in\MdN$. Definiere $g_k:[\frac{1}{k},\infty)\to\MdR$ durch $g_k(t):= \int_{\frac{1}{k}}^{t}\frac{du}{\phi(u)}$\\
Dann: $g_k \in C^1([\frac{1}{k},\infty)$, $g'_k=\frac{1}{\phi}>0$, $g_k$ ist streng wachsend, $g_k(\frac{1}{k})=0$, $g_k(t_0)>2(b-a)$\\
ZWS $\folgt [0,2(b-a)] \subseteq g_k([\frac{1}{k},\infty))$\\
F"ur $x \in I = [a,b]:2(x-a) \in [0,2(b-a)]$.\\ Definiere $\Psi_k:I\to\MdR$ durch $\Psi_k(x):=g_k^{-1}(2(x-a))$.\\
$\folgt (i): 2(x-a) = g_k(\Psi_k(x)) = \int_{\frac{1}{k}}^{\Psi_k(x)}\frac{du}{\phi(u)} \forall x \in I$.\\
$g_k$ streng wachsend $\folgt g_k^{-1}$ streng wachsend $\folgt \Psi_k$ streng wachsend.\\
$\Psi_k(a)=\Psi_k(x_0)=g_k^{-1}(0)=\frac{1}{k}$, $\Psi_k(x)>\frac{1}{k}\forall x \in (a,b]$.\\
$g_k$ ist stetig db $\folgt g_k^{-1}$ stetig db $\folgt \Psi_k$ stetig db.\\
Aus (i): $2 =g'_k(\Psi_k(x))\Psi'_k(x) = \frac{1}{\phi(\Psi_k(x))}\Psi'_k(x)\forall x \in I$\\
$\folgt$ (ii): $\Psi'_k=2\phi(\Psi_k(x))>0\forall x \in I$.\\

\textbf{(II):} \underline{Behauptung:} $\Psi_k(x)\to 0 (k\to \infty) \forall x \in I$.\\
\underline{Beweis:} Sei $x\in I$. \textbf{ Annahme: } $\Psi_k(x) \not\to 0 (k \to \infty)$. \\
Dann $\exists \epsilon_0 > 0$ und eine TF $(\Psi_{k_j}(x))$ von $(\Psi_k(x))$ mit:\\
$\epsilon_0 \ge 0 \Psi_{k_j}(x)\forall j \in \MdN$.\\
$c_j:=\int_{\frac{1}{k_j}}^{\epsilon_0}\frac{du}{\phi(u)} (j \in \MdN)$. Vorraussetzung $\folgt c_j \to \infty (j\to\infty)$.\\
Aber: $c_j=\int^{\ep_0}_{\frac{1}{k_j}}\frac{du}{\phi(u)} \le \int_{\frac{1}{k_j}}^{\Psi_{k_j}(x)}\frac{du}{\phi(u)} \gleichwegen{(1)} 2(x-a)\forall j\in\MdN$.\\
\textbf{Widerspruch zu $c_j \to \infty$!}\\

\textbf{(III):} Sei $y_1$, $y_2:I\to \MdR$ L"osungen von (A) auf $I$. $y:=y_1-y_2$. \\
Wir zeigen: $|y(x)|\le \Psi_k(x) \forall k \in \MdN \forall x \in I$. (Mit (II) folgt dann: $y == 0$ auf I.)\\
Sei $k \in \MdN$.\\
\textbf{Annahme:} $M:={x \in I:|y(x)| > \Psi_k(x)}\ne \emptyset$.\\
$y(a) = y(x_0)=y_1(x_0)-y_2(x_0)=0 \folgt a \not \in M. \xi := \inf M$\\
$y$,$\Psi_k$ stetig $\folgt |y(\xi)|\ge\Psi_k(\xi) \folgt \xi > a$ und $|y(x)| \le \Psi_k(x) \forall x \in [a,\xi)$ (iii)\\
$\folgtwegen{x\to\xi-} |y(\xi)|\le \Psi_k(\xi).$ Also: $|y(\xi)|=\Psi_k(\xi).$ D.h.: $+- y(\xi) = \Psi_k(\xi).$ o.B.d.A: $y(\xi)=\Psi_k(\xi)$. (sonst betrachte $y_2-y_1=-y$).\\
Aus (iii) folgt: $\exists \alpha > 0$ so, dass $\xi-\alpha \ge a$ und $0 < y \le \Psi_k$ auf $[\xi-a,\xi]$.\\
Sei $x\in(\xi-\alpha,\xi) \folgt y(x)\le \Psi_k(x) \folgt y(x)-y(\xi)\le \Psi_k(x)-\Psi_k(\xi)$\\
$\folgt \frac{y(x)-y(\xi)}{x-\xi}\ge \frac{\Psi_k(x)-\Psi_k(\xi)}{x-\xi} \folgtwegen{x\to\xi-} y'(\xi)\ge\Psi'_k(\xi) \folgt \Psi'_k(\xi)\le y'(\xi)=y'_1(\xi)-y'_2(\xi)$\\
$=f(\xi,y_1(\xi))-f(\xi,y_2(\xi)) \gleichwegen{(ii)} \frac{1}{2}\Psi'_k(\xi) \folgt \Psi'_k(\xi)\le 0$, Widersruch zu (ii)!.

\end{beweis}


\end{document}
