\chapter{Credits für Analysis III} Abgetippt haben die folgenden Paragraphen:\\% no data in Ana3Vorbereitung.tex
% no data in Ana3Vorwort.tex
\textbf{§ 1: Satz von Arzelà-Ascoli}: Joachim Breitner\\
\textbf{§ 2: Der Integralsatz von Gauss im $\MdR^2$}: Joachim Breitner, Florian Mickler\\
\textbf{§ 3: Flächen im $\MdR^3$}: Christian Schulz\\
\textbf{§ 4: Der Integralsatz von Stokes}: Bernhard Konrad\\
\textbf{§ 5: Der Integralsatz von Gauss im $\MdR^3$}: Bernhard Konrad\\
\textbf{§ 6: Differentialgleichungen: Grundbegriffe}: Pascal Maillard\\
\textbf{§ 7: Lineare Differentialgleichungen 1. Ordnung}: Pascal Maillard, Michael Knoll\\
\textbf{§ 8: Differentialgleichungen mit getrennten Veränderlichen}: Lars Volker, Wenzel Jakob\\
\textbf{§ 9: Einige Typen von Differentialgleichungen 1. Ordnung}: Wenzel Jakob\\
\textbf{§ 10: Exakte Differentialgleichungen}: Wenzel Jakob und Joachim Breitner\\
\textbf{§ 11: Hilfsmittel aus der Funktionalanalysis}: Joachim Breitner, Lars und Michael Volker - Knoll\\
\textbf{§ 12: Der Existenzsatz von Peano}: Christian Schulz, Ferdinand Szekeresch\\
\textbf{§ 13: Der Existenz- und Eindeutigkeitssatz von Picard - Lindelöf}: Ferdinand Szekeresch und Pascal Maillard\\
\textbf{§ 14: Matrizenwertige und vektorwertige Funktionen}: Pascal Maillard, Ferdinand Szekeresch und Christian Schulz\\
\textbf{§ 15: Existenz- und Eindeutigkeitssätze für Dgl.Systeme 1. Ordnung}: Christian Schulz\\
\textbf{§ 16: Lineare Systeme}: Wenzel Jakob, Bernhard Konrad\\
\textbf{§ 17: Lineare Systeme mit konstanten Koeffizienten}: Ferdinand Szekeresch und Joachim Breitner\\
\textbf{§ 18: Differentialgleichungen höherer Ordnung}: Jonathan Picht\\
\textbf{§ 19: Lineare Differentialgleichungen $m$-ter Ordnung}: Jonathan Picht und Ferdinand Szekeresch\\
\textbf{§ 20: Lineare Differentialgleichungen $m$-ter Ordnung mit konstanten Koeffizienten}: Ferdinand Szekeresch\\
\textbf{§ 22: Nicht fortsetzbare L"osungen}: Pascal Maillard\\
\textbf{§ 23: Minimal- und Maximallösung}: Christian Schulz\\
\textbf{§ 24: Ober- und Unterfunktionen}: Wenzel Jakob\\
\textbf{§ 25: Stetige Abhängigkeit}: Joachim Breitner\\
\textbf{§ 26: Zwei Eindeutigkeitssätze}: Joachim Breitner, Florian Mickler\\
\textbf{§ 27: Randwertprobleme (Einblick)}: Florian Mickler und Joachim Breitner\\
