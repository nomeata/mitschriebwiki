\chapter{Mathematisches} \label{mathe}

\section{Gruppen}

\subsection{Definition}

Eine Gruppe ist eine Menge $G$ mit einer Verkn�pfung $*$, sodass gilt:

\begin{itemize}
	\item die Verkn�pfung ist assoziativ
	\item es gibt ein Element $e_G$, sodass f�r alle $g \in G$ $g * e_G = e_G * g =g$ gilt (neutrales Element
	\item f�r alle $g \in G$ gibt es ein $g^{-1}$, sodass $g * g^{-1} = e_G$ gilt
\end{itemize}

Die Ordnung einer Gruppe ist die Anzahl ihrer Elemente.

\subsection{Satz von Lagrange}

Sei $G$ eine endliche Gruppe, $U \leq G$ eine Untergruppe von G. Dann teilt die Ordnung von $U$ die Ordnung von $G$.\\

Insbesondere erzeugt jedes Element $g \in G$ eine Untergruppe von $G$. Die Ordnung dieser Untergruppe teilt dann nat�rlich wieder die Ordnung von $G$.

\subsection{Kleiner Satz von Fermat}

\subsection{Eulersche $\varphi$-Funktion}

Die Eulersche $\varphi$-Funktion einer nat�rlichen Zahl $N$ ist definiert als die Anzahl der nat�rlichen Zahlen kleiner $N$, die zu $N$ teilerfremd sind: $\varphi(N) = \left|(\mathbb{Z}/N\mathbb{Z})^\times \right| = \left| \{ x \mid x \in \mathbb{N}, x < N, x \not| N \} \right|$

\section{Euklidischer Algorithmus}

\section{Chinesischer Restsatz (f�r $\mathbb{Z}$)}