\documentclass[a4paper,10pt,german]{scrbook}

\usepackage{latexki}
\lecturer{Prof. Dr. F. Herrlich}
\semester{Wintersemester 07/08}
\scriptstate{partial}

\usepackage[OT1,T1]{fontenc}
\usepackage{babel}
\usepackage{cmbright}
\usepackage{url}
\usepackage{makeidx}
\usepackage{framed}
\usepackage[pdftex]{xcolor}

\usepackage[utf8]{inputenc}

\usepackage{stmaryrd}
%\usepackage{ulsy}
%blitztest
\newcommand{\blitza}[0]{\lightning}
\newcommand{\blitzb}[0]{\lightning}
\newcommand{\blitzc}[0]{\lightning}
\newcommand{\blitzd}[0]{\lightning}

%erstmal auskommentiert, leider kann das amscd-Paket nicht so viel :/
%\usepackage{pictexwd,dcpic}

\usepackage{colortbl}

\usepackage{hyperref}

% Mathe-Pakete
\usepackage{amssymb}
\usepackage{amsmath}
\usepackage{amsfonts}
\usepackage{amsthm}
\usepackage{amscd}
\usepackage{mathtools}
\usepackage{faktor}

\usepackage{graphicx}

%\usepackage{algebra}
\newcommand{\da}{\coloneqq}
\newcommand{\ad}{\eqqcolon}


\newtheoremstyle{saetze}% Name des Stils
{3pt}% vertikaler Abstand zum vorangehenden Text
{3pt}% vertikaler Abstand zum folgenden Text
{}% Schriftart des Textkörpers
{}% Abstand des Erstzeileneinzugs der Kopfzeile
{\Large \bfseries}% Schriftart des Kopfes
{}% Punktierung nach dem Kopf
{\newline}% Abstand nach dem Kopf (z.B. \newline)
{}% Kopfspezifikation (leer bedeutet 'normal')

\newtheoremstyle{definitionen}% Name des Stils
{3pt}% vertikaler Abstand zum vorangehenden Text
{3pt}% vertikaler Abstand zum folgenden Text
{}% Schriftart des Textkörpers
{}% Abstand des Erstzeileneinzugs der Kopfzeile
{\large \bfseries}% Schriftart des Kopfes
{}% Punktierung nach dem Kopf
{\newline}% Abstand nach dem Kopf (z.B. \newline)
{}% Kopfspezifikation (leer bedeutet 'normal')

\theoremstyle{saetze}
    \newtheorem{Satz}{Satz}
    
\theoremstyle{definitionen}
    \newtheorem{Def}{Definition}[section]
    \newtheorem{DefBem}[Def]{Definition + Bemerkung}
    \newtheorem{Bem}[Def]{Bemerkung}
    \newtheorem{BemDef}[Def]{Bemerkung + Definition}
    \newtheorem{Prop}[Def]{Proposition}
    \newtheorem{PropDef}[Def]{Proposition + Definition}
    \newtheorem{Folg}[Def]{Folgerung}
    \newtheorem{Bsp}[Def]{Beispiel}
    \newtheorem{Bspe}[Def]{Beispiele}
    \newtheorem{DefProp}[Def]{Definition + Proposition}

\title{Algebra I}
\author{Prof. Dr. F. Herrlich}
\publishers{Die Mitarbeiter von \url{http://mitschriebwiki.nomeata.de/}}
\makeindex

\begin{document}


\setkomafont{sectioning}{\normalfont\normalcolor\bfseries}
\setkomafont{descriptionlabel}{\normalfont\normalcolor\bfseries}

%\renewcommand*{\othersectionlevelsformat}[1]{\llap{\csname the#1 \endcsname\autodot\enskip}}

%\newcommand{\ssubsection}[1]{\subsection{#1 \thesubsection} \label{\thesubsection}}

\newcommand{\emp}[1]{\textbf{\emph{#1\index{#1}}}}
\newcommand{\empind}[2]{\textbf{\emph{#1\index{#2}}}}

\newenvironment{define}
    { \begin{flushleft} \begin{description} }
    { \end{description} \end{flushleft} }

\newenvironment{enum} {
      \begin{enumerate}
      \renewcommand{\labelenumi}{(\alph{enumi})}

      }
      { \end{enumerate} }

\newcommand{\bla}[0]
{\begin{tiny} $\left\{
\begin{array}{l}
             . \\
             . \\
             . \end{array}
        \right\}$
\end{tiny}}

\newcommand{\blab}[0]
{\begin{tiny} $\left\{
\begin{array}{l}
             Halbgruppen \\
             Monoiden \\
             Gruppen \end{array}
        \right\}$
\end{tiny}}


\newenvironment{ourshaded}{%
  \def\FrameCommand{\colorbox{shadecolor}}%
  \edef\frmlw{\linewidth}%
  \MakeFramed{\setlength\hsize{\linewidth}\FrameRestore}%
  \setlength{\parskip}{1em}\setlength{\parindent}{0pt}}{\endMakeFramed}

\newcommand{\bew}[2]
{\definecolor{shadecolor}{rgb}{0.85,0.85,1}
 \begin{shaded}
 \textit{\textbf{Beweis: }}
 #1
 \begin{enum} #2
 \hfill \rule{2,1mm}{2,1mm} \end{enum}
 \end{shaded}
}

\newcommand{\sbew}[1]
{\definecolor{shadecolor}{rgb}{0.85,0.85,1}%
 \begin{ourshaded}%
 \textit{\textbf{Beweis:}} #1%
 \hfill \rule{2,1mm}{2,1mm}%
 \end{ourshaded}
}

\newcommand{\bsp}[1]
{\definecolor{shadecolor}{rgb}{0.85,1,0.85}%
 \begin{ourshaded}%
 \textit{\textbf{Beispiel:}} #1%
 \end{ourshaded} 
}

\newcommand{\ra}[0]{\rightarrow}
\newcommand{\ds}[0]{\displaystyle}
\newcommand{\cd}[0]{\cdot}
\newcommand{\lra}[0]{\Leftrightarrow}
\newcommand{\Ra}[0]{\Rightarrow}

\newcommand{\para}[1]{\noindent \Large\textbf{#1} \normalsize}

\newcommand{\defeqr}[0]{\mathrel{\mathop:}=}
\newcommand{\defeql}[0]{=\mathrel{\mathop:}}

\newcommand{\chk}[0]{\checkmark}
\newcommand{\wt}[0]{\widetilde}

\newcommand{\Kern}{\mathop{\rm Kern}\nolimits}
\newcommand{\Bild}{\mathop{\rm Bild}\nolimits}
\newcommand{\ord}{\mathop{\rm ord}\nolimits}

\maketitle

\addcontentsline{toc}{chapter}{Inhaltsverzeichnis}
\tableofcontents


\chapter{Gruppen}

\section{Grundlegende Definitionen}

\begin{Def} 
    Sei $M$ eine Menge.

    \begin{enum}
        \item Eine \emp{Verknüpfung} auf $M$ ist eine Abbildung $\cd: M
              \times M \ra M $

        \item Eine Menge $M$ zusammen mit einer Verknüpfung $\cd$ heißt
              \emp{Magma}.

        \item Eine Verknüpfung $\cd : M \times M \ra M$ heißt \emp{assoziativ},
        wenn \[\forall x,y,z \in M: (x\cd y)\cd z=x\cd(y\cd z)\]

        \item Eine \emp{Halbgruppe} ist ein assoziatives Magma.
    
        \item $e \in M$ heißt \emp{neutrales Element} für die Verknüpfung $\cd$,
        wenn \[\forall x \in M: x \cd e = e \cd x = x\]
          
        \item Eine Halbgruppe mit neutralem Element heißt \emp{Monoid}.

        \item Eine \emp{Gruppe} ist ein Monoid $(G,\cd)$, in dem es zu jedem $x
        \in G$ ein $x' \in G$ gibt mit \[ x \cd x' = x' \cd x = e \] $x'$ heißt
        dann \emp{zu $x$ inverses Element.}
    \end{enum}
\end{Def}

\begin{Bem}
    Sei $(M,\cd)$ ein Magma.
    
    \begin{enum}
        \item In $M$ gibt es höchstens ein neutrales Element.
        \sbew{Sind $e, e'$ neutrale Elemente, so ist $e=e\cd e' = e'$}

        \item Ist $M$ Monoid, so gibt es zu $x \in M$ höchstens ein inverses 
        Element.
        \sbew{Seien $x',x''$ zu $x$ invers, so ist $x'=(x'' \cd x)\cd x' =
        x'' \cd (x \cd x')=x''$}
    \end{enum}
\end{Bem}

\begin{DefBem} 
    Sei $(M,\cd)$ ein(e)
    \begin{small} $\left\{ \begin{array}{l}
            Magma \\
            Halbgruppe \\
            Monoid \\
            Gruppe \end{array}
    \right\}$
    \end{small}

    \begin{enum}
        \item $U \subseteq M$ heißt Unter-\bla, wenn $U \cd U \subseteq U$ und
        $(U,\cd)$ selbst ein(e) \bla ist.
        
        \item $U \subseteq M$ Unterhalbgruppe $\Leftrightarrow U \cd U \subseteq
        U$
        
        \item $U \subseteq M$ Untermonoid $\lra U \cd U \subseteq U$ und $e \in
        U$
        
        \item $U \subseteq M$ Untergruppe $\lra U \neq \emptyset$ und $\forall
        x,y \in U: x \cd y^{-1} \in U$
	\sbew{''$\Leftarrow$'': \newline Sei $x \in U \> \Rightarrow e=x \cd x^{-1}
        \in U \Rightarrow $ mit $x$ ist auch $x^{-1}$ in $U \Rightarrow$ mit
        $x,y$ ist auch $xy = x(y^{-1})^{-1} \in U$}
    \end{enum}
\end{DefBem}

\begin{Bem} 
    Sei $ (M,\cd)$ Monoid. Dann ist $M^x \defeqr \{x \in M:$ es gibt inverses
    $x^{-1}$ zu $x \in M\}$ eine Gruppe.
    \sbew{ $\\e \in M^x$, da $e\cd e = e$, also $M^x \neq \emptyset$.
    Sind $x,y \in M^x$, so ist $x \cd y \in M^x$, da $xy \cd
    (y^{-1}x^{-1})=e \Rightarrow \cd$ ist Verknüpfung auf $M^x \Rightarrow
    (M^x, \cd)$ ist Gruppe.}
\end{Bem}

\begin{DefBem}
    Seien $(M, \cd), (M',*)$ \bla
    \begin{enum}
        \item Eine Abbildung $f:M\ra M'$ heißt \emp{Homomorphismus}, wenn
        $\forall \; x,y \in M:\\\\$ \bigskip $ f(x\cd y) = f(x)*f(y)$ \hfill
        \textmd{(i)} \newline Hat M ein neutrales Element, so muß außerdem
        gelten: $\\\\f(e) = e'$ \hfill \textmd{(ii)}
        
        \item Ist $f:G \ra G'$ Abbildung von Gruppen, die \textmd{(i)} erfüllt,
        so ist $f$ Homomorphismus.
	\sbew{ $f(e) = f(e \cd e) =
        f(e) * f(e) \overset{\cd f(e)^{-1}}{\Rightarrow} e' = f(e)$} 
        
        \item Ein Homomorphismus $f:M \ra M'$ heißt \emp{Isomorphismus}, wenn es
        einen Homomorphismus $g:M'\ra M$ gibt, mit $f \circ g = id_{M'}$ und $g
        \circ f = id_M$

        \item Jeder bijektive Homomorphismus ist Isomorphismus. \newline
        \sbew{Sei $f:M\ra M'$ bijektiver Homomorphismus und $g:M' \ra M$
        die Umkehrabbildung. z.z.: $g$ ist Homomorphismus. \newline Seien $x,y
        \in M'$. Schreibe $x=f(\hat{x}), y=f(\hat{y})$ für passende $\hat{x},\hat{y} \in M 
        \Rightarrow g(x\cd y) = g(f(\hat{x}) \cd f(\hat{y})) = g(f(\hat{x} \cd \hat{y})) = 
        \hat{x} \cd \hat{y} = g(f(\hat{x})) \cd g(f(\hat{y})) = g(x) \cd g(y) $}

        \item Die Komposition von Homomorphismen ist wieder ein Homomorphismus.
    \end{enum}
\end{DefBem}

\begin{Def} 
    Sei $f:M \ra M'$ Hom von \bla.
    
    \begin{enum}
        \item $\Bild(f) \defeqr \{f(x):x \in M\} \subseteq M'$ ist ein
        Unter-\bla. 
        \sbew{Sind $x,x' \in M$, so ist $f(x)*f(x')=f(x \cd x') \in 
        \Bild(f)$. Sind $M,M'$ Monoide, so gilt: $f(e) = e' \in$ $\Bild(f)$.
        Sind $M, M'$ Gruppen, so gilt: $f(x)^{-1} = f(x^{-1}) \in$ $\Bild(f)$, da
        $f(x\cd x^{-1}) = f(e) = e' = f(x) * f(x^{-1})$}
        
        \item Sind $M, M'$ Monoide/Gruppen, so ist $\Kern(f) \defeqr \{x \in M :
        f(x) = e'\}$ Untermonoid/-gruppe von $M$. 
        \sbew{$x,y \in$ $\Kern(f) \Rightarrow f(xy) = f(x) * f(y) = e'*e' =
        e' \Rightarrow xy \in$ $\Kern(f), e \in$ $\Kern(f) \;\chk$ \\
        $x\in$ $\Kern(f) \Rightarrow f(x^{-1}) = f(x)^{-1} = (e')^{-1} = e'
        \Rightarrow x^{-1} \in$ $\Kern(f)$}

        \item Sind $G,G'$ Gruppen, so ist $f$ genau dann injektiv, wenn $\Kern(f)
        = \{e\}$
    \end{enum}
\end{Def}

\section{Beispiele und Konstruktionen}

\begin{enum}
    \item[(1)] Sei $M$ eine Menge. $\\M^M \defeqr \{ f:M \ra M$ Abbildung $\}$
    ist mit der Verknüpfung $\cd$ ein Monoid. $(M^M)^X = \{f: M \ra M$ bijektiv
    $\} \defeql$ Perm$(M) = S_M$. \newline
    insbesondere: $M=\{1,\dots ,n\}: S_{ \{1, \dots, n\} } = S_n$ Ist $(M,\cd)$
    ein \bla, so ist $End(M) \defeqr \{f \in M^M :f\;$ Hom.$\}$ ein Untermonoid
    von $M^M$ und \newline
    $Aut(M) \defeqr$ Perm$(M) \cap$ End$(M)$ Untergruppe von Perm$(M)$

    \item[(2a)] Sei $X$ Menge, $M$ ein(e) \bla. Dann ist $M^X = \{f:X \ra M$
    Abbildung $\}$ mit der Verknüpfung $(f\cd g)(x) = f(x)\cd g(x)$ ein(e) \bla
    
    \item[(2b)] Ist $(M,\cd)$ Halbgruppe, $(H,+)$ kommutative Halbgruppe, so
    ist Hom$(M,H) \defeqr \{f \in H^M: f\;$ Homomorphismus$\}$ eine kommutative
    Unterhalbgruppe von $H^M$. \newline\textbf{denn}: Sind $f,g: M \ra H$
    Homomorphismen, so ist $\forall x,y \in M$: \newline
    $(f+g)(x\cd y) = f(x \cd y) + g(x \cd y) = f(x) + f(y) + g(x) + g(y) = f(x)
    + g(x) + f(y) + g(y) = (f + g)(x) + (f + g)(y)$

    \item[(3)] Sei $I$ eine Indexmenge. Für jedes $i \in I$ sei $(M_i, \cd)$
    ein(e) \bla.
    \begin{enum}
        \item $\displaystyle \prod_{i \in I} M_i$ ist mit komponentenweiser
        Verknüpfung ein(e) \bla.
        
        \item Sind $M_i$ Monoide, so ist \[ \bigoplus_{i \in I} M_i \defeqr \{ 
        (x_i)_{i \in I} \in \prod_{i \in I} M_i, x_i = e_i \mbox{ ffa.} i\}\]
        ein Monoid.
    \end{enum}
\end{enum}

\begin{DefBem}
\mbox{}
\begin{enum}
\item $\prod$ heißt \emp{direktes Produkt} \newline
$\bigoplus$ heißt \emp{direkte Summe}
\item Ist $I$ endlich, so ist $\prod M_i \cong \bigoplus M_i$
\item Sei $M$ ein(e) \bla und für jedes $i \in I: g_i:M \ra
M_i$ ein Homomorphismus. Dann gibt es genau einen Homomorphismus
$\displaystyle G:M \ra \prod_{i \in I} M_i$, so dass $g_i = pr_i
\circ G$, wobei $\displaystyle pr_i: \prod_{j \in I} M_j \ra M_i$
Projektion.

%\[\begindc{\commdiag} \obj(1,1){$M$}
%                      \obj(5,1){$M_i$}
%                      \obj(3,3){$\pi M_j$}
%                      \mor{$M$}{$M_i$}{$g_i$}
%                      \mor{$M$}{$\pi M_j$}{$\exists!\;G$}[1,1]
%                      \mor{$\pi M_j$}{$M_i$}{pr$_i$}
%\enddc\]

\sbew{Setze $G(m) \defeqr (m_j)_{j \in I}$ mit $m_j =
g_j(m)$ für $m \in M$. $G$ ist Homomorphismus. $\chk\;\\$ $G$ ist
eindeutig, da $pr_i(G(m)) = g_i(m)$ sein muss.}
\item Ist $(M,+)$ ein kommutatives Monoid, und für jedes $i \in I\; f_i:M_i
\ra M$ ein Homomorphismus, so gibt es genau einen Homomorphismus \[
F: \bigoplus_{j\in I} M_j \ra M \mbox{, so dass für jedes } i \in I: f_i = F
\circ \nu_i \mbox{, wobei } \nu_i:M_i \ra \bigoplus_{j\in I} M_j\] \[m \mapsto
(m_j)_{j \in I}\mbox{, wobei } m_j = \left\{
\begin{array}{rl}
            m & i=j \\
            e_j & \mbox{sonst}
          \end{array}\right.\]

%\[\begindc{\commdiag} \obj(1,3){$M_i$}
%                      \obj(3,3){$M$}
%                      \obj(2,1){$\oplus_{j\in I} M_j$}
%                      \mor{$M_i$}{$M$}{$f_i$}
%                      \mor{$M_i$}{$\oplus_{j\in I} M_j$}{$\nu_i$}
%                      \mor{$M$}{$\oplus_{j\in I} M_j$}{$\exists!\;F$}[1,1]
%\enddc\]

\sbew{Setze $F((m_j)_{j \in I}) = \displaystyle \sum_{j \in I} f_j(m_j)$
\newline Brauche: $F((e,\dots, e,m_i,e,\dots,e)) = F(\nu_i(m_i)) \overset{!}{=}
f_i(m_i)$
\newline $\Rightarrow F((e,\dots,e,m_i,e,\dots,e,m_j,e,\dots,e)) = f_i(m_i) +
f_j(m_j) = F((e,\dots,e,m_i,e,\dots,e)) + F((e,\dots,e,m_j,e,\dots,e))$}

\end{enum}
\begin{enum}
\item[(4)] Sei $S$ eine Menge (''Alphabet'') $F^a(S) \defeqr
\displaystyle \bigcup^{\infty}_{n=1} S^n$ ist Halbgruppe mit Verknüpfung
''Nebeneinanderschreiben'' $\underset{\in S^n}{(x_1,\dots,x_n)}\cd
\underset{\in S^m}{(y_1,\dots,y_m)} \defeqr \underset{\in
S^{n+m}}{(x_1, \dots, x_n, y_1, \dots, y_m)}$ $F^a(S)$ heißt freie Halbgruppe oder ''Worthalbgruppe'' über
$S$.

Definiert man $S^0 \defeqr \{\varepsilon\}$, dann ist $F_0^a(S) \defeqr \bigcup_{n=0}^\infty S^n$ ein Monoid mit neutralem Element $\varepsilon$, dem „leeren Wort“. Für $S=\{1\}$ ist $F_0^a(S) = (\mathbb N_0,+)$.
\end{enum}
\end{DefBem}

\begin{Bem} 
Ist $(H,\cd)$ Halbgruppe, $f:S\ra H$ eine
Abbildung, so gibt es genau einen Homomorphismus $\varphi:F^a(S) \ra H$
mit $\varphi(s) = f(s)$ für alle $s\in S$, wobei man $S$ als $S^1\subset F^a(S)$ auffasst.

\sbew{Für $(x_1,\ldots,x_n)\in S^n$ muss gelten: $\varphi(x_1,\ldots,x_n) = \varphi(x_1)\cdot\cdots\cdot\varphi(x_n) = f(x_1)\cdot\cdots\cdot f(x_n)$. Also ist $\varphi$ eindeutig und existiert, da es so definiert werden kann.
}
\end{Bem}

\begin{BemDef}
\label{1.9}
Sei $(M,\cd)$ ein Monoid und $(G,\cd)$ eine Gruppe
\begin{enum} 

\item Für $x \in M$ ist $\varphi_x: \mathbb{N}_0
\to M$, $n \mapsto x^n$ ein Homomorphismus.

\item Für $g \in G$, so ist $\varphi_g: \mathbb{Z} \to G$, $n \mapsto g^n$
ein Gruppenhomomorphismus.

\item $\langle g \rangle \defeqr \Bild(\varphi_g)$ heißt die von
$g$ erzeugte \emp{zyklische Untergruppe} von $G$.

\item $G$ heißt zyklisch, wenn es ein $g\in G$ gibt mit $\langle g\rangle =G$.

\item $| \langle g \rangle |\in \mathbb N\cup\{\infty\}$ heißt \emp{Ordnung} von $g$

\item Ist $G$ endlich, so heißt $| G |$ die \emp{Ordnung} von $G$.
\end{enum}


%\begin{enum}
%\item[(6)] Sei $G$ Gruppe, für $g \in G$ sei $\tau_g:G \ra G$, $x\mapsto g \cd 
%x$ (''Linksmultiplikation'')\newline
%$\tau_g(e)=g \Rightarrow$ kein Gruppenhomomorphismus, $\tau_e = id_G$
%\end{enum}
\end{BemDef}

\begin{DefBem}[Satz von Cayley]
\mbox{}
\begin{enum}
\item Für $g\in G$ heißt die Abbildung $\tau_g:G\to G$, $h \mapsto gh$ die \emph{Linksmultiplikation} mit $g$.

\item Für jedes $g\in G$ ist $\tau_g$ bijektiv, da $\tau_{g^{-1}}$ die Umkehrabbildung ist.

\item Die Abbildung:
\[ \begin{array}{lcc}
    \tau    &   :   &   G \ra \mbox{Perm}(G) \\
            &       &   g \mapsto \tau_g
    \end{array} \]
ist ein injektiver Gruppenhomomorphismus. 
\bew{}{ \item[(1)]
$\tau_g \in$ Perm$(G) : \tau_g$ ist bijektiv mit Umkehrabbildung
$\tau_{g^{-1}}$
\item[(2)] $\tau$ ist Homomorphismus: $\tau(g_1 g_2) = \tau(g_1)
\circ \tau(g_2)$, denn: $\forall x \in G: \tau(g_1 \circ g_2)(x) = (g_1 g_2)x =
g_1(g_2 x) = \tau_{g_1}(\tau_{g_2}(x)) = (\tau_{g_1}
\circ \tau_{g_2})(x)$ \item[(3)] $\Kern(\tau)$ = $\{e\}$, denn ist
$\tau(g) = id_g$, so ist $\forall x \in G: \tau_g(x) = gx = x$, also $g = e$}
\end{enum}
\end{DefBem}

\begin{DefBem}
\label{kernundnormalteiler}
Sei $G$ Gruppe, $g \in G$
\begin{enum}
\item Die Abbildung $c_g:G \ra G, x \mapsto gxg^{-1}$ ist ein
\emp{Automorphismus}, sie heißt \emp{Konjugation} mit $g$.

\sbew{$c_g$ ist Homomorphismus: $c_g(x_1 x_2) = g(x_1
x_2)g^{-1} \\ c_g(x_1) c_g(x_2) = (g x_1 g^{-1})(g x_2 g^{-1}) = c_g(x_1)\cdot c_g(x_2) \\
c_g$ ist bijektiv: Die Umkehrabbildung ist $c_{g^{-1}}$ }

\item Die Abbildung $c:G \ra$ Aut$(G), g \mapsto c_g$ ist ein
Gruppenhomomorphismus.
\sbew{$\forall x \in G: c(g_1 g_2)(x) = (g_1
g_2)x(g_1 g_2)^{-1} = g_1(g_2 x g_2^{-1})g_1^{-1} = (c(g_1) \circ
c(g_2))(x)$ }

\item Die Elemente von $\Bild(c) \defeql$ Aut$_i(G)$ heißen \emp{innere
Automorphismen} von $G$.

\item $Z(G)\defeqr$ $\Kern(c)$ heißt \emp{Zentrum} von $G$. Es ist $Z(G) =
\{ g \in G:\forall x \in G:gx=xg \}$

\item Eine Untergruppe $N \subseteq G$ heißt \emp{Normalteiler} in
$G$, wenn $\forall g \in G: c_g(N) \subseteq N$.
Äquivalent: $\forall g \in G, x \in N: g x g^{-1} \in N$

\item Ist $f:G \ra G'$ Gruppenhomomorphismus, so ist $\Kern(f)$
Normalteiler in G.

\sbew{Sei $x \in$
$\Kern(f), g \in G$. Dann ist $f(g x g^{-1}) = f(g)
\underset{e'}{\underbrace{f(x)}} f(g)^{-1} = e'$.}

\item Aut$_i(G)$ ist Normalteiler in Aut$(G)$

\sbew{Sei $\varphi \in \mbox{Aut}(G), g\in G :
\mbox{z.z.: } \varphi \cd c_g \cd \varphi^{-1} \in \mbox{Aut}_i(g).
\\ \mbox{Es ist } (\varphi \cd c_g \cd \varphi^{-1})(x) =
\varphi(c_g(\varphi^{-1}(x))) = \varphi(g \cd \varphi^{-1}(x)\cd
g^{-1}) = \varphi(g) \cd \varphi(\varphi^{-1}(x)) \cd
\varphi(g^{-1}) = \varphi(g) \cd x \cd \varphi(g)^{-1} =
c_{\varphi(g)}(x) \Rightarrow \varphi \circ c_g \circ \varphi^{-1} =
c_{\varphi(g)} \in \mbox{Aut}_i(G)$}
\end{enum}
\end{DefBem}

\begin{DefBem}
\label{1.12}
Sei $G$ Gruppe, $H \subseteq
G$ Untergruppe. \begin{enum}
\item Für $g \in G$ heißt $g \cd H = \{g\cd h : h \in H\} =
\tau_g(H)$ \emp{Linksnebenklasse} von $G$ bzgl. $H$ und $H \cd g=
\{h \cd g : h \in H \}$ \emp{Rechtsnebenklasse}


\item Für $g_1$, $g_2 \in G$ gilt: $g_1 H \cap g_2 H \neq
\emptyset \lra g_1 H = g_2 H$

\sbew{Sei $ y = g_1
h_1 = g_2 h_2 \in g_1 H \cap g_2 H$ und $h1, h2, h \in H \Rightarrow g_1 = g_2
h_2 h_1^{-1} \Rightarrow g_1 h = g_2h_2h_1^{-1} \in g_2 H \Rightarrow g_1 H \subseteq g_2 H$, die Umkehrung
folgt analog.}
\item $H$ ist genau dann Normalteiler, wenn $\forall g\in G: g\cd H = H \cd g
$ 
\sbew{$gH = Hg \lra H =
gHg^{-1}$}

\item Alle Nebenklassen von $G$ bzgl. $H$ sind gleichmächtig.

\sbew{$\tau_g: \underset{e\cd H}{\underbrace{H}} \ra g\cd H, h
\mapsto g\cd h$ ist bijektiv.}
\item Die Anzahl der Linksnebenklassen bzgl. $H$ ist gleich der
Anzahl der Rechtsnebenklassen. Sie heißt \emp{Index} $[G:H]$ von $H$
in $G$. 
\sbew{Die Zuordnung \[\begin{array}{lcl}
    \{\mbox{Linksnebenklasse}\} & \ra & \{\mbox{Rechtsnebenklasse}\} \\
    g\cd H                      & \mapsto & H \cd g^{-1}
    \end{array}\]
ist \emp{wohldefiniert} und bijektiv.

\textbf{Wohldefiniertheit:} ist $g_1 H = g_2 H$, also $g_2 = g_1 h$
für ein $h \in H \Rightarrow Hg_2^{-1} = H(g_1 h)^{-1} = H\cd
h^{-1}g_1^{-1} = Hg_1^{-1}$}


\item
\emp{Satz von Lagrange:} Ist $G$ endlich, so ist
\[[G:H] = \frac{|G|}{|H|}\] \sbew{$G$ ist disjunkte Vereinigung
der $[G:H]$ Linksnebenklassen bzgl. $H$. Diese haben alle $|H|$
Elemente.} \label{\thesubsection \theenumiii}
\end{enum}
\end{DefBem}

\section{Quotientenbildung}

\begin{DefBem}
    \label{restklassendefinition}
    Sei $f: M \ra M'$ eine Abbildung von Mengen.
    
    \begin{enum}
        \item Die Relation $\sim_f$ auf $M: x \sim_f y \lra f(x) = f(y)$ ist
        eine Äquivalenzrelation.

        \item Für $x \in M$ sei $\bar x \defeqr [x]_f \defeqr \{ y \in M : y\sim_f x\} = \{ y\in M: f(y)= f(x)\}$.
	Es ist $\bar x = f^{-1}(f(x))$
	
	Weiter sei $\bar M \defeqr M/\sim_f \defeqr \{ \bar x : x \in M\}$

	\item $\bar f : \bar M\to \Bild(f)$, $\bar x\mapsto f(x)$ ist eine bijektive Abbildung.

    \end{enum}
\end{DefBem}

\begin{Def}
        Ist $(M,\cd)$ und $(M',\ast)$ ein \bla, und $(M,\cd) \ra (M', \ast)$ ein Homomorphismus, so wird durch
        $\bar x \cd \bar y \defeqr \overline{x \cd y}$ eine Verknüpfung auf $\bar M$ 
        definiert. So wird $(\bar M,\cd)$ auch zu einem \bla.
        \sbew{z.z.: $\cd$ ist wohldefiniert. 
        Seien also $x' \in \bar x, y' \in \bar y$ zu zeigen: $\overline{x'\cdot y'} =
        \overline{x\cdot y}$ dh. $f(x'\cdot y') = f(x\cdot y)$ dh. $f(x')=f(x), f(y') = f(y)$ Es ist
        $f(x'\cdot y') = f(x') \ast f(y') = f(x) \ast f(y) = f(x\cdot y)$}
\end{Def}

\begin{DefBem}
\label{1.14}
    \label{gruppenfaktorgruppe}
    Sei $f:G \ra G'$ Gruppenhomomorphismus.
    
    \begin{enum}
        \item $\bar G = G/\sim_f$ ist
        die Menge der Linksnebenklassen bzgl. $\Kern(f)$ also ist für jedes $g\in G$: $[g]_f = g\cdot \Kern{}(f) = \Kern(f)\cdot g$.

        \item  $\bar G = G/\Kern(f)$ heißt \emp{Faktorgruppe} von $G$ bzgl.
        $\Kern(f)$. 
        \sbew{Seien $x,y \in G$. Dann gilt: $\bar x = \bar y \lra
        f(x) = f(y) \lra f(x) \cdot f(y^{-1}) = e' \lra xy^{-1} \in \Kern{}(f)
        \lra y=(xy^{-1})^{-1} x \in \Kern{}(f) \cd  x \lra x^{-1}y \in
        \Kern{}(f) \lra y = x(x^{-1}y) \in x \cd \Kern{}(f) \lra y \cd
        \Kern{}(f) = x \cd \Kern{}(f)$}
    \end{enum}
\end{DefBem}

\textbf{Beispiel:} $\exp:(\mathbb R, +) \to (\mathbb C^\times, \cdot)$, $t\mapsto e^{2\pi i t}$ ist ein Gruppenhomomorphismus. Es ist $\exp(t_1) = \exp(t_2) \iff 1 = e^{2\pi i(t_2-t_1)} \iff t_2 - t_1 \in \mathbb Z$, also ist $\Kern(\exp) = \mathbb Z$.

Die Abbildung $[0,1)\to\mathbb R/\mathbb Z$, $t \mapsto [t]_f$ ist bijektiv, spiegelt aber die Eigenschaften dieser Gruppe nicht wieder. Besser geeignet ist die Bijektion $\mathbb R/\mathbb Z$, $\bar t \mapsto e^{2\pi i t}$.

\begin{Bem}
    Sei $G$ Gruppe. Es ist $N \subseteq G$ Normalteiler, genau dann, wenn es eine Gruppe $G'$ mit einem surjektivem Gruppenhomomorphismus $f:G \ra G'$ und $N
    =\Kern(f)$ gibt. 
    
    \bew{Die Richtung $\impliedby$ folgt aus \ref{kernundnormalteiler} f). 
    Sei $G' \defeqr \{x \cd N, x\in G\}$ $(\subseteq \mathcal{P}(G))$ Für
    $x,y \in G$ setze $(x \cd N)(y \cd N) = (xy \cd N)$ \newline
    \textit{Behauptung:} $({G}', \cd)$ ist Gruppe, \textit{\textbf{denn:}}}{ 
    
    \item[(i)] Die Verknüpfung ist wohldefiniert: Seien $x, x', y, y' \in G$ mit
    $x \cd N = x' \cd N,\; y \cd N = y' \cd N$. Dann gibt es $n,m \in N
    \mbox{ mit } x'=xn,
    y'=ym \Ra x',y' = x(ny)m$. Da $N$ Normalteiler ist, gibt es $n' \in N$ mit
    $ny=yn' \Ra x'y' = xyn'm \Ra x'y' \cd N = xy \cd N$
    
    \item[(ii)] alle übrigen Eigenschaften ''vererben'' sich von $G \mbox{ auf }
    G'\\$ $f: G \ra G',\; x \mapsto x \cd N$ ist surjektiver 
    Gruppenhomomorphismus mit $\Kern(f)$ = N }
\end{Bem}
    
\begin{DefBem}
    Sei $G$ Gruppe, $N\subset G$ Normalteiler. Die Gruppe $G'$ aus dem vorherigen Beweis heißt Faktorgruppe von $G$ nach $N$, und wir schrieben $G' = G/N$ („$G$ modulo $N$“). Sie ist gleich der Faktorgruppe $G/\Kern(f)$ für das $f$ aus der  vorherigen Bemerkung (ii).
\end{DefBem}

\begin{Satz}
\label{Satz 1}
\mbox{}
\begin{enum}
\item 
Sei $f: M \ra M'$ eine Abbildung. $\bar M \defeqr
M/\sim_f$ und $p: M \ra \bar M, x \mapsto \bar x$ die
Restklassenabbildung.
Dann exisitiert genau eine Abbildung $\bar f: \bar M \to M'$ mit $f = \bar f\circ p$. Es ist $p$ surjektiv und $\bar f$ injektiv.

\item Ist $f:M \to M'$ ein Homomorphismus von \bla, so ist $\bar M$ auch ein \bla und $p$, $\bar f$ sind Homomorphismen.

\item \emp{Homomorphiesatz} \newline
Ist $f:G \to G'$ ein Gruppenhomomorphismus, so ist $G/\Kern(f) \cong \Bild(f)$
%\[\begindc{\commdiag} \obj(1,3){$M$}
%                      \obj(3,3){$M'$}
%                      \obj(2,1){$M_2$}
%                      \mor{$M$}{$M'$}{$f$}[-1,0]
%                      \mor{$M$}{$M_2$}{$p$}[-1,0]
%                      \mor{$M_2$}{$M'$}{$\overline{f}$}[-1,1]
%\enddc\]
\item \emp{Universelle Abbildungseigenschaft (UAE) der Faktorgruppe} \newline
Sei $G$ Gruppe, $N \subseteq G$ Normalteiler. Dann gibt es zu jedem
Gruppenhomomorphismus $f:G \ra G'$ mit $N \subseteq$ $\Kern(f)$ genau
einen Gruppenhomomorphismus $f_N: G/N \ra G' \mbox{ mit } f = f_N \circ p_N$, wobei $p_N$ die Restklassenabbildung ist.
\end{enum} \noindent

\bew{}{
\item
$\bar f(\bar x)= f(x)$, wie in \ref{restklassendefinition} c)
\item[(c)] $\bar f : G / \Kern(f) \to \Bild(f)$ ist injektiv, ein Gruppenhomomorphismus nach a), b) und \ref{gruppenfaktorgruppe}. Also ist $\bar f$ ein bijektiver Homomorphismus, also eine Isomorphie.
\item[(d)] Setze $f_N(x \cd N) \defeqr f(x)$ \newline
$f_N$ ist wohldefiniert: Ist $gN = g'N$, so ist $(g')^{-1}g \in N \subseteq \Kern(f)$, also $f( (g')^{-1} g) = e' \implies f(g') = f(g)$. Die Eindeutigkeit von $\bar f$, sowie dass $\bar f$ ein Homomorphismus ist, ist klar.
}
\end{Satz}

\section{Abelsche Gruppen}

\begin{Bem} % 1.4.1
\begin{enum}

        \item Jede zyklische Gruppe ist isomorph zu $\mathbb{Z}$ oder zu 
        $\mathbb{Z}/n\mathbb{Z}$ für genau ein $n \in \mathbb{N} \setminus
        \{0\}$. 

        \sbew{
            Sei $G=\langle g \rangle$, $\varphi_g: \mathbb{Z} \ra G,\; n \mapsto
            g^n$ (siehe \ref{1.9}) \newline $\varphi_g$ ist surjektiver 
            Gruppenhomomorphismus. \newline
            Nach Satz \ref{Satz 1} ist $G \cong
            \mathbb{Z}/{\Kern(\varphi_g)}$ \newline
            Da jede Untergruppe von $\mathbb{Z}$ von der Form $H=n\mathbb{Z}$
            für ein $n \in \mathbb{N}$ ist, folgt die Behauptung.
        }

        \item Jede Untergruppe einer zyklischen Gruppe ist zyklisch.

        \sbew{
            Sei $G = \langle g \rangle$ zyklisch, $H \subseteq G$ Untergruppe. 
	    Ist $H=\{e\}$, so ist $H= \langle e\rangle$ zyklisch. Anderenfalls sei 
            $n \defeqr \min \{k \in \mathbb{N} \setminus \{0\} : g^k \in
            H\}$.
	    
	    Behauptung: $\langle g^n\rangle = H$, denn sonst gibt es ein $m>0$ mit $g^m \in H
	    \setminus \langle g^n\rangle$. Sei $m$ minimal mit dieser Eigenschaft. Dann ist
	    $0<m-n<m$. Aber: $g^{m-n} = g^m g^{-n}\in H \implies g^{m-n}\in \langle g^n\rangle
	    \implies g^m = g^{m-n}g^n \in \langle g^n\rangle$ Wid!
	    % Das sollte man jemand überprüfen!
            %% Bis auf den gerade korrigierten Typo hats gestimmt.
        }
\end{enum}
\end{Bem}


\begin{DefBem}
    \mbox{}
   
    \begin{enum}
        \item Die Abbildung $\varphi: \mathbb{N} \setminus \{0\} \ra 
        \mathbb{N},\; n \mapsto \varphi(n) \defeqr |\{k \in \{1,\ldots,n\}:
	\mbox{ggT}(k,n) = 1\}|$ heißt \emp{Eulersche 
        $\varphi$ -Funktion}.

	\item $\varphi(1) = 1 = \varphi(2)$, $\varphi(p) = p-1$ für $p$ Primzahl, $\varphi(m\cdot n) = \varphi(m) \cdot \varphi(n)$, falls $m,n$ teilerfremd, $\varphi(p^k) = p^{k-1}(p-1)$, für $p$ Primzahl.

        \item Für jedes $n \in \mathbb{N} \setminus \{0\}$ gilt: $\ds n =
        \sum_{d \mid n} \varphi(d)$

        \sbew{
            $n = |G| = \displaystyle \sum_{d \mid n} | \{x \in G, \mbox{ord}(x)
            = d\}| 
            \overset{(d)}{=} \sum_{d \mid n} \varphi(d)$
        }
        
	\item Ist $G$ zyklische Gruppe der Ordnung $n$, so gilt für jeden 
        Teiler $d$ von $n$: $|\{x \in G : \mbox{ord}(x) = d\}| = \varphi(d)$

        \sbew{
            Sei $G=\langle g \rangle$. Für $x = g^k \in G$ ist ord$(x) = 
            \frac{n}{\mbox{\small ggT}(k,n)}$. Also ist ord$\ds (x) = d \lra 
            \mbox{ggT}(k,n) = \frac{n}{d}$
	    $\implies |\{g\in G\mid \mbox{ord}(g)=d\}| = |\{l\in \{1,\ldots,n\}\mid \mbox{ggT}(l,d)=1\}| = \varphi(d)$.
        }

    \end{enum}

    \bsp{
    \begin{enumerate}
        \item[(1)] \[\{ e^{\frac{2\pi \imath k}{n}} : n\in \mathbb{N} \setminus
        \{0\}, 0 \leq k < n \}\] ist zyklische Untergruppe von $\mathbb{C}^*$ der
        Ordnung $n$. ($n$-te Einheitswurzel)

        \item[(2)] Sei $V = \{ id, \tau, \sigma_1, \sigma_2 \}$ mit $\tau=$ 
        Drehung im $\mathbb{R}^2$ $\begin{pmatrix} -1 & 0 \\ 0 & -1 
        \end{pmatrix}$, \newline
        $\sigma_1=$ Spiegelung an der $x$-Achse $\begin{pmatrix} 1 & 0 \\ 0 & -1
        \end{pmatrix}$, \newline
        $\sigma_2=$ Spiegelung an der $y$-Achse $\begin{pmatrix} -1 & 0 \\ 0 & 1
        \end{pmatrix}$.
        $V$ ist abelsche Gruppe, aber \textbf{nicht} zyklisch. $V$ heißt 
        \emp{Kleinsche Vierergruppe} $V \cong \mathbb{Z}/2\mathbb{Z} \oplus 
        \mathbb{Z}/2\mathbb{Z}$

        \item[(3)] $\begin{array}{ccccc} \mathbb{Z}/6\mathbb{Z} & \cong &
        \mathbb{Z}/2\mathbb{Z} & \oplus & \mathbb{Z}/3\mathbb{Z} \\ 
        \{1,a,a^2,a^3,a^4,a^5\} & & \{1,\sigma\} & &\{1, \tau, \tau^2\}  \\    
        a & \mapsto & (\sigma, \tau) \end{array}$
    \end{enumerate}
    }
\end{DefBem}

\begin{DefBem}
    Sei $G$ Gruppe, $A \subseteq G$ Teilmenge.
    
    \begin{enum}
        \item $\displaystyle \langle A \rangle \defeqr \bigcap_{\substack{H 
        \subseteq G\; Ugr. \\ A \subseteq H}} H\;$ heißt die \emp{von 
        $\mathbf{A}$ erzeugte Untergruppe von $\mathbf{G}$}. \newline
        \sbew{
        z.z.: $\displaystyle \langle A \rangle = \bigcap_{\substack{H
        \subseteq G\; Ugr.\\ A \subseteq H}} H$ ist Untergruppe in G.

        \begin{enum}
            \item[(i)] $\forall H \subseteq G$, $H$ Untergruppe$: e \in H \Ra e
            \in \langle A \rangle \Ra \langle A \rangle \neq \emptyset$

            \item[(ii)] Seien $x, y \in \langle A \rangle$, $H$ Untergruppe von
            $G$
            mit $A \subseteq H \Ra x,y \in H \overset{H \; Ugr.}{\Ra} xy^{-1}
            \in H \Ra
            xy^{-1} \in \langle A \rangle. \Ra \langle A \rangle$ Untergruppe
            von $G$.
        \end{enum}
        }
       
        \item $\langle A\rangle = \{g_1^{\varepsilon_1}\cdots g_n^{\varepsilon_n}, n\in \mathbb N, g_i \in A, \varepsilon_i\in\{\pm1\}\}$

    \end{enum}
\end{DefBem}

\begin{DefProp}
    \label{1.18}
    Sei $(A,+)$ eine abelsche Gruppe, $X \subseteq A$.
    \begin{enum}
        \item $A$ heißt \emp{freie abelsche Gruppe} mit Basis $X$, wenn gilt: $A=\langle X\rangle$ und für alle paarweisen verschiedenen Elemente $x_1,\ldots,x_n\in X$ ist $\sum_{i=1}^n n_i x_i = 0$, $n_i \in \mathbb Z$, nur dann möglich ist, wenn alle $n_i=0$ sind.
	
	Jedes $a \in A$ hat dann eine eindeutige Darstellung $\ds a = \sum_{x\in X} n_x x$ mit
        $n_x \in \mathbb{Z}\;, n_x \neq 0$ nur für endlich viele $x \in X$.
	
        \sbew{
            $A \ra \mathbb{Z}^X : \sum n_x x \mapsto (n_x)_{x \in X}$ ist 
            Isomorphismus.
        }

	\item $\mathbb Z$ ist frei mit Basis $\{1\}$.

	\item $A$ ist frei mit Basis $X$ genau dann, wenn $A \cong \bigoplus_{x\in X}\mathbb Z$.
	\item Ist $A$ frei mit Basis $X$, und $X$ endlich, so heißt $|X|$ der Rang von $A$.

        \item (UAE der freien abelschen Gruppe) \newline
        Ist $A$ frei mit Basis $X$, dann gibt es zu jeder abelschen Gruppe $A'$ und jeder Abbildung $f:X \ra A'$
        genau einen Homomorphismus $\varphi: A\ra A'$ mit $\forall x
        \in X: \varphi(x) = f(x)$ \newline
        \sbew{
            Setze $\ds \varphi(\sum_{x\in X} n_x x) \defeqr \sum_{x\in X} n_x 
            f(x)$
        } 
    \end{enum}


    \bsp{
        (wichtig!) $X$ endlich, $X=\{x_1, \dots, x_n\}$. Dann ist $\mathbb{Z}^X
        \cong \mathbb{Z}^n\\$ $\mathbb{Z}^n$ ist ''so etwas ähnliches'' wie ein
        Vektorraum \textit{(''freier Modul'')}. Insbesondere lassen sich die
        Gruppenhomomorphismen $\mathbb{Z}^n \ra \mathbb{Z}^m$ durch eine $m
        \times n$-Matrix mit Einträgen in $\mathbb{Z}$ beschreiben.
    }

    \bsp{
        Ist $(\mathbb Q,+)$ frei? $(\mathbb Q,+)$ ist nicht frei von Rang 1, sonst wäre $\mathbb Q=r\mathbb Z$ für ein $r\in \mathbb Q$.

	Sei also $(\mathbb Q,+)$ frei mit Basis $X$ und $x_1\ne x_2\in X$. Es gilt $x_i=\frac{n_i}{m_i}$, $n_i,m_i\in \mathbb Z$. Dann ist $n_2m_1x_1 - n_1m_2x_2 = 0$, also sind $x_1,x_2$ linear abhängig.
    }
\end{DefProp} 


\begin{Satz}[Elementarteilersatz]
\label{Satz 2}
Jede Untergruppe einer freien abelschen Gruppe von endlichem Rang $n$ ist frei mit Rang $r\le n$. Genauer:

    Sei $H$ eine Untergruppe von $\mathbb{Z}^n$ $(n \in \mathbb{N} \setminus
    \{0\})$. Dann gibt es eine Basis $\{x_1, \dots, x_n\}$ von $\mathbb{Z}^n$,
    ein $r \in \mathbb{N}$ mit $0 \leq r \leq n$ und $a_1, \dots, a_r \in 
    \mathbb{N} \setminus \{0\}$ mit $a_i$ teilt $a_{i+1}$ für $i = 1,\dots,r-1$,
    so dass $a_1 x_1, \dots, a_r x_r$ eine Basis von $H$ ist. Die $a_i$ sind eindeutig bestimmt.
    
    \sbew{\textbf{1. Schritt}: $H$ ist endlich erzeugt: Induktion über $n$:
        \newline
        \textbf{$n=1$ }:
            $\chk$\newline
        \textbf{$n > 1$}:
            Sei $e_1, \dots, e_n$ Basis von $\mathbb{Z}^n$,
            $\pi: \mathbb{Z}^n \ra \mathbb{Z}$, $\displaystyle \sum_{i=1}^n a_i
            e_i \mapsto a_n$ \newline
            (Projektion auf letze Komponente).
        \bigskip \newline
        \textbf{1. Fall}:
            $\pi(H) = \{0\} \Ra H \subseteq \mathbb{Z}^{n-1}$, also endlich erzeugt
            nach \textbf{IV}.
            \smallskip\newline
        \textbf{2. Fall}:
            $\pi(H) = l\mathbb{Z}$ für ein $l \in \mathbb{N} \setminus \{0\}$
            Sei $y \in H$ mit $\pi(y) = l$ \newline
            \textbf{Beh}.:
                $H \cong \langle y \rangle \oplus (H \cap \Kern(\pi))$
            Dann folgt die Behauptung von Schritt 1, da $\Kern(\pi) \cong
            \mathbb{Z}^{n-1}$, $H \cap$ $\Kern(\pi)$ Untergruppe von
            $\mathbb{Z}^{n-1}$, existiert also nach \textbf{IV} $\Ra$
            \smallskip\newline
            \textbf{Bew. der Beh.}.:
                $\langle y \rangle \cap (H \cap \Kern(\pi)) = \{0\}$ nach
                Definition von $y \Ra$ Summe direkt. \newline Sei $z \in H$ mit
                $\pi(z) = k \cd l$ für ein $k \in \mathbb{Z} \Ra z - ky \in H
                \cap \Kern(\pi) \Ra$ Beh. 

        \noindent\textbf{2. Schritt}:
            Sei $y_1, \dots, y_r$ Erzeugendensystem von $H$. Nach Schritt 1 kann
            $r \leq n$ erreicht werden. Schreibe $y_j = \displaystyle \sum_{i=1}^n a_{ij}
            e_i$. Dann ist $A\defeqr(a_{ij}) \in \mathbb{Z}^{n \times r}$ eine 
            Darstellungsmatrix der Einbettung $H \hookrightarrow \mathbb{Z}^n$ 
            bzgl. der Basen $\{y_1, \dots, y_r\}$ von $H$ und 
            $\{e_1,\dots,e_n\}$ von $\mathbb{Z}^n$. Zeilen- und
            Spaltenumformungen entsprechen Basiswechseln in $H$ bzw. $\mathbb{Z}^n$.
            \newline
            \textbf{Vorsicht}:
                Dabei dürfen nur \textbf{ganzzahlige} Basiswechselmatrizen 
                benutzt werden, deren inverse Matrix ebenfalls ganzzahlige Einträge hat!
            \newline \textbf{Ziel}: Bringe $A$ durch elementare Zeilen- und 
            Spaltenumformungen auf Diagonalgestalt: \newline $\widetilde{A} = 
            \begin{pmatrix} a_1 & & 0 \\ &  \ddots \\ 0  & & a_r \end{pmatrix}$
            mit $a_i \in \mathbb{Z}$ und $a_i$ teilt $a_{i+1}\; \forall\;
            i=1,\dots,r-1$

    \noindent\textbf{3. Schritt}:
            Das geht! Ganzzahliger Gauß-Algorithmus, „Elementarteileralgorithmus“.
            \begin{enum}
                \item[(i)] Suche den betragsmäßig kleinsten Matrixeintrag $\neq
                0$ und bringe diesen nach $a_{11}$. Dazu braucht man höchstens
                eine Zeilen- und eine Spaltenumformung.
                
                \item[(ii)] Stelle fest, ob alle $a_{i1}\; (i=2,\dots,n)$ durch
                $a_{11}$ teilbar sind. Falls nicht, teile $a_{i1}$ mit Rest durch
                $a_{11} : a_{i1} = q a_{11} + r$ mit $0 < r < |a_{11}|$. Ziehe
                dann von der $i$-ten Zeile das $q$-fache der ersten ab. Die
                neue $i$-te Zeile beginnt nun mit $\widetilde{a_{i1}} = r \Ra$
                Zurück zu (i)
                
                \item[(iii)] Sind schließlich alle $a_{i1}$ durch $a_{11}$
                teilbar, so wird die erste Spalte zu \[ \left( \begin{array}{c}
                a_{11} \\ 0 \\ \vdots \\ 0 \end{array} \right) \] gemacht,
                indem man von der $i$-ten Zeile das $\frac{a_{i1}}{a_{11}}$-fache
                der ersten Zeile abzieht. Gegebenenfalls zurück zu (i).

            \item[(iv)] Genauso wird die erste Zeile zu $(a_{11}, 0, \dots, 0)$
            
            \item[(v)] Gibt es jetzt noch einen Matrixeintrag, der nicht durch 
            $a_{11}$ teilbar ist, schreibe $a_{ij} = q a_{11} + r$ mit $0 < r <
            |a_{11}|$ Ziehe von der $i$-ten Zeile das $q$-fache der ersten ab. 
            Die neue $i$-te Zeile lautet dann: \[(-q a_{11}, a_{i2}, \dots, 
            a_{ij}, \dots, a_{ir})\] (da $a_{i1} = 0, a_{1k} = 0$ für $1<k\leq
            r$) \newline
            Addiert man zur $j$-ten Spalte die erste, so ist das neue Element
            $\widetilde{a_{ij}} = a_{ij} - q a_{11} = r \Ra$ Zurück zu (i)
            
            \item[(vi)] Nach endlich vielen Schritten erhalte Matrix 
            \[\begin{pmatrix} a_{11} & 0 & \hdots & 0 \\ 0 \\ \vdots &  & 
            \mbox{\LARGE $A'$}\\ 0 \end{pmatrix},\] in der alle Einträge von
            $A'$ durch $a_{11}$ teilbar sind. Wende nun den Algorithmus auf $A'$ an.
        \end{enum}


        Noch zu zeigen: Die Eindeutigkeit der $a_i$:

	$r$ ist eindeutig, da $r$ der Rang von $H$ ist.

	Ist $x_1,\ldots,x_n$ Basis von $\mathbb Z^n$, und $a_1x_1,\ldots,a_rx_r$ eine Basis
	von $H$ wie im Satz, so ist $H \subseteq \mathbb Z^r$ und $\mathbb Z^r/H \cong
	\bigoplus_{i=1}^r \mathbb Z/ a_i\mathbb Z$, denn $\varphi : \mathbb Z^r\to \oplus_{i=1}^r \mathbb Z/a_i\mathbb Z$, $x_i\mapsto e_i\defeqr (0,\ldots,s_i,\ldots,0)$, ($s_i$ Erzeuger von $\mathbb Z/a_i\mathbb Z$) ist ein surjektiver Homomorphismus. 

	$\Kern(\varphi) \supseteq \langle\{a_1 x_1,\ldots,a_rx_r\}\rangle = H$, sowie $\Kern(\varphi) \subseteq H$, denn für $y\in \Kern(\varphi)$, $y=\sum_{i=1}^r b_ix_i$ gilt: $\varphi(y) = \sum_{i=1}^r b_i e_i = (b_1s_1,\ldots,b_rs_r) = (0,\ldots,0)$, also gilt $a_i\mid b_i$, $i=1,\ldots,r$, also $y\in H$. Nach dem Homomorphiesatz gilt also: $\mathbb Z^r/H \cong
	\bigoplus_{i=1}^r \mathbb Z/ a_i\mathbb Z$.

	Zu zeigen ist nun: Für $T \defeqr \bigoplus_{i=1}^r \mathbb Z / a_i \mathbb Z \cong\bigoplus_{i=1}^s \mathbb Z / b_i \mathbb Z \defeql \tilde T$ mit $a_i\mid a_{i+1}$, $i=1,\ldots,r-1$ und $b_i\mid b_{i+1}$, $i=1,\ldots,s-1$ gilt: $r=s$ und $a_i=b_i$, $i=1,\ldots,r$.

	Für $z\in T$ gilt: $ord(z)\mid a_r$, denn mit $z=(z_1,\ldots,z_n)$, $z_i\in \mathbb Z/a_i \mathbb Z$ gilt $a_r \cdot z = (a_rz_1,\ldots,a_rz_r) = (0,\ldots,0)$. Genauso: $ord(z) \mid b_s$. $T$ enhält das Element $(0,\ldots,0,s_r)=e_r$ und $ord(e_r)=a_r$, also gilt $a_r\mid b_s$ und $b_s\mid a_r$, also $a_r = b_s$. Die Behauptung folgt dann per Induktion über $r$
    }

    \textbf{Ergänzung}:
        \begin{enumerate}
            \item[(1)]In der Situation von Satz 2 heißen die $a_{ii}\; 
            i=1,\dots,r$ die \emp{Elementarteiler} von $H$.
            
            \item[(2)] Ist $A = (h_1,\dots,h_r) \in \mathbb{Z}^{n\times r}$, so
            erzeugen die Spalten $h_1,\dots,h_r$ eine Untergruppe von 
            $\mathbb{Z}^n$. $A$ ist Darstellungsmatrix der Einbettung $H \hookrightarrow 
            \mathbb{Z}^n$.\newline Die Elementarteiler von $H$ heißen auch 
            Elementarteiler von $A$.
        \end{enumerate}
\end{Satz}

\begin{Folg}[Struktursatz für endlich erzeugte abelsche Gruppen]
\label{Satz 3}

Jede endlich erzeugte abelsche Gruppe $A$ ist die direkte Summe von zyklischen Gruppen:

    \[ A \cong \mathbb{Z}^r \oplus \bigoplus_{i=1}^m
    \mathbb{Z}/a_i\mathbb{Z}\]
    mit $r,m,a_1,\dots,a_m \in \mathbb{N}$, $\forall i: a_i \geq 2$, $a_i \mbox{
    teilt }
    a_{i+1}$ für $i=1,\dots,m-1$. Dabei sind $r,m$ und die $a_i$ eindeutig
    bestimmt.

    \sbew{
        Sei $x_1,\dots,x_n$ ein Erzeugendensystem von A.\newline
        Nach \ref{1.18} gibt es einen surjektiven Gruppenhomomorphismus
        $\varphi: \mathbb{Z}^n \to A$ mit $\varphi(e_i) = x_i$, für
        $i=1,\dots,n$.\newline
        Nach Homomorphiesatz (Satz \ref{Satz 1}) ist dann $A \cong \mathbb{Z}^n /
        $$\Kern(\varphi)$. \newline
        Nach Satz \ref{Satz 2} gibt es $m \in \mathbb{N}, m \leq
        n$, eine Basis $\{z_1, \dots , z_n\}$ von $\mathbb{Z}^n$ und
        Elementarteiler $a_1, \dots, a_m$ mit $a_i$ teilt $a_{i+1}$ für $i = 1,
        \dots, m-1$, so dass $\{a_1z_1, \dots, a_mz_m\}$ Basis von $\Kern(\varphi)$
        ist. Dann ist $A \cong \mathbb{Z}^n /$$\Kern(\varphi) \cong \left(
        \displaystyle \bigoplus_{i=1}^n z_i \mathbb{Z} \right) / \left(
        \displaystyle \bigoplus_{i=1}^m a_iz_i \mathbb{Z} \right) \cong 
        \displaystyle \bigoplus_{i=1}^m \left( z_i \mathbb{Z} / a_i z_i
        \mathbb{Z} \right) \oplus \displaystyle \bigoplus_{i=m+1}^n z_i
        \mathbb{Z} \cong \displaystyle \bigoplus_{i=1}^m \mathbb{Z} / a_i
        \mathbb{Z} \oplus \mathbb{Z}^{n-m}$

	Ist $a_1=1$, so lassen wir die $\mathbb Z/1\mathbb Z = \{e\}$ weg.
        
    }
\end{Folg}

\section{Freie Gruppen}

\begin{DefBem}
    Sei $F$ eine Gruppe und $X \subseteq F$
    \begin{enum}
        \item $F$ heißt \empind{freie Gruppe mit Basis $\mathbf{X}$}{freie Gruppe},
        wenn jedes $y
        \in F$ eine eindeutige Darstellung $y = x_1^{\varepsilon_1} \cd
        \dots \cd x_n^{\varepsilon_n}$ hat, in der
        \begin{itemize}
            \item $n \geq 0$ (f\"ur $n=0$ ist $y$ das ''leere Wort'', es ist das neutrale
            Element in $F$)

            \item $x_i \in X$ für $i=1,\dots,n$

            \item $\varepsilon_i \in \{+1,-1\}$ für $i=1,\dots,n$

            \item $x_{i+1}^{\varepsilon_{i+1}} \neq x_i^{-\varepsilon_i}$ für $i
             =1,\dots,n-1$
        \end{itemize}

        \item Ist $F$ frei mit Basis $X$, so gilt für jedes $x \in X$:  $x^{-1}
        \not \in X$.
        \item Ist $F$ frei mit Basis $X$, so ist $F$ torsionsfrei, das heißt: $ord(x) = \infty$ für jedes $x\in F$, $x\ne e$.

        \item $(\mathbb{Z},+)$ ist frei mit Basis $\{1\}$ oder Basis $\{-1\}$

        \item Ist $F$ frei mit Basis $X$ und $|X| \geq 2$, so ist $F$ nicht 
        abelsch. 
        
        \sbew{
            Seien $x_1, x_2 \in X: x_1 \neq x_2 \Ra x_1 x_2 x_1^{-1} x_2^{-1}
            \neq e \Ra x_1 x_2 \neq x_2 x_1$
        }
    \end{enum}
\end{DefBem}

\begin{Satz}
\label{Satz 4}
    \mbox{}
    
    \begin{enum}
        \item Zu jeder Menge $X$ gibt es eine freie Gruppe $F(X)$ mit Basis $X$.

        \item Zu jeder Gruppe $G$ und jeder Abbildung $f: X \ra G$ gibt es genau
        einen Gruppenhomomorphismus $\phi: F(X) \ra G$ mit $\phi(x) = f(x)$ für
        alle $x \in X$.
        
        \item Jede Gruppe ''ist'' (d.h. ist isomorph zu einer) Faktorgruppe einer freien Gruppe.  
        \item $F(X) \cong F(Y) \Leftrightarrow |X| = |Y|$
                                
      \end{enum}
\bew{}{  
\item Sei $X^{\pm} = X \times \{1, -1\}$ und $i: X^{\pm} \ra
X^{\pm}$ die Abbildung: $i(x, \varepsilon) = (x, -\varepsilon)$.
Die Abbildung $i$ ist bijektiv und $i^2 = id$. \newline Schreibweise: $(x,1)
\defeql x\;,\;(x,-1) \defeql x^{-1} \Ra i(x) = x^{-1}\;,\; i(x^{-1}) = x$
\newline Ein Element $g = (x_1,\dots,x_n) \in F^a_0(X^{\pm})$ (freie
Worthalbgruppe) heißt \emp{reduziert}, wenn $x_{\nu +1} \not= i(x_{\nu})$
für $\nu=1,\dots,n-1$. Sei $F(X)$ die Menge der reduzierten Wörter
in $F^a_0(X^{\pm})$
\newline \textbf{Def}.: Zwei Wörter in $F^a_0(X^{\pm})$ heißen
\emp{äquivalent}, wenn sie durch endliches Einfügen oder Streichen
von Paaren der Form $(x,i(x)), x\in X^{\pm}$ auseinander
hervorgehen.
\newline \textbf{Bsp}.: $x_1 \sim x_1 x_2 x_2^{-1} \sim x_1 x_2
x_3^{-1} x_3 x_2^{-1}$
\newline \textbf{Beh}.: In jeder Äquivalenzklasse gibt es genau
ein reduziertes Wort. Dann definiere Verknüpfung auf $F(X) :
(x_1,\dots,x_n) \star (y_1,\dots,y_m)$ sei \emp{das} reduzierte Wort
in der Äquivalenzklasse von $(x_1,\dots,x_n,y_1,\dots,y_m)$. Dieses
Produkt ist \textbf{assoziativ}: Für $x,y,z \in F(X)$ ist $(xy)z$
das eindeutig bestimmte reduzierte Wort in der Klasse von
$(x_1,\dots,x_n,y_1,\dots,y_m,z_1,\dots,z_l)$, und das gleiche gilt für
$x(yz)$.
\newline neutrales Element: $e=()$ \newline
inverses Eement zu $(x_1,\dots,x_n)$ ist $(i(x_n), i(x_{n-1}),
\dots, i(x_1)) \Ra F(X)$ ist Gruppe. $F(X)$ ist frei mit Basis $X$
nach Konstruktion. \newline \textbf{Bew. der Beh}.: In jeder Klasse
gibt es ein reduziertes Wort: \textbf{ja}! \newline
\textbf{Eindeutigkeit}: Seien $x,y$ reduziert und äquivalent. Dann
gibt es ein Wort $w$, aus dem sowohl $x$ als auch $y$ durch
Streichen hervorgehen. Zu zeigen also: Jede Reihenfolge von
Streichen führt zum selben reduzierten Wort. \newline Induktion über die Länge
$l(w)$:\newline $l(w) = 0\; \chk$ \newline $l(w) = 1\; \chk$ \newline Sei $l(w) \geq 2$; Ist
$w$ reduziert, so ...\newline Enthält $w$ genau ein Paar
$(x_{\nu},i(x_{\nu}))$, so muß dies als erstes gestrichen werden. Es
entsteht $w'$ mit $l(w') = l(w) - 2 \overset{\textbf{IV}}{\Ra}$ Beh.
Enthält $w$ Paare $(x_{\nu},i(x_{\nu}))$ und $(x_{\mu},
i(x_{\mu}))$, so gibt es zwei Fälle: (Sei oBdA $\mu > \nu$)\newline
$\mu = \nu +1$: $x_{\nu} i(x_{\nu}) x_{\nu}$ Dann führen beide
Streichungen zum selben Wort. \newline $\mu \geq \nu+2$: Streichen
beider Paare, erhalte $w''$ mit $l(w'') = l(w) - 4
\overset{\textbf{IV}}{\Ra}$ Beh.

\item[(b)] Sei $f:X\to G$ eine Abbildung. Für $w=x_1^{\varepsilon_1}\cdots x_n^{\varepsilon_n}$ setze
\[
\phi(w) = f(x_1)^{\varepsilon_1}\cdot \cdots \cdot f(x_n)^{\varepsilon_n}\,.
\]
Dies muss eindeutig so sein, und so wird ein Homomorphismus definiert.
\item[(c)] Sei $S \subseteq G$ ein Erzeugendensystem (d.h. die einzige
Untergruppe $H$ von $G$ mit $S \subseteq H$ ist $G$ selbst). Sei
$F(S)$ die freie Gruppe mit Basis $S$, $f:S \ra G$ die Inklusion und
$\phi:F(S) \ra G$ der Homomorphismus aus (b). $\phi$ ist
surjektiv, weil $\phi(F(S))$ Untergruppe ist, die $S$ enthält.
Also ist nach Homomorphiesatz $G \cong F(S)/\Kern(\phi)$

\paragraph{Beispiele:}
\begin{enumerate}
\item $G$ zyklisch von Ordnung $n\in \mathbb N$, dann ist $G\cong \mathbb Z / n \mathbb Z$.

\item $\mathbb Z^2 \da \mathbb Z \times \mathbb Z$, $S=\{(0,1) \ad x, (1,0) \ad y\}$. Der Homomorphismus $\varphi : F(S) \to \mathbb Z^2$, $x\mapsto (0,1)$, $y\mapsto (1,0)$ bildet $w = x^{n_1}y^{m_1}\cdots x^{n_d}y^{m_d} \in F(S)$ auf $\varphi(w) = (\sum_{i=1}^d n_i, \sum_{i=1}^d m_i)$ ab, also ist $\Kern\varphi = \{ w = x^{n_1}y^{m_1}\cdots x^{n_d}y^{m_d} \in F(S) \mid \sum_{i=1}^d n_i = \sum_{i=1}^d m_i = 0\} = \langle \{ w_1w_2w_1^{-1}w_2^{-1}, w_1, w_2\in F(S)\}\rangle = G^{\text{ab}}$. $\Kern\varphi$ ist kleinster Normalteiler von $F(\{x,y\})$, der $xyx^{-1}y^{-1}$ enthält, daher ist $\mathbb Z^2 \cong F(\{x,y\}) / \langle xyx^{-1}y^{-1}\rangle_\text{NT}$.
\end{enumerate}

\item[(d)] Erstmal ist klar, dass für jede Abbildung $g : X \ra Y$ ein eindeutiger
Gruppenhomomorphismus $\varphi_g : F(X) \ra F(Y)$ mit
$\varphi_g\left(x\right) = g\left(x\right)$ für alle $x\in X$ existiert. (Dies folgt
aus (b), wenn die Abbildung $X \to F(Y),\ x \mapsto g\left(x\right)$ als $f$
eingesetzt wird.)

''$\Leftarrow$'' Sei $f:X \ra Y$ bijektive Abbildung. Dazu gibt
es Gruppenhomomorphismen $\varphi_f: F(X) \ra F(Y)$ sowie
$\varphi_{f^{-1}} : F(Y) \ra F(X)$.
Es ist sowohl
$\varphi_{f^{-1}} \circ \varphi_{f}
|_X = id_X$ als auch $id_{F(X)}|_X = id_X$, also folgt aus der Eindeutigkeit (b), dass $\varphi_{f^{-1}} \circ \varphi_{f} = id_{F(X)}$. Analog;
$\varphi_f \circ \varphi_{f^{-1}} = id_{F(Y)}$. Also ist $\varphi_f$ ein Isomorphismus.
\newline ''$\Ra$'' Die Anzahl der
Gruppenhomomorphismen von $F(X)$ in $\mathbb{Z}/2\mathbb{Z}$ ist
gleich der Anzahl der Abbildungen von $X$ nach
$\{0,1\}$ (wegen (b)), und diese ist $|2^x| = |\mathcal P(X)|$

Sei $|X| \neq |Y|$, dann ist $|\mathcal P(X)|\ne |\mathcal P(Y)|$. 
}
\end{Satz}

\section{Kategorien und Funktoren}


\begin{Def}
    Eine \emp{Kategorie} $\mathcal{C}$ besteht aus einer Klasse $Ob
    \;\mathcal{C}$ von Objekten und für je zwei Objekte $A,B \in Ob\; \mathcal{C}$ aus 
    einer Menge $Mor_{\mathcal{C}}(A,B)$ von \emp{Morphismen} von $A$ nach $B$, für
    die folgende Eigenschaften erfüllt sind.
    
    \begin{enum}
        \item[(i)] Für jedes $A \in Ob\; \mathcal{C}$ gibt es ein Element $id_A
        \in Mor_{\mathcal{C}}(A,A)$

        \item[(ii)] Für je drei Objekte $A,B,C \in Ob\;\mathcal{C}$ gibt es eine Abbildung
        $\circ$:
        \[\begin{array}{ccccc} Mor(B,C) & \times & Mor(A,B) & \ra & Mor(A,C) \\
        (g  &,& f)  & \mapsto & g \circ f \end{array}\] mit 
        \[\begin{array}{cccc} g \circ id_A & = & g & \mbox{für alle } g \in
        Mor(A,B) \\ id_B \circ f &=& f & \mbox{für alle } f \in Mor(A,B) \\ (h
        \circ g)\circ f & = & h\circ(g\circ f) & \mbox{für alle } f \in
        Mor(A,B), g \in Mor(B,C), h \in Mor(C,D)\end{array}\]
\end{enum}


\bsp{\begin{enumerate}
\renewcommand{\labelenumi}{(\theenumi)}
\item Mengen mit Abbildungen
\item Mengen mit bijektiven Abbildungen
\item $K$-Vektorräume mit $k$-linearen Abbildungen
\item Halbgruppen mit Homomorphismen
\item Monoide mit Homomorphsimen
\item Magmen mit Homomorphismen
\item Gruppen mit Homomorphismen
\item abelsche Gruppen mit Homomorphismen
\item topologische Räume mit stetigen Abbildungen
\end{enumerate}
}
\end{Def}

\begin{Def}
Seien $\mathcal{A}$ und $\mathcal{B}$
Kategorien.
\begin{enum}
\item Ein \emp{kovarianter Funktor} $F:\mathcal{A} \ra \mathcal{B}$
besteht aus einer Abbildung $F: Ob\;\mathcal{A} \ra Ob\;\mathcal{B}$,
sowie für je zwei Objekte $X,Y \in Ob\; \mathcal{A}$ aus einer Abbildung
$F: Mor_{\mathcal{A}}(X,Y) \ra Mor_{\mathcal{B}}(F(X),F(Y))$, so dass
gilt:
\begin{enumerate}
\renewcommand{\theenumi}{(\roman{enumi})}
\item[(i)] $F(id_X) = id_{F(X)}$ für alle $X \in Ob\; \mathcal{A}$
\item[(ii)] $F(g \circ f) = F(g) \circ F(f)$ für alle $f \in Mor_{\mathcal{A}}(A,B), g \in Mor_{\mathcal{A}}(B,C)$
\end{enumerate}
\item Ein \emp{kontravarianter} Funktor $F:\mathcal{A} \ra
\mathcal{B}$ ist ebenso wie in (a) definiert. Ausnahme: $F:
Mor_{\mathcal{A}} (X,Y) \ra Mor_{\mathcal{B}}(F(Y),F(X))$, ... und
$F(g\circ f) = F(f) \circ F(g)$
\end{enum}
\end{Def}

\bsp{\begin{enumerate}
\renewcommand{\labelenumi}{(\theenumi)}
\item $V:$ \underline{Gruppen} $\ra$ \underline{Mengen}, $(G,\cd) \mapsto G$, $V(f) = f$ ist der „Vergissfunktor“

\item
\begin{enumerate}
\item $Im:$ \underline{Mengen} $\ra$ \underline{Mengen}, $Im(X) = \mathcal P(X)$, für $f: X \to Y$ ist $Im(f): \mathcal P(X) \to \mathcal P(Y)$, $Im(f)(U) = f(U)$, $U\in \mathcal P(X)$ ist kovariant.
\item $Urb:$ \underline{Mengen} $\ra$ \underline{Mengen}, $Urb(X) = \mathcal P(X)$, für $f: X \to Y$ ist $Urb(f): \mathcal P(Y) \to \mathcal P(X)$, $Urb(f)(V) = f^{-1}(V)$, $V\in \mathcal P(Y)$ ist kontravariant.
\end{enumerate}

\item Sei $\mathcal{C}$ Kategorie, $X$ ein Objekt in $\mathcal{C}$.
Definiere Funktoren $\mathcal{C} \ra$ \underline{Mengen} durch \newline
Hom$(X,\cd): Y \mapsto Mor_{\mathcal{C}}(X,Y)$ (kovariant)\newline
Hom$(\cd,X): Y \mapsto Mor_{\mathcal{C}}(Y,X)$ (kontravariant) \newline
Für $f\in $Mor$(Y,Z)$ ist Hom$(X,\cd)(f): Mor(X,Y) \ra Mor(X,Z)$ gegeben durch $g
\mapsto f \circ g$ und
Hom$(\cd, X)(f): Mor(Z,X) \ra Mor(Y,X),\;g \mapsto g \circ f$
\item Sei $X$ Menge, $F_X$: \underline{Gruppen} $\ra$ \underline{Mengen}.
$G \mapsto Abb(X,G) = Mor_{Mengen}(X,G)$ ist kovarianter Funktor (also Komposition des Vergissfunktors und des Homomorphismen-Funktors Hom($X,\cdot$)).
\end{enumerate}
}

\begin{Def}
Sei $\mathcal C$ eine Kategorie, $X,Y$ Objekte in $\mathcal C$. $f\in Mor_{\mathcal C}(X,Y)$ heißt \emp{Isomorphismus}, wenn es $g\in Mor_{\mathcal C}(Y,X)$ gibt, so dass $g \circ f = id_X$ und $f \circ g = id_Y$.
\end{Def}

\begin{Def}
Seien $\mathcal A$, $\mathcal B$ Kategorien und $F,G:\mathcal A \to \mathcal B$ kovariante Funktoren. $F$ und $G$ heißen \emp{isomorph}, wenn es zu jedem Objekt $A\in Ob\;\mathcal A$ einen Isomorphismus $\alpha_A: F(A) \to G(A)$, also $\alpha_A\in Mor_{\mathcal B}(F(A),G(A))$ gibt, so dass für alle Morphismen $f:A \to A'$ in $\mathcal A$ das folgende Diagramm kommutiert:
\[
\begin{CD}
F(A) @>\alpha_A>> G(A) \\
@VF(f)VV         @VVG(f)V \\
F(A') @>>\alpha_{A'}> G(A') 
\end{CD}
\]
Also: $G(f)\circ \alpha_A = \alpha_{A'} \circ F(f)$.

Sind die $\alpha_A$ nur Morphismen (also nicht notwendigerweise Isomorphismen), so heißt $\alpha: F\to G$ eine \emp{natürliche Transformation} von Funktoren.
\label{funktorisomorphismus}
\end{Def}

\begin{Prop}
Sei $X$ eine Menge, $F(X)$ die freie Gruppe mit Basis $X$. Dann sind die Funktoren $F_X$ und $Hom(F(X),\cdot):$ \underline{Gruppen} $\to$ \underline{Mengen} isomorph.
\end{Prop}

\sbew{
Nach Satz \ref{Satz 4} gibt es für jede Gruppe $G$ eine bijektive Abbildung $\alpha_G:F_X(G)=Abb(X,G) \to Hom(F(X),G)$, $f\mapsto \varphi = \hat f$. Sei $\rho:G\to G'$ ein Gruppenhomomorphismus. Dann kommutiert:
\[\begin{CD}
F_X(G) @>\alpha_G>> Hom(F(X),G) \\
@VF_X(\rho)VV         @VVHom(F(X),\cdot)(\rho)V \\
F_X(G') @>>\alpha_{G'}> Hom(F(X),G')
\end{CD}\]
Denn für $f\in F_X(G)$ ist $\alpha_G(f)=\hat f$ und $\big((Hom(F(X),\cdot)(\rho)\big)(\hat f) = \rho \circ \hat f$ sowie $F_X(\rho)(\hat f) = \rho \circ f$ und $\alpha_{G'}(\rho \circ f) = \widehat{g\circ f}$. Beides ist \emp{der} eindeutig bestimmte Gruppenhomomorphismus $F(X)\to G'$, der auf $X$ die Abbildung $g\circ f$ ist.
}

\begin{DefBem}
\strut
\begin{enumerate}
\item Sei $\mathcal C$ eine Kategorie und $F:\mathcal C\to$ \underline{Mengen} ein kovarianter Funktor. Ein Objekt $U\in \mathcal C$ heißt \emp{darstellendes Objekt} für $F$, wenn $F$ isomorph zu $Hom(U,\cdot)$ ist.

Analog gilt das für kontravariante Funktoren, wenn $F$ isomorph zu $Hom(\cdot, U)$ ist.
\item $F$ heißt \emp{darstellbar}, wenn es ein darstellendes Objekt für $F$ gibt.
\item Ist $F$ darstellbar, so sind je zwei darstellende Objekte für $F$ isomorph.
\end{enumerate}
\end{DefBem}

\sbew{Seien $U$, $W$ darstellende Objekte für $F$. Dann gibt es einen Isomorphismus von Funktoren $\alpha \da h_U \da Hom(U,\cdot) \to Hom(W,\cdot)$, insbesondere also bijektive Abbildungen $\alpha_U : Mor(U,U) \to Mor(W,U)$ und $\alpha_W: Mor(U,W)\to Mor(W,W)$. Sei $\varphi \da \alpha_U(id_U)$, $\psi \da \alpha_{W}^{-1}(id_W)$. Zu zeigen: $\varphi \circ \psi = id_U$, $\psi \circ \varphi = id_W$.

Das kommutative Diagramm aus Definition \ref{funktorisomorphismus} für den Morphismus $\psi$ ist:
\[
\begin{CD}
Mor(U,U) @>\alpha_U>> Mor(W,U)\\
@Vh_U(\psi)VV         @VVh_W(\psi)V \\
Mor(U,W) @>>\alpha_{W}> Mor(W,W)
\end{CD}
\]
Also gilt 
\begin{align*}
id_W &= \alpha_W(\psi) \\
&= \alpha_W(\psi \circ id_U) \\
&= (\alpha_W \circ h_U(\psi))(id_U)\\
&= (h_W(\psi) \circ \alpha_U)(id_U)\\
&= h_W(\psi)(\varphi) \\
&= \psi\circ \varphi
\end{align*}
und analog folgt $\varphi\circ\psi = id_U$.
}

\section{Gruppenaktionen und die Sätze von Sylow}

\begin{DefBem}
\label{1.22}
    Sei $G$ eine Gruppe, $X$ eine Menge.
    \begin{enum}
        \item Eine \emp{Aktion} (Wirkung) von $G$ auf $X$ ist ein 
        Gruppenhomomorphismus $\rho: G \ra$ Perm$(X)$. $G$ \emp{operiert} dann auf $X$.

        \item Die Aktionen von $G$ auf $X$ entsprechen bijektiv den Abbildungen:
        $G \times X \ra X,\; (g,x) \mapsto gx$, für die gilt

        \begin{enum}
            \item[(i)] $ex = x$ für alle $x \in X$

            \item[(ii)] $(g_1 g_2)x = g_1(g_2 x)$ für alle $g_1, g_2 \in G,\; x
            \in X$
        \end{enum}
        
        \sbew{
Sei $\rho:G\to \operatorname{Perm}(X)$ ein Homomorphismus. Dann erfüllt $G\times X\to X$, $(g,x)\mapsto \rho(g)(x)$ die Eigenschaften (i) und (ii), denn $\rho(e)=id_X$ und $\rho(g_1g_2)(x)=\rho(g_1)(\rho(g_2)(x))$.

Ist umgekehrt $\mu:G\times X\to X$ mit (i), (ii) gegeben, so sei für $g\in G$ die Abbildung $\rho(g):X\to X$ definiert durch $\rho(g)(x)=\mu(g,x)$. $\rho(g)$ ist bijektiv, da $\rho(g^{-1})$ die Umkehrabbildung ist:
\begin{align*}
\rho(g^{-1})(\rho(g)(x)) &= \mu (g^{-1},\mu(g,x))\\
&=g^{-1}\cdot(g\cdot x) \\
&= (g^{-1}\cdot g)\cdot x\\
&=e\cdot x\\
&=x
\end{align*}
Dann ist $\rho: G\to \operatorname{Perm}(X)$, $g\mapsto \rho (g)$ wegen (ii) ein Homomorphismus.
	}

        \bsp{
            \begin{enumerate}
                \renewcommand{\labelenumi}{(\theenumi)}
                \item $G \times G \ra G,\; (g_1, g_2) \mapsto g_1 g_2$ 
                (''Linksmultiplikation'') ist eine Gruppenaktion.

		\item $(g,h)\mapsto h\cdot g$ ist im Allgemeinen keine
                Gruppenaktion, aber $(g,h)\mapsto hg^{-1}$ ist eine.

                \item $G \times G \ra G,\; (g,h) \mapsto ghg^{-1}$ 
                (''Konjugation'') ist eine Gruppenaktion.
                
                \item Ist $X$ eine beliebige Menge, so operiert
                $S_n$ auf $X^n$ durch Vertauschen
                der Komponenten: $\sigma(x_1,\dots,x_n) = 
                (x_{\sigma^{-1}(1)},\dots,x_{\sigma^{-1}(n)})$.
            \end{enumerate}
        }

        \item Eine Aktion $\rho$ heißt \emp{effektiv} (oder \emp{treu}), wenn 
        $\Kern(\rho) = \{e\}$. \newline Allgemein heißt $\Kern(\rho)$ 
        \emp{Ineffektivitätskern} (''Nichtsnutz'') der Aktion.

\bsp{\begin{enum}
\item ist effektiv
\item[c)] Der Ineffektivitätskern ist das Zentrum $Z(G)$
\item[d)] ist effektiv für $|X| \geq 2$
\end{enum}
}
\item Für $x \in X$ heißt $Gx \defeqr \{ gx : g \in G\}$ die \emp{Bahn}
von $x$ unter $G$.
\item $X$ ist disjunkte Vereinigung von $G$-Bahnen.

\sbew{
Durch $x\sim y \iff \exists g\in G: y=gx$ wird eine Äquivalenzrelation definiert, deren Äquivalenzklassen gerade die $G$-Bahnen sind.
}
\item Für $x \in X$ heißt $G_x \defeqr \{ g \in G : gx = x\}$ die
\emp{Fixgruppe} von $x$ unter $G$ (auch \emp{Stabilisator}
oder \emp{Isotropiegruppe} von $x$ genannt). Dies ist eine
Untergruppe von $G$.
\item Für $x \in X,\; g \in G$ ist $G_{gx} = g G_{x} g^{-1}$

\end{enum}
\end{DefBem}


\begin{Prop}[Bahnbilanz]
\label{1.23}
    Sei $X$ endliche Menge, $G$ Gruppe, die auf $X$ operiert. Sei
    $x_1,\dots,x_n$ ein Vertretersystem der $G$-Bahnen in $X$. (dh. aus jeder
    $G$-Bahn genau ein Element). Dann gilt: \newline
    $\ds |X| = \sum_{i=1}^r [G:G_{x_i}]$ \newline
    \sbew{
        Nach \ref{1.22} ist $|X| = \displaystyle \sum_{i=1}^r |G x_i|$. Zu zeigen
        bleibt also: $|G x_i| = [G:G_{x_i}]$. \newline
        \textbf{Beh}.:
            \[\alpha_i = \left\{ \begin{array}{ccc} Gx_i & \ra & G/G_{x_i} \\ gx_i & \mapsto & g G_{x_i} \end{array}
            \right.\] ist bijektive Abbildung, denn:

\begin{itemize}
\item $\alpha_i$ ist wohldefiniert: Ist $g\cdot x_i = h\cdot x_i$, so ist $(h^{-1}g)x_i = x_i$, also $h^{-1}g\in G_{x_i} \implies g\in hG_{x_i} \implies gG_{x_i} \cap hG_{x_i} \ne\emptyset \implies gG_{x_i} = hG_{x_i}$
\item $\alpha_i$ ist injektiv. Ist $gG_{x_i} = hG_{x_i}$, so ist $g\in hG_{x_i}\implies h^{-1}g\in G_{x_i} \implies (h^{-1}g)x_i = x_i \implies g\cdot x_i = h\cdot x_i$
\item $\alpha_i$ ist offensichtlich surjektiv.
\end{itemize}
    }
\end{Prop}

\begin{Satz}[Sylow]
    Sei $G$ endliche Gruppe, $|G| = n$, $p$ eine Primzahl. Sei $n = p^k m$ mit
    $k \geq 0$ und $p\nmid m$. Dann gilt:
    
    \begin{enum}
        \item $G$ enthält eine Untergruppe $S$ der Ordnung $p^k$. Jede solche 
        Untergruppe heißt $\mathbf{p}$\emp{-Sylowgruppe} von $G$.
        
        \item Je zwei $p$-Sylowgruppen sind konjugiert.

        \item Die Anzahl $s_p$ der $p$-Sylowgruppen in $G$ erfüllt: $s_p \mid m$ und
        $s_p \equiv 1 \mod p$.
    \end{enum}

\end{Satz}

    \bew{$k=0:\chk$ Sei also $k \geq 1$.}
    {
        \item Sei $\ds\mathcal{M} = \{ M \subseteq G: |M| = p^k\} \subset 
        \mathcal{P}(G)$. \newline Es ist $\ds|\mathcal{M}| = \binom{n}{p^k} = \binom{p^k
        m}{p^k}$ \newline
        \textbf{Beh.1}:
            $p \nmid |\mathcal{M}|$ \newline
        $G$ operiert auf $\mathcal{M}$ durch die Linksmultiplikation $gM = 
        \{ gx : x \in M\} \in \mathcal{M} \Ra |\mathcal{M}|$ ist 
        Summe der Bahnlängen. Wegen Beh.1 gibt es eine Bahn $G M_0$ mit $p \nmid
        |G M_0|$. \newline $\ds \overset{\ref{1.23}}{\Ra} |G M_0|
        =[G:G_{M_0}] = \frac{|G|}{|G_{M_0}|} = \frac{p^km}{|G_{M_0}|} \Ra p^k \mid |G_{M_0}|$.\newline
        Andererseits ist $|G_{M_0}| \leq p^k = |M_0|$, denn für $x \in M_0$ ist
        $g\mapsto gx$ injektive Abbildung $G_{M_0} \ra M_0 \Ra |G_{M_0}| = p^k$,
        dh. $G_{M_0}$ ist $p$-Sylowgruppe. \newline
        \textbf{Bew. von Beh.1}:
            \[ \binom{p^k m}{p^k} = \prod_{i=0}^{p^k-1} \frac{p^k m - i}{p^k
            -i}\] Schreibe jedes dieser $i$ in der Form $p^{\nu_i} m_i$, mit $p
            \nmid m_i (0\leq \nu_i < k)$ $\ds \Ra \frac{p^k m - i}{p^k
            - i} = \frac{m p^{k-\nu_i} - m_i}{p^{k-\nu_i} - m_i} \Ra$ weder
            Zähler noch Nenner sind durch $p$ teilbar. $\Ra$ Beh.

        \item[(b)] Sei $S \subseteq G$ $p$-Sylowgruppe. \newline
        $\mathcal{S} \defeqr \{S' \le G:S' = gSg^{-1}$ für ein $g \in G$ \}
        \newline
        \textbf{Beh.2}:
            $p \nmid |\mathcal{S}|$.
        \newline
        \textbf{Bew.2}:
            $G$ operiert auf $\mathcal{S}$ durch Konjugation. Diese Aktion ist
            transitiv, d.h. es gibt nur eine Bahn. Die Fixgruppe von $S'$ unter
            dieser Aktion ist $N_{S'} \defeqr \{g\in G: gS'g^{-1} = S'\}$
            \newline
            $N_{S'}$ heißt der \emp{Normalisator} von $S'$ in $G$.\newline
            ($S'$ ist Normalteiler in $N_{S'}$ und maximal mit dieser
            Eigenschaft.) \newline
            $\ds\Ra |\mathcal{S}| = [G:N_{S}] = \frac{|G|}{|N_{S}|} =
            \frac{p^k m}{|N_{S}|}\\$ $S$ ist Untergruppe von $N_{S} \Ra p^k
            \mid |N_{S}| \Ra |\mathcal{S}|$ ist Teiler von $m$. \newline
            Sei $\widetilde{S}$ eine $p$-Sylowgruppe in G. zu zeigen:
         $\widetilde{S} \in \mathcal{S}$. \newline 
         $\widetilde{S}$ operiert auf 
         $\mathcal{S}$ (da $\tilde S \subset G$). Sei nun $s_1, \dots, s_r$ ein
         Vertretersystem der Bahnen. \[\ds \Ra |\mathcal{S}| = \sum_{i=1}^r
         [\tilde S : \widetilde{S_{s_i}}] = \sum_{i=1}^r
         \frac{p^k}{|\widetilde{S_{s_i}}|} \overset{\mbox{\small Beh.2}}{\Ra}
         \mbox{Es gibt ein } i \mbox{ mit } \widetilde S = \widetilde{S_{s_i}}\]
         Dann ist $\widetilde{S} \subseteq N_{S_i}$.
        
	\textbf{Beh.3}:
            Dann ist $\widetilde{S} \subseteq S_i$, also $\widetilde{S} = S_i$,
            da beide $p^k$ Elemente haben. \newline
        \textbf{Bew.3}:
            $S_i$ ist Normalteiler in $N_{s_i}$, $\tilde S$ ist Untergruppe in
            $N_{s_i} \Ra \widetilde{S} S_i$ ist Untergruppe von $N_{s_i}$ (Übung) \newline
            Wäre $\widetilde{S} \not \subseteq S_i$, dann wäre $\widetilde{S}
            S_i \supsetneq S_i$, also $|\widetilde{S} S_i| = p^k d$ mit $d>1$.
            (und $p \nmid d$) \newline
            $\overset{\mbox{Übung}}{\Ra} \widetilde{S} S_i / S_i \cong 
            \widetilde{S} / \widetilde{S} \cap S_i \Ra |\widetilde{S} S_i| = 
            \frac{|S_i| |\widetilde{S}|}{|\widetilde{S} \cap S_i|} = 
            \frac{p^{2k}}{|\widetilde S \cap S_i|} = p^l$ für ein $l\in\mathbb N$. $p^l=p^k d$,
            $d \neq 1 \Ra \blitzb$!
    
    \item[(c)] $s_p = |\mathcal{S}| \Ra s_p \mid m$ und $\mathcal{S} =
    \displaystyle \sum_{i=1}^r [\widetilde{S}: \widetilde{S}_{S_i}]\\$
    $[\widetilde{ S}:\widetilde{S}_{S_i}] = 1 \lra \widetilde{S} =
    \widetilde{S}_{S_i} \overset{Beh.3}{\lra} \widetilde{S} = S_i$, also genau
    \textbf{einmal}. Alle anderen Summanden sind durch $p$ teilbar.}

\begin{Folg}
Ist $G$ eine endliche Gruppe und $p$ eine
Primzahl, die die Gruppenordnung $|G|$ teilt, so enthält $G$ ein Element
von Ordnung $p$.

\sbew{Sei $|G| = p^k m$ mit $p \nmid m$, $k\geq 1$.
$S \subseteq G$ eine $p$-Sylowgruppe und $x \in S$, $x \neq e$.
$\overset{\mbox{\scriptsize \ref{1.12}}}{\Ra}$ ord$(x)$ ist Teiler
von $|S| = p^k \Ra$ ord$(x) = p^d$ für ein $d$, $1 \leq d \leq k$ $\Ra
x^{p^{d-1}}$ hat dann Ordnung $p$. }
\end{Folg}

\bsp{
Wieviele Gruppen $G$ gibt es mit $|G|=15$? Mindestens eine: $G=\mathbb Z/15\mathbb Z\cong \mathbb Z/3\mathbb Z\times \mathbb Z/5\mathbb Z$.

Nach Sylow gibt es nicht 5 3-elementigen Untergruppen, da $5\equiv 1\mod 3$ nicht gilt, und nicht 3, da $3\nmid s_3$. also gibt es nur eine $S_3$ und ebenso nur eine $S_5$.

Daher gibt es genau zwei Elemente der Ordnung 3 und vier Elemente der Ordnung 5. Übrig gleiben 8 Elemente, die Ordnung 15 haben müssen, also ist $G\cong \mathbb Z/15\mathbb Z$.
}

\section{Symmetrische und alternierende Gruppen}

\begin{DefBem}
Sei $n\ge 0$.
\begin{enum}
\item $S_n = \operatorname{Perm}(\{1,\ldots,n\})$ heißt \textit{symmetische Gruppe}.\index{symmetrische Gruppe}
\item $|S_n| = n!$
\item $\xi \in S_n$ heißt \emp{Zyklus} wenn es ein $k$ gibt (mit $1\le k\le n$) und paarweise verschiedene Elemente $i_1, \cdots, i_k$ von $\{1,\ldots,n\}$ mit $\xi(i_\nu) = i_{\nu+1}$ für $\nu=1,\ldots,k-1$, $\xi(i_k) = i_1$ und $\xi(j)=j$ für $j\notin \{i_1, \ldots, i_k\}$. In diesem Fall heißt $\xi$ ein \emp{$k$-Zyklus}, und $k$ wird die \emp{Länge} dieses Zykels $\xi$ genannt.
\item Jedes $\sigma\in S_n$ lässt sich als Produkt von paarweise disjunkten Zykeln schreiben (wobei zwei Zykeln als disjunkt gelten, wenn jedes Element von $\{1,\ldots,n\}$ von mindestens einem der beiden unverändert gelassen wird). Diese Darstellung ist eindeutig bis auf die Reihenfolge.
\item 2-Zykel heißen auch Transpositionen.
\item Jeder $k$-Zyklus ist Produkt von $k-1$ Transpositionen:
\[
(1\ 2 \cdots k) = (1\ 2)\circ(2\ 3)\circ\cdots\circ(k-1\ k)
\]
\item $\sigma \in S_n$ heißt \textit{gerade}, wenn es als Produkt einer geraden Anzahl von Transpositionen geschrieben werden kann, anderenfalls \textit{ungerade}.
\item $\operatorname{sign} : S_n \to \{+1,-1\}$,\[\operatorname{sign}(\sigma) = \begin{cases}+1, &\sigma\text{ gerade}\\-1,&\sigma\text{ ungerade}\end{cases}\] ist ein Homomorphismus.

$A_n\da \Kern(\operatorname{sign}) = \{\sigma \in S_n\mid \sigma \text{ gerade}\}$ heißt \textit{alternierende Gruppe}\index{alternierende Gruppe}.
\end{enum}
\end{DefBem}

\begin{Bem}
\begin{enum}
\item
Je zwei $k$-Zykel in $S_n$ sind konjugiert.
\sbew{
Für $\sigma \in S_n$ ist $\sigma (1\ 2\ \cdots\ k)\sigma^{-1} = (\sigma(1)\ \sigma(2)\ \cdots \ \sigma(k))$, also kann man jedes $k$-Zykel so darstellen.
}
\item Daraus folgt: Zwei Permutationen in $S_n$ sind genau dann konjugiert, wenn sie die gleiche „Zykelstruktur“ haben (d. h. derart jeweils als Produkte disjunkter Zykel dargestellt werden können, dass die Längen der Zykel in der Darstellung der ersten Permutation gleich den Längen der entsprechenden Zykel in der Darstellung der zweiten Permutation sind).
\end{enum}
\end{Bem}

\begin{Bem}
\label{konj3zykel}
\begin{enum}
\item In $A_4$ kann die vorangehende Bemerkung nicht stimmen: $(1\ 2\ 3)$ und $(3\ 2\ 1)$ sind nicht konjugiert.
\item Für $n\ge 5$ sind je zwei 3-Zykel in $A_n$ konjugiert.
\end{enum}
\end{Bem}

\bew{}
{\item Ausprobieren
\item $(1\ 3\ 2) = \sigma (1\ 2\ 3)\sigma^{-1}$ mit $\sigma = (1\ 2)(4\ 5)$\\
      $(i\ j\ k) = \sigma (1\ 2\ 3)\sigma^{-1}$ mit $\sigma = (1\ i\ 2\ j)(3\ k)$ für $i,j,k>3$. Weitere Fälle: Übung.
}

\begin{Bem}
\label{gerPerm3Zykel}
Jede gerade Permutation ist als Produkt von 3-Zykeln darstellbar
\end{Bem}

\sbew{
$(1\ 2)(3\ 4) = (1\ 2\ 3)(2\ 3\ 4)$, $(1\ 2)(2\ 3)= (1\ 2\ 3)$
}

\begin{Satz}
Für $n\ne 4$ enthält $A_n$ nur die Normalteiler $\{1\}$ und $A_n$
\end{Satz}
\sbew{
In $A_4$ ist $\{\text{id},(1\ 2)(3\ 4), (1\ 3)(2\ 4), (2\ 3)(1\ 4)\}$ Normalteiler, $A_1=A_2=\{\text{id}\}$ und $A_3 = \mathbb Z/3\mathbb Z$.

Sei also $n\ge 5$ und $N\ne\{\text{id}\}$ ein Normalteiler von $A_n$.

Es genügt zu zeigen: $N$ enthält einen 3-Zyklus, denn nach \ref{konj3zykel} sind dann alle 3-Zykel in $N$ und nach \ref{gerPerm3Zykel} ist damit $N=A_n$.

Es genügt auch zu zeigen, dass $N$ das Produkt von zwei Transpositionen enthält, denn ist $\sigma=(1\ 2)(3\ 4)\in N$, so ist auch $(3\ 4\ 5)=\sigma(\tau \sigma^{-1} \tau)\in N$, mit $\tau = (1\ 2)(3\ 5)$.

Das Ziel ist also zu zeigen, dass $N$ ein Element $\sigma$ enthält mit $\sigma(i)\ne i$ für höchstens vier $i$, denn dann ist $\sigma\in A_4$, also 3-Zykel oder Produkt von zwei Transpositionen.

Für $\sigma\in A_n$ sei $k_\sigma\da |\{i:\sigma(i)\ne i\}|$. Wähle $\sigma \in N\setminus\{\text{id}\}$ so dass $k_\sigma \le k_\alpha$ für alle $\alpha \in N\setminus\{\text{id}\}$.

Annahme: $k_\sigma\ge 5$. 1. Fall: $\sigma$ enthält einen Zyklus der Länge $\ge 3$, also $\sigma(1)=2$, $\sigma(2)=3$, $\sigma(4)\ne 4$, $\sigma(5)\ne 5$. Sei $\alpha \da \sigma^{-1}(3\ 4\ 5)\sigma(3\ 5\ 4)\in N$. Ist $\sigma(i)=i$, so ist $\alpha(i)=i$ für $i\ge 6$. Außerdem ist $\alpha(1)=1$ und $\alpha(2)=\sigma^{-1}(4)\ne 2$, also ist $\alpha\ne \text{id}$ und $k_\alpha < k_\sigma$.

2. Fall: $\sigma$ ist Produkt von disjunkten
Transpositionen (mind. 4).
Ohne Einschränkung der Allgemeinheit ist $\sigma = (12)(34)(56)(78)\widetilde{\sigma}$ mit
$\widetilde{\sigma} \in A_n$, $\widetilde{\sigma}(i) = i$ für
$i=1,\dots,8$ $\alpha = \sigma^{-1}(345)\sigma(354)$ erfüllt
$\alpha(i)=i$, falls $\sigma(i) = i$, und $\alpha(1) = 1 \Ra
k_{\alpha} < k_{\sigma}$

Also enthält $N$ ein $\sigma$, das höchstens vier $i$ nicht gleich lässt. Damit ist $N=A_n$ gezeigt.
}


\section{Kompositionsreihen}

\textbf{Vorüberlegung}: $G$ Gruppe, $N
\trianglelefteq G$ Normalteiler und $G/N$ die Faktorgruppe.
Läßt sich nun $G$ aus $N$ und $G/N$
rekonstruieren? Nicht unbedingt, wie das Beispiel $D_n$ zeigt. $D_n$ hat als Normalteiler $\mathbb Z/n\mathbb Z$, und ${D_n}/({\mathbb Z/n \mathbb Z}) \cong \mathbb Z/2\mathbb Z$, aber $D_n \ncong \mathbb Z/n \mathbb Z \times \mathbb Z/2\mathbb Z$.

\begin{Def}
Sei $(\ast) \dots \ra G_{i-1}
\overset{\alpha_{i-1}}{\ra} G_i \overset{\alpha_i}{\ra} \dots$ eine
Sequenz (Folge) von Gruppen und Gruppenhomomorphismen.
\newline $(\ast)$ heißt \emp{exakt} an der Stelle $i$, wenn
$\Kern(\alpha_i) =$ $\Bild(\alpha_{i-1})$.
\newline Die Sequenz $(\ast)$ heißt \emp{exakt}, wenn sie an jeder
Stelle exakt ist.
 \smallskip\newline \bsp{
\[0 \ra \mathbb{Z}/2\mathbb{Z} \ra \mathbb{Z}/4\mathbb{Z} \ra
\mathbb{Z}/2\mathbb{Z} \ra 0\] und \[0 \ra \mathbb{Z}/2\mathbb{Z}
\ra \mathbb{Z}/2\mathbb{Z} \oplus \mathbb{Z}/2\mathbb{Z} \ra
\mathbb{Z}/2\mathbb{Z} \ra 0\] sind exakt. Allgemein ist die Sequenz
\[\begin{array}{lr}1 \ra N \ra G \ra G/N \ra 1 & (\ast) \end{array}\]
exakt, wann immer $G$ eine Gruppe und $N$ ein Normalteiler von $G$
sind.
}
\end{Def}

Die Aufgabe, Gruppen zu klassifizieren zerfällt in zwei
Teilaufgaben:
\begin{enumerate}
\renewcommand{\labelenumi}{(\theenumi)}
\item Geg.: $N$ und $G/N$. Welche Möglichkeiten gibt es für $G$?
\item Welche ''unzerlegbaren'' Gruppen gibt es?
\end{enumerate}

\begin{Def}
Sei $G$ eine Gruppe.
\begin{enum}
\item Eine Reihe der Form
\[\begin{array}{lr} G = G_0 \triangleright G_1 \triangleright G_2 \triangleright
\dots \triangleright G_n = \{e\} & (\ast \ast) \end{array}\] (mit $n \in \mathbb{N}$) heißt \emp{Normalreihe}, wenn $G_{i+1}$
Normalteiler in $G_i$ ist ($i=0,\dots,n-1$) und $G_{i+1} \neq G_i$.
\item Die Faktorgruppen $G_i / G_{i+1}$ in einer Kompositionsreihe
$G_0 \triangleright G_1 \triangleright G_2 \triangleright
\dots \triangleright G_n$ heißen die \textbf{Faktoren} (oder \textbf{Faktorgruppen})
dieser Kompositionsreihe.
\item $G$ heißt \emp{einfach}, wenn $G\triangleright \{e\}$ die einzige Normalreihe ist, das hießt: $G$ besitzt nur die trivialen
Normalteiler $G$ und $\{e\}$ und $G\ne \{e\}$.
\item Eine Normalreihe heißt \emp{Kompositionsreihe}, wenn sie sich
nicht verfeinern läßt, dh. wenn $G_i/G_{i+1}$ einfach ist für
$i=0,\dots,n-1$
\end{enum}
\end{Def}

\begin{Bem}
\mbox{}
\begin{enum}
\item $\mathbb{Z}/n\mathbb{Z}$ ist einfach $\lra$ $n$ ist Primzahl.
\item Eine abelsche Gruppe $G$ ist einfach $\lra G \cong
\mathbb{Z}/p\mathbb{Z}$ für eine Primzahl $p$.
\item $\mathbb{Z}$ besitzt keine Kompositionsreihe.
\item Jede endliche Gruppe besitzt eine Kompositionsreihe.
\item Ist $G$ endlich, $(\ast \ast)$ eine Normalreihe, so gilt:
\[ |G| = \prod_{i=0}^{n-1} [ G_i:G_{i+1} ]
= \prod_{i=0}^{n-1} \frac{|G_i|}{|G_{i+1}|}\]
\item 
Es ist eine Kompositionsreihe:
\[ S_4 \triangleright A_4 \triangleright D_2 = \mathbb Z/2\mathbb Z \oplus \mathbb Z/2 \mathbb Z \triangleright \mathbb Z/2\mathbb Z \triangleright \{1\}\]
\item 
Für $n\geq 5$ ist eine Kompositionsreihe:
\[ S_n \triangleright A_n \triangleright \{1\}\]
\end{enum}
\end{Bem}

\begin{Satz}[Jordan-Hölder]
Sei $G$ eine Gruppe, und seien \[G = G_0
\triangleright G_1 \triangleright \dots \triangleright G_m = \{1\}\]\[G = H_0 \triangleright
H_1 \triangleright \dots \triangleright H_l = \{1\}\] Kompositionsreihen für $G$.
\smallskip\newline Dann ist $m =l$ und es gibt eine Permutation
$\sigma \in $Perm$(\{0,\ldots ,m-1\})$ mit $G_i / G_{i+1} \cong
H_{\sigma(i)}/H_{\sigma(i)+1}$ für $i=0,\dots,m-1$.

\sbew{Induktion über $m$:
\begin{description}
\item[$\mathbf{m=1}$:] Dann ist $G$ einfach, also auch $l=1$
\item[$\mathbf{m>1}$:] Sei $\bar G \defeqr G/G_1, \pi: G \ra \bar G$ die
Restklassenabbildung.
\newline$\Ra \bar{H_i} = \pi(H_i)$ ist
Normalteiler in $\bar{H_{i-1}}$ für $i=1,\ldots,l$, denn für 
$\bar{h_i} \in \bar H_i$, $\bar
g \in \bar{H_{i-1}}$ ist $\bar g \bar{h}_i \bar{g}^{-1} = \pi(gh_ig^{-1}) \in \bar{H_i}$  (da $gh_ig^{-1} \in H_i$).
\newline Nach Voraussetzung ist $\bar G$ einfach, also $\bar H_0 = \bar G$, $\bar H_1 =\bar G$ oder $\bar H_1 = \{1\}$, usw. $\Ra \exists j \in
\{0,\dots,l-1\}$ mit $ \bar{H_0} = \dots = \bar{H_j} = \bar{G},
\{1\} = \bar{H_{j+1}} = \dots = \bar{H_l}$.
\newline Sei $C_i \defeqr H_i \cap G_1$, $i=0,\ldots,l$.
\smallskip\newline \textbf{Beh.1}:
\[G_1 = C_0 \triangleright C_1 \triangleright \dots \triangleright C_j \triangleright C_{j+2} \triangleright \dots \triangleright C_l = \{1\}\]
ist Kompositionsreihe für $G_1$ wenn $j\le l-2$, bzw.
\[G_1 = C_0 \triangleright C_1 \triangleright \dots \triangleright C_j  = \{1\}\]
ist Kompositionsreihe für $G_1$ wenn $j=l-1$.
\newline Aber
\[G_1 \triangleright G_2 \triangleright G_3
\triangleright \dots \triangleright G_m = \{1\}\] ist ebenfalls
Kompositionsreihe. $\overset{\mbox{IV}}{\Ra} m-1 = l-1$, also $m=l$ und es gibt
$\sigma: \{0,\dots,j,j+2,\dots,l-1\} \to \{1, \dots, l-1\} $
bijektiv mit \[C_{i}/C_{i+1} \cong G_{\sigma(i)}/G_{\sigma(i)+1}
\text{ für } i\in\{0,\ldots,j,j+2,\ldots,l-1\}\,.\] % und $\ds C_j/C_{j+2} \cong G_{\sigma(j)}/G_{\sigma(j)+1}$.
\newline\textbf{Beh.2}
\begin{enum}
\item $C_j = C_{j+1}$
\item $C_{i}/C_{i+1} \cong H_{i}/H_{i+1}$ für $i \neq j$
\item $H_j/H_{j+1} \cong \bar G = G/G_1$
\end{enum}
\smallskip\textbf{Beh.1 folgt aus Beh.2:}
\newline $C_{i+1}$ ist Normalteiler in $C_{i}$ ($i=0,\dots,l-1$), denn für 
$x \in C_{i+1} = H_{i+1} \cap G_1$ und $y \in C_i = H_{i} \cap G_1$ ist $yxy^{-1} \in H_i \cap G_1 = C_i$.
\newline $C_{j+2}$ ist Normalteiler in $C_j$ wegen Beh.2(a).
\newline $C_{i-1}/C_i$ sind wegen Beh.2(b) einfach und $\neq \{1\}$
($i\neq j+1$) 

\pagebreak[1]
\textbf{Bew. von Beh.2:}\nopagebreak
\begin{enum}
\item $\bar{H}_{j+1} = \{1\}$, dh. $H_{j+1} \subseteq \Kern{\pi} = G_1 \Ra C_{j+1} =
H_{j+1}$. $C_j = H_j \cap G_1$ ist Normalteiler in $H_j$. (weil $G_1$
Normalteiler in $G$ ist)
\newline Da $\bar{H_j} = \bar G \neq \{1\}$, ist $C_j \neq H_j \Ra H_{j+1} = C_{j+1}
\trianglelefteq C_j \triangleleft H_j$, und weil $H_j/H_{j+1}$ einfach ist, folgt $C_j = H_{j+1} = C_{j+1}$
\item Für $i \geq j+1$ ist $\bar{H_i} = \{1\}$, also $H_i \subseteq
G_1$ und damit $C_i = H_i$.
\newline Für $i < j$ ist $\bar{H_{i+1}} = \bar G = G/G_1 \Ra H_{i+1}\cdot G_1
= G_1\cdot H_{i+1} = G$
\[ C_{i}/C_{i+1} = C_{i}/(H_{i+1} \cap C_{i})
\overset{\text{Übung}}{\cong} 
C_{i}\cdot H_{i+1}/H_{i+1}\] zu zeigen also: $C_{i}\cdot H_{i+1} = H_{i}$
\smallskip\newline \textbf{denn:}
„$\subseteq$“: $\chk$
\newline „$\supseteq$“: Da $G_1 H_{i+1} = G$ ist, gibt es zu $x \in
H_{i}$ ein $h \in H_{i+1}$ und $g \in G_1$ mit $x=gh \Ra g=xh^{-1} \in H_i \cdot H_{i+1} \subseteq H_i$, also $g \in H_{i}
\cap G_1 = C_{i}$ und folglich $x = gh \in C_i H_{i+1}$.
\item $\ds H_{j+1} \subseteq G_1 \Ra H_j/H_{j+1} = H_j/C_{j+1}
\overset{(a)}{=} H_j/C_j = H_j/H_j \cap G_1 \cong H_j G_1/G_1 = G/G_1$ 
\end{enum}\end{description}}
\end{Satz}

\begin{DefBem}
\mbox{}
\begin{enum}
\item Eine Gruppe heißt \emp{auflösbar}, wenn sie eine Normalreihe
mit abelschen Faktorgruppen besitzt.
\item Eine endliche Gruppe ist genau dann auflösbar, wenn die
Faktoren in ihrer Kompositionsreihe zyklisch von Primzahlordung
sind.

\sbew{
„$\impliedby$“: Klar

„$\implies$“: Sei
\[
G=G_0\triangleright G_1 \triangleright\cdots\triangleright G_m=\{1\}
\]
eine Normalreihe mit $G_i/G_{i+1}$ abelsch für $i=0,\ldots,m-1$. Verfeinere sie zur Kompositionsreihe
\[
G= G_0 = H_{0,0} \triangleright H_{0,1} \triangleright \cdots \triangleright H_{0,d_0} = G_1 = H_{1,0} \triangleright \cdots \triangleright H_{1,d_1} = G_2 \triangleright  \cdots \triangleright  G_m = \{1\}
\]
Dabei ist 
\[
\faktor{H_{i,j}}{H_{i,j+1}} \cong \faktor{ H_{i,j}/G_{i+1} }{ H_{i,j+1}/G_{i+1}} \subseteq \faktor{ G_i/G_{i+1} }{ H_{i,j+1}/G_{i+1}} 
\]
also ist $H_{i,j}/{H_{i,j+1}}$ isomorph zu einer Untergruppe einer Faktorgruppe einer abelschen Gruppe, also selbst auch abelsch.
}
\end{enum}
\end{DefBem}

\bsp{
\begin{itemize}
\item $D_n = \{ 1, \tau, \ldots, \tau^{n-1}, \sigma, \sigma\tau,\ldots, \sigma\tau^{n-1}\} \triangleright \{1, \tau, \ldots, \tau^{n-1}\} \cong \mathbb Z / n\mathbb Z \triangleright \{1\}$, also ist $D_n$ auflösbar.
\item Für $n\ge 5$ ist $S_n \triangleright A_n \triangleright \{1\}$ Kompositionsreihe, also ist $S_n$ nicht auflösbar.
\end{itemize}
}

\begin{Prop}
Sei $1 \ra G' \ra G \ra G'' \ra 1$ kurze exakte Sequenz von Gruppen, das heißt: $G'$ ist Normalteiler zu $G$ und $G'' = G/G'$.
Dann ist $G$ auflösbar genau dann, wenn $G'$ und $G''$ auflösbar sind.
\end{Prop}

\sbew{
„$\implies$“: Sei
\[
G=G_0\triangleright G_1 \triangleright\cdots\triangleright G_m=\{1\}
\]
eine Normalreihe mit $G_i/G_{i+1}$ abelsch für $i=0,\ldots,m-1$. Dann ist 
\[
G'=G_0\cap G'\triangleright G_1\cap G' \triangleright\cdots\triangleright G_m\cap G'=\{1\}
\]
nach Weglassen von Wiederholungen eine Normalreihe für $G'$. Die Faktorgruppen
\[
\faktor{ G_i \cap G' }{ G_{i+1} \cap G'} \cong \faktor{ G_{i+1} \cdot (G_i \cap G')}{G_{i+1}} \subseteq \faktor{G_i}{G_{i+1}}
\]
sind abelsch.
\[
G''=G_0/(G_0\cap G')\triangleright G_1/(G_1\cap G') \triangleright\cdots\triangleright G_m/(G_m\cap G')=\{1\}
\]
ist ebenso nach Weglassen von Wiederholungen eine Normalreihe für $G''$ mit abelschen Faktorgruppen.


„$\impliedby$“: Ist
\[
G' = G_0 \triangleright G_1 \triangleright\cdots\triangleright G_m = \{1\}
\]
eine Normalreihe für $G'$ mit abelschen Faktoren, 
\[
G'' = H_0 \triangleright H_1 \triangleright\cdots\triangleright H_n = \{1\}
\]
eine solche für $G''$ und $\pi : G\to G/G'=G''$ die Restklassenabbildung, dann ist
\[
G = \pi^{-1}(H_0) \triangleright \pi^{-1}(H_1) \triangleright \cdots \triangleright \pi^{-1}(H_n) = G_0 \triangleright G_1\triangleright \cdots\triangleright G_m = \{1\}
\]
eine Normalreihe für $G$, da $\pi^{-1}(H_{i+1})$ Normalteiler in $\pi^{-1}(H_i)$ ist und $\pi^{-1}(H_i)/\pi^{-1}(H_{i+1}) \cong H_i/H_{i+1}$ abelsch ist.
}


\chapter{Ringe}


\section{Grundlegende Definitionen und Eigenschaften}

\begin{DefBem}
\begin{enum}
\item Ein \emp{Ring} ist eine Menge $R$ mit Verknüpfungen $+$ und
$\cd$, so dass gilt:
\begin{enumerate}
\renewcommand{\labelenumii}{(\roman{enumii})}
\item $(R,+)$ ist abelsche Gruppe
\item $(R,\cd)$ ist Halbgruppe
\item Die Distributivgesetze gelten:
\[ \left. \begin{array}{lcl}
    x\cd(y+z) &=& xy + xz \\
    (x+y)\cd z &=& xz + yz
    \end{array}
    \right\} \mbox{für alle }x,y,z \in R\]
\end{enumerate}

\item $R$ heißt \emp{Ring mit Eins}, wenn $(R,\cd)$ Monoid ist.
\item $R$ heißt \emp{kommutativer Ring}, wenn $(R,\cd)$ kommutativ
ist.
\item Ein Ring $R$ mit Eins heißt \emp{Schiefkörper}, wenn $R^x = (R,\cdot)^\times = R \setminus \{0\}$,
dh. wenn jedes $x \in R \setminus \{0\}$ invertierbar bzgl. $\cd\;$
ist.
\bsp{
\[
\mathbb H \da \{
\begin{pmatrix}
w & z \\ -\bar z & \bar w
\end{pmatrix}, w,z \in \mathbb C\}
\]
ist ein Schiefkörper, genannt die \emp{Hamilton-Quaternionen}.
}
\item Ein kommutativer Schiefkörper heißt \emp{Körper}.

%  \bsp{ \[H = \{a+bi + cj +dk: a,b,c,d \in R\}\] mit
%  komponentenweiser Addition und folgender Multiplikation: \[ i^2 =
%  j^2 = k^2 = -1;\; ij = k = -ji \] (z.B. ist dann $ik = iij = -j,\;
%  kj = ijj = i(-1) = -i$ etc.)\newline
%  
%  \textbf{Es gilt}: $H$ ist Schiefkörper (\emp{Hamilton-Quaternionen})
%  
%  \[\ds (a+bi+cj+dk)(a-bi-cj-dk) =\] \[a^2 - abi - acj -adk + abi + b^2
%  + \dots + c^2 + d^2 = a^2 + b^2 + c^2 + d^2\] \[\Ra
%  \frac{1}{a+bi+cj+dk} = \frac{a}{a^2+b^2 +c^2+d^2} + \frac{b}{\sum}
%  +\frac{c}{\sum} + \frac{d}{\sum}\;\hfill((a,b,c,d) \neq (0,0,0,0)) \] }

\item In jedem Ring gilt:
\[\left. \begin{array}{l}
x\cd 0 = 0 = 0\cd x \\
x(-y) = - (xy) = (-x)y \\
(-x)(-y) = xy \end{array} \right\} \mbox{für alle } x,y \in R\]
\sbew{$x\cd 0 = x\cd(0+0) = x\cd0 +x\cd0$ (genauso für $0\cd x$)
\newline $x(-y) + xy = x(-y +y) =x\cd 0 = 0$
\newline $(-x)(-y) = -((-x)y) = -(-(xy)) = xy$}

\item Ist $R$ ein Ring mit Eins und $R \neq \{0\}$, so ist $0 \neq
1$ in $R$ 
\sbew{Wäre $0 = 1$, so gälte für jedes $x
\in R :x =x\cd 1 = x\cd 0 = 0$, also doch $R = \{0\}$}
\end{enum}
\end{DefBem}

\begin{Def}
Sei $(R,+,\cd)$ ein Ring.
\begin{enum}
\item $R' \subseteq R$ heißt \emp{Unterring}, wenn $(R',+,\cdot)$ Ring
ist. Umgekehrt heißt $R$ dann \emp{Ring\-erweiterung} von $R'$.
\item $I \subseteq R$ heißt (zweiseitiges) \emp{Ideal}, wenn $(I,+)$
Untergruppe von $(R,+)$ ist und $rx \in I, xr \in I$ für alle $x \in
I, r \in R$.

\bsp{
In $R=\mathbb Z$ sind $n\mathbb Z$ für jedes $n\in \mathbb Z$ Ideale. In $R=\mathbb Q$ dagegen sind diese für $n\ne 0$ keine Ideale.
}

\end{enum}
\end{Def}

\begin{DefBem}
\label{idealeimkoerper}
Sei $R$ ein kommutativer Ring.
\begin{enum}
\item Für $a$ ist $(a) \da a \cdot R = \{a \cdot r, r\in R\}$ ein Ideal in $R$.
\item Ein Ideal $I$ in $R$ heißt \emp{Hauptideal}, wenn es ein $a\in R$ gibt mit $I=(a)$.
\item $R$ heißt \emp{Hauptidealring}, wenn jedes Ideal in $R$ ein Hauptideal ist.
\item $\mathbb Z$ ist ein Hauptidealring.
\item Sei $R$ ein kommutativer Ring mit Eins, $R\ne \{0\}$. Dann ist $R$ ein Körper genau dann, wenn $(0)$ und $R$ die einzigen Ideale in $R$ sind.
\sbew{''$\Ra$'' Sei $ I \subset R$ Ideal, $a \in I
\setminus \{0\} \Ra$ es gibt $a^{-1} \in R \Ra 1 = a a^{-1} \in I
\Ra I = R\;(x \in R \Ra x = 1x)$
\newline ''$\Leftarrow$'' Sei $a \in R \setminus\{0\} \Ra (a) = R \Ra
\exists b \in R : ab =1$ }

\bsp{
$\mathbb Z/n \mathbb Z$ ist ein kommutativer Ring mit Eins für jedes $n\in \mathbb N$. Ist $n=p$ für eine Primzahl $p$, so ist $\mathbb F_p \da \mathbb Z / p\mathbb Z$ ein Körper, und $(\mathbb F_p \da \mathbb Z / p\mathbb Z)^\times = \{\bar a, a\in \mathbb Z, \operatorname{ggT}(a,n)=1\}$. In $\mathbb Z/6\mathbb Z$ dagegen gilt $\bar 2\cdot \bar 3 = \bar 0$.
}
\end{enum}
\end{DefBem}

\begin{Def}
Sei $R$ ein kommutativer Ring.
\begin{enum}
\item $x\in R$ heißt \emp{Nullteiler}, wenn es ein
$y \in R \setminus \{0\}$ gibt mit $xy = 0$.
\item $R\ne \{0\}$ heißt \emp{nullteilerfrei}, wenn $0$ der einzige
Nullteiler in $R$ ist. (Das heißt: Aus $xy = 0$ folgt, dass $x=0$ oder
$y=0$.)
\item $R$ heißt \emp{Integritätsbereich} (engl. \emp{integral domain}), wenn er
nullteilerfrei und kommutativ ist sowie eine Eins besitzt.
\end{enum}

\end{Def}
  
\begin{DefBem}
\begin{enum}
\item Eine Abbildung $\varphi: R \ra R'$ ($R,R'$ Ringe) heißt
\emp{Homomorphismus von Ringen}, wenn $\varphi: (R,+) \ra (R',+)$ ein
Homomorphismus von Gruppen und $\varphi: (R,\cd) \ra (R',\cd)$ ein 
Homomorphismus von Halbgruppen ist.
\item Sind $R,R'$ Ringe mit Eins, so heißt ein Homomorphismus von Ringen
$\varphi: R \ra R'$ ein \emp{Homomorphismus von Ringen mit Eins}, wenn
$\varphi(1_R) = 1_{R'}$.
\item Die Ringe bilden mit Ringhomomorphismus eine Kategorie
\item Die Ringe mit Eins bilden mit Homomorphismen von Ringen mit
Eins eine Kategorie (echte Unterkategorie der Ringe)
\item ${(R,+,\cd)}
\hookrightarrow {(R,+)}$ ist
kovarianter Funktor: Ringe $\ra$ abelsche Gruppen.
\medskip\newline$(R,+, \cd) \mapsto (R^x, \cd)$ ist kovarianter
Funktor: Ringe mit Eins $\ra$ Gruppen.
\end{enum}
\end{DefBem}

\bsp{Sei $R$ ein kommutativer Ring mit Eins und $R^{n \times n}$ der Ring der $n \times n$-Matrizen mit
Einträgen in $R$ Für $n \geq 2$ ist $R^{n \times n}$ nicht
kommutativ und nicht nullteilerfrei.

Die Eins in $R^{n \times n}$ ist die Einheitsmatrix:
\[ E_n \da \begin{pmatrix} 1_R & & 0 \\
                 & \ddots \\
                 0 & &  1_R
\end{pmatrix}\]
Die Einheiten in $R^{n \times n}$ sind die invertierbaren Matrizen:
$(R^{n\times n})^x = GL_n(R) = \{A\in R^{n\times n}: \exists B\in R^{n\times n}: A\cdot B = B\cdot A = E_n\} = \{A \in R^{n \times
n} : \det A \in R^\times \}$, denn für die Adjungierte $A^\#$ von $A$ gilt: $A\cdot A^\# = \det (A) \cdot E_n$.

($A^\#=(b_{ij})$ mit $b_{ij}=(-1)^{i+j}\det A_{ji}$, wobei $A_{ji}$ die Matrix $A$ ohne die $j$-te Zeile und $i$-te Spalte ist.)
}
\begin{Bem}
Sei $\varphi: R \ra R'$ Ringhomomorphismus.
Dann gilt:
\begin{enum}
\item $\Bild(\varphi)$ ist Unterring von $R'$
\item $\Kern(\varphi)\da \varphi^{-1}(0)$ ist Ideal in $R$

\sbew{Sei $x \in$ $\Kern(\varphi)$,
$r \in R \Ra \varphi(rx) = \varphi(r)\varphi(x) = \varphi(r)0 = 0
\Ra rx \in$ $\Kern(\varphi)$}
\item Ist $R$ Schiefkörper, $R'$ Ring mit Eins, $\varphi$ Homomorphismus von Ringen mit
Eins, so ist $\varphi$ injektiv oder die Nullabbildung.

\sbew{Sei $x \in R \setminus \{0\} \Ra \varphi(x)
\varphi(x^{-1}) = \varphi(1) \neq 0$, sofern $\varphi$ nicht die Nullabbildung $\Ra \varphi(x) \neq 0 \Ra$ $\Kern(\varphi)$ = $\{0\} \Ra
\varphi$ injektiv.}
\end{enum}
\end{Bem}

\begin{DefBem}
Sei $R$ Ring mit Eins.
\begin{enum}
\item \[\varphi_R: \mathbb{Z} \ra R,\; n \mapsto \left\{ \begin{array}{lc}
n \cd 1_R = \underset{n}{\underbrace{1_R+\dots+1_R}} & n \geq 0 \\
-((-n) \cd 1_R) & n \leq 0 \end{array} \right.\] ist Homomorphismus von
Ringen mit Eins. %\end{enum}
\item Ist $\Kern(\varphi_R$)$=n\mathbb{Z}\;(n\geq 0)$, so heißt $n$
die \emp{Charakteristik} von $R:\;n =$char($R$)
\item Ist $R$ nullteilerfrei, so ist char($R$)$= 0$, oder
char($R$)$=p$ für eine Primzahl $p$.
\item $\Bild(\varphi_R) \cong \mathbb Z/n\mathbb Z$, $n=\operatorname{char}(R)$
\item Ist $K$ (Schief-)Körper der Charakteristik $p>0$, so ist
$\Bild(\varphi_K$)$\cong \mathbb{Z}/p\mathbb{Z} = \mathbb{F}_p$ der
kleinste Teilkörper von $K$. Er heißt \emp{Primkörper}. Ist
char($K$)$=0$, so läst sich $\varphi_K$ eindeutig fortsetzen zu einem injektiven Homomorphismus $\tilde \varphi_K:\mathbb Q\to K$ mit $\tilde\varphi_K(\frac nm)=\varphi_K(n)\cdot\varphi_K(m)^{-1}$.
\end{enum}
\end{DefBem}

\begin{DefBem}
Sei $R$ Ring. Dann gilt:
\begin{enum}
\item Ist $J$ eine Indexmenge und sind $I_j$, $j\in J$ Ideale in $R$, so ist $\bigcap_{i\in J}I_j$ ein Ideal in $R$.
\item Sind $I_1, I_2$ Ideale in $R$, dann ist $I_1 + I_2 = \{a+b:a \in I_1, b\in I_2\}$ ein Ideal.
\item Sind $I_1, I_2$ Ideale in $R$, dann ist $I_1 \cd I_2 = \{\displaystyle \sum_{i=1}^{<\infty} a_i b_i: a_i \in I_1,
b_i \in I_2 \}$ ein Ideal.
\item Sind $I_1, I_2$ Ideale in $R$, dann ist $I_1 \cd I_2 \subseteq I_1 \cap I_2$ (aber im allgemeinen
$\neq$!)
\item Sei $R$ kommutativ mit Eins, $X \subseteq R$. Die Menge
\[(X) = \bigcap_{\substack{I \subseteq R \mbox{ \scriptsize Ideal} \\
X \subseteq I}} I = \{ \sum_{\mbox{\scriptsize endl.}} r_i x_i:\;
r_i \in R, x_i \in X\}\] heißt das von $X$ erzeugte Ideal.
\item Sind $I_1, I_2$ Ideale in einem kommutativen Ring $R$ mit Eins, so ist $\ds I_1 + I_2 = (I_1 \cup I_2)$ und $\ds I_1 \cd I_2 = (\{ab: a \in I_1, b \in I_2\})$.
% Das stimmt doch nicht, oder?
%% hier http://www.math.hu-berlin.de/~roczen/teaching/2004/la_1_2_29.pdf wird das auch anders definiert (oder ist das dasselbe?)
%%% Da wird aber nur die Summe definiert, und die genauso wie hier, oder?
\end{enum}
\end{DefBem}

\section{Polynomringe}

\begin{DefBem}
Sei $R$ ein kommutativer Ring
mit Eins, $R \neq \{0\}$.
\begin{enum}
\item Ein \emp{Polynom} über $R$ ist eine  Folge $f=(a_i)_{i\in \mathbb N}$ mit einem $n_0\in \mathbb N$ so, dass $\forall i>n_0: a_i=0$.
\newline Symbolische Schreibweise: $\ds f=\sum_{i=0}^n a_i X^i$
\item Die Menge $R[X]$ der Polynome über $R$ ist kommutativer Ring mit Eins mit den Verknüpfungen
\[\begin{array}{lclcl}
(a_i)_{i\in \mathbb N}& + &(b_i)_{i\in \mathbb N}& = &(a_i+ b_i)_{i\in \mathbb N} \\
(a_i)_{i\in\mathbb N} & \cd &(b_i)_{i\in\mathbb N}& = &(c_i)_{i\in \mathbb N} \text{ mit } c_i  = \sum_{k=0}^i a_k b_{i-k}\end{array}\]
\item $R \ra R[X]$, $a \mapsto (a,0,\dots)$ ist injektiver
Ringhomomorphismus
\item Für $f=\sum a_iX^i\in R[X]$, $f\ne 0$, heißt $\operatorname{Grad}(f)\da \max\{i\in \mathbb N, a_i\ne 0\}$ der Grad von $f$.
%\item Für $n \geq 2$ heißt $R[X_1,\dots,X_n] =
%(R[X_1,\dots,X_{n-1}])(X_n)$ \newline\emp{Polynomring in
%$\mathbf{\mathit{n}}$ Variablen über $\mathbf{\mathit{R}}$}
\item Für $f,g$ ist Grad($f+g$) $\leq \max($Grad($f$), Grad($g$)), falls $f,g,f+g\ne 0$
\item Für $f,g$ ist Grad($f\cd g$) $\begin{array}{cc} \leq & \mbox{Grad}(f) +
\mbox{Grad}(g) \\
= & \mbox{, falls } R \mbox{ nullteilerfrei} \end{array}$ für $f,g,f\cdot g\ne 0$.
\end{enum}
\end{DefBem}

\begin{Folg}
Ist $R$ Integritätsbereich, so ist $R[X]$
ebenfalls Integritätsbereich und $R[X]^x = R^x$
\end{Folg}
\begin{Prop}
Sei $R$ kommutativer Ring mit Eins.
\begin{enum}
\item Zu jedem $x \in R$ gibt es genau einen Ringhomomorphismus.
$\varphi_x: R[X] \ra R$ mit $\varphi_x|R = id_R$ und $\varphi_x(X) =
x$. Es ist $\ds \varphi_x(a_0, a_1, \dots) = \sum_{i\geq 0} a_i x^i$
\item Zu jedem Homomorphismus $\alpha: R \ra R'$ von Ringen mit Eins
und jedem $y \in R'$ gibt es genau einen Ringhomomorphismus
$\varphi_y:R[X]\ra R'$, $\varphi_y|R = \alpha$ und $\varphi_y(X) =
y$. Explizit: $\varphi_y(\sum a_iX^i)=\sum\alpha(a_i)y^i$.
\end{enum}
\bew{}{
\item ist (b) für $R' = R$ und $\alpha = id_R$
\item Die angegebene Formel
ist die einzig mögliche Definition dieses Ringhomomorphismus, weil
$\ds \varphi_y(a_0,a_1,\dots) = \varphi_y(\sum_{i=0}^n a_i X^i) = \sum_{i=0}^n
\varphi_y(a_i) \varphi_y(X)^i$ sein muß. }
\end{Prop}

\begin{Bem}
Die vorangehende Folgerung bleibt richtig, wenn $R'$ nicht kommutativ ist, solange $\alpha(R)\subseteq Z(R)$ ist, also $\alpha(a)\cdot b = b\cdot \alpha(a)$ für alle $a\in R, b\in R'$ gilt.
\end{Bem}

\begin{Bem}
Die Zuordnung $R \mapsto R[X]$ ist ein
kovarianter Funktor: \underline{Ringe mit Eins} $\ra$ \underline{Ringe mit Eins}.

\sbew{Ist $\alpha: R \ra R'$ Ringhomomorphismus,
so sei $\Psi: R[X] \ra R'[X]$ der Homomorphismus, der durch
$\alpha:R \ra R' \underset{2.8(c)}{\hookrightarrow} R'[X]$ und $X
\mapsto X$ bestimmt ist.}
\end{Bem}

%  \begin{DefBem}[Verallgemeinerung des Polynomrings]
%  \label{2.13}
%  Sei $R$ kommutativer Ring mit Eins, $(H,\cd)$ Halbgruppe.
%  \begin{enum}
%  \item $R[H] = \{(a_h)_{h \in H},\; a_h \neq 0$ nur für endlich viele
%  $h\in H\}$ ist mit den Verknüpfungen \[\begin{array}{ccc}
%  (a_h)+(b_h) &\defeqr& (a_h + b_h) \\
%  (a_h)\cd(b_h) &\defeqr& (c_h) = \sum_{h_1 h_2 = h} a_{h_1} b_{h_2}
%  \end{array}\]
%  ein Ring. $R[H]$ heißt \emp{Halbgruppenring} zu $H$ über $R$.
%  \newline Schreibweise: (auch) $\sum_{h \in H} a_h h$ für $(a_h)$
%  \item $R[(\mathbb{N},+)] \cong R[X],\; R[(\mathbb{N}^n,+)] \cong
%  R[X_1,\dots,X_n]$
%  \item $R[H] \left\{\begin{array}{l} \mbox{kommutativ} \\ \mbox{hat Eins}
%  \end{array}\right\} \lra H \left\{\begin{array}{l} \mbox{kommutativ} \\ \mbox{hat Eins}
%  \end{array}\right\}$
%  \item $(H,\cd) \mapsto (R[H],\cd), h \mapsto 1_R h$ ist injektiver
%  Halbgruppenhomomorphismus.
%  \item Ist $(H,\cd)$ Monoid, so ist $R \ra R[H], \; r\mapsto r\cd
%  1_H$ injektiver Ringhomomorphismus. \end{enum} \sbew{-}
%  \end{DefBem}
%  
%  \begin{Satz}[Universelle Abbildungseigenschaft des Monoidrings]
%  Sei $R$ Ring mit Eins, $(H,\cd)$ Monoid. Dann gibt es zu jedem
%  Homomorphismus $\varphi: R \ra R'$ von Ringen mit Eins und jedem
%  Monoidhomomorphismus $\sigma: H\ra (R',\cd)$ genau einen
%  Ringhomomorphismus $\Phi: R[H] \ra R'$ mit $\Phi_{|R} = \varphi$ und
%  $\Phi_{|H} = \sigma$. Dabei werden $R$ und $H$ wie in \ref{2.13} in
%  $R[H]$ eingebettet. 
%  \sbew{Es muß gelten: $\Phi(\sum_h
%  a_h h) = \sum_h \varphi(a_h) \sigma(h)$. Dies zeigt die
%  Eindeutigkeit, taugt aber auch als Definition von $\Phi$, was die
%  Existenz zeigt.}
%  \end{Satz}

\begin{DefBem}
\begin{enum}
\item $R\llbracket X\rrbracket = \{ (a_i)_{i \in \mathbb{N}}: a_i \in
R\}$ ist mit $+$ und $\cd$ wie im Polynomring ein kommutativer Ring
mit Eins. $R\llbracket X\rrbracket$ heißt \emp{Ring der (formalen)
Potenzreihen} über $R$.
\newline Schreibweise (auch): \[f = \sum_{i=0}^\infty a_i x^i\] für
$f=(a_i)_{i \in \mathbb{N}}$
\item $R[X]$ ist Unterring von $R\llbracket X \rrbracket$.
\item Sei $0 \neq f = \sum_{i=0}^\infty a_i x^i \in
R\llbracket X\rrbracket$. Dann heißt $o(f) \mathrel{\mathop:}=
\min\{i \in \mathbb{N},\; a_i \neq 0 \}$ der \emp{Untergrad} von
$f$. Es gilt für alle $f,g \in R\llbracket X\rrbracket \setminus
\{0\}:$ \[o(f+g) \geq \min \{o(f), o(g)\} \mbox{ und } o(f\cd g)
\geq o(f) + o(g)\]
wobei in der Ungleichung für die Multiplikation Gleichheit gilt, wenn $R$ nullteilerfrei ist.
\end{enum}
\end{DefBem}

\begin{Prop}
\begin{enum}
\item Ist $R$ Integritätsbereich, so ist $o(f\cd g) = o(f) + o(g)\;
\forall f,g \in R\llbracket X\rrbracket \setminus \{0\}$ und es
gilt: $R\llbracket X\rrbracket^x = \{ f = \sum_{i=0}^\infty a_i X^i
\in R\llbracket X\rrbracket : a_0 \in R^x \}$

\item Ist $R = K$ Körper, so ist $m \defeqr K\llbracket
X\rrbracket \setminus K\llbracket X\rrbracket^x = \{ \sum a_i X^i :
a_0 = 0\}$ Ideal in $K\llbracket X\rrbracket$, und das einzige maximale.

\sbew{(a), (b), (d) $\chk$

(c) ''$\subseteq$'': Sei $f = \sum a_i X^i \in R\llbracket
X\rrbracket^x$. Dann gibt es $g = \sum b_i X^i \in R\llbracket
X\rrbracket$ mit $1=fg = a_0 b_0 + (a_1 b_0 + a_0 b_1)X + \dots \Ra
a_0 \in R^x$
\newline ''$\supseteq$'': Definiere $g = \sum b_i X^i$ rekursiv durch
$b_0 = a_0^{-1}, b_i \defeqr a_0^{-1} \cd \sum_{k=1}^i (-1)^k
a_k b_{i-k},\; i \geq 1$. Dann ist $fg = 1$
}

\bsp{$i=1:b_i=a_0^{-1}(a_1 b_0)$}
\end{enum}
\end{Prop}

\section{Faktorringe}

Sei $R$ ein kommutativer Ring mit Eins.

\begin{DefBem}
\begin{enum}
\item Sei $I$ Ideal in $R$. Durch die Verknüpfung $\bar x \cd \overline
y \defeqr \overline{xy}$ wird die Faktorgruppe $(R,+)/(I,+)$ ein
kommutativer Ring mit Eins. Dieser Ring $R/I$ heißt \emp{Faktorring} oder
\emp{Quotientenring} von $R$ und $I$. (Man verwechsle diesen
Begriff des Quotientenrings nicht mit dem Quotientenkörper eines
Integritätsbereiches, siehe weiter unten!)

\item Die Restklassenabbildung $\pi: R \ra R/I, \; x \mapsto
\bar x$ ist surjektiver Ringhomomorphismus mit $\Kern(\pi)=I$.

\item (UAE des Faktorrings:) Sei $\varphi: R \ra R'$ ein
Ringhomomorphismus. Dann gibt es zu jedem Ideal $I \subseteq R$ mit
$I \subseteq \Kern(\varphi)$ einen eindeutig bestimmten
Ringhomomorphismus $\bar \varphi: R/I \ra R'$ mit $\varphi = \bar
\varphi \circ \pi$

%\[\begindc{\commdiag}
%\obj(1,3){$R$}
%\obj(3,3){$R'$}
%\obj(2,1){$R/I$}
%\mor{$R$}{$R'$}{$\varphi$}[-1,0]
%\mor{$R$}{$R/I$}{$\pi$}[-1, 0]
%\mor{$R/I$}{$R'$}{$\exists!\; \bar{\varphi}$}[-1,1]
%\enddc\]

\item (Homomorphiesatz für Ringe:) Ist $\varphi: R \ra R'$
surjektiver Ringhomomorphismus, dann ist $R' \cong
R/ \Kern(\varphi)$.
\end{enum}
\bew{}{\item \emp{Wohldef. des Produkts:} Seien $x', y' \in R$ mit
$\overline{x'} = \overline x, \; \overline{y'} = \overline{y}$. Dann gibt es $a,b \in I$
mit $x' = x+a,\; y' = y+b$. $\Ra x' y' = (x+a) (y+b) = xy + \underset{\in
I}{\underbrace{ay + bx + ab}} \Ra \bar{x'} \bar{y'} = \bar x
\bar y$. \newline Die restlichen Eigenschaften vererben sich dann
von $R$.
\item $\pi$ ist surjektiver Gruppenhomomorphismus mit
$\Kern(\varphi$)$=I$ nach Satz \ref{Satz 1}(a). $\pi(xy) = \pi(x) \cd \pi(y)$
nach Definition der Verknüpfung.
\item Nach Satz \ref{Satz 1}(d) gibt es einen eindeutig bestimmten
Gruppenhomomorphismus $\bar \varphi: R/I \ra R'$ mit $\varphi = \bar
\varphi \circ \pi$.
\newline Zeige also: $\bar \varphi$ ist Ringhomomorphismus: Für
$x,y \in R$ ist $\bar \varphi(\overline{x}\overline{y}) = \varphi(xy) =
\varphi(x) \varphi(y) = \bar\varphi(\bar x) \bar\varphi(\bar y)$
\item Folgt aus (c) und Satz \ref{Satz 1}(a)
}
\end{DefBem}

\begin{Def}
\begin{enum}
\item Ein Ideal $I \subsetneq R$ heißt \emp{maximal}, wenn es kein
Ideal $I'$ in $R$ gibt mit $I \subsetneq I' \subsetneq R$.
\bsp{In $R=K[X]$, $K$ Körper, ist $(X) = \{f=\sum_{i=0}^n a_iX^i, a_0=0\}$ maximal.}
\item Ein Ideal $I \subsetneq R$ heißt \emp{Primideal}, wenn für $x,y
\in R$ mit $xy \in I$ gilt: $x \in I$ oder $y \in I$.

\bsp{\begin{enum} \item[(1)] Für $p\in \mathbb Z$, $p > 0$ gilt: $p$ prim $\lra p \mathbb{Z}$
ist Primideal in $\mathbb{Z}$ (sogar maximal)
\item[(2)] $(X)$ ist Primideal in $R \llbracket X \rrbracket \lra R$
ist Körper.\end{enum}}
\item[(3)] $\{0\}$ ist Primideal in $\mathbb Z$.
\end{enum}
\end{Def}

\begin{Bem}
\label{prim-nullteilerfrei}
\begin{enum}
\item $R$ ist nullteilerfrei $\lra (0)$ ist Primideal.
\item $I\subsetneq R$ ist Primideal genau dann, wenn $R/I$ nullteilerfrei ist.
\end{enum}
\bew{}{\item[(a)] $R$ ist nicht nullteilerfrei $\lra \exists a,b \in
R \setminus\{0\}$: $ab = 0 \lra (0)$ kein Primideal.
\item[(b)] Seien $x,y\in R$ mit $x\cdot y=I$, also $\bar x \cdot \bar y =0$ in $R/I$. $I$ Primideal $\iff x \in I$ oder $y\in I \iff \bar x = 0 $ oder $\bar y=0 \iff R/I$ ist nullteilerfrei.
%\item[(c)] Seien $x,y \in R$ mit $xy \in I$ und $x \not \in I$. Dann
%ist $(x) + I \supset I \overset{I \text{ max.}}{\Ra} (x)
%+ I = R \Ra 1 \in (x) + I,$ dh. es gibt $x \in R, a \in I$ mit $1 = rx
%+a \Ra y = \underset{\in I}{\underbrace{rxy}} + \underset{\in
%I}{\underbrace{ay}} \in I \Ra I$ Primideal.
}
\end{Bem}

\begin{Bem}
Sei $I \subset R$ ein Ideal. Dann gilt:
\begin{enum}
\item Jedes maximale Ideal ist Primideal.
\item $I$ ist maximales Ideal $\lra R/I$ ist Körper.
\end{enum}
\bew{}{
\item folgt aus (b) und Bemerkung \ref{idealeimideal}.
\item Nach \ref{idealeimkoerper} (e) ist $R/I$ genau dann Körper, wenn $(0)$ und $R/I$
die einzigen Ideal in $R/I$ sind. Die Behauptung folgt dann aus:
$I \subsetneq J \subsetneq R$ in $R \Leftrightarrow 0 \neq \bar J \neq R/I$ in $R/I$ wobei $\bar{J}$ das Bild von $J$ in $R/I$ ist.
}
\end{Bem}

\begin{Bem}
\label{idealeimideal}
Sei $I$ ein Ideal in $R$. Dann entsprechen die Ideale in $R/I$ bijektiv den Idealen in $R$, die $I$ enthalten.

\sbew{
Sei $\pi:R\to R/I$ die Restklassenabbildung. Für jedes Ideal $\bar J$ in $R/I$ ist $\pi^{-1}(\bar J)$ ein Ideal in $R$. Es gilt $\pi^{-1}(\bar J)\supseteq \pi^{-1}(0)=\Kern{\pi}=I$.

Sei umgekehrt $J\subsetneq R$ ein Ideal mit $I\subseteq J$. Dann ist $\bar J \da \pi(J)$ ein Ideal in $R/I$, da $\pi$ surjektiv ist.

Weiter ist $\pi^{-1}(\pi(J))=J$, da $\Kern \pi \subseteq J$, und $\pi(\pi^{-1}(\bar J))=\bar J$, da $\pi$ surjektiv ist.
}
\end{Bem}

\begin{Bsp}[Algebraische Konstruktion der reelen Zahlen]
\label{konstruktion-reele-zahlen}
Sei $C = \{(a_n)_{n \in \mathbb{N}}: (a_n)$ Cauchy-Folge, $a_n
\in \mathbb{Q}\}$ (dh. für $k \in \mathbb{N}\; \exists n \in
\mathbb{N}: |a_i - a_j| < \frac{1}{k}$ für $i,j \geq n$) \newline
$C$ ist Ring mit komponentenweiser $+$ und $\cd$ (vornehm: $C
\subset \prod_{n \in \mathbb{N}} \mathbb{Q})$.
\newline $N = \{(a_n) \in C : (a_n)$ Nullfolge $\}$ (dh. für $k \in
\mathbb{N}\; \exists n \in \mathbb{N} : |a_i| < \frac{1}{k}\; \forall i
> n$)\newline
 $N$ ist Ideal in $C:\chk$ \newline
\textbf{Beh.}: $C/N$ ist Körper (bzw. $N$ ist maximal)

\sbew{Sei $a = (a_n)_{n \in \mathbb{N}} \in C
\setminus N$. zu zeigen: $1 \in (N + (a))$. \newline $(a_n) \not \in N \Ra a_n = 0$ nur für
endlich viele $n$, dh. $a_i\neq 0$ für $i > n_0$. \[b_n \defeqr \left\{
\begin{array}{crl} 0 &,& a_i = 0 | i \leq n_0
\\ \frac{1}{a_i} &,& a_i \neq 0 | i > n_0 \end{array} \right.\]
$b=(b_n) \in C$.
\[ ab = (c_n),\; c_n = \left\{ \begin{array}{crl} 0
&:& n < n_0 \\ 1 &:& n \geq n_0 \end{array} \right.\] \[\Ra 1 -ab=(d_n),\;
d_n = \left\{
\begin{array}{crl} 1 &:& n < n_0 \\ 0 &:& n \geq n_0 \end{array}
\right.\] $\Ra (d_n) \in N \Ra 1 = (d_n) + ba \in N + (a) \Ra N$
maximal.
\[\Ra C/N = \mathbb{R} \mbox{!}\] }
\end{Bsp}

\begin{Satz}[Chinesischer Restsatz]
\label{Satz 8}
Sei $R$ kommutativer Ring mit Eins, $I_1,\dots,I_n$
Ideale in $R$ mit $I_\nu + I_\mu = R$ für alle $\nu \not= \mu$ (dann
heißen $I_\nu, I_\mu$ \emp{relativ prim} oder \emp{koprim}) Für $\nu
= 1,\dots,n$ sei $\pi_\nu : R \ra R/I_\nu$ die Restklassenabbildung.
Dann gilt:
\begin{enum}
\item $\begin{array}{ccc}\varphi: R &\ra& R/I_1 \times \dots \times R/I_n \\
x &\mapsto& (\pi_1(x),\dots,\pi_n(x)) \end{array}$ ist surjektiv.

\item Wegen dem Homomorphiesatz und $\Kern(\varphi$)$= \bigcap_{\nu=1}^n I_\nu$ gilt:
\[R/I_1 \times \dots \times R/I_n \cong R/\bigcap_{\nu=1}^n
I_\nu\]
\item (Simultane Kongruenzen:) \newline Für paarweise teilerfremde ganze
Zahlen $m_1,\dots,m_n$ und beliebige $r_1,$ $\dots,r_n \in
\mathbb{Z}$ gibt es $x \in \mathbb{Z}$ mit $x \equiv r_\nu \mod
m_\nu$ für $\nu = 1,\dots,n$ (Spezialfall von (a) für
$R=\mathbb{Z}$) \end{enum}
\end{Satz}
\sbew{Es genügt z.z.:
$\begin{array}{lcl}\bar {e_\nu}=(0,\dots,0,&\underbrace{1},&0,\dots,0) \\ &
\nu\mbox{-te Stelle} &
\end{array} \in$ $\Bild(\varphi)$ für jedes $\nu$, dh. es gibt $e_\nu
\in R\; (\nu=1,\dots,n)$ mit $e_\nu \in I_\mu$ für $\nu \neq \mu$ und
$1-e_\nu \defeql a_\nu \in I_\nu$ (Denn für $x = (\bar
x_1,\dots, \bar x_n) \in R/I_1 \times \dots \times R/I_n$ sei $e
\defeqr \sum_{\nu=1}^n r_\nu e_\nu$ mit $r_\nu \in p_\nu^{-1}(\bar
x_\nu) \Ra \varphi(e) = \sum p_\nu (r_\nu e_\nu) = x$.)
\newline Nach Voraussetzung gibt es für jedes $\ds \mu \neq \nu\; a_\mu \in
I_\nu,b_\mu \in I_\mu$ mit \[a_\mu + b_\mu = 1 \Ra 1 =
\prod_{\substack{\mu = 1 \\ \mu \neq \nu}}^n (a_\mu + b_\mu) =
\underset{\mathclap{\ds \defeql e_\nu \in \bigcap_{\substack{\mu=1 \\ \mu \neq
\nu}}^n I_\mu}}{\underbrace{\prod_{\substack{\mu = 1 \\ \mu \neq
\nu}}^n b_\mu}} + \underset{\in I_\nu}{\underbrace{a_\nu}}\]
$\Ra 1 - e_\nu = a_\nu$ wie gewünscht.}

\section{Teilbarkeit}

Sei $R$ ein Integritätsbereich.
\begin{DefBem}
Seien $a,b \in R\setminus\{0\}$.
\begin{enum}
\item $a$ \emp{teilt} $b$ (Schreibweise $a \mid b$) $:\lra b \in (a)$
($\lra \exists x \in R: b = ax$)

\item $d \in R$ heißt \emp{größter gemeinsamer Teiler} von $a$ und
$b$, (Schreibweise ggT($a$,$b$)) wenn gilt:
    \begin{enum}
        \item[(i)] $d \mid a$ und $d \mid b$ bzw. $a \in (d),b\in (d)$
        \item[(ii)] ist $d' \in R$ auch Teiler von $a$ und $b$, so
        gilt $d' \mid d$ bzw. $d \in (d')$
    \end{enum}
\item Ist $d \in R$ ein ggT von $a$ und $b$ und $e \in R^x$, so ist
auch $e \cd d$ ein ggT. Sind $d,d'$ beide ggT von $a$ und $b$, so gibt es $e \in R^x$ mit
$d' = ed$. \sbew{Nach Definition gibt es $x,y \in R$ mit $d' = xd$ und
$d=yd' \Ra d' = xy d' \Ra d'(1-xy) = 0 \underset{\substack{d'\neq
0\\R \mbox{ \scriptsize nullteilerfrei}}}{\Longrightarrow} 1 = xy
\lra x,y \in R^x$}
\item In analoger Weise wird das kleinste gemeinsame Vielfache definiert.
\end{enum}
\end{DefBem}

\bsp{
\begin{enum}
\item In $\mathbb Z$ gibt es einen größten gemeinsamen Teiler.
\item In jedem nullteilerfreiem Hauptidealring $R$ gibt es zu je zwei Elementen $a,b$ einen größten gemeinsamen Teiler: Denn $(a,b)=(a)+(b)$ ist ein Hauptideal, das heißt, es gibt ein $d\in R$ mit $(a,b)=(d)$. Also gilt $d\mid a$ und $d\mid b$ und für jedes $d'\in R$, für das $d'\mid a$ und $d'\mid b$ gilt, gilt auch: $(a)\subseteq (d')$, $(b)\subseteq(d')$, also $(a,b)\subseteq(d')$ und somit $(d)\subseteq(d')$, also $d'\mid d$.
\item In $\mathbb Z[\sqrt{-5}]$ gibt es zu $6$ und $4 + 2\sqrt{-5}$ keinen größten gemeinsamen Teiler.
\end{enum}
}

\begin{Def}
Ein Integritätsbereich $R$ heißt \emp{euklidisch}, wenn es
eine Abbildung: $\delta: R\setminus\{0\} \ra \mathbb{N}$ mit
folgender Eigenschaft gibt: zu $f,g \in R, g\neq 0$ gibt es $q,r \in
R$ mit $f = qg + r$ mit $r=0$ oder $\delta(r) < \delta(g)$.
\end{Def}

\bsp{$\mathbb{Z}$ mit $\delta(a) = |a|, K[X]$ mit $\delta(f)
=$ Grad($f$)}


\begin{Bem}
\begin{enum} Sei $R$ euklidisch.
\item Für $a,b \in R\setminus\{0\}$ gilt:
    \begin{enum}
        \item[(i)] in $R$ gibt es einen ggT von $a$ und $b$, er heiße $d$.
        \item[(ii)] $(d) = (a,b) = (a) + (b)$
    \end{enum}
\item Jeder euklidische Ring ist ein Hauptidealring.
\end{enum}
\end{Bem}
\bew{}{\item \OE sei $\delta(a) \geq
\delta(b)$. Nach Voraussetzung gibt es $q_1,r_1 \in R$ mit $a = q_1
b + r_1$, $\delta(r_1) < \delta(b)$ oder $r_1 = 0$.
\newline Ist $r_1 = 0$, so ist $a \in (b) = (a,b)$ und
ggT($a,b$)$=b$. Sonst gibt es $q_2,r_2 \in R$ mit $b=q_2r_1 +r_2$
und $r_2 = 0$ oder $\delta(r_2) < \delta(r_1)$. usw... \[\Ra
\begin{array}{ccccc} r_i &= &q_{i+2} r_{i+1} &+& r_{i+2} \\
                    \vdots & & \vdots & & \\
                    r_{n-2} &=& q_n r_{n-1} && \end{array}\]
(da $\delta(r_{i+2}) < \delta(r_{i+1})$)
\newline \textbf{Beh.}: $d \defeqr r_{n-1}$ ist ggT von $a$ und $b$.
\newline\textbf{denn:} $d \mid r_{n-2}$ (vorletzte Zeile: $r_{n-3} = q_{n-1}
r_{n-2} + r_{n-1} \Ra d \mid r_{n-3}$)
\newline Induktion: $d \mid r_i$ für alle $i \Ra d \mid b \Ra d \mid a$
\newline \textbf{umgekehrt:} Sei $d'$ Teiler von $a$ und $b \Ra
d' \mid r_1 \underset{\mbox{\scriptsize Induktion}}{\Ra} d' \mid r_i\; \forall i
\Ra d' \mid d$. \newline noch zu zeigen ist $(d)=(a,b)$:

''$\subseteq$'': $d \in (a,b)$ Nach Konstruktion ist $r_{i+2} \in (r_i,r_{i+1}) \subset \dots \subset (a,b)\; \forall i$
\newline
''$\supseteq$'' $a \in (d), b \in (d)$ nach Definition.
\item Sei $I \subseteq R$ Ideal, $I \neq \{0\}$. Wähle $a \in
I$ mit $\delta(a)$ minimal. Dann gilt für jedes $b \in I : b=qa+r$
mit $r \in I$ und $\delta(r) < \delta(a)\;\blitza$ also $r=0 \Ra I
=(a)$}

\begin{DefBem}
Sei $R$ kommutativer Ring mit
Eins.
\begin{enum}
\item $x,y \in R$ heißen \emp{assoziiert}, wenn es $e \in R^x$ mit
$y=xe$ gibt. ''assoziiert'' ist eine Äquivalenzrelation.

\item $x \in R\setminus R^x$, $x\ne 0$ heißt \emp{irreduzibel} (unzerlegbar), wenn aus $x=y_1
\cd y_2$ mit $y_1,y_2 \in R$ folgt: $y_1 \in R^x$ oder $y_2 \in
R^x$.

\item $x \in R\setminus R^x$ heißt \emp{prim} (oder
\emp{Primelement}), wenn $(x)$ ein Primideal ist, dh. aus $x \mid y_1
y_2$ folgt $x \mid y_1$ oder $x \mid y_2$.

\item Sind $x,y \in R \setminus R^x$ assoziiert, so ist $x$ genau
dann irreduzibel (bzw. prim), wenn $y$ irreduzibel (bzw. prim) ist.

\item Ist $R$ nullteilerfrei, so ist jedes von Null verschiedene Primelement irreduzibel.

\sbew{Sei $(x)$ Primideal und $x=y_1 y_2,\; y_1,y_2 \in R
\Ra$ \OE: $y_1 \in (x)$, dh. $y_1 = xa$ für ein $a \in R$ (R
nullteilerfrei, $x \neq 0$) $\Ra x = xay_2 \Ra x(1-ay_2) = 0 \underset{\mbox{\scriptsize $x\neq 0$}}{\Ra}
ay_2 = 1 \Ra
y_2 \in R^x$
}
\end{enum}
\end{DefBem}

\begin{Bsp}
\begin{enum}
\item In $\mathbb Z/6\mathbb Z$ ist $2$ nicht irreduzibel: $2\cdot(-2)=2$.
\item 
In $R=\mathbb{Z}[\sqrt{-5}] = \{a + b\sqrt{-5}:\;a,b \in \mathbb{Z} \}
\subset \mathbb{C}$ ist $(1+\sqrt{-5})(1-\sqrt{-5}) =  6= 2\cdot 3$
\newline In $R$ ist $2$ kein Primelement, weder $1+\sqrt{-5}$ noch
$1-\sqrt{-5}$ sind durch $2$ teilbar, \textbf{aber} $2$ ist
irreduzibel!.
\newline \textbf{denn}: Sei $2 = (a+b\sqrt{-5})(c+d\sqrt{-5}) \Ra 4
= |2|^2 = (a+b\sqrt{-5})(a-b\sqrt{-5})(\dots) = (a^2 +
5b^2)(c^2+5d^2) = a^2c^2 + \underset{P \geq 0}{\underbrace{5P}} \Ra P =
0 \Ra b = d = 0 \Ra a^2 = 1,\; c^2 = 4$
\end{enum}
\end{Bsp}

\begin{PropDef}
\label{2.21}
Sei $R$ ein Integritätsbereich.
\begin{enum}
\item Folgende Eigenschaften sind äquivalent:
    \begin{enumerate}
        \item[(i)] Jedes $x \in R\setminus\{0\}$ läßt sich eindeutig
        als Produkt von Primelementen schreiben.
        \item[(ii)] Jedes $x \in R\setminus\{0\}$ läßt sich
        ''irgendwie'' als Produkt von Primelementen schreiben.
        \item[(iii)] Jedes $x \in R\setminus\{0\}$ läßt sich eindeutig als
        Produkt von irreduziblen Elementen schreiben.
    \end{enumerate}

\item Sind diese drei Eigenschaften für $R$ erfüllt, so heißt $R$
\emp{faktorieller} Ring. (Oder \emp{ZPE-Ring} (engl.: UFD)). Dabei
ist in (a) ''eindeutig'' gemeint, bis auf Reihenfolge und
Multiplikation mit Einheiten. Präziser: Sei $\mathcal{P}$ ein
Vertretersystem der Primelemnte ($\neq 0$) bezüglich
''assoziiert''.
\newline Dann heißt (i) $\forall x \in R \setminus\{0\}\; \exists!\;e \in
R^x$ und für jedes $p \in \mathcal{P}$ ein $\ds\nu_p(x) \geq
0:x=e\prod_{p \in \mathcal{P}} p^{\nu_p}$. (beachte $\nu_p \neq 0$
nur für endlich viele $p$).
\end{enum}
\sbew{\begin{description}\item[(i) $\Ra$ (ii)] $\chk$
\item[(ii) $\Ra$ (iii)] Sei $x \neq 0, x = ep_1\cd \dots \cd
p_r,\; p_i \in \mathcal{P},\; e \in R^x$. \newline Sei weiter $x = q_1 \cd \dots
\cd q_s$ mit irreduziblem Element $q_j$. \newline Es ist $x \in
(p_1) \Ra \exists j$ mit $q_j \in (p_1)$. \mbox{\OE}: $j=1$ dh. $q_1 =
\varepsilon_1 p_1$ mit $\varepsilon_1 \in R^x$ (da $q_1$ irreduzibel)
$\Ra \varepsilon_1 q_2 \cd \dots \cd q_s = e p_2 \cd \dots \cd p_r$.
Mit Induktion über $r$ folgt die Behauptung.
\item[(iii) $\Ra$ (i)] Noch zu zeigen: Jedes irreduzible Element in
$R$ ist prim.
\newline Sei $p \in R \setminus R^x$ irreduzibel, $x,y \in R$ mit
$xy \in (p)$, also $xy = pa$ für ein $a \in R$. Schreibe
$x=q_1,\dots,q_m,\;y=s_1\cd \dots \cd s_n,\;a=p_1\cd \dots \cd p_l$ mit
irreduziblen Elementen $q_i,s_j,p_k$.
\newline $\Ra xy = q_1\dots q_m s_1 \dots s_n = pa = p\cd p_1 \cd
\dots \cd p_l \overset{\mbox{\scriptsize
Eindeutigkeit}}{\Longrightarrow} p \in
\{q_1,\dots,q_m,s_1,\dots,s_n\}$ (bis auf
Einheiten)\end{description}}
\end{PropDef}

\begin{Bem}
Ist $R$ faktorieller Ring, so gibt es zu
allen $a,b \in R\setminus\{0\}$ einen ggT($a$,$b$).

\sbew{Sei $\mathcal{P}$ wie in \ref{2.21} Vertretersystem der
Primelemente. \[a = e_1 \prod_{p \in \mathcal{P}} p^{\nu_p(a)}, \; b
= e_2 \prod_{p \in \mathcal{P}} p^{\nu_p(b)} \Longrightarrow d
\defeqr \prod_{p \in \mathcal{P}} p^{\nu_p(d)}\] mit $\nu_p(d) =
\min(\nu_p(a),\nu_p(b))$ ist ggT von $a$ und $b$.}
\end{Bem}

\begin{Satz}
\label{Satz 9}
Jeder nullteilerfreie Hauptidealring ist faktoriell.
\newline\newline\bew{}{\item[(1)] Jedes $x\in R \setminus\{0\}$ läßt sich
als Produkt von irreduziblen Elementen schreiben.
\item[(2)] Jedes irreduzible $p \in R \setminus\{0\}$ erzeugt ein
maximales Ideal. Mit \ref{2.21} folgt dann die Behauptung.
\item[B(2)] Sei $p \in R\setminus\{0\}$ irreduzibel, $I$ Ideal in
$R$ mit $(p) \subseteq I \subset R$.
\newline Nach Voraussetzung gibt es $a \in R$ mit $I=(a)$, $a\not
\in R^x$, da $I \neq R$.
\newline Da $p \in (p) \subseteq I = (a)$, gibt es $\varepsilon \in
R$ mit $p = a \varepsilon \overset{p\mbox{ \scriptsize
irreduzibel}}{\Longrightarrow} \varepsilon \in R^x \Ra (p) = (a) =
I$
\item[B(1)] $x \in R\setminus\{0\}$ heiße Störenfried, wenn $x$
nicht als Produkt von irreduziblen Elementen darstellbar ist.
\newline Sei $x$ Störenfried. Dann ist $x \not \in R^x$ und $x$
nicht irreduzibel, also $x = x_1 y_1$ mit $x_1,y_1 \not \in R^x$.
\newline \OE sei $x_1$ Störenfried (sonst ist $x$
doch Produkt von irreduziblen Elementen). Also $x_1 = x_2y_2,\; x_2,
y_2 \not \in R^x$.
\newline \OE sei $x_2$ Störenfried. Induktiv erhalten
wir $x,x_1,x_2,\dots$ alles Störenfriede mit $(x) \subset (x_1)
\subset (x_2) \subset \dots$.
\newline Sei nun $I = \bigcup_{i\geq 1} (x_i)$. $I$ ist Ideal
$\chk \Ra$ \newline Es gibt $a \in R$ mit $I = (a) \Ra \exists i$
mit $a \in (x_i) \Ra x_j \in (x_i)$ für alle $j>i \blitzb$ }
\end{Satz}

\begin{Bem}
\label{p-adische-bewertung}
Sei $R$ ein faktorieller Ring, $\mathcal P$ ein Vertretersystem der Primelemente $\ne 0$. Für $x\in R\setminus\{0\}$ sei $x=e\prod_{p\in\mathcal P}p^{\nu_p(x)}$ die eindeutige Darstellung, also $e\in R^\times$, $\nu_p(x)\in \mathbb N$, $\nu_p(x)\ne 0$ nur für endlich viele $p\in \mathcal P$. Dann gilt für jedes $p\in\mathcal P$:
\begin{enum}
\item $\nu_p(x)=n \iff p^n \mid x$ und $p^{n+1} \nmid x$
\item Die Abbildung $\nu_{p}\to \mathbb N$ erfüllt
\begin{enumerate}
\item[(i)] $\nu_p(x\cdot y) = \nu_p(x) + \nu_p(y)$
\item[(ii)] $\nu_p(x+y) \ge \min(\nu_p(x), \nu_p(y))$, falls $x+y\ne 0$
\end{enumerate}
\item Sei $\rho\in \mathbb R$, $0<\rho<1$. Dann ist die Abbildung $|\cdot|_\rho:R\to \mathbb R$,
\[
|x|_\rho = 
\begin{cases}
\rho^{\nu_p(x)}, & x \ne 0\\
0 & x = 0
\end{cases}
\]
ein „nichtarchimedischer Betrag“ auf $R$, d.h. $|x\cdot y|_\rho = |x|_\rho \cdot |y|_\rho$ und $|x+y|_\rho \le \max(|x|_\rho, |y|_\rho)$.
\item $d_\rho(x,y) = |x-y|_\rho$ ist eine Metrik auf $R$.
\end{enum}
\end{Bem}

\bsp{
$R=\mathbb Z$, $\mathcal P = \{p\in\mathbb N_{>0}, p$ Primzahl$\}$. $\nu_p$ ist die \emp{$p$-adische Bewertung} und $|\cdot|_{\frac1p}$ ist der \emp{$p$-adische Betrag} auf $\mathbb Z$ (und $\mathbb Q$).
}

\begin{Satz}[Irreduzibilitätskriterium für Polynome]
Sei $R$ ein faktorieller Ring, $p\in \mathcal P$, $f = \sum_{i=0}^n a_i X^i \in R[X]$ mit $a_n \neq 0$, $\operatorname{ggT}(a_0,\ldots,a_n)=1$, $p\nmid a_n$.
\begin{enum}
\item (Eisenstein) Ist $n>0$, $p\mid a_i$ oder $a_i=0$ für $i=0,\ldots,n-1$, $p^2\nmid a_0\ne 0$, so ist $f$ irreduzibel.
\end{enum}
\end{Satz}

\bew{Sei $f = gh$ mit $\ds g=\sum_{i=0}^r b_i X^i$, $\ds h =
\sum_{i=0}^s c_i X^i$ $b_r \neq 0 \neq c_s \Ra n=r+s,\;a_n = b_r
c_s$, $a_0 = b_0 c_0 \Ra p \nmid b_r$, $p \nmid c_s$}{
\item \OE $p
\mid b_0$, $p \nmid c_0$.
\newline Sei $t$ maximal mit $p \mid b_i$ für $i=0,\dots,t$
\newline Dann ist $0 \leq t \leq r-1$ und
\[ \underset{\not \in (p)}{\underbrace{a_{t+1}}} = \underset{\not \in
(p)}{\underbrace{b_{t+1} \cd c_0 }} + \underset{\in
(p)}{\underbrace{\sum_{i=0}^t b_i c_{t+1-i}}}\notin (p)\] $\Ra t+1 = n
\Longrightarrow r =n \Ra s= 0 \Ra f=c_0\cdot g$, nach Voraussetzung ist dann $c_0\in R^\times$.}

\section{Brüche}

\textbf{Ziel:} Verallgemeinere die Konstruktion von $\mathbb{Q}$
aus $\mathbb{Z}$. \[\mathbb{Q} = \{ \frac{m}{n}\;:\; m,n \in
\mathbb{Z} \neq 0\}/_{\sim}\] mit $\frac{m}{n} \sim \frac{m'}{n'}
\lra mn' = m'n$

\begin{DefBem}
Sei $R$ kommutativer Ring mit
Eins, $S \subseteq (R,\cd)$ ein Untermonoid.
\begin{enum}
\item $S^{-1} R = R_S = (R \times S)/_{\sim}$ mit der
Äquivalenzrelation $(a_1,s_1) \sim (a_2,s_2) \mathrel{\mathop:}\lra
\exists t \in S\;: t(a_2 s_1 - a_1 s_2) = 0$ heißt \emp{Ring der
Brüche} von $R$ mit Nennern in $S$. (oder \emp{Lokalisierung} von
$R$ nach $S$) Schreibweise: $\frac{a}{s}$ sei eine Äquivalenzklasse
von $(a,s)$

\sbew{
z.z.: $\sim$ ist Äquivalenzrelation:
\newline reflexiv $\chk$
\newline symmetrisch $\chk$
\newline transitiv: $\left. \begin{array}{lcc}
                    (1) & a_2 s_1 & = a_1 s_2 \\
                    (2) & a_3 s_2 & = a_2 s_3\end{array} \right\}
\overset{?}{\Longrightarrow} a_3 s_1 = a_1 s_3$ \[a_3 s_2 s_1
\overset{(2)}{=} a_2 s_3 s_1 \overset{(1)}{=} a_1 s_3 s_2 \Ra
s_2(a_3 s_1 - a_1 s_3) = 0\] (falls $R$ nullteilerfrei und $0 \notin S
\Ra a_3 s_1 = a_1 s_3$)

Andernfalls sei nun mit $t,t' \in S$ $\left. \begin{array}{c} t(a_2
s_1 - a_1 s_2) = 0 \\  t'(a_2 s_3 - a_3 s_2) = 0 \end{array}
\right\} \Ra t t' s_2 (a_3 s_1 - a_1 s_3) = t(t' a_3 s_2 s_1 - t'
a_1 s_3 s_2) \overset{(2)}{=} t(t' a_2 s_3 s_1 - t' a_1 s_3 s_2) =
\\t s_3 t' (a_2 s_1 - a_1 s_2) \overset{(1)}{=} 0$}

\item Mit $\frac{a_1}{s_1} \cd \frac{a_2}{s_2} = \frac{a_1 a_2}{s_1
s_2}$ und $\frac{a_1}{s_1} + \frac{a_2}{s_2} = \frac{a_1 s_2 + a_2
s_1}{s_1 s_2}$ ist $R_S$ ein kommutativer Ring mit Eins.

\sbew{$\mathbf{\cd}$ \textbf{wohldefiniert}: Sei $\frac{a_1'}{s_1'} =
\frac{a_1}{s_1} \Ra \exists t \in S: t(a_1' s_1 - a_1 s_1') = 0
(\ast) \Ra t(a_1' a_2 s_1 s_2 - a_1 a_2 s_2 s_1')
\overset{(\ast)}{=} (ta_1 s_1' a_2 s_2 - t a_1 a_2 s_2 s_1') = 0$
\newline $\mathbf{\mathop+}$ \textbf{wohldefiniert}: Seien die
$\frac{a_1'}{s_1'}$, $\frac{a_1}{s_1}$ wie oben. $\Ra t(s_1' s_2(a_1s_2 + 
a_2 s_1) - s_1 s_2(a_1' s_2 + a_2 s_1')) = t s_2(a_1 s_2
s_1' + a_2 s_1 s_1' - a_1' s_1 s_2 - a_2 s_1 s_1')
\overset{(\dots)}{=} 0$. Die restlichen Eigenschaften vererben sich
von $R$}
\end{enum}
\end{DefBem}

\begin{DefBem}
Sei $R$ Integritätsbereich, $S = R \setminus \{0\}$. Dann ist
Quot($R$)$\defeqr R_S$ ein Körper, denn das Inverse zu $\frac ba$ mit
$a\ne 0$ ist $\frac ab$. Er heißt der \emp{Quotientenkörper} von $R$.
(Dieser Begriff hat mit dem Quotientenring $R/I$ von $R$ modulo einem
Ideal $I$ nichts zu tun.)

\bsp{
\begin{enum}
\item $R = \mathbb{Z}[X] \Ra$ Quot($R$)$ = \mathbb{Q}(X)$
\item $R = K[X_1,\dots,X_n],\; K$ Körper $\Ra$ Quot($R$)$=
K(X_1,\dots,X_n)$ Körper der rationalen Funktionen in $n$ Variablen.
\end{enum}}
\end{DefBem}

\begin{Bspe}
\begin{enum}
\item Ist $0\in S$, so ist $R_S=0$.
\item $x \in R\setminus\{0\},\; S=\{x^n : n \geq 0\}$ $R_S \defeql
R_x = \{ \frac{a}{x^n}\;:a\in R, n \geq 0\}$
\newline z.B.: $R = \mathbb{Z}$, $x=2 \Ra R_S =
\mathbb{Z}[\frac{1}{2}] = \{ \frac{m}{2^n}\;: m \in \mathbb{Z},\;
n\in\mathbb{N}\}$
\item Sei $\mathfrak{p} \subset R$ Primideal, dann ist $S = R \setminus \mathfrak{p}$ ist Monoid.
\newline $R_S \defeql R_\mathfrak{p}$ heißt Lokalisierung von $R$
nach $\mathfrak{p}$.
\bsp{
\begin{enum}
\item $R = \mathbb{Z}$, $\mathfrak{p} = (2) \Ra
\mathbb{Z}_{(2)} = \{\frac{m}{n}\;:m\in \mathbb{Z},\; n$ ungerade
$\}$ 
\item $\mathfrak p = (0)$, $R$ nullteilerfrei, dann ist $R_{\mathfrak p} = \operatorname{Quot}(R)$.
\item $R=K[X]$, $\mathfrak p=(X)$, dann ist $R_{\mathfrak p} = \{\frac fg, f,g\in K[X], g(0)\ne 0\}$.
\end{enum}}

\item $\mathfrak{p}R_\mathfrak{p} = \{\frac{x}{y}\;:x\in
\mathfrak{p},\; y\in R \setminus \mathfrak{p}\}$ ist maximales Ideal
in $R_\mathfrak{p}$ und zwar das einzige.

\textbf{denn}: Sei $\frac{z}{y} \in R_\mathfrak{p}
\setminus \mathfrak{p}R_\mathfrak{p}$, dh. $z \in R \setminus \mathfrak{p}$,
$y \in R \setminus \mathfrak{p} \Ra \frac{y}{z} \in R_\mathfrak{p}
\Ra \frac{z}{y} \in (R_\mathfrak{p})^x$,\newline typisches Beispiel:
$R = \mathbb{R}[X]$ (oder $R = C^0([-1,1])$) $\mathfrak{p} = \{f \in
R\;: f(0) = 0\}$ ist Primideal in $R$. $R_\mathfrak{p} =
\{\frac{f}{g}\;: f,g \in R, g(0) \neq 0\}$

\end{enum}
\end{Bspe}

\begin{Bem}
Sei $R$ kommutativer Ring mit Eins, $S
\subset (R,\cd)$ Monoid.
\begin{enum}
\item Die Abbildung $i_S: R\ra R_S, a\mapsto \frac{a}{1}$ ist
Ringhomomorphismus

\item $i_S$ ist injektiv genau dann, wenn $S$ keinen Nullteiler von $R$
enthält. ($0 \not \in S$)

\sbew{$\frac{a}{1} = 0 = \frac{0}{1}$ in $R_S \Ra
\exists s \in S$ mit $s(a 1 - 0 1) = 0$}

\item $i_S(S) \subset (R_S)^x$

\sbew{$(\frac{s}{1})^{-1} = \frac{1}{s}$}

\item (UAE) Zu jedem Homomorphismus $\varphi: R \ra R'$ von Ringen
mit Eins mit $\varphi(S) \subset (R')^x$ gibt es genau einen
Homomorphismus $\widetilde{\varphi}:R_S \ra R'$ mit $\varphi =
\widetilde{\varphi} \circ i_S$

\sbew{$\widetilde{\varphi}(\frac{a}{s}) = \widetilde{\varphi}(a
\frac{1}{s}) = \widetilde{\varphi}(\frac{a}{1} (\frac{s}{1})^{-1}) =
\varphi(a) \varphi(s)^{-1}$}

%\[\begindc{\commdiag} \obj(1,3){$R$}
%                      \obj(3,3){$R'$}
%                      \obj(2,1){$R_s$}
%                      \mor{$R$}{$R'$}{$\varphi$}[-1,0]
%                      \mor{$R$}{$R_s$}{$i_s$}[-1,0]
%                      \mor{$R_s$}{$R'$}{$\exists!\;\wt{\varphi}$}[-1,1]
%\enddc\]
\end{enum}
\end{Bem}

\section{Der Satz von Gauß}

Sei $R$ faktorieller Ring, $\mathcal{P}$ Vertretersystem
der von Null verschiedenen Primelemente in $R$.

\begin{Bem}
Für jedes $p\in \mathcal P$ lässt sich $\nu_p$ fortsetzen zu einer Abbildung $\nu_p: \operatorname{Quot}(R)\setminus\{0\} \to \mathbb Z$, die die Eigenschaften von \ref{p-adische-bewertung} b) erfüllt. Dabei gilt für $a,b\in R\setminus\{0\}: \nu_p(\frac ab) = \nu_p(a) - \nu_p(b)$.
\end{Bem}

\bsp{
\begin{enum}
\item $R=\mathbb Z$, $\mathcal P = \{p\in \mathbb N, p \text{ Primzahl}\}$. $\nu_p$ ist die \emp{$p$-adische Bewertung} auf $\mathbb Q$. Die Vervollständigung von $\mathbb Q$ wie in Beispiel \ref{konstruktion-reele-zahlen} ergibt den Körper $\mathbb Q_p$ der $p$-adischen Zahlen.
\item $R=\mathbb C[X]$, $\mathcal P=\{X-a, a\in \mathbb C\}$. Für $p=X-a\in \mathcal P$, $f\in \mathbb C[X]$ ist $\nu_p(f) = \ord_a(f)$ die Nullstellenordnung der Nullstelle $a$.
\end{enum}
}

\begin{DefProp}
\label{nu-multiplikativ}
Sei $R$ faktorieller Ring, $\mathcal P$ Vertretersystem der von Null verschiedenen Primelemente in $R$, $p\in \mathcal P$ und $K=\operatorname{Quot}(R)$.
\begin{enum}
\item Für $f=\sum_{i=0}^n a_iX^i\in K[X]\setminus\{0\}$ sei $\nu_p(f) \defeqr \min\{\nu_p(a_i), i=0,\dots,n\}$.
\item $f\in K[X]\setminus\{0\}$ heißt \emp{primitiv}, wenn $\nu_p(f) = 0$ für alle $p \in \mathcal{P}$ ist.
\item (Gauß) Für $f,g\in K[X]\setminus\{0\}$ gilt: $\nu_p(f\cdot g) = \nu_p(f) + \nu_p(g)$ für alle $p\in \mathcal P$.
\sbew{
Sei $f=\sum_{i=0}^na_iX^i$, $g=\sum_{j=0}^mb_jX^j$, $f\cdot g = \sum_{k=0}^{m\cdot n} c_kX^k$, also $c_k=\sum_{i+j=k}a_ib_j$.

\paragraph{1. Fall:} Sei $m=0$. Dann ist $c_k=a_kb_0$ für $k=0,\ldots,n$ und \begin{align*}
\nu_p(f\cdot g) & = \min_{i=0}^n(\nu_p(a_ib_0)) \\&= \min_{i=0}^n(\nu_p(a_i) + \nu_p(b_0))\\ &= \min_{i=0}^n(\nu_p(a_i)) + \nu_p(b_0) = \nu_p(f) + \nu_p(g)
\end{align*}

\paragraph{2. Fall:} Sei $f,g\in R[X]$ und primitiv, also $\nu_p(f)=\nu_p(g)=0$. Sei $i_0 \defeqr \min_{i=0}^n\{i: p\nmid a_i\}$ und $j_0 \defeqr \min_{j=0}^n\{j: p\nmid b_j\}$. Es ist:
\[
c_{i_0+j_0}
= \underbrace{a_{i_0}b_{j_0}}_{p\nmid} + \sum_{i=0}^{i_0-1} \underbrace{a_i}_{p\mid} b_{i_0+j_0-i} + \sum_{j=0}^{j_0-1} a_{i_0+j_0-j} \underbrace{b_j}_{p\mid}
\]
also gilt $p\nmid c_{i_0+j_0}$ und damit $\nu_p(f\cdot g)=0$.

\paragraph{3. Fall:} $f,g$ sind beliebig. Es gibt $c,d\in K\setminus\{0\}$, so dass $\tilde f = c\cdot f$, $\tilde g = d\cdot g$ primitiv sind. Dann folgt aus Fall 1 und Fall 2, dass:
\begin{align*}
\nu_p(f\cdot g)
&= \nu_p(\frac 1c \tilde f \cdot \frac 1d \tilde g) \\
&= \nu_p(\frac 1c) + \nu_p(\frac 1d) + \nu_p(\tilde f \cdot \tilde g) \\
&= \nu_p(\frac 1c) + \nu_p(\tilde f) + \nu_p(\frac 1d) + \nu_p(\tilde g) \\
&= \nu_p(f) + \nu_p(g)
\end{align*}
}
\end{enum}
\end{DefProp}

\begin{Satz}[Gauß]
Ist $R$ faktorieller Ring, so ist $R[X]$ faktoriell.
\end{Satz}

\sbew{
Sei $K=\operatorname{Quot}(R)$. Dann ist $K[X]$ faktoriell (sogar euklidisch), und $R[X]\subseteq K[X]$ ist ein Unterring. Sei $\mathcal P$ Vertretersystem der von Null verschiedenen Primelemente in $K[X]$. O.B.d.A. ist jedes Primpolynom in $\mathcal P$ ein primitives Polynom in $R[X]$. Sei weiter $\tilde{\mathcal P}$ ein Vertretersystem der von Null verschiedenen Primelemente in $R$. Sei nun $f\in R[X]\setminus\{0\}$. Schreibe $f = c \cdot f_1\cdots f_n$ mit $f_i\in \mathcal P$ und $c\in (K[X])^\times = K\setminus\{0\}$.

Es ist $c\in R$, denn: für $p\in \tilde{\mathcal P}$ ist nach \ref{nu-multiplikativ}
\[
\underbrace{\nu_p(f)}_{\ge 0} = \nu_p(c) + \sum_{i=1}^n \underbrace{\nu_p(f_i)}_{=0}\,,
\]
also ist $\nu_p(c)\ge 0$.

Schreibe also $c=e\cdot p_1\cdots p_m$ mit $e\in R^\times$ und $p_i\in \tilde{\mathcal P}$.

\paragraph{Behauptung 1:} Jedes $p_i\in \tilde{\mathcal P}$ ist auch prim in $R[X]$:

Sei $(p)\defeqr p\cdot R[X]$ das von $p$ in $R[X]$ erzeugte Ideal. Es genügt zu zeigen: $R[X]/(p)$ ist nullteilerfrei (nach \ref{prim-nullteilerfrei} b)). Sei $\bar R\defeqr R/(p\cdot R)$. $\bar R$ ist nullteilerfrei, da $p\in \tilde{\mathcal P}$ ist, also ist auch $\bar R[X]$ nullteilerfrei.

Die Restklassenabbildung $\pi: R\to \bar R$ ist surjektiv und induziert einen surjektiven Ringhomomorphismus $\tilde \pi: R[X]\to \bar R[X]$. Es ist $\Kern{\pi} = \{f=\sum_{i=0}^na_iX^i\in R[X], p\mid a_i, i=0,\ldots,n\} = p\cdot R[X]$, also ist $\bar R[X] \cong R[X]/(p)$.

\paragraph{Behauptung 2:} Jedes $f_i\in \mathcal P$ ist auch prim in $R[X]$:

Seien $g,h \in R[X]$ mit $g\cdot h\in (f_i) \defeqr f_i\cdot R[X]$. Da $f_i$ prim in $K[X]$ ist, ist o.B.d.A: $g\in f_i\cdot K[X]$, also $g=f_i\cdot \tilde g$ für ein $\tilde g \in K[X]$. Für jedes $p\in \tilde{\mathcal P}$ ist $0\le \nu_p(g) = \nu_p(f_i) + \nu_p(\tilde g) = \nu_p(\tilde g)$, also ist $\tilde g\in R[X]$ und damit $(f_i)$ ein Primideal in $R[X]$.
}


\begin{Bsp}
\label{Bsp 2.27}
$f(X) = X^{p-1} + X ^{p-2} + \dots + X + 1
\in \mathbb{Q}[X]$, $p$ Primzahl. Beh.: $f$ ist irreduzibel.
\newline Beobachte: \[f(X) = \frac{X^p - 1}{X - 1}\] (f heißt ''p-tes
Kreisteilungspolynom'' (Zeichnung fehlt))
\newline \textbf{Trick}: $g(X) = f(X + 1)$ ist genau dann
irreduzibel, wenn $f(X)$ irreduzibel ist. \[g(X) = \frac{(X+1)^p -
1}{X} = \sum_{k=1}^p \binom{p}{k} X^{k-1}\mbox{, }(n=p-1)\mbox{, }(\binom{p}{p} = 1 =
a_{p-1},\;\binom{p}{1} = p = a_0)\] Noch zu überlegen: $\binom{p}{k}$
ist durch $p$ teilbar für $k=1,\dots,p-1$, bekannt: $\binom{p}{k} =
\frac{p!}{k!(p-k)!} \Ra$ $\binom{p}{k}$ ist durch $p$ teilbar. Mit
Eisenstein folgt die Behauptung.
\end{Bsp}

\section{Maximale Ideale}

\begin{Prop}
Sei $R$ ein kommutativer Ring mit Eins. Dann gibt es zu jedem echten Ideal $I\triangleright R$ ein maximales Ideal $\mathfrak m$ mit $I\subseteq \mathfrak m$.
\end{Prop}

\subsection*{Lemma von Zorn}

Sei $M$ eine nicht leere, geordnete Menge. Hat jede total geordnete Teilmenge von $M$ eine obere Schranke in $M$, so besitzt $M$ ein maximales Element.

Zur Erinnerung:
\begin{itemize}
\item $\le$ heißt \emp{Ordnung} wenn $\le$ reflexiv, transitiv und antisymmetrisch ist.
\item $N\subset M$ ist \emp{total geordnet}, falls für $x,y\in N$ gilt: $x\le y$ oder $y\le x$.
\item $x\in M$ ist eine \emp{oberere Schranke} für $N$ wenn für alle $y\in N$ gilt: $y\le x$.
\item $m\in M$ heißt \emp{maximal}, wenn für alle $x\in M$ aus $m\le x$ folgt, dass $x=m$ ist.
\end{itemize}

\sbew{(der Proposition)
Sei $M$ die Menge aller echten Ideale in $R$, die $I$ enthalten. $I\in M$, also $M\ne \emptyset$. $M$ ist durch $\subseteq$ geordnet.

\textbf{Behauptung:} $n=\bigcup_{J\in N} J$ ist obere Schranke für $N\subseteq M$. Nach Zorn enthält $M$ dann ein maximales Element $\mathfrak m$. $\mathfrak m$ ist ein maximales Ideal in $R$.
}

\sbew{(der Behauptung)
\begin{itemize}
\item $n$ ist ein Ideal: Seien $x,y\in n$, also $x\in J_1$, $y\in J_2$. O.B.d.A.A. $J_1 \subseteq J_2$, also $x\in J_2$ und damit auch $x+y\in J_2\subseteq n$. Auch gilt für alle $a\in R$: $a\cdot x \in J \subseteq n$.
\item $I\subseteq n$ $\chk$
\item $n$ ist eine obere Schranke von $N$. $\chk$
\item $n \ne R$, denn sonst wäre $1\in n$, also $1\in J$ für ein $J\in N$, im Widerspruch zu $J\in M$.
\end{itemize}
}

\section{Moduln}

Sei $R$ kommutativ mit Eins.

\begin{DefBem}
\begin{enum}
\item Eine abelsche Gruppe $(M,+)$ zusammen mit einer Abbildung
$\bullet: R \times M \ra M$ heißt $\mathbf{R}$\emp{-Modul}, wenn für
alle $a,b \in R,\; x,y\in M$ gilt:
\begin{enumerate}
\item[(i)] $a(x+y) = ax + ay$
\item[(ii)] $(a+b)x = ax + bx$
\item[(iii)] $(ab)x = a(bx)$
\item[(iv)] $ 1x = x$
\end{enumerate}
\bsp{\begin{enumerate}
\item[(1)] $R$ ist $R$-Modul. (mit $\cd$ als Ringmultiplikation)
\item[(2)] Ist $R$ ein Körper, so ist $R$-Modul = $R$-Vektorraum.
\item[(3)] $R=\mathbb{Z}$, $M = \mathbb{Z}/2\mathbb{Z} = \{\bar 0,
\bar 1\}$ ist $\mathbb{Z}$-Modul durch $n \cd \bar 0 = \bar 0,\; n
\cd \bar 1 = \bar n$. Jede abelsche Gruppe $A$ ist
$\mathbb{Z}$-Modul durch $nx = \underset{\mbox{\scriptsize
n-mal}}{\underbrace{x+\dots+x}}$ und $(-n)x=n(-x)$ für $n \in \mathbb{N}_1, x \in A$
\item[(4)] Jedes Ideal in $R$ ist $R$-Modul.
\end{enumerate}}

\item Eine Abbildung $\varphi: M \ra M'$ von $R$-Moduln heißt
$\mathbf{R}$\emp{-Modulhomomorphismus} (oder
$\mathbf{R}$\emp{-linear}), wenn $\varphi$ Gruppenhomomorphismus ist
und für alle $x \in M, a \in R$ gilt: $\varphi(ax) = a \varphi(x)$

\item $Hom_R(M,M') \defeqr \{ \varphi: M \ra M' : \varphi$
$R$-linear$\}$ ist $R$-Modul durch
\newline$\left.\begin{array}{ll}
(\varphi_1 + \varphi_2)(x) &\defeqr \varphi_1(x) + \varphi_2(x) \\
(a \varphi)(x) &\defeqr a \varphi(x) \end{array}\right\}$ $\forall$
$\varphi_1,\varphi_2 \in Hom_R(M,M'),\;a\in R,\; x \in M$

\item Die $R$-Moduln bilden mit den $R$-linearen Abbildungen eine
Kategorie

\item Die Kategorien $\mathbf{\mathbb{Z}}$\emp{-Mod.} und \textbf{Abelsche
Gruppen} sind isomorph. denn: \[ \dots \varphi(nx) =
\varphi(x+\dots+x) = \varphi(x) + \dots + \varphi(x) = n
\varphi(x)\] ($\varphi: A \ra A'$ Gruppenhomomorphismus, $x \in A$,
$n \in \mathbb{N}$) $\Ra$ Jeder Gruppenhomomorphismus von abelschen
Gruppen ist $\mathbb{Z}$-linear.
\end{enum}
\end{DefBem}

\begin{DefBem}
Sei $M$ ein $R$-Modul.
\begin{enum}
\item Eine Untergruppe $U$ von $(M,+)$ heißt
$R$\emp{-Untermodul} von $M$, wenn $R\cd U \subseteq U$
ist, dh. wenn $U$ selbst $R$-Modul ist.

\item Ist $\varphi:M \ra M'$ $R$-linear, so sind $\Kern(\varphi$) und
$\Bild(\varphi$) Untermoduln von $M$ bzw. $M'$ (denn $\varphi(x) = 0
\Ra \varphi(ax)=0 \forall \dots$ und $a\varphi(x) = \varphi(ax)
\forall \dots$)

\item Sei $U \subseteq M$ Untermodul.
\newline Dann wird $M/U$ zu einem $R$-Modul durch $a \overline{x} \defeql
\overline{ax}$ (denn: Ist $x' \in \overline{x}$, also $x-x' \in U$, so ist $ax'
- ax = a(x' - x) \in U$)
\newline Die Restklassenabbildung $p:M \ra M/U,\;x\mapsto \bar x$
ist dann $R$-linear ($p(ax) = \overline{ax} = a \overline{x} = a p(x)$)
\end{enum}
\end{DefBem}

\begin{DefBem}
\begin{enum}
\item Für $X \subseteq M$ heißt \[\langle X \rangle \defeqr
\bigcap_{\substack{U \mbox{ \scriptsize Untermodul von } M \\ X
\subseteq U}}\;U\] der von $X$ erzeugte Untermodul.

\item $\ds \langle X \rangle = \{ \sum_{i=0}^n a_i x_i,\; a_i \in R,
x_i \in X, n \in \mathbb{N}\}$.

\item Eine Teilmenge $B \subseteq M$ heißt \emp{linear unabhängig}, wenn $\ds 0 =
\sum_{b\in B} a_b b$ mit $a_b \in R$ (wobei $a_b = 0$ für alle bis auf endlich viele
$b\in B$ gelten soll, damit die Summe $\ds \sum_{b\in B} a_b b$ wohldefiniert ist)
nur möglich ist mit $a_i = 0\;\forall i$.

\item Eine Teilmenge $B \subseteq M$ heißt \emp{Basis}, wenn jedes $x \in M$
eindeutig als Linearkombination $\ds 0 =
\sum_{b\in B} a_b b$ mit $a_b \in R$ (wobei $a_b = 0$ für alle bis auf endlich viele
$b\in B$ gelten soll) darstellbar ist.
\newline äquivalent: $B$ linear unabhängig und $\langle B \rangle =
M$

\item $M$ heißt \emp{frei}(er $R$-Modul), wenn $M$ eine Basis
besitzt.
\end{enum}
\bsp{\begin{enumerate}
\item[(1)] $R$ ist freier $R$-Modul mit Basis $1$ (oder einer
anderen Einheit)
\item[(2)] Für jedes $n \in \mathbb{N}$ ist $R^n = R \oplus \dots
\oplus R$ freier $R$-Modul mit Basis $e_1,\dots,e_n,\; e_i =
(0,\dots,0,1,0,\dots,0)$ (hier steht die $1$ an der $i$-ten Stelle).
\item[(3)] Ist $I \subseteq R$ Ideal, so ist $M \defeqr R/I =
\langle \{\bar{1} \} \rangle$. Für $I \neq \{0\}$ ist $R/I$ \emp{nicht}
frei. denn: Sei $\bar x \in M, a \in I \setminus \{0\} \Ra a \bar x
= \overline{ax} = \overline{0} \Ra$ in $M$ gibt es kein linear unabhängiges
Element (oder, um formal zu sein, keine linear unabhängige einelementige
Teilmenge).
\end{enumerate}}
\end{DefBem}



\chapter{Algebraische Körpererweiterungen}

\section{Algebraische und transzendente Elemente}

\begin{Def}
Sei $L$ ein Körper, $K \subset L$
Teilkörper.
\begin{enum}
\item Dann heißt $L$ Körpererweiterung von $K$. Schreibweise: $L/K$
Körpererweiterung.

\item $[L:K] =$ dim$_K$ $L$ heißt \emp{Grad} von $L$ über $K$

\item $L/K$ heißt \emp{endlich}, wenn $[L:K] < \infty$

\item $\alpha \in L$ heißt \emp{algebraisch} über $K$, wenn es ein
$0\neq f \in K[X]$ gibt mit $f(\alpha) = 0$

\item $\alpha \in L$ heißt \emp{transzendent} über $K$, wenn $\alpha$
nicht algebraisch über $K$ ist.

\item $L/K$ heißt \emp{algebraische Körpererweiterung}, wenn jedes
$\alpha \in L$ algebraisch über $K$ ist.
\end{enum}

\bsp{
\begin{enumerate}
\item[(1)] Für $a \in \mathbb{Q}$ und $n \geq 2$ ist $\sqrt[n]{a}$
algebraisch über $\mathbb{Q}$, da Nullstelle von $X^n - a$
\newline Summe und Produkt von solchen Wurzeln sind auch algebraisch
über $\mathbb{Q}$
\newline z.B.: $\sqrt{2} + \sqrt{3}$ ist Nullstelle von $X^4 - 10 X^2 + 1$, $i$ ist Nullstelle von $X^2+1$.

Klassische Frage: Hat jedes $f\in \mathbb Q[X]$ eine Nullstelle, die ein „Wurzelausdruck“ ist?.

\item[(2)] Sei $L = K(X) =$Quot$(K[X])$. Dann ist $X$ transzendent
über $K$. Das gleiche gilt für jedes $f \in K(X) \setminus K$

\item[(3)] In $\mathbb{R}$ gibt es sehr viele über $\mathbb{Q}$
transzendente Elemente. Da $\mathbb{Q}$ abzählbar ist, ist auch
$\mathbb{Q}[X]$ abzählbar, da jedes $f \in \mathbb{Q}[X]$ endlich viele
Nullstellen hat. Das heißt, es gibt nur abzählbar viele Elemente in
$\mathbb{R}$, die algebraisch über $\mathbb{Q}$ sind. $\mathbb{R}$
ist aber nicht abzählbar.
\end{enumerate} }
\end{Def}

\begin{DefBem}
\label{minimalpolynome}
Sei $L/K$ Körpererweiterung,
$\alpha \in L$, $\\\varphi_\alpha: K[X] \ra L,\;f\mapsto f(\alpha)$
Einsetzungshomomorphismus.
\begin{enum}
\item $\Kern(\varphi_\alpha)$ ist Primideal in $K[X]$

\sbew{$\Kern(\varphi_\alpha)$ ist Ideal, da $\varphi_\alpha$
Homomorphismus ist. Seien nun $f,g \in K[X]$ mit $f g \in$
$\Kern(\varphi_\alpha) \Ra
(fg)(\alpha) = f(\alpha)g(\alpha) = 0 \overset{L \text{ Körper}}{\Ra} f(\alpha)
= 0$ oder $g(\alpha) =0$}

\item $\alpha$ algebraisch genau dannn, wenn $\varphi_\alpha$ nicht injektiv ist.

\item Ist $\alpha$ algebraisch über $K$, so gibt es ein eindeutig
bestimmtes, irreduzibles und normiertes Polynom $f_\alpha \in K[X]$
mit $f_\alpha(\alpha) = 0$ und $\Kern(\varphi_\alpha) = (f_\alpha)$.
\newline $f_\alpha$ heißt \emp{Minimalpolynom} von $\alpha$.

\sbew{$K[X]$ ist Hauptidealring $\Ra \exists
\widetilde{f_\alpha}$ mit $\Kern(\varphi_\alpha) =
(\widetilde{f_\alpha})$. Wegen (a) ist $\widetilde{f_\alpha}$
irreduzibel, eindeutig bis auf Einheit in $K[X]$, also ein Element
aus $K^x \Ra \exists! \lambda \in K^x$, so dass $\lambda
\widetilde{f_\alpha} = f_\alpha$ normiert ist.}

\item $K[\alpha] \defeqr$ $\Bild(\varphi_\alpha) = \{f(\alpha):\;f
\in K[X]\} \subset L$ ist der kleinste Unterring von $L$, der $K$
und $\alpha$ enthält.

\item $\alpha$ ist transzendent $\lra$ $K[\alpha] \cong K[X]$

\sbew{
    $\alpha$ ist transzendent $\Ra$ $\Kern(\varphi_\alpha) = \{ 0 \} \Ra \varphi_\alpha$
    injektiv }

\item Ist $\alpha$ algebraisch über $K$, so ist $K[\alpha]$ ein
Körper und $[K[\alpha]:K] =$ deg$(f_\alpha)$

\sbew{Nach Homomorphiesatz ist $K[\alpha] \cong
K[X]/$$\Kern(\varphi_\alpha)$.
\newline $\Kern(\varphi_\alpha)$ ist maximales Ideal, da Primideal
$\neq (0)$ in $K[X]$ (siehe Bew. Satz \ref{Satz 9}, Beh. 2) $\Ra K[\alpha]$ ist
Körper.
\newline $f_\alpha(\alpha) = 0$, also $\alpha^n + c_{n-1} \alpha^{n-1} + \dots
+ c_1 \alpha + c_0 = 0$ mit $c_i \in K,\; c_0 \neq 0$ (da $f_\alpha$
irreduzibel), $\alpha(\alpha^{n-1} + \dots + c_1) = -c_0$. Ebenso:
$1,\alpha,\alpha^2,\dots,\alpha^{n-1}$ ist $K$-Basis von $K[\alpha]$, denn ist $\sum_{i=0}^{n-1}c_i\alpha^i = 0$ mit $c_i\in K$, so ist $\sum_{i=0}^nc_iX^i \in \Kern{\varphi_\alpha}$, also sind alle $c_i=0$, also sind $1,\alpha,\ldots,\alpha^{n-1}$ linear unabhängig. Sei $g(\alpha)\in K[\alpha]$ für ein $g\in K[X]$, und schreibe $g = q\cdot f_\alpha + r$ mit $\operatorname{Grad}(r)<n$. Also ist $g(\alpha)=r(\alpha)$ und $r=\sum_{i=0}^{n-1}c_iX^i$, also erzeugen $1,\alpha,\ldots,\alpha ^{n-1}$ ganz $R[\alpha]$.
}
\end{enum}
\end{DefBem}

\begin{Def}
Sei $L/K$ Körpererweiterung.
\begin{enum}
\item Für $A \subset L$ sei $K(A)$ der kleinste Teilkörper von $L$,
der $A$ und $K$ umfaßt; $K(A)$ heißt der \empind{von $\mathbf{A}$
erzeugte Teilkörper}{von A erzeugte Teilkörper} von $L$. Es ist
\[ K(A) = \left\{
\frac{f(\alpha_1,\dots,\alpha_n)}{g(\alpha_1,\dots,\alpha_n)} : n
\geq 1, \alpha_i \in A, f, g \in K[X_1,\dots,X_n], g \neq 0 \right\} \]

\item $L/K$ heißt \emp{einfach}, wenn es $\alpha \in L$ gibt mit $L
= K(\alpha)$

\item $L/K$ heißt \emp{endlich erzeugt}, wenn es eine endliche Menge
$\{\alpha_1,\dots,\alpha_n\} \subset L$ gibt mit $L =
K(\alpha_1,\dots,\alpha_n)$
\end{enum}
\end{Def}

\begin{Bem}
\label{3.1.4}
Sind $M/L$ und $L/K$ endlich, so auch $M/K$ und es gilt $[M:K]
= [M:L]\cd[L:K]$

\sbew{Sei $b_1,\dots,b_m$ $K$-Basis von $L$ und
$e_1,\dots,e_n$ $L$-Basis von $M \Ra B = \{e_i
b_j:\;i=1,\dots,n;j=1,\dots,m\}$ ist $K$-Basis von $M$.
\newline \textbf{denn}: $B$ erzeugt $M$: Sei $\alpha \in M$, $\ds
\alpha = \sum_{i=1}^n \lambda_i e_i$ mit $\lambda_i \in L$, $\ds
\lambda_i = \sum_{j=1}^m \mu_j b_j$ einsetzen $\Ra$ Behauptung.

$B$ linear unabhängig:

Ist $\sum \mu_{ij} e_i b_j = 0$, so ist für jedes feste $i$ :
$\displaystyle \sum_{j=1}^n \mu_{ij} b_j = 0$, da $e_i$ über $L$ linear unabhängig
sind. Da die $b_j$ linear unabhängig sind, sind die $\mu_{ij} = 0$}
% \newline 
\textbf{Notation}: $L/K$ Körpererweiterung, $\alpha \in L$,
$K[\alpha] = $ $\Bild(\varphi_\alpha) = \dots\\$ $K(\alpha)=$
Quot$(K[\alpha]) = K[\alpha]$, falls $\alpha$ algebraisch.
\end{Bem}

\begin{Bem}
\label{3.4}
Für eine Körpererweiterung $L/K$ sind
äquivalent:
\begin{enumerate}
\item[(i)] $L/K$ ist endlich.
\item[(ii)] $L/K$ ist endlich erzeugt und algebraisch.
\item[(iii)] $L$ wird von endlich vielen über $K$ algebraischen
Elementen erzeugt.
\end{enumerate}

\sbew{
\begin{description}
\item[(i) $\Ra$ (ii)] 
Jede $K$-Basis in $L$ ist auch Erzeugendensystem von $L/K$. Ist $\alpha\in L$ transzendent über $K$, so ist $K[\alpha] \cong K[X]$ ein unendlichdimensionaler $K$-Vektorraum in $L$, Widerspruch. Also sind alle Elemente in $L$ algebraisch.
\item[(ii) $\Ra$ (iii)] $\chk$
\item[(iii) $\Ra$ (i)] Induktion über die Anzahl $n$ der Erzeuger:

$n=1$: $[K(\alpha):K]=\operatorname{Grad}(f_\alpha)$ nach \ref{minimalpolynome} (f).

$n>1$: $K(\alpha_1,\ldots,\alpha_n) = K(\alpha_1,\ldots,\alpha_{n-1})(\alpha_n)$, $K'\defeqr K(\alpha_1,\ldots,\alpha_{n-1})/K$ ist endlich nach Induktionsvorraussetzung und $L/K'$ ist endlich nach Fall 1, also folgt aus \ref{3.1.4} $L/K$ ist endlich.
\end{description}
}
\end{Bem}

\bsp{$\cos \frac{2\pi}{n}$ ist für jedes $n \in \mathbb{Z} \setminus
\{0\}$ algebraisch über $\mathbb{Q}$.

\textbf{denn}: \[\cos \frac{2\pi}{n} = \Re\left(e^{\frac{2\pi
i}{n}}\right) = \frac{1}{2}\left(e^{\frac{2\pi i}{n}} +
\overline{e^{\frac{2\pi i}{n}}}\right) = \frac{1}{2}\left(e^{\frac{2\pi i}{n}}
+ e^{-\frac{2\pi i}{n}}\right)\]

$e^{\frac{2\pi\imath}{n}}$ ist Nullstelle von $X^n - 1$, also
algebraisch (über $\mathbb{Q}$) $\Ra K =
\mathbb{Q}\left(e^{\frac{2\pi\imath}{n}}\right)$ ist endliche
Körpererweiterung von $\mathbb{Q}$, $\cos \frac{2\pi}{n} \in K
\overset{3.5(i)\Ra(ii)}{\Ra} \cos \frac{2\pi}{n}$ ist algebraisch.

$\mathbb{Q} \subset \mathbb{Q}\left(\cos \frac{2\pi}{n}\right)
\subsetneq K\;(n\geq3)$}


\begin{Bem}
\label{3.5}
Seien $K\subset L \subset M$ Körper.
Sind $M/L$ und $L/K$ algebraisch, so auch $M/K$
\sbew{ Sei $\alpha \in M$, $f_\alpha = \displaystyle \sum_{i=0}^n c_i X^i
\in L[X]$ mit $f_\alpha(\alpha) = 0$. Dann ist $\alpha$ algebraisch über
$K(c_0,\dots,c_n) \defeql L' \subset L, L'$ ist endlich erzeugt über 
$K \overset{\ref{3.4}}{\Ra} L'/K$ endlich.
Außerdem ist $L'(\alpha)/L'$ endlich. $\overset{(b)}{\Ra} L'(\alpha)/K$
endlich $\Ra \alpha$ algebraisch über $K$.}
\end{Bem}


\section{Algebraischer Abschluss}

\begin{Prop}[Kronecker]
\label{Prop:Kronecker}
Sei $K$ Körper, $f\in K[X]$, $f$ nicht konstant.

Es gibt eine endliche Körpererweiterung $L/K$, so dass $f$ in
$L$ eine Nullstelle hat. Genauer: $[L:K] \le \operatorname{Grad} f$.
\end{Prop}

\sbew{\OE $f$ irreduzibel. Setze $L \defeqr K[X]/(f)$. $L$ ist
Körper, da $(f)$ maximales Ideal ist. $\alpha = \bar X$ = Klasse von $X$ in
$L$ ist Nullstelle von $f$. Genauer: $f$ ist das Minimalpolynom von $\alpha$.
}

\begin{Bem}
Ist $f\in K[X]\setminus\{0\}$ und $\alpha\in K$ mit $f(\alpha)=0$, dann ist $X-\alpha$ ein Teiler von $f$.
\end{Bem}

\sbew{
$\{f\in K[X]: f(\alpha)=0\}$ ist ein Ideal im  Hauptidealring $K[X]$ und $X-\alpha$ sein Erzeuger.
}

\begin{BemDef}
Sei $K$ Körper, $f\in K[X]\setminus K$
\begin{enum}
\item Es gibt eine endliche Körpererweiterung $L/K$, so dass $f$
über $L$ in Linearfaktoren zerfällt.
\newline
\sbew{Induktion über $n =$ deg$(f)$:
\begin{description}
\item[$n=1$] $\chk$
\item[$n\geq 1$] $L_1$ wie in Proposition \ref{Prop:Kronecker}. Dann ist $f(X) = (X - \alpha)
\cd f_1(X)$ in $L_1[X]$, deg$(f_1) = n-1$. Also gibt es $L_2/L_1$,
so dass $f_1(X) =\displaystyle \prod_{i=1}^{n-1} (X-\alpha_i)$ mit $\alpha_i \in
L_2$. Dabei ist $L_2/L_1$ endlich, $L_1/K$ endlich, also $L_2/K$
endlich.
\end{description}}

\item $L/K$ heißt \emp{Zerfällungskörper} von $f$, wenn $f$ über $L$
in Linearfaktoren zerfällt, und $L$ über $K$ von den Nullstellen von
$f$ erzeugt wird.
\item Es gibt einen Zerfällungskörper $Z(f)$.

\sbew{
Induktion über den Grad und die Anzahl über die irreduziblen Faktoren:

\OE Sei $f$ irreduzibel. Sei $L_1\da K[X]/(f)$ und $\alpha \da \bar X\in L$. Dann ist $L_1=K(\alpha)$ und $f=(X-\alpha)\cdot g$ in $L_1[X]$. Nach Induktionsvorraussetzung gibt es einen Zerfällungskörper $Z(g)$ von $g$ über $L_1$, also wird $Z(g)$ über $K$ von $\alpha$ und den Nullstellen von $g$ erzeugt.
}

\item Ist $f$ irreduzibel und $n =$ deg$(f)$, so ist $[Z(f) : K]
\leq n!$
\newline
\sbew{
In Proposition \ref{Prop:Kronecker} ist $[L:K] = n =$ deg$(f)$ und $f = (X-\alpha) \cd f_1$
mit deg$(f_1) = n-1$. Mit Induktion folgt die Behauptung. }
\end{enum}
\end{BemDef}

\bsp{\begin{enumerate}
\item[(1)] $f \in K[X]$ irreduzibel vom Grad 2. Dann ist $L=K[X]/(f)$
der Zerfällungskörper von $f$. $f(X) = (X - \alpha)(X-\beta)$,
$\alpha,\beta \in L$. Ist $f(X) = X^2 + pX + q$, so ist $\alpha +
\beta = -p$

\item[(2)] $f(X) = X^3 - 2 \in \mathbb{Q}[X]$.
Sei $\alpha = \sqrt[3]{2} \in \mathbb{R}$ Nullstelle von $f$. In
$\mathbb{Q}(\alpha)$ liegt keine weitere Nullstelle von $f$, da
$\mathbb{Q}(\alpha) \subset \mathbb{R}$
\[X^3 - 2 = (X-\alpha)\underset{\mbox{\scriptsize{irreduzibel über}
}\mathbb{Q}(\alpha)}{\underbrace{(X^2 + \alpha X + \alpha^2)}} \Ra
[Z(f) : \mathbb{Q}] = 6\]

\item[(3)] $K = \mathbb{Q}$, $p$ Primzahl, $f(X) = X^p - 1 = (X
-1)\underset{f_1}{\underbrace{(X^{p-1} + X^{p-2} +\dots + X + 1)}}$ $\\f_1$ irreduzibel
(siehe \ref{Bsp 2.27}). $\\\\L \defeqr \mathbb{Q}[X]/(f_1) \defeql
\mathbb{Q}(\zeta_p)$; $(\zeta_p^k)^p = \zeta_p^{pk} = 1$;
$k=1,\dots,p-1$

$\Ra \mathbb{Q}(\zeta_p) = Z(f)$ \end{enumerate} }

\begin{DefBem}
\label{3.7}
Sei $K$ ein Körper.
\begin{enum}
\item $K$ heißt \emp{algebraisch abgeschlossen}, wenn jedes
nichtkonstante Polynom $f \in K[X]$ in $K$ eine Nullstelle hat.
\item Die folgenden Aussagen sind äquivalent:

\begin{enumerate}
\item[(i)] $K$ ist algebraisch abgeschlossen
\item[(ii)] Jedes $f \in K[X]\setminus K$ zerfällt über $K$ in Linearfaktoren
\item[(iii)] $K$ besitzt keine echte algebraische
Körpererweiterung.
\end{enumerate}
\end{enum}

\sbew{
\begin{description}
\item[(i)$\Ra$(ii)] Induktion über den Grad von $f$.
\item[(ii)$\Ra$(iii)] Angenommen $L/K$ algebraisch, $\alpha \in L
\setminus K$. Dann sei $f_\alpha \in K[X]$ das Minimalpolynom
von $\alpha$; $f_\alpha$ ist irreduzibel und zerfällt in Linearfaktoren
$\Ra$ deg$(f) = 1 \;\blitzb$
\item[(iii)$\Ra$(ii)] Sei $f\in K[X]$ irreduzibel, $L\da K[X]/(f)$, dann folgt aus der Voraussetzung $L=K$ und damit $\operatorname{Grad}f =1$.
\end{description}
}
\end{DefBem}

\begin{Satz}
Zu jedem Körper $K$ gibt es eine algebraische Körpererweiterung
$\bar K/K$, so dass $\bar K$ algebraisch abgeschlossen ist. $\bar K$
heißt \emp{algebraischer Abschluss} von $K$.

\sbew{

\textbf{Hauptschritt}: Es gibt algebraische Körpererweiterung
$K'/K$, so dass jedes nichtkonstante $f\in K[X]$ in $K'$ eine
Nullstelle hat.

\textbf{Dann}: sei $K'' \defeqr (K')'$ und weiter $K^i \defeqr
(K^{i-1})',\;i\geq 3$; Es ist $K^i \subset K^{i+1}$.

$L\defeqr \displaystyle \bigcup_{i\geq 1} K^i$. Es gilt:
\begin{enumerate}
\item[(i)] $L$ ist Körper: $a+b \in L$ für $a \in K^i, b\in K^j$,
da \OE: $i \leq j \Ra a$ auch in $K^j$

\item[(ii)] $L$ ist algebraisch über $K$: jedes $\alpha \in L$ liegt
in einem $K^i,\;K^i$ ist algebraisch über $K$.

\item[(iii)] $L$ ist algebraisch abgeschlossen.
\newline \textbf{denn}: Sei $f \in L[X],\;f = \displaystyle \sum_{i=0}^n c_i
X^i,\; c_i \in L$. Also gibt es $j$ mit $c_i \in K^j$ für
$i=0,\dots,n \Ra f$ hat Nullstelle in $(K^j)' = K^{j+1} \subset L
\Ra$ Behauptung \end{enumerate}

\textbf{Bew.(Hautpschritt)}: Für jedes $f \in K[X] \setminus K$ sei $X_f$
ein Symbol. $\mathcal{X} \defeqr \{X_f : f \in K[X] \setminus K\}$, $R
\defeqr K[\mathcal{X}]$, $I$ sei das von allen $f(X_f)$ in $R$
erzeugte Ideal.

Behauptung: $I\ne R$. 
\newline Dann gibt es ein maximales Ideal $\mathfrak{m} \subset R$ mit $I \subset
\mathfrak{m}$, $K'\defeqr R/\mathfrak{m}$, $K'$ ist Körper, $K'/K$
ist algebraisch,
\newline \textbf{denn}: $K'$ wird über $K$ erzeugt von den $\bar X_f \in
\mathcal{X}$ und $f(\bar X_f) = 0$ in $K'$, weil $f(\bar X_f) \in I \subset
\mathfrak{m}$. $f$ hat in $K'$ die Nullstellen (Klasse von) $\bar X_f$.

\textbf{Beweis der Behauptung}
Angenommen $I = R$, also $1 \in I$. Dann gibt es $n \geq
1, f_1,\dots,f_n \in K[X] \setminus K$ und $g_1,\dots,g_n \in R$ mit $1 =
\ds \sum_{i=1}^n g_i f_i (X_{f_i})$. Sei $L/K$ Körpererweiterung, in
der jedes $f_i, i=1,\dots,n$ Nullstelle $\alpha_i$ hat (z.B. der
Zerfällungskörper von $f_1\cd\dots\cd f_n$).

Setze nun $\alpha_i$ für $X_{f_i}$ ein ($i=1,\dots,n$) (und $42$ für
alle anderen $X_f$). Dann ist $1 = \ds\sum_{i=1}^n g_i (\alpha_1,
\dots, \alpha_n, 42, \dots) \cd
\underset{=0}{\underbrace{f_i(\alpha_i)}} = 0 \blitzb$
}
\end{Satz}

\section{Fortsetzung von Körperhomomorphismen}

Sei $f(x)=x^2-2$, $K=\mathbb Q$, $L=\mathbb Q[X]/(f)$ und $\alpha=\bar X$, also $f(\alpha)=0$. Es gibt zwei Einbettungen von $L$ in $\mathbb R$: Schreibe $x\in L$ als $x=a + b\alpha$ mit $a,b\in \mathbb Q$ (dies ist eindeutig), dann sind $\varphi_1(x)\defeqr a+ b\sqrt 2$ und $\varphi_2(x)\defeqr a-b\sqrt 2$ Homomorphismen $L\to \mathbb R$.

\begin{Prop}
\label{3.8}
Sei $L = K(\alpha)$, $K$ Körper (also einfache Körpererweiterung).
Sei $\alpha$ algebraisch über K, $f = f_\alpha \in K[X]$ das Minimalpolynom.
Sei $K'$ Körper und $\sigma: K \ra K'$ ein Körperhomomorphismus. Sei
$f^\sigma$ das Bild von $f$ in $K'[X]$ unter dem Homomorphismus
$K[X] \ra K'[X], \;\sum a_i X^i \mapsto \sum \sigma(a_i) X^i$. Dann
gilt:

\begin{enum}
\item Ein Homomorphismus $\wt\sigma:L\to K'$ heißt \emp{Fortsetzung} von $\sigma$, wenn $\wt\sigma(a)=\sigma(a)$ für alle $a\in K$ gilt.
\item Ist $\wt{\sigma}: L\ra K'$ Fortsetzung von $\sigma$, so ist $\wt{\sigma}(\alpha)$ Nullstelle von $f^\sigma$.
\item Zu jeder Nullstelle $\beta$ von $f^\sigma$ in $K'$ gibt es
genau eine Fortsetzung $\wt{\sigma}:L \ra K'$ von $\sigma$ mit
$\wt{\sigma}(\alpha) = \beta$.

\end{enum}

\bew{}{\item[(b)] $f^\sigma(\wt{\sigma}(\alpha)) =
f^{\wt{\sigma}}(\wt{\sigma}(\alpha)) = \wt{\sigma}(f(\alpha)) = 0$

\item[(c)] Eindeutigkeit: $\chk$ $\wt{\sigma}$ ist auf den
Erzeugern von $L$ festgelegt.

Existenz: \[\begin{array}{llll}
            \varphi: K[X] &\ra K',& X &\mapsto \beta  \\
            && \underset{=g}{\sum a_i
            X^i} &\mapsto \sum \sigma(a_i) \beta ^i =
            g^\sigma(\beta) \end{array}\]

$\Ra \varphi(f) = f^\sigma(\beta)$
$\overset{\mbox{\scriptsize Hom.satz}}{\Ra} \varphi$ induziert
$\wt{\sigma}: \underset{=L}{K[X]}/(f) \ra K'$ }
\end{Prop}

\begin{Folg}
Sei $f \in K[X] \setminus K$. Dann ist der
Zerfällungskörper $Z(f)$ bis auf Isomorphie eindeutig.

\sbew{Seien $L,L'$ Zerfällungskörper, $L =
K(\alpha_1,\dots,\alpha_n)$, $\alpha_i$ die Nullstelle von $f$. Sei
weiter $\beta_1 \in L'$ Nullstelle von $f$. Nach \ref{3.8} gibt es
$\sigma: K(\alpha_1) \ra L'$ mit $\sigma_{|K} = $id$_K$ und
$\sigma(\alpha_1) = \beta_1$ und $\tau: K(\beta_1) \ra L$ mit
$\tau(\beta_1) = \alpha_1$ und $\tau_{|K} =$ id$_K$.

$\tau \circ \sigma =$ id$_{K(\alpha_1)}$, $\sigma \circ \tau =$
id$_{K(\beta_1)} \Ra K(\alpha_1) \cong K(\beta_1)$

Mit Induktion über $n$ folgt die Behauptung.}
\end{Folg}

\begin{Bem}
\label{3.10}
Sei $L/K$ algebraische Körpererweiterung,
$\bar K$ ein algebraisch abgeschlossener Körper. $\sigma: K \ra \bar
K$ ein Homomorphismus. Dann gibt es eine Fortsetzung $\wt{\sigma}: L
\ra \bar K$.

\sbew{ Ist $L/K$ endlich, so folgt die Aussage aus \ref{3.8}.
Für den allgemeinen Fall sei $\mathcal{M} \defeqr \{(L', \tau):
L'/K$ Körpererw., $L' \subseteq L, \tau:L'\ra \bar K$ Fortsetzung
von $\sigma\}$, $\mathcal{M}\neq \emptyset: (K,\sigma) \in \mathcal{M}$

$\mathcal{M}$ ist geordnet durch $(L_1, \tau_1) \subseteq (L_2, \tau_2) : \lra
L_1 \subseteq L_2$ und $\tau_2$ Fortsetzung von $\tau_1$. Sei
$\mathcal{N} \subset \mathcal{M}$ totalgeordnet $\ds \wt{L} \defeqr
\bigcup_{(L',\tau) \in \mathcal{N}} L'$.

$\wt{L}$ ist Körper, $\wt{L} \subseteq L$, $\wt{\tau}: \wt{L} \ra
\bar K$, $\wt{\tau}(x) = \tau(x)$, falls $x\in L'$ und $(L',\tau) \in
\mathcal{N}$.

Wohldefiniertheit: ist $x \in L''$, so ist \OE $(L',\tau) \subseteq
(L'', \tau'')$ und damit $\tau''(x) = \tau(x)$.
$\Ra (\wt{L},\wt{\tau})$ ist obere Schranke
$\overset{Zorn}{\Ra} \mathcal{M}$ hat maximales Element $(\wt{L},\wt{\sigma})$


\textbf{Zu zeigen}: $\wt{L} = L$. Sonst sei $\alpha \in L\setminus
\wt{L}$ und $\sigma'$ Fortsetzung von $\wt{\sigma}$ auf
$\wt{L}(\alpha)$ (nach \ref{3.8})

$\Ra (\wt{L}(\alpha), \sigma') \in \mathcal{M}$ und $(\wt{L},
\wt{\sigma}) \subsetneq (\wt{L}(\alpha), \sigma') \;\blitzc$}
\end{Bem}

\begin{Folg}
Für jeden Körper $K$ ist der algebraische
Abschluss $\bar K$ bis auf Isomorphie eindeutig bestimmt.

\sbew{ Seien $\displaystyle \underset{\substack{\supset \\
K}}{\bar{K}} \underset{\overset{\mbox{\scriptsize
id}}{\longrightarrow}}{\mbox{ und }} \underset{\substack{\supset
\\K}}{C}$ algebraische Abschlüsse von $K$. Nach Proposition
\ref{3.10} gibt es Körperhomomorphismus $\sigma:\bar K \ra C$, der
id$_K$ fortsetzt. Dann ist $\sigma(\bar K)$ auch algebraisch
abgeschlossen: ist $\underset{=\sum a_i X^i}{f} \in \sigma(\bar
K)[X] \Ra \underset{= \sum \sigma^{-1}(a_i) X^i}{f^{\sigma^{-1}}}
\in \bar K[X]$ hat Nullstelle $\alpha \in \bar K$. $\Ra
\sigma(\alpha)$ ist Nullstelle von $f$: 
\newline $\sum \sigma^{-1}(a_i)\alpha^i = 0 \Rightarrow 0 = \sigma(\sum
\sigma^{-1} (a_i) \alpha^i) = \sum a_i \sigma(\alpha^i) = \sum a_i
\sigma(\alpha)^i$

$C$ ist algebraisch über $K$, also erst recht über $\sigma(\bar K)
\overset{\ref{3.7}}{\Ra} \sigma(\bar K) = C$}
\end{Folg}

\begin{DefBem}
\label{3.12}
Seien $L/K, L'/K$ Körpererweiterungen von $K$.

\begin{enum}
\item
\[ \operatorname{Hom}_K(L,L') \defeqr \{\sigma:L\ra L'\text{ Körperhomomorphismus},\, \sigma_{|K} = \operatorname{id}_K\} \]

\[ \operatorname{Aut}_K(L) \defeqr \{\sigma: L\to L \text{ Körperautomorphismus},\, \sigma|_K = \operatorname{id}_K\}\]
\item Ist $L/K$ endlich, $\bar K$ algebraischer Abschluss von $K$, so
ist $|$Hom$_K(L,\bar K)| \leq [L:K]$.

\sbew{Sei $L = K(\alpha_1,\dots,\alpha_n)$, $\alpha_i$
algebraisch über $K$. Induktion über $n$:
\begin{description}
\item[$n=1$] Sei $f\in K[X]$ das Minimalpolynom von $\alpha_1$. Für
jedes $\sigma \in$ Hom$_K(L,\bar K)$ ist $\sigma(\alpha_1)$ Nullstelle
von $f^{\sigma} \in \bar K[X]$. Durch $\sigma_{|K} =$ id$_K$ und
$\sigma(\alpha_1)$ ist $\sigma$ eindeutig bestimmt. $\Ra
|$Hom$_K(L,\bar K)| = |$Nullstellen von $f^\sigma | \leq$
deg$(f^\sigma) = [L:K]$

\item[$n>1$] Sei $L_1 = K(\alpha_1,\dots,\alpha_{n-1}), f \in L_1[X]$
das Minimalpolynom von $\alpha_n$ über $L_1$. Für $\sigma \in $
Hom$_K(L,\bar K)$ ist $\sigma(\alpha_n)$ Nullstelle von $f^{\sigma_1}
\in \bar K[X]$ mit $\sigma_1 = \sigma_{|L_1} \Ra |$Hom$_K(L,\bar K)|
\leq |$Hom$_K(L_1,\bar K)| \cd$ deg$(f) \overset{\mbox{\scriptsize
IV}}{\leq} [L_1 : K] \cd [L:L_1] \overset{\ref{3.5}(b)}{=} [L:K]$
\end{description}
}
\end{enum}
\end{DefBem}

\section{Separable Körpererweiterungen}

\begin{DefBem}
\label{3.13}
Sei $L/K$ algebraische
Körpererweiterung und $\bar K$ algebraischer Abschluss von $K$.

\begin{enum}
\item $f \in K[X]$ heißt \emp{separabel}, wenn $f$ in $\bar K$
keine mehrfache Nullstelle hat (also deg$(f)$ verschiedene
Nullstellen).

\item $\alpha \in L$ heißt separabel, wenn das Minimalpolynom von
$\alpha$ über $K$ separabel ist.

\item $L/K$ heißt separabel, wenn jedes $\alpha \in L$ separabel
ist.

\item $f \in K[X] \setminus K$ ist genau dann separabel, wenn ggT$(f,f') =
1$. Dabei ist für $f = \ds\sum_{i=0}^n a_i X^i $ die \emp{Ableitung} definiert durch $f' \defeqr \sum_{i=1}^n i a_i X^{i-1}$

\sbew{Sei $f(X) = \ds\prod_{i=1}^n
(X-\alpha_i),\;\alpha_i \in \bar K \Ra f'(X) = \sum_{i=1}^n \prod_{j
\neq i} (X - \alpha_j)$ nach Definition ist $f$ separabel $\lra
\alpha_i \neq \alpha_j$ für $i \neq j$.

\textbf{Beh.}: $\alpha_1 = \alpha_i$ für ein $i \geq 2 \lra
(X-\alpha_1) \mid f'$

Aus der Behauptung folgt: $f$ separabel $\lra$ $f$ und $f'$
teilerfremd in $\bar K[X]$. Ist das so, dann ist ggT$(f,f') = 1$
(teilerfremd in $K[X]$). Ist umgekehrt ggT$(f,f') = 1$, so gibt es
$g,h \in K[X]$ mit $1 = g f + h f'$.

Das stimmt dann auch in $\bar K[X]$, also sind $f$ und $f'$ in $\bar
K[X]$ teilerfremd.

\textbf{Bew. der Beh.}: $(X-\alpha_1)$ teilt $\ds\prod_{j \neq i}
(X-\alpha_j)$, falls $i\neq 1$. Also gilt $X-\alpha_1$ teilt $f'
\lra X - \alpha_1$ Teiler von $\ds\prod_{j\neq 1} (X- \alpha_j) \lra
\alpha_1 = \alpha_j$ für ein $j \neq 1$.}

\item Ist $f \in K[X]$ irreduzibel, so ist $f$ separabel genau dann, wenn $f'
\not= 0$ (Nullpolynom) ist.

\sbew{Ist $f' = 0$, so ist ggT$(f,f') = f \neq 1$

Ist $f' \neq 0$, so ist deg $f'$ < deg $f$; ist $f$ irreduzibel und
$\alpha \in \bar K$ Nullstelle von $f$, so ist $f$ das
Minimalpolynom von $\alpha \overset{f' \neq 0}{\Ra} \alpha$ nicht
Nullstelle von $f' \Ra$ ggT$(f,f')=1$}
\end{enum}
\end{DefBem}

\begin{Folg}
Ist char$(K) = 0$, so ist jede algebraische
Körpererweiterung separabel.
\end{Folg}

\begin{Bspe}
Sei $p$ Primzahl, $K = \mathbb{F}_p(t)$ =
Quot$(\mathbb{F}_p[t])$. Sei $f(X) = X^p-t \in K[X]$.
\newline $f'(X) = pX^{p-1} = 0$, $t \in \mathbb{F}_p[t]$ ist Primelement
$\overset{\mbox{\scriptsize Eisenstein}}{\Ra} f$ irreduzibel in
($\mathbb{F}_p[t])[X] \overset{\mbox{\scriptsize \ref{2.29}}}{\Ra} f$
irreduzibel in $K[X]\\$
\newline $f(X) = X^p - a \in \mathbb{F}_p
\Ra f' = 0$, $f$ ist nicht irreduzibel, da $f$ Nullstelle in
$\mathbb{F}_p$ hat, dh. es gibt ein $b \in \mathbb{F}_p$ mit $b^p =
a$.
\newline Denn: $\varphi: \mathbb{F}_p \ra \mathbb{F}_p, b
\mapsto b^p$ ist Körperhomomorphismus! (denn $(a+b)^p = a^p + b^p$)
\end{Bspe}

\begin{Prop}
Sei char$(K) = p > 0$, $f \in K[X]$ irreduzibel, $\bar K$ ein algebraischer Abschluss von $K$.

\begin{enum}
\label{3.16}
\item Es gibt ein separables irreduzibles Polynom $g \in K[X]$, so
dass $f(X) = g(X^{p^r})$ für ein $r \geq 0$.

\item Jede Nullstelle von $f$ in $\bar K$ hat Vielfachheit $p^r$.
\end{enum}
\sbew{ Sei $f$ nicht separabel, $f = \ds\sum_{i=0}^n a_i X^i$, $f' =\ds\sum_{i=1}^n i a_i
X^{i-1} = 0 \Ra i a_i = 0$ für $i=1,\dots,n \Ra a_i = 0$, falls $i$
nicht durch $p$ teilbar $\Ra f$ ist Polynom in $X^p$, dh. $f =
g_1(X^p)$. Mit Induktion folgt die Behauptung.}
\end{Prop}

\begin{PropDef}
\label{Satz 13}
Sei $L/K$ endliche Körpererweiterung, $\bar K$
algebraischer Abschluss von $L$.
\begin{enum}
\label{Satz 13a}\item $[L:K]_s \defeqr |$Hom$_K(L,\bar K)|$ heißt
\emp{Separabilitätsgrad} von $L$ über $K$.

\label{Satz 13b}\item Ist $L'$ Zwischenkörper von $L/K$, so ist
$[L:K]_s = [L:L']_s \cd [L':K]_s$

\label{Satz 13c}\item $L/K$ ist separabel $\lra [L:K] = [L:K]_s$

\label{Satz 13d}\item Ist char$(K) = p > 0$, so gibt es ein $r \in
\mathbb{N}$ mit $[L:K] = p^r \cd [L:K]_s$
\end{enum}
\end{PropDef}

\bew{}{\item[(b)] Sei Hom$_K(L',\bar K) =
\{\sigma_1,\dots,\sigma_n\}$, Hom$_{L'}(L,\bar K) =
\{\tau_1,\dots,\tau_m\}$. Sei $\wt{\sigma_i}: \bar K \ra \bar K$
Fortsetzung von $\sigma_i,\;i=1,\dots,n$. Dann ist $\wt{\sigma_i}
\in$ Aut$_K(\bar K)$.

\textbf{Beh.}: \begin{description}
\item[(1)] Hom$_K(L,\bar K) = \{\wt{\sigma_i} \circ \tau_j:\;
i=1,\dots,n,j=1,\dots,m\}$
\item[(2)] $\wt{\sigma_i} \circ \tau_j = \wt{\sigma_{i'}} \circ
\tau_{j'} \lra i = i'$ und $j = j'$.
\end{description}

Aus (1) und (2) folgt (b).

\textbf{Bew.(1)}: ''$\supseteq$'' $\chk$ ''$\subseteq$'': Sei
$\sigma \in$ Hom$_K(L,\bar K)$. Dann gibt es ein $i$ mit
$\sigma_{|L'} = \sigma_i \Ra \wt{\sigma_i}^{-1} \circ \sigma_{|L'} =
\mbox{id}_{L'} \Ra \exists\; j$ mit $\wt{\sigma_i}^{-1} \circ \sigma = \tau_j
\Ra \sigma = \wt{\sigma_i} \circ \tau_j$.

\textbf{Bew.(2)}: Sei $\wt{\sigma_i} \circ \tau_j = \wt{\sigma_{i'}}
\circ \tau_{j'} \Ra
\underset{=\sigma_i}{\underbrace{\wt{\sigma_i}_{|L'}}} =
\underset{\sigma_{i'}}{\underbrace{\wt{\sigma_{i'}}_{|L'}}} \Ra i =
i' \Ra \tau_j = \tau_{j'} \Ra j = j'$.

\item[(c)] ''$\Ra$'': Sei $L = K(\alpha_1,\dots,\alpha_n)$. Induktion
über $n$:
\begin{description}
\item[n=1] $L=K(\alpha)$, $f = f_\alpha \in K[X]$ das Minimalpolynom
von $\alpha$ über $K \Ra [L:K]_s \overset{\ref{3.12}}{=}
|\{$Nullstellen von $f$ in $\bar K\}| =$ deg $f = [L:K]$.

\item[n>1] $L_1 \defeqr K(\alpha_1,\dots,\alpha_{n-1})$, $f \in
L_1[X]$ das Minimalpolynom von $\alpha_n$. Zu jedem $\sigma_1 \in$
Hom$_K(L_1,\bar K)$ und jeder Nullstelle von $f$ in $\bar K$ gibt es
genau eine Fortsetzung $\wt{\sigma_1}:L\ra \bar K$.

$\overset{f \mbox{ separabel}}{\Ra}[L:K]_s = |$Hom$_K(L,\bar K)| =$ deg$(f) \cd
|$Hom$_K(L_1,\bar K)| = [L:L_1] \cd [L_1:K]_s
\overset{\mbox{\scriptsize IV}}{=} [L:L_1]\cd[L_1:K] = [L:K]$.

\end{description}
''$\Leftarrow$'': Ist char$(K) = 0$, so ist $L/K$ separabel.
Sei also char$(K) = p > 0$ und $\alpha \in L$; $f \in K[X]$ das
Minimalpolynom von $\alpha$. Nach \ref{3.16} gibt es $r \geq 0$ und
ein separables, irreduzibles Polynom $g \in K[X]$ mit $f(X) =
g(X^{p^r}) \Ra [K(\alpha):K]_s = |\{$Nullstellen von $g$ in $\bar
K\}| \overset{g \mbox{ \scriptsize separabel}}{=}$ deg$(g)\;(\ast)
\Ra [K(\alpha) : K] =$ deg$(f) = p^r \cd$ deg$(g) = p^r \cd
[K(\alpha) : K]_s \Ra [L:K] = [L:K(\alpha)] \cd [K(\alpha) : K] \geq
[L:K(\alpha)]_s \cd p^r [K(\alpha):K]_s \overset{(b)}{=} [L:K]_s
\overset{\mbox{\scriptsize Voraussetzung}}{\Ra} p^r = 1 \Ra g = f
\Ra \alpha$ separabel.

\item[(d)] folgt aus $(\ast)$
}

\begin{Satz}[Satz vom primitiven Element]
\label{Satz 14}
Jede endliche separable
Körpererweiterung $L/K$ ist einfach, also gibt es $\alpha\in L$ mit $L=K(\alpha)$. $\alpha$ heißt \emp{primitives Element}.

\sbew{Ist $K$ endlich, so folgt aus \ref{3.17}, dass $L^x$ zyklische
Gruppe ist. Ist $L^x = \langle \alpha \rangle$, so ist $L =
K[\alpha]$.

Sei also $K$ unendlich, $L=K(\alpha_1,\dots,\alpha_r)$. \OE: $r=2$,
also $L=K(\alpha,\beta)$. Sei $\bar K$ algebraischer Abschluss von
$L$, $[L:K] = n$. Sei Hom$_K(L,\bar K) = \{\sigma_1, \dots,
\sigma_n\}$ (\ref{Satz 13}(c)).

Sei $g(X) \defeqr \ds\prod_{1 \leq i < j \leq n} (\sigma_i(\alpha) -
\sigma_j(\alpha)) + (\sigma_i(\beta) - \sigma_j(\beta))X) \in \bar
K[X]$, $g \neq 0$, denn aus $\sigma_i(\alpha) = \sigma_j(\alpha)$
und $\sigma_i(\beta) = \sigma_j(\beta)$ folgt $\sigma_i = \sigma_j$.
Da $K$ unendlich ist, gibt es $\lambda \in K$ mit $g(\lambda) \neq
0$.

\textbf{Beh.}: $\gamma \defeqr \alpha + \lambda \beta \in L$ erzeugt
$L$ über $K$.

\textbf{denn}: Sei $f \in K[X]$ das Minimalpolynom von $\gamma$ über
$K$. Für jedes $i$ ist $f(\sigma_i(\gamma)) \overset{{\sigma_i}_{|K} = id_K}{=} \sigma_i(f(\gamma))$.
Angenommen, $\sigma_i(\gamma) =
\sigma_j(\gamma)$ für ein $i \neq j$. Dann wäre $(\sigma_i(\alpha) +
\sigma_i(\beta) \lambda) - (\sigma_j(\alpha) + \sigma_j(\beta)
\lambda) = 0 \Ra g(\lambda) = 0\;\blitzb \Ra f$ hat mindestens $n$
Nullstellen $\Ra$ deg$(f) = [K(\gamma) : K] \geq n = [L:K]$, da
$\gamma \in L$, folgt $K(\gamma) = L$.}
\end{Satz}

\section{Endliche Körper}

\begin{Prop}
\label{3.17}
Ist $K$ ein Körper, so ist jede endliche
Untergruppe von $(K^x,\cd)$ zyklisch.

\sbew{Sei $G \subseteq K^x$ endliche Untergruppe, $a \in G$ ein
Element maximaler Ordnung. Sei $n=$ord$(a)$, $G_n \defeqr \{b\in G:$
ord$(b) \mid n$\}.

\textbf{Beh.}: $G_n = \langle a \rangle$

\textbf{denn}: jedes $b\in G_n$ ist Nullstelle von $X^n -1$. Diese
sind $1,a,a^2,\dots,a^{n-1} \Ra |G_n| = |\langle a \rangle| = n$.
\newline Nach Folgerung \ref{Satz 3} ist $G \cong \ds\bigoplus_{i=1}^r \mathbb{Z}/a_i
\mathbb{Z}$ mit $a_i|a_{i+1} \Ra$ Für jedes $b \in G$ ist ord$(b)$
Teiler von $a_r = n$.}
\end{Prop}

\begin{DefBem}
Sei $K$ Körper mit Charakteristik $p>0$.
\begin{enum}
\item Dann ist die Abbildung $\varphi:K\to K$, $x\mapsto x^p$ ein Homomorphismus. Er heißt \emp{Frobenius}-Homomorphismus.
\item Es ist $\varphi(x) =x \iff x\in \mathbb F_p$ (als Primkörper in $K$).
\end{enum}
\end{DefBem}

\begin{Satz}
Sei $p$ Primzahl, $n \geq 1, q = p^n$. Sei
$\mathbb{F}_q$ der Zerfällungskörper von $X^q - X \in
\mathbb{F}_p[X]$.

Dann gilt: \begin{enum}
\item $\mathbb{F}_q$ hat $q$ Elemente.
\item Zu jedem endlichen Körper $K$ gibt es ein $q = p^n$ mit $K
\cong \mathbb{F}_q$
\end{enum}

\sbew{\begin{enum}
\item $f(X) = X^q - X$ ist separabel, da $f'(X) = -1 \Ra$
ggT$(f,f') = 1 \Ra f$ hat $q$ verschiedene Nullstellen in
$\mathbb{F}_q \Ra |\mathbb{F}_q| \geq q$.

Umgekehrt: Jedes $a \in \mathbb{F}_q$ ist Nullstelle von $f$.

\textbf{denn}: $\mathbb{F}_q$ wird erzeugt von den Nullstellen von
$f$. Sind also $a,b$ Nullstellen von $f$, so ist $a^q = a$, $b^q =
b$, also auch $(ab)^q = ab, (a+b)^q = a^q + b^q = a+b$.

\item $(K^x, \cd)$ ist Gruppe der Ordnung $q-1 \Ra$ Für jedes
$a \in K$ gilt $a^q = a \Ra$ Jedes $a \in K$ ist Nullstelle von $X^q
- X \Ra K$ liegt im Zerfällungskörper von $X^q - X \Ra K$ enthält
$\mathbb{F}_q$ (bis auf Isomorphie).
\[ \overset{|K| = |\mathbb{F}_q| = q}{\Ra} K \cong \mathbb{F}_q\]
\end{enum}
}
\end{Satz}

\begin{Folg}
Jede algebraische Erweiterung eines
endlichen Körpers ist separabel.

\sbew{$\mathbb{F}_q/\mathbb{F}_p$ separabel, da $X^q - X$
separables Polynom ist. Ist $K$ endlich, also $K = \mathbb{F}_q$,
$L/K$ algebraisch, $\alpha \in L$, so ist $K(\alpha)/K$ endlich,
also separabel (da $K(\alpha) = \mathbb{F}_{q^r}$ für ein $r \geq 1$)
\newline \newline \textbf{Definition}: Ein Körper $K$ heißt \emp{vollkommen} 
(oder perfekt), wenn jede algebraische Körpererweiterung $L/K$ separabel ist.}
\end{Folg}

\section{Konstruktion mit Zirkel und Lineal}

\textbf{Aufgabe}: Sei $M \subset \mathbb{C} = \mathbb{R}^2$, z.B.:
$M=\{0,1\}$.
\[\mbox{Linien: }\mathcal{L}(M) \defeqr \{L \subset \mathbb{R}^2 \mbox{ Gerade: }
|L \cap M| \geq 2\}\; \cup \;\{K_{z_1 - z_2}(z_3): z_1, z_2, z_3
\in M\}\qquad\qquad\] $(K_r(z) = \{y \in \mathbb{R}^2:\;|z-y| =
r\}) \\\\ K_1(M) \defeqr \{z \in \mathbb{C}:\; z$ liegt auf zwei
verschiedenen Linien in $\mathcal{L}(M)\}\\$ $K_n(M) \defeqr
K_1(K_{n-1}(M))$ für $n \geq 2\\$ $K(M) \defeqr \bigcup_{n=1}^\infty
K_n(M)$

\begin{Satz}
Sei $M\subseteq \mathbb R^2$ mit $0,1\in M$ und $K(M)$ die Menge der mit Zirkel und Lineal konstruierbaren Punkte.
\begin{enum}
\item $K(M)$ ist ein Teilkörper von $\mathbb C$.
\item $K(M)/\mathbb Q(M)$ ist eine algebraische Körpererweiterung, dabei sei $\mathbb Q(M)$ der kleinste Teilkörper von $\mathbb C$, der $\mathbb Q$ und $M$ umfasst und mit $a$ auch $\bar a$ enthält.
\item Eine komplexe Zahl $a\in \mathbb C$ liegt genau dann in $K(M)$, wenn es eine Kette
\[
\mathbb Q(M) = L_0\subset L_1\subset \cdots \subset L_n
\]
gibt mit $a\in L_n$ und $[L_i:L_{i-1}]=2$ für $i=1,\ldots,n$.
\end{enum}

\bew{} {
\item 
Seien $a,b \in K(M)$. Zu zeigen: $a+b, -a, a \cd b, \frac{1}{a} \in
K(M)$.

$a+b \in K(M):$
\begin{center}
\includegraphics[width=0.6\textwidth]{alg16a1.png}
\end{center}
$-a \in K(M):$
\begin{center}
\includegraphics[width=0.6\textwidth]{alg16a2.png}
\end{center}
$a \cd b \in K(M):$\newline
Strahlensatz: $\frac{1}{a} = \frac{b}{x}$, also $x = a \cd b$. Winkel addieren
$\chk \Ra a \cd b$ allgemein $\chk$
\begin{center}
\includegraphics[width=0.6\textwidth]{alg16a3.png}
\end{center}

$\frac{1}{a} \in K(M):$ \OE $a \in \mathbb{R}$
\begin{center}
\includegraphics[width=0.6\textwidth]{alg16a4.png}
\end{center}

\item folgt aus (a)

\item Zeige mit Induktion über $n$: Jedes $a\in K_n(M)$ ist algebraisch über $\mathbb Q(M)$. Wegen $K_n(M)=K_1(\mathcal L_n(M))$ genügt es, die Behauptung für $n=1$ zu zeigen. Sei also $z\in K_1(M)$.

Vorüberlegung: Für $z \in M$ ist $\Re(z) =
\frac{1}{2}(z+\bar z) \in \mathbb{Q}(M)$ und $\Im(z) = \frac{1}{2}(z
-\bar z) \in \mathbb{Q}(M).$
\begin{enum}
\item $z$ ist Schnittpunkt zweier Geraden in $\mathcal{L}(M) \Ra
z$ ist Lösung zweier linearer Gleichungen $z_1 + \lambda z_2 = z_1'
+ \mu z_2'$
\item $z$ ist Schnittpunkt einer Geraden und eines Kreises: $\Ra$
quadratische Gleichung mit Koeffizienten in $\mathbb{Q}(M)$
\item $z$ ist Schnittpunkt zweier Kreise $K_{r_1}(m_1)$ und
$K_{r_2}(m_2)$ mit Mittelpunkten $m_1,m_2 \in M$. Radien: $r_1 =
|z_1 - z_1'|$, $r_2 = \dots$ also $r_1^2 = (z_1 - z_1') (\overline{z_1 -
z_1'}) \in \mathbb{Q}(M)$.

Dann ist $|z-m_1|^2 = r_1^2$.

$\Ra z\bar z - (z\bar{m_1} + \bar z m_1) = r_1^2 - m_1 \bar{m_1}$
und $z \bar z - (z \bar{m_2} + \bar z m_2) = r_2^2 - m_2 \bar{m_2}
\Ra 2 \Re[z(\bar{m_1} - \bar{m_2})] = r_1^2 - r_2^2 - (m_1 \bar{m_1}
- m_2 \bar{m_2})$

Das ist eine lineare Gleichung, die $\Re(z)$ und $\Im(z)$ enthält.
Einsetzen in $(1)$ ergibt quadratische Gleichung für $\Re(z)$ (mit
Koeffizienten in $\mathbb{Q}(M)$).
\end{enum}

Noch zu zeigen: Ist $a\in \mathbb C$ und gibt es eine Kette 
\[
\mathbb Q(M) = L_0\subset L_1\subset \cdots \subset L_n
\]
von Körpererweiterungen mit $[L_i:L_{i-1}]=2$ und $a\in L_n$, so ist $a\in K(M)$.

Sei also $L/K$ quadratische Erweiterung von Körpern (mit Charakteristik ungleich 2). Dann gibt es $\alpha\in L$ und $a \in K$, so dass $L=K(\alpha)$ und $\alpha^2=a$, das heißt $L=K(\sqrt{a})$. Zu zeigen ist also: Ist $K\subset K(M)$, so ist $\sqrt a\in K(M)$:

Wurzelziehen: $a \in \mathbb{R}$
\begin{center}
\includegraphics[width=0.6\textwidth]{alg16b.png}
\end{center}
$\overset{\scriptsize\mbox{Thales}}{\Ra}$ Winkel ist rechtwinklig
$\overset{\scriptsize\mbox{Höhensatz}}{\Ra} b^2=|-a| \cd 1 = a$
}
\end{Satz}

\bsp{
Das regelmäßige Fünfeck ist aus 0 und 1 konstruierbar. Ziel: Konstruiere Nullstellen von $X^5-1=(X-1)\cdot f$, $f\defeqr X^4+X^3+X^2+X+1$. Trick von Lagrange: $f(X) = X^2(X^2 + \frac1{X^2} + X + \frac1{X} + 1)$. Mit $Y\defeqr X+ \frac 1X$ ist dann $\frac 1{X^2} \cdot f(X) = Y^2+Y-1\defeql g(Y)$. Ist $y$ Nullstelle von $g$ und $\xi$ Nullstelle von $f$, so ist $\mathbb Q\subset \mathbb Q(y) \subset \mathbb Q(\xi)$ eine Kette wie im Satz.
}


\chapter{Galois-Theorie}

\section{Der Hauptsatz}

\begin{DefProp}
\label{4.1}
    Sei $L/K$ algebraische Körpererweiterung, $\bar K$ ein algebraischer Abschluss von $L$.

    \begin{enum}
        \item $L/K$ heißt \emp{normal}, wenn es eine Familie $\mathcal{F} 
        \subset K[X]$ gibt, so dass $L$ Zerfällungskörper von $\mathcal{F}$ ist.

        \item Ist $L/K$ normal, so ist Hom$_K(L,\bar K) =$ Aut$_K(L)$
        \sbew{
            ''$\supseteq$'' gilt immer. ''$\subseteq$'': Sei $L =
            Z(\mathcal{F})$, $f \in \mathcal{F}$, $\alpha \in L$ Nullstelle von
            $f \Ra$ Für $\sigma \in$ Hom$_K(L,\bar K)$ ist $\sigma(\alpha)$ auch
            Nullstelle von $f$. Sei $f(X) = \displaystyle \sum_{i=0}^n a_i X^i
            \Ra 0 = \sigma(f(\alpha)) = \sum_{i=0}^n 
            \underset{=a_i}{\underbrace{\sigma(a_i)}} \sigma(\alpha^i) = 
            f(\sigma(\alpha)) \Ra \sigma(\alpha) \in L \Ra \sigma(L) \subseteq L$. $\sigma$ ist surjektiv, da $L$ von den
            Nullstellen der $f \in \mathcal{F}$ erzeugt wird und jedes $f\in \mathcal F$ endlich viele Nullstellen hat, die durch $\sigma$ permutiert werden.
        }

        \item $L/K$ heißt \emp{galoissch}, wenn $L/K$ normal und separabel ist.

        \item Ist $L/K$ galoissch, so heißt \emp{Gal}$\mathbf(L/K)$ $\defeqr$
        Aut$_K(L)$ die \emp{Galoisgruppe} von $L/K$.

        \item Eine endliche Erweiterung $L/K$ ist genau dann galoissch, wenn 
        $|$Aut$_K(L)| = [L:K]$
        \sbew{''$\Ra$'' Aus (b) folgt \[|\mbox{Aut}_K(L)| = 
        |\mbox{Hom}_K(L,\bar K)| = [L:K]_s \overset{\mbox{\scriptsize
\ref{Satz 13}}}{=} [L:K] (\ast)\]

''$\Leftarrow$'' In $(\ast)$ gilt stets $|\mbox{Aut}_K(L)| \leq 
|\mbox{Hom}_K(L,\bar K)| = [L:K]_s \leq [L:K]$. Aus
$|$Aut$_K(L)| = [L:K]$ folgt also $[L:K]_s = [L:K] \Ra L/K$
separabel $\overset{\ref{Satz 14}}{\Ra} L=K(\alpha)$ für ein $\alpha
\in L$; Sei $f \in K[X]$ das Minimalpolynom von $\alpha$. Sei $\beta
\in \bar K$ Nullstelle von $f$. Nach \ref{3.8} gibt es $\sigma \in$
Hom$_K(L,\bar K)$ mit $\sigma(\alpha) = \beta$. Wegen $(\ast)$ ist
$\sigma \in$ Aut$_K(L) \Ra \beta \in L \Ra L =$ Z$(f)$. }
\end{enum}
\end{DefProp}

\bsp{
Sei $K$ Körper mit Charakteristik nicht 2, $d\in K^\times \setminus (K^\times)^2$. Dann ist $K{\sqrt{d}}/K$ eine Galois-Erweiterung, denn $X^2-d$ ist irreduzibel und separabel und zerfällt in $K(\sqrt d)[X]$ in $(X-\sqrt d)(X+ \sqrt d)$.
}

\begin{Bem}
\label{4.1.2}
\begin{enum}

\item Ist $L/K$ galoissch und $E$ ein Zwischenkörper, so ist $L/E$
galoissch und Gal$(L/E) \subseteq$ Gal$(L/K)$.

\sbew{$L/E$ normal, da Zerfällungskörper von $\mathcal{F}
\subset K[X] \subseteq E[X]$. $L/E$ separabel, da $L/K$ separabel und das Minimalpolynomm von $\alpha\in L$ über $E$ in $E[X]$ Teiler des Minimalpolynoms über $K$ ist.}

\item Ist in (a) zusätzlich auch $E/K$ galoissch, so ist \[1 \ra
\mbox{Gal}(L/E) \ra \underset{\sigma \mapsto
\sigma_{|E}}{\mbox{Gal}(L/K) \overset{\beta}{\ra} \mbox{Gal}(E/K)}
\ra 1\] exakt.

\sbew{Für $\sigma \in$ Gal$(L/K) =$ Aut$_K(L)$ ist
$\sigma_{|E}: E \ra L$, also $\sigma \in$ Hom$_K(E,L) \subseteq$
Hom$_K(E, \bar K) =$ Aut$_K(E)$, da $E/K$ galoissch ist. $\Ra \beta$
ist wohldefiniert.

$\beta$ \textbf{surjektiv}: Sei $\sigma \in$ Gal$(E/K)$. Nach
\ref{3.10} läßt sich $\sigma$ fortsetzen zu $\wt{\sigma}: L\ra \bar
K$, $\wt{\sigma} \in$ Hom$_K(L, \bar K) =$ Aut$_K(L)$ = Gal$(L/K)$
und $\beta(\wt{\sigma}) = \wt{\sigma}_{|E} = \sigma$

$\Kern \beta = \{ \sigma \in$ Gal$(L/K):\; \sigma_{|E} = id_E\} =$
Aut$_E(L) =$ Gal$(L/E)$}
\end{enum}
\end{Bem}

\begin{Satz}[Hauptsatz der Galoistheorie]
\label{Satz 17}
Sei $L/K$ endliche Galois-Erweiterung.
\begin{enum}

\item Die Zuordnungen
\[\begin{array}{ccc}
\{\mbox{Zwischenkörper von } L/K \}
&\begin{array}{c} \overset{\Psi}{\longrightarrow} \\
\underset{\Phi}{\longleftarrow}\end{array} &\{\mbox{Untergruppen von
Gal}(L/K)\} \\
E & \longmapsto & \mbox{Gal}(L/E) \\
L^H = \{\alpha \in L: \sigma(\alpha) = \alpha\; \forall \sigma \in
H\} &\longmapsfrom & H\end{array}\] sind bijektiv und zueinander
invers.

\item Ein Zwischenkörper $E$ von $L/K$ ist genau dann galoissch über
$K$, wenn Gal$(L/E)$ Normalteiler in Gal$(L/K)$ ist.
\end{enum}

\bew{}{\item $L^H$ ist Zwischenkörper: $\chk$

''$\Psi \circ \Phi = id$'': Sei $H \subseteq$ Gal$(L/K)$
Untergruppe. z.z.: Gal$(L/L^H) = H$

''$\supseteq$'' Nach Def. von $L^H$ ''$\subseteq$'': Nach \ref{4.1}
ist $|$Gal$(L/L^H)| = [L:L^H]$. Es genügt also z.z.: $[L:L^H] \leq
|H|$. Sei $\alpha \in L$ primitives Element von $L/L^H$, also
$L=L^H(\alpha)$. Sei $f \defeqr \displaystyle \prod_{\sigma \in H} (X-
\sigma(\alpha)) \in L[X]$; dann ist deg$(f) = |H|$. Für jedes $\tau
\in H$ ist $f^\tau = f$ (mit $\sigma$ durchläuft auch $\sigma \circ
\tau$ alle Elemente von $H$) $\Ra f \in L^H[X] \Ra$ Das
Minimalpolynom $g$ von $\alpha$ über $L^H$ ist Teiler von $f$. $\Ra
[L:L^H] =$ deg$(g) \leq$ deg$(f) = |H|$

''$\Phi \circ \Psi = id$'': Sei $E$ Zwischenkörper, $H \defeqr$
Gal$(L/E)$. zu zeigen: $E = L^H$.

''$\subseteq$'': Definition. ''$\supseteq$'': Da $L^H/E$ separabel
ist, genügt es zu zeigen $[L^H :E]_s = 1$. Sei also $\sigma \in$
Hom$_E(L^H,\bar K)$, $\wt{\sigma} \in$ Hom$_E(L,\bar K) =$ Aut$_E(L)
=$ Gal$(L/E)=H$ Fortsetzung $\Ra
\underset{=\sigma}{\wt{\sigma}_{|L^H}} = id_{L^H}$

\item ''$\Ra$'': \ref{4.1.2} b), da $\operatorname{Gal}(L/E)=\operatorname{Kern}\beta$.
''$\Leftarrow$'': Sei $H \defeqr$ Gal$(L/E)$ Normalteiler in
Gal$(L/K)$. Wegen \ref{4.1} c) genügt es zu zeigen: Für jedes $\sigma
\in$ Hom$_K(E,\bar K)$ ist $\sigma(E) \subseteq E$. Sei also $\sigma
\in$ Hom$_K(E,\bar K)$, $\underset{=\mbox{\scriptsize
Gal}(L/K)}{\wt{\sigma} \in\mbox{ Hom}_K(L,\bar K)}$ Fortsetzung.

Sei nun $\alpha \in E$, $\tau \in H$. Dann ist $\tau(\sigma(\alpha))
= (\tau \circ \wt{\sigma})(\alpha) =
(\wt{\sigma}\circ\tau')(\alpha)=\wt{\sigma}(\alpha) = \sigma(\alpha) $ mit $\wt{\sigma}$ wie eben und
$\tau' \defeqr \wt{\sigma}^{-1}\circ \tau \circ \wt{\sigma} \in H $
(nach Voraussetzung) $\Ra
\sigma(\alpha) \in L^H  = E\; \chk$}
\end{Satz}

\begin{Folg}
Sei $L/K$ endliche Galoiserweiterung. Dann
gilt für Zwischenkörper $E,E'$ bzw. Untergruppen $H,H'$ von
Gal$(L/K)$: \begin{enum}
\item $E \subseteq E' \iff \operatorname{Gal}(L/E) \supseteq \operatorname{Gal}(L/E')$

      $H \subseteq H' \iff L^H \supseteq L^{H'}$

\item $\operatorname{Gal}(L/E \cap E') = \langle \operatorname{Gal}(L/E), \operatorname{Gal}(L/E')\rangle$

      $E\cap E' = L^{\langle \operatorname{Gal}(L/E), \operatorname{Gal}(L/E')\rangle}$

      $L^{H\cap H'} = L^H \cdot L^{H'} \defeqr K(L^H \cup L^{H'})$ (das \emp{Kompositum} von $L^H$ und $L^{H'}$)

\end{enum}
\end{Folg}

\begin{Folg}
Zu jeder endlichen separablen
Körpererweiterung gibt es nur endlich viele Zwischenkörper.

\sbew{Ist $L/K$ endliche Galoiserweiterung, so entsprechen die
Zwischenkörper (nach \ref{Satz 17}) bijektiv den Untergruppen der
endlichen Gruppe$(L/K)$. Im allgemeinen ist $L=K(\alpha)$ (\ref{Satz 14}). Sei
$f$ das Minimalpolynom von $\alpha$ über $K$. $f$ ist separabel, da $L/K$
separabel. Sei $\wt{L}$ der Zerfällungskörper von $f$ über $K$.
$\Ra \wt{L}/K$ ist galoissch, $K \subseteq L \subseteq \wt{L} \Ra
L/K$ hat nur endlich viele Zwischenkörper. }
\end{Folg}

\begin{Prop}
\label{4.4}
Sei $L$ ein Körper, $G \subseteq$ Aut$(L)$
eine endliche Untergruppe. $K \defeqr L^G = \{\alpha \in
L:\,\sigma(\alpha) = \alpha\;\forall\; \sigma \in G\}$

Dann ist $L/K$ Galoiserweiterung und Gal$(L/K) = G$

\sbew{\begin{itemize}
\item $L/K$ ist algebraisch und separabel. Sei dazu $\alpha \in L$.
$\{\sigma(\alpha):\; \sigma \in G\} = G \alpha$  ist endlich. Sei
$G\alpha = \{\sigma_1(\alpha),\dots,\sigma_r(\alpha)\}$ mit
$\sigma_i(\alpha) \neq \sigma_j(\alpha)$ für $i\neq j$ und $\sigma_1
= id_L$. Dabei ist $r$ ein Teiler von $n := |G|$. Sei $f_\alpha(X)
\defeqr \displaystyle \prod_{i=1}^r (X- \sigma_i(\alpha)) \in L[X]$. Zu zeigen:
$f_\alpha \in K[X]$. \textbf{denn}: für $\sigma \in G$ ist
$f_\alpha^\sigma(X) = \displaystyle \prod_{i=1}^r (X -\sigma(\sigma_i(\alpha)))$
(selbe Faktoren wie $f_\alpha(X)$) $\Ra f_\alpha = f_\alpha^\sigma$
$\Ra f_\alpha \in K[X]$

$\Ra \alpha$ algebraisch, $\alpha$ separabel (da $f_\alpha$
separabel), $[K(\alpha):K] \leq n \hfill(\ast)$

\item $L/K$ normal: Der Zerfällungskörper von $f_\alpha$ ist in $L$
enthalten. $\Ra L$ ist der Zerfällungskörper der Familie
$\{f_\alpha:\;\alpha \in L\}$

\item $L/K$ endlich: Sei $(\alpha_i)_{i\in I}$ Erzeugendensystem von
$L/K$. Für jede endliche Teilmenge $I_0 \subseteq I$ ist
$K(\{\alpha_i:\;i\in I_0\})$ endlich über $K$, also
$K(\{\alpha_i:\;i\in I_0\}) = K(\alpha_0)$ für ein $\alpha_0 \in L
\overset{(\ast)}{\Ra} [K(\{\alpha_i:\;i\in I_0\}):K] \leq n$. Sei
$I_1 \subseteq I$ endlich, so dass $K_1 \defeqr K(\{\alpha_i:\;i\in
I_1\})$ maximal unter den $K(\{\alpha_j:\;j\in J\})$ für $J \subseteq I$
endlich.

\textbf{Ann.}: $K_1 \neq L$. Dann gibt es $i \in I$ mit $\alpha_i
\not \in K_1 \Ra K_1(\alpha_i) \supsetneq K_1$, trotzdem endlich im
Widerspruch zu Wahl von $K_1 \Ra L/K$ endlich, genauer $[L:K] \leq
n$ wegen $(\ast)$.

\item Gal$(L/K) = G$: ''$\supseteq$'': nach Definition. Nach
\ref{4.1} ist $n = |G| \leq |$Gal$(L/K)| = [L:K] \leq n$
\end{itemize}}
\end{Prop}


\section{Die Galoisgruppe einer Gleichung}

\begin{DefBem}
Sei $K$ ein Körper, $f \in K[X]$ ein separables Polynom.

\begin{enum}
\item Sei $L= L(f)$ Zerfällungskörper von $f$ über $K$. Dann heißt
Gal$(f) \defeqr$ Gal$(L/K)$ \empind{Galoisgruppe von $\mathbf{f}$}{Galoisgruppe
von f}.

\item Ist $n =$ deg$(f)$, so gibt es injektiven
Gruppenhomomorphismus Gal$(f) \hookrightarrow S_n$ (durch
Permutation der Nullstellen von $f$)

\item Ist $L/K$ separable Körpererweiterung vom Grad $n$, so ist
Aut$_K(L)$ isomorph zu einer Untergruppe von $S_n$.

\sbew{Sei $L=K(\alpha)$, $f \in K[X]$ Minimalpolynom
von $\alpha$, $\alpha= \alpha_1,\dots,\alpha_d$ die Nullstellen von
$f$ in $L \Ra$ jedes $\sigma \in$ Aut$_K(L)$ permutiert
$\alpha_1,\dots,\alpha_d$.}
\end{enum}
\end{DefBem}

\begin{Bspe}
Die Galoisgruppe von $f(X) = X^5 - 4X + 2 \in
\mathbb{Q}[X]$ ist $S_5$.
\newline\newline\textbf{Bew.}:\begin{itemize}

\item $f$ ist irreduzibel: Eisenstein für $p=2$

\item $f$ hat 3 relle und 2 zueinander konjugiert komplexe
Nullstellen $f(-\infty) = -\infty,\;f(0)=2,f(1) =
-1,f(\infty)=\infty$ $\Ra f$ hat mindestens $3$ reelle Nullstellen.

$f'(X) = 5X^4 - 4 = 5(X^2 - \frac{2}{\sqrt{5}})(X^2 +
\frac{2}{\sqrt{5}})$ hat $2$ reelle Nullstellen $\Ra f$ hat genau
$3$ reelle Nullstellen. Ist $\alpha \in \mathbb{C}$ Nullstelle von
$f$, so ist $f(\bar \alpha) = \overline{f(\alpha)} = 0$.

\item $G=$ Gal$(f)$ enthält die komplexe Konjugation $\tau$. $\tau$
operiert als Transposition: $2$ Nullstellen werden vertauscht, $3$
bleiben fix.

\item $G$ enthält ein Element von Ordnung $5$: Ist $\alpha$
Nullstelle von $f$, so ist $[\mathbb{Q}(\alpha):\mathbb{Q}] = 5$ und
$\mathbb{Q}(\alpha) \subseteq L(f) \overset{\ref{Satz 17}}{\Ra} 5$
teilt $|G| \overset{\mbox{\scriptsize Sylow}}{\Ra}$ Beh.

\item $G$ enthält also einen $5$-Zyklus und eine Transposition
$\overset{\mbox{(!)}}{\Ra} G = S_5$.
\end{itemize}
\end{Bspe}

\begin{Bem}
Allgemeine Gleichung $n$-ten Grades: Sei $k$
ein Körper, $L = k(T_1,\dots,T_n) =$ Quot$(k[T_1,\dots,T_n])$

\begin{itemize}
\item $S_n$ operiert auf $L$ durch $\sigma(T_i) = T_{\sigma(i)}$

\item Sei $K\defeqr L^{S_n}$. $L/K$ ist Galois-Erweiterung (nach
Proposition \ref{4.4}) vom Grad $n!$

\item $L$ ist (über $K$) Zerfällungskörper von $f(X) =
\displaystyle \prod_{i=1}^n(X-T_i) \in K[X]$

\item Gal$(f) = S_n$

\item $f(X) = \displaystyle \sum_{\nu = 0}^n (-1)^{\nu} s_{\nu}
(T_1,\dots,T_n)X^{n-\nu}$ mit $s_{\nu}(T_1,\dots,T_n) =
\displaystyle \sum_{1\leq i_1 < \dots < i_{\nu} \leq n} T_{i_1} \cd \dots \cd 
T_{i_\nu}$

z.B.: $s_1(T_1,\dots,T_n) = T_1 + \dots + T_n$, $s_2 = T_1 T_2 + T_1
T_3 + \dots + T_{n-1}T_n$, $s_n = T_1 \cd \dots \cd T_n$

\item $K = k(s_1,\dots,s_n)$
\end{itemize}
\end{Bem}
\section{Einheitswurzeln}

\begin{BemDef}
\label{4.8}
Sei $K$ ein Körper, $\bar K$
algebraischer Abschluss von $K$. Sei $n$ eine positive ganze
Zahl.
Angenommen, char$(K)$ ist entweder $0$ oder teilerfremd zu $n$.

\begin{enum}
\item Die Nullstellen von $X^n - 1$ in $\bar K$ heißen
$\mathbf{n}$\emp{-te Einheitswurzeln}.

\item $\mu_n(\bar K) \defeqr \{\zeta \in \bar K: \zeta^n = 1\}$ ist
zyklische Untergruppe von ${\bar K}^x$ der Ordnung $n$.

\sbew{$\mu_n(\bar K)$ Untergruppe $\checkmark$, also zyklisch
nach \ref{3.17}. $f(X) = X^n-1$ ist separabel, da $f'(X) = nX^{n-1}$
(Bem \ref{3.13})}

\item Eine $n$-te Einheitswurzel $\zeta$ heißt \emp{primitiv}, wenn
$\langle \zeta \rangle = \mu_n(\bar K)$
\end{enum}
\end{BemDef}

\begin{Satz}
(Voraussetzungen wie eben.)
\begin{enum}
\item Die Anzahl der primitiven Einheitswurzeln in $\bar K$ ist
$\varphi(n) = |(\mathbb{Z}/n\mathbb{Z})^x| = \{ m\in \{1,\dots,n\} :
$ggT$(m,n) = 1\}$ ($n \mapsto \varphi(n)$ ist Eulersche
$\varphi$-Funktion)

\sbew{Ist $\zeta$ primitive $n$-te Einheitswurzel, so ist
$\mu_n(\bar K) = \{1,\zeta, \zeta^2, \dots, \zeta^{n-1}\}$, $\zeta^k$
erzeugt $\{1,\zeta, \zeta^2, \dots, \zeta^{n-1}\} \lra$ ggT$(n,k) = 1$.}

\item Ist $n = p_1^{\nu_1} \dots p_r^{\nu_r}$,
(Primfaktorzerlegung) so ist $\varphi(n) = \displaystyle \prod_{i=1}^r
p_i^{\nu_i-1} (p_i - 1)$

\sbew{Nach Satz \ref{Satz 8} ist $\mathbb{Z}/n\mathbb{Z} \cong
\mathbb{Z}/p_1^{\nu_1}\mathbb{Z} \bigoplus \dots \bigoplus
\mathbb{Z}/p_r^{\nu_r} \mathbb{Z}$ (als Ringe) $\Ra
(\mathbb{Z}/n\mathbb{Z})^x  = (\mathbb{Z}/p_1^{\nu_1} \mathbb{Z})^x
\bigoplus \dots \bigoplus (\mathbb{Z}/p_r^{\nu_r} \mathbb{Z})^x$
(als Gruppen). Doch für jede Primzahl $p$ und jedes positive $\nu$
ist

$|(\mathbb{Z}/p^\nu \mathbb{Z})^x| = p^{\nu} - p^{\nu-1} = p^{\nu -
1}(p-1)$.}

\item Sind $\zeta_1,\dots,\zeta_{\varphi(n)}$ die primitiven
Einheitswurzeln, so heißt $\Phi_n(X) \defeqr
\displaystyle \prod_{i=1}^{\varphi(n)} (X- \zeta_i) \in \bar K[X]$ das $n$-te
\emp{Kreisteilungspolynom}

\item $X^n - 1 = \displaystyle \prod_{d \mid n} \Phi_d(X)$

\sbew{$X^n - 1 = \displaystyle \prod_{\zeta \in \mu_n} (X-\zeta) =
\prod_{d \mid n} \prod_{\substack{\zeta \in \mu_n \\ ord(\zeta) = d}} (X-\zeta) = \prod_{d
\mid n} \Phi_d(X)$}

\item Sei $\zeta$ primitive $n$-te Einheitswurzel. Dann ist
$K(\zeta)/K$ Galois-Erweiterung.
\sbew{$K(\zeta)$ ist
Zerfällungskörper von $X^n - 1$ über $K$, also normal. $X^n - 1$ ist
separabel (\ref{4.8})}

\item \[\chi_n: \begin{array}{ccc} \mbox{Gal}(K(\zeta)/K) &\ra
&(\mathbb{Z}/n\mathbb{Z})^x \\ \sigma &\mapsto &\chi_n(\sigma)
\end{array}\] ist injektiver Gruppenhomomorphismus, wobei
\newline $\sigma(\zeta) = \zeta^{\chi_n(\sigma)}$. ($\chi_n$ heißt
\emp{zyklotomischer Charakter})

\sbew{$\chi_n(\sigma) \in (\mathbb{Z}/n\mathbb{Z})^x$, da
$\sigma(\zeta)$ primitive Einheitswurzel sein muß.

$\chi_n$ ist Gruppenhomomorphismus: $\sigma_1, \sigma_2 \in$
Gal$(K(\zeta)/K) \Ra \sigma_1(\sigma_2(\zeta)) =
\sigma_1(\zeta^{\chi_n(\sigma_2)}) =
(\sigma_1(\zeta))^{\chi_n(\sigma_2)} = \zeta^{\chi_n(\sigma_1)
\chi_n(\sigma_2)}$

$\chi_n$ injektiv: $\chi_n(\sigma) = 1 \Ra \sigma(\zeta) = \zeta \Ra
\sigma = id$}

\item $\Phi_n(X) \in K[X]$, genauer $\Phi_n(X) \in \left\{
\begin{array}{ll} \mathbb{Z}[X] \mbox{ (primitiv) } &:\mbox{char}(K) =
0 \\ \mathbb{F}_p[X] &:\mbox{char}(K) = p \end{array}\right.$
\newline
\sbew{
Induktion über $n$: $n=1\;\checkmark$

$n>1$: $\underset{(\ast)}{\underbrace{X^n - 1}} \overset{(d)}{=} \Phi_n(X)
\underset{ (\ast \ast)}{\underbrace{\displaystyle \prod_{\substack{d \mid n\\
d < n}} \Phi_d(X)}}$

char$(K) = p : (\ast) \in \mathbb{F}_p[X]$, $(\ast \ast) \in \mathbb{F}_p[X]
\mbox{ nach IV} \Ra \Phi_n(X) \in \mathbb{F}_p[X]:$ (weil
Polynomdivision zweier Polynome in $\mathbb{F}_p[X]$ nie die
Koeffizienten aus dem Körper $\mathbb{F}_p$ herausführt).

char$(K) = 0: (\ast) \in \mathbb{Z}[X]$ (primitiv), $(\ast \ast) \in
\mathbb{Z}[X]$ primitiv nach IV

% TODO Lemma von Gauß = Satz von Gauß = Satz 11???
$\overset{\mbox{\scriptsize Lemma von Gauß}}{\Ra} \Phi_n(X) \in
\mathbb{Z}[X]$ primitiv.
}

\item Ist $K = \mathbb{Q}$, so ist $\Phi_n$ irreduzibel und $\chi_n$
ein Isomorphismus. $\mathbb{Q}(\zeta)$ heißt $n$-ter
\emp{Kreisteilungskörper}.

\sbew{
Es genügt zu zeigen: $\Phi_n$ irreduzibel (dann folgt $\chi_n$
Isomorphismus aus (e) und (f))

Sei $f \in \mathbb{Q}[X]$ Minimalpolynom von $\zeta$, $f \in
\mathbb{Z}[X]$ wegen (g)

\textbf{Beh.}: $f(\zeta^p) = 0$ für jede Primzahl $p$ mit $p \nmid
n$. Dann ist auch $f(\zeta^m) = 0$ für jedes $m$ mit ggT$(m,n) = 1\Ra
f(\zeta_i) = 0$ für jede primitive Einheitswurzel $\zeta_i \Ra
\Phi_n|f \Ra \Phi_n = f$

\textbf{Bew.}: Sei $X^n - 1 = f \cd h$. Wäre $f(\zeta^p) \neq 0 \Ra
h(\zeta^p) = 0$ dh. $\zeta$ Nullstelle von $h(X^p) \Ra h(X^p)$ ist
Vielfaches von $f \Ra \exists\;g \in \mathbb{Z}[X]$ mit $h(X^p) = f
\cd g\\ \overset{mod\;p}{\Ra} \bar f \bar g = {\bar h}^p$ in $\bar
{\mathbb{F}}_p [X] \Ra \bar f$ und $\bar h$ haben gemeinsame
Nullstellen in $\bar {\mathbb{F}}_p \Ra X^n - \bar 1 = \bar f \bar
h$ hat doppelte Nullstelle $\blitzb$ zu $X^n - 1$ separabel.
}
\end{enum}

\textbf{\newline Beispiele:} $\ds\Phi_1(X) = 1$, $\Phi_2(X) = X+1$, $\Phi_p(X) =
X^{p-1} + X^{p-2} + \dots + X + 1$ für $p$ prim.

\[\Phi_4(X) = \frac{X^4-1}{\Phi_2 \cd \Phi_1} = \frac{X^4-1}{X^2 -1} =
X^2 + 1\]

\[\ds\Phi_6(X) = \frac{X^6-1}{\Phi_3 \Phi_2 \Phi_1} = \dots = X^2 - X
+ 1\]

\[\ds\Phi_8(X) = X^4 +1\] \newline\newline Für $n < 105$ sind alle
Koeffizienten $0,1$ oder $-1$.
\end{Satz}

\begin{Folg}
Das regelmäßige $n$-Eck ist genau dann mit
Zirkel und Lineal (aus $\{0,1\}$) konstruierbar, wenn $\varphi(n)$ eine Potenz von
$2$ ist.
\sbew{z.z.: $\zeta_n$ (primitive
$n$-te Einheitswurzel) $\in K(\{0,1\}) \lra \varphi(n) = 2^l$ für
ein $l \geq 1\; \lra
\underset{\varphi(n)}{\underbrace{[\mathbb{Q}(\zeta_n) :
\mathbb{Q}]}} = 2^l$ und es gibt Kette $\mathbb{Q}(M) = L_0 \subset L_1
\subset \dots \subset L_n = \mathbb{Q}(\zeta_n)$ und $[L_i :
L_{i-1}] = 2$.

''$\Leftarrow$'': Gal$(\mathbb{Q}(\zeta_n):\mathbb{Q})$ ist abelsch
von Ordnung $2^l$. Dazu gehört Kompositionsreihe mit Faktoren
$\mathbb{Z}/2\mathbb{Z} \overset{\mbox{\scriptsize Hauptsatz d.
Galoistheorie}}{\Ra}$}
\end{Folg}
\section{Norm, Spur und Charaktere}

\begin{DefProp}
\label{4.10}
Sei $G$ eine Gruppe, $K$
ein Körper.
\begin{enum}
\item  Ein \emp{Charakter} von $G$ (mit Werten in $K$) ist ein
Gruppenhomomorphismus $\chi: G\ra K^x$

\item $X_K(G) \defeqr \{ \chi: G \ra K^x,\; \chi$ Charakter$\} =$
Hom$(G,K^x)$ heißt \emp{Charaktergruppe} von $G$ (mit Werten in
$K$)

\item (Lineare Unabhängigkeit der Charaktere, E.Artin)
$X_K(G)$ ist linear unabhängige Teilmenge des $K$-Vektorraums
Abb$(G,K)$


\sbew{Angenommen $X_K(G)$ ist linear abhängig. Dann sei
$n > 0$ minimal, so dass es in $X_K(G)$ $n$ paarweise verschieden
linear abhängige Elemente gibt. Es gebe also paarweise verschiedene
Charaktere $\chi_1,\dots,\chi_n \in X_K(G)$ und Körperelemente
$\lambda_1,\dots,\lambda_n \in K$ mit $\displaystyle \sum_{i=1}^n
\lambda_i \chi_i = 0$. Dazu muß $n\geq 2$ sein. Ferner sind die
Körperelemente $\lambda_1,\dots,\lambda_n \in K$ von $0$
verschieden, da sonst $n$ nicht minimal wäre.

Sei $g \in G$ mit $\chi_1(g) \neq \chi_2(g)$. Dann gilt für alle
$\ds h \in G$: \[0 = \sum_{i=1}^n \lambda_i \underbrace{\chi_i(gh)}_{=\chi_i(g)\chi_i(h)} =
\sum_{i=1}^n \underset{\defeql \mu_i \in K^x}{\underbrace{\lambda_i \chi_i
(g)}} \chi_i(h) = \sum_{i=1}^n \mu_i \chi_i(h) \Ra \sum_{i=1}^n \mu_i
\chi_i = 0\]

Sei $\nu_i \defeqr \mu_i - \lambda_i \chi_1(g),\;i=1,\dots,n$. Dann ist
$\displaystyle \sum_{i=1}^n \nu_i\chi_i = 0$ (da $\displaystyle
\sum_{i=1}^n \mu_i \chi_i = 0$ und $\displaystyle \sum_{i=1}^n \lambda_i \chi_i = 0$
ist). Da $\nu_1 = \lambda_1 \chi_1(g)
-\lambda_1\chi_1(g)=0$ ist, bedeutet dies: $\displaystyle \sum_{i=2}^n \nu_i\chi_i = 0$.
Wegen $\nu_2 = \lambda_2 \chi_2(g) - \lambda_2
\chi_1(g) = \underbrace{\lambda_2}_{\neq 0}\underbrace{(\chi_2(g) - \chi_1(g))}_{\neq 0} \neq 0$ sind also $\chi_2,...,\chi_n$
linear abhängig. Dies steht im Widerspruch zur
Minimalität von $n$.}
\end{enum}
\end{DefProp}

Es sei angemerkt, daß der Begriff eines ``Charakters'' in der Mathematik in sehr vielen,
teilweise stark unterschiedlichen Bedeutungen anzutreffen ist. So bedeutet ``Charakter''
in der Darstellungstheorie von Gruppen etwas anderes als in der obigen Definition \ref{4.10}.

\begin{DefBem}
Sei $L/K$ endliche
Körpererweiterung, $q\defeqr \frac{[L:K]}{[L:K]_s}$ ($=p^r,\;p=$char$(K)$), $n
\defeqr [L:K]_s$, Hom$_K(L,\bar K) = \{\sigma_1,\dots,\sigma_n\}$

\begin{enum}
\item Für $\alpha \in L$ heißt tr$_{L/K}(\alpha) \defeqr q \cd
\displaystyle \sum_{i=1}^n \sigma_i (\alpha) \in \bar K$ die \emp{Spur} von
$\alpha$ (über $K$)

\item $\forall \alpha \in L:$ tr$_{L/K}(\alpha) \in K$

\sbew{\OE $L/K$ separabel. Ist $L/K$ normal, also
galoissch, so ist Hom$_K(L,\bar K) =$ Gal$(L/K) \defeql G$ und
tr$_{L/K}(\alpha) \in L^G = K$ (da invariant unter allen $\sigma_i$). Andernfalls sei
$\wt{L}$ normale Erweiterung von
$K$ mit $L \subset \wt{L}$. Für $\tau \in$ Hom$_K(\wt{L},\bar K) =$
Gal$(\wt{L}/K)$ und jedes $i=1,\dots,n$ ist $\tau \circ \sigma_i
\in$ Hom$_K(L,\bar K)$ (da $\sigma_i(L) \subseteq \wt{L}$) $\Ra$
tr$_{L/K}(\alpha) \in {\wt{L}}^{\scriptsize\mbox{Gal}(\wt{L}/K)} = K$ }

\item tr$_{L/K}$ ist $K$-linear.

\item Für $\alpha \in L$ heißt $\ds N_{L/K}(\alpha) =
\left(\prod_{i=1}^n \sigma_i(\alpha)\right)^q$ die \emp{Norm} von
$\alpha$ (über $K$).

\item $N_{L/K}(\alpha) \in K$

\item $N_{L/K}: L^x \ra K^x$ ist Gruppenhomomorphismus

\sbew{\begin{enum}
\item[(e)] Ist $L/K$ separabel, so
argumentiere wie in (b). Sonst siehe Bosch.
\end{enum}
}
\end{enum}
\end{DefBem}

\begin{Bem}
Sei $L/K$ endliche Körpererweiterung. Für
$\alpha \in L$ sei $m_\alpha: L \ra L,\;x\mapsto \alpha x$.
$m_\alpha$ ist $K$-linear und es gilt:
\[\mbox{tr}_{L/K}(\alpha) = \mbox{Spur}(m_\alpha),\;N_{L/K}(\alpha)
= \mbox{det}(m_\alpha)\]

\sbew{Ist $L/K$ separabel, so sei $L=K(\alpha)$. Dann ist
$1,\alpha,\alpha^2,\dots,\alpha^{n-1}$ eine $K$-Basis von $L$,
$[L:K] = n$. Weiter sei $f(X) = X^n + c_{n-1} X^{n-1} + \dots + c_1
X + c_0\;\in K[X]$ das Minimalpolynom von $\alpha$ über $K$. Dann
ist die Abbildungsmatrix von $m_\alpha$ bezüglich der Basis
$1,\dots,\alpha^{n-1}$

\[ D = \begin{pmatrix}
0 & 0 &\dots & 0 & -c_0 \\
1 & 0 &      & \vdots & -c_1 \\
0 & 1 &      & \vdots & \vdots   \\
\vdots & \vdots & \ddots & 0 & \vdots \\
0 & 0 & \dots 0 & 1 & -c_{n-1}
\end{pmatrix}\]

$\Ra$ Spur$(m_\alpha) = -c_{n-1}$, det$(m_\alpha) = (-1)^n c_0$.

In $\bar K[X]$ zerfällt $f$ in Linearfaktoren:

$f = \prod_{i=1}^n(X-\sigma_i(\alpha)) \Ra c_{n-1} = \sum_{i=1}^n
\sigma_i(\alpha)$, $c_0 = (-1)^n \prod_{i=1}^n \sigma_i(\alpha)$

Ist $L\neq K(\alpha)$, so sei $b_1,\dots,b_n$ eine $K(\alpha)$-Basis
von $L$. Dann ist $B = \{ b_i
\alpha^j,\;i=1,\dots,m,\;j=0,\dots,n-1\}$ eine $K$-Basis von $L$.
Dann ist die Darstellungsmatrix von $m_\alpha$ bezüglich $B$:

\[ \wt{D} = \begin{pmatrix}
D & 0 & \dots & 0\\
0 & D & & \\
  &   & \ddots & \\
0 & 0 & & D \end{pmatrix} \] $\Ra$ Spur$(m_\alpha) = m(-c_{n-1})$,
det$(m_\alpha) = \left((-1)^n c_0 \right)^m$

Für jedes $\sigma_i \in$ Hom$_K(L,\bar K)$ ist $\sigma_i(\alpha)$
Nullstelle von $f$. Jede Nullstelle von $f$ wird dabei gleichoft
angenommen, nämlich $m = [L : K(\alpha)]$-mal $\Ra$
tr$_{L/K}(\alpha) = m \cd$ tr$_{K(\alpha)/K}(\alpha) = m(-c_{n-1})$
und $N_{L/K}(\alpha) = \left(N_{K(\alpha)/K}\right)^m = \left((-1)^n c_0 \right)^m$}
\end{Bem}

\begin{Satz}[''Hilbert(s Satz) 90'']
\label{Satz 19}
Sei $L/K$ zyklische
Galois-Erweiterung. (dh. Gal$(L/K) = \langle \sigma \rangle$ für ein
$\sigma$)
\begin{enum}

\item Ist $\beta \in L$ mit $N_{L/K}(\beta) = 1$, so gibt es ein
$\alpha \in L^x$ mit $\beta = \frac{\alpha}{\sigma(\alpha)}$

\sbew{ $n \defeqr [L:K]$. Nach \ref{4.10} sind die Charaktere
$id, \sigma,\dots,\sigma^{n-1}:\; L^x \ra L^x$ linear unabhängig
über $L$.

Nun ist $f = id + \beta \sigma + \beta\sigma(\beta) \sigma^2 + \dots
+ \beta \sigma(\beta) \dots \sigma^{n-2}(\beta) \sigma^{n-1}$
nicht die Nullabbildung $\Ra \exists \gamma \in L$ mit $\alpha
\defeqr f(\gamma) \neq 0$

$\beta \sigma(\alpha) = \beta \sigma(\gamma) + \beta \sigma(\beta)
\sigma^2 (\gamma) + \dots + \underset{N_{L/K}(\beta) =
1}{\underbrace{\beta \sigma(\beta) \dots \sigma^{n-1}(\beta)}}
\underset{=\gamma}{\underbrace{\sigma^n(\gamma)}} = \alpha$ }

\item Sei $L/K$ zyklische Galoiserweiterung, $n = [L:K]$, $\sigma \in$ Gal$(L/K)$
ein Erzeuger. Zu $\beta \in L$ mit tr$_{L/K}(\beta) = 0$ gibt es $\alpha
\in L$ mit $\beta = \alpha - \sigma(\alpha)$

\sbew{Sei $\gamma \in L$ mit tr$_{L/K}(\gamma) \neq 0$ und $\\\alpha
\defeqr \frac{1}{\mbox{\small tr}_{L/K}(\gamma)} \cd [ \beta \sigma(\gamma) +
(\beta + \sigma(\beta))\sigma^2(\gamma) + \dots + (\beta +
\sigma(\beta) + \dots + \sigma^{n-2}(\beta))\sigma^{n-1}(\gamma)]$
$\\\Ra \sigma(\alpha) = \frac{1}{\mbox{\small
tr}_{L/K}(\gamma)}[\sigma(\beta)\sigma^2(\gamma) +(\sigma(\beta) +
\sigma^2(\beta))\sigma^3(\gamma)+\dots+(\sigma(\beta) + \dots +
\sigma^{n-1}(\beta))\sigma^n(\gamma)]$ $\\\Ra (\alpha -
\sigma(\alpha))\mbox{tr}_{L/K}(\gamma) = \beta\sigma(\gamma) + \beta
\sigma^2(\gamma) + \dots + \beta\sigma^{n-1}(\gamma) -
\underset{-\beta}{\underbrace{(\sigma(\beta)+\dots+\sigma^{n-1}(\beta))}}
\gamma = \beta \cd \mbox{tr}_{L/K}(\gamma)$}
\end{enum}
\end{Satz}

\begin{Folg}
Voraussetzungen wie in Satz \ref{Satz 19}.
\begin{enum}

\item Ist char$(K)$ kein Teiler von $n=[L:K]$ und enthält $K$ eine
primitive $n$-te Einheitswurzel $\zeta$, so gibt es ein primitives
Element $\alpha \in L$, so dass das Minimalpolynom von $\alpha$ über
$K$ von der Form \[X^n - \gamma\] ist für ein $\gamma \in K$.
(\textit{''Kummer-Erweiterung''})

\item Ist char$(K) = [L:K] = p$, so gibt es ein primitives Element
$\alpha \in L$, so dass das Minimalpolynom von $\alpha$ über $K$ die
Form \[ X^p - X - \gamma\] hat für ein $\gamma \in K$.
(\textit{''Artin-Schreier-Erweiterung''})

\end{enum}
\bew{}{\item Es ist $N_{L/K}(\zeta) = \zeta^n = 1 =
N_{L/K}(\zeta^{-1}) \overset{\mbox{\scriptsize Satz \ref{Satz
19}}}{\Ra}$ es gibt $\alpha \in L$ mit $\sigma(\alpha) = \zeta
\alpha \Ra \sigma^i(\alpha) = \zeta^i \alpha,\;i=1,\dots,n-1$ $\Ra$
Das Minimalpolynom von $\alpha$ über $K$ hat $n$ verschiedene
Nullstellen $\Ra L = K(\alpha)$.

Außerdem ist $\sigma(\alpha^n) = \sigma(\alpha)^n = \alpha^n \Ra
\gamma \defeqr \alpha^n \in K$ $\Ra$ Das Minimalpolynom von $\alpha$
ist $X^n - \gamma$

\item tr$_{L/K}(1) = 1 + \dots + 1 = p = 0
\overset{\scriptsize\ref{Satz 19}}{\Ra}$ es gibt $\alpha \in L$ mit
$\sigma(\alpha) = \alpha + 1 \Ra \sigma^i(\alpha) = \alpha +
i,\;i=0,\dots,n-1 \Ra K(\alpha) = L$

$\sigma(\alpha^p - \alpha) = \sigma(\alpha)^p - \sigma(\alpha) =
\alpha^p + 1 - (\alpha + 1) = \alpha^p - \alpha \Ra \alpha^p -
\alpha \defeql \gamma \in K$ und $X^p -X - \gamma$ ist
Minimalpolynom von $\alpha$.}
\end{Folg}

\begin{Prop}
Sei $L/K$ einfache Körpererweiterung, $L =
K(\alpha)$

\begin{enum}

\item Ist $\alpha$ Nullstelle eines Polynoms $X^n - \gamma$ für ein
$\gamma \in K$ und enthält $K$ eine primitive $n$-te Einheitswurzel
$\zeta$, so ist $L/K$ galoissch, Gal$(L/K)$ zyklisch, $d\defeqr
[L:K]$ ist Teiler von $n$, $\alpha^d \in K$, $X^d - \alpha^d$ ist
Minimalpolynom von $\alpha$

\item Ist char$(K) = p > 0$ und $\alpha \in L\setminus K$ Nullstelle
eines Polynoms $X^p - X - \gamma$ für ein $\gamma \in K$, so ist
$L/K$ galoissch und Gal$(L/K) \cong \mathbb{Z}/p\mathbb{Z}$

\end{enum}

\bew{}{\item Die Nullstellen von $X^n - \gamma$ sind
$\alpha,\zeta\alpha,\dots,\zeta^{n-1}\alpha \Ra L$ ist
Zerfällungskörper von $X^n - \gamma$, also normal und separabel,
also galoissch.

Für $\sigma \in$ Gal$(L/K)$ ist $\sigma(\alpha) =
\zeta^{\nu(\sigma)} \alpha$ für ein $\nu(\sigma) \in
\mathbb{Z}/n\mathbb{Z}$.

$\sigma \mapsto \nu(\sigma)$ ist injektiver Gruppenhomomorphismus
Gal$(L/K) \ra \mathbb{Z}/n\mathbb{Z} \Ra$ Gal$(L/K)$ ist zyklisch,
da Untergruppe von $\mathbb{Z}/n\mathbb{Z} \Ra d = [L:K]$ teilt $n$.

Für $\sigma \in$ Gal$(L/K)$ ist $\sigma(\alpha^d) =
\left(\zeta^{\nu(\sigma)}\right)^d \alpha^d = \alpha^d \Ra \alpha^d
\in K$; $X^d - \alpha^d$ ist Minimalpolynom, da $L=K(\alpha)$ und
$[K(\alpha):K] = d$.

\item Für $i \in \mathbb{F}_p$ ist $(\alpha+i)^p - (\alpha + i) -
\gamma = \alpha^p + \underset{=i}{\underbrace{i^p}} - \alpha - i -
\gamma = 0 \Ra X^p - X - \gamma$ hat $p$ verschieden Nullstellen
$\Ra L$ ist Zerfällungskörper von $X^p - X - \gamma$ und $L/K$ ist
separabel. Außerdem folgt: Gal$(L/K) \cong \mathbb{Z}/p\mathbb{Z}$ }
\end{Prop}
\section{Auflösung von Gleichungen durch Radikale}

\begin{Def}\label{radikalerweiterung}
Sei $K$ ein Körper.
\begin{enum}

\item Eine einfache Körpererweiterung $L=K(\alpha)$ heißt
\emp{elementare (oder einfache) Radikalerweiterung}, wenn entweder

\begin{enumerate}
\item[(i)] $\alpha$ ist eine Einheitswurzel.
\item[(ii)] $\alpha$ ist Nullstelle von $X^n - \gamma$ für ein
$\gamma \in K$ und char$(K) \nmid n$
\item[(iii)] $\alpha$ ist Nullstelle von $X^p - X - \gamma$ für
$\gamma \in K$, char$(K) = p$
\end{enumerate}

\item Eine endliche Körpererweiterung $L/K$ heißt
\emp{Radikalerweiterung}, wenn es eine Körpererweiterung $L'/L$ gibt
und eine Kette $K=L_0 \subset L_1 \subset \dots \subset L_n = L'$
von Zwischenkörpern, so dass $L_{i+1}/L_i$ elementare
Radikalerweiterung ist für $i=0,\dots,n-1$

\item Ist $f \in K[X]$ separabel, nicht konstant, so heißt die
Gleichung $f(X) = 0$ \emp{durch Radikale auflösbar}, wenn der
Zerfällungskörper von $f$ Radikalerweiterung ist.
\end{enum}

\bsp{$K=\mathbb{Q}$, $f(X) = X^3 - 3X + 1$

\textbf{Beh.}: Ist $\alpha$ Nullstelle von $f$, so ist
$\mathbb{Q}(\alpha)$ Zerfällungskörper von $f$, hat also Grad $3$
über $\mathbb{Q}$. $\mathbb{Q}(\alpha)/\mathbb{Q}$ ist
\textbf{keine} einfache Radikalerweiterung.

Die Nullstellen von $f$ sind: \[\ds\begin{array}{l}\alpha_1 =
e^{2\pi i/9} + e^{16\pi i / 9} \\ \alpha_2 = e^{8\pi i/9} + e^{10
\pi i/9}
\\ \alpha_3 = e^{14 \pi i / 9} + e^{4\pi i /9} \end{array}\]

Es ist $\alpha_1^2 = e^{4\pi i /9} + e^{14 \pi i/9} + 2 = \alpha_3+2
\Ra \alpha_3 \in \mathbb{Q}(\alpha_1) \Ra \alpha_2 = -\alpha_1 -
\alpha_3 \in \mathbb{Q}(\alpha_1)$}
\end{Def}

\begin{Satz}
Sei $K$ ein Körper, $f \in K[X]$ separabel, nicht konstant.
\begin{enum}

\item Die Gleichung $f(X) = 0$ ist genau dann durch Radikale
auflösbar, wenn ihre Galoisgruppe auflösbar ist (dh. $G$ hat
Normalreihe $G=G_0 \vartriangleright \dots \vartriangleright G_n =
\{e\}$ mit $G_i/G_{i+1}$ abelsch).

\item Eine endliche Körpererweiterung $L/K$ ist genau dann
Radikalerweiterung, wenn es eine endliche Galoiserweiterung $L'/K$
gibt mit $L\subseteq L'$, so dass Gal$(L'/K)$ auflösbare Gruppe ist.
\end{enum}

\bsp{ $X^5 - 4X + 2$ hat Galoisgruppe $S_5$ und ist deshalb nicht
durch Radikale auflösbar, denn $S_5 \supset A_5 \supset \{e\}$ ist
Kompositionsreihe. Nach Jordan-Hölder tritt $A_5$ in jeder
Kompositionsreihe für $S_5$ als Faktorgruppe auf.}

\sbew{
''$\Ra$'': Sei $K=L_0 \subset L_1 \subset \dots
\subset L_m$ Kette wie in Def. \ref{radikalerweiterung} (b)
mit $L \subseteq L_m$.

\textbf{Induktion über $\mathbf{m}$:}
\begin{description}
\item[m=1:] Ist $L_1/K$ vom Typ (i), so ist $L_1 = K(\zeta)$ für
eine primitive $n$-te Einheitswurzel $\zeta$ und Gal$(K(\zeta)/K)
\subseteq (\mathbb{Z}/n\mathbb{Z})^x$, also auflösbar.

Ist $L_1/K$ vom Typ (ii), so ist $L_1/K$ galoissch und Gal$(L_1/K) =
\mathbb{Z}/p\mathbb{Z}$.

Sei $L_1/K$ vom Typ (iii). Enthält $K$
eine primitive $n$-te Einheitswurzel, so ist $K(\alpha)/K$ galoissch
und Gal$(K(\alpha)/K) \cong \mathbb{Z}/n\mathbb{Z}$

Andernfalls sei $F=K(\zeta)$ der Zerfällungskörper von $X^n-1$ über
$K$ und $L_1' = L_1(\zeta) = F(\alpha) = F \cd L_1$ das
''\emp{Kompositum}'' von $F$ und $L_1$.

$L_1'$ ist galoissch über $K$ (Zerfällungskörper von $X^n - \gamma$
über $K$) und es gibt exakte Sequenz
\[ 1 \ra \underset{\mbox{\scriptsize zyklisch}}{\underbrace{\mbox{Gal}(L_1'/F)}}
\ra \mbox{Gal}(L_1'/K) \ra \underset{\mbox{\scriptsize
abelsch}}{\underbrace{\mbox{Gal}(F/K)}} \ra 1 \]

$\Ra$ Gal$(L_1'/K)$ auflösbar.

\item[m$>$1:] Eine endliche Körpererweiterung heißt \emp{auflösbar},
wenn es eine endliche Erweiterung $L'/L$ gibt, so dass $L'/K$
galoissch und Gal$(L'/K)$ auflösbar ist.

Nach Induktionsvoraussetzung ist $L_{m-1}/K$ auflösbar. Außerdem ist
$L_m/L_{m-1}$ auflösbar. (m=1)

zu zeigen also: Sind $K \subset \underset{=L_{m-1}}{\underbrace{L}}
\subset \underset{=L_m}{\underbrace{M}}$ Körpererweiterungen und ist
$L/K$ auflösbar und $M/L$ auflösbar, so ist $M/K$ auflösbar.

Seien dazu $L'/L$ und $M'/M$ Erweiterungen wie in Def.:

%\[\begindc{\undigraph} \obj(1,1){$K$}[\south]
%                      \obj(2,1){$L$}[\south]
%                      \obj(3,1){$M$}[\south]
%                      \obj(4,1){$M'$}[\east]
%                      \obj(2,2){$L'$}[\north]
%                      \obj(4,2){$L'M'$}[\north]
%                      \mor{$K$}{$L$}{}
%                      \mor{$L$}{$M$}{}
%                      \mor{$M$}{$M'$}{}
%                      \mor{$L$}{$L'$}{}
%                      \mor{$M'$}{$L'M'$}{}
%                      \mor{$L'$}{$L'M'$}{}
%                      \mor{$K$}{$L'$}{gal.}
%                      \cmor((2,1)(3,0)(4,1))
%                            \pup(4,0){gal.}
%\enddc\]

\textbf{Beh.}: $L'M'/L'$ ist galoissch und Gal$(L'M'/L)$ ist
auflösbar.

\textbf{denn}: Nach Voraussetzung ist $M'/L$ galoissch, also
Zerfällungskörper eines Polynoms $f \in L[X] \Ra M'L'$ ist
Zerfällungskörper von $f \in  L'[X]$ über $L'$.

Außerdem: Gal$(L'M'/L') \ra$ Gal$(M'/L)$, $\sigma \mapsto
\sigma_{|M'} \overset{(!)}{\in}$ Gal$(M'/L)$ ist wohldefiniert
und injektiv: Ist $\sigma_{|M'} = id_{M'}$, so ist $\sigma =
id_{L'M}$, da $\sigma_{|L'} = id_{L'}$ nach Voraussetzung.

Also \OE $L=L'$, $L'M' = M$.

\item[m$>$1 (Forts.)] Ist $M/K$ galoissch, so ist Gal$(M/K)$
auflösbar, da dann \[ 1 \ra \underset{\mbox{\scriptsize
auflösbar}}{\underbrace{\mbox{Gal}(M/K)}} \ra \mbox{Gal}(M/K) \ra
\underset{\mbox{\scriptsize
auflösbar}}{\underbrace{\mbox{Gal}(L/K)}} \ra 1\] exakt ist.

Andernfalls sei $\wt{M}/M$ (minimale) Erweiterung, so dass $\wt{M}/K$
galoissch ist. $\wt{M}$ wird (über $K$) erzeugt von den $\sigma(M)$,
$\sigma \in$ Hom$_K(M,\bar K)$. ($\bar K$ fest gewählter
algebraischer Abschluss von $K$) Für jedes $\sigma \in$ Hom$_K(M,\bar
K)$ ist $\sigma(M)$ Galoiserweiterung von $\sigma(L) = L$.

Dann ist \[\begin{array}{ccc} \mbox{Gal}(\wt{M}/L) &\ra
&\prod_{\sigma \in \mbox{
\scriptsize Hom}_K(M,\bar K)}\mbox{ Gal}(\sigma(M)/L) \\
\tau &\mapsto &(\tau_{|\sigma(M)})_\sigma \end{array}\] injektiver
Gruppenhomomorphismus.

Für jedes $\sigma \in$ Hom$_K(M,\bar K)$ ist Gal$(\sigma(M)/L)
\cong$ Gal$(M/L)$, also auflösbar $\Ra \prod_\sigma$
Gal$(\sigma(M)/L)$ ist auflösbar. (!) $\Ra$ Gal$(\wt{M}/L)$
auflösbar (als Untergruppe einer auflösbaren Gruppe) $\Ra$
Gal$(\wt{M}/K)$ ist auflösbar wegen $1 \ra$ Gal$(\wt{M}/L) \ra$
Gal$(\wt{M}/K) \ra$ Gal$(L/K) \ra 1$ exakt.
\end{description}

''$\Leftarrow$'':
\newline $G \defeqr$ Gal$(L'/K)$ sei auflösbar, $G
= G_0 \supset G_1 \supset \dots \supset G_m = \{1\}$ Normalreihe, so
dass $G_{i+1}$ Normalteiler in $G_i$ und $G_i/G_{i+1} \cong
\mathbb{Z}/p\mathbb{Z}$ mit Primzahlen $p_i,\;i=0,\dots,m-1$ ist.
\newline \newline Dazu gehört eine Kette von Zwischenkörpern $K = K_0 \subset K_1
\subset \dots K_m = L'$, in der $K_i/K_{i-1}$ Galoiserweiterung ist
und Gal$(K_i/K_{i-1}) \cong \mathbb{Z}/p_i\mathbb{Z}$.
\newline \newline Fall 1: Ist $p_i =$ char$(K)$, so ist $K_i/K_{i-1}$
elementare Radikalerweiterung vom Typ (iii), also Minimalpolynom der Form $X^{p_i}-X-\gamma$.

Fall 2: Ist $p_i \neq$ char$(K)$, so ist $K_i/K_{i-1}$ vom Typ (ii), \textbf{falls}
$K_{i-1}$ eine primitive $n$-te Einheitswurzel $\zeta$ enthält.

Fall 3: $p_i\ne \operatorname{char}(K)$, $K_{i-1}$ enthält keine primitive Einheitswurzel. Sei also \[d \defeqr \prod_{\substack{\text{$p$ prim} \\ p \mid |G|}}p\] und $F$ der Zerfällungskörper von $X^d - 1$ über $K$.
$\Ra F/K$ ist Erweiterungskörper vom Typ (i).

Sei $\wt{L} = F L' \Ra \wt{L}/F$ ist Galoiserweiterung (siehe
hier ausgelassenes Diagramm). Die Abbildung $\operatorname{Gal}(\tilde L/F) \to \operatorname{Gal}(L' / K)$, $\sigma \mapsto \sigma|_{L'}$, ist injektiver Gruppenhomomorphismus, also ist $\operatorname{Gal}(\tilde L/F)$ auflösbar und $|\operatorname{Gal}(\tilde L,F)|$ teilt $|G|$. Erhalte Kette $K \subset F \subset F_1 \subset \dots \subset
F_r = \wt{L}$ von Zwischenkörpern, $F_i/F_{i-1}$ Galoiserweiterung,
Gal$(F_i/F_{i-1}) \cong \mathbb{Z}/p_i\mathbb{Z}$ elementare
Radikalerweiterung vom Typ (ii).}
\end{Satz}

%\appendix
%\renewcommand{\indexname}{Stichwortverzeichnis}
%\addcontentsline{toc}{chapter}{Stichwortverzeichnis}
%\printindex

\end{document}

