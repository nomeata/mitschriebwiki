\documentclass{article}
\usepackage[utf8]{inputenc}
\usepackage{amsmath}
\usepackage{amsfonts}
\usepackage{amssymb}
\usepackage{amsthm}
\usepackage{mathrsfs}
\usepackage{german}
\usepackage{enumerate}
\usepackage{stmaryrd}
\title{4. Topologie Übung}
\author{Ferdinand Szekeresch}

\DeclareMathOperator{\so}{SO}
\DeclareMathOperator{\gl}{GL}
\DeclareMathOperator{\id}{id}

\begin{document}
\maketitle

\textbf{Aufgabe 1}\\
$(X,\leq),X$ Menge, $\leq$ Ordnungsrelation auf $X$.\\
Beh.: $O:=\{U\subseteq X|\forall u\in U,x\in X:u\leq x \Rightarrow x\in U\}$ ist Topologie auf $X$.\\
Bew.: \begin{itemize}
\item $\emptyset, X$ klar.
\item Seien $U_i$ Mengen aus $O$, $i\in I$ bel. Indexmenge. Dann gilt:\\
Sei $u\in\bigcup\limits_{i\in I}u_i,x\in X$ mit $u\leq x$ $\Rightarrow\exists i\in I: u\in U_i\\
\stackrel{U_i\in O}{\Rightarrow}x\in U_i\Rightarrow x\in\bigcup\limits_{i\in I}U_i\\
u\in\bigcap\limits_{i\in I}U_i,x\in X$ mit $u\leq x\Rightarrow\forall i\in I:x\in U_i\Rightarrow x\in\bigcap\limits_{i\in I}U_i$
Beh.: Die Ordnungserhaltenden und die stetigen Abbildungen von $X$ nach $Y$ stimmen überein.\\
Bew.:\begin{itemize}\item[\glqq$\supseteq$\grqq:] Sei $F:X\rightarrow Y$ stetig, $x_1,x_2\in X$ mit $x_1\leq x_2$. Zu zeigen: $f(x_1)\leq f(x_2)$.\\
Betrachte folgende offene Menge in $Y$:
$$V:=\{u\in Y|f(x)\leq y\}$$
Das Urbild $f^{-1}(V)=:U$ ist eine offene Menge in $X$, da $f$ stetig ist.\\
$\Rightarrow\forall u\in U\forall x\in X: u\leq x\stackrel{x_1\in U}{\Rightarrow}{x_1\leq x_2}x_2\in U\Rightarrow f(x_2)\in V$\\
$\Rightarrow\stackrel{\text{Def. }V}{\Rightarrow}f(x_1)\leq f(x_2)$. Also: $f$ ist abstandserhaltend.
\item[\glqq$\subseteq$\grqq:] Sei $f$ ordungserhaltend. Sei $V$ eine offene Menge in $Y$. Zu zeigen: $U:=f^{-1}(V)$ ist offen in $X$.\\
Seien also $u\in U,x\in X$ mit $u\leq x\stackrel{f\text{ ordnungserh.}}{\Rightarrow}f(u)\leq f(x)$\\
Da $V$ offen in $Y$ ist und $f(u)\in V$ ist, nach Definition von \glqq offen\grqq ist auch $f(x)\in V\Rightarrow x\in U$.\\
Also: $U$ ist offen in $X$.\\
\end{itemize}
$\blacksquare$
\end{itemize}

\textbf{Aufgabe 2}\\
Beh.: $\so(n):=\{A\in\mathbb{R}^{n\times n}|\det(A)=1\text{ und }A^TA=E\}$ ist zusammenhängend.\\
Bew.: $\so(n)$ ist wegzusammenhängend, denn:\\
Lineare Algebra: Zu jedem $A\in\so(n)$ existiert $U\in\gl(n,\mathbb{R})$ mit:
$$A=U\cdot\left(\begin{array}{ccccc}1 & & & & 0\\ & \ddots & & & \\ & & D_{\theta_1} & & \\ & & & \ddots & \\ 0 & & & & D_{\theta_m}\end{array}\right)\cdot U^{-1}$$
und $D_{\theta_i}=\left(\begin{array}{cc}\cos(\theta_i) & -\sin(\theta_i) \\ \sin(\theta_i) & \cos(\theta_i)\end{array}\right)$ Somit: Definiere Weg von $E$ nach $A$ durch $\gamma:[0,1]\rightarrow\so(n)$ 
$$t\mapsto U\cdot \left(\begin{array}{ccccc}1 & & & & 0 \\ & \ddots & & & \\ & & D_{\theta_1(t)} & & \\ & & & \ddots & \\0 & & & & D_{\theta_m(t)}\end{array}\right)\cdot U^{-1}$$ mit $\theta_i(t):=t\cdot\theta_i$\\
Das ist ein Weg von $E$ nach $A$! Also: $\so(n)$ ist zusammenhängend (da wegzusammenhängend).\\
Beh.: $O(n):=\{A\in\mathbb{R}^{n\times n}|A^TA=E\}$ ist nicht zusammenhängend.\\
Bew.: es gilt: $\det:O(n)\rightarrow\{-1,1\}$ ist stetig (Leibniz-Formel) und surjektiv.\\
$\Rightarrow O(n)=f^{-1}(-1)\cup f^{-1}(1)\Rightarrow O(n)$ ist nicht zusammenhängend\\
$\blacksquare$\\

\textbf{Aufgabe 3}\\
\begin{enumerate}[(a)]
\item Beh.: Für $U\subseteq\mathbb{R}^n$ offen gilt: $U$ zusammenhängend $\Leftrightarrow U$ wegzusammenhängend.
Bew.:\glqq$\Leftarrow$\grqq klar.\\
\glqq$\Rightarrow$\grqq: Definiere Äquivalenzrelation $\sim$ auf $\mathbb{R}^n$ durch $x\sim y:\Leftrightarrow x$ kann mit $y$ durch Weg verbunden werden.\\
Sei $U\neq\emptyset$ zusammenhängende offene Teilmenge von $\mathbb{R}^n$. Zu zeigen: $U$ ist wegzusammenhängend.\\
Wähle $a\in U$ beliebig und setze $A:=\{x\in U|x\sim a\}$. Zu zeigen: $A=U$.\\
Dazu zeige: $A$ ist offen und abgeschlossen in $U$ bzgl. der Teilraumtopologie. (Dann folgt $A=U$ oder $A=\emptyset$, weil $U$ zsh. $\Rightarrow A=U$, da $a\in A$ ist.)\\
Beh.: $A$ ist offen in $U$.\\
Bew.: $U$ offen in $\mathbb{R}^n, A\subseteq U\Rightarrow\forall x\in A\exists\varepsilon>0 B_\varepsilon(x)\subseteq U$\\
$B_\varepsilon(x)$ ist konvex $\Rightarrow$ jedes $y\in B_\varepsilon(x)$ ist durch einen Weg mit $x$ verbindbar.\\
$\forall y\in B_\varepsilon(x):x\sim y$.\\
$x\in A\Rightarrow x\sim a\stackrel{\sim\text{ transitiv}}{\Rightarrow}y\sim a\Rightarrow y\in A$. Also: $B_\varepsilon(x)\subseteq A\Rightarrow A$ ist offen.\\
Beh.: $A$ ist abgeschlossen un $U$.\\
bew.: Sei $x\in\bar A$, dem Abschluss von $A$ bzgl. der Teilraumtopologie.\\
$x\in U\Rightarrow\exists\varepsilon>0:B_\varepsilon(x)\subseteq U x\in\bar A\stackrel{\text{Blatt 3}}{\Rightarrow}\emptyset\neq(B_\varepsilon(x)\cap U)\cap A = B_\varepsilon(x)\cap A$.\\
Wie eben gilt für $y\in B_\varepsilon(x)\cap A:y\sim x$.\\
Wegen $y\in A$ gilt auch $y\sim a\stackrel{y\text{ transitiv}}{\Rightarrow}x\sim a\rightarrow x\in A\Rightarrow\bar A=A\Rightarrow A$ abgeschlossen.\\
$\blacksquare$
\item $U:=\{(x,\sin\left(\frac{1}{x}\right)|x>0\}\cup\{(0,0)\}$\\
$U\backslash\{(0,0)\}$ ist zsh., da $U\backslash\{(0,0)\}$ Bild von $(0,1]$ unter der stetigen Abbildung
\begin{align*}
(0,1]&\rightarrow\mathbb{R}^2\\
x&\mapsto(x,\sin\left(\frac1x\right))
\end{align*}

$(0,0)$ liegt im Abschluss von $U\backslash\{(0,0)\}$, da jede Umgebung von $(0,0)$ einen Punkt aus $U\backslash(0,0)$ enthält.\\
$\stackrel{\text{Blatt 3}}{\Rightarrow}U$ ist zusammenhängend.\\
Ann.: U ist wegzsh. $\Rightarrow$ es ex. ein stetiger Weg $\gamma:[0,1)\rightarrow U,\gamma(0)=(0,0),\gamma(1)=(1,0)$.\\
Sei $\gamma(t)=(\gamma_1(t),\gamma_2(t))\Rightarrow\gamma_1$ ist ebenfalls stetig.\\
Zwischenwertsatz $\Rightarrow\forall y\in(0,1)\exists t\in(0,1):\gamma(t)=y$\\
Sei insbes. $y=\frac{1}{n}\Rightarrow\exists t_n\in (0,1): \gamma_1(t_n)=\frac{1}{n}\Rightarrow\gamma(t_n)=(\gamma_1(t_n),\gamma_2(t_n))=(\frac{1}{n},\sin(n))$\\
$\gamma$ ist stetig $\Rightarrow\gamma(t_n)\stackrel{n\rightarrow\infty}{\longrightarrow}(0,0)\lightning$
\end{enumerate}

\textbf{Aufgabe 4}\\
$X$ Top. Raum, $\sim$ Äquivalenzrel. auf $X$.\begin{enumerate}[(a)]
\item $Y:=X/\sim$ sei versehen mit der Quotiententopologie.\\
Beh.: Ist $Z$ weiterer top. Raum und $f: Y\rightarrow Z$, dann ist $f$ stetig $\Leftrightarrow f\circ\pi$ stetig.
\glqq$\Rightarrow$\grqq Sei $f:X/\sim\rightarrow Z$ stetig. Dann ist $f\circ\pi$ Verkettung stetiger Abbildungen, also stetig.\\
\glqq$\Leftarrow$\grqq Sei $f\circ\pi$ stetig und $U\subseteq Z$ offen.\\
$\Rightarrow\pi^{-1}(f^{-1}(U)) = (f\circ\pi)^{-1}(U)$ offen in $X \Rightarrow f^{-1}$ ist offen in $X/\sim\Rightarrow f$ ist stetig.
\item Beh.: Durch (a) ist die Quotiententopologie eindeutig bestimmt.\\
Bew.: Seien $J_1,J_2$ zwei Topologien auf $X/\sim$, die obige Eigenschaften erfüllen. Z.z. $J_1=J_2$.\\
Betrachte $\id_{X/\sim}:(X/\sim,J_1)\rightarrow(X/\sim,J_2)$. Nach obiger Eigenschaft ist $\id$ stetig $\Rightarrow$ alle $U\in J_2$ sind in $J_1$ enthalten.\\
$J_1\subseteq J_2$.\\
Analog umgekehrt,

\end{enumerate}


\end{document}
