\documentclass{article}
\usepackage{info}
\usepackage[left=2cm,right=2cm,top=1.5cm,bottom=2cm]{geometry}

\title{Vereinbarungen für Info2}
\author{Felix Brandt, Mathias Ziebarth}
% Wer nennenswerte Änderungen macht, schreibt sich bei \author dazu

\begin{document}

\def\A{\mathcal{A}}
\def\U{\mathcal{U}}
\def\I{\mathcal{I}}

\begin{center}
	Dieses Dokument erhebt keinen Anspruch auf Richtig- oder Vollständigkeit!\\
	Alle Angaben also ohne Gewähr!
\end{center}
	
\section{Syntax der Prädikatenlogik}
	\subsection{Terme}
		\begin{tabular}{ll}
			Variablen        & $X$, $Y$, $Z$ oder $X_1$, $X_2$, $X_3$, $\ldots$\\
			Konstanten       & $a$, $b$, $c$ oder $a_1$, $a_2$, $a_3$, $\ldots$\\
			Funktionssymbole & $f$, $g$, $h$ oder $f_1$, $f_2$, $f_3$, $\ldots$\\
											 &Sind $t_1, \ldots, t_n$ Terme, dann auch $f(t_1, \ldots, t_n)$.\\
		\end{tabular}
	\subsection{Formeln}
		\begin{tabular}{ll}
			Prädikatensymbole & $P$, $Q$, $R$ oder bspw. $P_1$, $Q_2$, $R_5$, $\ldots$\\
												&Sind $t_1, \ldots, t_n$ Terme, dann ist auch $P(t_1, \ldots, t_n)$ eine atomare Formel.\\
												&Beispiele: $(F \wedge G)$, $(F \vee G)$, $\forall xF$, $\exists xF$, $\neg F$\\
		\end{tabular}

\section{Semantik der Prädikatenlogik}
	\subsection{Struktur}
		$$\A(\U_\A, \I_\A)$$	
		
		\begin{description}
		\item[$\U_\A$] nichtleere Menge (Universum)
		
		\item[$\I_\A$] eine Abbildung die
			\begin{itemize}
				\item jedem Prädikatensymbol ein Prädikat
				\item jedem Funktionssymbol eine Funktion
				\item jeder Variablen $X$ ein Element der Grundmenge $\U_\A$
			\end{itemize}	
			zuordnet.
		\end{description}
			
			Falls $\I_\A$ für alle Symbole in $F$ definiert ist so "`passt"' $\A$ zu $F$.
			Ist $F$ eine Formel und $\A$ passt zu $F$ dann sei
			
			\begin{enumerate}
				\item Falls $F$ die Form $F=P(t_1, \ldots, t_k)$ mit Termen $t_1, \ldots, t_k$, so ist\\ $\A(F)=
					\begin{cases}
					1 & \text{falls } ((\A(t_1), \ldots, \A(t_k)) \in P^\A) \\
					0 & \text{sonst }
					\end{cases}$
				\item Falls "`$F = \neg G$"' hat, so ist\\ $\A(F)=
					\begin{cases}
					1 & \text{falls $\A(G)=0$} \\
					0 & \text{sonst}
					\end{cases}$
				\item Falls "`$F=(G \begin{array}{c}\wedge\\\vee\end{array} H)$"' so ist\\ $\A(F)=
					\begin{cases}
					1 & \text{falls }\A(G)=1 \begin{array}{c}\text{und}\\\text{oder}\end{array} \A(H)=1 \\
					0 & \text{sonst}
					\end{cases}$
				\item Falls "`$F=\forall xG$"' so ist\\ $\A(F)=
					\begin{cases}
					1 & \text{falls für alle $d \in \U_\A$ gilt $\A_{[x/d]}(G)=1$} \\
					0 & \text{sonst}
					\end{cases}$
				\item Falls "`$F=\exists xG$"' so ist\\ $\A(F)=
					\begin{cases}
					1 & \text{falls es ein $d \in \U_\A$ gibt, mit $\A_{[x/d]}(G)=1$}\\
					0 & \text{sonst}
					\end{cases}$
				\end{enumerate}

% ----------------------
% O-Kalkül Konventionen: 
% ----------------------

\section{O-Kalkül Vereinbarungen}

Zugelassen sind nur asymptotisch monoton wachsende Funktionen. \\
(Hinweis: Polynome besitzen eine endliche Anzahl an Nullstellen, sind also ab einem $n_0$ monoton) \\
Dann gilt: $O(f)*O(g)=O(f*g)$ und $O(f+c)=O(f)$.

\end{document}
