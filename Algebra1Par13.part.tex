\section{Quotientenbildung}

\begin{DefBem}
    \label{restklassendefinition}
    Sei $f: M \ra M'$ eine Abbildung von Mengen.
    
    \begin{enum}
        \item Die Relation $\sim_f$ auf $M: x \sim_f y \lra f(x) = f(y)$ ist
        eine Äquivalenzrelation.

        \item Für $x \in M$ sei $\bar x \defeqr [x]_f \defeqr \{ y \in M : y\sim_f x\} = \{ y\in M: f(y)= f(x)\}$.
	Es ist $\bar x = f^{-1}(f(x))$
	
	Weiter sei $\bar M \defeqr M/\sim_f \defeqr \{ \bar x : x \in M\}$

	\item $\bar f : \bar M\to \mbox{Bild}(f)$, $\bar x\mapsto f(x)$ ist eine bijektive Abbildung.

    \end{enum}
\end{DefBem}

\begin{Def}
        Ist $(M,\cd)$ und $(M',\ast)$ ein \bla, und $(M,\cd) \ra (M', \ast)$ ein Homomorphismus, so wird durch
        $\bar x \bar y \defeqr \overline{xy}$ eine Verknüpfung auf $\bar M$ 
        definiert. So wird $(\bar M,\cd)$ auch zu einem \bla.\newline
        \sbew{1.0}{\newline z.z.: $\cd$ ist wohldefiniert. 
        Seien also $x' \in \bar x, y' \in \bar y$ zu zeigen: $\overline{x'\cdot y'} =
        \overline{x\cdot y}$ dh. $f(x'\cdot y') = f(x\cdot y)$ dh. $f(x')=f(x), f(y') = f(y)$ Es ist
        $f(x'\cdot y') = f(x') \ast f(y') = f(x) \ast f(y) = f(x\cdot y)$}
\end{Def}

\begin{DefBem}
\label{1.14}
    \label{gruppenfaktorgruppe}
    Sei $f:G \ra G'$ Gruppenhomomorphismus.
    
    \begin{enum}
        \item $\bar G = G/\sim_f$ ist
        die Menge der Linksnebenklassen bzgl. $Kern(f)$ also ist für jedes $g\in G$: $[g]_f = g\cdot \Kern{}(f) = \Kern{}(f)\cdot g$.

        \item  $\bar G = G/\Kern(f)$ heißt \emp{Faktorgruppe} von $G$ bzgl.
        $\Kern(f)$. \newline
        \sbew{1.0}{\newline Seien $x,y \in G$. Dann gilt: $\bar x = \bar y \lra
        f(x) = f(y) \lra f(x) \cdot f(y^{-1}) = e' \lra xy^{-1} \in \Kern{}(f)
        \lra y=(xy^{-1})^{-1} x \in \Kern{}(f) \cd  x \lra x^{-1}y \in
        \Kern{}(f) \lra y = x(x^{-1}y) \in x \cd \Kern{}(f) \lra y \cd
        \Kern{}(f) = x \cd \Kern{}(f)$}
    \end{enum}
\end{DefBem}

\textbf{Beispiel:} $\exp:(\mathbb R, +) \to (\mathbb C^\times, \cdot)$, $t\mapsto e^{2\pi i t}$ ist ein Gruppenhomomorphismus. Es ist $\exp(t_1) = \exp(t_2) \iff e^{2\pi i(t_2-t_1)} \iff t_2=t_1 \in \mathbb Z$, also ist $\Kern(\exp) = \mathbb Z$

Die Abbildung $[0,1)\to\mathbb R/\mathbb Z$, $t \mapsto [t]_f$ ist bijektiv, spiegelt aber die Eigenschaften dieser Gruppe nicht wieder. Besser geeignet ist die Bijektion $\mathbb R/\mathbb Z$, $\bar t \mapsto e^{2\pi i t}$.

\begin{Bem}
    Sei $G$ Gruppe. Es ist $N \subseteq G$ Normalteiler, genau dann, wenn es eine Gruppe $G'$ mit einem surjektivem Gruppenhomomorphismus $f:G \ra G'$ und $N
    =\Kern(f)$ gibt. \newline
    
    \bew{Die Richtung $\impliedby$ folgt aus \ref{kernundnormalteiler} f). 
    Sei $G' \defeqr \{x \cd N, x\in G\}$ $(\subseteq \mathcal{P}(G))$ Für
    $x,y \in G$ setze $(x \cd N)(y \cd N) = (xy \cd N)$ \newline
    \textit{Behauptung:} $({G}', \cd)$ ist Gruppe, \textit{\textbf{denn:}}}{ 
    
    \item[(i)] Die Verknüpfung ist wohldefiniert: Seien $x, x', y, y' \in G$ mit
    $x \cd N = x' \cd N,\; y \cd N = y' \cd N$. Dann gibt es $n,m \in N
    \mbox{ mit } x'=xn,
    y'=ym \Ra x',y' = x(ny)m$. Da $N$ Normalteiler ist, gibt es $n' \in N$ mit
    $ny=yn' \Ra x'y' = xyn'm \Ra x'y' \cd N = xy \cd N$
    
    \item[(ii)] alle übrigen Eigenschaften ''vererben'' sich von $G \mbox{ auf }
    G'\\$ $f: G \ra G',\; x \mapsto x \cd N$ ist surjektiver 
    Gruppenhomomorphismus mit Kern$(f)$ = N }
\end{Bem}
    
\begin{DefBem}
    Sei $G$ Gruppe, $N\subset G$ Normalteiler. Die Gruppe $G'$ aus dem vorherigen Beweis heißt Faktorgruppe von $G$ nach $N$, und wir schrieben $G' = G/N$ („$G$ modulo $N$“). Sie ist gleich der Faktorgruppe $G/\Kern(f)$ für das $f$ aus der  vorherigen Bemerkung (ii).
\end{DefBem}

\begin{Satz}
\label{Satz 1}
\mbox{}
\begin{enum}
\item 
Sei $f: M \ra M'$ eine Abbildung. $\bar M \defeqr
M/\sim_f$ und $p: M \ra \bar M, x \mapsto \bar x$ die
Restklassenabbildung.
Dann exisitiert genau eine Abilldung $\bar f: \bar M \to M'$ mit $f = \bar f\circ p$. Es ist $p$ surjektiv und $\bar f$ injektiv.

\item Ist $f:M \to M'$ ein Homomorphismus von \bla, so ist $\bar M$ auch ein \bla und $p$, $\bar f$ sind Homomorphismen.

\item \emp{Homomorphiesatz} \newline
Ist $f:G \to G'$ ein Gruppenhomomorphismus, so ist $G/\Kern f \cong \Bild f$
%\[\begindc{\commdiag} \obj(1,3){$M$}
%                      \obj(3,3){$M'$}
%                      \obj(2,1){$M_2$}
%                      \mor{$M$}{$M'$}{$f$}[-1,0]
%                      \mor{$M$}{$M_2$}{$p$}[-1,0]
%                      \mor{$M_2$}{$M'$}{$\overline{f}$}[-1,1]
%\enddc\]
\item \emp{Universelle Abbildungseigenschaft (UAE) der Faktorgruppe} \newline
Sei $G$ Gruppe, $N \subseteq G$ Normalteiler. Dann gibt es zu jedem
Gruppenhomomorphismus $f:G \ra G'$ mit $N \subseteq$ Kern$(f)$ genau
einen Gruppenhomomorphismus $f_N: G/N \ra G' \mbox{ mit } f = f_N \circ p_N$, wobei $p_N$ die Restklassenabbildung ist.
\end{enum} \noindent

\bew{}{
\item
$\bar f(\bar x)= f(x)$, wie in \ref{restklassendefinition} c)
\item $\bar f : G / \Kern f \to G'$ ist injektiv, ein Gruppenhomomorphismus nach a), b) und \ref{gruppenfaktorgruppe}. Also ist $\bar f$ ein bijektiver Homomorphiesatz, also eine Isomorphie.
\item Setze $f_N(x \cd N) \defeqr f(x)$ \newline
$f_N$ ist wohldefiniert: Ist $gN = g'N$, so ist $(g')^{-1}g \in N \subset \Kern f$, also $f( (g'){^-1} g) = e' \implies f(g') = f(g')$. Die Eindeutigkeit von $\bar f$, sowie dass $\bar f$ ein Homomorphismus ist, ist klar.
}
\end{Satz}
