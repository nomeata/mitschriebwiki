\section{Quotientenbildung}

\begin{DefBem}
    Sei $f: M \ra M'$ eine Abbildung von Mengen.
    
    \begin{enum}
        \item Die Relation $\sim_f$ auf $M: x \sim_f y \lra f(x) = f(y)$ ist
        eine Äquivalenzrelation.

        \item Für $x \in M$ sei $\bar x \defeqr \{ y \in M : y\sim_f x\}$ Es ist
        $\bar x = f^{-1}(f(x))$ $\bar M \defeqr M/\sim_f \defeqr \{ \bar x : x \in M\}$

        \item Ist $f:(M,\cd) \ra (M', \ast)$ ein Homomorphismus, so wird durch
        $\bar x \bar y \defeqr \overline{xy}$ eine Verknüpfung auf $\bar M$ 
        definiert.\newline
        \sbew{0.9}{\newline z.z.: $\cd$ ist wohldefiniert. 
        Seien also $x' \in \bar x, y' \in \bar y$ zu zeigen: $\overline{x'y'} =
        \overline{xy}$ dh. $f(x'y') = f(xy)$ dh. $f(x')=f(x), f(y') = f(y)$ Es ist
        $f(x'y') = f(x') \ast f(y') = f(x) \ast f(y) = f(xy)$}

        \item Ist $(M,\cd)$ \bla, so auch $(\bar M, \cd)$
    \end{enum}
\end{DefBem}

\begin{DefBem}
\label{1.14}
    Sei $f:G \ra G'$ Gruppenhomomorphismus.
    
    \begin{enum}
        \item $\bar G = G/\sim_f$ ist
        die Menge der Linksnebenklassen bzgl. Kern$(f)$

        \item  $\bar G \defeql G$/Kern$(f)$ heißt \emp{Faktorgruppe} von G bzgl.
        Kern$(f)$. \newline
        \sbew{0.9}{\newline Seien $x,y \in G$. Dann gilt: $\bar x = \bar y \lra
        f(x) = f(y) \lra f(x) = f(y^{-1}) = e' \lra xy^{-1} \in \mbox{Kern}(f)
        \lra y=(xy^{-1})^{-1} x \in \mbox{Kern}(f) \cd  x \lra x^{-1}y \in
        \mbox{Kern}(f) \lra y = x(x^{-1}y) \in x \cd \mbox{Kern}(f) \lra y \cd
        \mbox{Kern}(f) = x \cd \mbox{Kern}(f)$}
    \end{enum}
\end{DefBem}

\begin{Bem}
    Sei $G$ Gruppe, $N \subseteq G$ Normalteiler. Dann gibt es eine Gruppe $\bar
    G$ und einen surjektiven Gruppenhomomorphismus $f:G \ra \bar G$ mit $N
    =$Kern$(f)$. \newline
    
    Folgerung: Nach \ref{1.14} ist dann $\bar G \cong G/\mbox{Kern}(f)
    \defeql G/N$.
    Man kann also nach jedem Normalteiler eine Faktorgruppe bilden. \newline
    
    \bew{Sei $\bar G \defeqr \{x \cd N, x\in G\} (\subseteq \mathcal{P}(G))$ Für
    $x,y \in G$ setze $(x \cd N)(y \cd N) = (xy \cd N)$ \newline
    \textit{Behauptung:} $(\bar{G}, \cd)$ ist Gruppe, \textit{\textbf{denn:}}}{ 
    
    \item[(i)] Die Verknüpfung ist wohldefiniert: Seien $x, x', y, y' \in G$ mit
    $x \cd N = x' \cd N,\; y \cd N = y' \cd N$. Dann gibt es $n,m \in N
    \mbox{ mit } x'=xn,
    y'=ym \Ra x',y' = x(ny)m$. Da $N$ Normalteiler ist, gibt es $n' \in N$ mit
    $ny=yn' \Ra x'y' = xyn'm \Ra x'y' \cd N = xy \cd N$
    
    \item[(ii)] alle übrigen Eigenschaften ''vererben'' sich von $G \mbox{ auf }
    \bar G\\$ $f: G \ra \bar G,\; x \mapsto x \cd N$ ist surjektiver 
    Gruppenhomomorphismus mit Kern$(f)$ = N }
\end{Bem}

\begin{Satz}
\label{Satz 1}
\mbox{}
\begin{enum}
\item \emp{Homomorphiesatz} \newline
Sei $f: M \ra M'$ Homomorphismus von \blab, \newline $\bar M \defeqr
M/\sim_f$, $p: M \ra \bar M, x \mapsto \bar x$ die
Restklassenabbildung.
\begin{enumerate}
\item[(i)] $p$ ist surjektiver Homomorphismus.
\item[(ii)] $\exists!$ Homomorphismus $\bar f: \bar M \ra M'$ mit $f=\bar f
\circ p$
\item[(iii)] $\bar f$ ist injektiv. Ist $f$ surjektiv, so ist $\bar{f}$ bijektiv.
\end{enumerate}

%\[\begindc{\commdiag} \obj(1,3){$M$}
%                      \obj(3,3){$M'$}
%                      \obj(2,1){$M_2$}
%                      \mor{$M$}{$M'$}{$f$}[-1,0]
%                      \mor{$M$}{$M_2$}{$p$}[-1,0]
%                      \mor{$M_2$}{$M'$}{$\overline{f}$}[-1,1]
%\enddc\]
\item \emp{Universelle Abbildungseigenschaft (UAE)} \newline
Sei $G$ Gruppe, $N \subseteq G$ Normalteiler. Dann gibt es zu jedem
Gruppenhomomorphismus $f:G \ra G'$ mit $N \subseteq$ Kern$(f)$ genau
einen Gruppenhomomorphismus $\bar f: G/N \ra G' \mbox{ mit } f = \bar
f \circ p$ \end{enum} \noindent

\bew{}{
\item
\begin{enumerate}
\item[(i)] $\chk$
\item[(ii)] Setze $\bar f(\bar x) = f(x)$ (einzige Möglichkeit) $\Ra
\bar f$ ist eindeutig, sofern es existiert. \newline $\bar f$ ist
wohldefiniert: Ist $y \in \bar x$, also $y \sim_f x \Ra f(y) = f(x)
\Ra \bar f(\bar y) = f(y) = f(x) = \bar f(\bar x)\\ \bar f$ ist
Homomorphismus: $\bar f(\bar x \bar y) = f(x y) \overset{Hom.}{=}
f(x) f(y) = \bar f(\bar x) \bar f(\bar y)$
\item[(iii)] $\chk$
\end{enumerate}
\item Setze $\bar f(x \cd N) \defeqr f(x)$ (Eindeutigkeit wie in (a)) \newline
$\bar f$ wohldefiniert: Sei $y \in G$ mit $y \cd N = x \cd N \Ra y=xn$ für ein
$n \in N \subseteq$ Kern$(f) \Ra f(y) = f(xn) = f(x) f(n) =
f(x) \Ra \mbox{Homomorphismus } \chk$}
\end{Satz}
