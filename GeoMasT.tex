\documentclass[a4paper,twoside,DIV15,BCOR12mm]{scrbook}

\usepackage{mathe}
\usepackage{saetze-hug}
\usepackage{enumerate}
\usepackage{dsfont}

\newcommand{\A}{\mathcal A}
\newcommand{\ind}{\mathds 1}
\DeclareMathOperator{\im}{im}

\pdfinfo{
	/Author (Die Mitarbeiter von http://mitschriebwiki.nomeata.de/)
	/Title   (Geometrische Maßtheorie)
	/Subject (Geometrische Maßtheorie)
}

\author{PD. Dr. Daniel Hug}
\publishers{Die Mitarbeiter von \url{http://mitschriebwiki.nomeata.de/}}
\title{Geometrische Maßtheorie}
\date{Wintersemester 2009/2010}
\makeindex

\begin{document}

\maketitle

\tableofcontents

\chapter*{Vorwort}

\section*{Über dieses Skriptum}
Dies ist ein Mitschrieb der Vorlesung \glqq Geometrische Maßtheorie\grqq\
von Herrn PD. Dr. Daniel Hug im Wintersemester 2009/2010 an der Universität Karlsruhe (TH).
 Die Mitschriebe der Vorlesung werden mit ausdrücklicher Genehmigung von Herrn Hug hier veröffentlicht,
Herr Hug ist für den Inhalt nicht verantwortlich.

\section*{Wer}
Beteiligt am Mitschrieb ist Joachim Breitner.

\section*{Wo}
Alle Kapitel inklusive \LaTeX-Quellen können unter \url{http://mitschriebwiki.nomeata.de} abgerufen werden.
Dort ist ein \emph{Wiki} eingerichtet und von Joachim Breitner um die \LaTeX-Funktionen erweitert.
Das heißt, jeder kann Fehler nachbessern und sich an der Entwicklung
beteiligen. Auf Wunsch ist auch ein Zugang über \emph{Subversion} möglich.

\chapter{Grundlage: Maß und Integral}

\section{Äußere Maße und Meßbarkeit}

\begin{definition}
Sei $X$ eine Menge. Eine Abbildung
\[
\mu : \mathcal P(X) \to [0,\infty]
\]
heißt \emph{äußeres Maß} auf $X$, falls gilt:
\begin{enumerate}
\item $\mu(\emptyset) = 0$
\item Für $A,A_n \subset X$, $i\in \MdN$ mit $A\subset \bigcup_{i\ge 1} A_i$ gilt 
\[
\mu(A) \le \sum_{i\ge1} \mu(A_i).
\]
\end{enumerate}
\end{definition}

Beobachte folgende einfache Folgerungen der Definition:

\begin{itemize}
\item $A\subset B \subset X \implies \mu(A)\le \mu(B)$
\item $A\subset B \cup \emptyset \cup \emptyset \cup \ldots \implies \mu(A) \le \mu(B) + \mu(\emptyset) + \mu(\emptyset) + \cdots = \mu(B)$
\end{itemize}

\begin{beispiel}
\begin{align*}
\mu_1(A) &= 
\begin{cases}
\#A, & A \text{ endlich} \\
0, & \text{sonst}
\end{cases}
&
\mu_2(A) &= 
\begin{cases}
1, & A \ne 0\\
0, & \text{sonst}
\end{cases} \\
\mu_3(A) &= 
\begin{cases}
\infty, & A \ne \emptyset \\
0, & \text{sonst}
\end{cases}
&
\mu_4(A) &= 
\begin{cases}
\infty, & A^c \text{ endlich} \\
0, & \text{sonst}
\end{cases} \\
\mu_5(A) &= 
\begin{cases}
0, & \text{$A$ abzählbar oder $A^c$ abzählbar} \\
1, & \text{sonst}
\end{cases}
\end{align*}
\end{beispiel}

Für die Konstruktion eines äußeren Maßes aus Rohdaten ist folgender Satz nützlich:

\begin{satz}
Sei $\mathcal E \subset \mathcal P (X)$ mit $\emptyset \in \mathcal E$, sei $\eta: \mathcal E \to [0,\infty]$ mit $\eta(\emptyset)=0$. Dann wird durch
\[
\mu(A) \da \inf\{\sum_{i=1}^\infty \eta(A_i) \mid A_i\in \mathcal E, i\in \MdN, A\subset\bigcup_{i\ge1}A_i\}
\]
($\inf\emptyset = \infty$) für $A\subset X$ ein äußeres Maß erklärt, das von $(\mathcal E, \eta)$ induzierte äußere Maß.
\end{satz}

\begin{beweis}
Es ist $0\le \mu(\emptyset) \le \sum_{i=1}^\infty \eta(\emptyset)$, da $\emptyset \subset \bigcup_{i=1}^\infty\emptyset$ und $\emptyset\in\mathcal E$.

Seien $A, A_i\subset X$ und $A\subset \bigcup_{i\ge1}A_i$. Wir müssen zeigen: $\mu(A) \le \sum_{i\ge 1}A_i$.

Ist für ein $i\in\MdN$ bereits $\mu(A_i)=\infty$, so sind wir fertig. Sei also $\mu(A_i)<\infty$ für alle $i\in\MdN$. Sei $\varepsilon>0$, dann existiert ein $E_{ij}\in \mathcal E$, $j\in \MdN$ mit $A_i\subset \bigcup_{j\ge 1} E_{ij}$ und
\[
\mu(A_i) + \frac{\varepsilon}{2^i}\ge \sum_{i\ge j}\mu(E_{ij}).
\]
Also gilt
\[
A \subset \bigcup_{i\ge j} Ai \subset \bigcup_{i,j\ge 1} E_{ij}, \quad E_{ij}\in \mathcal E
\]
und daraus folgt
\begin{align*}
\mu(A) &\le \sum_{i,j\ge 1}\eta(E_{ij}) \\
&= \sum_{i=1}^\infty \sum_{j=1}^\infty \mu(E_{ij}) \\
&\le \sum_{i=1}^\infty \mu(A_i) + \frac\varepsilon{2^i} \\
&\le \sum_{i=1}^\infty \mu(A_i) + \varepsilon
\end{align*}
was mit $\varepsilon \to 0$ bedeutet dass
\[
\mu(A) \le \sum_{i=1}^\infty \mu(A_i)
\]
\end{beweis}

\begin{definition}
Sei $\mu$ das äußere Maß auf $X$. Eine Menge $A\subset X$ heißt $\mu$-messbar, falls für alle $M\subset X$ gilt:
\[
\mu(M) = \mu(M\cap A) + \mu(M \cap A^c).
\]
Die Menge aller $\mu$-messbaren Mengen wird mit $\mathcal A_\mu$ bezeichnet.
\end{definition}

Es genügt bereits: $A$ ist $\mu$-messbar genau dann wenn
\[
\mu(M) \ge \mu(M\cap A) + \mu(M\cap A^c) \quad \forall M\subset X
\]

Denn wegen
\[
M\subset (M\cap A) \cup (M\cap A^c) \cup \emptyset \cup \emptyset\cdots
\]
gilt 
\[
\mu(M)\le \mu(M\cap A) + \mu(M\cap A^c) + \mu(\emptyset) + \cdots.
\]

Es gilt stets $\emptyset, X\in\mathcal A_\mu$.

\begin{bemerkung}
Sei $Y\subset X$. $\mu LY$ ist das durch
\[
(\mu LY)(M) \da \mu(M\cap Y)\quad M\subset X
\]
erklärte äußere Maß. Ferner ist $\mathcal A_\mu \subset \mathcal A_{\mu LY}$, denn: für $A\in\mathcal A_\mu$ und $M\subset X$ ist
\begin{align*}
\mu_{LY}(M)
&= \mu(Y\cap M) = \mu(Y\cap M\cap A) + \mu(Y\cap M \cap A^c) \\
&= (\mu LY) (M\cap A) + (\mu LY)(M\cap A^c)
\end{align*}
Es gilt
\[
A \in \mathcal A_\mu \iff \mu= (\mu LA) + (\mu L A^c)
\]
\end{bemerkung}

\begin{proposition}
Für ein äußeres Maß $\mu$ auf $X$ gelten die folgenden Aussagen:
\begin{enumerate}[a)]
\item $\emptyset,\, X\in \mathcal A_\mu$ sowie $A\in \mathcal A_\mu \iff A^c\in\mathcal A_\mu$.
\item Für $A\subset X$ mit $\mu(A)=0$ gilt $A\in\mathcal A_\mu$.
\item Für $A_i\in\mathcal A_\mu$, $i\in\MdN$ gilt $\bigcup_{i\ge1} A_i\in\mathcal M_\mu$ und $\bigcap_{i\ge1} A_i\in\mathcal M_\mu$.
\item Für $A_\in\mathcal A_\mu$, $B\in X$ gilt
\[
\mu(A\cap B) + \mu(A\cup B) = \mu(A) + \mu(B).
\]
\item Für $A_i\in\mathcal A_\mu$, paarweise disjunkt, gilt
\[
\mu(\bigcup_{i=1}^\infty A_i) = \sum_{i=1}^\infty A_i.
\]
\item Für $A_i\in \mathcal A_\mu$, $i\in\MdN$ und $A_i\subset A_{i+1}$ für alle $i\in\MdN$ gilt
\[
\mu(\bigcup_{i=1}^\infty A_i) = \lim_{i\to\infty}\mu(A_i).
\]
\item Für $A_i \in \mathcal A_\mu$, $i\in \MdN$ mit $\mu(A_1) < \infty$ und $A_i\supset A_{i+1}$ für alle $i\in\MdN$ gilt:
\[
\mu(\bigcap_{i=1}^\infty A_i) = \lim_{i\to\infty}\mu(A_i).
\]
\end{enumerate}
\end{proposition}

\begin{beweis}
\begin{enumerate}
\item[c)] $A_1,A_2\in \mathcal A_\mu$.
\begin{align*}
\mu(M) &= \mu(M\cap A_1) + \mu(M\cap A_1^c) \\
&= \mu(M\cap A_1) + \mu(M\cap A_1^c\cap A_2) + \mu(M\cap A_1^c \cap A_2^c) \\
&\ge \mu(M \cap (A_1 \cup (A_1^c \cap A_2))) + \mu)(M \cap A_1^c \cap A_2^c) \\
&= \mu(M\cap (A_1 \cup A_2)) + \mu(M\cap (A_1\cup A_2)^c)
\end{align*}
Daraus folgt, dass $A_1\cup A_2$ $\mu$-messbar ist. Per Induktion sieht man dann, dass für $A_1,\ldots,A_n\in\mathcal A_\mu$ gilt: $\bigcup_{i=1}^n A_i \in \mathcal A_\mu$.
\item[e)] Sind $A_1,\ldots,A_n\in\mathcal A_\mu$ und paarweise disjunkt. Dann gilt
\begin{align*}
\mu(A_1\cup A_2) = \mu( (A_1\cup A_2)\cap A_1) + \mu( (A_1\cup A_2)\cap A_1^c) 
= \mu(A_1) + \mu(A_2)
\end{align*}
woraus folgt dass
\[
\mu(\bigcup_{i=1}^n A_i) = \sum_{i=1}^n \mu(A_i).
\]
Wegen
\[
\sum_{i=1}^n \mu(A_i) \le \mu(\bigcup_{i=1}^n A_i) \quad \forall n\in\MdN
\]
gilt
\[
\sum_{i=1}^\infty \mu(A_i) \le \mu(\bigcup_{i=1}^\infty A_i) \le \sum_{i=1}^\infty \mu(A_i)
\]
und damit Gleichheit.
\item[f)] Wir definieren $B_1 \da A_1$, $B_2 \da A_2\setminus A_1$, $B_3 \da A_3\setminus A_2\ldots$ Es gilt nun dass $B_i\in \mathcal A_\mu$ für alle $i\in\MdN$ und die $B_i$ sind paarweise disjunkt. Nun kann ausrechnen dass
\begin{align*}
\lim_{k\to\infty} \mu(A_k) 
&= \lim_{k\to\infty} \mu(\bigcup_{i=1}^k B_i) \\
&= \lim_{k\to\infty} \sum_{i=1}^k \mu(B_i)\\
&= \sum_{i=1}^\infty \mu(B_i)\\
&= \mu(\bigcup_{i=1}^\infty B_i) \\
&= \mu(\bigcup_{i=1}^\infty A_i)
\end{align*}
\item[g)] Es ist
\begin{align*}
\mu(A_1) = \mu(A_2\cup A_1\setminus A_2) = \mu(A_2) + \mu(A_1\setminus A_2),
\end{align*}
das heißt
\[
\mu(A_1\setminus A_2) = \mu(A_1) - \mu(A_2).
\]
Damit zeigt man
\begin{align*}
\mu(A_1\setminus \bigcup_{i\ge1}A_i) 
&= \mu(A_1 \cap (\bigcap_{i\ge1}A_i)^c) \\
&= \mu(A_1 \cap (\bigcup_{i\ge1}A_i^c) \\
&= \mu(\bigcup_{i\ge 1}(A_1\cap A_i^c)) \\
\text{(nach f)) }&= \lim_{i\to\infty} \mu(\underbrace{\A_1\cap A_i^c}_{= A_1\setminus A_i}) \\
&= \lim_{i\to\infty} (\mu(A_1) -\mu(A_i)) \\
&= \mu(A_1) - \lim_{i\to\infty}\mu(A_i)
\end{align*}
\item[c)] Sei $M\subset X$. Wir definieren $C_k \da \bigcup_{i=1}^k A_i \in \A_\mu$. Damit gilt $C_1\subset C_2\subset \cdots$.

Sei ohne Beschränkung der Allgemeinheit $\mu(M) <\infty$. Dann gilt
\begin{align*}
\infty &> \mu(M) = (\mu LM)(X)\\
&= (\mu LM)(C_k) + (\mu LM)(C_k^c) \\
&= \lim_{k\to\infty} (\mu LM)(C_k) + \lim_{k\to\infty}(\mu LM)(C_k^c) \\
&= (\mu LM)(\bigcup_{i\ge 1}C_i) + (\mu LM)(\bigcap_{i\ge 1}C_i^c) \\
&= (\mu LM)(\bigcup_{i\ge 1}C_i) + (\mu LM)( (\bigcup_{i\ge 1}C_i)^c) \\
&= \mu(M\cap (\bigcup_{i\ge 1}A_i)) + \mu(M\cap (\bigcup_{i\ge 1}A_i)^c) 
\end{align*}
und somit $\bigcup_{i\ge 1}A_i \in \A_\mu$.

\item[d)] Für $A\in\A_\mu$ und $B\subset X$ gilt:
\begin{align*}
\mu(A\cup B) &= \mu( (A\cup B) \cap A) + \mu( (A\cup B)\cap A^c) \\
&= \mu(A) + \mu(B\cap A^c)\\
\text{sowie}\quad 
\mu(B) &= \mu(B\cap A) + \mu(B\cap A^c).
\intertext{Setzt man dies in die folgende Gleichung ein, so erhält man}
\mu(A) + \mu(B) &= \mu(A) + \mu(B\cap A) + \mu(B\cap A^c)\\
&= \mu(B\cap A) + \mu(A\cup B).
\end{align*}
\end{enumerate}
\end{beweis}

Hinweis: Es ist $\A_\mu$ eine (bezüglich $\mu$ vollständige) $\sigma$-Algebra und $\mu$ ist ein $\sigma$-additives Maß auf $\A_\mu$, wobei „$\A_\mu$ ist $\mu$-vollständig“ heißt, dass jede $\mu$-Nullmenge in $\A_\mu$ liegt. $(X,\A_\mu)$ ist ein messbarer Raum und $(X,\A_\mu,\mu)$ ist ein Meßraum.

\begin{definition}
Sei $\A$ eine $\sigma$-Algebra auf $X$. Ein äußeres Maß $\mu$ auf $X$ heißt $\A$-regulär, falls $\A\subset \A_\mu$ gilt und zu jeder Menge $M\subset X$ ein $A\in\A$ existiert mit $M\subset A$ und $\mu(M) = \mu(A)$. Das äußere Maß $\mu$ heißt regulär, falls $\mu$ ein $\A_\mu$-reguläres Maß ist.
\end{definition}

\begin{proposition}
Sei $\mathcal A$ eine $\sigma$-Algebra in $X$, $\mu$ ein $\A$-reguläres äußeres Maß auf $X$. Dann gilt:
\begin{enumerate}[a)]
\item Ist $M_i\subset X$, $M_i\subset M_{i+1}$ für alle $i\in\MdN$, so ist
\[
\mu(\bigcup_{i\ge 1}) = \lim_{i\to\infty}\mu(M_i)
\]
\item Zu jedem $M\in X$ mit $\mu(M)<\infty$ existiert ein $A\in\A$, so dass für alle $B\in \A_\mu$ gilt:
\[
\mu(B\cap M) = \mu(B\cap A)
\]
\item Ist $M_1\cup M_2\in \A$ und $\mu(M_1\cup M_2) = \mu(M_1)+\mu(M_2) <\infty$, so existiereren $A_1,A_2\in\A$ mit $M_i\subset A_i$, $i=1,2$ und $\mu(A_i\setminus M_i) = 0$. Insbesondere ist $M_1,M_2\in \A_\mu$.
\end{enumerate}
\end{proposition}

\begin{beweis}
\begin{enumerate}[a)]
\item Zu jedem $i\in\MdN$ finden wir ein $A_i\in \A$ so dass $M_i\subset A_i$ und $\mu(M_i) =\mu(A_i)$. Dazu definieren wir $B_i \da \bigcap_{j\ge i} A_j$. Damit gilt $M_i \subset B_i\subset A_i$, $B_i\subset B_{i+1}$ und $B_i\in\A$, $i\in\MdN$. Es folgt:
\begin{align*}
\mu(\bigcup_{i\ge 1} M_i) &\le \mu(\bigcup_{i\ge1}B_i) \\
&= \lim_{i\to\infty} \mu(B_i) \\
&\le \lim_{i\to\infty} \mu(A_i) \\
&\le \lim_{i\to\infty} \mu(M_i) \\
&\le \lim_{i\to\infty} \mu(\bigcup_{i\ge 1} M_i)
\end{align*}
\item  Zu $M$ existiert ein $A\in\A$ mit $M\subset A$ und $\mu(M) = \mu(A)$. Sei $B\in \A_\mu$. Dann folgt:
\begin{align*}
\mu(A) = \mu(M) &= \mu(M\cap B) + \mu(M\cap B^c) \\
&\le \mu(A\cap B) + \mu(M \cap B^c) \\
&\le \mu(A\cap B) + \mu(A \cap B^c) = \mu(A) \\
\end{align*}
woraus Gleichheit in obiger Ungleichung folgt. Wegen $\mu(M)<\infty$ ist auch $\mu(M\cap B^c)<\infty$, und wir können dies von zwei obigen Termen abziehen und erhalten
\[
\mu(M\cap B) = \mu(A\cap B).
\]
\item  Zu $M_1$ existiert $\tilde A_1\in\A$ mit $M_1 \subset \tilde A_1$ und $\mu(M_1) = \mu(\tilde A_1)$. Wir definieren $A_1 \da \tilde A_1 \cap (M_1\cup M_2)$. Für diese Menge gilt nun $M_1\subset A_1 \subset M_1\cup M_2$. Wir folgern 
\[
\mu(M_1) \le \mu(A_1) \le \mu (\tilde A_1) \le \mu(M_1)
\]
und
\begin{align*}
\mu(A_1\cap M_2) + \mu(A_1 \cup M_2) &= \mu(A_1) + \mu(M_2) \\
&= \mu(M_1) + \mu(M_2) \\
&= \mu(M_1\cup M_2) \\
&= \mu(A_1 \cup M_2) < \infty
\end{align*}
woraus $\mu(A_1\cap M_2) = 0$ folgt.

Nun liegt $A_1\setminus M_1\subset A_1\cap M_2$, also gilt $\mu(A_1\setminus M_1) = 0$ und somit $A_1\setminus M_1\in \A_\mu$. Damit gilt dann $M_1 = A_1\cap (A_1\setminus M_1)^c\in \A_\mu$.
\end{enumerate}
\end{beweis}

\begin{satz}
Sei $\A$ eine $\sigma$-Algebra in $X$ und $\nu$ ein Maß auf $\A$. Dann wird durch
\[
\mu(M) \da \inf\{\nu(A) \mid A\in\A,\, M\subset A\}
\]
für $M\subset X$ ein $\A$-reguläres äußeres Maß auf $X$ erklärt mit $\mu|_{\A} = \nu$. Ist $M\in\A_\mu$ und $\mu(M)<\infty$, so existiert ein $A\in\A$ mit $\mu(A\setminus M) = 0$ und $M\subset A$.
\end{satz}

\begin{beweis}
Für $M\subset X$ sieht man leicht:
\begin{align*}
\mu(M) &= \inf\{\sum_{i=1}^\infty \nu(A_i) \mid A_i \in \A,\, i\in \MdN,\, M\subset \bigcup_{i=1}^\infty A_i\}
\end{align*}
Also ist $\mu$ das von $(\A,\nu)$ induzierte äußeres Maß. Da $\nu$ monoton ist und nach der Definition von $\mu$ ist $\mu|_\A = \nu$.

Um die $\A$-Regularität zu zeigen nehmen wir ein $A\in\A$ und ein $M\subset X$. Für $B\in\A$ mit $M\subset B$ gilt:
\begin{align*}
\mu(M\cap A) + \mu(M\cap A^c) 
&\le \mu(B\cap A) + \mu(B\cap A^c) \\
&\le \nu(B\cap A) + \nu(B\cap A^c) \\
&= \nu(B)
\end{align*}
und daher
\[
\mu(M\cap A) + \mu(M\cap A^c) \le \mu(M)
\]
also ist $A\in\A_\mu$. Sei nun $M\subset X$ beliebig und ohne Beschränkung der Allgemeinheit $\mu(B)<\infty$. Es existiert also eine Folge $A_i\in\A$, $i\in\MdN$ mit $M\subset A_i$ und $\nu(A_i) \to \mu(M)$. Setze $A\da \bigcup_{i\ge 1} A_i$. Dann gilt $A \in \A$, $M\subset A$, sowie
\begin{align*}
\mu(M) = \lim_{i\to\infty} \nu(A_i) \ge \nu(\bigcap_{i=1}^\infty A_i) = \mu(\bigcap _{i=1}^\infty A_i) = \mu(A) \ge \mu(M)
\end{align*}
woraus $\mu(M) = \nu(M)$ folgt.

Sei nun $M\in\A_\mu$ mit $\mu(M)<\infty$. Es gibt ein $A\in\A$ mit $M\subset A$ und $\mu(M) = \mu(A)<\infty$. Es folgt
\begin{align*}
\infty \ge \mu(A) &= \mu(A\cap M) +\mu(A\cap M^c) \\
&= \mu(M) + \mu(A\cap M^c) \\
&= \mu(A) + \mu(A\setminus M).
\end{align*}
Wegen $\mu(A)=\mu(M)$ gilt also $\mu(A\setminus M) =0$.
\end{beweis}


\begin{anwendung}
Sei $\vartheta$ ein beliebiges äußeres Maß auf $X$. Dann ist $\vartheta|_{\A_\vartheta}$ ein Maß. Durch
\[
\mu(M) \da \inf\{\vartheta(A) \mid A\in\M_\vartheta,\, M\subset A\}
\]
wird also ein $\A_\vartheta$-reguläres äußeres Maß auf $X$ erklärt, das $\vartheta$ fortsetzt (also $\mu|_{\A_\vartheta} = \vartheta|_{\A_\vartheta}$).
\end{anwendung}

\begin{definition}
Seien $X,Y$ Mengen, $\mu$ ein äußeres Maß auf $X$ und $f:X\to Y$. Dann wird durch
\[
(f\mu)(M) \da \mu(f^{-1}(M))
\]
für $M\subset Y$ ein äußeres Maß $f\mu$ auf $Y$ erklärt. Man nennt $f\mu$ das \emph{Bild} von $\mu$ unter $f$ oder auch \emph{„push forward“} von $\mu$ bezüglich $f$ ($f_\#\mu$).
\end{definition}

\begin{bemerkung}
Für $B\subset Y$ gilt
\[
f^{-1}(B) \in \A_\mu \iff \forall M\subset X: B \in \A_{f(\mu LM)}.
\]
Seien hierzu $M\subset X$, $A,B\subset Y$.
\begin{align*}
&\phantom{=\ \ }\mu(M\cap f^{-1}(A) \cap f^{-1}(B)) + \mu(M\cap f^{-1}(A) \cap f^{-1}(B)^c) \\
&= (\mu LM)(f^{-1}(A\cap B)) + (\mu LM)(f^{-1}(A\cap B^c)) \\
&= f(\mu LM)(A\cap B) + f(\mu LM) (A\cap B^c)
\end{align*}
Insbesondere gilt: Ist $f^{-1}(A) \in \A_\mu$, so ist $A\in\A_{f(\mu)}$.
\end{bemerkung}

\begin{sprechweisen}
Sei $\mu$ ein äußeres Maß auf $X$. Eine Menge $N\subset X$ heißt $\mu$-Nullmenge, falls $\mu(N)=0$. Eine Eigenschaft $\mathcal E$ gilt für $\mu$-fast-alle $x\in X$ bzw. $\mu$-fast-überall, falls 
\[
\mu(\{x\in X\mid \mathcal E\text{ gilt für $x$ nicht}\}) = 0.
\]
Mit $\mathbb F_\mu(X,Y)$ wird die Menge aller Abbildungen $f:D\to Y$ bezeichnet mit $D\subset X$ und $\mu(X\setminus D) = 0$.
\end{sprechweisen}

\begin{definition}
Seien $X,Y$ Mengen und $\mu$ ein äußeres Maß auf $X$ und $\mathcal C$ eine $\sigma$-Algebra in $Y$. Dann heißt $f\in\mathbb F_\mu(X,Y)$ $\mu$-messbar bezüglich $\mathcal C$, falls $f^{-1}(\mathcal C)\subset \A_\mu$.
\end{definition}

Beachte dass für $f:D\to Y$ mit $\mu(X\setminus D)=0$ gilt: $D=f^{-1}(Y)\in \A_\mu$.

\begin{lemma}
Seien $X,Y$ Mengen, $\mu$ ein äußeres Maß auf $X$ und $\mathcal E\subset \mathcal P(Y)$.  Für $f\in\mathbb F_\mu(X,Y)$ sind äquivalent:
\begin{enumerate}[a)]
\item $f^{-1}(\mathcal E) \subset \A_\mu$
\item $f$ ist $\mu$-messbar bezüglich $\sigma(\mathcal E)$.
\end{enumerate}
\end{lemma}

\begin{definition}
Ist $(X,\mathcal T)$ ein topologischer Raum, so nennt man die von den offenen Mengen erzeugte $\sigma$-Algebra $\sigma(\mathcal T)$ die Borelsche $\sigma$-Algebra des topologischen Raumes $(X,\mathcal T)$ mit der Notation $\mathfrak B(X)$.

Spezielle Borelsche Algebren sind $\mathfrak B(\MdR)$, $\mathfrak B(\MdR^n)$, $\mathfrak B(\bar\MdR) \da \{B\in \bar\MdR : B\cap \MdR \in \mathfrak B(\MdR)\}$.
\end{definition}

\begin{definition}
Sei $X$ eine Menge, $\mu$ ein äußeres Maß auf $X$ und $f\in\mathbb F_\mu(X,\bar\MdR)$. Man nennt $f$ eine $\mu$-messbare Abbildung, falls $f$ dies bezüglich $\mathfrak B(\bar\MdR)$ ist.
\end{definition}

Im Folgenden schreiben wir für eine Relation $\varrho$ auf $\bar\MdR$, Mengen $D, D'\subset X$ und Abbildungen $f: D\to \bar\MdR$, $g:D'\to\bar\MdR$:
\[
\{ f\varrho g\} \da \{ x\in D\cap D' : f(x)\mathrel{\varrho} g(x) \}
\]

\begin{lemma}
Sei $\mu$ ein äußeres Maß auf $X$ und $f\in \mathbb F_\mu(X,\bar\MdR)$. Genau dann ist $f$ eine $\mu$-messbare Abbildung, wenn eine der folgenden Bedingungen für alle $a\in\MdR$ erfüllt ist:
\begin{align*}
\{f > a\} \in \A_\mu, && \{f\ge a\} \in \A_\mu, && \{f<a\} \in \A_\mu, && \{f\le a\}\in\A_\mu
\end{align*}
\end{lemma}

\begin{lemma}
Sei $\mu$ ein äußeres Maß auf $X$, seien $f,g,f_n\in \mathbb F_\mu(X,\bar\MdR)$, $n\in\MdN$, $\mu$-messbar. Dann gilt
\begin{enumerate}[(a)]
\item $\{f<g\}$, $\{f\le g\}$, $\{f=g\}$, $\{f\ne g\}$ sind $\mu$-messbare Mengen.
\item Die Funktionen
\begin{align*}
&f+ g, && f-g, && f \cdot g \text{ (falls $\mu$-fast-überall definiert)}, \\
&\sup_n f_n, && \inf_n f_n, && \\
&f^+ \da \max\{f,0\}, && f^- \da -\min\{f,0\}, && |f|, \\
&\limsup_n f_n, && \liminf_n f_n
\end{align*}
sind $\mu$-messbar.
\end{enumerate}
\end{lemma}

\begin{satz}
Ist $\mu$ ein äußeres Maß auf $X$, so ist $f\in \mathbb F_\mu(X,\bar\MdR)$ genau dann $\mu$-messbar, wenn für alle $M\subset X$, $a,b\in\MdR$ mit $a<b$ gilt
\[
\mu(M) \ge \mu(M\cap \{f \le a\}) + \mu(M\cap \{f\ge b\}).
\]
\end{satz}

\begin{beweis}
Sei $f$ zunächst $\mu$-messbar. Dann gilt mit $a<b$, $M\subset X$: 
\begin{align*}
\mu(M) &\ge \mu(M\cap\{f\le a\} ) + \mu(M\cap \{f> a\}) \\
&\ge \mu(M\cap\{f\le a\} ) + \mu(M\cap \{f\ge b\}) 
\end{align*}

Jetzt gelte die Bedingung des Satzes für alle $M\subset X$, $a<b$. Zu zeigen ist: $\{f\le r\}\in \A_\mu$ für beliebige $r\in\MdR$. Sei $M\subset X$ beliebig mit $\mu(M) <\infty$. Für $i\in \MdN$ sei
\[
A_i \da M\cap \{r + \frac1{i+1} \le f \le r+\frac 1i\}.
\]
Wir zeigen nun mit vollständiger Induktion, dass 
\[
\mu(\bigcup_{i=0}^n A_{2i+1})  \ge \sum_{0=1}^n \mu(A_{2i+1})
\]
gilt.

Für $n=0$ ist dies klar. Die Ungleichung gelte für ein $n\in\MdN$.
\begin{align*}
\mu(\bigcup_{i=0}^{n+1} A_{2i+1}) 
&\ge \mu(\bigcup_{i=0}^{n+1} A_{2i+1} \cap \{f\ge \underbrace{r+ \frac1{2n+2}}_{b}\}) + 
     \mu(\bigcup_{i=0}^{n+1} A_{2i+1} \cap \{f\le r+ \underbrace{\frac1{2n+3}}_{a}\}) \\
&= \mu(\bigcup_{i=0}^n A_{2i+1}) + \mu(A_{2n+3}) \\
&\ge \sum_{i=0}^n \mu(A_{2i+1}) + \mu(A_{2n+3}) \\
&\ge \sum_{i=0}^{n+1} \mu(A_{2i+1})
\end{align*}

Analog zeigt man 
\[
\mu(\bigcup_{i=1}^n A_{2i})  \ge \sum_{i=1}^n \mu(A_{2i})
\]
und erhält zusammen
\[
\sum_{i=1}^\infty \mu(A_i) \le 2 \mu(M) <\infty.
\]

Sei $\ep>0$. Dann gibt es ein $n\in\MdN$ mit $\sum_{i\ge n}\mu(A_i) <\ep$. Zunächst schätzen wir ab
\begin{align*}
\mu(M\cap \{r < f < r+\frac1n\}) 
&\le \mu(M\cap \{r<f\le r + \frac1n\}) \\
&= \mu(M\cap \bigcup_{i=n}^\infty \{r +\frac 1{i+1} \le f \le r+\frac 1i\}) \\
&= \mu(\bigcup_{i=n}^\infty A_i) \\
&\le \sum_{i\ge n} \mu(A_i) < \ep
\end{align*}
und damit
\begin{align*}
&\phantom{=\ \ }\mu(M\cap \{f\le r\}) + \mu(M\cap \{f>r\})  \\
&\le \mu(M\cap \{f\le r\}) + \mu(M\cap \{r<f<r+\frac1n\}) + \mu(M\cap \{f\ge r+\frac 1n\}) \\
&\le \mu(M) + \ep
\end{align*}
\end{beweis}

\begin{satz}
Seien $\mu$ ein äußeres Maß auf $X$, $f: X \to [0,\infty]$ eine $\mu$-messbare Abbildung und $(r_n)_{n\in\MdN}$ eine Folge in $[0,\infty)$ mit $\lim_{n\to\infty} r_n = 0$ und $\sum_{n=1}^\infty r_n = \infty$. Dann gibt es eine Folge $(A_n)_{n\in\MdN}$ $\mu$-messbarer Mengen mit
\[
f = \sum_{n\ge1} r_n \ind_{A_n}.
\]
\end{satz}

\begin{beweis}
Setze $A_1 \da \{ f \ge r_1 \}$ und allgemein $A_n \da \{f \ge r_n + \sum_{j=1}^{n-1} r_j \ind_{A_j}\}$, $n\ge 1$.

\textbf{Behauptung:} Es ist $f \ge \sum_{i=1}^n r_i \ind_{A_i}$, $n\in\MdN$. Dies gilt für $n=1$, und wenn es für ein $n\in\MdN$ gilt, dann folgt: Ist $x\notin A_{n+1}$, dann ist
\begin{align*}
f(x) \ge \sum_{i=1}^n r_i \ind_{A_i}(x) + \underbrace{r_{n+1}\ind_{A_{n+1}}(x)}_{=0}
\end{align*}
nach Induktionsvoraussetzung. Ist dagegen $x\in A_{n+1}$, so gilt nach der Definition von $A_{n+1}$
\begin{align*}
f(x) \ge \sum_{i=1}^n r_i \ind_{A_i}(x) + r_j = \sum_{i=1}^n r_i \ind_{A_i}(x) + r_j \underbrace{\ind_{A_{n+1}}(x)}_{=1}.
\end{align*}

Folglich ist $f\ge \sum_{i=1}^\infty r_i \ind_{A_i}$.

\textbf{Annahme:} Es gelte $f(x) > \sum_{i=1}^\infty r_i \ind_{A_i}(x)$ für ein $x\in X$.

Also ist $\sum_{i=1}^\infty r_i \ind_{A_i}(x)<\infty$. Da $\sum_{i=1}^\infty r_i = \infty$ gilt, muss es eine Folge natürlicher Zahlen $(j_k)_{k\in\MdN}$ geben mit $\ind_{A_{j_k}}(x)=0$ für alle $k\in\MdN$.
Wegen $\lim_{k\to\infty} r_{j_k} = 0$ gibt es ein $k\in\MdN$ mit
\begin{align*}
r_{j_k} < f(x) - \sum_{j=1}^\infty r_j \ind_{A_j}(x)
\end{align*}
und damit 
\begin{align*}
f(x) &> \sum_{j=1}^\infty r_j \ind_{A_j}(x) + r_{j_k} \\
&\ge \sum_{j=1}^{j_k-1} r_j \ind{A_j}(x) + r_{j_k}.
\end{align*}
Das bedeutet $x\in A_{j_k}$ im Widerspruch zu $\ind_{A_{j_k}}(x) \ne 0$.
\end{beweis}

\section{Integration}

In diesem Abschnitt wird generell vorausgesetzt, dass $X$ eine Menge und $\mu$ ein äußeres Maß auf $X$ sei.

\begin{definition}
Eine $\mu$-Treppenfunktion auf $X$ ist eine $\mu$-messbare Abbildung $h\in \mathbb F_\mu(X,\MdR)$ mit abzählbarer Wertemenge $\im(h)$ und
\[
\sum_{\substack{r\in \im(h)\\r< 0}} r \cdot \mu(\{h=r\}) > -\infty \quad\text{ oder }\quad
\sum_{\substack{r\in \im(h)\\r> 0}} r \cdot \mu(\{h=r\}) < \infty.
\]
Ist $h$ eine $\mu$-Treppenfunktion auf $X$, so wird durch
\[
\int hd\mu = \sum_{r\in \im(h)} r \cdot \mu(\{h=r\})
\]
das $\mu$-Integral von $h$ erklärt, wobei „$0\cdot \infty\da 0$“ gelte.
\end{definition}

\begin{bemerkung}
\begin{enumerate}
\item  Es gilt
\[
\int h d\mu = \int h^+ d\mu - \int h^- d\mu.
\]
\item $h=\ind_A$, $A\in\A_\mu$ ist eine $\mu$-Treppenfunktion, $\int \ind_A d\mu=\mu(A)$.
\end{enumerate}
\end{bemerkung}

\begin{lemma}
\label{lem1.10}
Seien $h,g$ $\mu$-Treppenfunktionen auf $X$. Es gelte $\int h^+ d\mu <\infty$ und $\int g^+ d\mu<\infty$  oder $\int h^- d\mu <\infty$ und $\int g^- d\mu<\infty$. Dann ist $h+g$ eine $\mu$-Treppenfunktion und es gilt
\[
\int (h+g) d\mu = \int h d\mu + \int g d\mu.
\]
\end{lemma}

\begin{beweis}
Es gilt zunächst $h+g \in \mathbb F_\mu(X,\MdR)$. Zur Additivität: Wir definieren $A(r,s) \da \{h=r\} \cap \{g=s\}$ für $r,s\in\MdR$.
\begin{align*}
\int hd\mu + \int gd\mu 
&= \sum_{r\in\im(h)} r\cdot \mu(\{h=r\}) + \sum_{s\in \im(h)} s \cdot \mu(\{g=s\}) \\
&= \sum_{r\in\im(h)} r \cdot \sum_{s\in \im(g)} \mu(A(r,s)) + \sum_{s\in\im(g)} s \cdot \sum_{r\in \im(h)} \mu(A(r,s)) \\
&= \sum_{\substack{r\in\im(h)\\s\in\im(g)}} (r+s)\cdot \mu(A(r,s)) \\
&= \sum_{t\in \im(g+h)} t \cdot \sum_{\substack{r\in \im(h) \\s\in\im(g)\\r+s=t}} \mu(A(r,s)) \\
&= \sum_{t\in \im(g+h)} t \cdot \mu\big(\bigcup_{\substack{r\in \im(h) \\s\in\im(g)\\r+s=t}} A(r,s)\big) \\
&= \sum_{t\in \im(g+h)} t \cdot \mu(\{g+h=t\}) \\
&= \int (h+g) d\mu
\end{align*}
Übung: Zeige, dass $g+h$ integrierbar ist.
\end{beweis}

\begin{bemerkung}
Sei $h$ eine $\mu$-Treppenfunktion mit $h\ge 0$. Dann gilt $\int hd\mu \ge 0$. Mit Lemma \ref{lem1.10} folgt für $\mu$-Treppenfunktionen $h,g$:
\[
h \le g \implies \int h d\mu \le \int g d\mu
\]
\end{bemerkung}

\begin{definition}
Sei $f\in\mathbb F_\mu(X,\bar\MdR)$. Eine $\mu$-Oberfunktion (bzw. $\mu$-Unterfunktion) von $f$ ist eine $\mu$-Treppenfunktion $h$ auf $X$ mit $f\le h$ $\mu$-fast-überall auf $X$ (bzw. $h\le f$ $\mu$-fast-überall auf $X$).

Durch
\[
\int^* fd\mu \da \inf\left\{\int hd\mu \mid \text{$h$ ist eine $\mu$-Oberfunktion von $f$}\right\}
\]
wird das $\mu$-Oberintegral von $f$ erklärt.
Analog wird durch
\[
\int_* fd\mu \da \sup\left\{\int hd\mu \mid \text{$h$ ist eine $\mu$-Unterfunktion von $f$}\right\}
\]
wird das $\mu$-Unterintegral von $f$ erklärt.
\end{definition}

\begin{lemma}
Für $f,g\in\mathbb F_\mu(X,\bar\MdR)$ gelten die folgenden Aussagen:
\begin{enumerate}
\item $\int_* fd\mu = - \int^* (-f)d\mu$.
\item Gilt $\mu$-fast-überall $f\le g$, so ist $\int^* fd\mu \le \int^* gd\mu$ und $\int_* fd\mu \le \int_* gd\mu$.
\item Gilt $\mu$-fast-überall $f\ge 0$, so ist $\int^* fd\mu \ge 0$ und $\int_* fd\mu \ge 0$.
\item Gilt $\int^* fd\mu <\infty$, so auch $\int^*f^+d\mu<\infty$ und $f<\infty$ $\mu$-fast-überall.
\item Für $c\in(0,\infty)$ gilt $\int^*(cf)d\mu = c\cdot \int^*fd\mu$ und $\int_*(cf)d\mu = c\cdot \int_*fd\mu$.
\item Ist $\int^* fd\mu <\infty$ und $\int^*g d\mu<\infty$, so ist $f+g \in \mathbb F_\mu(X,\bar\MdR)$ und $\int^*(f+g)d\mu \le \int^*fd\mu + \int^*gd\mu$.

Analog gilt: Ist $\int_* fd\mu >-\infty$ und $\int_*g d\mu>-\infty$, so ist $f+g \in \mathbb F_\mu(X,\bar\MdR)$ und $\int_*(f+g)d\mu \ge \int_*fd\mu + \int_*gd\mu$.
\item $\int_* fd\mu \le \int^* fd\mu$.
\end{enumerate}
\end{lemma}

\begin{bemerkung}
Ist $h$ eine $\mu$-Treppenfunktion, so ist $h$ eine $\mu$-Oberfunktion und eine $\mu$-Unterfunktion seinerselbst. Das heißt insbesondere
\[
\int hd\mu \le \int_* h d\mu \le \int^* h d\mu \le \int h d\mu.
\]
\end{bemerkung}

\begin{definition}
Ist $f\in \mathbb F_\mu(X,\bar\MdR)$ eine $\mu$-messbare Abbildung und stimmen das $\mu$-Oberintegral mit dem $\mu$-Unterintegral von $f$ überein, so wird durch
\[
\int fd\mu \da \int^* fd\mu = \int_* fd\mu
\]
das $\mu$-Integral von $f$ erkärt. Man sagt in diesem Fall, dass das $\mu$-Integral existiert. Ist $\int fd\mu \in \MdR$, so heißt $f$ $\mu$-integrierbar.
\end{definition}

\begin{satz}
Sei $f\in \mathbb F_\mu(X,\bar\MdR)$ nicht-negativ und $\mu$-messbar. Dann existiert das $\mu$-Integral von $f$. Es gilt $\int fd\mu \ge 0$ und 
\[
\int fd\mu = \sup\left\{ \int hd\mu \mid \text{$h$ ist $\mu$-Unterfunktion, $\im(h)$ ist endlich}\right\}.
\]
\end{satz}

\begin{beweis}
Ist $\mu(\{f=\infty\})>0$, dann ist für jedes $n\in\MdN$ die Funktion $n\cdot \ind_{\{f=\infty\}}$ eine $\mu$-Unterfunktion von $f$ und 
\[
\int^*fd\mu \ge \int_* fd\mu \ge \int n \cdot\ind_{\{f=\infty\}} d\mu = n \cdot \mu(\{f=\infty\})\to \infty\text{ für }n\to\infty.
\]
Also ist $\int^* fd\mu = \int_* fd\mu = \infty$.

Sei jetzt $f<\infty$ $\mu$-fast-überall. Für $t\in(1,\infty)$ sei
\[
U_t \da \sum_{n\in\MdZ} t^n \cdot \ind_{\{t^n \le f < t^{n+1}\}} \le f \le t\cdot U_t
\]
$U_t$ ist eine $\mu$-Unterfunktion von $f$, $tU_t$ eine $\mu$-Oberfunktion von $f$. Damit gilt
\begin{align*}
\int^* fd\mu \le \int t U_t d\mu = t \cdot \int U_t d\mu \le t \int_* f d \mu.
\end{align*}
Ist $\int_* fd\mu <\infty$, dann folgt $\int^* fd\mu \le \int_* fd\mu$ aus $t\to 1$. Ist $\int_* fd\mu = \infty$, so ist $\int^* fd\mu = \int_* fd\mu = \infty$.
\end{beweis}

\end{document}
