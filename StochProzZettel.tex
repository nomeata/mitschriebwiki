\documentclass[a4paper,DIV15]{scrartcl}

\usepackage{mathe}
\usepackage{enumerate}
\usepackage{tikz}
\usepackage{longtable}


\newcommand{\cF}{\mathcal F}
\newcommand{\borel}{{\mathfrak B}}

\author{Joachim Breitner}
\title{Stochastische Prozesse\\Stoffzusammenfassung}
\begin{document}
\maketitle

Diese Zusammefassung ist natürlich alles andere als vollständig und zu knapp, um immer alle Aussagen mit Voraussetzungen korrekt wiederzugeben. Man verwende sie mit Vorsicht.

\section{Vokabeln, Definitionen und äquivalente Charakterisierungen}
\subsection{Markov-Ketten in diskreter Zeit}

\begin{longtable}[h]{lp{0.8\linewidth}}
$(X_n)_{x\in\MdN_0}$, $X_n:\Omega\to S$
	& Markov-Kette mit Zustandsraum $S$\\
	& $P(X_{n+1}=i_{n+i} \mid X_0=i_0,\ldots,X_n=i_n) = p_{i_ni_{n+1}}$ \\
$P=(p_{ij})_{i,j\in S}$
	& Übergangsmatrix mit Übergangswahrscheinlichkeiten \\
$P^{(n)} = (p_{ij}^{(n)})_{i,j\in S}$
	& $n$-Schritt-Übergangsmatrix mit $n$-Schritt-Übergangswahrscheinlichkeiten\\
$i\rightsquigarrow j$
	& $\exists n\in\MdN p_{ij}^{(n)} > 0$, „$i$ führt nach $j$“\\
$i \leftrightarrow j$
	& $i\rightsquigarrow j \wedge j\rightsquigarrow i$, „$i$ kommuniziert mit $j$“ \\
$J\subseteq S$ abgeschlossen
	& $\not\exists j\in J, i\in S\setminus J: i\rightsquigarrow j$ \\
	& $(p_{ij},\, i,j\in S)$ ist stochastische Matrix \\
$(X_n)$ irreduzibel
	& $(X_n)$ hat nur eine Äquivalenzklasse bzgl. „$\leftrightarrow \vee =$“\\
$T_i$
	& $\inf\{n\in\MdN\mid X_n=i\}$, „Ersteintrittszeit“\\
$f_{ij}^{(n)}$
	& $P(T_j=n\mid X_0=i)$, insbesondere $f_{ij}^{(1)} = p_{ij}$\\
$f_{ij}^{*}$
	& $\sum_{n=0}^\infty f_{ij}^{(n)} = P_i(T_j < \infty)$\\
$i$ rekurrent
	& $f_{ii}^*=1$ \\
	& $\sum_{n=0}^\infty p_{ij}^{(n)} = \infty = E_i(\sum_{n=0}^\infty 1_{X_n=i})$, die erwartete Zahl der Besuche.\\
$i$ transient
	& $i$ nicht rekurrent \\
$\nu:S\to\MdR_{\ge0}$
	& Maß \\
	& Verteilung, wenn gilt: $\sum_{a\in S} \nu(a)=1$\\
$\nu$ invariant
	& $\sum_{i\in S} \nu(i)p_{ij} = \nu(j)$, also $\nu = \nu P$\\
$\gamma_k:S\to\MdR_{\ge0}$
	& $E_k(\sum_{n=1}^{T_k} 1_{(X_n=i)})$, Besucher pro Zyklus \\
	& invariant, $0<\gamma_k<\infty$, eindeutig mit $\gamma_k(k)=1$, wenn $(X_n)$ irreduzibel, rekurrent. \\
	& $(X_n)$ irreduzibel, transient: stationäre Verteilung existiert nicht. \\
$m_i$
	& $E_i(T_i) = \sum_{n=1}^\infty n\cdot f_{ii}^{(n)} + \infty \cdot(1-f_{ii}^*)$ \\
	& $i$ transient $\implies$ $m_i=\infty$.\\
$i$ positiv rekurrent 
	& $m_i<\infty$ \\
	& $(X_n)$ irreduzibel: Stationäre Verteilung existiert $\iff$ ein/alle Zustände positive rekurrent. Dann: $\pi(i)=\frac1{m_i}$ \\
$(Ph)(i)$
	& $\sum_{j\in S} p_{ij}h(j)$, vergleiche Matrix-Vektor-Multiplikation. \\
	& $Ph \ge h \implies h(X_n)$ Sub-Martingal\\
	& $Ph = h \implies h(X_n)$ Martingal\\
	& $Ph \le h \implies h(X_n)$ Super-Martingal\\
\end{longtable}

\subsection{Markov-Ketten in stetiger Zeit}

\subsubsection{Poisson-Prozess}
\begin{enumerate}[({A}1)]
\item $t\mapsto N(t,\omega)\in \{f:[0,\infty) \to \MdN_0\mid f(0)=0,\text{ $f$ monoton wachsend, $f$ stetig von rechts}\}$
\item Unabhängige Zuwächse
\item Identisch verteilte Zuwächse
\item Ereignisse einzeln: $P(N_h\ge 2) = o(h)$ für $h\to 0$
\end{enumerate}

Dann gilt:
\begin{itemize}
\item $\forall s,t\ge 0:N_{s+t}-N_s \sim \mathcal Po(\lambda t)$
\item Zeit zwischen Sprüngen $\operatorname{Exp}(\lambda)$-verteilt.
\end{itemize}

Intensitätsmatrix:
\[
\begin{pmatrix}
-\lambda & \lambda & 0 & \cdots \\
0 & -\lambda & \lambda & 0 & \\
\vdots & 0 & -\lambda & \lambda & \\
& & & \ddots & \ddots 
\end{pmatrix}
\]

\subsubsection{Der allgemeine Fall}

\begin{longtable}[h]{lp{0.8\linewidth}}
Markov-Eigenschaft
	& $P(X_{t_n+h} = i_{n+1} \mid X_{t_k}=i_k,\, k=1,\ldots,n) = P(X_{t_n+h} = i_{n+1} \mid X_{t_n}=i_n)  = P(X_{t+h} = i_{n+1} \mid X_{t}=i_n)$\\
$P(t) = (p_{ij}(t))$
	& $p_{ij}(t) \da P(X_t = j \mid X_0=i)$, Übergangsmatrizenfunktion \\
$P_{ij}$ SÜMF
	& $\lim_{t\to 0}p_{ij}(t) = \delta_{ij}$, „Standardübergangsmatrizenfunktion“\\
$Q=(q_{ij})$
	& Instensitätsmatrix \\
	& $q_{ij}\da \lim_{t\to0}\frac{p_{ij}(t)-\delta_{ij}}t = p_{ij}'(0)$\\
	& Anschaulich: Kehrwert der Diagonalelemente sagt, wie lange die Kette in dem Zustand bleibt, die anderen Elemente geben die Wahrscheinlichkeit des nächsten Zustands an. \\
$q_i$
	& $-q_{ii}$\\
$P$ konservativ
	& $\sum_{i\in S} q_{ij} = 0$ \\
\end{longtable}

\subsection{Brownsche Bewegung}
\begin{longtable}[h]{lp{0.8\linewidth}}
Gauss-Prozess
	& alle Fidis normalverteilt \\
Brownsche Bew.
	& $P(B_0=0)=1$, $P$-f.a. Pfade stetig, $B_t-B_s$ unabhängig von $\cF_s$, $\mathcal N (0,t-s)$-verteilt.\\
	& $\iff$ $EB_t=0$, $\operatorname{Cov}(B_s,B_t)=s\wedge t$, Gauss-Prozess mit $F$-f.s. stetigen Pfaden. \\
$P:\MdR \times \borel \to [0,1]$
	& Stochasticher Kern \\
	& $A\mapsto P(x,A)$ Wahrscheinlichkeitsmaß und $x\mapsto P(x,A)$ messbar\\
$\mathcal G$ Generator
	& $\lim_{t\to0}\frac 1t (P_t-P_0)$ \\
	& Bei BB: $\mathcal Gf= \frac 12 f''$ \\
Markov-Eigenschaft
	& $P(X_t\in A\mid \cF_s) = P_{t-s}(X_s,A)$ \\
$\cF_\tau$
	& $\{A\in\cF\mid \forall t\ge 0: A \cap \{\tau\le t\} \in \cF_t\}$\\
Progressiv messbar
	& $\forall t\ge 0: (s,\omega) \mapsto X_s(\omega)$ ist $\borel([0,t])\otimes \cF_t$-messbar.\\
\end{longtable}

\section{Formeln}

Methode des ersten Besuchs:
\[
p_{ij}^{(n)} = \sum_{k=1}^n f_{ij}^{(n)} p_{jj}^{(n-k)}
\]

Übergangswahrscheinlichsgrenzwert bei Periode $d_j$:
\[
\lim_{n\to\infty} p_{ij}^{n\cdot d_j + r} = \frac{d_j}{m_j} \sum_{k=0}^\infty f_{ij}^{k\cdot d_j + r}
\]
insbesondere ist $p_{ij}^{(n)}\to 0$ für $m_j=\infty$, d.h. $i$ transient oder null-rekurrent.

Chapman-Kolmogorov-Gleichungen
\begin{itemize}
\item diskret
\[
p_{ij}^{(n+m)} = \sum_{k\in S} p_{ik}^{(n)} p_{kj}^{(m)}
\]
\item stetig
\[
p_{ij}(t+s) = \sum_{k\in S} p_{ik}(t) p_{kj}(s)
\]
\item allgemein (eigentlich die Definition von stochastischem Kern)
\[
P_{t+s}(x,A) = \int P_s(y,A)P_t(x,dy)
\]
\end{itemize}

Kolmogorovsche Rückwärts-DGL:
\[
P'(t) = QP(t) \text{ d.h. } p_{ij}'(t) = -q_ip_{ij}(t) + \sum_{k\ne i}q_{ik}p_{kj}(t)
\]
Kolmogorovsch Vorwärts-DGL: \emph{Wann genau gilt die?}
\[
P'(t) = P(t)Q
\]
Ist $S$ endlich, kann man $P(t) = e^{tQ}$ schreiben.

Erfüllt $Q$ die Bedingungen
\begin{align*}
\sum_{j\in S} q_{ij} = 0\text{ und } 0<\sup_{i\in S} |q_ii| \ad \lambda <\infty \tag{($*$)}
\end{align*}
so gilt:
\begin{itemize}
\item Ist $N$ Poissonprozess und $Y_n$ Markov-Kette mit $P=E+\frac 1 \lambda Q$, dann ist $X_t=Y_{N_t}$ Markov-Kette mit Intensitätsmatrix $Q$.
\item $\mu$ ist invariantes Maß $\iff \mu Q = 0$
\item Ist $X_n$ rekurrent, irreduzibel, dann gilt $\lim_{t\to\infty} p_{ij}(t) = \frac1{m_jq_j}$
\end{itemize}

Ist $(B_t)$ eine Brownsche Bewegung, dann auch:
\begin{itemize}
\item $(-B_t)$, $(B_{a+t}-B_a)$, $(cB_{\frac{t}{c^2}})$
\item Zeitumkehr: $(tB_{\frac1t})$
\item Spiegelungsprinzip: Der nach $\tau$ gespiegelte Prozess.
\end{itemize}
Eigenschaften der Brownschen Bewegung:
\begin{itemize}
\item $\sup B_t=\infty$, $\inf B_t=-\infty$, also unendlich oft weit hoch und runter.
\item $P$-fast-sicher nie Lipschitzs-Stetig
\item Totalvariation $\infty$, quadratische Variation $\tonach{$P$}t$.
\item Stochastischer Kern $P_t(x,\cdot) = \mathcal N(x,t)$
\item $\mathcal G f = \frac 12 f''$
\item Ist $\tau$ endliche Stoppzeit, dann ist $(B_{\tau +t} - B_\tau)$ verteilt wie $(B_t)$ und unabhängig von $\cF_\tau$.
\item Identisch verteilt sind: $M_t\da \sup_{0\le s\le t}B_s$, $M_t-B_t$, $|B_t|$
\item $P$-fast-sicher Nullstellenmenge perfekt
\end{itemize}

Invarianzprinzip von Dansker:
Ist $E\xi_i=0$, $0<\operatorname{Var}(\xi_i)\ad \sigma^2<\infty$, $S_k=\sum_{j=1}^k \xi_i$, $Y_t = S_{\lfloor t\rfloor} + (t-\lfloor t\rfloor t)\xi_{\lfloor t\rfloor +1}$, $(X_t^{(n)}) \da \frac1{\sigma\sqrt n} Y_{nt}$, dann konvergieren die Wahrscheinlichkeitsmaße $P_n$ schwach gegen $P$, wobei $P$ so ist, dass die Projektionen $\pi_t$ eine Brownsche Bewegung sind.

\section{Wichtige Beweisideen}

\subsection{Konvergenz gegen stationäre Verteilung}

Voraussetzungen: $(X_n)$ irreduzibel, aperiodisch, positiv rekurrent. „Kopplungs-Argument“: $(Y_n)$ Kette mit gleicher Übergangsmatrix, $Y_n\sim\pi$, $T\da \inf\{n\in\MdN\mid X_n=Y_n\}$.
\begin{itemize}
\item Zeige $P(T<\infty)=1$
\item Definiere 
\[
Z_n \da 
\begin{cases}
X_n,n\le T
Y_n,n> T
\end{cases}
\]
\item Schätze ab
\[
|p_{ij}^{(n)}-\pi(j)| \le 2\cdot P_{\hat \nu}(T>n) \to P(T=\infty) = 0
\]
\end{itemize}

\subsection{$\mu Q = 0 \iff$ $\mu$ stationäre Verteilung}
\begin{itemize}
\item $P'(t) = P(t)Q = QP(t)$
\item $\implies P(t) = E + Q\int_0^t P(s)ds = E+\int_0^t P(s)ds Q$
\item $\implies \mu = \mu P(t) = \mu  + \int_0^t \mu P(s) ds Q = \mu  + t \cdot (\mu Q) = \mu$
\end{itemize}

\subsection{Solidaritätsprinzip}
$i$ rekurrent, $j\in K(i)$, also $\exists m,n\in\MdN: p_{ij}^(m)p_{ji}^{(n)} > 0$. Dann
\begin{align*}
\sum_{k=0}^\infty p_{jj}^{(k)} \ge \sum_{k=0}^\infty p_{jj}^{(n+m+k)}
\ge \sum_{k=0}^\infty p_{ji}^{(n)} p_{ii}^{(k)} p_{ij}^{(m)}
= p_{ij}^{(m)}p_{ji}^{(n)} \sum_{k=0}^\infty p_{ii}^{(k)} = \infty
\end{align*}

\section{Verteilungen}

\begin{itemize}
\item $\mathcal N(0,\sigma^2)$: $f(x) = \frac1{\sigma \sqrt{2\pi}} \exp(-\frac 12 (\frac x \sigma)^2)$
\item $\operatorname{Exp}(\lambda)$: $f(x) =\lambda e^{-\lambda x}$, $F(x)=1-e^{-\lambda x}$, $EX=\frac 1 \lambda$, $\operatorname{Var}X=\frac1{\lambda^2}$.
\item $\mathcal Po(\lambda)$: $P(X=k)=\frac{\lambda^k}{k!} e^{-\lambda}$, $EX=\operatorname{Var}X=\lambda$.
\end{itemize}

\end{document}
