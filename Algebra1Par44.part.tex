\section{Norm, Spur und Charaktere}

\begin{DefProp}
\label{4.10}
Sei $G$ eine Gruppe, $K$
ein Körper.
\begin{enum}
\item  Ein \emp{Charakter} von $G$ (mit Werten in $K$) ist ein
Gruppenhomomorphismus $\chi: G\ra K^x$

\item $X_K(G) \defeqr \{ \chi: G \ra K^x,\; \chi$ Charakter$\} =$
Hom$(G,K^x)$ heißt \emp{Charaktergruppe} von $G$ (mit Werten in
$K$)

\item (Lineare Unabhängigkeit der Charaktere, E.Artin)
$X_K(G)$ ist linear unabhängige Teilmenge des $K$-Vektorraums
Abb$(G,K)$


\sbew{0.9}{Angenommen $X_K(G)$ ist linear abhängig. Dann sei
$n > 0$ minimal, so daß es in $X_K(G)$ $n$ paarweise verschieden
linear abhängige Elemente gibt. Es gebe also $\chi_1,\dots,\chi_n
\in X_K(G)$, $\lambda_1,\dots,\lambda_n \in K^x$ mit $\displaystyle \sum_{i=1}^n
\lambda_i \chi_i = 0$. Dazu muß $n\geq 2$ sein.

Sei $g \in G$ mit $\chi_1(g) \neq \chi_2(g)$. Dann gilt für alle
$\ds h \in G$: \[0 = \sum_{i=0}^n \lambda_i \chi_i(gh) =
\sum_{i=1}^n \underset{\defeql \mu_i \in K^x}{\underbrace{\lambda_i \chi_i
(g)}} \chi_i(h) = \sum_{i=1}^n \mu_i \chi_i(h) \Ra \sum_{i=1}^n \mu_i
\chi_i = 0\]

Sei $\nu_i \defeqr \mu_i - \lambda_i \chi_1(g),\;i=1,\dots,n$. Dann ist
$\displaystyle \sum_{i=1}^n \nu_i\chi_i = 0,\; \nu_1 = \lambda_1 \chi_1(g)
-\lambda_1\chi_1(g)=0$, $\nu_2 = \lambda_2 \chi_2(g) - \lambda_2
\chi_1(g) = \lambda_2(\chi_2(g) - \chi_1(g)) \neq 0$ Widerspruch zur
Minimalität von $n$.}
\end{enum}
\end{DefProp}

\begin{DefBem}
Sei $L/K$ endliche
Körpererweiterung, $q\defeqr \frac{[L:K]}{[L:K]_s}$ ($=p^r,\;p=$char$(K)$), $n
\defeqr [L:K]_s$, Hom$_K(L,\bar K) = \{\sigma_1,\dots,\sigma_n\}$

\begin{enum}
\item Für $\alpha \in L$ heißt tr$_{L/K}(\alpha) \defeqr q \cd
\displaystyle \sum_{i=1}^n \sigma_i (\alpha) \in \bar K$ die \emp{Spur} von
$\alpha$ (über $K$)

\item $\forall \alpha \in L:$ tr$_{L/K}(\alpha) \in K$
\newline\sbew{0.9}{ \OE $L/K$ separabel. Ist $L/K$ normal, also
galoissch, so ist Hom$_K(L,\bar K) =$ Gal$(L/K) \defeql G$ und
tr$_{L/K}(\alpha) \in L^G = K$ (da invariant unter allen $\sigma_i$). Andernfalls sei
$\wt{L}$ normale Erweiterung von
$K$ mit $L \subset \wt{L}$. Für $\tau \in$ Hom$_K(\wt{L},\bar K) =$
Gal$(\wt{L}/K)$ und jedes $i=1,\dots,n$ ist $\tau \circ \sigma_i
\in$ Hom$_K(L,\bar K)$ (da $\sigma_i(L) \subseteq \wt{L}$) $\Ra$
tr$_{L/K}(\alpha) \in {\wt{L}}^{\scriptsize\mbox{Gal}(\wt{L}/K)} = K$ }

\item tr$_{L/K}$ ist $K$-linear.

\item Für $\alpha \in L$ heißt $\ds N_{L/K}(\alpha) =
\left(\prod_{i=1}^n \sigma_i(\alpha)\right)^q$ die \emp{Norm} von
$\alpha$ (über $K$).

\item $N_{L/K}(\alpha) \in K$

\item $N_{L/K}: L^x \ra K^x$ ist Gruppenhomomorphismus
\newline\sbew{0.9}{\begin{enum}
\item[(e)] Ist $L/K$ separabel, so
argumentiere wie in (b). Sonst siehe Bosch.
\end{enum}
}
\end{enum}
\end{DefBem}

\begin{Bem}
Sei $L/K$ endliche Körpererweiterung. Für
$\alpha \in L$ sei $m_\alpha: L \ra L,\;x\mapsto \alpha x$.
$m_\alpha$ ist $K$-linear und es gilt:
\[\mbox{tr}_{L/K}(\alpha) = \mbox{Spur}(m_\alpha),\;N_{L/K}(\alpha)
= \mbox{det}(m_\alpha)\]

\sbew{1.0}{Ist $L/K$ separabel, so sei $L=K(\alpha)$. Dann ist
$1,\alpha,\alpha^2,\dots,\alpha^{n-1}$ eine $K$-Basis von $L$,
$[L:K] = n$. Weiter sei $f(X) = X^n + c_{n-1} X^{n-1} + \dots + c_1
X + c_0\;\in K[X]$ das Minimalpolynom von $\alpha$ über $K$. Dann
ist die Abbildungsmatrix von $m_\alpha$ bezüglich der Basis
$1,\dots,\alpha^{n-1}$

\[ D = \begin{pmatrix}
0 & 0 &\dots & 0 & -c0 \\
1 & 0 &      & \vdots & -c1 \\
0 & 1 &      & \vdots & \vdots   \\
\vdots & \vdots & \ddots & 0 & \vdots \\
0 & 0 & \dots 0 & 1 & -c_{n-1}
\end{pmatrix}\]

$\Ra$ Spur$(m_\alpha) = -c_{n-1}$, det$(m_\alpha) = (-1)^n c_0$ }

\sbew{1.0}{In $\bar K[X]$ zerfällt $f$ in Linearfaktoren:

$f = \prod_{i=1}^n(X-\sigma_i(\alpha)) \Ra c_{n-1} = \sum_{i=1}^n
\sigma_i(\alpha)$, $c_0 = (-1)^n \prod_{i=1}^n \sigma_i(\alpha)$

Ist $L\neq K(\alpha)$, so sei $b_1,\dots,b_n$ eine $K(\alpha)$-Basis
von $L$. Dann ist $B = \{ b_i
\alpha^j,\;i=1,\dots,m,\;j=0,\dots,n-1\}$ eine $K$-Basis von $L$.
Dann ist die Darstellungsmatrix von $m_\alpha$ bezüglich $B$:

\[ \wt{D} = \begin{pmatrix}
D & 0 & \dots & 0\\
0 & D & & \\
  &   & \ddots & \\
0 & 0 & & D \end{pmatrix} \] $\Ra$ Spur$(m_\alpha) = m(-c_{n-1})$,
det$(m_\alpha) = \left((-1)^n c_0 \right)^m$

Für jedes $\sigma_i \in$ Hom$_K(L,\bar K)$ ist $\sigma_i(\alpha)$
Nullstelle von $f$. Jede Nullstelle von $f$ wird dabei gleichoft
angenommen, nämlich $m = [L : K(\alpha)]$-mal $\Ra$
tr$_{L/K}(\alpha) = m \cd$ tr$_{K(\alpha)/K}(\alpha) = m(-c_{n-1})$
und $N_{L/K}(\alpha) = \left(N_{K(\alpha)/K}\right)^m = \left((-1)^n c_0 \right)^m$}
\end{Bem}

\begin{Satz}[''Hilbert(s Satz) 90'']
\label{Satz 19}
Sei $L/K$ zyklische
Galois-Erweiterung. (dh. Gal$(L/K) = \langle \sigma \rangle$ für ein
$\sigma$)
\begin{enum}

\item Ist $\beta \in L$ mit $N_{L/K}(\beta) = 1$, so gibt es ein
$\alpha \in L^x$ mit $\beta = \frac{\alpha}{\sigma(\alpha)}$

\sbew{0.9}{ $n \defeqr [L:K]$. Nach \ref{4.10} sind die Charaktere
$id, \sigma,\dots,\sigma^{n-1}:\; L^x \ra L^x$ linear unabhängig
über $L$.

Nun ist $f = id + \beta \sigma + \beta\sigma(\beta) \sigma^2 + \dots
+ \beta \sigma(\beta) \dots \sigma^{n-2}(\beta) \sigma^{n-1}$
nicht die Nullabbildung $\Ra \exists \gamma \in L$ mit $\alpha
\defeqr f(\gamma) \neq 0$

$\beta \sigma(\alpha) = \beta \sigma(\gamma) + \beta \sigma(\beta)
\sigma^2 (\gamma) + \dots + \underset{N_{L/K}(\beta) =
1}{\underbrace{\beta \sigma(\beta) \dots \sigma^{n-1}(\beta)}}
\underset{=\gamma}{\underbrace{\sigma^n(\gamma)}} = \alpha$ }

\item Sei $L/K$ zyklische Galoiserweiterung, $n = [L:K]$, $\sigma \in$ Gal$(L/K)$
ein Erzeuger. Zu $\beta \in L$ mit tr$_{L/K}(\beta) = 0$ gibt es $\alpha
\in L$ mit $\beta = \alpha - \sigma(\alpha)$

\sbew{0.9}{Sei $\gamma \in L$ mit tr$_{L/K}(\gamma) \neq 0$ und $\\\alpha
\defeqr \frac{1}{\mbox{\small tr}_{L/K}(\gamma)} \cd [ \beta \sigma(\gamma) +
(\beta + \sigma(\beta))\sigma^2(\gamma) + \dots + (\beta +
\sigma(\beta) + \dots + \sigma^{n-2}(\beta))\sigma^{n-1}(\gamma)]$
$\\\Ra \sigma(\alpha) = \frac{1}{\mbox{\small
tr}_{L/K}(\gamma)}[\sigma(\beta)\sigma^2(\gamma) +(\sigma(\beta) +
\sigma^2(\beta))\sigma^3(\gamma)+\dots+(\sigma(\beta) + \dots +
\sigma^{n-1}(\beta))\sigma^n(\gamma)]$ $\\\Ra (\alpha -
\sigma(\alpha))\mbox{tr}_{L/K}(\gamma) = \beta\sigma(\gamma) + \beta
\sigma^2(\gamma) + \dots + \beta\sigma^{n-1}(\gamma) -
\underset{-\beta}{\underbrace{(\sigma(\beta)+\dots+\sigma^{n-1}(\beta))}}
\gamma = \beta \cd \mbox{tr}_{L/K}(\gamma)$}
\end{enum}
\end{Satz}

\begin{Folg}
Voraussetzungen wie in Satz 19.
\begin{enum}

\item Ist char$(K)$ kein Teiler von $n=[L:K]$ und enthält $K$ eine
primitive $n$-te Einheitswurzel $\zeta$, so gibt es ein primitives
Element $\alpha \in L$, so daß das Minimalpolynom von $\alpha$ über
$K$ \[X^n - \gamma\] ist für ein $\gamma \in K$.
(\textit{''Kummer-Erweiterung''})

\item Ist char$(K) = [L:K] = p$, so gibt es ein primitives Element
$\alpha \in L$, so daß das Minimalpolynom von $\alpha$ über $K$
\[ X^p - X - \gamma\] für ein $\gamma \in K$.
(\textit{''Artin-Schreier-Erweiterung''})

\end{enum}
\bew{}{\item Es ist $N_{L/K}(\zeta) = \zeta^n = 1 =
N_{L/K}(\zeta^{-1}) \overset{\mbox{\scriptsize Satz \ref{Satz
19}}}{\Ra}$ es gibt $\alpha \in L$ mit $\sigma(\alpha) = \zeta
\alpha \Ra \sigma^i(\alpha) = \zeta^i \alpha,\;i=1,\dots,n-1$ $\Ra$
Das Minimalpolynom von $\alpha$ über $K$ hat $n$ verschiedene
Nullstellen $\Ra L = K(\alpha)$.

Außerdem ist $\sigma(\alpha^n) = \sigma(\alpha)^n = \alpha^n \Ra
\gamma \defeqr \alpha^n \in K$ $\Ra$ Das Minimalpolynom von $\alpha$
ist $X^n - \gamma$

\item tr$_{L/K}(1) = 1 + \dots + 1 = p = 0
\overset{\scriptsize\ref{Satz 19}}{\Ra}$ es gibt $\alpha \in L$ mit
$\sigma(\alpha) = \alpha + 1 \Ra \sigma^i(\alpha) = \alpha +
i,\;i=0,\dots,n-1 \Ra K(\alpha) = L$

$\sigma(\alpha^p - \alpha) = \sigma(\alpha)^p - \sigma(\alpha) =
\alpha^p + 1 - (\alpha + 1) = \alpha^p - \alpha \Ra \alpha^p -
\alpha \defeql \gamma \in K$ und $X^p -X - \gamma$ ist
Minimalpolynom von $\alpha$.}
\end{Folg}

\begin{Prop}
Sei $L/K$ einfache Körpererweiterung, $L =
K(\alpha)$

\begin{enum}

\item Ist $\alpha$ Nullstelle eines Polynoms $X^n - \gamma$ für ein
$\gamma \in K$ und enthält $K$ eine primitive $n$-te Einheitswurzel
$\zeta$, so ist $L/K$ galoissch, Gal$(L/K)$ zyklisch, $d\defeqr
[L:K]$ ist Teiler von $n$, $\alpha^d \in K$, $X^d - \alpha^d$ ist
Minimalpolynom von $\alpha$

\item Ist char$(K) = p > 0$ und $\alpha \in L\setminus K$ Nullstelle
eines Polynoms $X^p - X - \gamma$ für ein $\gamma \in K$, so ist
$L/K$ galoissch und Gal$(L/K) \cong \mathbb{Z}/p\mathbb{Z}$

\end{enum}

\bew{}{\item Die Nullstellen von $X^n - \gamma$ sind
$\alpha,\zeta\alpha,\dots,\zeta^{n-1}\alpha \Ra L$ ist
Zerfällungskörper von $X^n - \gamma$, also normal und separabel,
also galoissch.

Für $\sigma \in$ Gal$(L/K)$ ist $\sigma(\alpha) =
\zeta^{\nu(\sigma)} \alpha$ für ein $\nu(\sigma) \in
\mathbb{Z}/n\mathbb{Z}$.

$\sigma \mapsto \nu(\sigma)$ ist injektiver Gruppenhomomorphismus
Gal$(L/K) \ra \mathbb{Z}/n\mathbb{Z} \Ra$ Gal$(L/K)$ ist zyklisch,
da Untergruppe von $\mathbb{Z}/n\mathbb{Z} \Ra d = [L:K]$ teilt $n$.

Für $\sigma \in$ Gal$(L/K)$ ist $\sigma(\alpha^d) =
\left(\zeta^{\nu(\sigma)}\right)^d \alpha^d = \alpha^d \Ra \alpha^d
\in K$; $X^d - \alpha^d$ ist Minimalpolynom, da $L=K(\alpha)$ und
$[K(\alpha):K] = d$.

\item Für $i \in \mathbb{F}_p$ ist $(\alpha+i)^p - (\alpha + i) -
\gamma = \alpha^p + \underset{=i}{\underbrace{i^p}} - \alpha - i -
\gamma = 0 \Ra X^p - X - \gamma$ hat $p$ verschieden Nullstellen
$\Ra L$ ist Zerfällungskörper von $X^p - X - \gamma$ und $L/K$ ist
separabel. Außerdem folgt: Gal$(L/K) \cong \mathbb{Z}/p\mathbb{Z}$ }
\end{Prop}