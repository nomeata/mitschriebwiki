\documentclass[a4paper,twoside,DIV15,BCOR12mm,chapterprefix=true,headings=onelinechapter]{scrbook}
\usepackage{ana}

\author{Die Mitarbeiter von \url{http://mitschriebwiki.nomeata.de/}}
\title{Analysis III - Bachelorversion}
\makeindex

\begin{document}
\maketitle

\renewcommand{\thechapter}{\Roman{chapter}}
%\chapter{Inhaltsverzeichnis}
\addcontentsline{toc}{chapter}{Inhaltsverzeichnis}
\tableofcontents

\chapter{Vorwort}

\section{Über dieses Skriptum}
Dies ist ein Mitschrieb der Vorlesung \glqq Analysis III\grqq\ von Herrn Schmoeger im
Wintersemester 2010 an der Universität Karlsruhe (KIT). Die Mitschriebe der Vorlesung werden mit ausdrücklicher Genehmigung 
von Herrn Schmoeger hier veröffentlicht, Herr Schmoeger ist für den Inhalt nicht 
verantwortlich.

\section{Wer}
Gestartet wurde das Projekt von Joachim Breitner. Beteiligt an diesem Mitschrieb sind \ldots na mal schaun.

\section{Wo}
Alle Kapitel inklusive \LaTeX-Quellen können unter \url{http://mitschriebwiki.nomeata.de} abgerufen werden.
Dort ist ein \emph{Wiki} eingerichtet und von Joachim Breitner um die \LaTeX-Funktionen erweitert.
Das heißt, jeder kann Fehler nachbessern und sich an der Entwicklung
beteiligen. Auf Wunsch ist auch ein Zugang über \emph{Subversion} möglich.


\renewcommand{\thechapter}{\arabic{chapter}}
\renewcommand{\chaptername}{§}
\renewcommand*{\chapterformat}{§\,\thechapter \enskip}
\setcounter{chapter}{-1}

\chapter{Vorbereitungen}
In diesem Paragraphen seien $X,Y,Z$ Mengen ($\ne\emptyset$) und $f:X\to Y, g:Y\to Z$ Abbildungen.
\begin{enumerate}
\index{Potenzmenge}
\index{Disjunktheit}
\item 
\begin{enumerate}
\item $\mathcal{P}(X):=\{A:A\subseteq X\}$ heißt \textbf{Potenzmenge} von $X$.
\item Sei $\fm\subseteq\mathcal{P}(X)$, so heißt $\fm$ \textbf{disjunkt}, genau dann wenn $A\cap B=\emptyset$ für $A,B\in\fm$ mit $A\ne B$.
\item Sei $(A_j)$ eine Folge in $\mathcal{P}(X)$ (also $A_j\subseteq X$), so heißt $(A_j)$ \textbf{disjunkt}, genau dann wenn $\{A_1,A_2,\ldots\}$ disjunkt ist. In diesem Fall schreibe: $\dot{\bigcup}_{j=1}^\infty:=\bigcup_{j=1}^\infty A_j$\\
Allgemein sei $\bigcup_{j=1}^\infty A_j:=\bigcup A_j$ und $\bigcap_{j=1}^\infty A_j:=\bigcap A_j$.
\end{enumerate}
\item Sei $A\subseteq X$, für $x\in X$ definiere
\[\mathds{1}_A(x):=\begin{cases}1, x\in A\\ 0, x\in A^C\end{cases}\]
wobei $A^C:=X\setminus A$.
\item Sei $B\subseteq Y$ dann ist $f^{-1}(B):=\{x\in X: f(x)\in B\}$ und es gelten folgende Eigenschaften:
\begin{enumerate}
\item $f^{-1}(B^C)=f^{-1}(B)^C$
\item Ist $B_j$ eine Folge in $\mathcal{P}(Y)$, so gilt:
\begin{align*}
f^{-1}(\bigcup B_j)=\bigcup f^{-1}(B_j)\\
f^{-1}(\bigcap B_j)=\bigcap f^{-1}(B_j)\\
\end{align*}
\item Ist $C\subseteq Z$, so gilt:
\[(g\circ f)^{-1}(C)=f^{-1}(g^{-1}(C))\]
\end{enumerate}
\item $\sum_{j=1}^\infty a_j =: \sum a_j$
\end{enumerate}

\chapter{$\sigma$-Algebren und Maße}
In diesem Paragraphen sei $\emptyset\ne X$ eine Menge.

\begin{definition}
\index{$\sigma$-!Algebra}
Sei $\fa\subseteq\mathcal{P}(X)$, $\fa$ heißt eine \textbf{$\sigma$-Algebra} auf $X$, wenn gilt:
\begin{enumerate}
\item[($\sigma_1$)] $X\in\fa$
\item[($\sigma_2$)] Ist $A\in\fa$, so ist auch $A^C\in\fa$.
\item[($\sigma_3$)] Ist $(A_j)$ eine Folge in $\fa$, so ist $\bigcup A_j\in\fa$.
\end{enumerate}
\end{definition}

\begin{beispiel}
\begin{enumerate}
\item $\{X,\emptyset\}$ und $\mathcal{P}(X)$ sind $\sigma$-Algebren auf $X$.
\item Sei $A\subseteq X$, dann ist $\{X,\emptyset, A, A^C\}$ eine $\sigma$-Algebra auf $X$.
\item $\fa:=\{A\subseteq X: A$ abzählbar oder $A^C$ abzählbar$\}$ ist eine $\sigma$-Algebra auf $X$.
\end{enumerate}
\end{beispiel}

\begin{lemma}
Sei $\fa$ eine $\sigma$-Algebra auf $X$, dann:
\begin{enumerate}
\item $\emptyset\in\fa$
\item Ist $(A_j)$ eine Folge in $\fa$, so ist $\bigcap A_j\in\fa$.
\item Sind $A_1,\ldots,A_n\in\fa$, so gilt:
\begin{enumerate}
\item $A_1\cup\cdots\cup A_n\in\fa$
\item $A_1\cap\cdots\cap A_n\in\fa$
\item $A_1\setminus A_2\in\fa$
\end{enumerate}
\end{enumerate}
\end{lemma}

\begin{beweis}
\begin{enumerate}
\item $\emptyset=X^C\in\fa$ (nach ($\sigma_2$)).
\item $D:=\bigcap A_j$. $D^C=\bigcup A_j^C\in\fa$ (nach ($\sigma_2$) und ($\sigma_3$)), also gilt auch $D=(D^C)^C\in\fa$.
\item \begin{enumerate}
\item $A_1\cup\cdots\cup A_n\in\fa$ folgt aus ($\sigma_3$) mit $A_{n+j}:=\emptyset$ ($j\ge 1$).
\item $A_1\cap\cdots\cap A_n\in\fa$ folgt aus (2) mit $A_{n+j}:=X$ ($j\ge 1$).
\item $A_1\setminus A_2=A_1\cap A_2^C\in\fa$
\end{enumerate}
\end{enumerate}
\end{beweis}

\begin{lemma}
Sei $\emptyset\ne\cf$ eine Menge von $\sigma$-Algebren auf $X$. Dann ist 
\[\fa_0:=\bigcap_{\fa\in\cf}\fa\]
eine $\sigma$-Algebra auf $X$.
\end{lemma}

\begin{beweis}
\begin{enumerate}
\item[($\sigma_1$)] $\forall\fa\in\cf:X\in\fa\implies X\in\fa_0$.
\item[($\sigma_2$)] Sei $A\in\fa_0$, dann gilt:
\begin{align*}
\forall\fa\in\cf:A\in\fa &\implies \forall\fa\in\cf:A^C\in\fa\\
&\implies A^C\in\fa_0
\end{align*}
\item[($\sigma_3$)] Sei $(A_j)$ eine Folge in $\fa_0$, dann ist $(A_j)$ Folge in $\fa$ für alle $\fa\in\cf$, dann gilt:
\begin{align*}
\forall\fa\in\cf:\bigcap A_j\in\fa \implies \bigcap A_j\in\fa_0
\end{align*}
\end{enumerate}
\end{beweis}

\begin{definition}
\index{Erzeuger}
Sei $\emptyset\ne\mathcal{E}\subseteq\mathcal{P}(X)$ und $\cf:=\{\fa:\fa$ ist $\sigma$-Algebra auf $X$ mit $\mathcal{E}\subseteq\fa\}$. Definiere
\[\sigma(\mathcal{E}):=\bigcap_{\fa\in\cf}\fa\]
Dann ist wegen 1.2 $\sigma(\mathcal{E})$ eine $\sigma$-Algebra auf $X$. $\sigma(\mathcal{E})$ heißt die \textbf{von $\mathcal{E}$ erzeugte $\sigma$-Algebra}. $\mathcal{E}$ heißt ein \textbf{Erzeuger} von $\sigma(\mathcal{E})$.
\end{definition}

\begin{lemma}
Sei $\emptyset\ne\mathcal{E}\subseteq\mathcal{P}(X)$.
\begin{enumerate}
\item $\mathcal{E}\subseteq\sigma(\mathcal{E})$. $\sigma(\mathcal{E})$ ist die "kleinste" $\sigma$-Algebra auf $X$, die $\mathcal{E}$ enthält.
\item Ist $\mathcal{E}$ eine $\sigma$-Algebra, so ist $\sigma(\mathcal{E})=\mathcal{E}$.
\item Ist $\mathcal{E}\subseteq\mathcal{E}'$, so ist $\sigma(\mathcal{E})\subseteq\sigma(\mathcal{E}')$.
\end{enumerate}
\end{lemma}

\begin{beweis}
\begin{enumerate}
\item Klar nach Definition.
\item $\fa:=\mathcal{E}$, dann gilt $\fa\subseteq\sigma(\mathcal{E})\subseteq\fa$.
\item $\mathcal{E}\subseteq\mathcal{E}'\subseteq\sigma(\mathcal{E}')$, also folgt nach Definition $\sigma(\mathcal{E})\subseteq\sigma(\mathcal{E}')$.
\end{enumerate}
\end{beweis}

\begin{beispiel}
\begin{enumerate}
\item Sei $A\subseteq X$ und $\mathcal{E}:=\{A\}$. Dann ist $\sigma(\mathcal{E})=\{X,\emptyset,A,A^C\}$.
\item $X:=\{1,2,3,4,5\}, \mathcal{E}:=\{\{1\},\{1,2\}\}$. Dann gilt:
\[\sigma(\mathcal{E}):=\{X,\emptyset, \{1\},\{2\},\{1,2\},\{3,4,5\},\{1,3,4,5\},\{2,3,4,5\}\}\]
\end{enumerate}
\end{beispiel}

\begin{erinnerung}
\index{Offenheit}\index{Abgeschlossenheit}
Sei $d\in\mdn, X\subseteq\mdr^d$. $A\subseteq X$ heißt \textbf{offen} (\textbf{abgeschlossen}) in $X$, genau dann wenn ein offenes (abgeschlossenes) $G\subseteq\mdr^d$ existiert mit $A=X\cap G$.\\
Beachte: $A$ abgeschlossen in $X$ $\iff$ $X\setminus A$ offen in $X$.
\end{erinnerung}

\begin{definition}
\index{Borel!$\sigma$-Algebra}\index{$\sigma$-!Algebra, Borelsche}
\index{Borel!Mengen}
Sei $X\subseteq\mdr^d$.
\begin{enumerate}
\item $\mathcal{O}(X):=\{A\subseteq X:A$ ist offen in $X\}$
\item $\fb(X):=\sigma(\mathcal{O}(X))$ heißt \textbf{Borelsche $\sigma$-Algebra} auf $X$.
\item $\fb_d:=\fb(\mdr^d)$. Die Elemente von $\fb_d$ heißen \textbf{Borelsche Mengen} oder \textbf{Borel-Mengen}.
\end{enumerate}
\end{definition}

\begin{beispiel}
\begin{enumerate}
\item Sei $X\subseteq\mdr^d$. Ist $A\subseteq$ offen (abgeschlossen) in $X$, so ist $A\in\fb(X)$.
\item Ist $A\subseteq\mdr^d$ offen (abgeschlossen) so ist $A\in\fb_d$.
\item Sei $d=1, A=\mdq$. $\mdq$ ist abzählbar, also $\mdq=\{r_1,r_2,\ldots\}$ (mit $r_i\ne r_j$ für $i\ne j$). Also ist $\mdq=\bigcup \{r_j\}$. Sei nun $r\in\mdq$, dann ist $B:=(-\infty,r)\cup(r,\infty)\in\fb_1$. Daraus folgt $\{r_j\}\in\fb_1$, also auch $\mdq\in\fb_1$.\\
Allgemeiner lässt sich zeigen: $\mdq^d:=\{(x_1,\ldots,x_n):x_j\in\mdq (j=1,\ldots,n)\}\in\fb_d$.
\end{enumerate}
\end{beispiel}

\begin{definition}
\index{Intervall}
\index{Halbraum}
\begin{enumerate}
\item Seien $I_1,\ldots,I_d$ Intervalle in $\mdr$. $I_1\times\cdots\times I_d$ heißt ein \textbf{Intervall} in $\mdr^d$.
\item Seien $a=(a_1,\ldots,a_d), b=(b_1,\ldots,b_d)\in\mdr^d$.
\[a\le b:\iff a_j\le b_j\quad (j=1,\ldots,d)\]
\item Seien $a,b\in\mdr^d$ und $a\le b$.
\begin{align*}
(a,b)&:=(a_1,b_1)\times\cdots\times(a_d,b_d)\\
(a,b]&:=(a_1,b_1]\times\cdots\times(a_d,b_d]\\
[a,b)&:=[a_1,b_1)\times\cdots\times[a_d,b_d)\\
[a,b]&:=[a_1,b_1]\times\cdots\times[a_d,b_d]
\end{align*}
mit der Festlegung $(a,b):=(a,b]:=[a,b):=\emptyset$, falls $a_j=b_j$ für ein $j\in\{1,\ldots,d\}$.
\item Für $k\in\{1,\ldots,d\}$ und $\alpha\in\mdr$ definiere die folgenden \textbf{Halbräume}:
\begin{align*}
H_k^-(\alpha):=\{(x_1,\ldots,x_d)\in\mdr^d:x_k\le\alpha\}\\
H_k^+(\alpha):=\{(x_1,\ldots,x_d)\in\mdr^d:x_k\ge\alpha\}
\end{align*}
\end{enumerate}
\end{definition}

\begin{satz}[Erzeuger der Borelschen $\sigma$-Algebra auf $\mdr$]
Es seien $\ce_1,\ce_2,\ce_3$ wie folgt definiert:
\begin{align*}
\ce_1&:=\{(a,b):a,b\in\mdq^d,a\le b\}\\
\ce_2&:=\{(a,b]:a,b\in\mdq^d, a\le b\}\\
\ce_3&:=\{H^-_k(\alpha):\alpha\in\mdq, k=1,\ldots,d\}
\end{align*}
Dann gilt:
\[\fb_d=\sigma(\ce_1)=\sigma(\ce_2)=\sigma(\ce_3)\]
Entsprechendes gilt für die anderen Typen von Intervallen und Haupträumen.
\end{satz}

\begin{beweis}
\begin{enumerate}
\item Sei $G\in\co(\mdr^d), \fm:=\{(a,b):a,b\in\mdq^d,a\le b, (a,b)\subseteq G\}$. Dann ist $\fm$ abzählbar und $G=\bigcup_{I\in\fm}I$. also gilt:
\[G\in\sigma(\ce_1)\implies \fb_d=\sigma(\co(\mdr^d))\subseteq\sigma(\ce_1)\]
\item Sei $(a,b)\in\ce_1$.\\
\textbf{Fall 1:} $(a,b)=\emptyset\in\ce_2\subseteq\sigma(\ce_2)$\\
\textbf{Fall 2:} $(a,b)\ne\emptyset, a=(a_1\ldots,a_d), b=(b_1\ldots,b_d)$. Dann gilt für alle $j\in\{1,\ldots,d\}:a_j<b_j$, also gilt auch:
\[\exists N\in\mdn:\forall n\ge N: \forall j\in\{1,\ldots,d\}:a_j<b_j-\frac1n\]
Definiere $c_n:=(\frac1n,\ldots,\frac1n)\in\mdq^d$. Dann gilt:
\[(a,b)=\bigcup_{n\ge N}(a,b-c_n]\in\sigma(\ce_2)\]
Also auch $\ce_1\subseteq\sigma(\ce_2)$ und damit $\sigma(\ce_1)\subseteq\sigma(\ce_2)$.
\item Seien $a = (a_1,\ldots,a_d), b=(b_1,\ldots,b_d) \in \mdq^d$ mit $a \leq b$. Nachrechnen:
\[(a,b] = \bigcap_{k=1}^d (H^-_k(b_k) \cap H^-_k(a_k)^C) \in \sigma(\ce_3). \]
Das heißt $\ce_2 \subseteq \sigma(\ce_3)$ und damit auch $\sigma(\ce_2) \subseteq \sigma(\ce_3)$. 
\item $H^-_k(\alpha)$ ist abgeschlossen, somit ist $H^-_k(\alpha)^C$ offen und damit $H^-_k(\alpha)^C \in \fb_d$, also auch $H^-_k(\alpha) \in \fb_d$. Damit ist $\ce_3 \subseteq \fb_d \implies \sigma(\ce_3) \subseteq \fb_d$. 
\end{enumerate}
\end{beweis}

\begin{definition}
\index{Spur}
Sei $\emptyset \neq \fm \subseteq \mathcal{P}(X)$ und $\emptyset \neq Y \subseteq X$. 
\[\fm_Y := \{A \cap Y : A \in \fm\}\] 
heißt die \textbf{Spur von $\fm$ in $Y$}.
\end{definition}

\begin{satz}[Spuren und $\sigma$-Algebren]
Sei $\emptyset \neq Y \subseteq X$ und $\fa$ eine $\sigma$-Algebra auf $X$.
\begin{enumerate}
\item $\fa_Y$ ist eine $\sigma$-Algebra auf $Y$.
\item $\fa_Y \subseteq \fa \iff Y \in \fa$
\item Ist $\emptyset \neq \ce \subseteq \mathcal{P}(Y)$, so ist $\sigma(\ce_Y) = \sigma(\ce)_Y$.
\end{enumerate}
\end{satz}

\begin{beweis}
\begin{enumerate}
\item \begin{enumerate}
\item[($\sigma_1$)] Es ist $Y=Y\cap X\in\fa_Y$, da $X\in\fa$.
\item[($\sigma_2$)] Sei $B\in\fa_Y$, dann existiert ein $A\in\fa$ mit $B=A\cap Y$. Also ist $Y\setminus B=(X\setminus A)\cap Y\in\fa_Y$, da $X\setminus A\in\fa$ ist.
\item[($\sigma_3$)] Sei $(B_j)$ eine Folge in $\fa_Y$, dann existiert eine Folge $(A_j)\in\fa^\mdn$ mit $B_j=A_j\cap Y$. Es gilt:
\[\bigcup B_j=\bigcup(A_j\cap Y)=(\bigcup A_j)\cap Y\in\fa_Y\]
\end{enumerate}
\item Der Beweis erfolgt durch Implikation in beiden Richtungen:
\begin{enumerate}
\item["`$\implies$"'] Es gilt $Y\in\fa_Y\subseteq\fa$.
\item["`$\impliedby$"'] Sei $B\in\fa_Y$, dann existiert ein $A\in\fa$ mit $B=A\cap Y\in\fa$.
\end{enumerate}
\item Es gilt:
\begin{align*}
\ce\subseteq\sigma(\ce)&\implies\ce_Y\subseteq\sigma(\ce)_Y\\
&\implies\sigma(\ce_Y)\subseteq\sigma(\ce)_Y
\end{align*}
Sei nun:
\[\cd:=\{A\subseteq X:A\cap Y\in\sigma(\ce_Y)\}\]
Übung: $\cd$ ist eine $\sigma$-Algebra auf $X$.\\
Sei $E\in\ce$ dann ist $E\cap Y\in\ce_Y\subseteq\sigma(\ce_Y)$ also $E\in\cd$ und damit $\ce\subseteq\cd$. Daraus folgt:
\begin{align*}
\sigma(\ce)_Y&\subseteq\sigma(\cd)_Y=\cd_Y=\{A\cap Y:A\in\cd\}\\
&\subseteq\sigma(\ce_Y)
\end{align*}
\end{enumerate}
\end{beweis}

\begin{folgerungen}
Sei $X\subseteq\mdr^d$. Dann gilt:
\begin{enumerate}
\item $\fb(X)=(\fb_d)_X$
\item Ist $X\in\fb_d$, so ist $\fb(X)=\{A\in\fb_d:A\subseteq X\}\subseteq\fb_d$.
\end{enumerate}
\end{folgerungen}

\begin{definition}
Wir fügen $\mdr$ das Symbol $+\infty$ hinzu. Es soll gelten:
\begin{enumerate}
\item $\forall a\in\mdr:a<+\infty$
\item $\pm a+(+\infty):=+\infty=:(+\infty)\pm a$
\item $(+\infty)+(+\infty):=+\infty$
\end{enumerate}
Sei etwa $[0,+\infty]:=[0,\infty)\cup\{+\infty\}$.
\begin{enumerate}
\item Sei $(x_n)$ eine Folge in $[0,+\infty]$. Es gilt:
\[x_n\stackrel{n\to\infty}{\to}\infty:\iff \forall c>0\exists n_c\in\mdn:\forall n\ge n_c: x_n> c\]
\item Sei $(a_n)$ eine Folge in $[0,+\infty]$. Es gilt
\[\sum_{n=1}^\infty a_n=\sum a_n = +\infty\]
genau dann wenn $a_j=+\infty$ für ein $j\in\mdn$ oder, falls alle $a_j<+\infty$, wenn $\sum a_n$ divergiert.
\end{enumerate} 
Wegen 13.1 Ana I können Reihen der obigen Form beliebig umgeordnet werden, ohne dass sich ihr Wert verändert.
\end{definition}

\begin{definition}
\index{Maß}
\index{$\sigma$-!Additivität}
\index{Maßraum}
\index{Maß!endliches}
\index{Wahrscheinlichkeitsmaß}\index{Maß!Wahrscheinlichkeits-}
Sei $\fa$ eine $\sigma$-Algebra auf $X$ und $\mu:\fa\to[0,+\infty]$ eine Abbildung. $\mu$ heißt ein \textbf{Maß} auf $\fa$, genau dann wenn gilt:
\begin{enumerate}
\item[$(M_1)$] $\mu(\emptyset)=0$
\item[$(M_2)$] Ist $(A_j)$ eine disjunkte Folge in $\fa$, so ist $\mu(\bigcup A_j)=\sum\mu(A_j)$. Diese Eigenschaft heißt \textbf{$\sigma$-Additivität}.
\end{enumerate}
Ist $\mu$ ein Maß auf $\fa$, so heißt $(X,\fa,\mu)$ ein \textbf{Maßraum}.\\
Ein Maß $\mu$ heißt \textbf{endlich}, genau dann wenn $\mu(X)<\infty$. Ein Maß $\mu$ heißt ein \textbf{Wahrscheinlichkeitsmaß}, genau dann wenn $\mu(X)=1$ ist.
\end{definition}

\begin{beispiel}
\index{Punktmaß}\index{Maß!Punkt-}
\index{Dirac-Maß}\index{Maß!Dirac-}
\index{Zählmaß}\index{Maß!Zähl-}
\begin{enumerate}
\item Sei $\fa=\cp(X)$ und $x_0\in X$. $\delta_{x_0}:\fa\to[0,+\infty]$ sei definiert durch:
\[\delta_{x_0}(A):=
\begin{cases}
1,\ x_0\in A\\
0,\ x_0\not\in A
\end{cases}\]
Klar ist, dass $\delta_{x_0}(\emptyset)=0$ ist.\\
Sei $(A_j)$ eine disjunkte Folge in $\fa$.
\[\delta_{x_0}(\bigcup A_j)=
\left.\begin{cases}
1,\ x_0\in\bigcup A_j\\
0,\ x_0\not\in\bigcup A_j
\end{cases}\right\}=\sum\mu(A_j)\]
$\delta_{x_0}$ ist ein Maß auf $\cp(X)$ und heißt \textbf{Punktmaß} oder \textbf{Dirac-Maß}.
\item Sei $X:=\mdn$, $\fa:=\cp(X)$ und $(p_j)$ eine Folge in $[0,+\infty]$. Definiere $\mu:\fa\to[0,+\infty]$ durch:
\begin{align*}
\mu(A):=
\begin{cases}
0&,A=\emptyset\\
\sum_{j\in A}p_j&,A\ne\emptyset
\end{cases}
\end{align*}
Übung:$\mu$ ist ein Maß auf $\fa=\cp(\mdn)$ und heißt ein \textbf{Zählmaß}. Sind alle $p_j=1$, so ist $\mu(A)$ gerade die Anzahld er Elemente von $A$.
\item Sei $(X,\fa,\mu)$ ein Maßraum, $\emptyset\ne Y\subseteq X$ und $\fa_0\subseteq\fa$ eine $\sigma$-Algebra auf $Y$. Definiere $\mu_0:\fa_0\to[0,+\infty]$ durch $\mu_0(A):=\mu(A)$ ($A\in\fa_0$). Dann ist $(Y,\fa_0,\mu_0)$ ein Maßraum.\\
Ist spezieller $Y\in\fa$, so ist $\fa_0:=\fa_Y\subseteq\fa$ und man definiert $\mu_{|Y}:\fa_Y\to[0,+\infty]$ durch $\mu_{|Y}(A)=\mu(A)$.
\end{enumerate}
\end{beispiel}

\begin{satz}
\((X,\fa,\mu)\) sei ein Ma\ss raum, es seien \(A,B\in\fa\) und \((A_{j})\) sei eine Folge in \(\fa\). Dann:
\begin{enumerate}
\item \(A\subseteq B\,\implies\,\mu(A)\leq\mu(B)\)
\item Ist \(\mu(A)<\infty\,\implies\,\mu(B\setminus A)=\mu(B)-\mu(A)\)
\item Ist \(\mu\) endlich, dann ist \(\mu(A)<\infty\) und \(\mu(A^{C})=\mu(X)-\mu(A)\)
\item \(\mu\left(\bigcup A_{j}\right)\leq\sum{\mu(A)}\) (\(\sigma\)-Subaddidivit\"at)
\item Ist \(A_{1}\subseteq A_{2}\subseteq A_{3}\subseteq\ldots\), so ist \(\mu(\bigcup A_{j})=\lim_{n\to\infty}{\mu(A_{n})}\)
\item Ist \(A_{1}\supseteq A_{2}\supseteq A_{3}\supseteq\ldots\) und \(\mu(A)<\infty\), so ist
	\(\mu(\bigcap A_{j})=\lim_{n\to\infty}{\mu(A_{n})}\)
\end{enumerate}
\end{satz}
\begin{beweis}
\begin{enumerate}
% Eigentlich muesste es in folgender Zeile statt B=(B\setminus A)\cup A korrekt 
% heissen: B=(B\setminus A)\cupdot A -- Spaeter...
\item[(1)-(3)] \(B=(B\setminus A)\cup A\). Dann: \(\mu(B)=\underbrace{\mu(B\setminus A)}_{\geq0}+\mu(A)\geq\mu(A)\)
\item[(4)] % Das muesste jetzt eigentlich Punkt 4 sein...
\(B_{1}=A_{1},\,B_{k}:=A_{k}\setminus\bigcup_{j=1}^{k-1}{A_{j}}\quad(k\geq 2)\)

Dann: \(B_{j}\in\fa,\,B_{j}\subseteq A_{j}\,(j\in\MdN);\,(B_{j})\) disjunkt und \(\bigcup A_{j}=\bigcup B_{j}\). Dann:
\[
\mu\left(\bigcup A_{j}\right)=\mu\left(\bigcup B_{j}\right)=\sum{\underbrace{\mu(B_{j})}_{\leq\mu(A_{j})}}\leq\sum{\mu(A_{j})}
\]
\item[(5)] % Das muesste jetzt eigentlich Punkt 5 sein...
\(B_{1}=A_{1},\,B_{k}=A_{k}\setminus A_{k-1}\,(k\geq 2)\)

Dann: \(B_{j}\subseteq\fa;\,B_{j}\subseteq A_{j}\,(j\in\MdN);\,\bigcup A_{j}=\bigcup B_{j}\) und \(A_{n}=\bigcup_{j=1}^{n}{B_{j}}\)%\bigcupdot_{j=1}^{n}{B_{j}}\)

Dann: \(\mu(\bigcup A_{j})=\mu(\bigcup B_{j})=\sum{\mu(B_{j})}=\lim_{n\to\infty}{\underbrace{\sum_{j=1}^{n}{\mu(B_{j})}}_{=\mu\left(\bigcup_{j=1}^{n}{B_{j}}\right)=\mu(A_{n})}}\)
\item[(6)] \"Ubung
\end{enumerate}
\end{beweis}

\chapter{Das Lebesguema\ss}
\index{Lebesguemass}
In diesem Kapitel sei \(X\) eine Menge, \(X\neq\varnothing\).
\begin{definition}
Sei \(\varnothing\neq\mathfrak{R}\subseteq\mathcal{P}(X)\). \(\mathfrak{R}\)
hei\ss t ein Ring (auf \(X\)), genau dann wenn gilt:
\begin{enumerate}
\item \(\varnothing\in\mathfrak{R}\)
\item \(A,B\in\mathfrak{R}\,\implies\,A\cup B,\,B\setminus A\in\mathfrak{R}\)
\end{enumerate}
\end{definition}

\appendix
\chapter{Satz um Satz (hüpft der Has)}
\listtheorems{satz,wichtigedefinition}

\renewcommand{\indexname}{Stichwortverzeichnis}
\addcontentsline{toc}{chapter}{Stichwortverzeichnis}
\printindex

\chapter{Credits für Analysis III} Abgetippt haben die folgenden Paragraphen:\\% no data in Ana2Vorwort.tex
\textbf{§ 1: $\sigma$-Algebren und Maße}: Rebecca Schwerdt, Peter Pan, Philipp Ost\\


\end{document}
