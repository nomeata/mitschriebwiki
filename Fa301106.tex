\documentclass[a4paper,11pt]{book}

\usepackage{amssymb}
\usepackage{amsmath}
\usepackage{amsfonts}
\usepackage{ngerman}
%\usepackage{graphicx}
\usepackage{fancyhdr}
\usepackage{euscript}
\usepackage{makeidx}
\usepackage{hyperref}
\usepackage[amsmath,thmmarks,hyperref]{ntheorem}
\usepackage{enumerate}
\usepackage{url}
\usepackage{mathtools}
\usepackage[arrow, matrix, curve]{xy}
%\usepackage{pst-all}
%\usepackage{pst-add}
%\usepackage{multicol}

\usepackage[latin1]{inputenc}

%%Zahlenmengen
%Neue Kommando-Makros
\newcommand{\R}{{\mathbb R}}
\newcommand{\C}{{\mathbb C}}
\newcommand{\N}{{\mathbb N}}
\newcommand{\Q}{{\mathbb Q}}
\newcommand{\Z}{{\mathbb Z}}
\newcommand{\K}{{\mathbb K}}
\newcommand{\sL}{{\mathcal L}}
\newcommand{\sll}{{\mathcal l}}
\newcommand{\sn}[1]{||#1||_{\infty}}
\newcommand{\eps}{{\varepsilon}}
\newcommand{\begriff}[1]{\textbf{#1}} %das sollte man noch ändern!
\newcommand{\eb}{\begin{flushright} \rule{1ex}{1ex} \end{flushright}}
\newcommand{\id}{1\hspace{-0,9ex}\raisebox{-0,2ex}{1}}


% Seitenraender
\textheight22cm
\textwidth14cm
\topmargin-0.5cm
\evensidemargin0,5cm
\oddsidemargin0,5cm
\headheight14pt

%%Seitenformat
% Keine Einrückung am Absatzbeginn
\parindent0pt

\DeclareMathOperator{\unif}{Unif}
\DeclareMathOperator{\var}{Var}
\DeclareMathOperator{\cov}{Cov}


\def\AA{ \mathcal{A} }
\def\PM{ \EuScript{P} } 
\def\EE{ \mathcal{E} }
\def\BB{ \mathfrak{B} } 
\def\DD{ \mathcal{D} } 
\def\NN{ \mathcal{N} } 

% Komische Symbole
\def\folgt{\ensuremath{\implies}}
\newcommand{\folgtnach}[1]{\ensuremath{\DOTSB\;\xRightarrow{\text{#1}}\;}}
\def\equizu{\ensuremath{\iff}}
\def\d{\mbox{d}}
\def\fs{\stackrel{f.s.}{\rightarrow }}

%Nummerierungen
\newtheorem{Def}{Definition}[chapter]
\newtheorem{Sa}[Def]{Satz}
\newtheorem{Lem}[Def]{Lemma}
\newtheorem{Kor}[Def]{Korollar}
\theorembodyfont{\normalfont}
\newtheorem{Bsp}[Def]{Beispiel}
\newtheorem{Bem}[Def]{Bemerkung}
\theoremsymbol{\ensuremath{_\blacksquare}}
\theoremstyle{nonumberplain}
\newtheorem{Bew}[Def]{Beweis}
\setcounter{chapter}{1}
\setcounter{Def}{65}

% Kopf- und Fusszeilen
\pagestyle{fancy}
\fancyhead[LE,RO]{\thepage}
\fancyfoot[C]{}
\fancyhead[LO]{\rightmark}

\title{29.11.06}
\author{Das \texttt{latexki}-Team\\[8 cm]}

\date{Stand: \today}
\begin{document}

%1.5
\section{Standardkonstruktionen}
\textbf{A) Produkte}\\
Seien $X,Y$ nVRe. Dann:
\[
X \times Y = \{ (x,y): x \in X, y \in Y \} \text{ist ein nVR bzgl.}
\]
\[
||(x,y)||_p = \left\{
\begin{array}{rl}
(||x||_X^p + ||y||_Y)^{\frac1{p}} ,& 1 \leq p < \infty \\
\max\{ ||x||_X,||y||_Y\} ,& p = \infty
\end{array} \right.
\]
Diese Normen sind alle "aquivalent.\\
Sind $X,Y$ vollst"andig, dann ist $(X \times Y, ||\cdot||_p)$ ein BR.

%Definition ohne Nummer
\begin*{Def}
Sei $Z$ ein nVR und $P \in B(Z)$ mit $P = P^2$. Dann hei"st $P$ \begriff{Projektion}.
\end*{Def}
Hier ist die kanonische Projektion auf $X$ gegeben durch $P(x,y) = (x,0)$.\\
\\
\textbf{B) Diskrete Summe}

%Def 1.72
\begin{Def}
Seien $X_1,X_2$ abg. UVRe eines BRes $X$ mit $X_1 + X_2 = X$ und $X_1 \cap X_2 = \{ 0 \}$. Dann ist $X$ die \begriff{direkte Summe} von $X_1$ und $X_2$. Mann schreibt $X = X_1 \oplus X_2$.\\
$X_2$ hei"st dann \begriff{Komplement} von $X_1$ in $X$.
\end{Def}

%Lemma 1.73
\begin{Lem}
Sei $X$ ein BR und $P \in B(X)$ eine Projektion. Dann ist $Q = I-P \in B(X)$ auch eine Projektion und es gelten $R(P) = N(Q) =: X_1,\ N(P) = R(Q) =: X_2,\ X = X_1 \oplus X_2.$ Man hat $||P|| \geq 1$, wenn $P \not= 0.$
\end{Lem}

\begin{Bew}
$Q^2 = I - 2P + P^2 = I - P = Q$.\\
Falls $y = Px$ f"ur ein $x \in X \Longrightarrow Qy = Px - P^2x = 0.$\\
Falls $Qx = 0$ f"ur ein $x \in X \Longrightarrow x - Px = 0 \Longleftrightarrow x = Px \Longrightarrow x \in R(P) \Longrightarrow R(P) = N(Q)$.
Also ist $X_1 = N(Q) = R(P)$ abgeschlossen (1.16). Genauso: $X_2$.\\
Ist $x \in X \Longrightarrow x = Px + (I-P)x \in X_1 \oplus X_2$. Wenn $x \in X_1 \cap X_2 \Longrightarrow Px = 0$ und $x = Py$ f"ur ein $y \in X \Longrightarrow x = P^2 y = Px = 0 \Longrightarrow X_1 \cap X_2 = \{ 0 \}$. Schlie"slich: $||P|| = ||P^2|| \leq ||P||^2 \Longrightarrow ||P|| \geq 1$,\ falls $P \not= 0$.
\end{Bew}

Umkehrung:\\
Sei $X = X_1 \oplus X_2$. Dann existiert f"ur jedes $x \in X$ eindeutig bestimmte $x_1 \in X_1, x_2 \in X_2$ mit $x = x_1 + x_2$. Setze $Px = x_1$. Dann ist $P$ linear und $P=P^2$. Ferner ist $P$ stetig nach dem Homomorphiesatz (Kap. 3). Somit ist die Existenz direkter Zerlegungen und Projektionen "aquivalent.

%Beispiel 1.74
\begin{Bsp}
\begin{enumerate}

\item[a)] $X = L^1(\R).\ Pf := \id_{\R^+} \cdot f,\ f \in X \Rightarrow P \in B(X), ||P|| = 1, P = P^2$. Ferner: $(I-P)f = \id_{(-\infty,0)} \cdot f.$ Die Abbildung $J: R(P) \rightarrow L^1(\R^+),\ Jf = f_{|\R^+}$ ist stetig und linear mit stetiger Inverser
\[
J^{-1}g = \left\{
\begin{array}{rl}
g ,& \text{ auf} \R^+ \\
0 ,& \text{ auf} (-\infty,0)
\end{array} \right.
\]
$\Rightarrow R(P) \equiv L^1(\R^+)$. Entsprechend: $N(P) \equiv L^1(\R_-) \Rightarrow L^1(\R) \equiv L^1(\R^+) \oplus L^1(\R_-)$

\item[b)] $c_0$ hat kein Komplement in $\sll^{\infty}$ (Werner, IV 6.5)

\item[c)] $X = \R^2, P = \begin{pmatrix} 1 & t \\ 0 & 0 \end{pmatrix},\ t \in \R. P^2 = P$ und $||P|| = 1 + |t|$ bzgl. $||\cdot||_1.\ P$ ist Projektion auf $x-$Achse.
\end{enumerate}
\end{Bsp}

\textbf{Quotienten}\\
Seien $X$ nVR, $Y$ ein UVR.
\[
X Y = \{ \hat{x} = x + Y,\ x \in X \} \quad \text{Quotientenraum}
\]
Die Quotientenabbildung $\Pi: X \Rightarrow X Y, \PiX = \hat{X}$ ist wohldefiniert, linear und surjektiv. Man schreibt codin\footnote{FXIME: kann dies jemand berichtigen, hier unlesbar} $Y = \dim X Y.$ Es gilt $N(\Pi) = Y$. Definiere $||\hat{x}|| = \inf_{y \in Y} ||x-y|| := d(x,Y) \quad \text{Quotientennorm}$. Gilt $\overline{x}+Y = x + Y$ f"ur gewisse $x,\overline{x} \in X$, dann gilt: $\overline{x}-x \in Y \Rightarrow d(x,Y) = d(\overline{x},Y) \Rightarrow$ Quotientennorm wohldefiniert.\\
Sei $\alpha \in \K \backslash \{0\}$. Dann:
\[
||\hat{\alpha x}|| = \inf_{y \in Y} ||\alpha x -\frac{\alpha}{\alpha} y|| = |\alpha| \inf_{z \in Y} ||x-z|| = |\alpha| ||\hat{x}||
\]
Sien $x_1,x_2 \in X$ und $\eps > 0$. Dann ex. $y_1,y_2 \in Y$ so, dass $||x_k - y_k|| \leq ||\hat{x_k}|| + \eps,\ h=1,2. \Rightarrow ||\hat{x_1}+\hat{x_2}|| = \inf_{y \in Y} ||x_1+x_2-y|| \leq ||x_1-y_1+x_2-y_2|| \leq ||\hat{x_1}|| + ||\hat{x_2}|| + 2 \eps \folgtnach{\eps \rightarrow 0} ||\hat{x_1}+\hat{x_2}|| \leq ||\hat{x_1}|| + ||\hat{x_2}|| \Rightarrow ||\hat{x}||$ ist ein Halbnorm auf $X Y$.\\
Sei nun $Y$ abgeschlossen. Ist $||\hat{x}|| = 0$, dann ex $y_n \in Y$ mit $||x - y_n|| \rightarrow 0 \ (n \rightarrow \infty).$ Da $Y$ abg $\Rightarrow x \in Y \Rightarrow \hat{x}=0$ und $X Y$ ist nVR.\\
Weiter: $||\Pi(x)|| = ||\hat{x}|| \leq ||x||_X \Longrightarrow \Pi \in B(X, X \setminus Y)$ mit $||\Pi|| \leq 1$.
\end{document}