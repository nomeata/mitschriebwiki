\section{Moduln}

Sei $R$ kommutativ mit Eins.

\begin{DefBem}
\begin{enum}
\item Eine abelsche Gruppe $(M,+)$ zusammen mit einer Abbildung
$\bullet: R \times M \ra M$ heißt $\mathbf{R}$\emp{-Modul}, wenn für
alle $a,b \in R,\; x,y\in M$ gilt:
\begin{enumerate}
\item[(i)] $a(x+y) = ax + ay$
\item[(ii)] $(a+b)x = ax + bx$
\item[(iii)] $(ab)x = a(bx)$
\item[(iv)] $ 1x = x$
\end{enumerate}
\sbsp{0.9}{\begin{enumerate}
\item[(1)] $R$ ist $R$-Modul. (mit $\cd$ als Ringmultiplikation)
\item[(2)] Ist $R$ ein Körper, so ist $R$-Modul = $R$-Vektorraum.
\item[(3)] $R=\mathbb{Z}$, $M = \mathbb{Z}/2\mathbb{Z} = \{\bar 0,
\bar 1\}$ ist $\mathbb{Z}$-Modul durch $n \cd \bar 0 = \bar 0,\; n
\cd \bar 1 = \bar n$. Jede abelsche Gruppe $A$ ist
$\mathbb{Z}$-Modul durch $nx = \underset{\mbox{\scriptsize
n-mal}}{\underbrace{x+\dots+x}}$ und $(-n)x=n(-x)$ für $n \in \mathbb{N}_1, x \in A$
\item[(4)] Jedes Ideal in $R$ ist $R$-Modul.
\end{enumerate}}

\item Eine Abbildung $\varphi: M \ra M'$ von $R$-Moduln heißt
$\mathbf{R}$\emp{-Modulhomomorphismus} (oder
$\mathbf{R}$\emp{-linear}), wenn $\varphi$ Gruppenhomomorphismus ist
und für alle $x \in M, a \in R$ gilt: $\varphi(ax) = a \varphi(x)$

\item $Hom_R(M,M') \defeqr \{ \varphi: M \ra M' : \varphi$
$R$-linear$\}$ ist $R$-Modul durch
\newline$\left.\begin{array}{ll}
(\varphi_1 + \varphi_2)(x) &\defeqr \varphi_1(x) + \varphi_2(x) \\
(a \varphi)(x) &\defeqr a \varphi(x) \end{array}\right\}$ $\forall$
$\varphi_1,\varphi_2 \in Hom_R(M,M'),\;a\in R,\; x \in M$

\item Die $R$-Moduln bilden mit den $R$-linearen Abbildungen eine
Kategorie

\item Die Kategorien $\mathbf{\mathbb{Z}}$\emp{-Mod.} und \textbf{Abelsche
Gruppen} sind isomorph. denn: \[ \dots \varphi(nx) =
\varphi(x+\dots+x) = \varphi(x) + \dots + \varphi(x) = n
\varphi(x)\] ($\varphi: A \ra A'$ Gruppenhomomorphismus, $x \in A$,
$n \in \mathbb{N}$) $\Ra$ Jeder Gruppenhomomorphismus von abelschen
Gruppen ist $\mathbb{Z}$-linear.
\end{enum}
\end{DefBem}

\begin{DefBem}
Sei $M$ ein $R$-Modul.
\begin{enum}
\item Eine Untergruppe $U$ von $(M,+)$ heißt
$\mathbf{R}$\emp{-Untermodul} von $M$, wenn $R\cd U \subseteq U$
ist, dh. wenn $U$ selbst $R$-Modul ist.

\item Ist $\varphi:M \ra M'$ $R$-linear, so sind Kern($\varphi$) und
Bild($\varphi$) Untermoduln von $M$ bzw. $M'$ (denn $\varphi(x) = 0
\Ra \varphi(ax)=0 \forall \dots$ und $a\varphi(x) = \varphi(ax)
\forall \dots$)

\item Sei $U \subseteq M$ Untermodul.
\newline Dann wird $M/U$ zu einem $R$-Modul durch $a \overline{x} \defeql
\overline{ax}$ (denn: Ist $x' \in \overline{x}$, also $x-x' \in U$, so ist $ax'
- ax = a(x' - x) \in U$)
\newline Die Restklassenabbildung $p:M \ra M/U,\;x\mapsto \bar x$
ist dann $R$-linear ($p(ax) = \overline{ax} = a \overline{x} = a p(x)$)
\end{enum}
\end{DefBem}

\begin{DefBem}
\begin{enum}
\item Für $X \subseteq M$ heißt \[\langle X \rangle \defeqr
\bigcap_{\substack{U \mbox{ \scriptsize Untermodul von } M \\ X
\subseteq U}}\;U\] der von $X$ erzeugte Untermodul.

\item $\ds \langle X \rangle = \{ \sum_{i=0}^n a_i x_i,\; a_i \in R,
x_i \in X, n \in \mathbb{N}\}$

\item $B \subseteq M$ heißt \emp{linear unabhängig}, wenn $\ds 0 =
\sum_{i=0}^n a_i b_i$ mit $a_i \in R, b_i \in B, n \in \mathbb{N}$
nur möglich ist mit $a_i = 0\;\forall i$

\item $B \subseteq M$ heißt \emp{Basis}, wenn jedes $x \in M$
eindeutig als Linearkombination $\ds x=\sum_{i=0}^n a_i b_i$
($\forall \dots$) darstellbar ist.
\newline äquivalent: $B$ linear unabhängig und $\langle B \rangle =
M$

\item $M$ heißt \emp{frei}(er $R$-Modul), wenn $M$ eine Basis
besitzt.
\end{enum}
\bsp{\begin{enumerate}
\item[(1)] $R$ ist freier $R$-Modul mit Basis $1$ (oder einer
anderen Einheit)
\item[(2)] Für jedes $n \in \mathbb{N}$ ist $R^n = R \oplus \dots
\oplus R$ freier $R$-Modul mit Basis $e_1,\dots,e_n,\; e_i =
(0,\dots,0,1,0\dots,0)$
\item[(3)] Ist $I \subseteq R$ Ideal, so ist $M \defeqr R/I =
\langle \{\bar{1} \} \rangle$ Für $I \neq \{0\}$ ist $R/I$ \emp{nicht}
frei. denn: Sei $\bar x \in M, a \in I \setminus \{0\} \Ra a \bar x
= \overline{ax} = \overline{0} \Ra$ in $M$ gibt es kein linear unabhäniges
Element.
\end{enumerate}}
\end{DefBem}